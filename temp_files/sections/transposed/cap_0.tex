Much of this book originated in literary periodicals and magazines. Were it not for editors Alex Star, Paul Laity, Mary-Kay Wilmers, Paul Meyerscough, Adam Shatz, John Palattella, and Jackson Lears, I never would have written about the right. It’s often assumed that academics who publish in nonacademic venues are trotting out their scholarly research for popular consumption, simplifying complex ideas first worked out in the laboratories of academe. For me, the process of writing this book has been the reverse: conservatism became a scholarly interest of mine through my nonacademic writing, and most of my ideas about the right were formulated in conversation with and writing for these editors, especially Alex and John.
 \par 
Intellectually, this book owes its inspiration to Arno Mayer and Karen Orren. No two scholars have done more to advance my under-standing of “the persistence of the old regime”—in Europe and the United States—than Karen and Arno. Against the conventional wisdom of the left and the right, which assumes that medievalism has been washed away by modernity, Karen and Arno opened my eyes to the “belated feudalism” of our post-feudal world. While they undoubtedly would disagree with my interpretation of conservatism, I could not have come to it without their enormously generative work.
 \par 
In the course of writing and revising these essays, I have been sustained by a broad circle of readers: historians and political
 \par 
Scientists, poets and essayists, theorists and philosophers, literary critics and sociologists, journalists and editors. For their contributions to one or more of these essays, I would like to thank Jed Abrahamian, Bruce Ackerman, Joel Allen, Gaston Alonso, Joyce Appleby, Moustafa Bayoumi, Seyla Benhabib, Mar-shall Berman, Sara Bershtel, Akeel Bilgrami, Norman Birnbaum, Steve Bronner, Dan Brook, Sebastian Budgen, Josh Cohen, Peter Cole, Paisley Currah, Lizzie Donahue, Jay Driskell, Tom Dumm, John Dunn, Sam Farber, Liza Featherstone, Jason Frank, Steve Fraser, Josh Freeman, Paul Frymer, Sam Goldman, Manu Goswami, Alex Gourevitch, Pete Hallward, Harry Harootunian, Chris Hayes, Doug Henwood, Dick Howard, David Hughes, Judy Hughes, Allen Hunter, Jack Jacobs, Ira Katznelson, Gordon Lafer, Jill Lepore, Penny Lewis, Joe Lowndes, Steven Lukes, Kieko Matteson, Kevin Mattson, John Medeiras, Kathy New-man, Molly Nolan, Anne Norton, Jolie Olcott, Christian Parenti, Di Paton, Rick Perlstein, Ros Petchesky, Kim Phillips-Fein, Katha Pollitt, Aziz Rana, Andy Rich, Andrew Ross, Kristin Ross, Saskia Sassen, Ellen Schrecker, George Scialabba, Richard Seymour, Nikhil Singh, Quentin Skinner, Jim Sleeper, Rogers Smith, Katrina van den Heuvel, John Wallach, Eve Weinbaum, Keith Whittington, Daniel Wilkinson, Wesley Yang, Brian Young, and Marilyn Young.
 \par 
A good portion of this material has been presented in work-shops and talks at universities across the country. I am grateful for the comments and suggestions I received on those occasions from Arash Abizadeh, Anthony Appiah, Banu Bargu, Seyla Benhabib, Akeel Bilgrami, Elizabeth Cohen, Josh Cohen, Julie Cooper, the late Jack Diggins, Matt Evans, Nancy Fraser, Mark Graber, Nan Keohane, Steve Macedo, Karuna Mantena, Andrew March, Tom Medvetz, Andrew Murphy, Andrew Norris, Anne Norton, Joshua Ober, Philip Pettit, Andy Polsky, Robert Reich, Austin Sarat, Peter
 \par 
I would like to thank the following institutions for providing much needed release time from my teaching: the American Council of Learned Societies; the Princeton University Center for Human Values; the Offi ce of the Provost at Brooklyn College; and the Professional Staff Congress of the City University of New York. An extra special thanks goes to my kitchen cabinet of first readers: Greg Grandin, Adina Hoff man, Robert Perkinson, and Scott Saul; to Marco Roth, who came up with the book’s title; to Charles Petersen, copy editor extraordinaire; to my students at Brooklyn College and the CUNY Graduate from Center, who have worked with me through the texts and tomes of the right; to Alexandra Dauler and Marc Schneider at Oxford University Press (OUP); and to David McBride, my editor at OUP, an unfailing source of excellent advice who believed in this project from its inception and shepherded it through with what seems to be effed ortless wisdom, patience, and grace.
 \par 
My greatest thanks go to Laura Brahm, who listened to these ideas when they were half-sentences and read them when they were half-baked. She has brought to these essays an eye for what matters and an unerring sense of taste. She is always and inevitably the only reader I want to please.