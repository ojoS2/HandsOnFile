\begin{ figure }
	\centering
	\\includegraphics[width=1.\textwidth]{ temp_files/images/UP_logo.png }
	\caption{Elena Lagadinova (right, with Angela Davis) (1930-2017): The youngest fe- male partisan fighting against Bulgaria’s Nazi-allied monarchy during World War II. She earned her PhD in agrobiology and worked as a research scientist before she became the president of the Committee of the Bulgarian Women’s Movement. Lagadinova led the Bulgarian delegation to the 1975 United Nations First World Conference on Women. Because free markets discriminate against those who bear children, Lagadinova believed that only state intervention could support women in their dual roles as workers and mothers. Courtesy of Elena Lagadinova.}
	\label{ }
\end{ figure }
 \par 
\chapter{AUTHOR'S NOTE}\label{AUTHOR'S NOTE}
 \par 
= the last twenty years, I have studied the social impacts of the political and economic transition from state socialism to capitalism in Eastern Europe. Although I first traveled through the region just months after the fall of the Berlin Wall in 1989, my professional interest began in 1997, when I started conducting research on the impacts of the collapse of communist ideology on ordinary people. First as a PhD student and later as a university professor, I lived for more than three years in Bulgaria and nineteen months in both eastern and Western Germany. In the summer of 1990, I also spent two months traveling through Yugoslavia, Romania, Hungary, Czechoslovakia, and the soon-to-disappear German Democratic Republic. In the intervening years, I’ve been a frequent visitor to Eastern Europe, delivering invited lectures in cities such as Belgrade, Bucharest, Budapest, and Warsaw. Because I often travel by car, bus, and train, I’ve seen firsthand the ravages of neoliberal capitalism across the region: bleak landscapes pockmarked with the decrepit remains of once thriving factories giving way to new suburbs with Walmart-style megastores selling forty-two different types of shampoo. I’ve also studied how the institution of unregulated free markets in
 \par 
Eastern Europe returned many women to a subordinate status, economically dependent on men.
 \par 
Since 2004, I’ve published six scholarly books and over three dozen articles and essays, using empirical evidence gathered from archives, interviews, and extended ethnographic fieldwork in the region. In this book, I draw on over twenty years of research and teaching to write an introductory primer for a general audience interested in European socialist feminist theories, the experience of twentieth-century state socialism, and their lessons for the present day. After the unexpected success of Bernie Sanders in the 2016 Democratic primaries, socialist ideas are circulating more broadly among the American public. It is essential that we pause and learn from the experiences of the past, examining both good and bad. Because I believe in the pursuit of historical nuance, and that there were some redeeming qualities of state socialism, I will inevitably be accused of being an apologist for Stalinism. Vitriolic ad hominem attacks are the reality of our hyperpolarized political climate, and I find it quite ironic that those who claim to abhor totalitarianism have no trouble silencing speech or unleashing hysterical Twitter mobs. The German political theorist Rosa Luxemburg once said: “Freedom is always and exclusively freedom for the one who thinks differently.” This book is about learning to think differently with regard to the state socialist past, our neoliberal capitalist present, and the path to our collective future.
 \par 
Throughout this book, I use the term “state socialism” or “state socialist” to refer to the states of Eastern Europe and the Soviet Union dominated by ruling Communist Parties where political freedoms were curtailed. I use the
 \par 
AUTHOR'S NOTE term “democratic socialism” or “democratic socialist” to refer to countries where socialist principles are championed by parties that compete in free and fair elections and where political rights are maintained. Although many parties re-
 \par 
Ferred to themselves as “communist,” that term denotes the ideal of a society where all economic assets are collectively owned and the state and law have withered away. In no case has real communism been achieved, and therefore I try to avoid this term when referring to actually existing states.
 \par 
On the topic of semantics, I have also endeavored to be sensitive to contemporary intersectional vocabularies. For example, when I talk about “women” in this book, Iam primarily referring to cisgender women. The nineteenth and twentieth-century socialist “woman question” did not consider the unique needs of trans women, but I have no desire to exclude or alienate trans women from the current discussion. Similarly, in my discussion of maternity, I do recognize that I am discussing those who are female assigned-at-birth (FAB), but for the sake of simplicity, I use the word “woman” even though this category includes some who identify as men or other genders.
 \par 
Because this is an introductory book, there will be places in the text where I don’t go into full detail about the debates surrounding topics such as Universal Basic Income (UBI), surplus value extraction, or gender-based quotas. In particular, although I believe that they are absolutely essential, I don’t spend a lot of time discussing universal single-payer health care or free public postsecondary education, because I feel these policies have been discussed at length elsewhere. I hope readers are inspired to explore more about the issues raised within these pages, taking this xi
 \par 
xii
 \par 
AUTHOR'S NOTE book as an invitation for further exploration of the intersections of socialism and feminism. I also want to make it clear that this is not a scholarly treatise; those in search of theoretical frameworks and methodological debates should consult the books I’ve published with university presses. I also recognize the long and important tradition of Western socialist feminism, although it is not discussed in these pages. Furthermore, I encourage interested readers to refer to the books listed in the suggestions for further reading.
 \par 
For all the direct quotations and statistical claims made throughout the book, I include consolidated citations in an endnote at the end of the relevant paragraph. Few substantive end notes accompany this text, so most readers can feel free to ignore the end notes unless they have a question about a particular source. General historical material can be found in the suggestions for further reading. When discussing personal anecdotes, I have changed the names and identifying details to preserve anonymity.
 \par 
Finally, with the many social ills plaguing the world today, some might find the chapters on intimate relations a bit too prurient for their taste; some might think that having better sex is a trivial reason to switch economic systems. But turn on the television, open a magazine, or surf the internet, and you will find a world saturated with sex. Capitalism has no problem commodifying sexuality and even preying on our relationship insecurities to sell us products and services we don't want or need. Neoliberal ideologies persuade us to view our bodies, our attentions, and our affections as things to be bought and sold. I want to turn the tables. To use the discussion of sexuality to expose the shortcomings of unfettered free markets. If we can better
 \par 
AUTHOR'S NOTE understand how the current capitalist system has co-opted
 \par 
And commercialized basic human emotions, we have taken the first step toward rejecting market valuations that purport to quantify our fundamental worth as human beings. The political is personal.
 \par 
xiii