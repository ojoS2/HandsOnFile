{\chapter{Conclusion} } {\label{Conclusion} }{\par}{\textit{	} } {\par}{\par} {\textbf{\textit{	} } } {\par} 
	Conservatism has dominated American politics for the past forty years. Just as the Republican administrations of Dwight Eisenhower and Richard Nixon demonstrated the resilience of the New Deal, so have the Democratic administrations of Bill Clinton and Barack Obama demonstrated the resilience of Reaganism. The conservative embrace of unregulated capitalism and imperial power still envelops our two parties. Consistent with this book’s argument about the private life of power, the most visible effort of the GOP since the 2010 midterm election has been to curtail the rights of employees and the rights of women. While the right’s success in these campaigns is by no means assured, the fact that the Republicans have taken aim at the last redoubt of the labor movement and the entirety of Planned Parenthood gives some indication of how far they’ve come. The end (in both senses of the word) of the right’s long march against the twentieth century may be in sight.{\par} 	The success of the right, however, is not an unmixed blessing. As conservatives have long noted, there is a dialectical synergy between the left and the right, in which the progress of the former spurs on the innovations of the latter. “It is ironic, although not historically unprecedented,” wrote Frank Meyer, the intellectual architect of the FM- Zionist strategy that brought together the libertarian and traditionalist wings of modern conservatism, “that such a burst of creative energy on the intellectual level” on the right “should occur simultaneously with a continuing spread of the influence of Liberalism in the practical political sphere.” Across the Atlantic, Roger Scruton, a more traditional type of British Tory, wrote that “in times of crisis. . . Conservatism does its best,” while Friedrich Hayek observed that the defense of the free market “became stationary when it was most influential” and “progressed” when it was “on the defensive.” {\color{blue} 1 } True, these were intellectuals writing about ideas; conservative operatives might be less sanguine about the prospect of trading four more years for a few good books. Even so, if the ultimate fate of a party is tied to the strength of its ideas—not the truth of its ideas, but the resonance and pertinence of those ideas, their cultural purchase and ability to travel across the political landscape—it should be a cause of concern on the right that its ideas have so roundly succeeded. As Burke warned long ago, victory may simply be a way station to death.{\par} Several recent books of conservative introspection suggest that many on the right are indeed concerned about the state of conservative ideas. {\color{blue} 2 } But most of these attempts at self-criticism seem motivated by a simple fear of defeat at the polls. Oriented as they are to the electoral cycle or to the pros and cons of particular policies, they don’t see that conservatism, like any party, can lose elections yet still control public debate. More important, these writers don’t understand that failure is the wellspring of conservative renewal. They imagine that conservatism can simply be reinvented or retooled to meet the needs of a changing electorate or the hobbyhorses of its theoreticians. But that is not how conservatism works. Conservatism requires defeat; failure is its most potent source of inspiration. Not failure in the brooding, romantic sense that Andrew Sullivan articulates in his paean to loss, but failure in the simultaneously threatening and galvanizing sense. {\color{blue} 3 } Loss—real social loss, of power and position, privilege and prestige—is the mustard seed of conservative innovation. What the right suffers from today is not loss but success, and until a signifi can't dominant group in society is forced to suff her loss—of the kind experienced by employers during the 1930s, white supremacists during the 1960s, or husbands in the 1970s—it will remain a philosophically flabby movement. Politically powerful, but intellectually moribund.{\par} Which leads me to wonder about the long-term prospects of the Tea Party, the latest variant of right-wing populism. Has the Tea Party given conservatism a new lease on life? Or is the Tea Party like the New Politics of the late 1960s and early 1970s, the last spark of a spent force, its frantic energies a mask for the decline of the larger movement of which it is a part? It’s impossible to say, but this much is clear: So long as there are social movements demanding greater freedom and equality, there will be a right to counter them. Except the gay rights movement, there are today no threatening social movements of the left. Once they arise, a new right will arise with them—not a right that needs to invent bogeymen like Obama’s socialism but a right with real monsters to destroy. Until then, we can chalk up the current state of the right not to its failures of imagination or excess of spleen— as some have done {\color{blue} 4 } —but to its overwhelming success.{\par} Modern conservatism came onto the scene of the twentieth century in order to defeat the great social movements of the left. As far as the eye can see, it has achieved its purpose. Having done so, it now can leave. Whether it will, and how much it will take with it on its way out, remains to be seen.{\par}