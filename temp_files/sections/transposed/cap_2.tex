\chapter{~ YOU MIGHT BE SUFFERING}\label{~ YOU MIGHT BE SUFFERING}
 \par 
TTTTTT the argument of this book can be summed up succinctly: Unregulated capitalism is bad for women, and if we adopt some ideas from socialism, women will have better lives. If done properly, socialism leads to economic independence, better labor conditions, better work/ family balance, and, yes, even better sex. Finding a way into a better future requires learning from the mistakes of the past, including a thoughtful assessment of the history of twentieth-century state socialism in Eastern Europe.
 \par 
»
 \par 
That’s it. If you like the idea of such outcomes, then come along for an exploration of how we might change things. If you are dubious because you don’t understand why capitalism as an economic system is uniquely bad for women, and if you doubt that there could ever be anything good about socialism, this short treatise will provide some illumination. If you don’t give a whit about women’s lives because you're a xenophobic right-wing internet troll, save your money and get back to your parents’ basement right now; this isn’t the book for you.
 \par 
2
 \par 
YOU MIGHT BE SUFFERING FROM CAPITALISM
 \par 
Of course, some might argue that unregulated capitalism sucks for almost everyone, but I want to focus on how capitalism disproportionately harms women. Competitive labor markets discriminate against those whose reproductive biology makes them primarily responsible for child bearing. Today, this means humans who get pink hats in the hospital and the letter “F” next to the name on their birth certificate (as if we've already failed by not coming into the world as a boy). Competitive labor markets also devalue those expected to be the primary caregivers of children. Although societal attitudes have evolved in this regard, our idealization of motherhood means that most of us still believe that baby needs mama a lot more than papa—at least until the child is old enough to play sports.
 \par 
Others will argue that unregulated capitalism is not bad for all women. Yes, for those women lucky enough to sit at the top of the income distribution, the system works pretty well. Although women at the executive level still face gender pay gaps and remain underrepresented in leadership positions, on the whole things aren't too shabby for the Sheryl Sandbergs of the world. Of course, sexual harassment still hinders progress even for those at the top, and too many women believe that if you want to run with the big dogs, you may have to suck it up and ignore the groping and unwanted advances. And race plays an important role as well; white women do a lot better in aggregate than do women of color. But when we look at society as a whole, on average, women are comparatively worse off in countries where markets are less encumbered by regulation, taxation, and public enterprises than they are in nations where state revenues support greater levels of redistribution and larger social safety nets.
 \par 
KRISTEN R. GHODSEE
 \par 
Choose your data source, and you find the same story. Unemployment and poverty plague women with children. Employers discriminate against women without children because they might have them in the future. In the United States in 2013, women over the age of sixty-five suffered from poverty at much greater rates than men and dominated those in the category of “extreme poverty.” Globally, women face higher rates of economic deprivation. Women are often the last to be hired and the first to be fired in cyclical downturns, and when they do find employment, bosses pay them less than men. When states need to slash government spending on education, health care, or old age pensions, mothers, daughters, sisters, and wives must pick up the slack, diverting their energies to care for the young, the sick, and the elderly. Capitalism thrives on women’s unpaid labor in the home because women’s care work supports lower taxes. Lower taxes mean higher profits for those already at the top of the income ladder—mostly men.'
 \par 
But capitalism was not always so savage. Throughout much of the twentieth century, state socialism presented an existential challenge to the worst excesses of the free market. The threat posed by Marxist ideologies forced Western governments to expand social safety nets to protect workers from the unpredictable but inevitable booms and busts of the capitalist economy. After the Berlin Wall fell, many celebrated the triumph of the West, consigning socialist ideas to the dustbin of history. But for all its faults, state socialism provided an important foil for capitalism. It was in response to a global discourse of social and economic rights—a discourse that appealed not only to the progressive populations of Africa, Asia, and Latin America but also to many men and women in Western Europe and North America—that
 \par 
3
 \par 
4
 \par 
YOU MIGHT BE SUFFERING FROM CAPITALISM politicians agreed to improve working conditions for wage laborers as well as create social programs for children, the poor, the elderly, the sick, and the disabled, mitigating exploitation and the growth of income inequality. Although there were important antecedents in the 1980s, once state socialism collapsed, capitalism shook off the constraints of market regulation and income redistribution. Without the looming threat of a rival superpower, the last thirty years of global neoliberalism have witnessed a rapid shriveling of social programs that protect citizens from cyclical instability and financial crises and reduce the vast inequality of economic outcomes between those at the top and bottom of the income distribution.
 \par 
For much of the twentieth century, Western capitalist countries also endeavored to outdo the East European countries in terms of women’s rights, fueling progressive social change. For example, the state socialists in the USSR and Eastern Europe were so successful at giving women economic opportunities outside the home that initially, for two decades after the end of World War I, women’s wage work was conflated with the evils of communism. The American way of life meant male breadwinners and female homemakers. But slowly, socialist championing of women’s emancipation began to chip away at the Leave It to Beaver ideal. The Soviet launch of Sputnik in 1957 spurred American leaders to rethink the costs of maintaining traditional gender roles. They feared the state socialists enjoyed an advantage in technological development because they had double the brainpower; the Russians educated women and funneled the best and the brightest into scientific research.”
 \par 
KRISTEN R. GHODSEE
 \par 
Fearing Eastern Bloc superiority in the space race, the American government passed the National Defense Education Act (NDEA) in 1958. Despite a continuing cultural desire for women to stay at home as dependent wives, the NDEA created new opportunities for talented girls to study science and math. Then, in 1961, President John F. Kennedy signed Executive Order 10980 to establish the first Presidential Commission on the Status of Women, citing national security concerns. This commission, chaired by Eleanor Roosevelt, laid the groundwork for the future US women’s movement. Americans received a further shock in 1963, when Valentina Tereshkova became the first female cosmonaut, spending more time orbiting the Earth than all male astronauts in the United States had, combined. Later, Soviet and East European dominance at the Olympics spurred the passage of Title IX, so that the United States could identify and train more female athletes to snatch gold medals away from the ideological enemy.’
 \par 
In response to state socialist prowess in the sciences, the American government sponsored an important study titled “Women in the Soviet Economy.” The head of the study visited the USSR in 1955, 1962, and 1965 to examine Soviet policies to integrate women into the formal labor force as an example for American legislators. “Concern in recent years on the waste of women’s talent and labor potential led to the appointment of the President's Commission on the Status of Women, which has issued a series of reports on various problems affecting women and their participation in economic, political, and social life,” the 1966 report began. “For any formulation of policy directed toward the better use of our women power, it is important to know the experience
 \par 
5
 \par 
6
 \par 
YOU MIGHT BE SUFFERING FROM CAPITALISM of other nations in utilizing the capabilities of women. For this reason as well as others, the Soviet experience is of particular interest at this time.” The precedent set by the state socialist countries in Eastern Europe acted as an influential example for American politicians at the same historical moment that Betty Friedan published The Feminine Mystique and revealed how unsatisfied middle-class, white women felt with their circumscribed domestic lives. But in the current political climate, it may be hard to fathom how a rivalry between superpowers could have sparked interest in the status of women.*
 \par 
Today, socialist ideas are enjoying a renaissance as young people across countries such as the United States and France find inspiration in politicians and movements like Bernie Sanders, Alexandria Ocasio-Cortez, Jean-Luc Mélenchon, and the Mouvement des gilets jaunts (yellow vests). Citizens desire an alternative political path that would lead to a more egalitarian and sustainable future. To move forward, we must be able to discuss the past with no ideologically motivated attempts to whitewash or backwash either our own history or the accomplishments of state socialism. On the one hand, any nuanced account of twentieth-century state socialism will inevitably encounter the sputtering and bluster of those who insist that it was pure evil, end of story. As the Czech writer Milan Kundera wrote in his famous novel The Unbearable Lightness of Being: “The people who struggle against what we call totalitarian regimes cannot function with queries and doubts. They, too, need certainties and simple truths to make the multitudes understand, to provoke collective tears.”° On the
 \par 
KRISTEN R. GHODSEE other hand, some young people today joke about “full communism now.” Leftist millennials might not know about (or prefer to ignore) the real horrors inflicted on citizens in one party states. Gruesome tales of the secret police, travel restrictions, consumer shortages, and labor camps are not just anticommunist propaganda. Our collective future depends on a balanced examination of the past so we can discard the bad and move forward with the good, especially where women’s rights are concerned.
 \par 
Since the middle of the nineteenth century, European social theorists argued that the female sex is uniquely disadvantaged in an economic system that prizes profits and private property over people. Throughout the 1970s, socialist feminists in the United States also asserted that smashing the patriarchy wasn’t enough. Exploitation and inequality would persist so long as financial elites built their fortunes on the backs of docile women reproducing the labor force for free. But these early critiques were based on abstract theories with little empirical evidence to substantiate them. Slowly, over the course of the first half of the twentieth century, new democratic socialist and state socialist governments in Europe began to test these theories in practice. In East Germany, Scandinavia, the Soviet Union, and Eastern Europe, political leaders supported the idea of women’s emancipation through their full incorporation into the labor force. These ideas soon spread to China, Cuba, and a wide variety of newly independent countries across the globe. Experiments with female economic independence fueled the twentieth-century women’s movement and resulted in-a revolution in the life paths open to women previously confined to the domestic sphere. And nowhere
 \par 
7
 \par 
8
 \par 
YOU MIGHT BE SUFFERING FROM CAPITALISM in the world were there more women in the workforce than under state socialism.°
 \par 
Women’s emancipation infused the ideology of almost all state socialist regimes, with the Franco-Russian revolutionary Inessa Armand famously declaring: “If women’s liberation is unthinkable without communism, then communism is unthinkable without women’s liberation.” Although important differences existed between countries and none achieved full equality in practice, these nations did expend vast resources to invest in women’s education and training and to promote them in professions previously dominated by men. Understanding the demands of reproductive biology, they also attempted to socialize domestic work and child care by building a network of public creches, kindergartens, laundries, and cafeterias. Extended, job-protected maternity leaves and child benefits allowed women to find at least a modicum of work/family balance. Moreover, twentieth-century state socialism did improve the material conditions of millions of women’s lives; maternal and infant mortality declined, life expectancy increased, and illiteracy all but disappeared. To take just one example, the majority of Albanian women were illiterate before the imposition of socialism in 1945. Just ten years later, the entire population under forty could read and write, and by the 1980s half of Albania’s university students were women.’
 \par 
While different countries pursued different policies, in general state socialist governments reduced women’s economic dependence on men by making men and women equal recipients of services from the socialist state. These policies helped to decouple love and intimacy
 \par 
KRISTEN R. GHODSEE from economic considerations. When women enjoy their own sources of income, and the state guarantees social security in old age, illness, and disability, women have no economic reason to stay in abusive, unfulfilling, or otherwise unhealthy relationships. In countries such as Poland, Hungary, Czechoslovakia, Bulgaria, Yugoslavia, and East Germany, women’s economic independence translated into a culture in which personal relationships could be freed from market influences. Women didn’t have to marry for money.®
 \par 
Of course, just as we can learn from the experiences of Eastern Europe, we shouldn't ignore the downsides. Women’s rights in the Eastern Bloc failed to include a concern for same-sex couples and gender nonconformity. Abortion served as a primary form of birth control in the countries where it was available on demand. Most East European states strongly encouraged women to become mothers, with Romania, Albania, and the USSR under Stalin forcing women to have children they didn’t want. State socialist governments suppressed discussions of sexual harassment, domestic violence, and rape. And although they tried to get men involved in housework and child care, men largely resisted challenges to traditional gender roles. Many women suffered under a double burden of mandatory formal employment and domestic work, as so well captured in Natalya Baranskaya’s brilliant novella, A Week Like Any Other. Finally, in no country were women’s rights promoted as a project to support women’s individualism or self-actualization. Instead, the state supported women as workers and mothers so they could participate more fully in the collective life of the nation.’
 \par 
9
 \par 
10
 \par 
YOU MIGHT BE SUFFERING FROM CAPITALISM
 \par 
After the fall of the Berlin Wall in 1989, new democratic governments rapidly privatized state assets and dismantled social safety nets. Men under these newly emerging capitalist economies regained their “natural” roles as family patriarchs, and women were expected to return home as mothers and wives supported by their husbands. Across Eastern Europe, post-1989 nationalists argued that capitalist competition would relieve women of the notorious double burden and restore familial and societal harmony by allowing men to reassert their masculine authority as breadwinners. However, this meant that men could once again wield financial power over women. For instance, the renowned historian of sexuality Dagmar Herzog shared a conversation with several East German men in their late forties in 2006. They told her that “it was really annoying that East German women had so much sexual self-confidence and economic independence. Money was useless, they complained. The few extra Eastern Marks that a doctor could make in contrast with, say, someone who worked in the theater, did absolutely no good, they explained, in luring or retaining women the way a doctor’s salary could and did in the West. ‘You had to be interesting. What pressure. And as one revealed: ‘I have much more power now as a man in unified Germany than I ever did in communist days.” Furthermore, following the publication of my New York Times op-ed, “Why Women Had Better Sex Under Socialism,” I did an interview with Doug Henwood on his radio show, Behind the News. One listener, a forty-six-year-old woman born in the Soviet Union, emailed the show to say that I had “nailed it” in my discussion of romantic relations in “the old
 \par 
»»
 \par 
KRISTEN R. GHODSEE country, as she called it, “but also the way men lord it over women with money here [in the United States].”"
 \par 
The collapse of state socialism in 1989 created a perfect laboratory to investigate the effects of capitalism on women’s lives. The world could watch as free markets were conjured from the rubble of the planned economy, and these new markets variously affected different categories of workers. After decades of shortages, East Europeans eagerly exchanged authoritarianism for the promise of democracy and economic prosperity, throwing their countries open to Western capital and international trade. But there were unforeseen costs.
 \par 
The rejection of the one-party state and the embrace of political freedoms came bundled with economic neoliberalism. New democratic governments privatized public enterprises to make room for new competitive labor markets where productivity would determine wages. Gone were the long lines for toilet paper and the black markets for jeans. Coming soon was a glorious consumer paradise free from shortages, famines, the secret police, and the labor camp. But after almost three decades, many Eastern Europeans still wait for a bright capitalist future. Others have abandoned all hope.”
 \par 
The evidence is incontrovertible: like so many other women across the globe, women in Eastern Europe are once again commodities to be bought and sold—their price determined by the fickle fluctuations of supply and demand. Writing in the immediate aftermath of state socialism’s collapse, the Croatian journalist Slavenka Drakulić explained, “We live surrounded by newly opened
 \par 
11
 \par 
12
 \par 
YOU MIGHT BE SUFFERING FROM CAPITALISM porno shops, porno magazines, peepshows, stripteases, unemployment, and galloping poverty. In the press they call Budapest ‘the city of love, the Bangkok of Eastern Europe. Romanian women are prostituting themselves for a single dollar at the Romanian-Yugoslav border. In the midst of all this, our anti-choice nationalist governments are threatening our right to abortion and telling us to multiply, to give birth to more Poles, Hungarians, Czechs, Croats, Slovaks.” Today, Russian mail-order brides, Ukrainian sex workers, Moldovan nannies, and Polish maids flood Western Europe. Unscrupulous middle men harvest blond hair from poor Belorussian teenagers for New York wigmakers. In St. Petersburg, women attend academies for aspiring gold diggers. Prague is an epicenter of the European porn industry. Human traffickers prowl the streets of Sofia, Bucharest, and Chișinău for hapless girls dreaming of a more prosperous life in the West."
 \par 
Older citizens of Eastern Europe fondly recall the small comforts and predictability of their life before 1989: free education and health care, no fear of unemployment and of not having money to meet basic needs. A joke, told in many East European languages, illustrates this sentiment:
 \par 
TTTTTT out of bed, eyes filled with terror. Her startled husband
 \par 
Watches her rush into the bathroom and open the medicine cabinet. She then dashes to the kitchen and inspects
 \par 
The inside of the refrigerator. Finally, she flings open a window and gazes out onto the street below their apart-
 \par 
Ment. She takes a deep breath and returns to bed.
 \par 
\[FROM CAPITALISM\]
 \par 
“What’s wrong with you?” her husband says. “What happened?”
 \par 
SSSSSS that we had the medicine we needed, that our refiner-
 \par 
Ator was full of food, and that the streets outside were
 \par 
\section{In the middle of the night a woman screams and jumps}
 \par 
SSSSSS thought the Communists were back in power.”
 \par 
Opinion polls throughout the region continue to show that many citizens believe their lives were better before 1989, under authoritarianism. Although these polls may say more about disappointment with the present than they do about the desirability of the past, they complicate the totalitarian narrative. For example, a 2013 random poll of 1,055 adult Romanians found that only a third reported that their lives were worse off before 1989: {\color{blue}44} percent said their lives were better, and {\color{blue}16} percent said there was no change. These results were also gendered in interesting ways: whereas {\color{blue}47} percent of women thought that state socialism was better for their country, only {\color{blue}42} percent of men said the same. Similarly, whereas {\color{blue}36} percent of men claimed that life was worse before 1989, only {\color{blue}31} percent of women said life under the dictator Nicolae Ceausescu was worse than the present. And this is from Romania, one of the most corrupt and oppressive regimes in the former Eastern Bloc where Ceausescu gold-plated the flushing handle on his private toilet. Similar results emerged from surveys in Poland in 2011 and from an opinion poll conducted in eight other
 \par 
13
 \par 
14
 \par 
TTTTTT former socialist nations in 2009. For citizens who have had the opportunity to live under two different economic systems, many now feel that capitalism is worse than the state socialism they were once so eager to cast aside.’*
 \par 
Back in the United States, the collapse of East European state socialism ushered in an era of Western capitalist triumphalism. Great Society ideas about how to regulate our economy and redistribute wealth to maximize the wellbeing of all citizens, including women, fell out of favor. The rise of what was called the Washington Consensus (born of Reaganomics) meant marketization, privatization, and the shredding of social safety nets in the name of efficiency. Throughout the 1990s and 2000s, citizens witnessed increasing deregulation of the financial, transportation, and utility sectors and the growing commodification of everyday life. We conflated freedom with free markets. After the global financial crisis in 2008, economic elites targeted already lean state budgets, slashing deeper into social programs while using taxpayer monies to bail out the bankers who created much of the mess in the first place. Occupy Wall Street called attention to structural inequality, but politicians on both sides of the aisle met rising public anger with the same old line: there is no alternative to capitalism.
 \par 
This is a lie. Conservative cold warriors will counter any attempt to complicate the history of twentieth-century state socialism with screaming about Stalin’s famines and purges. In their imagination, the entire experience of state socialism consisted of people standing in bread lines and snitching
 \par 
KRISTEN R. GHODSEE on their neighbors to the secret police. For seventy years in the Soviet Union and forty-five years in Eastern Europe, totalitarian leaders apparently shuttled everyone back and forth between labor camps and prisons, a godless Orwellian nightmare where people wore gray, unisex Mao suits and sported shaved heads. If babies were born, it’s not because people chose to start families, but because the Party mass-inseminated the population to meet predetermined human production quotas. Anticommunists refuse to acknowledge the important differences among the wide variety of societies that embraced socialism or to credit them for their various achievements in science, education, health outcomes, culture, and sport. In the stereotypes promoted by Western leaders, state socialism was an inefficient economic system doomed to inevitable collapse and a terrifying red menace requiring billions of dollars of taxpayer money to contain. It’s odd to consider how it could have been both.
 \par 
Within Eastern Europe today, numerous Westernfunded research institutes investigate the crimes of communism. In countries like Hungary, Bulgaria, and Romania (all German allies in World War II) the descendants of Nazi collaborators are keen to paint themselves as “victims of communism.” Local politicians and economic elites who benefitted from the transition to free markets (particularly those who had the previously nationalized property of their grandparents instituted to them after 1989) collude to create one official totalitarian narrative about the past. For example, after a lecture I gave in Vienna in 2011, a young Bulgarian woman in the audience sent an email thanking me for my courage in discussing some of the positive legacies
 \par 
15 ts
 \par 
YOU MIGHT BE SUFFERING FROM CAPITALISM
 \par 
\[KRISTEN R. GHODSEE\]
 \par 
TTTTTT of Todor Zhivkov, Bulgaria's leader from 1954 to 1989. “No one [in Bulgaria] can talk about the nostalgia and the pains of transition without being framed as a communist and as someone who denies the crimes of the Zhivkov regime. So the important issues you deal with are not present in the discourse or media.” In nearby Romania, the literary scholar Costi Rogozanu has criticized the East European practice of using horror stories about the state socialist past to justify the continued implementation of neoliberal economic policies in the present: “Do you want a salary raise? You are communist. Do you want public services? Do you want to tax the rich and ease the burden on small producers and wage earners? You are a communist and you killed my grandparents. Do you want public transportation instead of highways? You are mega-communist and an era: hipster.”
 \par 
Although it’s important not to romanticize the state socialist past, the ugly realities should not make us completely oblivious to the ideals of the early socialists, to the various attempts to reform the system from within (such as the Prague Spring, glasnost, or perestroika), or to the important role that socialist ideals played in inspiring national independence movements in the Global South. Acknowledging the bad does not negate the good. Just as there are those who would like to whitewash American history by downplaying, just for starters, Jim Crow, institutional racism, gun violence, or the unprecedented incarceration rate, there are those who would backwash the history of state socialism, insisting that everything was evil.’
 \par 
Today, we have over two hundred: years of experimentation with various forms of socialism, but the word
 \par 
TTTTTT areal i ls an ed the eee A Acai
 \par 
KRISTEN R. GHODSEE
 \par 
“Socialism” still carries negative connotations. Howls about Stalin’s Gulag and Ukrainian famines meet any mention of socialist principles. Opponents decry it as an economic system doomed to failure and inevitably leading to totalitarian terror, while ignoring the successful democratic socialist nations in Scandinavia. Europe was a battleground in the Cold War, and the northern European countries once had large, domestic communist and socialist parties that participated in the parliamentary process, promoting policies that ensured redistribution and social welfare. In the 1990s, while Russia, Hungary, and Poland liquidated state assets and dismantled their social safety nets, Denmark, Sweden, and Finland maintained generous public spending financed by government-owned industries and progressive taxation despite the global fashion for neoliberalism. The democratic socialist societies of Northern Europe show that it’s possible to find a humane alternative to neoliberal capitalism. And although they aren't perfect or easy to replicate— they are ethnically homogenous and increasingly hostile to immigrants—they have found ways to combine the political freedoms of the West with the social securities of the East.
 \par 
Northern Europe is not only the happiest place to live in the world but an oasis for women who enjoy more economic and political power than anywhere else on the planet. In a brilliant article in Dissent, “Cockblocked by Redistribution: A Pick-Up Artist in Denmark,” Katie J. M. Baker exposed how the American womanizer Daryush Valizadeh (aka Roosh) warned his fans that Denmark was a veritable desert for men on the hunt for easy women. The country’s generous social safety net and gender equalizing policies apparently render Valizadeh’s alpha male seduction
 \par 
17
 \par 
18
 \par 
YOU MIGHT BE SUFFERING FROM CAPITALISM techniques useless because Danish women don't need men for financial security. In less egalitarian countries, women understand that sexual relationships provide an avenue for social mobility—the Cinderella fantasy. But when women earn their own money and live in societies where the state supports their independence, Prince Charming loses his appeal. Roosh’s book, Don’t Bang Denmark, stands as a testament to the idea that redistributive policies can provide women the stability and security that mitigates the effects of discrimination in daily life.'” *
 \par 
Young people are rediscovering the idea that democratic governments have a role in ensuring a just economy. Today, corporations and wealthy elites influence politicians to do their bidding through campaign contributions and hired lobbyists: cut services for the poor to slash taxes for the rich. The 2010 Supreme Court decision in Citizens United v. FEC affirmed the idea that money equals speech and therefore deserved protection under the First Amendment of the Constitution. But as long as the United States remains a representative democracy, ordinary people can vote their economic interests and choose leaders who will pursue policies of redistribution and support social safety nets for all. By 2020, millennial voters will make up the largest demographic group of the American electorate. And young women make up half of the millennial population. The math here is simple.
 \par 
A June 2015 Gallup poll found that Americans ages eighteen to twenty-nine were more willing.to vote for a “socialist” presidential candidate than any other age cohort,
 \par 
KRISTEN R. GHODSEE and this was well before Bernie Sanders’s primary campaign was in full swing. In addition, a January 2016 YouGov poll asked Americans, “Do you have a favorable or an unfavorable opinion of socialism?” The results showed a stark difference in the opinions of different age cohorts. For those over the age of sixty-five, {\color{blue}60} percent had an unfavorable opinion of socialism, compared to the {\color{blue}23} percent that reported a favorable opinion. For those between the ages of thirty and sixty-four, about a quarter reported a positive idea of socialism, but half of thirty to forty-four-year-olds and {\color{blue}54} percent of forty-five to sixty-four-year-olds maintained a negative view. Among the eighteenth twenty nine-year-olds, only about one quarter had an unfavorable view of socialism. A whopping {\color{blue}43} percent had a favorable opinion, greater than the percentage of eighteenth twenty nine-year-olds who had a positive opinion of capitalism (32 percent)! A follow-up poll by the Victims of Communism Memorial Foundation in October 2017 found that support for socialism continued to increase among the young: “For starters, as of this year, more Millennials would prefer to live in a socialist country (44 percent) than in a capitalist one (42 percent). Or even a communist country (7 percent). The percentage of Millennials who would prefer socialism to capitalism is a full ten points higher than that of the general population. The significance of this finding cannot be overstated—as of last year, Millennials surpassed Baby Boomers as the largest generational cohort in American society.”
 \par 
This same study revealed fascinating gender differences in opinions on whether respondents viewed either capitalism or socialism as “favorable” or “unfavorable.” Of the
 \par 
19
 \par 
20
 \par 
YOU MIGHT BE SUFFERING FROM CAPITALISM
 \par 
2,300 Americans surveyed, women made up {\color{blue}51} percent of the sample, and their opinions often diverged significantly from those of men. When asked if they had a favorable view of capitalism as an economic system, {\color{blue}56} percent of men surveyed agreed compared to only {\color{blue}44} percent of women, a 12-percentage-point difference. Alternatively, {\color{blue}53} percent of men had an unfavorable view of socialism, compared to only {\color{blue}47} percent of women. Although men tended to have stronger political opinions overall, these gender differences suggest that women voters are more inclined toward redistributive policies. And these changes in political opinions are despite the efforts of conservative politicians to conflate all leftist ideals with the worst horrors of Stalinism. Perhaps millennials don’t trust the authority of the baby boomer cold warriors, or perhaps the economic realities of the present day, with growing inequality and stagnant earnings for the bottom half of the income distribution, are more real than ghost stories about an “evil empire” that fell before they were born.”
 \par 
George Orwell once wrote: “Who controls the past controls the future. Who controls the present controls the past.”’° Conservatives will do anything to suppress evidence that socialist experiments in the twentieth century (despite their collapse) did some good things for women, including policies that have been and can be implemented in democratic societies: paid maternity leaves, publicly funded child care, shorter and more flexible work weeks, free postsecondary education, universal health care, and other programs that would help both men and women to lead less precarious and more fulfilling lives. Many of these socialistic policies already exist in advanced Western countries,
 \par 
KRISTEN R. GHODSEE countries where Fox News and knee-jerk anticommunism don’t deter citizens from voting in their economic interests. The current hyperpolarized political climate mitigates against a more nuanced view of the past. Conservative critics care little about the history of twentieth-century state socialism and its policies toward women. They want to maintain the status quo. For instance, the Washington DC-based Victims of Communism Memorial Foundation claims that “an entire generation of Americans is open to collectivist ideas because they don’t know the truth. We tell the truth about communism because our vision is for a world free from the false hope of communism.” Notice the slippage between “collectivist ideas” and “communism” as if the former always and inevitably become the latter. (If I want to own my snow blower in common with my neighbors, it must be because I’m secretly hoping they'll get sent to the Gulag.) This foundation designs high school curricula, pays for anticommunist billboards on Times Square, and hopes to build a victim of communism museum near the National Mall in Washington, DC (with funding from explicitly right-wing donors). They want to control history in the same way that the Soviet Union once manufactured the past to suit its own political ends. If you challenge their single-minded focus on the worst aspects of the past, you challenge their assertion that socialism will always fail no matter how or where it is tried in the future.”
 \par 
Millennials and members of generation Z reject the Cold War baggage of their elders who once proclaimed, “Better dead than Red!” Young people wonder whether their lives would be less harried, insecure, and stressful if the government took a more active role in redistribution.
 \par 
21
 \par 
22
 \par 
YOU MIGHT BE SUFFERING FROM CAPITALISM
 \par 
They have incentives to vote for leaders who understand that markets boom and bust and that ordinary people need protection from the sudden and often savage fluctuations of free markets. Right-wing populist leaders will try to scapegoat women, people of color, and immigrants to deflect blame from the real roots of economic injustice: the high concentration of wealth in the hands of fewer and fewer people. As ordinary men and women struggle and scramble to cover their basic needs in an economy that promises equal opportunities for social mobility, but in which {\color{blue}78} percent of African American children born between 1985 and 2000 grew up in highly disadvantaged neighborhoods (compared to only {\color{blue}5} percent of white children), citizens must join together to effect real political change.”
 \par 
Let’s be clear: I don’t advocate a return to any form of twentieth-century state socialism. Those experiments failed under the weight of their own contradictions: the vast chasm between their stated ideals and the actual practices of authoritarian leaders. You shouldn't have to sacrifice toilet paper for medical care. Basic political freedoms don't need to be traded for guaranteed employment. But there were other paths not taken, such as those envisioned by early socialist theorists like Karl Liebknecht and Rosa Luxemburg. And no socialist experiment was ever allowed to flourish without facing the overt or covert opposition of the United States, whether direct confrontations like those in Korea and Vietnam or secret operations in places such as Cuba, Chile, or Nicaragua. Did somebody say, “Iran-Contra affair”? Besides, the historical circumstances of the twenty first century differ from those of the twentieth century. As our global economy evolves and changes in response to new
 \par 
KRISTEN R. GHODSEE technologies, citizens need access to a theoretical toolkit that contains the widest array of potential political solutions to the problems we will face in the coming years.
 \par 
Just as European peasants once believed that God anointed kings and aristocrats to rule over them, today many believe the super rich have earned their money in a fair competition in free markets. But as suspicions of the so-called rigged economy grow, more and more youth are searching for alternatives. The seventeenth-century philosopher Spinoza is supposed to have said, “If you want the future to be different from the present, study the past.” Even if past experiments with socialism failed, there were a few successes. We should study these successes and salvage what we can of the most powerful theoretical and practical tools we have to limit the worst excesses of global capitalism today. Young women in particular have little to lose and much to gain from a collective effort to build more just, equitable, and sustainable societies.” This book explains why.
 \par 
23
 \par 
\begin{ figure }
	\centering
	\\includegraphics[width=1.\textwidth]{ temp_files/images/UP_logo.png }
	\caption{Clara Zetkin (1857-1933): Editor of Die Gleichheit (Equality), a journal of the German Social Democratic Party, Zetkin was a key architect of social- ist women’s activism. She was the founder of International Women’s Day in 1910, celebrated each year on March {\color{blue}8}. After the outbreak of World War I, she split with the German Social Democratic Party and became active in the German Communist Party, serving as a member of the Constituent Assembly during the Weimar Republic. Zetkin believed that socialist men and women needed to work together to overthrow the bourgeoisie and dis- dained independent feminists. Courtesy of Archiv der sozialen Demokratie/ Friedrich-Ebert-Foundation.}
	\label{ }
\end{ figure }