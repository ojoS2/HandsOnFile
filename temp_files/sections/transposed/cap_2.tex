\chapter{Conservatism and Counterrevolution}\label{Conservatism and Counterrevolution}
 \par 
Whoever fi gets monsters should see to it that in the process he does not become a monster.
 \par 
When John McCain announced Sarah Palin as his running mate during the 2008 presidential campaign, voices in the conservative movement expressed surprise, even shock. It wasn’t just that McCain had chosen a political novice, an ingénue and outsider to the ways and means of governance in the lower forty-eight states. It was how he had chosen her: with little to no vetting and a great deal of faith in the superiority of intuition and impulse (his and hers) over reason and reflection. Furthermore, it was, it seemed, a most unconderivative decision: impetuous, ill-considered, imprudent.
 \par 
This was hardly the first time that a standard-bearer of conservatism failed to live up to the self-image of the conservative. In the spring of 2003, several conservatives voiced concern over the audacity of George W. Bush’s decision to fight what was essentially a war of choice. They also noted the liberal pedigree of one of the Iraq War’s justifications: spreading democracy and human rights. Here was a conservative leader, again it seemed, acting in the most
 \par 
This chapter originally appeared in Raritan 30, no. {\color{blue}1} (Summer 2010): 1–17.
 \par 
Unconservative of ways: jettisoning the realism of his father and his party for an internationalism long considered the exclusive property of the left, pressing the forward march of history against the status quo of the Middle East.
 \par 
Ever since Edmund Burke invented conservatism as an idea, the conservative has styled himself a man of prudence and moderation, his cause a sober—and sobering—recognition of limits. “To be conservative,” we heard Michael Oakeshott declare in the introduction, “is to prefer the familiar to the unknown. . . The tried to the untried, fact to mystery, the actual to the possible, the limited to the unbounded, the near to the distant.” {\color{blue}1} Yet the political efforts that have roused the conservative to his most profound real ections—the reactions against the French and Bolshevik revolutions; the defense of slavery and Jim Crow; the attack on social democracy and the welfare state; and the serial backlashes against the New Deal, the Great Society, civil rights, feminism, and gay rights— have been anything but that. Whether in Europe or the United States, in this century or previous ones, conservatism has been a forward movement of restless and relentless change, partial to risk taking and ideological adventurism, militant in its posture and populist in its bearings, friendly to upstarts and insurgents, outsiders and newcomers alike. While the conservative theorist claims for his tradition the mantle of prudence and moderation, there is a not-so-subterranean strain of imprudence and moderation running through that tradition—a strain that, however counterintuitive it seems, connects Sarah Palin to Edmund Burke.
 \par 
A consideration of this deeper strain of conservatism gives us a clearer sense of what conservatism is about. While conservatism is an ideology of reaction—originally against the French Revolution, more recently against the liberation movements of the sixties and seventies—that reaction has not been well understood. Far from yielding a knee-jerk defense of an unchanging old regime or
 \par 
A thoughtful traditionalism, the reactionary imperative presses conservatism in two rather different directions: first, to a critique and reconfiguration of the old regime; and second, to an absorption of the ideas and tactics of the very revolution or reform it opposes. What conservatism seeks to accomplish through that reconfiguration of the old and absorption of the new is to make privilege popular, to transform a tottering old regime into a dynamic, ideologically coherent movement of the masses. A new old regime, one could say, which brings the energy and dynamism of the street to the antique inequalities of a dilapidated estate.
 \par 
As the forty-year dominion of the right begins to fade, however fit fully, writers like Sam Tanenhaus, Andrew Sullivan, Jeff rey Hart, Sidney Blumenthal, and John Dean claim that conservatism went into decline when Palin, or Bush, or Reagan, or Goldwater, or Buckley, or someone took it off the rails. Originally, the argument goes, conservatism was a responsible discipline of the governing classes, but somewhere between Joseph de Maistre and Joe the Plumber, it got carried away with itself. It became adventurous, fanatical, populist, ideological. What this story of decline over-looks—whether it emanates from the right or the left—is that all of these supposed vices of contemporary conservatism were pre-sent at the beginning, in the writings of Burke and Maistre, only they weren’t viewed as vices. They were seen as virtues. Conservatism has always been a wilder and more extravagant movement than many realize—and it is precisely this wildness and extravagance that has been one of the sources of its continuing appeal.
 \par 
It is hardly provocative to say that conservatism arose in reaction to the French Revolution. Most historically minded conservatives would agree. {\color{blue}2} But if we look more carefully at two emblematic voices of that reaction—Burke and Maistre—we find several surprising and seldom-noticed elements. The first is an antipathy,
 \par 
Bordering on contempt, for the old regime they claim as their cause. The opening chapters of Maistre’s Considerations on France are an unrelenting assault on the three pillars of the ancien régime: the aristocracy, the church, and the monarchy. Maistre divides the nobility into two categories: the treasonous and the clueless. The clergy is corrupt, weakened by its wealth and lax morals. The monarchy is soft and lacks the will to punish. Maistre dismisses all three with a line from Racine: “Now see the sad fruits your faults produced, / Feel the blows you have yourselves induced.”{\color{blue}3}
 \par 
In Burke’s case, the criticism is subtler but runs deeper. (Though by the end of his life, he was speaking in the same modulated tones as Maistre.) {\color{blue}4} It comes during his account in Refl ections on the Revolution in France of the storming of the palace at Versailles and the capture of the royal family. There, Burke describes Marie Antoinette as a “delightful vision. . . Glittering like the morning star, full of life, and splendor, and joy.” Burke takes her beauty as a symbol of the loveliness of the old regime, where feudal manners and mores “made power gentle” and “by a bland assimilation, incorporated into politics the sentiments which beautify and soften private society.”{\color{blue}5}
 \par 
Ever since he wrote those lines, Burke has been mocked for his sentimentality. But readers of Burke’s earlier work on aesthetics, A Philosophical Inquiry into the Origins of Our Ideas of the Sublime and the Beautiful, will know that beauty, for Burke, is never a sign of power’s vitality; it is always a sign of decadence. Beauty arouses pleasure, which gives way to indifference or leads to a total dissolution of the self. “Beauty acts,” Burke writes, “by relaxing the solids of the whole system.” {\color{blue}6} It is this relaxation and dissolution of bodies—physical, social, political bodies—that makes beauty such a potent symbol and agent of degeneration and death. “Our most salutary and most beautiful institutions yield nothing but dust and smut.”{\color{blue}7}
 \par 
What these two opening statements of the conservative persuasion suggest is that the greatest enemy of the old regime is
 \par 
Neither the revolutionary nor the reformer; it is the old regime itself or, to be more precise, the defenders of the old regime. {\color{blue}8} They simply lack the ideological wherewithal to press the cause of the old regime with the requisite vigor, clarity, and purpose. As Burke declared of George Grenville, in the very different context of Britain't relationship with its American colonies:
 \par 
But it may be truly said, that men too much conversant in off ice, are rarely minds of remarkable enlargement. . . . Persons who are nurtured in off ice do admirably well as long as things go on in their common order; but when the high roads are broken up, and the waters out, when a new and troubled scene is opened, and the file affords no precedent, then it is that a greater knowledge of mankind, and a far more extensive comprehension of things, is requisite, than ever off ice gave, or than off ice can ever give.{\color{blue}9}
 \par 
Later conservatives will make this claim in various ways. Some-times they’ll accuse the defenders of the old regime of having been cowed by the revolutionary or reformist challenge. According to Thomas Dew, one of the earliest and most aggressive apologists for American slavery, the Nat Turner rebellion destroyed “all feeling of security and confidence” among the master class. So frightened were they that “reason was almost banished from the mind.” It wasn’t just the slaves’ violence that frightened them. It was the moral indictment leveled by the slaves and the abolitionists, which had somehow insinuated itself into the slaveholders’ minds and made them unsure of their own position. “We our-selves,” wrote William Harper, another defender of slavery, “have in some measure pleaded guilty to the impeachment.”{\color{blue}10}
 \par 
More than a century later, Barry Goldwater would take up the same theme. The very first paragraph of The Conscience of a Conservative directs its fire not at liberals or Democrats or even the
 \par 
Welfare state; it is aimed at the moral timidity of what will later be called the “Republican Establishment.”
 \par 
I have been much concerned that so many people today with Conservative instincts feel compelled to apologize for them. Or if not to apologize directly, to qualify their commitment in a way that amounts to breast beating. “Republican candidates,” Vice President Nixon has said, “should be economic conservatives, but conservatives with a heart.” President Eisenhower announced during his first term, “I am conservative when it comes to economic problems but liberal when it comes to human problems.” . . . These formulations are tantamount to an admission that Conservatism is a narrow, mechanistic economic theory that may work very well as a bookkeeper’s guide, but cannot be relied upon as a comprehensive political philosophy.{\color{blue}11}
 \par 
More often, conservatives have argued that the defender of the old regime is simply obtuse. He has grown lazy, fat, and complacent, so roundly enjoying the privileges of his position that he cannot see the coming catastrophe. Or, if he can see it, he can’t do anything to fend it off, his political muscles having atrophied long ago. John C. Calhoun was one such conservative, and throughout the 1830s, when the abolitionists began pressing their cause, he drove himself into a rage over the easy living and willful cluelessness of his comrades on the plantation. His fury reached a peak in 1837, when, in a speech on the Senate floor, he urged Congress not to receive an abolitionist petition—a moment, as we saw in the introduction, that he would remember to his dying day. “All we want is concert,” he pleaded with his fellow Southerners, to “unite with zeal and energy in repelling approaching dangers.” But, he went on, “I dare not hope that anything I can say will arouse the South to a due sense of danger. I fear it is beyond the power of the
 \par 
Mortal voice to awaken it in time from the fatal security into which it has fallen.”{\color{blue}12}
 \par 
In his influential essay, Oakeshott argued that conservatism “is not a creed or a doctrine, but a disposition.” Specifi cally, he thought, it is a disposition to enjoy the present. Not because the present is better than the alternatives or even because it is good on its own terms. That would imply a level of conscious reflection and ideological choice that Oakeshott believes is alien to the conservative. No, the reason the conservative enjoys the present is simply and merely because it is familiar, because it is there, because it is at hand.{\color{blue}13}
 \par 
Oakeshott’s view of the conservative—and this view is widely shared on both the left and the right—is not an insight; it is a conceit. It overlooks the fact that conservatism invariably arises in response to a threat to the old regime or after the old regime has been destroyed. (Oakeshott openly admits that loss or threatened loss makes us value the present, as I argued in the introduction, but he does not allow that insight to penetrate or dislodge his overall understanding of conservatism.) Oakeshott is describing the old regime in an easy chair, when its mortality is a distant notion and time is a warming medium rather than an acrid solvent. This is the old regime of Charles Loyseau, who wrote nearly two centuries before the French Revolution that the nobility has no “beginning” and thus no end. It “exists time out of mind,” without consciousness or awareness of the passage of history.{\color{blue}14}
 \par 
Conservatism appears on the scene precisely when—and precisely because—such statements can no longer be made. Walter Berns, one of the many future neoconservatives at Cornell who were traumatized in 1969 by the black students, takeover of Willard Straight Hall, stated in his farewell speech when he resigned from the university: “We had too good a world; it couldn’t last.” {\color{blue}15} Nothing so disturbs the idyll of inheritance as the sudden and often brutal replacement of one world with another. Having witnessed
 \par 
The death of what was supposed to live forever, the conservative can no longer look upon time as the natural ally or habitat of power. Time is now the enemy. Change, not permanence, is the universal governor, with change signifying neither progress nor improvement but death, and an early, unnatural death at that. “The decree of violent death,” says Maistre, is “written on the very frontiers of life.” {\color{blue}16} The problem with the defender of the old regime, says the conservative, is that he doesn’t know this truth or, if he does, he lacks the will to do anything about it.
 \par 
The second element we find in these early voices of reaction is a surprising admiration for the very revolution they are writing against. Maistre’s most rapturous comments are reserved for the Jacobins, whose brutal will and penchant for violence—their “black magic”—he plainly envies. The revolutionaries have faith, in their cause and themselves, which transforms a movement of mediocrities into the most implacable force Europe has ever seen. Thanks to their efforts, France has been purified and restored to its rightful pride of place among the family of nations. “The revolutionary government,” Maistre concludes, “hardened the soul of France by tempering it in blood.”{\color{blue}17}
 \par 
Burke, again, is more subtle but cuts more deeply. Great power, he suggests in The Sublime and the Beautiful, should never aspire to be—and can never actually be—beautiful. What great power needs is sublimity. The sublime is the sensation we experience in the face of extreme pain, danger, or terror. It is something like awe but tinged with fear and dread. Burke calls it “delightful horror.” Great power should aspire to sublimity rather than beauty because sublimity produces “the strongest emotion which the mind is capable of feeling.” It is an arresting yet invigorating emotion, which has the simultaneous but contradictory effect of diminishing and magnifying us. We feel annihilated by great power; at the same
 \par 
Time, our sense of self “swell[s]” when “we are conversant with terrible objects.” Great power achieves sublimity when it is, among other things, obscure and mysterious, and when it is extreme. “In all things,” writes Burke, the sublime “abhors mediocrity.”{\color{blue}18}
 \par 
In the Refl ections, Burke suggests that the problem in France is that the old regime is beautiful while the revolution is sublime. The landed interest, the cornerstone of the old regime, is “sluggish, inert, and timid.” It cannot defend itself “from the invasions of ability,” with ability standing in here for the new men of power that the revolution brings forth. Elsewhere in the Refl ections, Burke says that the moneyed interest, which is allied with the revolution, is stronger than the aristocratic interest because it is “more ready for any adventure” and “more disposed to new enterprises of any kind.” The old regime, in other words, is beautiful, static, and weak; the revolution is ugly, dynamic, and strong. And in the horrors that the revolution perpetrates—the rabble rushing into the bedchamber of the queen, dragging her half-naked into the street, and marching her and her family to Paris—the revolution achieves a kind of sublimity: “We are alarmed into real exion,” writes Burke of the revolutionaries’ actions. “Our minds. . . Are purified by terror and pity; our weak unthinking pride is humbled, under the dispensations of a mysterious wisdom.”{\color{blue}19}
 \par 
Beyond these simple professions of envy or admiration, the conservative actually copies and learns from the revolution he opposes. “To destroy that enemy,” Burke wrote of the Jacobins, “by some means or other, the force opposed to it should be made to bear some analogy and resemblance to the force and spirit which that system exerts.” {\color{blue}20} This is one of the most interesting and least understood aspects of conservative ideology. While conservatives are hostile to the goals of the left, particularly the empowerment of society’s lower castes and classes, they are often the left’s best students. Sometimes, their studies are self-conscious and strategic,
 \par 
As they look to the left for ways to bend new vernaculars, or new media, to their suddenly legitimated aims. Fearful that the philosophes had taken control of popular opinion in France, reactionary theologians in the middle of the eighteenth century looked to the example of their enemies. They stopped writing abstruse disquisitions for each other and began to produce Catholic agitprop, which would be distributed through the very networks that brought enlightenment to the French people. They spent vast sums funding essay contests, like those in which Rousseau made his name, to reward writers who wrote accessible and popular defenses of religion. Previous treatises of faith, declared Charles-Louis Richard, were “useless to the multitude who, without arms and without defenses, succumbs rapidly to Philosophie.” His work, by contrast, was written “with the design of putting in the hands of all those who know how to read a victorious weapon against the assaults of this turbulent Philosophie.”{\color{blue}21}
 \par 
Pioneers of the Southern Strategy in the Nixon administration, to cite a more recent example, understood that after the rights revolutions of the sixties they could no longer make simple appeals to white racism. From now on, they would have to speak in code, preferably one palatable to the new dispensation of color-blindless. As White House chief of staff H. R. Haldeman noted in his diary, Nixon “emphasized that you have to face the fact that the whole problem is really the blacks. The key is to devise a system that recognized this while not appearing to.” {\color{blue}22} Looking back on this strategy in 1981, Republican strategist Lee Atwater spelled out its elements more clearly:
 \par 
You start out in 1954 by saying, “Nigger, nigger, nigger.” By 1968 you can’t say “nigger”—that hurts you. Backfi res. So you say stuff like forced busing, states’ rights and all that stuff. You’re getting so abstract now you’re talking about cutting taxes, and
 \par 
All these things you’re talking about are totally economic things and a by-product of them is blacks get hurt worse than whites. And subconsciously maybe that is part of it.{\color{blue}23}
 \par 
More recently still, David Horowitz has encouraged conservative students “to use the language that the left has deployed so effectively in behalf of its own agendas. Radical professors have created a ‘hostile learning environment’ for conservative students. There is a lack of ‘intellectual diversity’ on college faculties and in academic classrooms. The conservative viewpoint is ‘underrepresented’ in the curriculum and on its reading lists. The university should be an ‘inclusive’ and intellectually ‘diverse’ community.”{\color{blue}24}
 \par 
At other times, the education of the conservative is unknowing, happening, as it were, behind his back. By resisting and thus engaging with the progressive argument day after day, he comes to be influenced, often in spite of himself, by the very movement he opposes. Setting out to bend a vernacular to his will, he finds his will bent by the vernacular. Atwater claims this is precisely what occurred within the Republican Party; after suggesting “subconsciously maybe that is part of it.” He adds:
 \par 
I’m not saying that. But I’m saying that if it is getting that abstract, and that coded, that we are doing away with the racial problem one way or the other. You follow me—because obviously sitting around saying, “We want to cut this,” is much more abstract than even the busing thing, and a hell of a lot more abstract than “Nigger, nigger.”{\color{blue}25}
 \par 
Republicans have learned to disguise their intentions so well, Atwater argues, that the disguise has seeped into and transformed the intention. Assuming such a transformation has indeed occurred,
 \par 
We might well ask whether the conservative has ceased to be what he set out to be. But that is a question for another day.
 \par 
Even without directly engaging the progressive argument, conservatives may absorb, by some elusive osmosis, the deeper categories and idioms of the left, even when those idioms run directly counter to their official stance. After years of opposing the women’s movement, for example, Phyllis Schlafl y seemed genuinely incapable of conjuring the prefeminist view of women as deferential wives and mothers. Instead, she celebrated the activist “power of the positive woman.” And then, as if borrowing a page from The Feminine Mystique, she railed against the meaninglessness and lack of fulfillment among American women; only she blamed these ills on feminism rather than on sexism. {\color{blue}26} When she spoke out against the Equal Rights Amendment (ERA), she didn’t claim that it introduced a radical new language of rights. Her argument was the opposite. The ERA, she told the Washington Star, “is a takeaway of women’s rights.” It will “take away the right of the wife in an ongoing marriage, the wife in the home.” {\color{blue}27} Schlafl y was obviously using the language of rights in a way that was opposed to the aims of the feminist movement; she was using rights talk to put women back into the home, to keep them as wives and mothers. But that is the point: conservatism adapts and adopts, often unconsciously, the language of democratic reform to the cause of hierarchy.
 \par 
One also can detect a certain sexual frankness—even feminist concern—in the early conversations of the Christian Right that would have been unthinkable prior to the women’s movement. In 1976, Beverly and Tim LaHaye wrote a book, The Act of Marriage, which Susan Faludi has rightly called “the evangelical equivalent of The Joy of Sex.” There, the LaHayes claimed that “women are much too passive in lovemaking.” God, the LaHayes told their female readers, “placed [your clitoris] there for your enjoyment.” They also complained that “some husbands are carryovers from the
 \par 
Dark Ages, like the one who told his frustrated wife, ‘Nice girls aren’t supposed to climax.’ Today’s wife knows better.”{\color{blue}28}
 \par 
What the conservative ultimately learns from his opponents, wit-tingly or unwittingly, is the power of political agency and the potency of the mass. From the trauma of revolution, conservatives learn that men and women, whether through willed acts of force or some other exercise of human agency, can order social relation-ships and political time. In every social movement or revolutionary moment, reformers and radicals have to invent—or rediscover— the idea that inequality and social hierarchy are not natural phenomena but human creations. If hierarchy can be created by men and women, it can be uncreated by men and women, and that is what a social movement or revolution sets out to do. From these efforts, conservatives learn a version of the same lesson. Where their predecessors in the old regime thought of inequality as a naturally occurring phenomenon, an inheritance passed on from generation to generation, the conservatives’ encounter with revolution teaches them that the revolutionaries were right after all: inequality is a human creation. And if it can be uncreated by men and women, it can be recreated by men and women.
 \par 
“Citizens!” exclaims Maistre at the end of Considerations on France. “This is how counterrevolutions are made.” {\color{blue}29} Under the old regime, monarchy—like patriarchy or Jim Crow—isn’t made. It just is. It would be difficult to imagine a Loyseau or Bossuet declaring, “Men”—much fewer citizens—“this is how a monarchy is made.” But once the old regime is threatened or toppled, the conservative is forced to realize that it is a human agency, the willed imposition of intellect and imagination upon the world, that generates and maintains inequality across time. Coming out of his confrontation with the revolution, the conservative voices the kind of affirmation of political agency one finds in this 1957 editorial
 \par 
From William F. Buckley’s National Review : “The central question that emerges” from the civil rights movement “is whether the White community in the South is entitled to take such measures as are necessary to prevail, politically and culturally, in areas in which it does not predominate numerically? The sobering answer is Yes— the White community is so entitled because, for the time being, it is the advanced race.”{\color{blue}30}
 \par 
The revolutionary declares the Year I, and in response the conservative declares the Year Negative I. From the revolution, the conservative develops a particular attitude toward political time, a belief in the power of men and women to shape history, to propel it forward or backward; and by virtue of that belief, he comes to adopt the future as his preferred tense. Ronald Reagan off and the perfect distillation of this phenomenon when he invoked, repeatedly, Thomas Paine’s dictum that “we have it in our power to begin the world over again.” {\color{blue}31} Even when the conservative claims to be preserving a present that’s threatened or recovering a past that’s lost, he is impelled by his own activism and agency to confess that he’s making a new beginning and creating the future.
 \par 
Burke was especially attuned to this problem and so was often at pains to remind his comrades in the battle against the Revolution that whatever was rebuilt in France after the restoration would inevitably, as he put it in a letter to a émigré, “be in some measure a new thing.” {\color{blue}32} Other conservatives have been less ambivalent, happily affirming the virtues of political creativity and moral originality. Alexander Stephens, vice president of the U.S. Confederacy, proudly declared that “our new government is the first, in the history of the world” to be founded upon the “great physical, philosophical, and moral truth” that “the negro is not equal to the white man; that slavery—subordination to the superior race—is his natural and normal condition.” {\color{blue}33} Barry Goldwater said simply, “Our future, like our past, will be what we make it.”{\color{blue}34}
 \par 
From revolutions, conservatives also develop a taste and talent for the masses, mobilizing the street for spectacular displays of power while making certain power is never truly shared or redistributed. That is the task of right-wing populism: to appeal to the mass without disrupting the power of elites or, more precisely, to harness the energy of the mass in order to reinforce or restore the power of elites. Far from being a recent innovation of the Christian Right or the Tea Party movement, reactionary populism runs like a red thread throughout conservative discourse from the very beginning.
 \par 
Maistre was a pioneer in the theater of mass power, imagining scenes and staging dramas in which the lowest of the low could see themselves reflected in the highest of the high. “Monarchy,” he writes, “is without contradiction, the form of government that gives the most distinction to the greatest number of persons.” Ordinary people “share” in its “brilliance” and glow, though not, Maistre is careful to add, in its decisions and deliberations: “man is honored not as an agent but as a portion of sovereignty.” {\color{blue}35} Arch-monarchist that he was, Maistre understood that the king could never return to power if he did not have a touch of the plebeian about him. So when Maistre imagines the triumph of the counterrevolution, he takes care to emphasize the populist credentials of the returning monarch. The people should identify with this new king, says Maistre, because like them, he has attended the “terrible school of misfortune” and suffered in the “hard school of adversity.” He is “human,” with humanness here connoting an almost pedestrian, and reassuring, capacity for error. He will be like them. Unlike his predecessors, he will know it which “is a great deal.”{\color{blue}36}
 \par 
But to appreciate fully the inventiveness of right-wing populism, we have to turn to the master class of the Old South. The slaveholder created a quintessential form of democratic feudalism,
 \par 
Turning the white majority into a lordly class, sharing in the privileges and prerogatives of governing the slave class. Though the members of this ruling class knew that they were not equal to each other, they were compensated by the illusion of superiority—and the reality of rule—over the black population beneath them.
 \par 
One school of thought—call it the equal opportunity school— located the democratic promise of slavery in the fact that it put the possibility of personal mastery within the reach of every white man. The genius of the slaveholders, wrote Daniel Hundley in his Social Relations in Our Southern States, is that they are “not an exclusive aristocracy. Every free white man in the whole Union has just as much right to become an Oligarch.” This was not just propaganda: by 1860, there were 400,000 slaveholders in the South, making the American master class one of the most democratic in the world. The slaveholders repeatedly attempted to pass laws' encouraging whites to own at least one slave and even considered granting tax breaks to facilitate such ownership. Their thinking, in the words of one Tennessee farmer, was that “the minute you put it out of the power of common farmers to purchase a Negro man or woman. . . You make him an abolitionist at once.”{\color{blue}37}
 \par 
That school of thought contended with a second, arguably more influential, school. American slavery was not democratic, according to this line of thinking, because it off and the opportunity for personal mastery to white men: American slavery was democratic because it made every white man, slaveholder or not, a member of the ruling class by virtue of the color of his skin. In the words of Calhoun: “With us the two great divisions of society are not the rich and poor, but white and black; and all the former, the poor as well as the rich, belong to the upper class, and are respected and treated as equals.” {\color{blue}38} Or as his junior colleague James Henry Hammond put it, “In a slave country every freeman is an aristocrat.” {\color{blue}39} Even without slaves or the material prerequisites for
 \par 
Freedom, a poor white man could style himself a member of the nobility and thus be relied upon to take the necessary measures in its defense.
 \par 
Whether one subscribed to the first or second school of thought, the master class believed that democratic feudalism was a potent counter to the egalitarian movements then roiling Europe and Jacksonian America. European radicals, declared Dew, “wish all mankind to be brought to one common level. We believe slavery, in the United States, has accomplished this.” By freeing whites from “menial and low off ices,” slavery had eliminated “the greatest cause of distinction and separation of the ranks of society.” {\color{blue}40} As the nineteenth-century ruling classes contended with challenge after challenge to their power, the master class off and up racial domination as a way of harnessing the energy of the white masses, in support of, rather than in opposition to, the privileges and powers of established elites. This program would find its ultimate fulfillment a century later and a continent away.
 \par 
These populist currents can help us make sense of a final element of conservatism. From the beginning, conservatism has appealed to and relied upon outsiders. Maistre was from Savoy, Burke from Ireland. Alexander Hamilton was born out of wedlock in Nevis and rumored to be part black. Disraeli was a Jew, as are many of the neoconservatives who helped transform the Republican Party from a cocktail party in Darien into the party of Scalia, d’Souza, Gonzalez, and Yoo. (It was Irving Kristol who first identified “the historical task and political purpose of neoconservatism” as the conversion of “the Republican Party, and American conservatism in general, against their respective wills, into a new kind of conservative politics suitable to governing a modern democracy.”) {\color{blue}41} Allan Bloom was a Jew and a homosexual. And as she never tired of reminding us during the 2008 campaign, Sarah Palin is a woman in a world of
 \par 
Men, an Alaskan who said no to Washington (though she really didn’t), a maverick who rode shotgun to another maverick.
 \par 
Conservatism has not only depended upon outsiders; it also has seen itself as the voice of the outsider. From Burke’s cry that “the gallery is in the place of the house” to Buckley’s complaint that the modern conservative is “out of place,” the conservative has served as a tribune for the displaced, his movement a conveyance of their grievances. {\color{blue}42} Far from being an invention of the politically correct, victimhood has been a talking point of the right ever since Burke decried the mob’s treatment of Marie Antoinette. The conservative, to be sure, speaks for a special type of victim: one who has lost something of value, as opposed to the wretched of the earth, whose chief complaint is that they never had anything to lose. His constituency is the contingently dispossessed—William Graham Sumner’s “forgotten man”—rather than the preternaturally oppressed. Far from diminishing his appeal, this brand of victimhood endows the conservative complaint with a more universal significance. It connects his disinheritance to an experience we all share—namely, loss—and threads the strands of that experience into an ideology promising that that loss, or at least some portion of it, can be made whole.
 \par 
People on the left often fail to realize this, but conservatism really does speak to and for people who have lost something. It may be a landed estate or the privileges of white skin, the unquestioned authority of a husband or the untrammeled rights of a factory owner. The loss may be as material as money or as ethereal as a sense of standing. It may be a loss of something that was never legitimately owned in the first place; it may, when compared with what the conservative retains, be small. Even so, it is a loss, and nothing is ever so cherished as that which we no longer possess. It used to be one of the great virtues of the left that it alone under-stood the often zero-sum nature of politics, where the gains of one
 \par 
Class necessarily entail the losses of another. But as that sense of conflict diminishes on the left, it has fallen to the right to remind voters that there really are losers in politics and that it is they—and only they—who speak for them. “All conservatism begins with loss,” Andrew Sullivan rightly notes, which makes conservatism not the Party of Order, as Mill and others have claimed, but the party of the loser.{\color{blue}43}
 \par 
The chief aim of the loser is not—and indeed cannot be—preservation or protection. It is recovery and restoration. That, I believe, is one of the secrets of conservatism’s success. For all of its demotic frisson and ideological grandiosity, for all of its insistence upon triumph and will, movement and mobilization, conservatism can be an ultimately pedestrian aff air. Because his losses are recent—the right agitates against reform in real time, not millennia after the fact—the conservative can credibly claim to his constituency, indeed to the polity at large, that his goals are practical and achievable. He merely seeks to regain what is his, and the fact that he once had it—indeed, probably had it for some time— suggests that he is capable of possessing it again. “It is not an old structure,” Burke declared of Jacobin France, but “a recent wrong.” {\color{blue}44} Where the left’s program of redistribution raises whether its beneficiaries are truly prepared to wield the powers they seek, the conservative project of restoration suffers from no such challenge. Unlike the reformer or the revolutionary, moreover, who faces the nearly impossible task of empowering the powerless—that is, of turning people from what they are into what they are not—the conservative merely asks his followers to do more of what they always have done (albeit, better and differently). As a result, his counterrevolution will not require the same disruption that the revolution has visited upon the country. “Four or five persons, perhaps,” writes Maistre, “will give France a king.”{\color{blue}45}
 \par 
For some, perhaps many, in the conservative movement, this knowledge comes as a source of relief: their sacrifice will be small, their reward great. For others, it is a source of bitter disappointment. To this subset of activists and militants, the battle is all. To learn that it soon will be over and will not require so much from them is enough to prompt a complex of despair: disgust over the shabbiness of their effort, grief over the disappearance of their foe, anxiety over the early retirement into which they have been forced. As Irving Kristol complained after the end of the Cold War, the defeat of the Soviet Union and the left more generally “deprived” conservatives like himself “of an enemy,” and “in politics, being deprived of an enemy is a very serious matter. You tend to get relaxed and dispirited. Turn inward.” {\color{blue}46} Depression haunts conservatism as surely as does great wealth. But again, far from diminishing the appeal of conservatism, this darker dimension only enhances it. Onstage, the conservative waxes Byronic, moodily surveying the sum of his losses before an audience of the lovelorn and the starstruck. Off-stage, and out of sight, his managers quietly compile the sum of their gains.