\chapter{Affi native Action Baby}\label{Affi native Action Baby}
 \par 
Next to Clarence Thomas, Antonin Scalia is the most conservative justice on the Supreme Court. He also loves the television show {\color{blue}24}. “Boy, those early seasons,” he tells his biographer, “I’d be up to two o’clock, because you’re at the end of one [episode], and you’d say, ‘No, I’ve got to see the next.’” Scalia is especially taken with Jack Bauer, the show’s fictional hero played by Kiefer Sutherland. Bauer is a government agent at a Los Angeles counterterrorism unit who foils mass-murder plots by torturing suspects, kidnapping innocents, and executing colleagues. Refusing to be bound by the law, he fights a two-front war against terrorism and the Constitution. And whenever he bends a rule or breaks a bone, Scalia swoons.
 \par 
Jack Bauer saved Los Angeles. . . . He saved hundreds of thou-sands of lives. . . . Are you going to convict Jack Bauer? Say that criminal law is against him? You have the right to a jury trial? Is any jury going to convict Jack Bauer? I don’t think so. So the question is really whether we really believe in these absolutes. And ought we believe in these absolutes? {\color{blue}1}
 \par 
Yet Scalia has spent the better part of his career as a lawyer, professor, and jurist telling us that the Constitution is an absolute, in which we must believe, even when—particularly when—it tells us something we do not want to hear. Scalia’s Constitution is not a warming statement of benevolent purpose, easily adapted to our changing needs. His Constitution is cold and dead, its prohibitions and injunctions frozen in time. Phrases like “cruel and unusual punishment” mean what they meant when they were written into the Constitution. If that produces objectionable results—say, the execution of children and the mentally retarded—too bad. “I do not think,” Scalia writes in Nixon v. Missouri Municipal League, that “the avoidance of unhappy consequences is adequate basis for interpreting a text.”{\color{blue}2}
 \par 
Scalia takes special pleasure in unhappy consequences. He relishes difficulty and dislikes anyone who would diminish or deny it. In Hamdi v. Rumsfeld, a plurality of the Court took what Scalia thought was a squishy position on executive power during wartime. The Court ruled that the Authorization for the Use of Military Force, passed by Congress after 9/11, empowered the president to detain U.S. citizens indefinitely as “illegal enemy combatants” without trying them in a court of law. It also ruled, however, that such citizens were entitled to due process and could challenge their detention before some kind of tribunal.
 \par 
Scalia was livid. Writing against the plurality—as well as the Bush administration and fellow conservatives on the Court—he insisted that a government at war, even one as unconventional as the war on terror, had two, and only two, ways to hold a citizen: try him in a court of law or have Congress suspend the writ of habeas corpus. Live by the rules of due process, in other words, or suspend them. Take a stand, make a choice.
 \par 
But the Court weaseled out of that choice, making life easier for the government and itself. Congress and the president could act
 \par 
As if habeas corpus were suspended, without having to suspend it, and the Court could act as if the writ hadn’t been suspended thanks to a faux due process of military tribunals. More than coloring outside the lines of the Constitution, it was the Court’s “Mr. Fix-It Mentality,” in Scalia’s words, its “mission to Make Everything Come Out Right,” that enraged him.{\color{blue}3}
 \par 
Scalia’s mission, by contrast, is to make everything come out wrong. A Scalia opinion, to borrow a phrase from New Yorker writer Margaret Talbot, is “the jurisprudential equivalent of smashing a guitar on stage.” {\color{blue}4} Scalia may have once declared the rule of law the law of rules—leading some to mistake him for a stereotypical conservative—but rules and laws have a particular frisson for him. Where others look to them for stabilizing checks or reassuring supports, Scalia looks for exhilarating impediments and vertiginous barriers. Where others seek security, Scalia seeks sublimity. Rules and laws make life harder, and harder is every-thing. “Being tough and traditional is a heavy cross to bear,” he tells one reporter. “ Duresse oblige.”{\color{blue}5}
 \par 
That, and not fidelity to the text or conservatism as it is conventionally understood, is the idée fixe of Scalia’s jurisprudence—and the source of his apparent man-crush on Jack Bauer. Bauer never makes things easy for himself; indeed, he goes out of his way to make things as hard as possible. He volunteers for a suicide mission when someone else would do (and probably do it better); he turns himself into a junkie as part of an impossibly baroque plan to stop an act of bioterrorism; he puts his wife and daughter at risk, not once but many times, and then beats himself up for doing so. He loathes what he does but does it anyway. That is his nobility—some might say masochism—and why he warms Scalia’s heart.
 \par 
It means something, of course, that Scalia identifies the path of most resistance in fidelity to an ancient text, while Bauer finds it in betrayal of that text. But not as much as one might think: as we’ve
 \par 
Come to learn from the marriages of our right-wing preachers and politicians, fidelity is often another word for betrayal.
 \par 
Scalia was born in Trenton, New Jersey, in March 1936, but he was conceived the previous summer in Florence, Italy. (His father, a doctoral student in romance languages at Columbia, had won a fellowship to travel there with his wife.) “I hated Trenton,” Scalia says; his heart belongs to Florence. A devotee of opera and hunting—“he loves killing unarmed animals,” observes Clarence Thomas—Scalia likes to cut a Medicean profile of great art and great cruelty. He peppers his decisions with stylish allusions to literature and history. Once upon a time, he enjoys telling audiences, he was too “faint-hearted” an originalism to uphold the eighteenth century’s acceptance of ear notching and flogging as forms of punishment. Not anymore. “I’ve gotten older and crankier,” he says, ever the diva of disdain.{\color{blue}6}
 \par 
When Scalia was six, his parents moved to the Elmhurst section of Queens. His lifelong conservatism is often attributed to his strict Italian Catholic upbringing there; alluding to Burke, he calls it his “little platoon.” He attended Xavier High School, a Jesuit school in Manhattan, and Georgetown, a Jesuit university in Washington, D.C. In his freshman year at Georgetown, the senior class voted Senator Joseph McCarthy as the Outstanding American.{\color{blue}7}
 \par 
But Scalia comes to his ethnicity and religion with an attitude, lending his ideology a defibrillator ant edge. (That defiance is often thought to be distinctive, out of keeping with conservative manners and mores; but as we have seen, it’s not.) He claims he didn’t get into Princeton, his first choice, because “I was an Italian boy from Queens, not quite the Princeton type.” Later, after Vatican II liberalized the liturgy and practices of the Church, including his neighborhood church in suburban Washington, D.C., he insisted on driving his brood of seven children miles
 \par 
Away to hear Sunday Mass in Latin. Later still, in Chicago, he did the same thing, only this time with nine children in tow. Commenting on how he and his wife managed to raise conservative children during the sixties and seventies—no jeans in the Scalia household—he says:
 \par 
They were being raised in a culture that wasn’t supportive of our values, that was certainly true. But we were helped by the fact that we were such a large family. We had our own culture. . . The first thing you’ve got to teach your kids is what my parents used to tell me all the time, “You’re not everybody else. . . . We have our own standards, and they aren’t the standards of the world in all respects, and the sooner you learn that the better.”{\color{blue}8}
 \par 
Scalia’s conservatism, it turns out, is less a little platoon than a Thoreauvian counterculture, a retreat from and rebuke to the main-stream, not unlike the hippie communes and groupuscules he once tried to keep at bay. It is not a conservatism of tradition or inheritance: his parents had only one child, and his mother-in-law often complained about having to drive miles and hours in search of the one true church. “Why don’t you people ever seem to live near churches?” she would ask Scalia and his wife. {\color{blue}9} It is a conservatism of invention and choice, informed by the very spirit of rebellion he so plainly loathes—or thinks he loathes—in the culture at large.
 \par 
In the 1970s, while teaching at the University of Chicago, Scalia liked to end the semester with a reading from A Man for All Seasons, Robert Bolt’s play about Thomas More. While the play’s antiauthoritarianism would seem at odds with Scalia’s conservatism, its protagonist, at least as he is portrayed by Bolt, is not. Literally more Catholic than the pope, More is a true believer in the law who refuses to compromise his principles in order to accommodate the wishes of Henry VIII. He pays for his integrity with his life.
 \par 
Scalia’s biographer introduces this biographical tidbit with a revealing setup: “Yet even as Scalia in middle age was developing a more rigid view of the law, he still had bursts of idealism.” {\color{blue}10} That “yet” is misplaced. Scalia’s rigidity is not opposed to his idealism; it is his idealism. His ultraconservative reading of the Constitution reflects neither cynicism nor conventionalism; orthodoxy and piety are, for him, the essence of dissidence and iconoclasm. No charge grieves him more than the claim, rehearsed at length in his 1995 Tanner Lectures at Princeton, that his philosophy is “wooden,” “unimaginative,” “pedestrian,” “dull,” “narrow,” and “hidebound.” {\color{blue}11} Call him a bastard or a prick, a hound from hell or a radical in robes. Just don’t say he’s a suit.
 \par 
Scalia’s philosophy of Constitutional interpretation—variously called originalism, original meaning, or original public meaning— is often confused with original intention. While the first crew of originalisms in the 1970s did claim that the Court should interpret the Constitution according to the intentions of the Framers, later originalisms like Scalia wisely recast that argument in response to criticisms it received. The intentions of a single author are often unknowable, and in the case of many authors, practically indeterminate. And whose intentions should count: those of the {\color{blue}55} men who wrote the Constitution, the 1,179 men who ratified it, or the even greater number of men who voted for the men who ratified it? From Scalia’s view, it is not intentions that govern us. It is the Constitution, the text as it was written and rewritten through amendment. That is the proper object of interpretation.
 \par 
But how to recover the meaning of a text that can careen from terrifying generality in one sentence (“the executive Power shall be vested in a President”) to an uneventful precision (presidential terms are four years) in the next? Look to the public meaning of the words at the time they were adopted, says Scalia. See how they
 \par 
Were used: consult dictionaries, other usages in the text, influential writings of the time. Consider the context of their utterance, how they were received. From these sources, construct a bounded uni-verse of possible meanings. Words don’t mean one thing, Scalia concedes, but neither do they mean anything. Judges should read the Constitution neither literally nor loosely but “reasonably”— that is, in such a way that each word or phrase is construed “to contain all that it fairly means.” And then, somehow or other, apply that meaning to our own much different times.{\color{blue}12}
 \par 
Scalia justifies his originalism on two grounds, both negative. In a constitutional democracy it is the job of elected representatives to make the law, the job of judges to interpret it. If judges are not bound by how the law, including the Constitution, was understood at the time of its enactment—if they consult their own morals or their own interpretations of the country’s morals—they are no longer judges but lawmakers, and often unelected lawmakers at that. By tying the judge to a text that does not change, originalism helps reconcile judicial review with democracy and protects us from judicial despotism.
 \par 
If Scalia’s first concern is tyranny from the bench, his second is anarchy on the bench. Once we abandon the idea of an unchanging Constitution, he says, we open the gates to any and all modes of interpretation. How are we to understand a Constitution that evolves? By looking at the polls, the philosophy of John Rawls, the teachings of the Catholic Church? If the Constitution is always changing, what constraints can we impose on what counts as an acceptable interpretation? None, Scalia says. When “every day” is “a new day” in the law, it ceases to be law.{\color{blue}13}
 \par 
This mix of tyranny and anarchy is no idle fantasy, Scalia and other originalisms insist. For a brief, terrible time—from the Warren Court of the 1960s to the Burger Court of the 1970s—it was a reality. In the name of a “living Constitution,” left-wing judges remade (or
 \par 
Tried to remake) the country in their own image, forcing an agenda of social democracy, sexual liberation, gender equality, racial integration, and moral relativism down the country’s throat. Ancient words acquired new implications and insinuations: suddenly “due process of law” entailed a “right to privacy,” code words for birth control and abortion (and later gay sex); “equal protection of the laws” required one man, one vote; the ban against “unreasonable searches and seizures” meant that evidence obtained unlawfully by the police could not be admitted in court; the proscription against the “establishment of religion” forbade school prayer. With each law it overturned and right it discovered, the Court seemed to invent a new ground of action. It was a constitutional Carnival, where exotic theories of adjudication were paraded with libidinous abandon. For originalisms, what was most outrageous about this revolution from above—beyond the left-wing values it foisted upon the nation—was how out of keeping it was with how the Court traditionally justified its decisions to strike down laws.
 \par 
Prior to the Warren Court, says Scalia, or the 1920s (it’s never clear when exactly the rot set in), everyone was an originalism. {\color{blue}14} That’s not quite true. Expansive constructions of Constitutional meaning are as old and august as the founding itself. And the theoretical self-consciousness Scalia and his followers bring to the table is a decidedly twentieth-century phenomenon. Scalia, in fact, often sounds like he’s a comp lit student circa 1983. He says it is a “sad commentary” that “American judges have no intelligible theory of what we do most” and “even sadder” that the legal profession is “by and large. . . Unconcerned with the fact that we have no intelligible theory.”{\color{blue}15}
 \par 
Conservatives used to mock that kind of theory fetishism as the mark of an inexperienced and artless ruling class; even an avowed originalism like Robert Bork concedes that “self-config dent legal institutions do not require so much talking about.” But Scalia
 \par 
And Bork forged their ideas in battle against a liberal jurisprudence that was self-conscious and theoretical, and, like so many of their predecessors on the right, they have come out of it looking more like their enemies than their friends. Bork, in fact, freely admits that it is not John Marshall or Joseph Story—the traditional greats of judicial review—to whom he looks for guidance; it is Alexander Bickel, arguably the most self-conscious of the twentieth-century liberal theoreticians, who “taught me more than anyone else about this subject.”{\color{blue}16}
 \par 
Like many originalisms, Scalia claims that his jurisprudence has nothing to do with his conservatism. “I try mightily to prevent my religious views or my political views or my philosophical views from affecting my interpretation of the laws.” Yet he has also said that he learned from his teachers at Georgetown never to “separate your religious life from your intellectual life. They’re not separate.” Only months before Ronald Reagan nominated him to the Supreme Court in 1986, he admitted that his legal views were “inevitably affected by moral and theological perceptions.”{\color{blue}17}
 \par 
And, indeed, in the deep grammar of his opinions lies a conservatism that, if it has little to do with advancing the immediate interests of the Republican Party, has even less to do with averting the threats of judicial tyranny and judicial anarchy. It is a conservatism that would have been recognizable to Social Darwinists of the late nineteenth century, that mixes freely of the premodern and the postmodern, the archaic and the advanced. It is not to be found in the obvious places—Scalia’s opinions about abortion, say, or gay rights—but in a dissenting opinion about that most un-Scaliaesque of places, the golf course.
 \par 
Casey Martin was a champion golfer (he’s now an ex-golfer) who because of a degenerative disease could no longer walk the eighteen holes of a golf course. After the PGA Tour refused his
 \par 
Request to use a golf cart in the final round of one of its qualifying tournaments, a federal court issued an injunction, based on the Americans with Disabilities Act (ADA), allowing Martin to use a cart. Title III of the ADA states that “no individual shall be discriminated against on the basis of disability in the full and equal enjoyment of the goods, services, privileges, advantages, or accommodations of any place of public accommodation by any per-son who owns, leases (or leases to), or operates a place of public accommodation.” By the time the case reached the Supreme Court in 2001, the legal questions had boiled down to these: Is Martin entitled to the protections of Title III of the ADA? Would allowing Martin to use a cart “fundamentally alter the nature” of the game? Ruling 7–2 in Martin’s favor—with Scalia and Thomas in dissent— the Court said yes to the first and no to the second.
 \par 
In answering the first question, the Court had to contend with the PGA’s claims that it was operating a “place of exhibition or entertainment” rather than a public accommodation, that only a customer of that entertainment qualified for Title III protections, and that Martin was not a customer but a provider of entertainment. The Court was skeptical of the first two claims. But even if they were true, the Court said, Martin would still be protected by Title III because he was in fact a customer of the PGA: he and the other contestants had to pay $3,000 to try out for the tournament. Some customers paid to watch the tournament, others to compete in it. The PGA could not discriminate against either.
 \par 
Scalia was incensed. It “seems to me quite incredible,” he began, that the majority would treat Martin as a “‘custom[r]’ of ‘competition’” rather than as a competitor. The PGA sold entertainment, the public paid for it, the golfers provided it; the qualifying rounds were their application for hire. Martin was no more a customer than is an actor who shows up for an open casting call. He was an employee, or potential employee, whose proper recourse, if he had any, was
 \par 
Not Title III of the ADA, which covered public accommodations, but Title I, which covered employment. But Martin wouldn’t have that recourse, admitted Scalia, because he was essentially an independent contractor, a category of employee not covered by the ADA. Martin would thus wind up in a legal no man’s land, without any protection from the law.
 \par 
In the majority’s suggestion that Martin was a customer rather than a competitor, Scalia saw something worse than a wrongly decided opinion. He saw a threat to the status of athletes everywhere, whose talent and excellence would be smothered by the bosomy embrace of the Court, and also a threat to the idea of competition more generally. It was as if the Homeric rivals of Ancient Greece were being plucked from their manly games and forced to walk the aisles of a modern boutique.
 \par 
Games hold a special valence for Scalia: they are the space where inequality rules. “The very nature of competitive sport is the measurement,” he says, “of unevenly distributed excellence.” That inequality is what “determines the winners and losers.” In the noonday sun of competition, we cannot hide our superiority or inferiority, our excellence or inadequacy. Games make our unequal natures plain to the world; they celebrate “the uneven distribution of God-given gifts.”
 \par 
In the Court’s transposition of competitor into customer, Scalia saw the forced entry of democracy (a “revolution,” actually) into this antique preserve. With “Animal Farm determination”—yes, Scalia goes there—the Court had destroyed our one and only opportunity to see how unequal we truly are, how unfairly God has chosen to bestow his blessings upon us. “The year was 2001,” reads the last sentence of Scalia’s dissent, “and ‘everybody was finally equal.’” Like the Social Darwinists and Nietzsche, Scalia is too much a modernist, even a postmodernist, to pine for the lost world of feudal phi cities. Modernity has seen too much flux to sustain a belief in
 \par 
Hereditary status. The watermarks of privilege and privation are no longer visible to the naked eye; they must be identified, again and again, through struggle and contest. Hence, the appeal of the game. In sports, unlike law, every day is a new day. Every competition is a fresh opportunity for mixing it up, for throwing our established hierarchies into anarchic relief and allowing a new face of supremacy or abjection to emerge. It thus off ers the perfect marriage of the feudal and the fallible, the unequal and the unsettled. To answer the second question—goes riding in a golf cart “fundamentally alter the nature” of golf—the majority undertook a thorough history of the rules of golf. It then formulated a two-part test for determining whether riding in a cart would change the nature of golf. The dutifulness and care, the seriousness with which the majority took its task, both amused and annoyed Scalia.
 \par 
It has been rendered the solemn duty of the Supreme Court of the United States. . . To decide What Is Golf. I am sure that the Framers of the Constitution, aware of the 1457 edict of King James II of Scotland prohibiting golf because it interfered with the practice of archery, fully expected that sooner or later the paths of golf and government, the law and the links, would once again cross, and that the judges of this august Court would some day have to wrestle with that age-old jurisprudential question, for which their years of study in the law have so well-prepared them: Is someone riding around a golf course from shot to shot really a golfer?
 \par 
Scalia is clearly enjoying himself, but his mirth is a little mystifying. The ADA defines discrimination as
 \par 
A failure to make reasonable modifications in the policies, practices, or procedures, when such modifications are necessary
 \par 
To afford such goods, services, facilities, privileges, advantages, or accommodations to individuals with disabilities, unless the entity can demonstrate that making such modifications would fundamentally alter the nature of such goods, services, facilities, privileges, advantages, or accommodations that the entity provides.
 \par 
Any determination of discrimination requires a prior determination about whether the “reasonable modification” would “fundamentally alter the nature” of the good in question. The language of the statute, in other words, compels the Court to inquire into and decide What is Golf.
 \par 
But Scalia won’t have any of it. Refusing to be bound by the text, he prefers to meditate on the futility and fatuity of the Court’s inquiry. In seeking to discover the essence of golf, the Court is looking for something that does not exist. “To say that something is ‘essential,’” he writes, “is ordinarily to say that it is necessary to the achievement of a certain object.” But games “have no object except amusement.” Lacking an object, they have no essence. It’s thus impossible to say whether a rule is essential. “All are arbitrary,” he writes of the rules, “none is essential.” What makes a rule is either tradition or, “in more modern times,” the edict of an authoritative body like the PGA. In an unguarded moment, Scalia entertains the possibility of there being “some point at which the rules of a well-known game are changed to such a degree that no reasonable person would call it the same game.” But he quickly pulls back from his foray into essentialism. No Plato for him; he’s with Nietzsche all the way.{\color{blue}18}
 \par 
It is difficult to reconcile this almost Rortyesque hostility to the idea of golf's essence with Scalia’s earlier statements about “the very nature of competitive sport” being the revelation of divinely ordained inequalities. (It’s also difficult to reconcile Scalia’s
 \par 
Indiff erence to the language of the statute with his textualism, but that’s another matter.) Left unresolved, however, the contradiction reveals the twin poles of Scalia’s faith: a belief in rules as arbitrary impositions of power—reflecting nothing (not even the will or standing of their makers) but the fl at surface of their lockbinary meaning—to which we must nevertheless submit; and a belief in rules, zealously enforced, as the divining rod of our ineradicable inequality. Those who make it past these blank and barren gods are winners; everyone else is a loser.
 \par 
In the United States, Tocqueville observed, a federal judge “must know how to understand the spirit of the age.” While the persona of a Supreme Court Justice may be “purely judicial,” his “prerogatives”—the power to strike down laws in the name of the Constitution—“are entirely political.” {\color{blue}19} If he is to exercise those prerogatives effectively, he must be as culturally nimble and socially attuned as the shrewdest pol.
 \par 
How then to explain the influence of Scalia? Here is a man who proudly, defiantly, proclaims his disdain for “the spirit of the age”— that is, when he is not embarrassingly ignorant of it. When the Court voted in 2003 to overturn state laws banning gay sex, Scalia saw the country heading down a slippery slope to masturbation. {\color{blue}20} In 1996, he told an audience of Christians that “we must pray for the courage to endure the scorn of the sophisticated world,” a world that “will not have anything to do with miracles.” We have “to be prepared to be regarded as idiots.” {\color{blue}21} In a dissent from that same year, Scalia declared, “Day by day, case by case, [the Court] is busy designing a Constitution for a country I do not recognize.” {\color{blue}22} As Maureen Dowd wrote, “He’s so Old School, he’s Old Testament.” {\color{blue}23} And yet, according to Elena Kagan, the newest member of the Court, appointed by Obama in 2010, Scalia “is the justice who has had the most important impact over the years on how we think
 \par 
And talk about the law.” John Paul Stevens, the man Kagan replaced and until his retirement the most liberal Justice on the Court, says that Scalia has “made a huge difference, some of it constructive, some of it unfortunate.” Scalia’s influence, moreover, will in all likelihood extend into the future. “He is in tune with many of the current generation of law students,” observes Ruth Bader Gins-burg, another Court liberal. {\color{blue}24} Give me a law student at an impressionable age, Jean Brodie might have said, and she is mine for life. It is not Scalia’s particular positions that have prevailed on the Court. Indeed, some of his most famous opinions—against abortion, affirmative action, and gay rights; in favor of the death penalty, prayer in school, and sex discrimination—are dissents. (With the addition of John Roberts to the Court in 2005 and Samuel Alito in 2006, however, that has begun to change.) Scalia’s hand is more evident in the way his colleagues—and other jurists, lawyers, and scholars—make their arguments.
 \par 
For many years, originalism was derided by the left. As William Brennan, the Court’s liberal titan of the second half of the twentieth century, declared in 1985: “Those who would restrict claims of right to the values of 1789 specifically articulated in the Constitution turn a blind eye to social progress and eschew adaptation of overarching principles to changes of social circumstance.” Against the originalisms, Brennan insisted that “the genius of the Constitution rests not in any static meaning it might have had in a world that is dead and gone, but in the adaptability of its great principles to cope with current problems and current needs.”{\color{blue}25}
 \par 
Just a decade later, however, the liberal Laurence Tribe, para-phrasing the liberal Ronald Dworkin, would say, “We are all originlists now.” {\color{blue}26} That’s even truer today. Where yesterday’s generation of constitutional scholars looked to philosophy— Rawls, Hart, occasionally Nozick, Marx, or Nietzsche—to interpret the Constitution, today’s looks to history, to the moment
 \par 
When a word or passage became part of the text and acquired its meaning. Not just on the right, but also on the left: Bruce Ackerman, Akhil Amar, and Jack Balkin are just three of the most prominent liberal originalisms writing today.
 \par 
Liberals on the Court have undergone a similar shift. In his Citizens United dissent, Stevens wrote a lengthy excursus on the “original understandings,” “original expectations,” and “original public meaning” of the First Amendment with regard to corporate speech. Opening his discussion with a dutiful sigh of obligation— “Let us start from the beginning”—Stevens felt compelled by Scalia, whose voice and name were present throughout, to demonstrate that his position was consistent with the original meaning of freedom of speech.{\color{blue}27}
 \par 
Other scholars and jurists have helped bring about this shift, but it is Scalia who has kept the flame at the highest reaches of the law. Not by tact or diplomacy. Scalia is often a pig, mocking his col-leagues’ intelligence and questioning their integrity. Sandra Day O’Connor, who sat on the bench from 1981 to 2006, was a frequent object of his ridicule and scorn. Scalia characterized one of her arguments as “devoid of content.” Another, he wrote, “cannot be taken seriously.” Whenever he is asked about his role in Bush v. Gore (2000), which put George W. Bush in the White House through a questionable mode of reasoning, he sneers, “Get over it!” {\color{blue}28} Nor, contrary to his camp followers, has Scalia dominated the Court by force of his intelligence. (“How bright is he?” exhales one representative admirer.) {\color{blue}29} On a Court where everyone is a graduate of Harvard, Yale, or Princeton, and Ivy League professors sit on either side of the bench, there are plenty of brains to go around.
 \par 
Several other factors explain Scalia’s dominance of the Court. For starters, Scalia has the advantage of a straightforward philosophy and nifty method. While he and his army march through the archives, rifling through documents on the right to bear arms, the
 \par 
Commerce clause, and much else, the legal left remains “confused and uncertain,” in the words of Yale law professors Robert Post and Reva Siegel, “unable to advance any robust theory of constitutional interpretation” of its own. {\color{blue}30} In an age when the left lacks certainty and will, Scalia’s self-confidence can be a potent and intoxicating force.
 \par 
Second, there’s an elective affinity, even a tight fit, between the originalism of duress oblige and Scalia’s idea of the game. And that is Scalia’s vision of what the good life entails: a daily and arduous struggle, where the only surety, if we leave things well enough alone, is that the strong shall win and the weak shall lose. Scalia, it turns out, is not nearly the iconoclast he thinks he is. Far from telling “people what they don’t like to hear,” as he claims, he tells the power elite exactly what they want to hear: that they are superior and that they have a seat at the table because they are superior. {\color{blue}31} Tocqueville, it seems, was right after all. It is not the plainness but the appositeness of Justice Scalia, the way he reflects rather than refracts the spirit of the age, that explains, at least in part, his influence.
 \par 
But there may be one additional, albeit small and personal, reason for Scalia’s outsized presence in our Constitutional firmament. And that is the patience and forbearance, the general decency and good manners, his liberal colleagues show him. While he rants and raves, smashing guitars and dive-bombing his enemies, they tend to respond with an indulgent shrug, a “that’s just Nino,” as O’Connor was wont to say.{\color{blue}32}
 \par 
The fact may be small and personal, but the irony is large and political. For Scalia preys on and prof its from the very culture of liberalism he claims to abhor: the toleration of opposing views, the generous allowances for other people’s failings, the “benevolent com passion” he derides in his golf course dissent. Should his colleagues ever force him to abide by the same rules of liberal civility, or treat him as he treats them, who knows what might
 \par 
Happen? Indeed, as two close observers of the Court have noted—in an article aptly titled “Don’t Poke Scalia!”—whenever advocates before the bench subject him to the gentlest of gibes, he is quickly rattled and thrown off his game. {\color{blue}33} Prone to tantrums, coddled by a different set of rules: now that’s an affix rmatime action baby.
 \par 
Ever since the 1960s, it has been a commonplace of our political culture that liberal niceties depend upon conservative not-soniceties. A dinner party on the Upper West Side requires a police force that doesn’t know from Miranda, the First Amendment a military that doesn’t know from Geneva. That, of course, is the conceit of {\color{blue}24} (not to mention a great many other Hollywood productions like A Few Good Men). But that formulation may have it exactly backward: without his more liberal colleagues indulging and protecting him, Scalia—like Jack Bauer—would have a much more difficult time. The conservatism of duress oblige depends upon the liberalism of noblesse oblige, not the other way around. That is the real meaning of Justice Scalia.