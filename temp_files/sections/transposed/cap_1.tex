\begin{ figure }
	\centering
	\\includegraphics[width=1.\textwidth]{ temp_files/images/UP_logo.png }
	\caption{Valentina Tereshkova (born 1937): The first woman in space, Tereshkova or- bited the Earth forty-eight times in June 1963 on Vostok {\color{blue}6}. ‘After her career as a cosmonaut, Tereshkova became a prominent politician and led the Soviet delegation to the 1975 United Nations World Conference on Women. She is still widely viewed as a national heroine in Russia today. Courtesy of Elena Lagadinova.}
	\label{ }
\end{ figure }
 \par 
\chapter{PREFACE 10 THE PAPERBACK EDITION}\label{PREFACE 10 THE PAPERBACK EDITION}
 \par 
|: become an academic, you must embrace a particular kind of guileless optimism. Earning a doctorate usually takes a minimum of six years beyond an undergraduate degree, and at least since the late 1980s, the chances of finding a tenure-track job in your chosen field have been abysmal. Back in 1997, when I decided I wanted to write my PhD dissertation on women’s labor in the postsocialist Bulgarian tourism sector, most of my friends and mentors thought me crazy. “You'll never get a job,” they warned, “with such an obscure topic.”
 \par 
Stubborn and perhaps a bit naive, I persisted and spent the late 1990s living and doing research in Eastern Europe, watching firsthand the slow and painful transformation of a state-owned economy into one of unfettered free markets. I observed that women were more likely than men to express a longing for the state socialist past because of the many tangible benefits women lost with the coming of democracy and capitalism. The privatization and liberalization of the economy had disproportionately affected women who lost access to once generous social safety nets that had allowed them to more easily combine work and family responsibilities before 1989. Since those early days interviewing
 \par 
XV xvi
 \par 
PREFACE T0 THE PAPERBACK EDITION chambermaids and receptionists on the Black Sea, I have spent the rest of my career studying the lived experience of state socialism and the effects of postsocialism on ordinary lives in Eastern Europe.
 \par 
As I write this preface in September 2019, more than two years have passed since the publication of the New York Times op-ed that eventually grew into this book, first _published in November 2018. In the intervening period, there has been a veritable explosion of interest in socialism among young Americans coupled with concomitant attacks by political leaders promising that “America will never be a socialist country.” When I submitted my manuscript in March 2018, no one had ever heard of Alexandria Ocasio-Cortez or imagined the record number of women and people of color who would win seats in the House of Representatives during the November 2018 midterm elections. But today ordinary Americans are discussing ideas like the Green New Deal, housing as a human right, free public colleges and technical schools, universal basic income, and Medicare for All as real political possibilities. There has never been a better time to think and write about the history of socialism in both theory and practice.
 \par 
And this conversation has gone global. In the face of a resurgence of right-wing nationalism, white supremacist, and neo-fascist nativist populism, citizens concerned with the trinity of impending disasters for the twenty-first century—the ecological catastrophes of climate change, the automation or algorithmization of most human jobs, and the poisonous growth of extreme income and wealth inequality—find possible solutions in the ideals of socialism. As of this writing, there are ten confirmed foreign editions of this
 \par 
PREFACE TO THE PAPERBACK EDITION book, five of which are translations into the languages of former state socialist countries in Eastern Europe: Russian, German, Polish, Czech, and Slovak. In addition to the official translations, the book has been reviewed and discussed in the East European media from Croatia in the Balkans to Estonia in the Baltics, and American and British press reviews have been translated into languages such as Russian,
 \par 
Ukrainian, Bulgarian, Romanian, and Serbian. I also know from my colleagues based in Eastern Europe that downloaded electronic copies of the English-language version of the book continue to circulate widely, sparking new discussions and reassessments of the state socialist past.
 \par 
Over the last two years, readers in and from Eastern Europe have also reached out to share their own stories or the stories of their parents and grandparents. In one case, a young scholar from Belarus told me he grew unexpectedly closer to his mother when she started sharing her own experiences as a woman living in the former Soviet Union for the first time after reading a review of my book. Another man from Azerbaijan explained that my original op-ed had been translated into Azeri and had sparked a national conversation on social media about the erosion of women’s rights in his country since the breakup of the USSR. Women from the former East Germany confirmed that their personal lives had been much easier—despite the many political hardships—when they lived in a society that valued and supported them as both workers and mothers. For too long these topics have remained taboo. The public discourse about the past made no room for discussions about what socialism had done well.
 \par 
\section{The Romanian scholars Liviu Chelcea and Oana Druta}
 \par 
Xvii xviii
 \par 
PREFACE TO THE PAPERBACK EDITION coined a phrase for the silencing of conversations about socialism in Eastern Europe. They argue that local elites deploy a “zombie socialism” to uphold and legitimize the unequal distribution of once state-owned wealth after 1989. In their usage, “zombie socialism” refers to the constant invocation of the crimes of socialism past to discredit any demands for egalitarian political projects in the present. The authors assert, “The usage of spectral and mythological representations of socialism has, for the winners of transition, the capacity to preempt social justice claims and to structure political relations in the allocation of wealth.” In other words, discussing the past horrors of socialism deflects attention from the very real and present horrors of capitalism. And this isn’t only true in Eastern Europe. As one of a handful of scholars researching and writing about women’s rights in the former state socialist East for many years, I have fought a long and hard battle to convince my Western colleagues that there was anything good on the other side of the Iron Curtain. Today, it heartens me to see that even mainstream publications like the Economist, the Financial Times, and Germany’s Der Spiegel admit that state socialist policies empowered women in profound and lasting ways. Even three decades after the end of the Cold War, statistics show that East European women continue to excel in previously male-dominated fields, particularly in medicine, science, and technology. Much research remains to be done on how socialist theories and practices concerning women’s emancipation changed the course of millions of women’s lives for the better.
 \par 
Of course, not everyone agrees with this point of view, and people have also shared negative personal experiences
 \par 
PREFACE TO THE PAPERBACK EDITION or substantive critiques of my conclusions. But in the nine months that have passed since the publication of the book in the United States and United Kingdom, the empirical evidence presented in this book has not been contradicted, and indeed, new evidence continues to emerge to further substantiate the arguments. Across-Eastern Europe, a rising generation of scholars is diving into the archives, conducting oral histories, and reexamining statistical data to complicate the overwhelmingly negative picture we have of the state socialist past. Even in the West, writers and researchers are rethinking the legacies of state socialist policies in
 \par 
Such fields as art, music, sport, cinema, architecture, urban planning, youth culture, and LGBT rights.
 \par 
This renaissance in interest in the history and culture of East European socialism has been met with a fierce backlash, spearheaded by those wielding the specter of zombie socialism to maintain the neoliberal status quo. Some conservatives insist on equating all things socialist with the worst crimes of Stalinism and will resort to outright lies in the face of any evidence that life behind the Iron Curtain was more than just one big gulag where everyone eventually starved to death waiting in line for toilet paper. One can ignore most of the hyperbole and recycled stereotypes, but there is one insidious tactic used to discredit anyone challenging the idea that socialism will always and inevitably lead to the famines, purges, and gulags: epistemic deplatforming.
 \par 
In the literal sense, a person is “deplatformed” when they are refused access to a venue where they wish to express controversial views. The term “epistemic” relates to knowledge or the ways in which knowledge is validated—how we xix
 \par 
XX
 \par 
PREFACE TO THE PAPERBACK EDITION know what we know. Putting them together, a person has been epistemically deplatformed if it is asserted that anything the person says about a particular subject is tainted— that is, the person should not be believed. One common negative response to my book has been an attempt to epistemically deplatform me on the grounds that I did not live through state socialism and experience it firsthand. A typical Goodreads comment in this vein might be: “Sex was better under socialism for a nut who never lived that.” Despite my academic credentials, decades of experience, and the fact that I was married to a Bulgarian and still have many friends, colleagues, and family members in the region, I can have no authority on state socialism in Eastern Europe because I didn’t “live that” (as if scholars of Ancient Greece or medieval France have firsthand experience of the historical epochs they research).
 \par 
Interestingly, however, conservatives also aibiek the credibility of those who say positive things about socialism even if they did live in Eastern Europe at the time. In the case of my younger colleagues, born after 1980, their critics attack their authority because they were too young to have experienced state socialism as adults. This is especially true for those born after 1991, whose only personal experiences of socialism are those vicariously lived through the stories of their parents and grandparents. It doesn’t matter that they have inherited the state socialist past as modern citizens of these countries—or that the transition to capitalism has intimately shaped their entire lives. Detractors question their ability to speak knowledgeably about the past.
 \par 
And what about my older colleagues who were born and grew up under state socialism in Eastern Europe? Certainly,
 \par 
PREFACE TO THE PAPERBACK EDITION if Westerners and younger East Europeans cannot speak knowledgeably about the past, the obvious authority falls to those who actually lived during the era in question. But alas, even those who write about the society in which they once lived, studied, and worked will also find themselves deplatformed if they dare to say anything positive about that society. Critics claim they are brainwashed, or nostalgic for their youth, or somehow permanently psychologically damaged by their experiences of totalitarianism. Like those suffering from Stockholm syndrome, they have fallen in love with their oppressors and cannot be trusted to pro-
 \par 
\section{Duce objective, unbiased scholarship about the past.}
 \par 
This strategy of epistemic deplatforming thus works out very nicely for conservatives: my claims can be disregarded because I didn’t live through it; my European colleagues can be disregarded either because they were too young during state socialism or because they were brainwashed by it. Therefore, you need not take anyone seriously if they have anything positive to say about state socialism; no need to look at evidence or arguments—just reject the source. Heads I win, tails you lose.
 \par 
Of course, there are legitimate methodological issues with evidence gathered before 1989, and particularly with official government sources of information that may have been manipulated for propaganda reasons. For the purposes of this book, I have mostly relied on (and cited in the end notes) scholarly work from more contemporary historians, sociologists, and anthropologists, including those from the West and those born and raised in countries such as Poland, the Czech Republic, Russia, Hungary, Serbia, and Bulgaria. The only legitimate voices allowed to speak about xxi
 \par 
xxii
 \par 
PREFACE TO THE PAPERBACK EDITION the state socialist past are not just those who have a vested interest in reducing the entire history of state socialism in Eastern Europe to the worst crimes of Stalin in the 1930s.
 \par 
This may seem a mere squabble among scholars, but it has important implications for contemporary politics. In our ever-polarizing world, the continued demonization of the experience of state socialism in Eastern Europe serves as a political bludgeon used to smash the dreams of anyone attempting to imagine a postcapitalist political future. Given the challenges facing us in the twenty-first century, we need to start seriously thinking about what happens when our current economic system gets impaled on a pike of its own inherent contradictions. That day may be closer than we imagine, and as citizens we need to shake off the shackles of Western post-Cold War triumphalist history and cultivate our knowledge of the widest, array of political possibilities. Narratives insisting that all redistributive political experiments end in terror exist to prevent us from believing in the possibility of profound social change. As a scholar and a teacher concerned with the material conditions of women’s lives, | am I making a small contribution to the current political discourse by trying—along with my many academic and journalistic colleagues around the world—to push back against the Cold War imagination that Westerners have of life in the twentieth-century state socialist countries of Eastern Europe. A more nuanced look at this history provides some ideas about what worked and what didn’t work, and allows us to consider new and innovative ways to move forward. If the means of production inhered in the factories of 1920, they inhere in the robots, algorithms, and artificial intelligence of 2020. The world
 \par 
PREFACE TO THE PAPERBACK EDITION has changed dramatically in the last one hundred years, but the logics of capitalism—with its tendency to produce inequality, bigotry, violence, and war—remain the same. This is why Marxism still resonates with the people who bother to read the original texts.
 \par 
To be clear, twentieth-century state socialism failed, and no one with a sincere interest in the well-being of our societies wants to recreate those versions of autocratic states with their clunky centrally planned economies, Draconian travel restrictions, and snooping secret police. Rather, new technological advances allow us to reimagine the relationship between markets and states in more just, equitable, and sustainable ways. But we aren't about to do this so long as horror stories about the past stymie our ability to dream. For too long, our political leaders have told us that there is no alternative to capitalism while at the same time suppressing and distorting the history of potential alternatives.
 \par 
Today, the rise of right-wing racist and xenophobic politics—in countries as diverse as Hungary, France, Brazil, Poland, and the United States—should mobilize us to fight for more nuanced and accurate histories. Right-wing politicians do not fight fair when it comes to discussing the past. Countries such as Ukraine have literally outlawed versions of history that do not serve their national interests. Journalists who say the wrong things about the country’s shared past can be fined or imprisoned. The Ukrainian 2015 “decommunization” laws also banned symbols like the hammer and sickle or images of Che Guevara. Contemporary commemorations of the “victims of communism” in Bulgaria and Croatia exonerate fascists who participated in the Holocaust during World War II.
 \par 
xxiii
 \par 
PREFACE TO THE PAPERBACK EDITION
 \par 
The United States has yet to officially legislate history, but when conservative politicians laud the free market, most consciously attempt to decouple capitalism from its past associations with slavery, imperialism, and monopolist. Yet at the first mention of socialism, they immediately link its history with that of labor camps, famines, and purges. This is not to deny any of this history—slavery, imperialism, monopolist, famines, gulags, and purges must animate our shared histories of both capitalism and socialism. But political and economic systems evolve and change over time; proponents of the unfettered free market embrace this dynamism for capitalism but reject it for socialism.
 \par 
At the center of any socialist project is the value of human life and the desire of individuals to live with meaning and dignity, free of exploitation and oppression. Too often we are told that real political freedom necessitates some form of economic exploitation. This is a lie. It is the bald-faced propaganda of those who gain from the idea that profits are more important than people. Another world is possible, and it is my sincere hope that this book inspires new ways of thinking about both the twentieth-century past and our twenty-first-century future. We can, and will, do better.
 \par 
\section{Kristen R. Ghodsee 9 September 2019 Philadelphia, USA}