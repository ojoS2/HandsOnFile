\chapter{Introduction}\label{Introduction}
 \par 
A political party may fi nd that it has had a history, before it is fully aware of or agreed upon its own permanent tenets; it may have arrived at its actual formation through a succession of metamorphoses and adaptations, during which some issues have been superannuated, and new issues have arisen. What its fundamental tenets are, will probably be found only by careful examination of its behavior throughout its history and by examination of what its more thoughtful and philosophical minds have said on its behalf; and only accurate historical knowledge and judicious analysis will be able to discriminate between the permanent and the transitory; between those doctrines and principles which it must ever, and in all circumstances, maintain, or manifest itself a fraud, and those called forth by special circumstances, which are only intelligible and justifi able in the light of those circumstances.
 \par 
Since the modern era began, men and women in subordinate positions have marched against their superiors in the state, church, workplace, and other hierarchical institutions. They have gathered under different banners—the labor movement, feminism, abolition, socialism—and shouted different slogans: freedom, equality, rights, democracy, revolution. In virtually every instance, their superiors have resisted them, violently and nonviolently, legally and illegally, overtly and covertly. That march and démarche of democracy is the story of modern politics or at least one of its stories.
 \par 
This book is about the second half of that story, the démarche, and the political ideas—variously called conservative, reactionary, revanchist, counterrevolutionary—that grow out of and give rise to it. These ideas, which occupy the right side of the political spectrum, are forged in battle. They always have been, at least since they first emerged as formal ideologies during the French Revolution, battles between social groups rather than nations; roughly speaking, between those with more power and those with less. To understand these ideas, we have to understand that story. For that is what conservatism is: a meditation on—and theoretical rendition of—the felt experience of having power, seeing it threatened, and trying to win it back.
 \par 
Despite the very real differences between them, workers in a factory are like secretaries in an off ice, peasants on a manor, slaves on a plantation—even wives in a marriage—in that they live and labor in conditions of unequal power. They submit and obey, heeding the demands of their managers and masters, husbands and lords. They are disciplined and punished. Furthermore, they do much and receive little. Sometimes their lot is freely chosen—workers contract with their employers, wives with their husbands—but its entailments seldom are. What contract, after all, could ever itemize the ins and outs, the daily pains and ongoing sufferance, of a job or a marriage? Throughout American history, in fact, the contract often has served as a conduit to unforeseen coercion and constraint, particularly in institutions like the workplace and the family where men and women spend so much of their lives. Employment and marriage contracts have been interpreted by judges, themselves friendly to the interests of employers and husbands, to contain all sorts of unwritten and unwanted provisions of servitude to which wives and workers tacitly consent, even when they have no knowledge of such provisions or wish to stipulate otherwise.{\color{blue}1}
 \par 
Until 1980, for example, it was legal in every state in the union for a husband to rape his wife. {\color{blue}2} The justification for these dates back to a 1736 treatise by English jurist Matthew Hale. When a woman marries, Hale argued, she implicitly agrees to give “up herself in this kind [sexually] unto her husband.” Hers is a tacit, if unknowing, consent “which she cannot retract” for the duration of their union. Having once said yes, she can never say no. As late as 1957—during the era of the Warren Court—a standard legal treatise could state, “A man does not commit rape by having sexual intercourse with his lawful wife, even if he does so by force and against her will.” If a woman (or man) tried to write into the marriage contract a requirement that express consent had to be given in order for sex to proceed, judges were bound by common law to ignore or override it. Implicit consent was a structural feature of the contract that neither party could alter. With the exit option of divorce not widely available until the second half of the twentieth century, the marriage contract doomed women to be the sexual servants of their husbands. {\color{blue}3} A similar dynamic was at work in the employment con-tract: workers consented to be hired by their employers, but until the twentieth century that consent was interpreted by judges to contain implicit and irrevocable provisions of servitude; mean-while, the exit option of quitting was not nearly as available, legally or practically, as many might think.{\color{blue}4}
 \par 
Every once in a while, however, the subordinates of this world contest their fates. They protest their conditions, write letters and petitions, join movements, and make demands. Their goals may be minimal and discrete—better safety guards on factory machines, an end to marital rape—but in voicing them, they raise the specter of a more fundamental change in power. They cease to be servants or supplicants and become agents, speaking and acting on their own behalf. More than the reforms themselves, it is this assertion of agency by the subject class—the appearance of an insistent and
 \par 
Independent voice of demand—that vexes their superiors. Guatemala’s Agrarian Reform of 1952 redistributed a million and a half acres of land to 100,000 peasant families. That was nothing, in the minds of the country’s ruling classes, compared to the riot of political talk the bill seemed to unleash. Progressive reformers, Guatemala’s arch-bishop complained, sent local peasants “gifted with facility with words” to the capital, where they were given opportunities “to speak in public.” That was the great evil of the Agrarian Reform.{\color{blue}5}
 \par 
In his last major address to the Senate, John C. Calhoun, former vice president and chief spokesman of the Southern cause, identityphi ed the decision by Congress in the mid-1830s to receive abolitionist petitions as the moment when the nation set itself on an irreversible course of confrontation over slavery. In a four-decade career that had seen such defeats to the slaveholder position as the Tariff of Abominations, the Nullifi cation Crisis, and the Force Bill, the mere appearance of slave speech in the nation’s capital stood out for the dying Calhoun as the sign that the revolution had begun. {\color{blue}6} And when, a half-century later, Calhoun’s successors sought to put the abolitionist genie back into the bottle, it was this same assertion of black agency that they targeted. Explaining the proliferation across the South in the 1890s and 1900s of constitutional conventions restricting the franchise, a delegate to one such convention declared, “The great underlying principle of this Convention movement. . . Was the elimination of the negro from the politics of this State.”{\color{blue}7}
 \par 
American labor history is filled with similar complaints from the employing classes and their allies in government: not that unionized workers are violent, disruptive, or unprofi table but that they are independent and self-organizing. Indeed, so potent is their self-organization that it threatens—in the eyes of their superiors— to render superfluous the employer and the state. During the Great Upheaval of 1877, striking railroad workers in St. Louis took to
 \par 
Running the trains themselves. Fearful the public might conclude the workers were capable of managing the railroad, the owners tried to stop them—in effect, launching a strike of their own in order to prove it was the owners, and only the owners, who could make the trains run on time. During the Seattle general strike of 1919, workers went to great lengths to provide basic government services, including law and order. So successful were they that the mayor concluded it was this, the workers’ independent capacity to limit violence and anarchy, that posed the greatest threat.
 \par 
The so-called sympathetic Seattle strike was an attempted revolution. That there was no violence does not alter the fact. . . . True, there were no fl ashing guns, no bombs, no killings. Revolution, I repeat, doesn’t need violence. The general strike, as practiced in Seattle, is of itself the weapon of revolution, all the more dangerous because quiet. . . . That is to say, it puts the government out of operation. And that is all there is to revolt— no matter how achieved.{\color{blue}8}
 \par 
Into the twentieth century, judges regularly denounced unionized workers for formulating their own definitions of rights and compiling their own register of shop-floor rules. Workers like these, claimed one federal court, saw themselves as “exponents of some higher law than that. . . Administered by courts.” They were exercising “powers belonging only to Government,” declared the Supreme Court, constituting themselves as a “self-appointed tribunal” of law and order. {\color{blue}9} Conservatism is the theoretical voice of this animus against the agency of the subordinate classes. It provides the most consistent and profound argument as to why the lower orders should not be allowed to exercise their independent will, why they should not be allowed to govern themselves or the polity. Submission is their first duty, agency, the prerogative of the elite.
 \par 
Though it is often claimed that the left stands for equality while the right stands for freedom, this notion misstates the actual disagreement between right and left. Historically, the conservative has favored liberty for the higher orders and constraint for the lower orders. What the conservative sees and dislikes in equality, in other words, is not a threat to freedom but its extension. For in that extension, he sees a loss of his own freedom. “We are all agreed as to our own liberty,” declared Samuel Johnson. “But we are not agreed as to the liberty of others: for in proportion as we take, others must lose. I believe we hardly wish that the mob should have liberty to govern us.” {\color{blue}10} Such was the threat Edmund Burke saw in the French Revolution: not merely an expropriation of property or explosion of violence but an inversion of the obligations of deference and command. “The levelers,” he claimed, “only change and pervert the natural order of things.”
 \par 
The occupation of a hairdresser, or of a working tallow-chandler, cannot be a matter of honor to any person—to say nothing of a number of other more servile employments. Such descriptions of men ought not to suff her oppression from the state; but the state suffers oppression, if such as they, either individually or collectively, are permitted to rule.{\color{blue}11}
 \par 
By virtue of membership in a polity, Burke allowed, men had a great many rights—to the fruits of their labor, their inheritance, education, and more. But the one right he refused to concede to all men was that “share of power, authority, and direction” they might think they ought to have “in the management of the state.”{\color{blue}12}
 \par 
Even when the left’s demands shift to the economic realm, the threat of freedom’s extension looms large. If women and workers are provided with the economic resources to make independent choices, they will be free not to obey their husbands and employers.
 \par 
That is why Lawrence Mead, one of the leading intellectual opponents of the welfare state in the 1980s and 1990s, declared that the welfare recipient “must be made less free in certain senses rather than more.” {\color{blue}13} For the conservative, equality portends more than a redistribution of resources, opportunities, and outcomes—though he certainly dislikes these, too. {\color{blue}14} What equality ultimately means is a rotation in the seat of power.
 \par 
The conservative is not wrong to construe the threat of the left in these terms. Before he died, G. A. Cohen, one of contemporary Marxism’s most acute voices, made the case that much of the left’s program of economic redistribution could be understood as entailing not a sacrifice of freedom for the sake of equality, but an extension of freedom from the few to the many. {\color{blue}15} And, indeed, the great modern movements of emancipation—from abolition to feminism to the struggle for workers’ rights and civil rights—have always posited a nexus between freedom and equality. Marching out of the family, the factory, and the field, where freedom and inequality are the flip sides of the same coin, they have made freedom and equality the irreducible yet mutually reinforcing parts of a single whole. The link between freedom and equality has not made the argument for redistribution any more palatable to the right. As one conservative wag complained of John Dewey’s vision of social democracy, “The definitions of liberty and of equality have been so juggled that both refer to approximately the same condition.” {\color{blue}16} Far from being a sleight of the progressive hand, how-ever, this synthesis of freedom and equality is a central postulate of the politics of emancipation. Whether the politics conforms to the postulate is, of course, another story. But for the conservative, the concern is less the betrayal of the postulate than its fulfillment.
 \par 
One of the reasons the subordinate’s exercise of agency so agitates the conservative imagination is that it takes place in an intimate
 \par 
Setting. Every great political blast—the storming of the Bastille, the taking of the Winter Palace, the March on Washington—is set off by a private fuse: the contest for rights and standing in the family, the factory, and the field. Politicians and parties talk of constitution and amendment, natural rights and inherited privileges. But the real subject of their deliberations is the private life of power. “Here is the secret of the opposition to woman’s equality in the state,” Elizabeth Cady Stanton wrote. “Men are not ready to recognize it in the home.” {\color{blue}17} Behind the riot in the street or debate in Parliament is the maid talking back to her mistress, the worker disobeying her boss. That is why our political arguments—not only about the family but also the welfare state, civil rights, and much else—can be so explosive: they touch upon the most personal relations of power. It is also why it has so often fallen to our novelists to explain to us our politics. At the height of the civil rights movement, James Baldwin traveled to Tallahassee. There, in an imagined handshake, he found the hidden transcript of a constitutional crisis.{\color{blue}18}
 \par 
I am the only Negro passenger at Tallahassee’s shambles of an airport. It is an oppressively sunny day. A black chauffeur, leading a small dog on a leash, is meeting his white employer. He is attentive to the dog, covertly very aware of me and respectful of her in a curiously watchful, waiting way. She is middamaged, beaming and powdery-faced, delighted to see both the beings who make her life agreeable. I am sure that it has never occurred to her that either of them has the ability to judge her or would judge her harshly. She might almost, as she goes toward her chauffeur, be greeting a friend. No friend could make her face brighter. If she were smiling at me that way I would expect to shake her hand. But if I should put out my hand, panic, bafflement, and horror would then overtake that
 \par 
Face, the atmosphere would darken, and danger, even the threat of death, would immediately fill the air.
 \par 
On such small signs and symbols does the southern cabala
 \par 
The conflict over American slavery—the looming precedent to this set piece of Baldwin’s imagination—off ers an instructive example. One of the distinguishing characteristics of slavery in the United States is that unlike slaves in the Caribbean or serfs in Russia, many slaves in the South lived on small holdings with their masters in residence. Masters knew their slaves’ names; tracked their births, marriages, and deaths; and held parties to honor these dates. The personal interaction between master and slave was unparalleled, leading a visiting Frederick Law Olmsted to remark upon the “close cohabitation and association of black and white” in Virginia, the “familiarity and closeness of intimacy that would have been noticed with astonishment, if not with manifest displeasure, in almost any chance company at the North.” {\color{blue}20} Only the “relations of husband and wife, parent and child, brother and sister,” wrote the slavery apologist Thomas Dew, produced “a closer tie” than that of master and slave; the latter relationship, declared William Harper, another defender of slavery, was “one of the most intimate relations of society.” {\color{blue}21} Conversely, after slavery was abolished, many whites' lamented the chill in relations between the races. “I’m fond of the Negro,” said one Mississippian in 1918, “but the bond between us is not as close as it was between my father and his slaves.”{\color{blue}22}
 \par 
Much of this talk was propaganda and self-delusion, of course, but in one respect it was not: the nearness of master to slave did make for an exceptionally personal mode of rule. Masters devised and enforced “unusually detailed” rules for their slaves, dictating when they had to get up, eat, work, sleep, garden, visit, and pray. Masters decided upon their slaves’ mates and marriages. They
 \par 
Named their children, and when the market dictated, separated those children from their parents. And while masters—as well as their sons and overseers—availed themselves of the bodies of their female slaves whenever they wished, they saw fit to patrol and punish any and all sexual congress between their slaves. {\color{blue}23} Living with their slaves, masters had direct means to control their behavior and a detailed map of all the behavior there was to control.
 \par 
The consequences of this proximity were felt not just by the slave but by the master as well. Living every day with his mastery, he became entirely identified with it. So complete was this identify cation that any sign of the slave’s disobedience—much less her emancipation—was seen as an intolerable assault upon his person. When Calhoun declared that slavery “has grown up with our society and institutions, and is so interwoven with them, that to destroy it would be to destroy us as a people,” he wasn’t just referring to society in the aggregate or abstract. {\color{blue}24} He was thinking of individual men absorbed in the day-to-day experience of ruling other men and women. Take that experience away, and you destroyed not only the master but also the man—and the many men who sought to become, or thought they already were like, the master.
 \par 
Because the master put so little distance between himself and his mastery, he would go to unprecedented lengths to keep his holdings. Throughout the Americas slaveholders defended their privileges, but nowhere with the intensity or violence of the master class in the South. Outside the South, wrote C. Vann Wood-ward, the end of slavery was “the liquidation of an investment.” Inside, it was “the death of a society.” {\color{blue}25} And when, after the Civil War, the master class fought with equal ferocity to restore its privileges and power, it was the proximity of command, the nearness of rule, that was uppermost in its mind. As Henry McNeal Turner, a black Republican in Georgia, put it in 1871: “They do not care so much about Congress admitting Negroes to their halls. . . But they
 \par 
Do not want the negroes over them at home.” One hundred years later, a black sharecropper in Mississippi would still resort to the most domestic of idioms to describe relations between blacks and whites: “We had to mind them as our children mind us.”{\color{blue}26}
 \par 
When the conservative looks upon a democratic movement from below, this (and the exercise of agency) is what he sees: a terrible disturbance in the private life of power. Witnessing the election of Thomas Jeff erson in 1800, Theodore Sedgwick lamented, “The aristocracy of virtue is destroyed; personal influence is at an end.” {\color{blue}27} Sometimes the conservative is personally implicated in that life, sometimes not. Regardless, it is his apprehension of the private grievance behind the public commotion that lends his theory its tactile ingenuity and moral ferocity. “The real object” of the French Revolution, Burke told Parliament in 1790, is “to break all those connections, natural and civil, that regulate and hold together the community by a chain of subordination; to raise soldiers against their officers; servants against their masters; tradesmen against their customers; artificers against their employers; tenants against their landlords; curates against their bishops; and children against their parents.” {\color{blue}28} Personal insubordination rapidly became a regular and consistent theme of Burke’s pronouncements on the unfolding events in France. A year later, he wrote in a letter that because of the Revolution, “no house is safe from its servants, and no Offi cer from his Soldiers, and no State or constitution from conspiracy and insurrection.” {\color{blue}29} In another speech before Parliament in 1791, he declared that “a constitution founded on what was called the rights of man” opened “Pandora’s box” throughout the world, including Haiti: “Blacks rose against whites, whites against blacks, and each against one another in murderous hostility; subordination was destroyed.” {\color{blue}30} Nothing to the Jacobins, he declared at the end of his life, was worthy “of the name of the public virtue, unless it indicates violence on the private.”{\color{blue}31}
 \par 
So powerful is that vision of private eruption that it can turn a man of reform into a man of reaction. Schooled in the Enlightenment, John Adams believed that “consent of the people” was “the only moral foundation of government.” {\color{blue}32} But when his wife suggested that a muted version of these principles be extended to the family, he was not pleased. “And, by the way,” Abigail wrote him, “in the new code of laws which I suppose it will be necessary for you to make, I desire you would remember the ladies and be more generous and favorable to them than your ancestors. Do not put such unlimited power into the hands of the husbands. Remember, all men would be tyrants if they could.” {\color{blue}33} Her husband’s response:
 \par 
We have been told that our struggle has loosened the bands of government everywhere; that children and apprentices were disobedient; that schools and colleges were grown turbulent; that Indians slighted their guardians, and Negroes grew insolent to their masters. But your letter was the first intimation that another tribe, more numerous and powerful than all the rest, were grown discontented.
 \par 
Though he leavened his response with playful banter—he prayed that George Washington would shield him from the “despotism of the petticoat” {\color{blue}34} —Adams was clearly rattled by this appearance of democracy in the private sphere. In a letter to James Sullivan, he worried that the Revolution would “confound and destroy all distinctions,” unleashing throughout society a spirit of insubordination so intense that all order would be dissolved. “There will be no end of it.” {\color{blue}35} No matter how democratic the state, it was imperative that society remain a federation of private dominions, where husbands ruled over wives, masters governed apprentices, and each “should know his place and be made to keep it.”{\color{blue}36}
 \par 
Historically, the conservative has sought to forestall the march of democracy in both the public and the private spheres, on the assumption that advances in the one necessarily spur advances in the other. “In order to keep the state out of the hands of the people,” wrote the French monarchist Louis de Bonald, “it is necessary to keep the family out of the hands of women and children.” {\color{blue}37} Even in the United States, this effs ort has periodically yielded fruit. Despite our Whiggish narrative of the steady rise of democracy, historian Alexander Keyssar has demonstrated that the struggle for the vote in the United States has been as much a story of retraction and contraction as one of progress and expansion, “with class tensions and apprehensions” on the part of political and economic elites constituting “the single most important obstacle to universal suffrage. . . From the late eighteenth century to the 1960s.”{\color{blue}38}
 \par 
Still, the more profound and prophetic stance on the right has been Adams’s: cede the field of the public, if you must, stand fast in the private. Allow men and women to become democratic citizens of the state; make sure they remain feudal subjects in the family, the factory, and the field. The priority of conservative political argument has been the maintenance of private regimes of power—even at the cost of the strength and integrity of the state. We see this political arithmetic at work in the ruling of a Federalist court in Massachusetts that a Loyalist woman who fl ed the Revolution was the adjutant of her husband, and thus should not be held responsible for fleeing and should not have her property config scatted by the state; in the refusal of Southern slaveholders to yield their slaves to the Confederate cause; and the more recent insistence of the Supreme Court that women could not be legally obliged to sit on juries because they are “still regarded as the center of home and family life” with their “own special responsibilities.” {\color{blue}39} Conservatism, then, is not a commitment to limited government and liberty—or a wariness of change, a belief in evolutionary
 \par 
Reform, or a politics of virtue. These may be the byproducts of conservatism, one or more of its historically specific and ever-changing modes of expression. But they are not its animating purpose. Neither is conservatism a makeshift fusion of capitalists, Christians, and warriors, for that fusion is impelled by a more elemental force—the opposition to the liberation of men and women from the fetters of their superiors, particularly in the private sphere. Such a view might seem miles away from the libertarian defense of the free market, with its celebration of the atomistic and autonomous individual. But it is not. When the libertarian looks out upon society, he does not see isolated individuals; he sees private, often hierarchical, groups, where a father governs his family and an owner his employees.{\color{blue}40}
 \par 
No simple defense of one’s own place and privileges—the conservative, as I’ve said, may or may not be directly involved in or benefit from the practices of rule he defends; many, as we’ll see, are not—the conservative position stems from a genuine conviction that a world thus emancipated will be ugly, brutish, base, and dull. It will lack the excellence of a world where the better man commands the worst. When Burke adds, in the letter quoted above, that the “great Object” of the Revolution is “to root out that thing called an Aristocrat or Nobleman and Gentleman,” he is not simply referring to the power of the nobility; he is also refer-ring to the distinction that power brings to the world. {\color{blue}41} If the power goes, the distinction goes with it. This vision of the connection between excellence and rule is what brings together in post-war America that unlikely alliance of the libertarian, with his vision of the employer’s untrammeled power in the workplace; the traditionalist, with his vision of the father’s rule at home; and the statist, with his vision of a heroic leader pressing his hand upon the face of the earth. Each in his own way subscribes to this typical statement, from the nineteenth century, of the conservative creed:
 \par 
“To obey a real superior. . . Is one of the most important of all virtues—a virtue absolutely essential to the attainment of any-thing great and lasting.”{\color{blue}42}
 \par 
The notion that conservative ideas are a mode of counterrevolutionary practice is likely to raise some eyebrows, even hackles. It has long been an axiom on the left that the defense of power and privilege is an enterprise devoid of ideas. “Intellectual history,” a recent study of American conservatism submits, “is never unwelcome,” but it “is not the most direct approach to explaining the power of conservatism in America.” {\color{blue}43} Liberal writers have always portrayed right-wing politics as an emotional swamp rather than a movement of considered opinion: Thomas Paine claimed counter-revolution entailed “an obliteration of knowledge”; Lionel Trilling described American conservatism as a mélange of “irritable mental gestures which seek to resemble ideas”; Robert Paxton called fascism an “aff air of the gut,” not “of the brain.” {\color{blue}44} Conservatives, for their part, have tended to agree. {\color{blue}45} It was Palmerston, after all, when he was still a Tory, who first attached the epithet “stupid” to the Conservative Party. Playing the part of the dull-witted country squire, conservatives have embraced the position of F. J. C. Hearnshaw that “it is commonly sufficient for practical purposes if conservatives, without saying anything, just sit and think, or even if they merely sit.” {\color{blue}46} While the aristocratic overtones of that discourse no longer resonate, the conservative still holds onto the label of the untutored and the unlettered; it’s part of his populist charm and demotic appeal. As the conservative Washington Times observes, Republicans “often call themselves the ‘stupid party.’” {\color{blue}47} Nothing, as we shall see, could be further from the truth. Conservatism is an idea-driven praxis, and no amount of preening from the right or polemic from the left can reduce or eff ace the catalog of mind one finds there.
 \par 
Conservatives themselves will likely be put off by this argument for a different reason: it threatens the purity and profundity of conservative ideas. For many, the word “reaction” connotes an unthinking, lowly grab for power. {\color{blue}48} But reaction is not real ex. It begins from a position of principle—that some are fit, and thus ought, to rule others—and then recalibrates that principle in light of a democratic challenge from below. This recalibration is no easy task, for such challenges tend by their very nature to dis-prove the principle. After all, if a ruling class is truly fit to rule, why and how has it allowed a challenge to its power to emerge? What does the emergence of the one say about the fitness of the other? {\color{blue}49} The conservative faces an additional hurdle: How to defend a principle of rule in a world where nothing is solid, all is in flux? From the moment conservatism came onto the scene, it has had to contend with the decline of ancient and medieval ideas of an orderly universe, in which permanent hierarchies of power reflected the eternal structure of the cosmos. The over-throw of the old regime reveals not only the weakness and incompetence of its leaders but also a larger truth about the lack of design in the world. (The idea that conservatism reflects the revelation that the world has no natural hierarchies might seem odd in our age of Intelligent Design. But as Kevin Mattson and others have pointed out, Intelligent Design is not based on the same kind of medieval assumption of a firm eternal structure to the universe, and there is more than a touch of relativism and skepticism to its arguments. Indeed, one of Intelligent Design’s leading proponents has claimed that though he’s “no postmodernist,” he has “learned a lot” from postmodernism. {\color{blue}50}) Reconstructing the old regime in the face of a declining faith in permanent hierarchies has proven to be a difficult feat. Not surprisingly, it also has produced some of the most remarkable works of modern thought.
 \par 
But there is another reason we should be wary of the effort to dismiss the reactionary thrust of conservatism, and that is the testimony of the tradition itself. Ever since Burke, it has been a point of pride among conservatives that theirs is a contingent mode of thought. Unlike their opponents on the left, they do not unfurl a blueprint in advance of events. They read situations and circumstances, not texts and tomes; their preferred mode is adaptation and intimation rather than assertion and declamation. There’s a certain truth to this claim, as we will see: the conservative mind is extraordinarily supple, alert to changes in context and fortune long before others realize they are occurring. With his deep awareness of the passage of time, the conservative possesses a tactical virtuosity few can match. It seems only logical that conservatism would be intimately bound up with, its antennae ever sensitive to, the movements and countermovements of power sketched above. These are, as I’ve said, the story of modern politics, and it would seem strange if a mind so attuned to the surrounding contingencies were not well versed in that story. Not just well versed, but awakened and aroused by it as by no other story.
 \par 
Indeed, from Burke’s claim that he and his ilk had been “alarmed into real exion” by the French Revolution to Russell Kirk’s admission that conservatism is a “system of ideas” that “has sustained men. . . In their resistance against radical theories and social transformation ever since the beginning of the French Revolution,” the conservative has consistently affirmed that his is a knowledge produced in reaction to the left. {\color{blue}51} (Burke would go on to lay down as his “foundation” the notion that “never greater” an evil had “existed” than the French Revolution.) {\color{blue}52} Sometimes, that affirmation has been explicit. Three times prime minister, Salisbury wrote in 1859 that “hostility to Radicalism, incessant, implacable hostility, is the essential definition of Conservatism. The fear that the Radicals may triumph is the only final cause that the Conservative Party can plead
 \par 
For its own existence.” {\color{blue}53} More than a half-century later, his son Hugh Cecil—among other things, best man at Winston Churchill’s wedding and provost of Eton—reaffirmed the father’s stance: “I think the government will find in the end that there is only one way of defeating revolutionary tactics and that is by presenting an organized body of thought which is non-revolutionary. That body of thought I call Conservatism.” {\color{blue}54} Others, like Peel, have taken a more circuitous route to get to the same place:
 \par 
My object for some years past, that which I have most earnestly labored to accomplish, has been to lay the foundation of a great party, which, existing in the House of Commons, and deriving its strength from the popular will, should diminish the risk and deaden the shock of a collision between the two deliberative branches of the legislature—which should enable us to check the too importunate eagerness of well-intending men, for hasty and precipitate changes in the constitution and laws of the country, and by which we should be enabled to say, with a voice of authority, to the restless spirit of revolutionary change, “Here are thy bounds, and here shall thy vibrations cease.”{\color{blue}55}
 \par 
Lest we think such sentiments—and circumlocutions—are peculiarly English, consider how the court historian of the American right approached the matter in 1976. “What is conservatism?” George Nash asked in his now classic The Conservative Intellectual Movement in America since 1945. After a page of hesitation— conservatism resists definition, it “varies enormously with time and place” (what political idea doesn’t?), it should not be “confused with the Radical Right”—Nash settled upon an answer that could have been given (indeed, was given) by Peel, Salisbury and son, Kirk, and most of the thinkers on the Radical Right. Conservatism, he said, is defined by “resistance to certain forces perceived to be
 \par 
Leftist, revolutionary, and profoundly subversive of what conservatives at the time deemed worth cherishing, defending, and perhaps dying for.”{\color{blue}56}
 \par 
These are the explicit professions of the counterrevolutionary creed. More interesting are the implicit statements, where antipathy to radicalism and reform is embedded in the very syntax of the argument. Take Michael Oakeshott’s famous definition in his essay “On Being Conservative”: “To be conservative, then, is to prefer the familiar to the unknown, to prefer the tried to the untried, fact to mystery, the actual to the possible, the limited to the unbounded, the near to the distant, the sufficient to the superabundant, the convenient to the perfect, present laughter to utopian bliss.” One cannot, it seems, enjoy fact and mystery, near and distant, laughter and bliss. One must choose. Far from affirming a simple hierarchy of preferences, Oakeshort’s either/or signals that we are on existential ground, where the choice is not between something and its opposite but between some-thing and its negation. The conservative would enjoy familiar things in the absence of forces seeking their destruction, Oakeshott con-cedes, but his enjoyment “will be strongest when” it “is combined with evident risk of loss.” The conservative is a “man who is acutely aware of having something to lose which he has learned to care for.” And while Oakeshott suggests that such losses can be engineered by a variety of forces, the engineers invariably seem to work on the left. (Marx and Engels are “the authors of the most stupendous of our political rationalism,” he writes elsewhere. “Nothing. . . Can compare with” their abstract utopianism.) For that reason, “it is not at all inconsistent to be conservative in respect of government and radical in respect of almost every other activity.” {\color{blue}57} Not at all inconsistent—or altogether necessary? Radicalism is the raison d’être of conservatism; if it goes, conservatism goes too. {\color{blue}58} Even when the conservative seeks to extricate himself from this dialogue with the left, he cannot, for his most lyrical motifs—organic change, tacit knowledge, ordered
 \par 
Liberty, prudence, and precedent—are barely audible without the call and response of the left. As Disraeli discovered in his Vindication of the English Constitution (1835), it is only by contrast to a putative revolutionary rationalism that the invocation of ancient and tacit wisdom can have any purchase on the modern mind.
 \par 
The formation of a free government on an extensive scale, while it is assuredly one of the most interesting problems of humanity, is certainly the greatest achievement of human wit. Perhaps I should rather term it a superhuman achievement; for it requires such refined prudence, such comprehensive knowledge, and such perspicacious sagacity, united with such almost illimitable powers of combination, that it is nearly in vain to hope for qualities so rare to be congregated in a solitary mind. Assuredly this sum mum bonus is not to be found ensconced behind a revolutionary barricade, or floating in the bloody gutters of an incendiary metropolis. It cannot be scribbled down—this great invention—in a morning on the envelope of a letter by some charter-concocting monarch, or sketched with ludicrous facility in the conceited commonplace book of a Utilitarian sage.{\color{blue}59}
 \par 
There is more to this antagonistic structure of argument than the simple antibodies of partisan politics, the oppositional position-taking that is a requirement of winning elections. As Karl Mannheim argued, what distinguishes conservatism from traditionalism—the universal “vegetative” tendency to remain attached to things as they are, which is manifested in nonpolitical behaviors such as a refusal to buy a new pair of pants until the current pair is shredded beyond repair—is that conservatism is a deliberate, conscious effort to pre-serve or recall “those forms of experience which can no longer be had in an authentic way.” Conservatism “becomes conscious and
 \par 
Refl ective when other ways of life and thought appear on the scene, against which it is compelled to take up arms in the ideological struggle.” {\color{blue}60} Where the traditionalist can take the objects of desire for granted—he can enjoy them as if they are at hand because they are at hand—the conservative cannot. He seeks to enjoy them precisely as they are being—or have been—taken away. If he hopes to enjoy them again, he must contest their divestment in the public realm. He must speak of them in a language that is politically serviceable and intelligible. But as soon as those objects enter the medium of political speech, they cease to be items of lived experience and become incidents of an ideology. They get wrapped in a narrative of loss—in which the revolutionary or reformist plays a necessary part—and presented in a program of recovery. What was tacit becomes articulate, what was fluid becomes formal, what was practice becomes polemic. {\color{blue}61} Even if the theory is a paean to practice—as conservatism often is—it cannot escape becoming a polemic. The fussiest conservative who would deign to enter the street is compelled by the left to pick up a paving stone and toss it at the barricades. As Lord Hailsham put it in his 1947 Case for Conservatism :
 \par 
Conservatives do not believe that political struggle is the most important thing in life. In this they diff her from Communists, Socialists, Nazis, Fascists, Social Creditors and most members of the British Labour Party. The simplest among them prefer fox-hunting—the wisest religion. To the great majority of Conservatives, religion, art, study, family, country, friends, music, fun, duty, all the joy and riches of existence of which the poor no less than the rich are the indefeasible freeholders, all these are higher in the scale than their handmaiden, the political struggle. This makes them easy to defeat—at first. But, once, defeated, they will hold to this belief with the fanaticism of a Crusader and the doggedness of an Englishman.{\color{blue}62}
 \par 
Because there is so much confusion about conservatism’s opposition to the left, it is important that we be clear about what the conservative is and is not opposing in the left. It is not change in the abstract. No conservative opposes change as such or defends order as such. The conservative defends particular orders—hierarchical, often private regimes of rule—on the assumption, in part, that hierarchy is order. “Order cannot be had,” declared Johnson, “but by sub-ordination.” {\color{blue}63} For Burke, it was axiomatic that “when the multitude are not under this discipline” of “the wiser, the more expert, and the more opulent,” “they can scarcely be said to be in civil society.” {\color{blue}64} In defending such orders, moreover, the conservative invariably launches himself on a program of reaction and counterrevolution, often requiring an overhaul of the very regime he is defending. “If we want things to stay as they are,” in Lampedusa’s classic formulation, “things will have to change.” {\color{blue}65} To preserve the regime, as I show in part 1, the conservative must reconstruct the regime. This pro-gram entails far more than clichés about “preservation through renovation” would suggest: often, it can require the conservative to take the most radical measures on the regime’s behalf.
 \par 
Some of the stuffiest partisans of order on the right have been more than happy, when it has suited their purposes, to indulge in a little bit of mayhem and madness. Kirk, the self-styled Burkean, wished to “espouse conservatism with the vehemence of a radical. The thinking conservative, in truth, must take on some of the out-ward characteristics of the radical, today: he must poke about the roots of society, in the hope of restoring vigor to an old tree strangled in the rank undergrowth of modern passions.” That was in 1954. Fifteen years later, at the height of the student movement, he wrote, “Having been for two decades a mordant critic of what is foolishly called the higher learning in America, I confess to relishing somewhat. . . The fulfillment of my predictions and the present plight of the educationist Establishment. I even own to a sneaking
 \par 
Sympathy, after a fashion, with the campus revolutionaries.” In God and Man at Yale, William F. Buckley declared conservatives “the new radicals.” Upon reading the first few issues of National Review, Dwight Macdonald was inclined to agree: “Had [Buckley] been born a generation earlier, he would have been making the cafeterias of 14th Street ring with Marxian dialectics.” {\color{blue}66} Even Burke himself wrote that “the madness of the wise” is “better than the sobriety of fools.”{\color{blue}67}
 \par 
There’s a fairly simple reason for the embrace of radicalism on the right, and it has to do with the reactionary imperative that lies at the core of conservative doctrine. The conservative not only opposes the left; he also believes that the left has been in the driver’s seat since, depending on who’s counting, the French Revolution or the Reformation. {\color{blue}68} If he is to preserve what he values, the conservative must declare war against the culture as it is. Though the spirit of militant opposition pervades the entirety of conservative discourse, Dinesh D’Souza has put the case most clearly.
 \par 
Typically, the conservative attempts to conserve, to hold on to the values of the existing society. But. . . What if the existing society is inherently hostile to conservative beliefs? It is foolish for a conservative to attempt to conserve that culture. Rather, he must seek to undermine it, to thwart it, to destroy it at the root level. This means that the conservative must. . . Be philosophically conservative but temperamentally radical.{\color{blue}69}
 \par 
By now, it should also be clear that it is not the style or pace of change that the conservative opposes. The conservative theorist likes to draw a “manifest marked distinction” between evolutionary reform and radical change. {\color{blue}70} The first is slow, incremental, and adaptive; the second is fast, comprehensive, and by design. But that distinction, so dear to Burke and his followers, is often less clear in practice
 \par 
Than the theorist allows. {\color{blue}71} Political theory is designed to be abstract, but what abstraction has impelled such diametrically opposed political programs as the preference for reform over radicalism, evolution over revolution? In the name of slow, organic, adaptive change, self-declared conservatives opposed the New Deal (Robert Nisbet, Kirk, and Whittaker Chambers) and endorsed the New Deal (Peter Viereck, Clinton Rossiter, and Whittaker Chambers). {\color{blue}72} A belief in evolutionary reform could lead one to adopt a Hayekian defense of the free market or the democratic socialism of Edward Bernstein. “Even Fabian Socialists,” Nash tartly observes, “who believed in ‘the inevitability of gradualness’ might be labeled conservatives.” {\color{blue}73} Conversely, as Abraham Lincoln pointed out, it’s just as easy for the left to claim the mantle of preservation as it is for the right. “You say you are conservative,” he declared to the slaveholders.
 \par 
Eminently conservative—while we are revolutionary, destructive, or something of the sort. What is conservatism? Is it not adherence to the old and tried, against the new and untried? We stick to, contend for, the identical old policy on the point in controversy which was adopted by “our fathers who framed the Government under which we live”; while you with one accord reject, and scout, and spit upon that old policy, and insist upon substituting something new. . . . Not one of all your various plans can show a precedent or an advocate in the century within which our Government originated. Consider, then, whether your claim of conservatism for yourself, and your charge of destructiveness against us, are based on the most clear and stable foundations.{\color{blue}74}
 \par 
More often, however, the blurriness of the distinction has allowed the conservative to oppose reform on the grounds either that it will lead to revolution or that it is revolution. (Indeed, with
 \par 
The exception to Peel and Baldwin, no Tory leader has ever pursued a consistent program of preservation through reform, and even Peel could not persuade his party to follow him. {\color{blue}75}) Burke him-self was not immune to the argument that reform leads to revolution. Even though he spent the better part of the decade preceding the American Revolution contesting that argument, he still wondered, “When you open” a constitution “to inquiry in one part,” which would seem to be the definition of slow reform, “where will the inquiry stop?” {\color{blue}76} Other conservatives have argued that any demand from or on behalf of the lower orders, no matter how tepid or tardy, is too much, too soon, too fast. Reform is revolution, improvement is insurrection. “It may be good or bad,” a gloomy Lord Carnarvon wrote of the Second Reform Act of 1867—a bill twenty years in the making that tripled the size of the British electorate—“but it is a revolution.” Minus the opening qualification, this was a repeat of what Wellington had said about the first Reform Act. {\color{blue}77} Across the Atlantic, Wellington’s contemporary Nicholas Biddle was denouncing Andrew Jackson’s veto of the Second Bank (that most constitutionally exercised of constitutional powers) in similar terms: “It has all the fury of a chained panther biting at the bars of his cage. It really is a manifesto of anarchy— such as Marat or Robespierre might have issued to the mob.”{\color{blue}78}
 \par 
Today’s conservative may have made his peace with some mancitations past; others, like labor unions and reproductive freedom, he still contests. But that does not alter the fact that when those emancipation first arose as a question, whether in the context of revolution or reform, his predecessor was in all likelihood against them. Michael Gerson, former speechwriter for George W. Bush, is one of the few contemporary conservatives who acknowledge the history of conservative opposition to emancipation. Where other conservatives like to lay claim to the abolitionist or civil rights mantle, Gerson admits that “honesty requires the recognition that
 \par 
Many conservatives, in other times, have been hostile to religiously motivated reform” and that “the conservative habit of mind once opposed most of these changes.” {\color{blue}79} Indeed, as Samuel Huntington suggested a half-century ago, saying no to such movements in real time may be what makes someone a conservative throughout time.{\color{blue}80}
 \par 
Forged in response to challenges from below, conservatism has none of the calm or composure that attends an enduring inheritance of power. One will look in vain throughout the canon of the right for steady assurances of a Great Chain of Being. Conservative statements of organic unity, such as they are, either have an air of quiet—and not so quiet—desperation about them or, as in the case of Kirk, lack the texture, the knowing feel, of a longstanding witness to power. Even Maistre’s professions of divine providence cannot conceal or contain the turbulent democracy that generated them. Made and mobilized to counter the claims of emancipation, such statements do not disclose a dense ecology of deference; they reveal instead a rapidly thinning forest. Conservatism is about power besieged and power protected. It is an activist doctrine for an activist time. It waxes in response to movements from below and wanes in response to their disappearance, as Hayek and other conservatives admit.{\color{blue}81}
 \par 
Far from compromising the vision of excellence set out above— in which the prerogatives of rule are supposed to bring an element of grandeur to an otherwise drab and desultory world—the activist imperative only strengthens it. “Light and perfection,” Matthew Arnold wrote, “consist, not in resting and being, but in growing and becoming, in a perpetual advance in beauty and wisdom.” {\color{blue}82} To the conservative, power in repose is power in decline. The “mere husbanding of already existing resources,” wrote Joseph Schumpeter about industrial dynasties, “no matter how painstaking, is always
 \par 
Characteristic of a declining position.” {\color{blue}83} If power is to achieve the distinction the conservative associates with it, it must be exercised. {\color{blue}84} And there is no better way to exercise power than to defend it against an enemy from below. Counterrevolution, in other words, is one of the ways in which the conservative makes feudalism seem fresh and medievalism modern.
 \par 
But it is not the only way. Conservatism also off ers a defense of rule, independent of its counterrevolutionary imperative, that is agonistic and dynamic and dispenses with the staid traditionalism and harmonic registers of hierarchies past. And here we come to the conservative’s deepest intimations of the good life, of that reactionary utopia he hopes one day to bring into being. Unlike the feudal past, where power was presumed and privilege inherited, the conservative future envisions a world where power is demonstrated and privilege earned: not in the antiseptic and anodyne halls of the meritocracy, where admission is readily secured—“the road to eminence and power, from obscure condition, ought not to be made too easy, nor a thing too much of course” {\color{blue}85} —but in the arduous struggle for supremacy. In that struggle, nothing matters, not inheritance, social connections, or economic resources, but one’s native intelligence and innate strength. Genuine excellence is revealed and rewarded, true nobility is secured. “‘ Nitor in adversum’ [I strive against adversity] is the motto for a man like me,” declares Burke, after dismissing a to-the-manor-born politician who was “swaddled, and rocked, and dandled into a legislator.” {\color{blue}86} Even the most biologically inclined and deterministic racist believes that the members of the superior race must personally wrest their entitlement to rule through the subjugation or elimination of the inferior races.
 \par 
The recognition that race is the substratum of all civilization must not, however, lead anyone to feel that membership in a
 \par 
Superior race is a sort of comfortable couch on which he can go to sleep. . . . The biological heritage of the mind is no more imperishable than the biological heritage of the body. If we continue to squander that biological mental heritage as we have been squandering it during the last few decades, it will not be many generations before we cease to be the superiors of the Mongols. Our ethnological studies must lead us, not to arrogance, but to action.{\color{blue}87}
 \par 
The battlefield, as we shall see in part 2, is the natural proving ground of superiority; there, it is only the soldier, with his wits and weapon, who determines his standing in the world. With time, however, the conservative will find another proving ground in the marketplace. Though most early conservatives were ambivalent about capitalism, {\color{blue}88} their successors will come to believe that warriors of a different kind can prove their mettle in the manufacture and trade of commodities. Such men wrestle the earth’s resources to and from the ground, taking for themselves what they want and thereby establishing their superiority over others. The great men of money are not born with privilege or right; they seize it for them-selves, without let or permission. {\color{blue}89} “Liberty is a conquest,” wrote William Graham Sumner. {\color{blue}90} The primal act of transgression— requiring daring, vision, and an aptitude for violence and violation {\color{blue}91} —is what makes the capitalist a warrior, entitling him not only to great wealth but also, ultimately, to command. For that is what the capitalist is: not a Midas of riches but a ruler of men. A title to property is a license to dispose, and if a man has the title to another’s labor, he has a license to dispose of it—to dispose, that is, of the body in motion—as he sees fit.
 \par 
Such have been called “captains of industry.” The analogy with military leaders suggested by this name is not misleading. The
 \par 
Great leaders in the development of the industrial organization need those talents of executive and administrative skill, power to command, courage, and fortitude, which were formerly called for in military affairs and scarcely anywhere else. The industrial army is also as dependent on its captains as a military body is on its generals. . . . Under the circumstances there has been a great demand for men having the requisite ability for this function. . . . The possession of the requisite ability is a natural monopoly.{\color{blue}92}
 \par 
The warrior and the businessman will become twin icons of an age in which, as Burke foresaw, membership in the ruling classes must be earned, often through the most painful and humiliating of struggles. “At every step of my progress in life (for in every step was I traversed and opposed), and at every turnpike I met, I was obliged to shew my passport, and again and again to prove my sole title to the honor of being useful to my Country. . . . Otherwise, no rank, no toleration even, for me.”{\color{blue}93}
 \par 
Even though war and the market are the modern agones of power—with Nietzsche the theoretician of the first and Hayek of the second—the embrace of capitalism on the right has never bee nunqualified. To this day, as I show in part 2, conservatives remain leery of the shabbiness and shallowness of making money, of the political autism the market seems to induce in the governing classes, and of the foolishness and frivolity of consumer culture. For this wing of the movement war will always remain the only activity where the best man can truly prove his right to rule. It’s a bloody business, to be sure, but how else to be an aristocrat when all that’s solid melts into air?
 \par 
In the last two decades, there has been a flurry of interest in the American right, resulting in a body of scholarship—much of it by younger historians, many of them on the left—that has dramatically transformed our understanding of conservatism in the United
 \par 
States. {\color{blue}94} Much of my own reading of conservative thought has been informed by this literature—its emphasis on the lived realities of race, class, and gender as they have manifested themselves in the partisan struggles of the last half-century; the syncretism between high politics and mass culture; and the creative tension between elites and activists, businessmen and intellectuals, suburbs and Southerners, movement and media. Believing with T. S. Eliot that conservatism is best understood by “careful examination of its behavior throughout its history and by examination of what its more thoughtful and philosophical minds have said on its behalf,” {\color{blue}95} I have read the theory in light of the practice (and the practice in light of the theory). With the help of this scholarship, I have listened for the “metaphysical pathos” of conservative thought—the hum and buzz of its implications, the assumptions it invokes and associations it evokes, the inner life of the movement it describes. {\color{blue}96} The felt presence of this scholarship is what distinguishes, I hope, my interpretation of conservative thought from other interpretations, which tend to read the theory in seclusion from the practice or in relation to a highly stylized account of that practice.{\color{blue}97}
 \par 
As sophisticated as the recent literature about conservatism is, however, it suffers from three weaknesses. The first is a lack of comparative perspective. Scholars of the American right rarely examine the movement in relation to its European counterpart. Indeed, among many writers, it seems to be an article of faith that, like all things American, conservatism in the United States is exceptional. “There is a distinctly American feel to Bush and his intellectual defenders,” writes Mattson. “A conservatism that draws on Edmund Burke, a conservatism of wisdom and tradition deeply rooted in a European context” is “the sort of conservatism that has never taken hold in America.” {\color{blue}98} The commitment to laissez-faire capitalism on this side of the Atlantic is supposed to differentiate American conservatism from the traditionalism of a Burke or Disraeli; a native
 \par 
Pragmatism renders American conservatism inhospitable to the pessimism and fanaticism of a Bonald; democracy and populism make untenable the aristocratic biases of a Tocqueville. But this assumption is premised, I will show, on misapprehensions about the European right: not even Burke was as traditional as writers have made him out to be, while Maistre held views on the economy that were—like so much else in his revanchist writings—surprisingly modern. {\color{blue}99} Indeed, there are deep points of contact—particularly over questions of race and violence—between the radical right in Europe and figures like Calhoun, Teddy Roosevelt, Barry Goldwater, and the neoconservatives. In the postwar era, many of conservatism’s leading lights self-consciously turned to Europe in search of guidance and instruction, a service European émigrés—most notably, Hayek, Ludwig von Mises, and Leo Strauss—were only too happy to provide. {\color{blue}100} Indeed, for all the focus on the Frankfurt School and Hannah Arendt, it seems that the only political movements in postwar America that truly felt the impress of the European mind were on the right.
 \par 
The second weakness is a lack of historical perspective. No matter how far back writers and scholars push the origins of con-temporary conservatism (the latest move argues for a long conservative movement that connects the Tea Party to the 1920s), {\color{blue}101} there is a notion in the recent literature that contemporary conservatism is fundamentally different from earlier iterations. At some point, the argument goes, American conservatism broke with its predecessors—it became populist, ideological, and so on—and it is this break, depending upon one’s perspective, that either saved or doomed it. {\color{blue}102} But this argument ignores the continuities between figures like Adams and Calhoun and more recent voices on the American right. Far from an innovation of the last decades, the populism of the Tea Party and the futurism of a Reagan or Gingrich can be found in the earliest voices of conservatism, on both sides
 \par 
Of the Atlantic. Likewise, the adventurism, racism, and penchant for ideological thinking.
 \par 
The third weakness derives from the second. The further back analysts trace the origins of contemporary conservatism, the less inclined they are to believe that it is a politics of reaction or back-lash. If the commitments of the contemporary conservative can be situated in the writings of Albert Jay Nock or John Adams, these scholars argue, conservatism must reflect ideas and commitments more transcendent than mere opposition to the Great Society would suggest. {\color{blue}103} But a recognition of the long history of the right need not undermine the claim that contemporary conservatism is a backlash politics. Instead, the long view should help us to under-stand better the nature and dynamics, as well as the idiosyncrasies and contingencies, of that backlash. Indeed, only by setting the contemporary right against the backdrop of its predecessors can we understand its specific city and particularity.
 \par 
Against these three assumptions, which dwell on difference and distinction, I treat the right as a unity, as a coherent body of theory and practice that transcends the divisions so often emphasized by scholars and pundits. {\color{blue}104} I use the words conservative, reactionary, and counterrevolutionary interchangeably: not all counterrevolutionaries are conservative—Walt Rostow immediately comes to mind—but all conservatives are, in one way or another, counterrevolutionary. I seat philosophers, statesmen, slaveholders, scribblers, Catholics, fascists, evangelicals, businessmen, racists, and hacks at the same table: Hobbes next to Hayek, Burke across from Palin, Nietzsche in between Ayn Rand and Antonin Scalia, with Adams, Calhoun, Oakeshott, Ronald Reagan, Tocqueville, Theodore Roosevelt, Margaret Thatcher, Ernst Jünger, Carl Schmitt, Winston Churchill, Phyllis Schlafl y, Richard Nixon, Irving Kristol, Francis Fukuyama, and George W. Bush interspersed throughout.
 \par 
This is not to say that there is no change in conservatism across time or space. If conservatism is a specific reaction to a specific movement of emancipation, it stands to reason that each reaction will bear the traces of the movement it opposes. As I argue in chapter 1, not only has the right reacted against the left, but in the course of conducting its reaction, it also has consistently borrowed from the left. As the movements of the left change—from the French Revolution too abolition to the right to vote to the right to organize to the Bolshevik Revolution to the struggles for black freedom and women’s liberation—so do the reactions of the right.
 \par 
Beyond these contingent changes, we can also trace a longer structural change in the imagination of the right: namely, the gradual acceptance of the entrance of the masses onto the political stage. From Hobbes to the slaveholders to the neoconservatives, the right has grown increasingly aware that any successful defense of the old regime must incorporate the lower orders in some capacity other than underlings or starstruck fans. The masses must either be able to locate themselves symbolically in the ruling class or be provided with real opportunities to become faux aristocrats themselves in the family, the factory, and the field. The former path makes for an upside-down populism, in which the lowest of the low see themselves projected in the highest of the high; the latter makes for a democratic feudalism, in which the husband or supervisor plays the part of a lord. The former path was pioneered by Hobbes, Maistre, and various prophets of racism and nationalism, the latter by Southern slaveholders, European imperialists, and Gilded Age apologists. (And neo–Gilded Age apologists: “There is no single elite in America,” writes David Brooks. “Everyone can be an aristocrat within his own Olympus.” {\color{blue}105}) Occasionally, as in the writing of Werner Sombart, the two paths con-verge: ordinary people get to see themselves in the ruling class by virtue of belonging to a great nation among nations, and they also get to govern lesser beings through the exercise of imperial rule.
 \par 
We Germans, too, should go through the world of our time in the same way, proud heads held high, in the secure feeling of being God’s people. Just as the German bird, the eagle, soars high over all animals on this earth, so the German must feel himself above all other peoples that surround him and that he sees in boundless depth below him.
 \par 
But aristocracy has its obligations, and this is true here, too. The idea that we are chosen people places formidable duties— and only duties—on us. We must above all maintain ourselves as a strong nation in the world.{\color{blue}106}
 \par 
While these historical differences on the right are real, there is an underlying affinity that draws these differences together. One cannot perceive this affinity by focusing on disagreements of policy or contingent statements of practice (states’ rights, federalism, and so on); one must look to the underlying arguments, the idioms and metaphors, the deep visions and metaphysical pathos evoked in each disagreement and statement. Some conservatives criticize the free market, others defend it; some oppose the state, others embrace it; some believe in God, others are atheists. Some are vocalists, others nationalists, and still others internationalists. Some, like Burke, are all three at the same time. But these are historical improvisations—tactical and substantive—on a theme. Only by juxtaposing these voices—across time and space—can we make out the theme amid the improvisation.
 \par 
For many, the notion of a unity on the right will be the most contentious claim of this book. Even though we continue to use the term “conservative” in our everyday discourse (indeed, political discussion would be inconceivable without it); even though conservatism in both Europe and the United States has managed, for more than a century, to attract and hold together a coalition of traditionalists, warriors, and capitalists; even though the opposition
 \par 
Between left and right has proven to be an enduring “political distinction” of the modern era (despite the attempts, every generation or so, to deny or overcome this opposition via a “third way”) {\color{blue}107} —many continue to believe the differences on the right are so great it would be impossible to say anything about the right. {\color{blue}108} But if it is impossible to say anything about the right—to defi né, describe, explain, analyze, and interpret the right as a distinctive formation—how can we say that it even exists?
 \par 
Hoping to avoid that radical skepticism, which would render unintelligible much of what goes on in our politics, some scholars have retreated to a nominalist position: conservatives are people who call themselves conservative or, more elaborately, conservatives are people who call themselves conservative call conservative. {\color{blue}109} This only begs the question: What do these people who call themselves conservative—or who others who call themselves conservative call conservative—mean by “conservative”? Why do they opt for that self-description as opposed to liberal, socialist, or aardvark? Unless these people think they are referring to idiosyncratic identities—in which case we’re back to the skeptical position—we need to understand what the term means, independent of its use. How else can we understand why individuals from different times and places, adopting different positions on different issues, would call themselves and their kindred spirits conservative? While not every reader need to accept my claim about what unites the right, it seems a necessary condition of intelligent discussion that we agree that there is something called the right and that it has some set of common features that make it right.
 \par 
The eleven chapters of this book have been culled from a decade’s worth of writing about the right. Some chapters originally appeared as lengthy review essays for periodicals like The Nation and the London
 \par 
Review of Books ; others are academic research articles, reported pieces, or stand-alone essays. I have made some alterations to these pieces to account for new developments or changes in my views. Occasionally, I have eliminated entire sections because they no longer seemed relevant. But on the whole I have tried to leave the pieces intact in the hope that their varied approaches capture this notion of the right as a set of historical improvisations on a continuous theme. The book is divided into two parts. Part {\color{blue}1} opens with a general statement about the counterrevolutionary thrust of conservative politics, from the French Revolution through today. This chapter focuses less on the aims and intentions of the counterrevolution and more on its moves and maneuvers: how it breaks with the very regime it is defending and looks to the left in its efforts to recon-struct the right. I then move chronologically, from an examination of Thomas Hobbes and the English Civil War to a concluding analysis of Justice Scalia and his originalism jurisprudence. Along the way, I discuss Rand, Goldwater, the New Right, and conservatives after the Cold War. Part {\color{blue}2} looks at the fraught topic of violence in conservatism. Though I open with a brief look back at the Latin American Cold War and conclude with a more general real ection on how the right has approached violence since Burke, most of the discussion in these chapters is drawn from the past decade: 9/11, the war on terror, the war in Iraq. These events, and the giddiness they inspired among conservatives, more than anything drove me to think and write about the right. As I came to realize, and as chapter {\color{blue}11} argues, the infatuation with violence on today’s right is not an aberration; it is constitutive of the tradition itself.