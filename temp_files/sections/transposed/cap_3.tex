TTTTTT hen I was in my twenties, a dear friend of mine, whom I will call Lisa, worked in human resources for a large corporation in San Francisco. Lisa loved fashion, and my wardrobe still includes elegant ensembles she put together for me on our frequent bargain shopping excursions to Filene’s Basement and various thrift stores on Fillmore Street. She had a knack for choosing discount designer treasures and assembling outfits that mixed Levi’s with vintage Dior. Over the years, we kept in touch, commiserating over marriage and new motherhood. But whereas I started my life as a working mom on the tenure track, Lisa quit her job to become a stay-at-home mom as soon as she realized she was pregnant. Her husband earned enough to support her, and he preferred that she not be employed. His own mother had stayed home, and among their immediate friends, neighbors, and peers, this was the normal arrangement. Lisa claimed this was her choice; she wanted a break from the rat race of corporate America. She had a second child soon after the first and abandoned the idea of returning to the workforce. Lisa thought it was easier this way; she
 \par 

 \par 
25
 \par 
26
 \par 
WOMEN—LIKE MEN, BUT CHEAPER: ON WORK would be physically there for her daughters in a way that I never could be for mine.
 \par 
In those early years, while she baked cookies and organized playmates, I dropped my daughter off at a full-time day care center, five days a week, costing me a small fortune. While her girls napped, Lisa read novels, worked out, and cooked lavish meals. My first four years of motherhood coincided with my first three years on the tenure track. My life was a crushing routine of harried days. The first time I taught class with my shirt inside out, I cringed with embarrassment when a sympathetic student pointed to my seams. But after the third time, I stopped caring. As long as my skirt wasn’t on backwards, it was fine. I often envied Lisa’s choice, but I'd earned my PhD and landed a good job. I didn’t want to quit. Once my daughter turned five, things got a bit easier. My first book came out, I earned tenure, and my daughter started first grade. Out from under the crippling day care bills, I started to reap the psychological and financial rewards of my perseverance.
 \par 
A few years later, I spent a weekend with Lisa. Her husband offered to stay in with our three girls so she and I could head to the mall: dinner, a movie, and maybe a little shopping. Our social engagements usually included our children, so this was a real treat. I longed for a few hours of adult conversation with an old friend and no urgent demands for juice or ice cream or unexpected tantrums. A real girls’ night out.
 \par 
I'd been upstairs at her house getting ready when I realized I'd forgotten my hair dryer. I wanted to ask Lisa if I could borrow hers, but as I started down the steps, I heard Lisa fighting with her husband.
 \par 
KRISTEN R. GHODSEE ne ee re an ee
 \par 
\chapter{WOMEN—LIKE MEN, BUT CHEAPER: ON WORK}\label{WOMEN—LIKE MEN, BUT CHEAPER: ON WORK}
 \par 
“... Please, Bill. It'll be embarrassing.” “No. You've spent enough money this month. I'll give
 \par 
\section{Com né}
 \par 
“But I shopped for the house and bought clothes for the girls. I didn’t buy anything for me.”
 \par 
“You're always buying things for yourself and saying it’s
 \par 
\section{You the card again after the statement rolls.”}
 \par 
“But it is for the girls. They keep growing.” “You have enough clothes. You don’t need anything else. I’ve given you enough for the dinner and the movie.”
 \par 
“Bill, please.” Lisa’s voice cracked. I turned to tiptoe back up the stairs, praying they hadn't heard me. I hid in the bathroom until Lisa came up, jaw clenched and eyes pink.
 \par 
We drove to the restaurant in silence. We ordered two courses, and I attempted to prolong the dinner until just before the film started. Lisa seemed grateful to linger.
 \par 
After our second glass of Malbec she said, “Bill and I had a fight.”
 \par 
I looked down at my plate. “He says we don’t have sex often enough.” I looked up. That’s not the fight I thought I heard. She swirled her empty glass. “You think we have time
 \par 
\section{For the girls.”}
 \par 
“You go ahead,” I said. “I'll drive.” She drank a third glass of wine, and we chatted about the reviews of the film we planned to see. When the check came, she opened her wallet and pushed some twenty-dollar bills across the table at me. I put down my credit card.
 \par 
She looked at the American Express with my name on it, and sighed. “Bill only gives me cash.”
 \par 
27
 \par 
28
 \par 
WOMEN—LIKE MEN, BUT CHEAPER: ON WORK
 \par 
“Why don’t you let me get this?” I slid the money back at her. “Keep it.”
 \par 
She stared down at the table for a long moment. Finally, she said, “Thanks,” and scooped the bills back into her wallet. “I'll fuck him tonight and pay you back tomorrow.”
 \par 
I sat there, stunned. Lisa looked at her watch. “If we hurry, I can hit the Shiseido counter before the movie starts.”
 \par 
Sitting in the restaurant that night, I swore to myself that no matter how hard it was to balance my full-time job with care for my daughter, I would never put myself in Lisa’s position if I had any choice in the matter. “Capitalism acts on women as a continual bribe to enter into sex relations for money, whether in or out of marriage; and against this bribe there stands nothing beyond the traditional respectability which Capitalism destroys by poverty,’ George Bernard Shaw wrote in 1928. Directly or indirectly, sex and money are always linked in women’s lives, a remnant of our long history of oppression.’
 \par 
Too many women find themselves in Lisa’s situation, economically dependent on men for their basic livelihoods. Divorce laws and court orders for child support and alimony will offer Lisa some (possibly inadequate) protection if Bill ever seeks to divorce her, but she remains at his mercy while they are married. All the labor she performs caring for their children, organizing their lives, and managing their home is invisible as far as the market is concerned. Lisa receives no wages and contributes no funds toward her own social security in old age. She accumulates no work
 \par 
KRISTEN R. GHODSEE experience and creates a black hole on her résumé, one that will require explaining away if she ever hopes to rejoin the labor force. She even accesses medical care through her husband’s employer. Everything she has she derives from Bill’s income, and he can deny her access to their joint credit cards at will.
 \par 

 \par 
In Margaret Atwood’s chilling dystopian novel, The Handmaid's Tale, the founders of the Republic of Gilead legislate a blanket prohibition on women’s employment and the seizure of their personal savings. All at once, anyone designated female is fired from her job, and the money in her bank account is transferred into the accounts of her husband or nearest male relative, the first step in returning women to their “rightful place.” The subjugation of women begins by making them economically dependent on men once more. Without money and without a means to earn it, women are helpless to determine the course of their own lives. Personal independence requires the resources to make your own choices.’
 \par 
Free markets discriminate against women workers. At the beginning of the industrial revolution, the big bosses considered women inferior to their male counterparts (weaker, more emotional, less reliable, and so forth). The only way to convince an employer to hire a woman was through financial incentives: women cost less and are more docile than men. If she demanded a wage equal to that of a man, the employer would just hire a man instead. Therefore, women’s comparative advantage in the workplace from the very earliest days of capitalism is that they will do the same work as a man for less money. The idea of the family wage compounds the problem. When women finally entered the
 \par 
29
 \par 
30
 \par 
WOMEN—LIKE MEN, BUT CHEAPER: ON WORK industrial labor force en masse and began to dominate light industries (like sewing, weaving, laundry), employers paid women wages for a single person, not a family, even if they were single mothers or widows. Society insisted that women were the dependents of men, and working women were conveniently imagined as wives and daughters earning pocket money to purchase lace doilies for their dressing tables. Husbands and fathers were supposed to meet their major needs for food, shelter, and clothing.
 \par 
Patriarchal cultures reduce women to economic dependence, treating them as a form of chattel to be traded among families. For centuries, the doctrine of overture rendered married women the property of their husbands with no legal rights of their own. All of a woman’s personal property transferred to her husband upon marriage. If your man wanted to hawk your rubies for rum, you had no right to refuse. Married West German women could not work outside the home without their husband’s permission until 1957. Laws prohibiting married women from entering into contracts without their husbands’ permission persisted in the United States until the 1960s. Women in Switzerland didn’t earn the right to vote at the federal level until 1971.°
 \par 
Under capitalism, industrialism reinforced a division of labor that concentrated men in the public sphere of formal employment and rendered women responsible for unpaid labor in the private sphere. In theory, male wages were high enough to allow men to support their wives and children. Women’s free labor in the home subsidized the profits of employers because workers’ families bore the cost of reproducing the future labor force. Without birth control, access to education, or opportunities for meaningful employment,
 \par 
KRISTEN R. GHODSEE the woman was trapped within the confines of the family in perpetuity. “Under the capitalist system women found themselves worse off than men,” Bernard Shaw wrote in 1928, “because, as Capitalism made a slave of the man, and then by paying women through him, made her his slave, she became the slave of a slave, which is the worst sort of slavery.”
 \par 
As early as the mid-nineteenth century, feminists and socialists diverged on how best to liberate women. Wealthier women advocated for the Married Women’s Property Acts and the right to vote without questioning the overall economic system that perpetuated women’s subjugation. Socialists, such as the German theorists Clara Zetkin and August Bebel, believed women’s liberation required their full incorporation into the labor force in societies in which the working classes collectively owned the factories and productive infrastructure. This was a much more audacious and perhaps utopian goal, but all subsequent experiments with socialism would include women’s labor force participation as part of their program to refashion the economy on a more just and equitable basis.
 \par 
The perception that a woman’s labor is less valuable than a man’s persists to this day. In a capitalist system, labor power (or the units of time we sell to our employers) is a commodity traded in the free market. The laws of supply and demand determine its price, as does the perceived value of that labor. Men are paid more because employers, clients, and customers perceive that they are worth more. Think about it: Why do cheap diners always have waitresses, but
 \par 
31
 \par 
32
 \par 
WOMEN—LIKE MEN, BUT CHEAPER: ON WORK expensive restaurants often have male waiters? In the comfort of our own homes, most of us grow up being served by women: grandmothers, mothers, wives, sisters, and sometimes daughters. But being served by men is rare, as is having men look after our basic needs. We pay a premium to have a man serve us our dinner because we perceive this service as more valuable, even if all he does is set a plate in front of you and grind fresh pepper onto your filet mignon. Similarly, although women have fed humanity for millennia, men dominate the culinary world. Apparently, customers prefer a side of testosterone with their mashed potatoes.* In the past, women understood the public valued their work less and took steps to mitigate against the effects of discrimination. Charlotte Bronté published her early novels under the pen name Currer Bell, and Mary Anne Evans wrote as George Eliot. More recently, both J. K. Rowling and E. L. James published books using their initials to obscure their gender. In Rowling’s case, her publisher asked her to do this to attract boy readers who might reject a book written by a woman. In the world of university teaching, having a female-sounding name results in worse teaching evaluations, as students consistently rank male professors higher than their female counterparts. One 2015 experimental study found that assistant instructors who taught the same online class under two different gender identities received lower ratings for their female persona.°
 \par 
Racism exacerbates gender discrimination. Hispanic and Black women suffer a larger wage gap than white women. When we talk about gender discrimination, we have to be careful not to privilege gender as the primary category of analysis, as some feminists have done in the
 \par 
KRISTEN R. GHODSEE past. The state of being female is complicated by other categories such as class, race, ethnicity, sexual orientation, disability, religious belief, and so on. Yes, {\color{blue}1} am a woman, but I am also a Puerto Rican—Persian from an immigrant and working-class background (my grandmother had a third grade education, and my mother only finished high school). The old concept of sisterhood ignores the structural aspects of capitalism that benefit white, middle-class women while disadvantaging working-class women of color, something that socialist women activists understood as early as the late nineteenth century. Within left circles, orthodox Marxists obsessed with class position are often called “socialists,” because they emphasize worker solidarity over issues of race and gender. Some feminists and socialists will argue that too much focus on identity politics divides people and undermines the potential power base for mass movements for social change, but when examining structures of oppression, we must be mindful of the hierarchies of subjugation even while building strategic coalitions.
 \par 
Taking an intersectional approach, for instance, helps us see how public-sector jobs have created important opportunities for different populations. While white working class men once dominated private manufacturing jobs, government employment provided important avenues for African Americans who were (and remain) more likely than whites to work in the public sector. The public sector has historically offered jobs to religious minorities, people of color, and women who faced discrimination in the private sector, creating career opportunities for those disadvantaged by race or gender in competitive labor markets. Cuts in public sector employment after the Great Recession hit
 \par 
33
 \par 
34
 \par 
WOMEN—LIKE MEN, BUT CHEAPER: ON WORK
 \par 
African American women particularly hard, forcing them to seek work in private companies, where perceptions of the value of their labor are more influenced by the color of their skin and their gender.
 \par 
A classic study showing the deep persistence of gender bias involved auditions for symphony orchestras. Women musicians were sorely underrepresented in professional orchestras before the introduction of an audition process whereby musicians played their instruments behind screens that separated them from the judges. In order to ensure total gender anonymity, musicians removed their shoes so that the footfalls of men and women would be indistinguishable to those making decisions. When those doing the auditioning judged musicians solely on their ability to play, “the percent of female musicians in the five highest-ranked orchestras in the nation increased from {\color{blue}6} percent in 1970 to {\color{blue}21} percent in 1993.” This screen audition system would also eliminate racial biases.’
 \par 
But we can’t hide ourselves behind screens for all of our job interviews and interactions with potential employers. Our names give us away, and even if we manage to hide our gender behind initials or male pseudonyms, references use pronouns and other words that reveal our gender. Proving discrimination is difficult, and there are few repercussions for those who systematically pay women less than men for the same work. Furthermore, because women earn less than men, it makes economic sense for mothers to stay home with young children when affordable child care is scarce. When women do enter the labor force as part-time or flexible workers, they often do so without benefits and without wages sufficient to cover their basic needs. And as
 \par 
KRISTEN R. GHODSEE women withdraw from the labor force to care for the young, the sick, or elderly relatives, discrimination against women workers becomes more entrenched, since employers view them as less reliable (more on this in the next chapter), and the cycle of women’s economic dependence on men continues.
 \par 
To counter the effects of discrimination and the wage gap, socialist countries devised policies to encourage or require women’s formal labor force participation. To a greater or lesser degree, all state socialist countries in Eastern Europe demanded the full incorporation of women into paid employment. In the Soviet Union and particularly in Eastern Europe after World War II, labor shortages drove this policy. Women have always been used as a reserve army of labor when the men are off at war (just like American women’s employment during the World War II era of Rosie the Riveter). But unlike the United States and West Germany, where women were “let go” after the soldiers returned home, East European states guaranteed women’s full employment and invested vast resources in their education and training. These nations promoted women’s labor in traditionally male professions such as mining and military service, and mass-produced images of women driving heavy machinery, especially tractors.*
 \par 
For example, while American women were stocking their kitchens with the latest appliances during the postwar economic boom, the Bulgarian government encouraged girls to pursue careers in the new economy. In 1954, the state produced a short documentary film to celebrate the lives of women helping to transform agricultural Bulgaria into a modern, industrial power. This film, J Am a Woman
 \par 
35
 \par 
36
 \par 
WOMEN—LIKE MEN, BUT CHEAPER: ON WORK
 \par 
Official statistics from the International Labor Organization (ILO) demonstrate the disparity between the workforce participation rates in state socialist economies and those in market economies. In 1950, the female share of the total Soviet labor force was {\color{blue}51}.{\color{blue}8} percent, and the female share of the total workforce in Eastern Europe was {\color{blue}40}.{\color{blue}9} percent, compared to {\color{blue}28}.{\color{blue}3} percent in North America and {\color{blue}29}.{\color{blue}6} percent in Western Europe. By 1975, the United Nations’ International Year of Women, women made up {\color{blue}49}.{\color{blue}7} percent of the Soviet Union’s workforce and {\color{blue}43}.{\color{blue}7} percent of that in the Eastern Bloc, compared to {\color{blue}37}.{\color{blue}4} percent in North America and {\color{blue}32}.{\color{blue}7} percent in Western Europe. These findings led the ILO to conclude that the “analysis of data on women’s participation in economic activity in the USSR and the socialist countries
 \par 
KRISTEN R. GHODSEE of Europe shows that men and women in these countries enjoy equal rights in all areas of economic, political, and social life. The exercise of these rights is guaranteed by granting women equal opportunities with men in access to education and vocational training and in work.”®
 \par 
Of course, women’s own accounts complicate the rosy picture painted by the ILO in 1985. Gender pay gaps still existed in East European countries. And despite the attempts to funnel women into traditionally male employment, there remained a gendered division of labor whereby women worked in white-collar professions, the service sector, and light industry, compared to men who worked in the higher-paid sectors of heavy industry, mining, and construction. But salaries mattered less when there was little to buy with one’s wages and where formal employment itself guaranteed social services from the state. In many countries women had no choice; they were forced to work when their children were old enough to go to kindergarten. And women in state socialist countries suffered the double burden of housework and formal employment (a problem very familiar to many working women today). Consumer shortages plagued the economy, and both men and women waited in lines to acquire basic goods. But as workers, women contributed to their own pensions and developed their own skill sets. They benefitted from free health care, public education, and a generous social safety net that subsidized shelter, utilities, public transportation, and basic foodstuffs. In some countries, women could retire from formal employment up to five years earlier than men.
 \par 
Despite the-shortcomings of the command economy, the socialist system also promoted a culture in which
 \par 
37
 \par 
38
 \par 
WOMEN—LIKE MEN, BUT CHEAPER: ON WORK women’s formal labor force participation was accepted and even celebrated. Before World War II, Eastern Bloc countries were deeply patriarchal, peasant societies with conservative gender relations emerging from both religion and traditional culture. Socialist ideologies challenged centuries of women’s subjugation. Because the state required girls’ education and compelled women into the labor force, their fathers and husbands couldn't force them to stay home. Women seized these opportunities for education and employment. When birthrates began to falter in the late 1960s, many Communist Party leaders worried that their investments in women would hurt their economies in the long run. They conducted sociological surveys and found that women indeed struggled with their dual responsibilities as workers and mothers. Some governments considered allowing women to return to the home, but when asked if they would be happier if their husbands earned enough to support the family, the majority of women rejected the traditional breadwinner/homemaker model. They wanted to work. In Natalya Baranskaya’s novella about a harried Soviet working mom, the protagonist never once fantasizes about quitting her job, stating unequivocally that she loves her work."°
 \par 
Reflecting on the achievements of state socialism compared to the situation of women in most East European countries prior to the Second World War, Hungarian sociologist Zsuzsa Ferge explained, “All in all. . The objective situation of women has probably improved everywhere compared to the pre-war situation. Their paid work outside the home contributed to the well-being of the family (at least it helped to make ends meet); their educational
 \par 
.
 \par 
KRISTEN R. GHODSEE advancement and the work outside the home enriched (at least in a majority of cases) their life experience; their status as earners weakened their former oppression within and outside the family and made them (somewhat) less subservient in some walks of life. Also, it attenuated female poverty, especially in the case of mothers who practically all started to work, and of older women who obtained a pension in their own right.” State socialist countries could promote women’s economic autonomy because the state was the primary employer, and it guaranteed each man and woman full employment as a right and a duty of citizenship. In the democratic socialist countries of Northern Europe, women’s employment is voluntary, but the state promotes their labor force participation by providing the social services necessary to help citizens combine their roles as workers and parents.”
 \par 
Socialist states also try to counter the persistent discrimination against women by expanding opportunities for public sector employment. Although not as sexy as start-ups, governments can ensure women are paid equal (decent) wages for equal work and support women in their work and family responsibilities. According to a report from the Organization for Economic Co-operation and Development (OECD), the Scandinavian countries lead the world not only in terms of gender equality but also in terms of public sector employment. This is no coincidence. In 2015, {\color{blue}30} percent of total employment in Norway was government employment, followed by {\color{blue}29}.{\color{blue}1} percent in Denmark, {\color{blue}28}.{\color{blue}6} percent in Sweden, and {\color{blue}24}.{\color{blue}9} percent in Finland. The United Kingdom,
 \par 
39
 \par 
40
 \par 
WOMEN—LIKE MEN, BUT CHEAPER: ON WORK by contrast, employed only {\color{blue}16}.{\color{blue}4} percent of its total employed population in the public sector, and in the United States this figure was {\color{blue}15}.{\color{blue}3} percent. Even more remarkable is that women account for around {\color{blue}70} percent of all public employees in Norway, Denmark, Sweden, and Finland, and the OECD average is {\color{blue}58} percent. The authors of the report explain women’s overrepresentation in the public sector partially because teachers and nurses are female-dominated
 \par 
Professions, but also because of “more flexible working conditions in the public than in the private sector. For example, in sixteen OECD countries the public sector offers more child and family care arrangements than the private sector.” Finally, studies reveal smaller wage gaps between men and women in the public sector.”
 \par 
Public sector employment rates used to be higher in the United States until federal agencies began outsourcing, subcontracting to the private sector, or just slashing jobs. A 2013 report analyzing US employment trends showed a precipitous decline in public sector employment after the
 \par 
Great Recession, as states and localities pruned budgets after the crisis. The Hamilton Project examined government responses to previous recessions and found that cutting the jobs of teachers, emergency responders, and air traffic controllers during a time of high unemployment slowed the recovery and inflicted greater economic pain on American citizens, particularly on the younger generation, who were crowded into larger classrooms with fewer educators. “The ongoing recovery, which began when the Great Recession ended in June 2009, dramatically deviates from the usual pattern,” write the project’s researchers. “In the forty-six months following the end of the five other recent recessions,
 \par 
KRISTEN R. GHODSEE government employment increased by an average of {\color{blue}1}.7 million. During the current recovery, however, government employment has decreased by more than 500,000, and a disproportionate number of those losing jobs were women. Put together, the policy differences have led to {\color{blue}2}.2 million fewer jobs today. Such a large contraction of the public sector during a recovery is unprecedented in recent American economic history.”
 \par 
Attitudes toward public sector employment reflect ideological divides about whether the government is more or less efficient than the market. Our banks are private because Americans believe that state-run banks (like a postal bank) would be more bureaucratic and less consumer friendly than those forced to compete on free markets for depositors’ money (even if the federal government provides deposit insurance up to $250,000 and bails out banks deemed “too big to fail”). Similarly, the United States rejects a national health system because our private health insurance supposedly provides better care at lower prices as a result of market competition. Although countless studies show that Americans pay more money for health care, Americans cling to the idea that markets produce better outcomes than state-run programs even when presented with copious evidence to the contrary. Another example is in higher education, with the expansion of for-profit universities. A 2016 study shows that employers don’t value for-profit college degrees as much as they value degrees from public universities. Yet government funds provide substantial financial aid for students at these universities, thus subsidizing profits for investors when those funds could be used to strengthen the quality of public education instead.
 \par 
4l
 \par 
42
 \par 
Citizens in other societies, even our close allies in Canada and the United Kingdom, understand that the profit motive sometimes undermines the public good.”
 \par 
Of course, some might argue that instead of expanding public employment, the government could legislate pay equality and enforce provisions to ensure that private sector firms pay women fairly, a step the Icelandic government took beginning in early 2018 and the state of Massachusetts took after July 1, 2018. But federal legislation on equal pay in the United States has been relatively weak and without real teeth, since the onus remains on women to prove pay discrimination in court (and who has the money needed for a lawsuit?). Attempts to strengthen the 1963 Equal Pay Act have failed to win Republican support in Congress, most recently in April 2017 with the Paycheck Fairness Act, which did not receive a single Republican vote.
 \par 
Critics will also claim that expanded public sector employment hurts growth and cripples the private sector, but private sector job expansion has not been able to reverse wage stagnation, the rise of the gig economy, or the incredible growth in inequality between the rich and the poor, as revealed by Thomas Piketty. Economists and legislators will have to debate the details, but given that as of 2017, just eight men own the same amount of wealth as the {\color{blue}3}.6 billion people who make up the poorest half of humanity, redistribution is going to come in one form or another. Current levels of inequality are unsustainable in the long term. In a global economy buoyed by the credit-fueled consumer spending of the masses, the bubble will burst eventually. An acute crisis of overproduction and underconsumption looms on the horizon.'°
 \par 
‘
 \par 
KRISTEN R. GHODSEE
 \par 
The expansion of public services would support women in a second way. A wider social safety net means that women's lower private sector wages don’t disadvantage them in terms of access to health care, clean water, child care, education, or security in old age. Rather than trying to legislate equality or coerce private companies into providing equal pay for equal work and giving women equal opportunities for promotion, women could join together to choose leaders who will lessen the social costs of gender discrimination through public policy. Another idea is some form of guaranteed employment like what they had in the state socialist countries. This is an old economic concept to prevent the human suffering caused during economic downturns. The United Kingdom’s Labour Party has proposed a job guarantee in which the state acts as the employer of last resort for young people ages eighteen to twenty-five who are willing to work but cannot find employment. Economists have debated job guarantees for decades, and in 2017 the Center for American Progress (CAP) threw its weight behind a proposal for a new “Marshall Plan for America,” which would create {\color{blue}4}.4 million new jobs. The CAP proposal calls for a “large-scale, permanent program of public employment and infrastructure investment—similar to the Works Progress Administration (WPA) during the Great Depression but modernized for the 21st century. It will increase employment and wages for those without a college degree while providing needed services that are currently out of reach for lower-income households and cash-strapped state and local governments.”
 \par 
In September 2017, I attended Mass with my eighty nine-year-old grandmother in the San Diego church I
 \par 
43 went to as a kid. On that Sunday, the priest introduced the parable of the workers in the vineyard (Matthew 20:1-16) by explaining that, for Americans, it was one of the most controversial of the parables. In Jesus’s story, a landowner goes into town to hire day laborers for a fair wage in the morning. He then returns to hire more men at noon and later in the afternoon. Near sunset, the landowner returns to find more idle men. He asks why they are not working, and they explain that no one has hired them that day. The landowner hires them and then proceeds to pay all the workers the same wage no matter how long they worked. When the workers hired early in the morning complain about the unfairness, the landowner chastises them: I offered you a fair wage, and you accepted it. The landowner says, I am not being unfair to you. Or are you envious because I am generous? Although parables are typically interpreted allegorically, that day the priest used the story to talk about fair wages and immigration in his homily. “The landowner went into town and hired the men who needed work,” he told us. “He didn’t ask to see their documents.” In the same way, perhaps, the parable also supports the idea of job guarantees. The landowner provided employment to all those willing and able to work, and he paid them a fair wage no matter how long they actually labored in the vineyard. From the landowner’s perspective, it was a generous thing to do for people in need. For Americans, such generosity sounds suspiciously socialist.'”
 \par 
But let’s face it: job guarantees would not only benefit women. In the long run if privately owned robots and A.l. take over our economy, organic men may find themselves just as devalued in competitive labor markets as
 \par 
KRISTEN R. GHODSEE today’s organic women. The owners of inorganic life may be the real beneficiaries of our future unregulated free markets. Fears of the increasing automation have led some to promote the idea of the Universal Basic Income (UBI), sometimes called a Universal Citizen’s Income or Citizen’s Dividend. This would guarantee that all qualifying citizens received a fixed monthly payment to meet their basic needs. A generous UBI experiment has been tried in Finland, and many people across the political spectrum support the idea of some kind of flat payment to save people from the ravages of unemployment. This revenue could be generated from taxation of the private sector or from the profits of public enterprises. UBI could go a long way in promoting gender equality, since women’s unpaid labor in the home would be compensated. Of course, some critics fear that UBI will make people lazy, while others worry that it is just a way for the hyper-rich to gut the welfare state and buy off the masses with small cash payments while they luxuriate in their riches. It is an idea that requires much more debate, particularly from a socialist perspective."
 \par 
Whatever happens, any move toward employment assurances will require a substantial expansion of the public sector, which will have the added benefit of promoting gender equality by eliminating the wage gap between men and women. The irony here is that where state socialist regimes reduced women’s economic dependence on men by making men and women equally dependent on the state, in a capitalist society our technological future may reduce women’s economic dependence on men by making men and women equally dependent on the generosity of those who own our robot overlords. Some day in the near future, Bill may be
 \par 
45
 \par 
46
 \par 
WOMEN—LIKE MEN, BUT CHEAPER: ON WORK begging a computer to give him his Friday night allowance so he can head to the sports bar with his buddies. There will be some cosmic justice when Siri informs Bill that he’s already watched enough sports this month, and should stay home and spend some quality time with his wife and daughters.
 \par 
