{\chapter{The First Counterrevolutionary} } {\label{The First Counterrevolutionary} }{\par}{\textit{	} } {\par}{\par} {\textbf{\textit{	} } } {\par} 
	Revolution sent Thomas Hobbes into exile; counterrevolution sent him back. In 1640, parliamentary opponents of Charles I, such as John Pym, were denouncing anyone “preaching for absolute monarchy that the king may do what he lists.” Hobbes had recently fin- ished writing The Elements of Law, which did just that. After the king’s top adviser and a theologian arguing for unlimited royal power were both arrested, Hobbes decided it was time to go. Not waiting for his bags to be packed, he fl ed England for France. {\color{blue} 1 } {\par} Eleven years and a civil war later, Hobbes fl ed France for Eng-land. This time, he was running from the royalists. As before, Hobbes had just finished a book. Leviathan, he would later explain, “fights on behalf of all kings and all those who under whatever name bear the rights of kings.” {\color{blue} 2 } It was the second half of that claim, with its seeming indifference about the identity of the sovereign, that was now getting him into trouble. Leviathan justified, no, demanded, that men submit to any person or persons capable of protecting them from foreign attack and civil unrest. With the monarchy abolished and Oliver Cromwell’s forces in control of england and providing for the people’s safety, Leviathan seemed to suggest that everyone, including the defeated royalists, profess their allegiance to the Commonwealth. Versions of that argument had already gotten Anthony Assam, ambassador for the Commonwealth, assassinated by royalist exiles in Spain. So when Hobbes learned that clergymen in France were trying to arrest him— Leviathan was also vehemently anticatholic, which off ended the Queen Mother—he slipped out of Paris and made his way back to London. {\color{blue} 3 } {\par} It’s no accident that Hobbes fl ed his enemies and then his friends, for he was fashioning a political theory that shredded long-standing alliances. Rather than reject the revolutionary argument, he absorbed and transformed it. From its deepest categories and idioms he derived an uncompromising defense of the most hide-bound form of rule. He sensed the centrifugal forces of early modern Europe—the priesthood of all believers; the democratic armies massing under the banner of ancient republican ideals; science and skepticism—and sought to channel them to a single center: a sovereign so terrible and benign as to make any challenge to such authority seem immoral and irrational. Not unlike the Italian Futurists, Hobbes put dissolution in the service of resolution. He was the first and, along with Nietzsche, the greatest philosopher of counterrevolution, a blender savant la letter of cultural modernism and political reaction who understood that to defeat a revolution, you must become the revolution.{\par} And how has he been treated by the right? Not well. T. S. Eliot (an adroit blender himself) called Hobbes “one of those extraordinary little upstarts whom the chaotic motions of the Renaissance tossed into an eminence which they hardly deserved.” {\color{blue} 4 } Of the four twentieth-century political theorists identified by Perry Anderson as “The Intransigent Right” {\color{blue} 5 } —Leo Strauss, Carl Schmitt, Michael Makeshift, and Friedrich Hayek—only Makeshift saw in Hobbes a kindred spirit. {\color{blue} 6 } The rest viewed him as the source of a malignant liberalism, Jacobitism, or even Bolshevism. {\color{blue} 7 } {\par} Orthodox custodians of the old regime often mistake the counterrevolutionary for the opposition because they can’t follow the alchemy of his argument. All they sense is what’s there—a newfangled way of thinking that sounds dangerously like the revolutionary’s—and what’s not there: the traditional justification for authority. To the orthodox, the counterrevolutionary looks like a revolutionary. That makes the counterrevolutionary a suspect, in their eyes, not a comrade. In this they are not entirely wrong. Neither left nor, conventionally speaking, right—one of Hayek’s most famous pieces of writing is called “Why I Am Not a Conservative” {\color{blue} 8 } —the counterrevolutionary is a pastiche of incongruities, high and low, old and new, irony and faith. The counterrevolutionary attempts nothing less than to square the circle, making prerogative popular and remaking a regime that claims never to have been made in the first place (the old regime was, is, and will be; it is not made). These are tasks no other political movement must undertake. The counterrevolutionary is not disposed to paradox; he’s simply forced to straddle historical contradictions, for power’s sake.{\par} But why even bring Hobbes before the bar of conservatism, the right, and counterrevolution? After all, none of these terms came into circulation until the French Revolution or later, and most historians no longer believe the English Civil War was a revolution. The forces that overthrew the monarchy may have been looking for the Roman Republic or the ancient constitution. They may have wanted a reformation of religious manners or limitations on royal power. But a revolution lay nowhere in their sights. How could Hobbes have been a counterrevolutionary if there was no revolution for him to oppose?Hobbes, for one, thought otherwise. In Behemoth, his most considered treatment of the issue, he firmly declared the English Civil war a revolution. {\color{blue} 9 } And though he meant by that term something like what the ancients meant—a cyclical process of regime change, more akin to the orbit of the planets than a great leap forward— Hobbes saw in the overthrow of the monarchy a zealous (and, to his mind, toxic) yearning for democracy, a firm desire to redistribute power to a greater number of men. That, for Hobbes, was the essence of the revolutionary challenge; and so it has remained ever since—whether in Russia in 1917, Flint in 1937, or Selma in 1965. That this democratic expansion was inspired by visions of the past rather than the future need not detain us any more than it did Hobbes—or Benjamin Constant or Karl Marx, for that matter, both of whom saw how easy it was for the French to make their revolution while (or even by) looking backward. {\color{blue} 10 } {\par} Hobbes clearly opposed the “democratically,” as he called the parliamentary forces and their followers. {\color{blue} 11 } A considerable sum of his philosophical energy was expended in this opposition, and his greatest innovations derived from it. {\color{blue} 12 } His specific target was the republicans’ conception of liberty, their notion that individual freedom entailed men collectively governing themselves. Hobbes unfastened the republican links between personal freedom and the possession of political power. He thus was able to argue that men could be free in an absolute monarchy—or at least no less free than they were in a republic or a democracy. It was “an epoch-making moment in the history of Anglophone political thought,” says Quentin Skinner. The result was a novel account of liberty to which we remain indebted to this day. {\color{blue} 13 } {\par} Every counterrevolutionary faces the same question: how to defend an old regime that has been or is being destroyed? The first impulse—to reiterate the regime’s ancient truths—is usually the worst, for it is often those truths that got the regime into trouble in the first place. Either the world has so changed that they no longer command assent, or they have grown so pliable that they mutate into arguments for revolution. Either way, the counterrevolutionary must look elsewhere for materials from which to fashion his defense of the old regime. This need can put him at odds, as Hobbes came to realize, not only with the revolution, but also with the very regime he claims as his cause.{\par} The monarchy’s defenders in the first half of the seventeenth century off and two types of arguments, neither of which Hobbes could endorse. The first was the divine right of kings. Itself a recent innovation—James I, Charles’s father, was the major exponent in Britain—it held that the king was God’s agent on earth (indeed, was rather like God on earth), that he was accountable only to God, and that he alone was authorized to govern and should not be restrained by the law, institutions, or the people. As Charles’s adviser allegedly put it, “the king’s little finger should be thicker than the loins of the law.” {\color{blue} 14 } {\par} While such absolutism appealed to Hobbes, the foundation of the theory was shaky. Most divine right theorists presumed what Hobbes and his contemporaries, particularly on the continent, believed no longer to exist: a teleology of human ends that mirrored the natural hierarchy of the universe and produced unassailable definitions of good and evil, just and unjust. After a century of bloodshed over the meaning of those terms and skepticism about the existence of a natural order or our ability to know it, defenses of divine right seemed neither credible nor reliable. With their dubious premises, they were just as likely to spark conflict as to settle it.{\par} Arguably more troubling was that the theory depicted a political theater in which there were only two actors of any consequence: God and king, each performing for the other. Though Hobbes believed the sovereign should never share the stage with anyone, he was too attuned to the democratic distemper of his times not to notice that the theory neglected a third actor: the people. That was all well and good when the people were quiet and deferential, but during the 1640s a closet drama between God and the king was no longer viable. The people were onstage, demanding a leading role; they could not be ignored or given a bit part.{\par} Changes in England, in short, had rendered divine right untenable. The challenge Hobbes faced was intricate: how to preserve the thrust of the theory (unquestioning submission to absolute, undivided power) while ditching its anachronistic premises. With his theory of consent, in which individuals contract with one another to create a sovereign with absolute power over them, and his theory of representation, in which the people are impersonated by the sovereign without his being obliged to them, Hobbes found his solution.{\par} The theory of consent made no assumptions about the defibrillator NI- tion of good and evil, nor did it rely upon a natural hierarchy inherent in the universe, whose meaning must be apparent to all. To the contrary, the theory of consent presumed that men dis-agreed about such things; indeed, that they disagreed so violently that the only way they could pursue their conflicting goals and survive was to cede all of their power to the state and submit to it without protest or challenge. Protecting men from one another, the state guaranteed them the space and security to get on with their lives. When combined with Hobbes’s account of representation, the theory of consent had an added advantage: though it gave all power to the sovereign, the people could still imagine them-selves in his body, in every swing of his sword. The people created him; he represented them; to all intents and purposes, they were him. Except that they weren’t: the people may have been the authors of Leviathan—Hobbes’s infamous name for the sovereign, derived from the Book of Job—but like any author they had no control over their creation. It was an inspired move, characteristic of all great counterrevolutionary theories, in which the people become actors without roles, an audience that believes it is onstage. The second argument off and in favor of the monarchy, the constitutional royalist position, had deeper roots in English thought and was therefore more difficult to counter. It held that England was a free society because royal power was limited by the common law or shared with Parliament. That combination of the rule of law and shared sovereignty, claimed Sir Walter Raleigh, was what distinguished the free subjects of the king from the benighted slaves of despots in the East. {\color{blue} 15 } It was this argument and its radical off shoots that quickened Hobbes’s most profound and daring re- fl actions about liberty. {\color{blue} 16 } {\par} Beneath the constitutionalist conception of political liberty lay a distinction between acting for the sake of reason and acting at the behest of passion. The first is a free act; the second is not. “To act out of passion,” writes Skinner in his account of the argument Hobbes arrayed himself against, “is not to act as a free man, or even distinctively as a man at all; such actions are not an expression of true liberty but of mere license or animal brutishness.” Freedom entails acting upon what we have willed; but will should not be confused with appetite or aversion. As Bishop Bram hall, Hobbes’s great antagonist, put it: “A free act is only that which proceeds from the free election of the rational will.” And “where there is no consideration nor use of reason, there is no liberty at all.” {\color{blue} 17 } Being free entails acting in accordance with reason or, in political terms, living under laws as opposed to arbitrary power.{\par} Like the divine right of kings, the constitutional argument had been rendered anachronistic by recent developments, most notably the fact that no English monarch in the first half of the seventeenth century claimed to believe it. Intent on turning England into a modern state, James and Charles were compelled to advance far more absolutist claims about the nature of their power than the constitutional argument allowed.{\par} More troubling for the regime, however, was how easily the constitutional argument could be turned into a republican one and used against the king. Common lawyers and parliamentary supplicants argued that by fl outing the common law and Parliament, Charles was threatening to turn England into a tyranny; radicals insisted that anything short of a republic or democracy, where men lived under laws to which they had consented, constituted a tyranny. All monarchy, in the eyes of the radicals, was despotism.{\par} Hobbes thought that the latter argument derived from the “Histories, and Philosophy of the Ancient Greeks, and Romans,” which were so influential among educated opponents of the king. {\color{blue} 18 } That ancient heritage was given new life by Machiavelli’s Discourse, translated into English in 1636, which may have been Hobbes’s ultimate target in his admonition against popular government. But the underlying premise of the republican argument—that what distinguishes a free man from a slave is that the former is subject to his own will while the latter is subject to the will of another—could also be found in English common law, as Skinner points out, in a “word-for-word” reproduction of “the Digest of Roman law,” as early as the thirteenth century. Likewise, the distinction between will and appetite, liberty and license, was “deeply embedded” in both the scholastic traditions of the Middle Ages and the humanist culture of the Renaissance. This philosophy of will thus found expression not only in the royalist positions of Bram hall and his ilk, but also among the radicals and regicides who overthrew the king. Beneath the chasm separating royalist and republican lay a deep and volatile bedrock of shared assumption about the nature of liberty. {\color{blue} 19 } Hobbes’s genius was to recognize that assumption; his ambition was to crush it.{\par} While the notion that freedom entails living under laws lent support to the constitutional royalists (who made much of the distinction between lawful monarchs and despotic tyrants) it did not necessarily lead to the conclusion that a free regime had to be a republic or a democracy. To advance that argument, the radicals had to make two additional claims: first, to equate arbitrariness or lawlessness with a will that is not one’s own, a will that is external or alien, like the passions; and second, to equate the decisions of a popular government with a will that is one’s own, like reason. To be subject to a will that is mine—the laws of a republic or democracy—is to be free; to be subject to a will that is not mine—the edicts of a king or foreign country—is to be a slave.{\par} In making these claims, Skinner argues, the radicals were aided by a peculiar, though popular, understanding of slavery. What made someone a slave, in the eyes of many, was not that he was in chains or that his owner impeded or compelled his movements. It was that he lived and moved under a net, the ever-changing, arbitrary will of his master, that might fall upon him at any moment. Even if that net never fell—the master never told him what to do or never punished him for not doing it, or he never desired to do something different from what the master told him—the slave was still enslaved. The fact that he “lived in total dependence” on the will of another, that he was under the master’s jurisdiction, “was sufficient in itself to guarantee the servility” that the master “expected and despised.” 20{\par} {\textbf{\textit{The mere presence of relations of domination and dependence. . . Is held to reduce us from the status of. . . “Free-men” to that of slaves. It is not sufficient, in other words, to enjoy our civic rights and liberties as a matter of fact; if we are to count as free-men, it is necessary to enjoy them in a particular way. We must never hold them merely by the grace or goodwill of} } }{\par} {\par} {\textbf{\textit{Anyone else; we must always hold them independently of any-one’s arbitrary power to take them away from us. {\color{blue} 21 } } } }{\par} At the individual level, freedom means being one’s own master; at the political level, it requires a republic or democracy. Only a full share in public power will ensure we enjoy our freedom in the “particular way” freedom requires; without full political participation, freedom will be fatally abridged. It is this double movement between the personal and the political that is arguably the most radical element of the theory of popular government and, from Hobbes’s view, the most dangerous.{\par} Hobbes sets about destroying the argument from the ground up. Breaking with traditional understandings, he argues for a materialist account of the will. The will, he says, is not a decision resulting from our reasoned deliberation about our desires and aversions; it is simply the last appetite or aversion we feel before we act, which then prompts the act. Deliberation is like the oscillating rod of a metronome—back and forth our inclinations go, alternating between appetite and aversion—but less steady. Wherever the rod comes to rest, and produces an action or, conversely, no action at all, turns out to be our will. If this conception seems arbitrary and mechanistic, it should: the will does not stand above our appetites and aversions, judging and choosing between them; the will is our appetites and aversions. There is no such thing as a free or autonomous will; there is only “the last Appetite, or Aversion, immediately adhering to the action, or to the omission thereof.” {\color{blue} 22 } {\par} Imagine a man with the keenest appetite for wine, racing into a building on fire in order to rescue a case of it; now imagine a man with the fiercest aversion to dogs, racing into that same building to escape a pack of them. Hobbes’s opponents would see in these examples the force of irrational compulsion; Hobbes sees the will in action. These may not be the wisest or sanest acts, Hobbes allows, but wisdom and sanity need not play any part in volition. Both acts may be compelled, but so are the actions of a man on a listing vessel who throws his bags overboard in order to lighten the load and save himself. Hard choices, actions taken under duress— these are as many expressions of my will as the decisions I make in the calm of my study. Extending the analogy, Hobbes would argue that the surrender of my wallet to someone holding a gun to my head is also a willed act: I have chosen my life over my wallet.{\par} Against his opponents, Hobbes suggests that there can be no such thing as voluntarily acting against one’s will; all voluntary action is an expression of the will. External constraints like being locked in a room can prevent me from acting upon my will; being on a chain gang can force me to act in ways I have not willed (when my neighbor takes a step forward or lifts his tool, I must follow him, unless I have sufficient physical force to resist him and the fellow behind me). But I cannot act voluntarily against my will. In the case of the mugger, Hobbes would say that his gun changed my will: I went from wanting to safeguard the money in my wallet to wanting to protect my life.{\par} If I can’t act voluntarily against my will, I can’t act voluntarily in accordance with a will that is not my own. If I obey a king because I fear that he will kill or imprison me, that does not signify the absence, forfeiture, betrayal, or subjection of my will; it is my will. I could have willed otherwise—hundreds of thousands during Hobbes’s lifetime did—but my survival or liberty was more important to me than whatever it was that may have called for my disobedience.{\par} Hobbes’s definition of freedom follows from his understanding of the will. Liberty, he says, is “the absence of. . . Externall Impediments of motion,” and a free man“ is he, that in those things, which by his strength and wit he is able to do, is not hundred to doe what he has a will to.” {\color{blue} 23 } I can be rendered unfree, Hobbes insists, only by external obstacles to my movement. Chains and walls are such obstacles; laws and obligations are another, albeit a more metaphorical, sort. If the obstacle lies within me—I don’t have the ability to do some-thing; I am too afraid to do it—I lack power or will, not freedom. Hobbes, in a letter to the earl of Newcastle, attributes these defibrillator - agencies to “the nature and intrinsic quality of the agent,” not the conditions of the agent’s political environment. {\color{blue} 24 } {\par} And that is the purpose of Hobbes’s effort: to separate the status of our personal liberty from the state of public affairs. Freedom is dependent on the presence of government but not on the form government takes; whether we live under a king, a republic, or a democracy does not change the quantity or quality of the freedom we enjoy. The separation between personal and political liberty had the dramatic effect of making freedom seem both less present and more present under a king than Hobbes’s republican and royalist antagonists had allowed.{\par} On the one hand, Hobbes insists that there is no way to be free and subject at the same time. Submission to government entails an absolute loss of liberty: wherever I am bound by law, I am not free to move. When republicans argue that citizens are free because they make the laws, Hobbes claims, they are confusing sovereignty with liberty: what the citizen has is political power, not freedom. He is just as obliged (perhaps more obliged, Rousseau will later suggest) to submit to the law, and thus just as unfree, as he would be under a monarchy. And when the constitutional royalists argue that the king’s subjects are free because the king’s power is limited by the law, Hobbes claims that they are just confused.{\par} On the other hand, Hobbes thinks that if freedom is unimpeded motion, it stands to reason that we are a lot freer under a monarch, even an absolute monarch, than the royalist and the republican realize (or care to admit). {\color{blue} 25 } First and most simply, even when we act out of fear, we are acting freely. “Free, and Liberty are consistent,” says Hobbes, because fear expresses our negative inclinations; these inclinations may be negative, but that doesn’t negate the fact that they are our inclinations. So long as we are not impeded from acting upon them, we are free. Even when we are most terrified of the King’s punishments, we are free: “all actions which men doe in Commonwealths, for free of the law, are actions, which the doers had liberty to omit.” {\color{blue} 26 } {\par} More important, wherever the law is silent, neither commanding nor prohibiting, we are free. One need only contemplate all the “ways a man may move himself,” Hobbes says in De Five, to see all the ways he can be free in a monarchy. These freedoms, Hobbes explains in Leviathan, include “the Liberty to buy, and sell, and otherwise contract with one another; to choose their own aboard, their own diet, their own trade of life, and institute their children as they themselves think fit; & the like.” {\color{blue} 27 } To whatever degree the sovereign can guarantee the freedom of movement, the ability to go about our business without the hindrance of other men, we are free. Submission to his power, in other words, augments our freedom. The more absolute our submission, the more powerful he is and the freer we are. Subjugation is emancipation.{\par} Despite the disclaimers of the “Intransigent Right,” the Hobbes argument continues to haunt modern conservatism. Hobbes’s idea of private liberty pervades libertarian discourse, while Leviathan casts a long shadow over the conservative ideal of a night watchman state—where the government’s primary purpose is to protect the citizenry from foreign attack and criminal trespass; where people are free to go about their business so long as they do not interfere with the movements of others; and where contracts are enforced and security is ensured.{\par} Libertarians will blanch at that association: whatever resonance Hobbes ideas may find in their writings, the Hobbes state is a good deal more repressive than any government a libertarian would ever countenance. Except for the fact that it’s not. Milton Friedman famously met with the Chilean dictator Augusto Pinochet in 1975 to advise him on economic matters; Friedman’s Chicago Boys worked even more closely with Pinochet’s junta. Sergio de Castro, Pinochet’s finance minister, made the observation, reminiscent of Hobbes, that “a person’s actual freedom can only be ensured through an authoritarian regime that exercises power by implementing equal rules for everyone.” Hayek admired Pinochet’s Chile so much that he decided to hold a meeting of his Mont Pelegrín Society in Viña del Mar, the seaside resort where the coup against Allende was planned. In 1978, he wrote to the London Times that he had “not been able to find a single person even in much maligned Chile who did not agree that personal freedom was much greater under Pinochet than it had been under Allende.” 28“Despite my sharp disagreement with the authoritarian political system of Chile,” Friedman would later claim, “I do not regard it as evil for an economist to render technical economic advice to the Chilean Government.” {\color{blue} 29 } The marriage between free markets and state terror cannot be annulled so easily. As Hobbes under-stood, it takes an enormous amount of repression to create the type of men who can exercise their “Liberty to buy, and sell, and otherwise contract with one another” without getting stroppy. {\color{blue} 30 } They must be free to move—or choose—but not so free as to think about redesigning the highway. Assuming an all-too-easy congruence between capitalism and democracy, the libertarian overlooks just how much coercion is required to make citizens who will use their freedom responsibly and accept distress without turning to the state for relief.{\par} It took Margaret Thatcher, of all people, to explain this fact to the libertarian right. When pressed by Hayek to impose Pinochet’s brand of shock therapy in Britain, Thatcher responded, “I am sure you will agree that, in Britain with our democratic institutions and the need for a high degree of consent, some of the measures adopted in Chile are quite unacceptable.” It was 1982, and British democracy being what it was, Thatcher had to go slow. But then came the Falklands War and the miners’ strike. Once Thatcher realized that she could do to the miners and the trade unions what she had done to President Gautier and his Argentine generals— “We had to fight the enemy without in the Falklands, and now we have to fight the enemy within, which is much more difficult but just as dangerous to liberty”—the stage was set for the full Bayesian Monty. 31{\par}