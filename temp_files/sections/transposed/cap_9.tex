{\chapter{Remembrance of Empires Past} } {\label{Remembrance of Empires Past} }{\par}{\textit{	In 2000, I spent the better part of a late summer interviewing William F. Buckley and Irving Bristol. I was writing an article for Lingua Franca (see chapter {\color{blue} 5 }) on the defections to the left of right-wing intellectuals and wanted to hear what the movement’s founding fathers thought of their wayward sons. Over the course of our conversations, however, it became clear that Buckley and Bristol were less interested in these ex-conservatives than they were in the sorry state of the conservative movement and the uncertain fate of the United States as a global empire. The end of Communism and the triumph of the free market, they told me, were mixed blessings. While they were conservative victories, these developments had nevertheless rendered the United States ill-equipped for the postcold War era. Americans now possessed the most powerful empire in history. At the same time, they were possessed by one of the most anti-political ideologies in history: the free market.} } {\par}{\par} {\textbf{\textit{	— Henry IV, Part 2} } } {\par} 
{\footnote{This chapter originally appeared as “Remembrance of Empires Past: 9/11 and the End of the Cold War,” in Cold War Triumphalism: The Misuse of History after the Fall of Communism, ed. Ellen Checker (New York: New Press, 2004), 274–297.} }	According to its idealists, or at least one camp of its idealists, the free market is a harmonious order, promising an international civil society of voluntary exchange, requiring little more from the state than the occasional enforcement of laws and contracts. For Buckley and Bristol, this was too bloodless a notion upon which to found a national order, much less a global empire. It did not provide the passion and élan, the gravitas and authority, that the exercise of American power truly required, at home and abroad. It encouraged triviality and small-minded politics, self-interest over the national interest—not the most promising base from which to launch an empire. What’s more, the right-wingers in charge of the Republican Party didn’t seem to realize this.“The trouble with the emphasis in conservatism on the market,” Buckley told me, as we saw in chapter 5, “is that it becomes rather boring. You hear it once, you master the idea. The notion of devoting your life to it is horrifying if only because it’s so repetitious. It’s like sex.” Conservatism, Bristol added, “is so influenced by business culture and by business modes of thinking that it lacks any political imagination, which has always been, I have to say, a property of the left.” Bristol confessed to a deep yearning for an American empire: “What’s the point of being the greatest, most powerful nation in the world and not having an imperial role? It’s unheard of in human history. The most powerful nation always had an imperial role.” But, he continued, previous empires were not “capitalist democracies with a strong emphasis on economic growth and economic prosperity.” Because of its commitment to the free market, the United States lacked the fortitude and vision to wield imperial power. “It’s too bad,” Bristol lamented. “I think it would be natural for the United States. . . To play a far more dominant role in world affairs. Not what we’re doing now but to command and to give orders as to what is being done. People need that. There are many parts of the world, Africa in particular, where an authority willing to use troops can make a very good difference, a healthy difference.” But with public discussion moderated by accountants, Bristol thought it unlikely that the United States would take its rightful place as the successor to empires past. “There’s the Republican Party tying itself into knots. Over what? Prescriptions for elderly people? Who gives a damn? I think it’s dis-gusting that. . . Presidential politics of the most important country in the world should revolve around prescriptions for elderly people. Future historians will find this very hard to believe. It’s not Athens. It’s not Rome. Furthermore, it’s not anything.” {\color{blue} 1 } {\par} Since 9/11, I’ve had many occasions to recall these conversations. September 11, we were told in the aftermath, shocked the United States out of the complacent peace and prosperity that set in after the Cold War. It forced Americans to look beyond their borders, to understand at last the dangers that confront a world power. It reminded us of the goods of civic life and of the value of the state, putting an end to that fantasy of creating a public world out of private acts of self-interested exchange. Furthermore, it restored to our woozy civic culture a sense of depth and seriousness, of things “larger than ourselves.” Most critical of all, it gave the United States a coherent national purpose and focus for imperial rule. A country that seemed for a time unwilling to face up to its international responsibilities was now prepared, once again, to bear any burden, pay any price, for freedom. This changed attitude, the argument went, was good for the world. It pressed the United States to create a stable and just international order. It was also good for the United States. Furthermore, it forced us to think about something more than peace and prosperity, reminding us that freedom was a fighting faith rather than a cushy perch.{\par} Like any historical moment, 9/11—not the terrorist attacks or the day itself, but the new wave of imperialism it spawned— has multiple dimensions. Some part of this rejuvenated imperial political culture is the product of a surprise attack on civilians and the efforts of U.S. leaders to provide some measure of security to an apprehensive citizenry. Some part of it flows from the subterranean political economy of oil, from the desire of U.S. elites to secure access to energy reserves in the Middle East and Central Asia, and to wield oil as an instrument of geopolitics. But while these factors play a considerable role in determining U.S. policy, they do not explain entirely the politics and ideology of the imperial moment itself. To understand that dimension, we must look to the impact on American conservatives of the end of the Cold War, of the fall of Communism and the ascendancy of the free market as the organizing principle of the domestic and international order. For it was conservative dissatisfaction with that order that drove, in part, their effort to create a new one.{\par} For neoconservatives who thrilled to Ronald Reagan’s crusade against communism, all that was left after the Cold War was Reagan’s other passion—his sunny entrepreneurialism and market joie de vivre—which found a welcome home in Bill Clinton’s America. While neocons are certainly not opposed to capitalism, they do not believe the free market is the highest achievement of civilization. Their vision is more exalted. They aspire to the epic grandeur of Rome, the ethos of the pagan warrior—or moral crusader—rather than that of the comfortable bourgeois. Since the end of the Cold War, the imperial vision has received short shrift, eclipsed by the embrace of free markets and free trade. Undone by their own success, neoconservatives are not happy with the world they created. And so they have taken up the call of empire, providing the basso prof undo to a swelling chorus. Though they have complete faith in American power, the neocons are uncomfortable using it for the mere extension of capitalism. They seek to create an international order that will be a monument for the ages, a world that is about something more than money and markets.{\par} But as we have come to learn this envisioned imperium may not provide such an easy resolution to the challenges confronting the United States. Even before the war in Iraq went south, the American empire was coming up against daunting obstacles in the Middle East and Central Asia, suggesting how elusive the reigning idea of the neocon imperialists—that the United States can govern events, that it can make history—truly is. (Indeed, it was not so long ago that the Bush administration was telling journalists, “We’re an empire now, and when we act, we create our own reality. And while you’re studying that reality—judiciously as you will— we’ll act again, creating other new realities, which you can study.”) {\color{blue} 2 } Domestically, the cultural and political renewal that many imagined 9/11 would produce has proven a chimera, the victim of a free-market ideology that shows no sign of abating. As it turns out, 9/11 did not—and, in all truth, probably could not—fulfill the role ascribed to it by the neocons of empire.{\par} Immediately following the attacks on the World Trade Center and the Pentagon, intellectuals, politicians, and pundits—not on the radical left, but mainstream conservatives and liberals—breathed an audible sigh of relief, almost as if they welcomed the strikes as a deliverance from the miasma Buckley and Bristol had been criticizing. The World Trade Center was still on fire and the bodies entombed there scarcely recovered when Frank Rich announced that “this week’s nightmare, it’s now clear, has awakened us from a frivolous if not decadent decade-long dream.” What was that dream? The dream of prosperity, of surmounting life’s obstacles with money. During the 1990s, Maureen Down wrote, we hoped “to overcome fl ab with diet and exercise, wrinkles with collagen and Botox, sagging skin with surgery, impotence with Viagra, mood swings with anti-depressants, myopia with laser surgery, decay with human growth hormone, disease with stem cell research and bioengineering.” We “renovated our kitchens,” observed David Brooks, “refurbished our home entertainment systems, invested in patio furniture, Jacuzzis and gas grills”—as if affluence might free us of tragedy and difficulty. {\color{blue} 3 } This ethos had terrible domestic consequences. For Francis Fukuyama, it encouraged “self-indulgent behavior” and a “preoccupation with one’s own petty affairs.” It also had international repercussions. According to Lewis “Scooter” Libby, the cult of peace and prosperity found its purest expression in Bill Clinton’s weak and distracted foreign policy, which made “it is easier for someone like Osama bin Laden to rise up and say credibly ‘The Americans don’t have the stomach to defend themselves. They won’t take casualties to defend their interests. They are morally weak.’” According to Brooks, even the most casual observer of the pre-9/11 domestic scene, including Alqaeda, “could have concluded that America was not an entirely serious country.” {\color{blue} 4 } {\par} But after that day in September, more than a few commentators claimed, the domestic scene was transformed. America was now “more mobilized, more conscious and therefore more alive” wrote Andrew Sullivan. George Packer remarked upon “the alertness, grief, resolve, even love” awakened by 9/11. “What I dread now,” Packer confessed, “is a return to the normality we’re all supposed to seek.” For Brooks, “the fear that is so prevalent in the country” after 9/11 was “a cleanser, washing away a lot of the self-indulgence of the past decade.” Revivifying fear eliminated the anxiety of prosperity, replacing a disabling emotion with a bracing passion. “We have traded the anxieties of affluence for the real fears of war.” 5{\par} {\textbf{\textit{Now upscales who once spent hours agonizing over which Men faucet head would go with their copper farmhouse-kitchen sink are suddenly worried about whether the water coming out of pipes has been poisoned. People who longed for} } }{\par} {\par} {\textbf{\textit{Prada bags at Bloomingdale's are suddenly spooked by unattended bags at the airport. America, the sweet land of liberty, is getting a crash course in fear. {\color{blue} 6 } } } }{\par} Today, Brooks concluded, “commercial life seems less important than public life. . . . When life or death fighting is going on, it’s hard to think of Bill Gates or Jack Welch as particularly heroic.” {\color{blue} 7 } {\par} Writers repeatedly welcomed the galvanizing moral electricity now coursing through the body politic. A pulsing energy of public resolve and civic commitment, which would restore trust in government—perhaps, according to some liberals, even authorize a revamped welfare state—and bring about a culture of patriotism and connection, a new bipartisan consensus, the end of irony and the culture wars, a more mature, more elevated presidency. {\color{blue} 8 } According to a reporter at USA Today, President Bush was especially keen on the promise of 9/11, off bring himself and his generation as Exhibit A in the project of domestic renewal. “Bush has told advisors that he believes confronting the enemy is a chance for him and his fellow baby boomers to refocus their lives and prove they have the same kind of valor and commitment their fathers showed in WWII.” And while the specific source of Christopher Hitchens’s elation may have been peculiarly his own, his self-declared schadenfreude assuredly was not: “I should perhaps con-fess that on September {\color{blue} 11 } last, once I had experienced all the usual mammalian gamut of emotions, from rage to nausea, I also discovered that another sensation was contending for mastery. On examination, and to my own surprise and pleasure, it turned out to be exhilaration. Here was the most frightful enemy—theocratic barbarism—in plain view. . . . I realized that if the battle went on until the last day of my life, I would never get bored in prosecuting it to the utmost.” {\color{blue} 9 } With its shocking spectacle of fear and death, 9/11 off and a dead or dying culture the chance to live again.{\par} Internationally, 9/11 forced the United States to reengage the world, to assume the burden of empires without embarrassment or confusion. Where the first George Bush and Bill Clinton had fumbled in the dark, searching for a doctrine to guide the exercise of U.S. power after the collapse of the Soviet Union, the mission of the United States was now clear: to defend civilization against barbarism, freedom against terror. As Condoleezza Rice told the New Yorker, “I think the difficulty has passed in defining a role. I think September 11th was one of those great earthquakes that clarify and sharpen. Events are in much sharper relief.” An America thought to be lost in the quicksand of free markets, individualism, and isolation was now recalled to a consciousness of a world beyond its borders, and inspired to a commitment to sustain casualties on behalf of a U.S.-led global order. As Clinton’s former undersecretary of defense concluded, “Americans are unlikely to slip back into the complacency that marked the first decade after the Cold War.” They now understood, in the words of Brooks, that “evil exists” and that “to preserve order, good people must exercise power over destructive people.” {\color{blue} 10 } {\par} A decade later, it’s difficult to recapture, let alone fathom, the mindset of that moment. Not just because it disappeared so quickly, with the country relapsing to its strange and sour partisan-ship—where the volume of rhetorical antagonism between the parties is matched only by the depth of their agreement about the economic fundamentals (in that respect, we’re still living in Bill Clinton’s America)—before Bush’s first term had even ended. More bewildering is how so many writers and politicians could open their arms to the political fallout from mass death, taking 9/11 as an opportunity to express their apparently long-brewing contempt for the very peace and prosperity that preceded it. On September 12, one might have expected expressions of sorrow over the bursting of bubbles—economic, cultural, and political. Instead,Many saw 9/11 as a thunderous judgment upon, and necessary corrective to, the frivolity and emptiness of the 1990s. We would have to reach back almost a century—to the opening days of World War I, when the “marsh gas of boredom and vacuity” enveloping another free-trading, globalizing fin de siècle exploded—to find a remotely exact parallel. {\color{blue} 11 } {\par} To understand this spirit of exuberant relief, we must revisit the waning days of the Cold War, when American elites first saw that the United States would no longer be able to defi né its mission in terms of the Soviet menace. While the end of the Cold War unleashed a wave of triumphalism, it also provoked among elites an anxious uncertainty about U.S. foreign policy. With the defeat of Communism, many asked, how should the United States defi né its role in the world? Where and when should it intervene in foreign conflicts? How big a military should it field?Underlying these arguments was a deep unease about the size and purpose of American power. The United States seemed to be suffering from a surfeit of power, which made it difficult for elites to formulate any coherent principles to govern its use. As Richard Cheney, then serving as the first President Bush’s secretary of defense, acknowledged in February 1992, “We’ve gained so much strategic depth that the threats to our security, now relatively distant, are harder to defi né.” Almost a decade later, the United States would still seem, to its leaders, a floundering giant. As Condoleezza Rice noted during the 2000 presidential campaign, “The United States has found it exceedingly difficult to defi né its ‘national interest’ in the absence of Soviet power.” So uncertain about the national interest did political elites become that a top Clinton defense aide—and later dean of Harvard’s Kennedy School— eventually threw up his hands in defeat, declaring the national interest to be whatever “citizens, after proper deliberation, say it is”—an abdication simply unthinkable during the Cold War reign of the Wise Men. {\color{blue} 12 } {\par} When Clinton assumed off ice, he and his advisers took stock of this unparalleled situation—where the United States possessed so much power that it faced, in the words of Clinton National Security Adviser Anthony Lake, no “credible near-term threat to [its] existence”—and concluded that the primary concerns of American foreign policy were no longer military but economic. After summarily rehearsing the various possible military dangers to the United States, President Clinton declared in a 1993 address, “We still face, overarching everything else, this amorphous but profound challenge in the way humankind conducts its commerce.” The great imperative of the post–Cold War era was to organize a global economy where citizens of the world could trade across borders. For that to happen, the United States had to get its own economic house in order—“renewal starts at home,” said Lake—by reducing the deficit (in part through reductions in military spending), lowering interest rates, supporting high-tech industry, and promoting free trade agreements. Because other nations would also have to conduct a painful economic overhaul, Lake concluded that the primary goal of the United States was the “enlargement of the world’s free community of market democracies.” {\color{blue} 13 } {\par} Clinton’s assessment of the challenges facing the United States was partially inspired by political calculation. He had just won an election against a sitting president who not only had led the United States through victory in the Cold War, but also had engineered a stunning rout over the Iraqi military. A Southern governor with no foreign policy experience—and a draft-dodger to boot—Clinton concluded that his victory over Bush meant that questions of war and peace no longer resonated with American voters the way they might have in an earlier age. {\color{blue} 14 } But Clinton’s vision also reflected a conviction, common to the 1990s, that the globalization of the free market had undermined the efficacy of military power and the viability of traditional empires. Force was no longer the sole, or most effective, instrument of national will. Power now hinged upon the dynamism and success of a nation’s economy and the attractiveness of its culture. As Joseph Nye, Clinton’s assistant secretary of defense, would come to argue, “soft power”—the cultural capital that made the United States so admired around the globe—was as important to national preeminence as military power. In perhaps a first for a U.S. official, Nye invoked Gram sci to argue that the United States would only maintain its position of hegemony if it persuaded—rather than forced—others to follow its example. “If I can get you to want to do what I want,” wrote Nye, “then I do not have to force you to do what you do not want to do.” {\color{blue} 15 } To maintain its standing in the world, the United States would have to out-compete other national economies, all the while ensuring the spread of its free market model and pluralist culture. The greatest danger confronting the United States was that it would not reform its economy or that it would abuse its military superiority and provoke international hatred. The problem was not that the United States did not have enough power, but that it had too much. To render the world safe for globalization, the United States would have to be deranged or, at a minimum, significantly curtailed in its imperial aspirations.{\par} For conservatives who yearned for and then celebrated social-ism’s demise, Clinton’s promotion of easygoing prosperity was a horror. Affluence produced a society without difficulty and adversity. Material satisfaction induced a loss of social depth and political meaning, a lessening of resolve and heroic verve. “In that age of peace and prosperity,” David Brooks would write, “the top sitcom was Seinfeld, a show about nothing.” Robert Kaplan emitted barb after barb about the “healthy, well-fed” denizens of “bourgeois society,” too consumed with their own comfort and pleasure to lend a hand—or shoulder a gun—to make the world a safer place. “Material possessions,” he concluded, “encourage docility.” {\color{blue} 16 } Through-out the 1990s, the lead item of intellectual complaint, across the political spectrum, was that the United States was insufficiently civic-minded or martial, its leaders and citizens too distracted by prosperity and affluence to take care of its inherited institutions, common concerns, and worldwide defense. Respect for the state was supposed to be dwindling, as was political participation and local volunteerism. {\color{blue} 17 } Indeed, one of the most telling signs of the waning imperative of the Cold War was the fact that the 1990s began and ended with two incidents—the Clarence Thomas–Anita Hill controversy and the Supreme Court decision Bush v. Gore — that cast scandalous suspicion on the nation’s most venerated political institution.{\par} For influential neocons, Clinton’s foreign policy was even more anathema. Not because the neocons were unilateralist arguing against Clinton’s multilateralism, or isolationists or realists critical of his internationalism and humanitarianism. {\color{blue} 18 } Clinton’s foreign policy, they argued, was too driven by the imperatives of free market globalization. It was proof of the oozing decadence taking over the United States after the defeat of the Soviet Union, a sign of weakened moral fiber and lost martial spirit. In an influential manifesto published in 2000, Donald and Frederick Kagan could barely contain their contempt for “the happy international situation that emerged in 1991,” which was “characterized by the spread of democracy, free trade, and peace” and which was “so congenial to America” with its love of “domestic comfort.” According to Kaplan, “the problem with bourgeois societies” like our own “is a lack of imagination.” The soccer mom, for instance, so insistently championed by Republicans and Democrats alike, does not care about the world outside her narrow confines. “Peace,” he complained, “is pleasurable, and pleasure is about momentary satisfaction.” It can be obtained “only through a form of tyranny, however subtle and mild.” It erases the memory of bracing conflict, robust disagreement, the luxury of defining ourselves “by virtue of whom we were up against.” {\color{blue} 19 } {\par} Though conservatives are often reputed to favor wealth and prosperity, law and order, stability and routine—all the comforts of bourgeois life—Clinton’s conservative critics hated him for his pursuit of these very virtues. Clinton’s free-market obsessions betrayed an unwillingness to embrace the murky world of power and violent conflict, of tragedy and rupture. His foreign policy was not just unrealistic; it was insufficiently dark and brooding. “The striking thing about the 1990s zeitgeist,” complained Brooks, “was the presumption of harmony. The era was shaped by the idea that there were no fundamental conflicts anymore.” Conservatives thrive on a world filled with mysterious evil and unfathomable hatreds, where good is always on the defensive and time is a precious commodity in the cosmic race against corruption and decline. Coping with such a world requires pagan courage and an almost barbaric virtue, qualities conservatives embrace over the more prosaic goods of peace and prosperity. It is no accident that Paul Wolfowitz, the darkest of these dark princes of pessimism, was a student of Allan Bloom (in fact, Horowitz makes a cameo appearance in Ravenstein, Saul Bellow’s novel about Bloom). For Bloom—like many other influential neoconservatives—was a fol-lower of Leo Strauss, whose quiet odes to classical virtue and ordered harmony veiled his Nietzsche an vision of torturous con- fl ICT and violent struggle. {\color{blue} 20 } {\par} But there was another reason for the neocons’ dissatisfaction with Clinton’s foreign policy. Many of them found it insufficiently visionary and consistent. Clinton, they claimed, was reactive and ad hoc, rather than proactive and forceful. He and his advisers were unwilling to imagine a world where the United States shaped,Rather than responded to, events. Breaking again with the usual stereotype of conservatives as nonideological pragmatists, figures like Horowitz, Libby, Kaplan, Perl, Frank Gaff Na, Kenneth Adelman, and the father-and-son teams of Kagan and Bristol called for a more ideologically coherent projection of U.S. power, where the “benign hegemony” of American might would spread “the zone of democracy” rather than just extend the free market. They wanted a foreign policy that was, in words that Robert Kagan would later use to praise Senator Joseph Lieberman, “idealistic but not naïve, ready and willing to use force and committed to a strong military, but also committed to using American power to spread democracy and do some good in the world.” As early as the first Bush administration, the neocons were insisting that the United States ought, in Cheney’s words, “to shape the future, to determine the outcome of history,” or, as the Kagan's would later put it, “to intervene decisively in every critical region” of the world, “whether a visible threat exists there.” They criticized those Republicans, in Robert Kagan’s words, who “during the dumb decade of the 1990s” suffered from a “hostility to ‘nation-building,’ the aversion to ‘international social work’ and the narrow belief that ‘superpowers don’t do windows.’” {\color{blue} 21 } What these conservatives longed for was an America that was genuinely imperial—not just because they believed it would make the United States safer or richer, and not just because they thought it would make the world better, but because they literally wanted to see the United States make the world.{\par} At the most obvious level, 9/11 confirmed what the conservatives had been saying for years: the world is a dangerous place, filled with hostile forces who will stop at nothing to see the United States felled. More important, 9/11 gave conservatives an opportunity to articulate, without embarrassment, the vision of imperial American power they had been quietly nourishing for decades. “People are now coming out of the closet on the word empire,” Charles Kraut-hammer accurately observed soon after 9/11. Unlike empires past, this one would be guided by a benign, even benefit cent vision of worldwide improvement. Because of America’s sense of fair play and benevolent purpose—unlike Britain or Rome, the United States had no intention of occupying or seizing territory of its own—this new empire would not generate the backlash that all previous empires had generated. As a Wall Street Journal writer said, “we are an attractive empire, the one everyone wants to join.” In the words of Rice, “Theoretically, the realists would predict that when you have a great power like the United States it would not be long before you had other great powers rising to challenge it. And I think what you’re seeing is that there’s at least a predilection this time to move to productive and cooperative relations with the United States, rather than to try to balance the United States.” {\color{blue} 22 } In creating an empire, the United States would no longer have to respond to immediate threats, to “wait upon events while dangers gather,” as President Bush put it in his 2002 State of the Union address. It would now “shape the environment,” anticipate threats, thinking not in months or years, but in decades, perhaps centuries. The goals were what Cheney, acting on the advice of Horowitz, first outlined in the early 1990s: to ensure that no other power ever arose to challenge the United States and that no regional powers ever attained preeminence in their local theaters. The emphasis was on the preemptive and predictive, to think in terms of becoming, rather than in terms of being. As Richard Perle put it, vis-à-vis Iraq: “What is essential here is not to look at the opposition to Saddam as it is today, without any external support, without any realistic hope of removing that awful regime, but to look at what could be created.” {\color{blue} 23 } For conservatives, the two years after 9/11 were a heady time, a moment when their simultaneous commitment and hostility to the free market could finally be satisfied. No longer hamstrung by the numbing politics of affluence and prosperity, they believed they could count on the public to respond to the call of sacrifice and destiny, confrontation and evil. With “danger” and “security” the watchwords of the day, the American state would be newly sanctified—without opening the floodgates to economic redistribution. 9/11 and the American empire they hoped, would at last resolve the cultural contradictions of capitalism that Daniel Bell had noticed long ago but which had only truly come to the fore after the defeat of Communism.{\par} What a difference a decade makes—or for that matter even a couple of years. Long before the United States would essentially have to declare victory in Iraq and (kind of) go home, long before George W. Bush left his off ice in disgrace, long before the war in Afghanistan proved to be far more than the American people could stomach, it was clear that the neocon imperium rested upon a shaky foundation. In late October and early November 2001, for example, after mere weeks of bombing had failed to dislodge the Taliban, critics started murmuring their fears that the war in Afghanistan would be a reprise of the Vietnam quagmire. {\color{blue} 24 } As soon as the war in Iraq seemed to be not quite the cakewalk its defenders had proclaimed it would be, Democrats began to probe, however tentatively, the edges of acceptable criticism. As early as the 2004 presidential campaign, voicing criticism of the war became something of a litmus test among the Democratic candidates.{\par} None of these critics, of course, would challenge the full-throttle military premise of Bush’s policies—and even under Obama, few would question the basic premises of America’s global reach—but periodic appearance of such critics, particularly in times of trouble or defeat, suggests that the imperial vision is politically viable only so long as it is successful. This is as it must be: because the centerpiece of the imperial promise is that the United Stales can govern events,That it can determine the outcome of history, the promise stands or falls on success or failure. With any suggestion that events lie beyond the empire’s control, the imperial vision blurs. Indeed, it only took a week in March 2002 of horrific bloodshed in Israel and the Occupied Territories—and the resulting accusations that “Bush fiddles in the White House or Texas, playing Nero as the Mideast burns”—for the planned empire to be called into question. No sooner had violence in the Middle East begun to escalate then even the administration’s defenders began jumping ship, suggesting that any invasion of Iraq would have to be postponed indefinitely. As one of Reagan’s high-level national security aides put it, “The supreme irony is that the greatest power the world has ever known has proven incapable of managing a regional crisis.” The fact, this aide added, that the administration had been so maniacally “focused on either Afghanistan or Iraq”—the two key outposts of imperial confrontation— while the Middle East was going up in flames, “reflects either appalling arrogance or ignorance.” {\color{blue} 25 } {\par} Ironically, insofar as the Bush administration avoided those conflicts, such as that between the Israelis and Palestinians, where it might fail—and, indeed, as of this writing, the Obama administration seems to be following the same path with regard to Israel and Palestine—it was forced to forgo the very logic of imperialism that it sought to avow. Premised as it was on the ability of the United States to control events, the neocon imperial vision could not accommodate failure. But by avoiding failure, the imperialists were forced to acknowledge that they could not control events. As former Secretary of State Lawrence Eagle burger observed of the Israelipalestinian conflict, Bush realized “that simply to insert himself into this mess without any possibility of achieving any success is, in and of itself, dangerous, because it would demonstrate that in fact we don’t have any ability right now to control or affect events” {\color{blue} 26 } —precisely the admission the neocons could not aff ord to make. This Catch-22 was no mere problem of logic or consistency: it betrayed the essential fragility of the imperial position itself.{\par} That fragility also reflected the domestic hollowness of the neocons’ imperial vision. Though the neocons saw and continue to see imperialism as the cultural and political counterpart to the free market, they have never come to terms—even ten years later— with how the conservative opposition to government spending and the commitment to tax cuts renders the United States unlikely to make the necessary investments in nation-building that imperialism requires.{\par} Domestically, there is little evidence to suggest that the political and cultural renewal imagined by most commentators—the revival of the state, the return of shared sacrifice and community, the deepening of moral awareness—ever took place, even in the headiest days of the aftermath of 9/11. Of all the incidents one could cite from that time, two stand out. In March 2002, sixty-two senators, including nineteen Democrats, rejected higher fuel-efficiency standards in the automobile industry, which would have reduced dependence upon Persian Gulf oil. Missouri Republican Christopher Bond felt so unencumbered by the need to pay homage to state institutions in a time of war that he claimed on the Senate floor, “I don’t want to tell a mom in my home state that she should not get an S.U.V. because Congress decided that would be a bad choice.” Even more telling was how vulnerable proponents of higher standards were to these antistatist arguments. John Mc cain, for example, was instantly put on the defensive by the notion that the government would be interfering with people’s private market choices. He was left to argue that “no American will be forced to drive any different automobile,” as if that would have been a dreadful imposition in this new era of wartime sacrifice and solidarity. {\color{blue} 27 } {\par} A few months earlier, Ken Feinberg, head of the September {\color{blue} 11 } Victims’ Compensation Fund, announced that families of victims would receive compensation for their loss based in part on the salary each victim was earning at the time of his or her death. After the attacks on the World Trade Center and the Pentagon, Congress had taken the unprecedented step of assuming national responsibility for restitution to the families of the victims. Though the inspiration for this decision was to forestall expensive lawsuits against the airline industry, many observers took it as a signal of a new spirit in the land: in the face of national tragedy, political leaders were finally breaking with the jungle survivalist of the Reaganclinton years. But even in death, the market—and the inequalities it generates—was the only language America’s leaders knew how to speak. Abandoning the notion of shared sacrifice, Feinberg opted for the actuarial tables to calculate appropriate compensation packages. The family of a single sixty-five-year-old grandmother earning $10,000 a year—perhaps a minimum-wage kitchen worker—would draw $300,000 from the fund, while the family of a thirty-year-old Wall Street trader would get $3,870,064. The men and women killed on September {\color{blue} 11 } were not citizens of a democracy; they were earners, and rewards would be distributed accordingly. Virtually no one—not even the commentators and politicians who denounced the Feinberg calculus for other reasons—criticized this aspect of his decision. {\color{blue} 28 } {\par} Even within and around the military, the ethos of patriotism and shared destiny remained secondary to the ideology of the market. In a little-noticed October 2001 article in the New York Times, military recruiters confessed that they still sought to entice enlistees not with the call of patriotism or duty but with the promise of economic opportunity. As one recruiter put it, “It’s just business as usual. We don’t push the ‘Help our country’ routine.” When the occasional patriot burst into a recruiting off ice and said,“I want to fight,” a recruiter explained, “I’ve got to calm them down. We’re not all about fighting and bombing. We’re about jobs. Furthermore, we’re about education.” {\color{blue} 29 } Recruiters admitted that they continued to target immigrants and people of color, on the assumption that it was these constituencies’ lack of opportunity that drove them to the military. The Pentagon’s publicly acknowledged goal, in fact, was to increase the number of Latinos in the military from {\color{blue} 10 } percent to {\color{blue} 22 } percent. Recruiters even slipped into Mexico, with promises of instant citizenship to poor noncitizens willing to take up arms on behalf of the United States. According to one San Diego recruiter, “It’s more or less common practice that some recruiters go to Tijuana to distribute pamphlets, or in some cases they look for someone to help distribute information on the Mexican side.” {\color{blue} 30 } In December 2002, as the United States prepared to invade Iraq, New York Democratic congressman Charles Rangel decided to confront this issue head-on by proposing a reinstatement of the draft. Noting that immigrants, people of color, and the poor were shouldering a greater percentage of the military burden than their numbers in the population warranted, Rangel argued that the United States should distribute the domestic costs of empire more equitably. If middle-class white kids were forced to shoulder arms, he claimed, the administration and its supporters might think twice before going to war. The bill went nowhere.{\par} The fact that the war never imposed the sort of sacrifices on the population that normally accompany national crusades provoked signifi can't concern among political and cultural elites. “The danger, over the long term,” wrote the Times's R. W. Apple before he died, “is loss of interest. With much of the war to be conducted out of plain sight by commandos, diplomats and intelligence agents, will a nation that has spent decades in easy self-indulgence stay focused?” Not long after he had declared the age of glitz and glitter over, Frank Rich found himself publicly agonizing that “you’d never guess this is a nation at war.” Prior to 9/11, “the administration said we could have it all.” Since 9/11, the administration had been saying much the same thing. A former aide to Lyndon Johnson told the New York Times, “People are going to have get involved in this. So far it’s a government effort, as it should be, but people aren’t engaged.” {\color{blue} 31 } Without consecrating the cause in blood, observers feared, Americans would not have their commitment tested, their resolve deepened. As Doris Kearns Goodwin complained on The Newshour:{\par} {\textbf{\textit{Well, I think the problem is we understand that it’s going to be a long war, but it’s hard for us to participate in that war in a thousand and one ways the way we could in World War II. You could have hundreds of thousands joining the armed forces. They could go to the factories to make sure to get those ships, tanks, and weapons built. They could have victory gardens. Furthermore, they could feel not simply as we’re being told: Go back to your ordinary lives. It’s harder now. We don’t have a draft in the same way we did, although there’s some indication I’d like to believe that that younger generation will want to participate. My own youngest son who just graduated from Harvard this June has joined the military. He wants that three-year commitment. He wants to be part of what this is all about instead of just going to work for a year and going to law school, he wants to be a part of this. And I suspect there will be a lot of others like that as well. But somehow you just keep wishing that the government would challenge us. Maybe we need a Manhattan Project for this antibiotics' vaccine production. We were able to get cargo ships down from {\color{blue} 365 } days in World War II to one day by the middle with that kind of collective enterprise. And I think we need to be mobilized, our spirit, our productivity, much more than we were. {\color{blue} 32 } } } }{\par} In what may have been the strangest spectacle of the entire war, the nation’s leaders wound up scrambling to find things for people to do—not because there was much to be done, but because with-out something to do, the ardor of ordinary Americans would grow cold. Since these tasks were unnecessary, and mandating them would have violated the norms of market ideology, the best the president and his colleagues could come up with was to announce Websites and toll-free numbers where enterprising men and women could find information about helping out the war effort. As Bush declared in North Carolina the day after his 2002 State of the Union address, “If you listened to the speech last night, you know, people were saying, ‘Well, gosh, that’s nice, he called me to action, where do I look?’ Well, here’s where: at usafreedomcorps. Gov. Or you can call this number—it sounds like I’m making a pitch, and I am. This is the right thing to do for America. 1-877-USA- CORPS.” The government couldn’t even count on the citizenry to pay for the phone call. And what were the duties these volunteers were to perform? If they were doctors or health care workers, they could enlist to help out during emergencies. And everyone else? They could serve in neighborhood watch programs to guard against terrorist attacks—in North Carolina. {\color{blue} 33 } {\par} Ever since the end of the Cold War, some might even say Vietnam, there has been a growing disconnect between the culture and ideology of U.S. business elites and that of political warriors like Horowitz and the other neocons. Where the Cold War saw the creation of a semicoherent class of Wise Men who brought together, however jaggedly, the worlds of business and politics—men like Dean Acheson and the Dulles brothers—the Reagan years and beyond have witnessed something altogether different. On the one hand, we have a younger generation of corporate magnates who, though ruthless in their efforts to secure benefits from the state,Have none of the respect or passion for the state of their older counterparts. Certainly willing to take from the public till, they are contemptuous of politics and government. These new CEOs respond to their counterparts in Tokyo, London, and other global cities; so long as the state provides them with what they need and does not interfere unduly with their operations, they leave it to the apparatchiks. {\color{blue} 34 } Asked by Thomas Friedman how often he talks about Iraq, Russia, or foreign wars, one Silicon Valley executive said, “Not more than once a year. We don’t even care about Washington. Money is extracted by Silicon Valley and then wasted by Washington. I want to talk about people who create wealth and jobs. I don’t want to talk about unhealthy and unproductive people. If I don’t care about the wealth destroyers in my own country, why should I care about the wealth destroyers in another country?” {\color{blue} 35 } {\par} On the other hand, we have a new class of political elites who have little contact with the business community, whose primary experiences outside of government have been in academia, journalism, think tanks, or some other part of the culture industry. Men like Horowitz and Brooks, the Kagan's and the Bristol, traffic in ideas and see the world as a landscape of intellectual projection. Unconstrained by even the most interested of interests, they see themselves as free to advance their cause, in the Middle East and elsewhere. Like their corporate counterparts, the neocons view the world as their stage; unlike their corporate counterparts, they are preparing for an altogether more theatrical, otherworldly drama. Their endgame, if they have one, is an apocalyptic confrontation between good and evil, civilization and barbarism—categories of pagan conflict diametrically opposed to the world-without-borders vision of America’s free-trading, globalizing elite.{\par}