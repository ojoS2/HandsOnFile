{\chapter{Potomac Fever} } {\label{Potomac Fever} }{\par}{\textit{	} } {\par}{\par} {\textbf{\textit{	} } } {\par} 
	The year 1948, John Cheever once wrote, was “the year everybody in the United States was worried about homosexuality.” And nobody was more worried than the federal government, rumored to be teeming with gays and lesbians. One might think that Washington’s attentions would have been focused elsewhere—on the Soviet Union, say, or on Communist spies. But in 1950 President Truman’s advisers warned him that “the country is more concerned about the charges of homosexuals in the government than about Communists.” The executive branch responded immediately. That year, the State Department phi red “perverts” at the rate of one a day, more than twice the figure for suspected Communists. Charges of homosexuality ultimately accounted for a quarter to a half of all dismissals in the State Department, the Commerce department and the CIA. Only {\color{blue} 25 } percent of Joseph Mc carthy’s fan letters complained of “red infiltration”; the rest fretted about “sex depravity.” {\color{blue} 1 } {\par} The Lavender Scare, as it’s been called, lasted from 1947 through the 1970s, and thousands lost their jobs. It was an exercise in humiliation—and hilarity. For the men and women charged with rinsing the pink from the Potomac were astonishingly ignorant about their quarry. Senator Clyde Hey, head of the first congressional inquiry into the threat, had to ask an aide: “Can you please tell me, what can two women possibly do?” Senator Margaret Chase Smith asked one Hey committee witness whether there wasn’t a “quick test like an X-ray that discloses these things.” {\color{blue} 2 } {\par} The official justification for the purge was that homosexuals were vulnerable to blackmail and could be turned into Soviet spies. But investigators never found a single instance of this kind of blackmail during the Cold War. The best they could come up with was a dubious case from before World War I, when the Russians allegedly used the homosexuality of Austria’s top spy to force him to work for them. {\color{blue} 3 } {\par} The real justification was even more suspect: gays were social misfits whose pathology made them susceptible to Communist indoctrination. Many conservatives also believed that the Communist Party was a movement of and for libertines, and the Soviet Union a haven of free love and open marriage. Gays, they concluded, couldn’t resist the temptation of freedom from bourgeois constraint. Drawing parallels with the decline of the Roman Empire, Mc carthy regarded homosexuality as a cultural degeneracy that could only weaken the United States. It was, as one tabloid put it, “Stalin’s Atom Bomb.” {\color{blue} 4 } {\par} How could a nation confronting so many foreign threats allow itself to be so distracted? (This is not just a question for historians: throughout the first decade of the twenty-first century, while the united States was supposedly confronting a threat to its very existence, the U.S. military devoted considerable energy to purging its gay and lesbian service members. As of 2009, the military had phi red at least sixty Arabic speakers for being gay. {\color{blue} 5 } One case was uncovered after investigators asked a soldier if he had ever participated in community theater.) With the Soviets in possession of the bomb and Korea on the march, why was Secretary of State Dean Ache-son dispatched to Congress to defend his heterosexuality and that of his “powder puff diplomats”? {\color{blue} 6 } Didn’t he have more important things to do than host rowdy gatherings of politicians and journalists that were{\par} {\textbf{\textit{Reminiscent of “stag parties,” featuring copious amounts of Scotch and bourbon, and smiling women “whose identity remained undisclosed.” As one senator remarked, “It reminded me somewhat of the fraternity rushing season at college.” Dean Acheson tried to appear as “one of the boys,” slapping senators on the back. A journalist reported that “his hair was rumpled, his tie awry. The stiff and precise manner and speech which have antagonized many of us had disappeared. He even seemed to have removed the wax from his mustache.” {\color{blue} 7 } } } }{\par} The Lavender Scare off ERS an instructive parable about that proverbial balance between freedom and security, which so vexes us today. It suggests that not only do we seldom strike the right balance between freedom and security, but that the metaphor of balance may itself be deeply fl awed.{\par} The first problem with the metaphor of balance between freedom and security is its assumption that security is a transparent concept, unsullied by ideology and self-interest. Because security Ben- EFI ts everyone—“the most vital of all interests,” John Stuart Mill called it, which no one can “possibly do without”—it is immune to politics. {\color{blue} 8 } Yet, as Arnold Golfers wrote years ago, security is an “ambiguous symbol,” which “may not have any precise meaning at all.” {\color{blue} 9 } Under the banner of a seemingly neutral, universal value, political elites are allowed, indeed encouraged, to pursue partisan and ideological courses of action they would ordinarily find hard to justify. The actions of the U.S. government during the war on terror bear out this claim. According to two official commissions, one of the reasons U.S. intelligence agencies did not anticipate 9/11 was that turf wars prevented them from sharing information. The “obstacles to information sharing were more bureaucratic than legal” and had little to do “with the constitutional principles of due process, accountability, or checks and balances.” {\color{blue} 10 } But while the government rides roughshod over Constitutional principles, it has done little to remove these bureaucratic obstacles. Even the Department of Homeland Security, which was supposed to unite competing agencies, “is bogged down by bureaucracy” and a “lack of strategic planning,” according to one wire report. {\color{blue} 11 } {\par} In the counterterrorism community, to cite another example, it is widely acknowledged that the preemptive arrest and preventive detention of suspected terrorists frustrates the gathering of intelligence. Yet since 9/11 the United States government consistently has relied on such policies. In the two years following 9/11, federal authorities preemptively rounded up more than 5,000 foreign nationals. As of 2006 not a single one of those individuals stood “convicted of any terrorist crime.” {\color{blue} 12 } {\par} The pattern is clear: measures that would improve security are not taken, while the measures that are taken, either fail to improve security or undermine it. There are several explanations for this paradox, including the blinkered interests of the intelligence bureaucracy. But a key factor is that conservatives view national security through the lens of their ongoing Kulturkampf against the 1960s.{\par} This belief influences Republican policies, as we saw during the Bush years, but it also affects Democrats, who are perennially on the defensive against the charge that they are insufficiently hawkish. Consider the career of John Ashcroft, Bush’s first attorney general, who helped design so many of the Draconian measures of the war on terror. As attorney general in Missouri, Ashcroft nearly got cited for contempt—not usually a good career move in American politics—for fighting the court-ordered desegregation of schools in St. Louis and Kansas City. As a senator, he received an honorary degree from Bob Jones University, which has barred interracial dating, and gave a friendly interview to Southern Partisan, a magazine sympathetic to the Old Confederacy. Like the biblical kings, he had his father anoint his head with oil when he became a governor and then a senator. After his father’s death, he had Clarence Thomas do the honors when Bush appointed him attorney general. Convinced that calico cats were signs of the devil, he reportedly had his team make sure that the International Court at The Hague had none on its premises. {\color{blue} 13 } {\par} Ashcroft’s peculiar notions reflect the broader discontent of his party with the political culture bequeathed to or foisted on the United States in the 1960s and 1970s. During those years, liberals and leftists not only toppled legalized racial and gender hierarchies; they also attempted to rein in the security apparatus. They limited executive power, championed an activist judiciary, increased the rights of dissenters and criminals, and separated law enforcement from intelligence gathering. Though these reforms proved short-lived—they were significantly undermined by Reagan and Clinton—the legal legacy of the 1960s has come to stand for the larger culture of freedom that conservatives have loathed and liberals have loved for years.{\par} Conservatives like to eschew any talk of terrorism’s “root causes,” but when it comes to the decadent liberalism that has allegedly hampered the government’s ability to fight evildoers at home and abroad, they are willing to make an exception. Constitutional rights, Ashcroft insisted after 9/11, are “weapons with which to kill Americans.” Terrorists “exploit our openness.” According to Republican Senator Orin Hatch, terrorists “would like nothing more than the opportunity to use all our traditional due process protections to drag out the proceedings.” {\color{blue} 14 } For conservatives, 9/11 was a thunderous judgment on thirty years of treason—as if the attacks on the Pentagon and the World Trade Center were caused not by Alqaeda but by reading criminals their Miranda rights— and a golden opportunity to move in the opposite direction: to expand the power of the presidency at the expense of Congress and the courts, and to blur the lines between intelligence gathering, political surveillance, and law enforcement. {\color{blue} 15 } {\par} This synergy between national security and conservative anxiety is hardly new. The Lavender Scare reflected a general backlash against the loosening of sexual mores and gender roles that resulted from the New Deal and World War II. Roosevelt’s welfare state, conservatives argued, sapped the nation’s energy and patriarchal vigor. Instead of sturdy husbands and firm fathers controlling their wives and children, lisping bureaucrats and female social workers were now running the show. World War II exacerbated the problem: with so many men away at the front, and women working in the factories, male authority was further eroded. Citing these “social and family upheavals,” J. Edgar Hoover argued that “the wartime spirit of abandon and ‘anything goes’ led to a decline of morals among people of all ages.” {\color{blue} 16 } {\par} Washington was the center of this Cultural Revolution. A boom town for young single people in the 1930s and 1940s, it had a tight housing market, forcing men to bunk with men, and off and women plentiful opportunities to support themselves through government jobs. What with the anonymous cruising sites of lafayette Park (right in front of the White House) and the company of tolerant female colleagues in the federal bureaucracy, homosexuals managed to turn Washington into a “very gay city.” Hoover grew up in Washington, when it was a racist backwater of the Old South, and despite his own ambiguous sexuality, he was not happy about these changes. {\color{blue} 17 } {\par} After the war, conservatives stirred a panic about gender roles. “A great emphasis,” according to Cheever, “by way of defense, was put upon manliness, athletics, hunting, fishing and conservative clothing, but the lonely wife wondered, glancing, about her husband at his hunting camp, and the husband wondered with whom he shared a rude bed of pines. Was he? Had he? Did he want to? Had he ever?” In generating that panic, conservatives deftly turned the public against a government bent on making everyone gay. The New Deal, they claimed, was a Queer Deal; America was run by “fairies and Fair Dealers.” {\color{blue} 18 } Because of this ungodly union of Democrats, Communists, and fags, the United States was now vulnerable to the Soviet Union.{\par} Today’s conservatives believe that decades of domestic reform, driven this time by an excessive tenderness about the Constitution, have created a devitalized society that lacks the will and wherewithal to face down foreign threats. That is why Bush promised after 9/11 that there would be “no yielding. No equivocation. No layering this thing to death.” It’s also why Ashcroft bridled at the notion that the U.S. government should read Alqaeda “the Miranda rights, hire a flamboyant defense lawyer, bring them back to the United States to create a new cable net-work of Osama TV.” {\color{blue} 19 } It’s not clear who, if anyone, was recommending such a policy, but that Ashcroft felt compelled to denounce it gives an indication of what he finds at issue when he talks about security. Conservatives certainly believe the Patriot Act and other restrictions of civil liberties will protect the American people—whether it’s terrorism they’re being protected from is another question.{\par} There is a second problem with the notion of a balance between freedom and security. Ever since warfare became the business of peoples rather than kings, the compass of security has steadily expanded beyond the barracks and high command of the military. Frederick II waged war, Lukas wrote, “in such a manner that the civilian population simply would not notice it.” Modern war insinuates itself into “the inner life of a nation.” {\color{blue} 20 } It requires the full mobilization of a country’s resources and active support of its citizenry. Limiting freedom in the most remote parts of society can thus be justified as a legitimate act of national defense. One can find a clear and present danger in the nation’s political economy, its schools and popular culture, even in its beds, and resolve to suppress liberty there in order to avert the threat. When liberals and conservatives affirm the priority of security over freedom in wartime, they are not just endorsing government restrictions on what the press reports about the military; they are also licensing the suppression of all manner of dissent, throughout the entire social order.{\par} Consider the National Security Agency’s (NSA) surveillance of telecommunications traffic in the United States, which was first reported in the New York Times in 2005. As James Risen, who helped break the story, writes, the NSA is “the largest organization in the United States intelligence community, double the size of the CIA and truly the dominant electronic spy service in the world.” Thanks to a secret order issued by Bush in 2002, it “is now eavesdropping on as many as five hundred people in the United States at any given time and potentially has access to the phone calls and emails of millions more. It does this without court-approved search warrants and with little independent oversight.” {\color{blue} 21 } {\par} The Bush administration’s justification for this program, which may be “the largest domestic spying operation since the 1960s,” is that, in order to monitor international traffic between terrorists, it must tap into the domestic network. “The switches carrying calls from Cleveland to Chicago. . . May also be carrying calls from Islamabad to Jakarta,” with the result that “it is now difficult to tell where the domestic telephone system ends and the international network begins.” The administration authorized the NSA to work secretly with telecommunications companies to spy on this inter-national traffic and encouraged them to route more of it through the United States. If they aren’t already, the NSA and its helpers in industry may soon be spying not only on America but on Europe and Asia as well. {\color{blue} 22 } {\par} The expansion of the security domain into all areas of society does more than curtail freedom in the abstract: it also empowers the conservative forces of political repression. Influential conservatives argue that national unity is an essential weapon of war, that opposition undermines the war effort, and that dissenters are dangerous, subversive, or traitorous. After antiwar candidate Ned Lamont’s victory over Joe Lieberman in the 2006 Connecticut Democratic senatorial primary, Vice President Cheney declared that Lamont’s election would only embolden “the alqaeda types,” who were “betting on the proposition that ultimately they can break the will of the American people.” {\color{blue} 23 } {\par} The Patriot Act, passed by Congress six weeks after 9/11, takes the equation of dissent with subversion a step further, suggesting that opponents of the war on terror are not just helping terrorists but may be terrorists themselves. Section {\color{blue} 802 } of the Act defines “domestic terrorism” as “acts dangerous to human life that are a violation of the criminal laws” and that “appear, to be intended. . . To influence the policy of a government by intimidation or coercion.” {\color{blue} 24 } A definition as broad and vague as this could easily be used against demonstrators marching without a permit (a protest might make it impossible for ambulances or other emergency vehicles to get through). {\color{blue} 25 } After antiwar protesters caused a disruption in Port-land, Oregon, in the autumn of 2002, state legislators drafted an antiterrorist bill along these lines. They defined terrorism as, among other things, any act intended “by at least one of its participants” to disrupt “commerce or the transportation systems of the State of Oregon.” {\color{blue} 26 } {\par} During the Republican National Convention in September 2004, the New York City Police Department arrested 1,800 antiwar protesters on various charges, most of which were later thrown out of court. Justifying these arrests, the city’s mayor, Michael Bloomberg, said: “Some people think that we shouldn’t allow people to express themselves. That’s exactly what the terrorists did, if you think about it, on 9/11. Now this is not the same kind of terrorism, but there’s no question that these anarchists are afraid to let people speak out.” {\color{blue} 27 } {\par} Because war mobilizes all spheres of society, defenders of the social order claim that any disruption to that order—from, say, striking labor unions—is as threatening to the war effort as opposition to the war itself. It was on these grounds that in 1950 the Supreme Court upheld the federal government’s denial of labor protection to Communist-led unions. These union leaders, the court argued, might use their positions of power “at a time of external or internal crisis” to call “political strikes” and disrupt the channels of commerce. {\color{blue} 28 } In January 2003, the off ice of Tom De lay, then the House majority leader, sent out a fundraising letter to supporters of the National Right to Work Foundation, a business group seeking to rid America of unions. Claiming that the labor movement “presents a clear-and-present-danger to the security of the United States at home and the safety of our Armed Forces Over-seas,” the letter denounced “Big Labor Bosses. . . Willing to harm freedom-loving workers, the war effort and the economy to acquire more power!” {\color{blue} 29 } {\par} Republicans in Congress also worked closely with Bush to deny union rights and whistle-blower protections to 170,000 employees in the Department of Homeland Security. Even though many of them are clerical workers, and even though employees in the Defense Department are not denied these rights, the administration claimed that eliminating these rights and protections would make the department as “agile and aggressive as the terrorists themselves.” After Congress passed the antiunion bill in November 2002, a White House official declared it to be a model for all federal employees. {\color{blue} 30 } The expansive nature of security authorizes the government not only to deploy these weapons but also to share them with private employers, who are often better positioned to use and abuse them. Because employers aren’t subject to the constraints of the First Amendment, they are generally free to use their powers of hiring and phi ring, promotion and demotion, to silence dissent. During the Mc carthy years, for example, the government imprisoned fewer than two hundred men and women for political reasons. But anywhere between {\color{blue} 20 } and {\color{blue} 40 } percent of the workforce was monitored for signs of ideological nonconformity, which included support for civil rights and labor unions. {\color{blue} 31 } {\par} To eff ects of this outsourcing of repression are particularly visible in the media, for the U.S. media practices a form of censor-ship that must be the envy of tyrants everywhere. Without the government lifting a finger, informal pressure and newsroom careerism are enough to make reporters toe the line. The former CBS news anchor Dan Rather claims that conservatives are “all over your telephones, all over your email.” As a result, “you say to yourself: ‘You know, I think we’re right on this story. I think we’ve got it in the right context, I think we’ve got it in the right perspective, but we better pick another day.’” {\color{blue} 32 } Those at the bottom get the message fast. The television reporter Sam Donaldson, who covered the White House during the Reagan years, tells Eric Buehler:{\par} {\textbf{\textit{Today, not all the bosses support their reporters. So if you’re a reporter at the White House, and you’re thinking about further successes in the business, and you’re nervous about your boss getting a call, maybe you pull your punches because of the career track. {\color{blue} 33 } } } }{\par} Journalists afraid for their careers aren’t likely to question their government in time of war. And they haven’t. ABC’s Ted Koppel, one of the most aggressive interviewers in the business, admits that “we were too timid before the war” in Iraq. The PBS anchor Jim Leader says: “It would have been difficult to have had debates [about occupying Iraq]. . . You’d have had to have gone against the grain.” The few journalists who bucked the trend were swiftly punished. After criticizing the media for its coverage of the war, Ashleigh Banff ELD was “taken to the woodshed” by her bosses, according to a Noonday report, and her career at NBC was finished. A Wall Street Journal reporter sent a personal email describing the terrible situation in Iraq: her editors pulled her out of the country and off the story. {\color{blue} 34 } {\par} The last problem with the notion of a balance between freedom and security is that it mistakenly assumes that the benefits and bur-dens of freedom and security will be distributed equally among all members of society. But it is always some members of society, often the most marginalized and despised—gays and leftists during the Cold War, Arabs and Muslims (and still gays and leftists, albeit to a lesser degree) today—who are forced to give up their freedoms so that the rest can enjoy their security. Indeed, it is precisely because these groups are powerless, and not because they are d angerous, that the powerful can require them to bear the cost. (Even though {\color{blue} 2 } percent of American men aged {\color{blue} 18 } to {\color{blue} 21 } are arrested for drunk driving, the Supreme Court has ruled that this fact does not justify denying men of that age the right to buy alcohol. Many fewer than {\color{blue} 2 } percent of Arabs and Muslims in the United States are engaged in terrorist activity but the U.S. government has denied these groups far more fundamental rights.) {\color{blue} 35 } What the metaphor of balance between freedom and security conceals is the fundamental imbalance of power between groups in society; unequal costs are paid in return for unequal gains.{\par} In No Equal Justice (1999), David Cole turned a commonplace— that white and/or wealthy Americans get better treatment from the cops and courts than black and/or poor citizens—into a startling theorization of a dual justice system in America. Granting maximal rights to all citizens would have a high cost in terms of safety, he observed, while denying those rights would have a high cost in terms of freedom. So what does America do? It does both: it formally grants rights to all, but systematically denies them to blacks and the poor. White, wealthy America gets maximal freedom and maximal safety, and “sidesteps the difficult question of how much constitutional protection we could afford if we were willing to ensure that it was enjoyed equally by all people.” {\color{blue} 36 } {\par} In Enemy Aliens and Terrorism and the Constitution, Cole extends this argument to noncitizens in wartime. Ever since the Alien Act of 1798 America’s first impulse when faced with a foreign threat has been to restrict the rights of immigrants. The attraction of such measures is similar to the attraction of the dual system of criminal justice. It is a “politically tempting way to mediate the tension between liberty and security. Citizens need not forgo their rights” in order to be—or to feel—protected. Noncitizens forgo theirs, and because they “have no direct voice in the democratic process by which to register their objections,” few people complain. {\color{blue} 37 } {\par} After 9/11, security measures that would have affected all citizens—such as Operation TIPS, in which utility employees, delivery men, and other individuals were to spy on their fellow citizens, or the Pentagon’s Total Information Awareness program, a massive surveillance project of public and private computer records—were quickly blocked, even by leading Republicans. But measures affect- ING noncitizens, particularly Muslims and Arabs, received over-whelming public support. Perhaps that is why a year after 9/11, only {\color{blue} 7 } percent of Americans believed themselves to have sacrificed basic rights and liberties. {\color{blue} 38 } {\par} But there is one difference between the treatment of aliens in wartime and the treatment of blacks and the poor in peacetime. Wartime measures inflicted on noncitizens eventually influence measures against U.S. citizens, especially liberals or progressives. In 1942, the federal government put Japanese noncitizens and Japanese Americans in internment camps (on the assumption that even if they were citizens, their racial heritage made them aliens). Several years later, the FBI compiled a secret list of 12,000 citizens to be detained in the event of a national emergency—an initiative rate- phi ed in 1950 by the passage of the Internal Security Act, which remained on the books until 1971. {\color{blue} 39 } Whether a similar mutation will occur in the war on terror is anyone’s guess, but the evidence so far is not encouraging.{\par} What a fuller analysis of the metaphor reveals is that the items being balanced on the scale are not freedom and security but power and powerlessness. It thus makes perfect sense for conservatives to use the metaphor, for it conceals and protects their natural constituency. The real question is: why do liberals oblige them?Perhaps it is because it was liberals who invented the argument. It was liberals who first argued that individuals should be free to say and do whatever they wish, as long as they don’t harm anyone else. Liberal democracies should use coercion only to punish acts or attempted acts of harm, including threats to the security of the nation. One can see variants of this argument in Locke’s account of religious toleration, which could be sacrificed only for “the safety and security of the commonwealth”; Mill’s theory of liberty, which could be limited only to avert harm; and Oliver Wendell Holmes’s defense of freedom of speech, which could be abridged only to thwart “a clear and present danger.” {\color{blue} 40 } {\par} The problem with these arguments is that it is nearly impossible to defi né harm—or danger, threat, menace—in a neutral way. Every definition of harm and its national security cognates rests on ideological assumptions about human nature, morality, and the good life. And in this regard, liberals are as guilty as conservatives. The only difference is that they often have less power to act on their convictions—and to stop their opponents from acting on theirs.{\par} As a philosophical footnote to the Lavender Scare, we might recall that at the very moment the United States was conducting its purge of gays and lesbians, two Englishmen—conservative jurist Patrick Devlin and liberal philosopher H. L. A. Hart—were engaged in a debate of surprising relevance to events across the water. It began in 1957, when the Offended Committee in the United Kingdom recommended, among other things, that gay sex between consenting adults in private be decriminalized. Speaking at the British Academy in March 1959, Devlin bridled at the committee’s contention that there is “a realm of private morality and immorality which is, in brief and crude terms, not the law’s business” and that only concrete acts of injury or harm should be prosecuted and punished by law. Not so, said Devlin: “What makes a society of any sort is community of ideas, not only political ideas but also ideas about the way its members should behave and govern their lives.” Any challenge to those ideas—no matter how private, incidental, or symbolic—undermined social cohesion and posed as great a threat to the civic order as treason. In the same way that treason could lead to the overthrow of a government, homosexuality could produce a “loosening of moral bonds,” which “is often the first stage of disintegration.” Thus, “the suppression of vice is as much the law’s business as the suppression of subversive activities.” {\color{blue} 41 } {\par} Hart’s response was fast—he took to the airwaves in July, delivering a lecture on BBC Radio that was subsequently published in The Listener —and furious. {\color{blue} 42 } “It is grotesque,” he declared, “to think of the homosexual behavior of two adults in private as in any way like treason or sedition.” Not just grotesque but obtuse: Devlin mistakenly assumed “that deviation from a general moral code is bound to affect that code, and to lead not merely to its modify cation but to its destruction.” If one man’s private acts did alter a society’s beliefs—a big if, Hart insisted—such a shift would constitute not a collapse but a transformation of social morality. The proper political analogue to gay sex, then, was not treason but “a peaceful change” in a form of government. {\color{blue} 43 } {\par} Critics tend to think that Hart got the better of Devlin. But I wonder. Hart, after all, never defined harm with any precision or persuasiveness, and it’s not clear that he could have. So what was to stop Devlin from claiming that homosexuality was as harmful as treason—or, as his American counterparts claimed, that homosexuality was treason? Very little, it seems, either politically or philosophically. For when harm comes in shades of gray, someone, somewhere, will inevitably see it in lavender and pink—or any other disfavored color of the rainbow.{\par}