\chapter{Garbage and Gravitas}\label{Garbage and Gravitas}
 \par 
Saint Petersburg in revolt gave us Vladimir Nabokov, Isaiah Berlin, and Ayn Rand. The first was a novelist, the second a philosopher. The third was neither but thought she was both. Many other people have thought so too. In 1998 readers responding to a Modern Library poll identified Atlas Shrugg ed and The Fountainhead as the two greatest novels in English of the twentieth century— surpassing Ulysses, To the Lighthouse, and Invisible Man. In 1991 a survey by the Library of Congress and the Book-of-the-Month Club found that except the Bible, no book has influenced more American readers than Atlas Shrugg ed.{\color{blue}1}
 \par 
One of those readers might well have been Farrah Fawcett. Not long before she died, the actress called Rand a “literary genius” whose refusal to make her art “like everyone else’s” inspired Fawcett’s own experiments in painting and sculpture. The admiration, it seems, was mutual. Rand watched Charlie’s Angels each week and, according to Fawcett, “saw something” in the show “that the critics didn’t.”
 \par 
She described the show as a “triumph of concept and casting.” Ayn said that while Angels was uniquely American, it was also the exception to American television in that it was the only show to capture true “romanticism”—it intentionally depicted the world not as it was, but as it should be. Aaron Spelling was probably the only other person to see Angels that way, although he referred to it as “comfort television.”
 \par 
So taken was Rand with Fawcett that she hoped the actress (or if not her, Raquel Welch) would play the part of Dagny Taggart in a TV version of Atlas Shrugg ed on NBC. Unfortunately, network head Fred Silverman killed the project in 1978. “I’ll always think of ‘Dagny Taggart’ as the best role I was supposed to play but never did,” Fawcett said.{\color{blue}2}
 \par 
Rand’s following in Hollywood has always been strong. Barbara Stanwyck and Veronica Lake fought to play the part of Dominique Francon in the movie version of The Fountainhead. Never to be out-done in that department, Joan Crawford threw a dinner party for Rand in which she dressed as Francon, wearing a streaming white gown dotted with aquamarine gemstones. {\color{blue}3} More recently, the author of The Virtue of Selfi shness and the statement “if civilization is to survive, it is the altruist morality that men have to reject” has found an unlikely pair of fans in the Hollywood humanitarian set. {\color{blue}4} Rand “has a very interesting philosophy,” says Angelina Jolie. “You re-evaluate your own life and what’s important to you.” The Fountainhead “is so dense and complex,” marvels Brad Pitt, “it would have to be a six-hour movie.” (The 1949 film version has a running time of {\color{blue}113} minutes, and it feels long.) Christina Ricci claims that The Fountainhead is her favorite book because it taught her that “you’re not a bad per-son if you don’t love everyone.” Rob Lowe boasts that Atlas Shrugg ed is “a stupendous achievement, and I just adore it.” And any boyfriend of Eva Mendes, the actress says, “has to be an Ayn Rand fan.”{\color{blue}5}
 \par 
But Rand, at least according to her fiction, shouldn’t have attracted any fans at all. The central plot device of her novels is the conflict between the creative individual and the hostile mass. The greater the individual’s achievement, the greater the mass’s resistance. As Howard Roark, architect hero of The Fountainhead, puts it:
 \par 
The great creators—the thinkers, the artists, the scientists, the inventors—stood alone against the men of their time. Every great new thought was opposed. Every great new invention was denounced. The first motor was considered foolish. The airplane was considered impossible. The power loom was considered vicious. Anesthesia was considered sinful. But the men of borrowed vision went ahead. They fought, they suffered and they paid.{\color{blue}6}
 \par 
Rand clearly thought of herself as one of these creators. In an inter-view with Mike Wallace she declared herself “the most creative thinker alive.” That was in 1957, when Arendt, Quine, Sartre, Camus, Lukács, Adorno, Murdoch, Heidegger, Beauvoir, Rawls, Anscombe, and Popper were all at work. It was also the year of the first performance of Endgame and the publication of Pnin, Doctor Zhivago, and The Cat in the Hat. Two years later, Rand told Wallace that “the only philosopher who ever influenced me” was Aristotle. Otherwise, everything came “out of my own mind.” She boasted to her friends and to her publisher at Random House, Bennett Cerf, that she was “challenging the cultural tradition of two and a half thousand years.” She saw herself as she saw Roark, who said, “I inherit nothing. I stand at the end of no tradition. I may, perhaps, stand at the beginning of one.” Yet tens of thousands of fans were already standing with her. In 1945, just two years after its publication, The Fountainhead sold 100,000 copies. In 1957, the year Atlas Shrugg ed was published, it sat on the New York Times bestseller list for twenty-one weeks.{\color{blue}7}
 \par 
Rand may have been uneasy about the challenge her popularity posed to her worldview, for she spent much of her later life spinning tales about the chilly response she and her work had received. She falsely claimed that twelve publishers rejected The Fountainhead before it found a home. She styled herself the victim of a terrible but necessary isolation, claiming that “all achievement and progress has been accomplished, not just by men of ability and certainly not by groups of men, but by a struggle between man and mob.” But how many lonely writers emerge from their study, having just written “The End” on the last page of their novel, to be greeted by a chorus of congratulations from a waiting circle of fans? {\color{blue}8}
 \par 
Had she been a more careful reader of her work, Rand might have seen this irony coming. However much she liked to pit the genius against the mass, her fiction always betrayed a secret communion between the two. Each of her two most famous novels gives its estranged hero an opportunity to defend himself in a lengthy speech before the untutored and the unlettered. Roark declaims before a jury of “the hardest faces” that includes “a truck driver, a bricklayer, an electrician, a gardener and three factory workers.” John Galt takes to the airwaves in Atlas Shrugg ed, addressing millions of listeners for hours on end. In each instance, the hero is understood, his genius acclaimed, his alienation resolved. And that’s because, as Galt explains, there are “no conflicts of interest among rational men”—which is just a Randian way of saying that every story has a happy ending.{\color{blue}9}
 \par 
The chief conflict in Rand’s novels, then, is not between the individual and the masses. It is between the demigod-creator and all those unproductive elements of society—the intellectuals, bureaucrats, and middlemen—that stand between him and the masses. Aesthetically, this makes for kitsch; politically, it bends toward fascism. Admittedly, the argument that there is a connection between fascism and kitsch has taken a beating over the years.
 \par 
Yet surely the example of Rand is suggestive enough to put the question of that connection back on the table.
 \par 
She was born on February 2, three weeks after the failed revolution of 1905. Her parents were Jewish. They lived in Saint Petersburg, a city long governed by hatred of the Jews. By 1914, its register of antisemitic restrictions ran to nearly 1,000 pages, including one statute limiting Jews to no more than {\color{blue}2} percent of the population. They named her Alissa Zinovievna Rosenbaum.{\color{blue}10}
 \par 
When she was four or five years old she asked her mother if she could have a blouse like the one her cousins wore. Her mother said no. She asked for a cup of tea like the one being served to the grown-ups. Again her mother said no. She wondered why she couldn’t have what she wanted. Someday, she vowed, she would. In later life, Rand would make much of this experience. Her biographer does too: “The elaborate and controversial philosophical system she went on to create in her forties and fifties was, at its heart, an answer to this question.”{\color{blue}11}
 \par 
The story, as told, is pure Rand. There’s the focus on a single incident as portent or precipitant of dramatic fate. There’s the elevation of a childhood commonplace to grand philosophy. What child, after all, hasn’t bridled at being denied what she wants? Though Rand seems to have taken youthful selfishness to its outermost limits—as a child she disliked Robin Hood; as a teenager she watched her family nearly starve while she treated herself to the theater—her solipsism was neither so rare nor so precious as to war-rant more than the usual amount of adolescent self-absorption. {\color{blue}12} There is, finally, the inadvertent revelation that one’s worldview constitutes little more than a case of arrested development. “It is not that chewing gum undermines metaphysics,” Max Horkheimer once wrote about mass culture, “but that it is metaphysics—this is what must be made clear.” {\color{blue}13} Rand made it very, very clear.
 \par 
But the anecdote suggests something additionally distinctive about Rand. Not her opinions or tastes, which were middlebrow and conventional. Rand claimed Victor Hugo as her primary inspiration in matters of fiction; Edmond Rostand’s Cyrano de Bergerac was another touchstone. She deemed Rachmaninoff superior to Bach, Mozart, and Beethoven. She was off ended by a reviewer’s admittedly foolish comparison of The Fountainhead to The Magic Mountain. Mann, Rand thought, was the inferior author, as was Solzhenitsyn. {\color{blue}14} Nor was it her sense of self that set Rand apart from others. True, she tended toward the cartoonish and the grandiose. She told Nathaniel Branden, her much younger lover and disciple of many years, that he should desire her even if she were eighty and in a wheelchair. Her essays often quote Galt’s speeches as if the character were a real person, a philosopher on the order of Plato or Kant. She claimed to have created herself with the help of no one, even though she was the lifelong beneficiary of social democratic largesse. She got a college education thanks to the Russian Revolution, which opened universities to women and Jews and, once the Bolsheviks had seized power, made tuition-free. Subsidizing theater for the masses, the Bolsheviks also made it possible for Rand to see cheesy operettas on a weekly basis. After Rand’s first play closed in New York City in April 1936, the Works Progress Administration took it on the road to theaters across the country, giving Rand a handsome income of $10 a performance throughout the late 1930s. Librarians at the New York Public Library assisted her with the research for The Fountainhead. {\color{blue}15} Still, her narcissism was probably no greater—and certainly no less sustaining—than that of your run-of-the-mill struggling author.
 \par 
No, what truly distinguished Rand was her ability to translate her sense of self into reality, to will her imagined identity into material fact. Not by being great, but by persuading others, even shrewd biographers, that she was great. Anne Heller, for example, author of Ayn
 \par 
Rand and the World She Made, repeatedly praises Rand’s “original, razor sharp mind” and “lightning-quick logic,” making one wonder if she’s read any of Rand’s work. She claims that Rand was able “to write more persuasively from a male point of view than any female writer since George Eliot.” {\color{blue}16} Does Heller really believe that Roark or Galt is more credible or persuasive than Lawrence Selden or New-land Archer? Or little James Ramsay, who seems to have acquired more psychic depth in his six years than any of Rand’s protagonists, male or female, demonstrate throughout their entire lives? Jennifer Burns, an intellectual historian and author of Goddess of the Market: Ayn Rand and the American Right, writes that Rand was “among the first to identify the modern state’s often terrifying power and to make it an issue of popular concern,” which is true only if one sets aside Montesquieu, Godwin, Constant, Tocqueville, Proudhon, Bakunin, Spencer, Kropotkin, Malatesta, and Emma Goldman. She claims that Rand disliked the “messiness of the Bohemian student protestors” of the sixties because she was “raised in the high European tradition.” But what kind of high European tradition includes operettas and Rachmaninoff, melodrama and movies? She concludes that “what remains” of enduring value in Rand is her injunction to “be true to yourself.” Yet it hardly took Rand to teach us that; indeed, the very same notion figures in a play about a Danish prince written roughly five centuries before Rand’s birth.{\color{blue}17}
 \par 
To understand how Alissa Rosenbaum created Ayn Rand, we need to trace her itinerary not to prerevolutionary Russia, which is the mistaken conceit of her biographers, but to her destination upon leaving Soviet Russia in 1926: Hollywood. For where else but in the dream factory could Rand have learned how to make dreams—about America, capitalism, and herself ?
 \par 
Even before she was in Hollywood, Rand was of Hollywood. In 1925 alone, she saw {\color{blue}117} movies. It was in movies, Burns says, that Rand “glimpsed America”—and, we might add, developed her
 \par 
Enduring sense of narrative form. Once there, she became the subject of her very own Hollywood story. She was discovered by Cecil B. DeMille, who saw her mooning about his studio looking for work. Intrigued by her intense gaze, he gave her a ride in his car and a job as an extra, which she quickly turned into a screenwriting gig. Within a few years her scripts were attracting attention from major players, prompting one newspaper to run a story with the headline“ Russian Girl Finds End of Rainbow in Hollywood.” {\color{blue}18} Rand, of course, was not the only European who came to Hollywood during the interwar years. But unlike Fritz Lang, Hanns Eisler, and all the other exiles in paradise, Rand did not escape to Hollywood; she went there willingly, eagerly. Billy Wilder arrived and shrugged his shoulders; Rand came on bended knee. Her mission was to learn, not refi né or improve, the art of the dream factory: how to turn a good yarn into a suspenseful plot, an ordinary person into an outsize hero (or villain)—all the tricks of melodramatic narrative designed to persuade millions of viewers that life is really lived at a fever pitch. Most important, she learned how to per-form that alchemy upon herself. Ayan Rand was Norma Desmond in reverse: she was small; it was the pictures that got big.
 \par 
When playing the part of the Philosopher, Rand liked to claim Aristotle as her tutor. “Never have so many”—uncharacteristically, she included herself here—“owed so much to one man.” {\color{blue}19} It’s not clear how much of Aristotle’s work Rand actually read: when she wasn’t quoting Galt, she had a habit of attributing to the Greek philosopher statements and ideas that don’t appear in any of his writings. One alleged Aristotelianism Rand was fond of citing did appear, complete with false attribution, in the autobiography of Albert Jay Nock, an influential libertarian from the New Deal era. In Rand’s copy of Nock’s memoir, Burns observes in an endnote, the passage is marked “with six vertical lines.”{\color{blue}20}
 \par 
Rand also liked to cite Aristotle’s law of identity or noncontradiction—the notion that everything is identical to itself, captured by the shorthand “A is A”—as the basis of her defense of selfishness, the free market, and the limited state. That particular transport sent Rand’s admirers into rapture and drove her critics, even the friendliest, to distraction. Several months before his death in 2002, Harvard philosopher Robert Nozick, the most analytically sophisticated of twentieth-century libertarians, said that “the use that’s made by people in the Randian tradition of this principle of logic. . . Is completely unjustified so far as I can see; it’s illegitimate.” {\color{blue}21} In 1961 Sidney Hook wrote in the New York Times :
 \par 
Since his baptism in medieval times, Aristotle has served many strange purposes. None have been odder than this sacramental alliance, so to speak, of Aristotle with Adam Smith. The extraordinary virtues Miss Rand finds in the law that A is A suggests that she is unaware that logical principles by themselves can test only consistency. They cannot establish truth. . . . Swearing phi - deity to Aristotle, Miss Rand claims to deduce not only matters of fact from logic but, with as little warrant, ethical rules and economic truths as well. As she understands them, the laws of logic license her in proclaiming that “existence exists,” which is very much like saying that the law of gravitation is heavy and the formula of sugar sweet.{\color{blue}22}
 \par 
Whether or not Rand read Aristotle, it’s clear that he made little impression upon her, particularly when it came to ethics. Aristotle had a distinctive approach to morality, quite out of keeping with modern sensibilities; and while Rand had some awareness of its distinctiveness, its substance seems to have been lost on her. Like a set of faux-leather classics on the living room shelf, Aristotle was there to impress the company—and, in Rand’s case, distract from the real business at hand.
 \par 
Unlike Kant, the emblematic modern who claimed that the rightness of our deeds is determined solely by reason, unsullied by need, desire, or interest, Aristotle rooted his ethics in human nature, in the habits and practices, the dispositions and tendencies, that make us happy and enable our flourishing. And where Kant believed that morality consists of austere rules, imposing unconditional duties upon us and requiring our most strenuous sariphi ce, Aristotle located the ethical life in the virtues. These are qualities or states, somewhere between reason and emotion but combining elements of both, that carry and convey us, by the gentlest and subtlest of means, to the outer hills of good conduct. Once there, we are inspired and equipped to scale these lower heights, whence we move onto the higher reaches. A person who acts virtuously develops a nature that wants and is able to act virtuously and that finds happiness in virtue. That coincidence of thought and feeling, reason and desire, is achieved over a lifetime of virtuous deeds. Virtue, in other words, is less a codex of rules, which must be observed in the face of the self's most violent opposition, than it is the food and fiber, the grease and gasoline, of a properly functioning soul.
 \par 
If Kant is an athlete of the moral life, Aristotle is its virtuoso. Rand, by contrast, is a melodramatic of the moral life. Apprenticed in Hollywood rather than Athens, she has little patience for the quiet habituation in the virtues that Aristotelian ethics entails. She returns instead to her favored image of a heroic individual con-fronting a difficult path. Diffi culty is never the result of confusion or ambiguity; Rand loathed “the cult of moral grayness,” insisting that morality is first and always “a code of black and white.” {\color{blue}23} What makes the path treacherous—not for the hero, who seems to have been born fully outfitted for it, but for the rest of us—are the obstacles along the way. Doing the right thing brings hardship, penury, and exile, while doing the wrong thing brings wealth, status, and
 \par 
Acclaim. Because he refuses to submit to architectural conventions, Roark winds up splitting rocks in a quarry. Peter Keating, Roark’s doppelgänger, betrays everyone, including himself, and is the toast of the town. Ultimately, of course, the distribution of rewards and punishments will reverse: Roark is happy, Keating miserable. But ultimately is always and inevitably a long way off.
 \par 
In her essays, Rand seeks to apply to this imagery a superficial Aristotelian gloss. She, too, roots her ethics in human nature and refuses to draw a distinction between self-interest and the good, between ethical conduct and desire or need. But Rand’s metric of good and evil, virtue and vice, is not happiness or flourishing. It is the stern and stark exigencies of life and death. As she writes in “The Objectivist Ethics”:
 \par 
I quote from Galt’s speech: “There is only one fundamental alternative in the universe: existence or nonexistence—and it pertains to a single class of entities: to living organisms. The existence of inanimate matter is unconditional, the existence of life is not: it depends on a specific course of action. Matter is indestructible, it changes its forms, but it cannot cease to exist. It is only a living organism that faces a constant alternative: the issue of life or death. Life is a process of self-sustaining and self-generated action. If an organism fails in that action, it dies; its chemical elements remain, but its life goes out of existence. It is only the concept of ‘Life’ that makes the concept of ‘Value’ possible. It is only to a living entity that things can be good or evil.”{\color{blue}24}
 \par 
Rand’s defenders like to claim that what Rand has in mind by “life” is not simply biological preservation but the good life of Aristotle’s great-fouled man, what Rand characterizes as “the survival of man qua man.” {\color{blue}25} And it’s true that Rand isn’t much taken with mere life
 \par 
Or life for life’s sake. That would be too pedestrian. But Rand’s naturalism is far removed from Aristotle’s. For him life is a fact for her, it is a question, and that very question is what makes life, on its own, such an object and source of reflection.
 \par 
What gives life value is the ever-present possibility that it might (and one day will) end. Rand never speaks of life as a given or ground. It is a conditional, a choice we must make, not once but again and again. Death casts a pall, lending our days an urgency and weight they otherwise would lack. It demands wakefulness, an alertness to the fatefulness of each and every moment. “One must never act like a zombie,” Rand enjoins. {\color{blue}26} Death, in short, makes life dramatic. It makes our choices—not just the big ones but the little ones we make every day, every second—matter. In the Randian universe, it’s high noon all the time. Far from being exhausting or enervating, such an existence, at least to Rand and her characters, is enlivening and exciting.
 \par 
If this idea has any moral resonance, it will be heard not in the writings of Aristotle nor in the superficially similar existentialism of Sartre, but in the drill march of fascism. The notion of life as a struggle against and unto death, of every moment laden with destruction, every choice pregnant with destiny, every action weighed upon by annihilation, its lethal pressure generating moral meaning— these are the watchwords of the European night. In his Berlin Sportpalest speech of February 1943, Goebbels declared, “Whatever serves it and its struggle for existence is good and must be sustained and nurtured. Whatever is injurious to it and its struggle for existence is evil and must be removed and eliminated.” {\color{blue}27} The “it” in question is the German nation, not the Randian individual. But if we strip the pronoun of its antecedent—and listen for the background hum of Sein oder Nichtsein, preservation versus elimination—the similarities between the moral syntax of Randianism and of fascism become clear. Goodness
 \par 
Is measured by life, life is a struggle against death, and only our daily vigilance ensures that one does not prevail over the other.
 \par 
Rand, no doubt, would object to the comparison. There is, after all, a difference between the individual and the collective. Rand thought the former an existential fundament, the latter—whether it took the form of a class, race, or nation—a moral monstrosity. And where Goebbels talked of violence and war, Rand spoke of commerce and trade, production and economy. But fascism is hardly hostile to the heroic individual. That individual, moreover, often finds his deepest calling in economic activity. Far from demonstrating a divergence from fascism, Rand’s economic writings register its presence indelibly.
 \par 
Here is Hitler speaking to a group of industrialists in Düssel-
 \par 
You maintain, gentlemen, that the German economy must be constructed on the basis of private property. Now such a conception of private property can only be maintained in practice if it in some way appears to have a logical foundation. This conception must derive its ethical justification from the insight that this is what nature dictates.{\color{blue}28}
 \par 
Rand, too, believes that capitalism is vulnerable to attack because it lacks “a philosophical base.” If it is to survive, it must be rationally justified. We must “begin at the beginning,” with nature itself. “In order to sustain its life, every living species has to follow a certain course of action required by its nature.” Because reason is man’s “means of survival,” nature dictates that “men prosper or fail, survive or perish in proportion to the degree of their rationality.” (Notice the slippage between success and failure and life and death.) Capitalism is the one system that acknowledges and incorporates this dictate of nature. “It is the basic, metaphysical
 \par 
Fact of man’s nature—the connection between his survival and his use of reason—that capitalism recognizes and protects.” {\color{blue}29} Like Hitler, Rand finds in nature, in man’s struggle for survival, a “logical foundation” for capitalism.
 \par 
Far from privileging the collective over the individual or subsuming the latter under the former, Hitler believed that it was the “strength and power of individual personality” that determined the economic (and cultural) fate of the race and nation. {\color{blue}30} Here he is in 1933 addressing another group of industrialists:
 \par 
Everything positive, good and valuable that has been achieved in the world in the field of economics or culture is solely attributable to the importance of personality. . . . All the worldly goods we possess we owe to the struggle of the select few.{\color{blue}31}
 \par 
The exceptional men, the innovators, the intellectual giants. . . . It is the members of this exceptional minority who lift the whole of a free society to the level of their own achievements, while rising further and ever further.{\color{blue}32}
 \par 
If the first half of Hitler’s economic views celebrates the romantic genius of the individual industrialist, the second spells out the in egalitarian implications of the first. Once we recognize “the outstanding achievements of individuals,” Hitler says in Düsseldorf, we must conclude that “people are not of equal value or of equal importance.” Private property “can be morally and ethically justified only if [we] admit that men’s achievements are different.” An understanding of nature fosters a respect for the heroic individual, which fosters an appreciation of inequality in its most vicious guise. “The creative and decomposing forces in a people always fight against one another.”{\color{blue}33}
 \par 
Rand’s appreciation of inequality is equally pungent. I quote
 \par 
From Galt’s speech:
 \par 
The man at the top of the intellectual pyramid contributes the most to all those below him, but gets nothing except his mate-rial payment, receiving no intellectual bonus from others to add to the value of his time. The man at the bottom who, left to himself, would starve in his hopeless ineptitude, contributes nothing to those above him, but receives the bonus of all their brains. Such is the nature of the “competition” between the strong and the weak of the intellect. Such is the pattern of “exploitation” for which you have damned the strong.{\color{blue}34}
 \par 
Rand’s path from nature to individualism to inequality also ends in a world divided between “the creative and decomposing forces.” In every society, says Roark, there is a “creator” and a parasitic “second-hander,” each with its own nature and code. The first “allows man to survive.” The second is “incapable of survival.” {\color{blue}35} One produces life, the other induces death. In Atlas Shrugg ed the battle is between the producer and the “looters” and “moochers.” It, too, must end in life or death.
 \par 
To find Rand in such company should come as no surprise, for she and the Nazis share a patrimony in the vulgar Nietzscheanism that has stalked the radical right, whether in its libertarian or fascist variants, since the early part of the twentieth century. As both of her biographers show, Nietzsche exerted an early grip on Rand that never really loosened. Her cousin teased Rand that Nietzsche “beat you to all your ideas.” When Rand arrived in the United States, Thus Spake Zarathustra was the first book in English she bought. With Nietzsche on her mind, she was inspired to write in her journals that “the secret of life” is “you must be
 \par 
Nothing but will. Know what you want and do it. Know what you are doing and why you are doing it, every minute of the day. All will and all control. Send everything else to hell!” Her entries frequently include phrases like “Nietzsche and I think” and “as Nietzsche said.”{\color{blue}36}
 \par 
Rand was much taken with the idea of the violent criminal as moral hero, a Nietzschean translator of all values; according to Burns, she “found criminality an irresistible metaphor for individualism.” A literary Leopold and Loeb, she plotted out a novella based on the actual case of a murderer who strangled a twelveyear-old girl. The murderer, said Rand, “is born with a wonderful, free, light consciousness—resulting from the absolute lack of social instinct or herd feeling. He does not understand, because he has no organ for understanding, the necessity, meaning or importance of other people.” {\color{blue}37} That is not a bad description of Nietzsche’s master class in The Genealogy of Morals.
 \par 
Though Rand’s defenders claim she later abandoned her infatuation with Nietzsche, there is too much evidence of its persistence. There’s the figure of Roark himself: “As she jotted down notes on Roark’s personality,” writes Burns, “she told herself, ‘See Nietzsche about laughter.’ The book’s famous first line indicates the centrality of this connection: ‘Howard Roark laughed.’” {\color{blue}38} And then there’s Atlas Shrugg ed, which Ludwig von Mises, one of the pre-siding eminences of neoclassical economics, praised thus:
 \par 
You have the courage to tell the masses what no politician told them: you are inferior and all the improvements in your conditions which you simply take for granted you owe to the effort of men who are better than you.{\color{blue}39}
 \par 
But Nietzsche’s influence saturated Rand’s writing in a deeper way, one emblematic of the overall trajectory of the right since its
 \par 
Birth in the crucible of the French Revolution. Rand was a lifelong atheist with a special animus for Christianity, which she called the “best kindergarten of communism possible.” {\color{blue}40} Far from representing a heretical tendency within conservatism, Rand’s statement channels a tradition of right-wing suspicion about the insidious effects of religion, particularly Christianity, on the modern world. Where many conservatives since 1789 have rallied to Christianity and religion as an antidote to the democratic revolutions of the eighteenth and nineteenth centuries, a not insignifi can't subset among them have seen religion, or at least some aspect of religion, as the adjutant of revolution.
 \par 
Joseph de Maistre was one of the first. An arch-Catholic, he traced the French Revolution to the acrid solvents of the Reformation. With its celebration of “private interpretation” of the Scriptures, Protestantism paved the way for century upon century of regicide and revolt originating in the lower classes.{\color{blue}41}
 \par 
It is from the shadow of a cloister that there emerges one of mankind’s very greatest scourges. Luther appears; Calvin fol-lows him. The Peasants’ Revolt; the Thirty Years’ War; the civil war in France. . . The murders of Henry II, Henry IV, Mary Stuart, and Charles I; and finally, in our day, from the same source, the French Revolution.{\color{blue}42}
 \par 
Nietzsche, the child of a Lutheran pastor, radicalized this argument, painting all of Christianity—indeed all of Western religion, going back to Judaism—as a slave morality, the psychic revolt of the lower orders against their betters. Before there was religion or even morality, there was the sense and sensibility of the master class. The master looked upon his body—its strength and beauty, its demonstrated excellence and reserves of power—and saw and said that it was good. As an afterthought he looked upon the slave,
 \par 
And saw and said that it was bad. The slave never looked upon himself: he was consumed by envy of and resentment toward his master. Too weak to act upon his rage and take revenge, he launched a quiet but lethal revolt of the mind. He called all the master’s attributes—power, indifference to suffering, thoughtless cruelty—evil. He spoke of his own attributes—meekness, humility, forbearance—as good. Furthermore, he devised a religion that made selfishness and self-concern a sin, and compassion and concern for others the path to salvation. Furthermore, he envisioned a universal brotherhood of believers, equal before God, and damned the master’s order of unevenly distributed excellence. {\color{blue}43} The modern residue of that slave revolt, Nietzsche makes clear, is found not in Christianity, or even in religion, but in the nineteenth-century movements for democracy and socialism:
 \par 
Another Christian concept, no less crazy, has passed even more deeply into the tissue of modernity: the concept of the “equality of souls before God.” This concept furnishes the prototype of all theories of equal rights: mankind was first taught to stammer the proposition of equality in a religious context, and only later was it made into morality: no wonder that man ended by taking it seriously, taking it practically!—that is to say, politically, democratically, socialistically.{\color{blue}44}
 \par 
When Rand inveighs against Christianity as the forebear of socialism, when she rails against altruism and sacrifice as inversions of the true hierarchy of values, she is cultivating the strain within conservatism that sees religion as not a remedy to, but a helpmate of, the left. And when she looks, however ineptly, to Aristotle for an alternative morality, she is recapitulating Nietzsche’s journey back to antiquity, where he hoped to find a master-class morality untainted by the egalitarian values of the lower orders.
 \par 
Though Rand’s antireligious defense of capitalism might seem out of place in today’s political firmament, we would do well to recall the recent revival of interest in her books. More than 800,000 copies of her novels were sold in 2008 alone; as Burns rightly notes, “Rand is a more active presence in American culture now than she was during her lifetime.” Indeed, Rand is regularly cited as a formative influence upon an entire new generation of Republican leaders; Burns calls her “the ultimate gateway drug to life on the right.” {\color{blue}45} Whether she is invoked by name, Rand’s presence is palpable in the concern, heard increasingly on the right, that there is something sinister afoot in the institutions and teachings of Christianity.
 \par 
I beg you, look for the words “social justice” or “economic justice” on your church website. If you find it, run as fast as you can. Social justice and economic justice, they are code words. Now, am I advising people to leave their church? Yes.
 \par 
That was Glenn Beck on his March 2, 2010, radio show, taking a stand against, well, pretty much every church in the Christian faith: Catholic, Episcopalian, Methodist, Baptist—even his very own Church of Jesus Christ of Latter-day Saints.{\color{blue}46}
 \par 
On her own, Rand is of little significance. It is only her resonance in American culture—and the unsavory associations her resonance evokes—that makes her of any interest. She’s not unlike the “secondhand” described by Roark: “Their reality is not within them, but somewhere in that space which divides one human body from another. Not an entity, but a relation. . . . The secondhand acts, but the source of his actions is scattered in every other living person.” {\color{blue}47} For once, it seems, he knew whence he spoke.
 \par 
But after all the Nietzsche is said and Aristotle is done, we’re still left with a puzzle about Rand: How could such a mediocrity, not just a secondhand but a second-rater, exert such a continuing influence on the culture at large?
 \par 
We possess an entire literature, from Melville to Mamet, devoted to the con man and the hustler, and it’s tempting to see Rand as one of the many fakes and frauds who periodically light up the American landscape. But that temptation should be resisted. Rand represents something different, more unsettling. The con man is a liar who can ascertain the truth of things, often better than the rest of us. He has to: if he is going to fleece his mark, he has to know who the mark is and who the mark would like to be. Working in that netherworld between fact and fantasy, the con man can gild the lily only if he sees the lily for what it is. But Rand had no desire to gild anything. The gilded lily was reality. What was there to add? She even sported a lapel pin to make the point: made of gold and fashioned in the shape of a dollar sign, it was bling of the most literal sort.
 \par 
Since the nineteenth century, it has been the task of the left to hold up to liberal civilization a mirror of its highest values and to say, “You do not look like this.” You claim to believe in the rights of man, but it is only the rights of property you uphold. You claim to stand for freedom, but it is only the freedom of the strong to dominate the weak. If you wish to live up to your principles, you must give way to their demiurge. Allow the dispossessed to assume power, and the ideal will be made real, the metaphor will be made material.
 \par 
Rand believed that this meeting of heaven and earth could be arranged by other means. Rather than remake the world in the image of paradise, she looked for paradise in an image of the world. Political transformation wasn’t necessary. Transubstantiation was enough. Say a few words, wave your hands and the ideal is real, the metaphor material. An idealist of the most primitive sort, Rand took a century of socialist dichotomies and flattened them. Small wonder
 \par 
So many have accused her of intolerance: When heaven and earth are pressed so closely together, where is there room for dissent?
 \par 
Far from needing explanation, her success explains itself. Rand worked in that quintessential American proving ground—alongside the likes of Richard Nixon, Ronald Reagan, and Glenn Beck—where garbage achieves gravitas and bullshit gets blessed. There she learned that dreams don’t come true. They are true. Turn your metaphysics into chewing gum, and your chewing gum is meta-physics. A is A.