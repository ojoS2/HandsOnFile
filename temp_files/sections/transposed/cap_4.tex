\begin{ figure }
	\centering
	\\includegraphics[width=1.\textwidth]{ temp_files/images/UP_logo.png }
	\caption{Lily Braun (1865-1916): Feminist writer and a politician within the German So- cial Democratic Party. Her 1901 book, The Women’s Question: Historical Devel- opment and Economic Aspect proposed many novel solutions to the challenges faced by working mothers, including proposals for what she called “maternity insurance.” Braun was a moderate and a reformer and did not believe that revo- lution was necessary to achieve socialism. Courtesy of Lebendiges Museum On- line (Deutsches Historisches Museum).}
	\label{ }
\end{ figure }
 \par 
\chapter{WHAT TO EXPECT WHEN YOU'RE EXPECTING EXPLOITATION: ON MOTHERHOOD}\label{WHAT TO EXPECT WHEN YOU'RE EXPECTING EXPLOITATION: ON MOTHERHOOD}
 \par 
 né of my childhood friends, whom I will call Jake, © hungered for financial success in a society where financial success reflected a kind of moral superiority. Jake valorized the idea of the American Dream. He saw goodness in the kind of Horatio Alger, pull-yourself-up-byte-bootstraps hard work required to “make something” of yourself. Back then, I was already a feminist with concerns about economic inequality, while Jake, true to the spirit of the 1980s, believed that whoever dies with the most toys wins. We spent hours debating the pros and cons of capitalism, and the ways that Thatcherism and Reaganomics sucked or didn’t suck. Jake embraced the Gordon Gekko zeitgeist of the age: “Greed is good.” I wasn't buying it. But back in those days when domestic politics weren't so polarized, we managed to maintain our friendship throughout our college years. In the 1990s, while I was off teaching English and reading Karl Polanyi in Japan, Jake was hustling his way up the corporate ladder at a tech start-up.
 \par 
49
 \par 
50
 \par 
WHAT TO EXPECT WHEN YOU'RE EXPECTING EXPLOITATION: ON MOTHERHOOD
 \par 
One day in 1997, Jake informed me with great pleasure that he’d hired a promising young woman for a strategic position in his firm. She’d been a finalist with two other men, and with my voice ringing in his ears, he decided to take a chance on her. “They were all equally qualified on paper,” he told me, “But after years of listening to your feminist rants, I convinced my boss that since women face so many barriers in tech, she had actually worked harder to get where she was than the men in the pool.” I was struggling through my first year of graduate school at the time, and Jake’s news warmed my heart; I’d made a little difference in the world.
 \par 
Over the next few years, the woman proved herself clever, competent, and hard-working. Jake’s company gave her a three-month paid sabbatical for some additional training, grooming her for a promotion. Then she announced she was pregnant. The start-up had no formal maternity leave policy, but Jake asked his boss to give her twelve paid weeks to stay home with her baby and make child care arrangements. Jake argued that they had already invested so much money in her training that a twelve-week leave would pay for itself in the long run. His boss reluctantly agreed.
 \par 
\textit\textbf{ {The woman returned to work after the birth of her baby and tried her best to keep up with the demands of a small start-up. But she was nursing. And the baby kept her up at night. She would attend meetings bleary-eyed and unprepared. She called in sick when the nanny didn’t show. Furthermore, she found a place in a good nursery, but if her son got sick, they sent him home. Her husband traveled for business, and she had no family in the area. Jake, always the optimist, believed things would improve once the child was older. He} }
 \par 
KRISTEN R. GHODSEE even offered to babysit in a pinch. His star employee managed to hold on for six months. Then she quit.
 \par 
That night Jake called me to share the news. Dejected and frustrated, he told me, “I’m never hiring a woman again.
 \par 
“But she’s just one woman,’ I said. “Not every woman is going to make her choice.”
 \par 

 \par 
I think I hung up on him. But it really wasn’t Jake’s fault. What could he do in a system that provides no support for women when they become mothers, that forces women to choose between their careers and their families? Economists call this “statistical discrimination.” The basic idea is that since employers can’t directly observe the productivity of individual workers, they can make observations about demographic characteristics that are correlated with worker productivity. They make decisions based on the averages: if women are more likely to quit than men for personal reasons, employers assume that any given woman is more likely to quit than a man. Economists observe that the theory of statistical discrimination can create a vicious cycle. If women are (or used to be) more likely to quit, they will be paid less. If they are paid less, they are more likely to quit. This vicious cycle provides a very good justification for government intervention.’
 \par 
51
 \par 
52
 \par 
WHAT TO EXPECT WHEN YOU'RE EXPECTING EXPLOITATION: ON MOTHERHOOD
 \par 
The perception of women’s comparative inferiority as workers is linked to their biological capacity for child bearing and nursing, and the concomitant social expectation that women will be the primary caregivers for babies and young children. And in some patriarchal fantasy world, our supposedly innate caring nature also makes us perfectly suited for nursing other sick, weak, or aged relatives. And since women are at home anyway, so the argument goes, we might as well do all the shopping, cooking, cleaning, and emotional labor required to maintain a household, right? Someone has to do it, and that someone is almost always a woman, in part because the location of the tasks align, but also because she has been socialized from infancy to believe that it’s her natural role. Baby dolls, EZ Bake ovens, and toy vacuum cleaners allow girls to play-practice the labors they will perform when they grow up.
 \par 
Employers discriminate against those whose bodies can produce children because society attributes certain characteristics to the owners of those bodies. When scholars talk about men and women, they often make a distinction between the terms “sex” and “gender.” The word “sex” means the biological difference between males and females and the word “gender” connotes the social roles that cultures expect to match the biology. For example, by sex 1am a woman because I have the physiological equipment necessary for baby manufacturing, but my gender is also female because in many ways I conform to contemporary American society’s imagination of what a woman should be: I have long hair; I wear skirts, jewelry, and makeup; I enjoy romantic comedies and nice bath products; and although I might claim it’s for my general health, I do a daily hour on the elliptical
 \par 
KRISTEN R. GHODSEE trainer because I worry about my weight (okay, well, maybe it’s only forty-five minutes, and it’s not every day, but you get the idea). In other ways, however, my gender identity is more masculine: I have always worked full-time and earned my own money; I enjoy watching soccer, science fiction, and action movies; I love a good beer; and although I try to be polite about it, I always speak my mind even if my thoughts and opinions may offend. I suffer no fools, while according to some, real women tolerate gropers, mansplainers, and plain old idiots with a smile.
 \par 
Gender discrimination arises because society constructs archetypes of the ideal man and the ideal woman based on their supposedly natural biological differences. This is not to say that men and women are the same—they are not— but only that our beliefs about how men and women behave are a figment of our collective imaginations—a powerful figment, yes, but a figment nonetheless. When a student ranks a professor with a female name lower than a professor with a male name, the student may assume that the male professor has more time and energy to dedicate to his teaching because he is not distracted by his care obligations outside of work. When employers like my friend Jake’s boss see a woman’s name on a job application, they immediately think that “woman” equals potential mother with priorities in life that take precedence over their careers. Employers also assume that men will put their careers over their families because they are supposedly less biologically attached to children. It doesn’t matter if individual men decide to stay home with their children or if individual women sterilize themselves to overcome the challenges of work/family balance; our gender stereotypes of how men and women
 \par 
53
 \par 
54
 \par 
WHAT TO EXPECT WHEN YOU'RE EXPECTING EXPLOITATION: ON MOTHERHOOD behave are rooted in our ideas about the “natural” link between biological sex and how this informs our life choices.
 \par 
I used to do a classroom exercise with my students to get them to think about the relationship between sex and gender. I borrowed a scenario from Ursula Le Guin’s classic science fiction novel, The Left Hand of Darkness, where a man from earth is sent to work on a planet of “bisexual hermaphrodites.” This means that all people have both male and female sexual organs and hormones. Throughout the month, there are seven-day periods when a portion of the population experiences a form of heat: an irresistible desire to copulate. At the initiation of sexual contact, one of the members of the pair becomes the male, and the other person becomes the female. In any given sexual encounter, an individual will randomly become either the male or the female. The member of the pair who becomes female can become pregnant and will then have a nine-month gestation period before giving birth. When an individual is not copulating or pregnant, they revert to a neutral state until their next sexual encounter, when the process repeats. Any one individual can therefore be both a father and a mother, and everyone is equally “at risk” for pregnancy and childbirth.
 \par 
I asked my students to try to imagine how the society on this fictional planet would be arranged compared to our society in the United States. The first thing to go would be sex discrimination, since everyone would be biologically identical. All people are “hermaphrodites,” so you couldn't use biological sex to create hierarchies. Of course, more attractive “bisexual hermaphrodites” might enjoy more privileges than the ugly ones, and the old might have more power over the young, but discrimination would not
 \par 
KRISTEN R. GHODSEE be based on whether you can make babies. Similarly, the social roles linked to biology would be the same for everyone, since most members of this society would be both mothers and fathers to multiple children. My students also imagined that the society on this fictional planet would be organized to accommodate the demands of pregnancy and childbirth, since every member of that society would benefit from collectively organized forms of support.
 \par 
Socialists have long understood that creating equity between men and women despite their biological sex differences requires collective forms of support for child-rearing. By the mid-nineteenth century, as women flooded into the industrial labor force of Europe, socialists theorized that you could not build strong worker’s movements without the participation of women. The German feminist Lily Braun promoted the idea of a state-funded “maternity insurance” as early as 1897. In this scheme, working women would enjoy paid furloughs from their jobs both before and after delivery, with guarantees that their jobs would be held in their absence. It’s important to remember that as late as 1891, in Germany female industrial workers toiled for a minimum of sixty-five hours per week, even if they were with child. Under these circumstances, pregnant women and girls stayed at the assembly line until they gave birth, and if they had no husband or family to support them, they returned to work soon afterward. The infant and maternal mortality rate for working women was more than double that of middle-class women because of the harsh conditions.
 \par 
Although British and American feminists wanted to
 \par 
55
 \par 
56
 \par 
WHAT TO EXPECT WHEN YOU'RE EXPECTING EXPLOITATION: ON MOTHERHOOD support working mothers through nonstate charities, Braun proposed that funds for the maternity insurance be raised through a progressive income tax. The German government could then pay a woman’s wages for a fixed period before and after the birth of her child. Everyone would contribute to a special pot of money that new mothers could draw on, much like unemployment insurance or a state pension. Braun asserted that since society benefitted from children, it should help bear the costs of raising them. Children are future soldiers, workers, and taxpayers. They are a benefit to all, not just to the parents who bring them into the world (and some parents of teenagers might argue that they are more of a benefit to society than they are to their parents). This is especially true in ethnically homogenous states, where societies place a premium on preserving a particular national identity.’
 \par 
But Braun’s proposal was expensive. It required new taxes and would redistribute wealth to the working classes, an idea that many middle-class men and women opposed. Braun’s ideas also faced initial opposition from the Left. Because Braun was a reformer and believed that her maternity scheme could be implemented under capitalism, more radical German socialists like Clara Zetkin initially rejected her ideas, claiming they could only be realized under a socialist economy. Braun also favored communal living arrangements (communes) over state-funded nurseries and kindergartens, whereas Zetkin believed that housework and child care should be socialized. Nonetheless, Braun’s proposals, in watered down form at least, were passed into law as early as 1899. And by the Second International Conference of Socialist Women in 1910, Braun’s
 \par 
KRISTEN R. GHODSEE ideas were incorporated into the official socialist platform with the support of Clara Zetkin and the Russian Alexandra Kollontai.
 \par 
The fourth point on the 1910 socialist platform laid the foundation for all subsequent socialist policies regarding state responsibilities toward women workers. Under the title “Social Protection and Provision for Motherhood and Infants,” the women of the Second International demanded an eight-hour working day. They proposed that pregnant women stop working (without previous notice) for eight weeks prior to the expected delivery date, and that women be granted a paid “motherhood insurance” of eight weeks if the child lived, which could be extended to thirteen weeks if the mother was willing and able to nurse the infant. Women would get a six-week leave for stillborn children, and all working women would enjoy these benefits, “including agricultural laborers, home workers and maid servants.” These policies would be paid for by the permanent establishment of a special maternity fund out of tax revenues.”
 \par 
Seven years later, Kollontai attempted to implement some of these policies in the Soviet Union after the Bolshevik revolution. Instead of burdening individual women with household chores and child care in addition to their industrial labor, the young Soviet state proposed to build kindergartens, creches, children’s homes, and public cafeterias and laundries. By 1919, the Eighth Congress of the Communist Party handed Kollontai a mandate to expand her work for Soviet women, and she secured state commitments to expend the funds necessary to build a wide network of social services. The year 1919 also saw the creation
 \par 
57
 \par 
58
 \par 
WHAT TO EXPECT WHEN YOU'RE EXPECTING EXPLOITATION: ON MOTHERHOOD of an organization called the Zhenotdel, the Women’s Section, which would oversee the work of implementing the radical program of social reform that would lead to women’s full emancipation.*
 \par 
But Soviet enthusiasm for women’s emancipation soon evaporated in the face of more pressing demographic, economic, and political concerns. After the country was devastated by the brutal years of the First World War, followed by the Civil War and the horrendous famine of 1921 and 1922, Lenin and the Bolsheviks did not have the funds to support Kollontai’s plan. Hundreds of thousands of war orphans roamed the major cities, plaguing residents with petty crime and theft. The state lacked the resources to care for them; children’s homes were overburdened and understaffed. Liberalization of divorce laws meant that fathers abandoned their pregnant wives, and poor enforcement of child support and alimony laws meant that those men who had survived the First World War, the Civil War, and the famine routinely skipped out on their responsibilities. Working women couldn’t look after their children and hoped the state would step in and help, as Kollontai and the other women’s activists had promised. In 1920, the Soviet Union had also become the first country in Europe to legalize abortion on demand during the first twelve weeks of pregnancy. Birthrates plummeted as women sought to limit the size of their families. Eventually there was fear that the falling birthrate combined with the devastation of war and famine would derail the country’s plans for rapid modernization.°
 \par 
No one ever wanted women’s economic independence to come at the cost of motherhood, but this is what
 \par 
KRISTEN R. GHODSEE happened. As the demands on Soviet women’s time increased, they chose to delay or limit childbearing. Eventually, Stalin disbanded the Zhenotdel, declaring that the
 \par 
“Woman question” had been solved. In 1936, he reversed most liberal policies, banned abortion, and reinstated the traditional family, on top of his sustained program of state terror and arbitrary purges. The rapidly industrializing Soviet state needed women to work, have babies, and do all the care work the world’s first socialist state could not yet afford to pay for. Soviet women were far from emancipated, and Alexandra Kollontai spent most of her remaining years in diplomatic exile.
 \par 
While the Soviet experiment failed, Braun’s ideas and the program of the socialist women in 1910 found fertile soil in the Scandinavian social democracies. The Danes introduced a two-week leave for working women as early as 1901, and by 1960 a universal, state-funded paid maternity leave was extended to all working women. In 1919, Finland passed maternity leave provisions for factory workers and professional women, and added job protections in 1922. Sweden introduced an unpaid maternity leave of four weeks as early as 1901, and by 1963, the government guaranteed women {\color{blue}180} days of job-protected maternity leave at {\color{blue}80} percent of their salaries. Compare this with the United States, which did not even pass a law outlawing discrimination against pregnant women until 1978. And American women didn’t have a federal law for job-protected unpaid leave until 1993. We still don’t have mandated paid maternity leave (but then again, we don’t have mandated paid sick leave either).
 \par 
59
 \par 
60
 \par 
WHAT TO EXPECT WHEN YOU'RE EXPECTING EXPLOITATION: ON MOTHERHOOD
 \par 
Eastern European countries also made early use of maternity leave provisions. Poland granted twelve weeks of fully paid maternity leave in 1924, but most countries introduced these provisions after World War II. These nations needed women to work because there was a shortage of male labor, but they had also invested heavily in women’s education and professional training and did not want to lose their expertise (think back to Jake’s reasoning in the beginning of this chapter). For example, the Czechoslovaks introduced the first maternity support policies in 1948, and by 1956 the Labor Code guaranteed women eighteen weeks of paid, job-protected leave. In Bulgaria, the 1971 constitution guaranteed women the right to maternity leaves. In 1973, Bulgarian women enjoyed a fully paid maternity leave of {\color{blue}120} days before and after the birth of the first child as well as an extra six months of leave paid at the national minimum wage. New mothers could also take unpaid leave until their child reached the age of three, when a place in a public kindergarten would be made available. Time on maternity leave counted as labor service toward a woman’s pension, and all leaves were job-protected. Later, an amended law allowed fathers and grandparents to take parental leave in the place of the mother. The Bulgarians covered for those on parental leave with the labor of new university graduates. (In Bulgaria, postsecondary education was free for ‘students who agreed to complete a period of mandatory national service after earning their degrees. These internships allowed young people to get work experience and ensured that a parent’s job would be waiting when he or she returned from leave.)’ The 1973 Bulgarian Politburo decision also included language about reeducating men to be more active in the
 \par 
TTTTTT home: “The reduction and alleviation of woman’s household work depends greatly on the common participation of the two spouses in the organization of family life. It is therefore imperative: a) to combat outdated views, habits, and attitudes as regards the allocation of work within the family; b) to prepare young men for the performance of household duties from childhood and adolescence both by the school and society and by the family.”*
 \par 
In the pages of the Bulgarian women’s magazine The Woman Today, editors published articles about men doing their fair share of the housework and encouraging men to be more active fathers to their children. In the Young Pioneers and the Komsomol, two gender-integrated youth organizations, boys and girls were socialized to treat each other as equals who both had important (albeit different) roles to play in building a socialist society. Where men did mandatory military service after secondary school, women’s reproductive labors counted as an equivalent form of national service. In the end, these policies failed to challenge traditional gender roles, but it is important to recognize that there were at least attempts to redefine ideas about masculinity and femininity. Indeed, specific state efforts to encourage men to be more active fathers and participate more in housework can be found as early as the 1950s in Eastern Germany and Czechoslovakia. However, in the face of male recalcitrance, governments focused their efforts instead on the socialization of housework and child care, hoping to expand the network of communal kitchens and public laundries throughout the country.
 \par 
As early as 1817, the British utopian socialist Robert Owen had suggested that children over the age of three
 \par 
61
 \par 
62 should be raised by local communities rather than in nuclear families, and this idea of the public provision of child care influenced all twentieth-century experiments with state socialism. In addition to maternity leaves, countries like Poland, Hungary, Czechoslovakia, Bulgaria, East Germany, and Yugoslavia invested state funds to expand the network of nursery schools (for children from birth to age three) and kindergartens (for children ages three to six) to support women’s continued labor force participation. Of course, the quality of these child care facilities was uneven across the region and often left much to be desired; children got sick with more communicable diseases, and caregivers were often overwhelmed by the demands of too many children (problems common in day care centers today). But as with so many things in the command economy, planners allocated resources inefficiently, and demand always exceeded supply. In my research in the archives of the Bulgarian Women’s Committee, for instance, I discovered many letters to the relevant ministries complaining about the lack of funds allocated for the creches and kindergartens. Here again, the northern European countries of Sweden, Norway, Denmark, and Finland did much better. They invested state funds to build child care facilities to promote women’s full employment. By the end of the Cold War, Scandinavian female labor force participation rates were second only to those of women in the Eastern Bloc.’
 \par 
Upon publication of my op-ed in the New York Times, I received countless messages from Western readers who discussed their own frustrations. Many women who grew up in the Eastern Bloc also wrote me to relate their memories
 \par 
\[KRISTEN R. GHODSEE\]
 \par 
TTTTTT and opinions about life under socialism, confirming with their personal anecdotes that not all was so bleak behind the Iron Curtain. My favorite letter came from a woman living in Switzerland, born into a middle-class family in Czechoslovakia in 1943. She detailed her own recollections of life under state socialism:
 \par 
SSSSSS maternity leave was eight months after which I went back to work. I had to gently wake our little daughter every morning at 5:30 am as the day care center opened at 6:00 am, and it took us {\color{blue}15} minutes by tram to get there. Once at the day care center, I had to dress her in a uniform and hurry to take the bus at 6:30 am to get to work. I often only just managed to catch the bus, and it was not unusual that the doors of the bus would close behind me with part of my coat still hanging outside. At the time, my husband was getting off work at {\color{blue}2} pm which meant that he could pick up our daughter, buy some groceries and prepare dinner in time for my return at around {\color{blue}5} pm. Shortly after that, we would put our daughter to bed as the next day promised the same rushed routine as the day before. My husband and I were both tired after such a day.”
 \par 
The Swiss-Czechoslovak woman actually meant this description of her former life as a criticism of the German version of the op-ed. She felt that her life was too harried
 \par 
63
 \par 
\[NC ANN A\]
 \par 
64 for sex with her husband. As a working mother, I certainly understand how difficult it is to manage work/family balance, but I don’t think this woman (age seventy-four when she wrote me in 2017) realized the extent of her privilege in state socialist Czechoslovakia compared to the situation of working women today. In her criticism, she mentions that she and her husband had their own private flat, she had eight months of maternity leave, their child had a spot in a state-funded day care center fifteen minutes from home, and her husband got off work at two p.m. and picked up their daughter, bought groceries, and prepared dinner before she returned home at five. She tells me that she and her husband were exhausted by this “rushed routine,” but I suspect she has no idea how luxurious this routine might sound to women, even European women, trying to balance work and family today. In fact, the Cambridge Women’s Pornography Cooperative publishes a book called Porn for Women that features men who pick up their children, buy groceries, and cook dinner before their wives get home from work."
 \par 
For many women, access to affordable and quality child care is more important than maternity leave, especially if the latter is not job-protected. When I first started out as an assistant professor, I was far removed from my family, and I placed my infant daughter in the on-campus day care center full-time for five days a week. One of my colleagues had three children under the age of four—two three-year-old twin girls and a one-year-old son. This colleague, whom I
 \par 
KRISTEN R. GHODSEE will call Leslie, had been an established professional before motherhood and had no desire to forfeit her career. She had accepted a three-quarters-time job well below her qualifications, and her husband also arranged to drop to a four-day week. Leslie paid for the remaining three full days of child care for her three children directly through a payroll deduction. At the end of each month she would waltz into my office with her pay stub. After taxes, insurance payments, and the cost of childcare, Leslie earned about seventy cents a month. She worked thirty hours a week, and often put in unpaid extra time for evening events, for less than $9.{\color{blue}00} of take-home pay per year. And she did this for three years!
 \par 
I once asked Leslie why she didn’t just stay home with the kids, and she admitted that she often fantasized about it. But she refused to give up her work life, and she feared having a gap on her résumé. “I’ve seen too many professional women get completely derailed after taking time out of the labor force,” she explained. “I’m working for nothing now, but it will pay off when my kids are old enough to go to school and I can just go out and get another full-time position.”
 \par 
Consider Leslie’s situation compared to that of Ilse, a composite woman based on research into the experiences of a typical East German woman growing up in the 1980s. Immediately after World War II, the East Germans mobilized women into the labor force. The East German state fully supported women in the workplace, and while it encouraged marriage, being a wife was not considered a precursor to motherhood. Since there weren’t enough men to go around, the state invested heavily in supporting single
 \par 
65
 \par 
66
 \par 
WHAT TO EXPECT WHEN YOU'RE EXPECTING EXPLOITATION: ON MOTHERHOOD mothers. In particular, the East German government idealized early motherhood and built special “mother-and child” housing at universities where students could live with their babies. If Ilse was an average East German woman, she had her first child by the age of twenty-four, probably before she graduated from college, which meant she avoided the fertility decline associated with delayed childbearing. The government heavily subsidized housing, children’s clothing, basic foods, and other expenses associated with child-rearing, as well as providing women like Ilse with access to child care whenever they needed it. By 1989, out-of wedlock births accounted for about {\color{blue}34} percent of all births (compared to only {\color{blue}10} percent in West Germany), but unlike most places in the capitalist West, single motherhood did not lead to destitution. One of my Bulgarian friends earned his degree in Leipzig in the 1990s. He recalls knowing two female students for three years before he realized that they were the mothers of small children. Nothing about motherhood interfered with their education, because their infants were cared for in campus nurseries."”
 \par 
By contrast, women in Western Germany, like women in the United States, returned home to be dependent housewives and mothers after World War II, confined to the Kinder, Kiiche, Kirche (children, kitchen, church). As noted earlier, West German law required a husband’s consent before a woman could work outside the home until 1957, and until 1977 family law insisted that married women were not to let their jobs interfere with their household responsibilities. On a practical level, school schedules and a lack of after school care rendered it almost impossible for West German women to work full-time. Married mothers
 \par 
KRISTEN R. GHODSEE worked mostly in part-time jobs with a larger gender wage gap than that found in the East."
 \par 
Of course, not all socialist countries supported women’s economic independence to the extent of the East Germans (who were locked in their own Cold War rivalry with the West Germans). The Soviets relegalized abortion in 1955 but remained decidedly pronatalist, and even the most basic sex education was absent in the public discourse. Romania and Albania were terrible in terms of women’s reproductive freedoms, with the state forcing women to have babies by restricting access to birth control, sex education, and abortion. Although initially legal in Romania, the infamous Decree {\color{blue}770} of 1966 outlawed abortion in an effort to reverse the population decline, and the law was strengthened in the 1980s to include mandatory gynecological exams for women of reproductive age. The Romanian state essentially nationalized women’s bodies, and many women sought dangerous, illegal abortions, as dramatized in the brilliant 2007 film {\color{blue}4} Months, {\color{blue}3} Weeks and {\color{blue}2} Days.”
 \par 
The key message here is that you do not have to have an authoritarian regime to implement policies that ease the conflict between fertility and employment. Today, almost every country in the world has some form of guaranteed paid maternity leave for women, and many are instituting parental leaves with mandatory paternity leave components. In Iceland, the most gender-equal country on the globe according to the World Economic Forum, fathers get ninety days of leave, and {\color{blue}90} percent of them take it. The state supports both parents to combine their work and
 \par 
67
 \par 
68
 \par 
WHAT TO EXPECT WHEN YOU'RE EXPECTING EXPLOITATION: ON MOTHERHOOD family responsibilities, providing the way for full gender equality in the home as well as the workplace.”
 \par 
While state socialism had its downsides, the sudden change of East European women’s fortunes after 1989 amply demonstrates how free markets quickly erode women’s potential for economic autonomy. In Central Europe, for instance, post-1989 governments pursued conscious policies of “rehabilitation” to support the transition from state socialism to neoliberal capitalism. As state enterprises closed or were sold to private investors, unemployment rates skyrocketed. Too many workers competed for too few jobs. At the same time, the new democratic states reduced their public expenditures by defunding creches and kindergartens. Public child care establishments closed, and new private facilities required substantial fees. Some governments made up for closing kindergartens by extending parental leaves for up to four years, but at far lower rates of wage compensation and without job protections.'®
 \par 
These policies conspired to force women back into the home. Without state-funded child care or well-paid maternity leave, and in a new economic climate where employers had a large army of the unemployed from which to choose, many women were pushed out of the labor market. From a macroeconomic perspective, this proved a boon to transitioning states. Unemployment rates dropped (and thus the need for social benefits), and women now performed for free the care work the state had once subsidized in order to promote gender equality. Later, when deeper budget cuts hit pensioners and the health care system, women already at home looking after their children could now care for the sick and the old—at great savings to the state budget.””
 \par 
KRISTEN R. GHODSEE
 \par 
Given that many women preferred formal employment to the unpaid drudgery of housework, it should not be surprising that post-1989 birthrates plunged. Although birthrates in Eastern Europe were higher than those in Western Europe before 1989, they began to fall as soon as the rehabilitation process began. The institution of free markets actually hindered rather than helped new family formation. Nowhere was this more profound than in Eastern Germany, where skyrocketing unemployment and the collapse of support for child care contributed to an unprecedented and uncoordinated drop in fertility, what the West German press called the “birth strike.” Over a five-year period, the birthrate in the East German states of reunified Germany fell by {\color{blue}60} percent. Although the fertility rates have climbed out of the pits of the 1990s in some countries, the former state socialist nations of Eastern Europe have some of the lowest birthrates in the world today. In 2017, Bulgaria had the fastest-shrinking population in the world, and sixteen of the top twenty nations facing the steepest expected population declines by 2030 were former state socialist nations.” The irony is that as women were being forced back into the home in Eastern Germany, many East German women moved to the West looking for better paid jobs, and these women brought with them a set of expectations that helped West German women find their way into the workplace. The young East Germans who flooded into West Germany after 1989 were the children of working mothers, and they thought it absolutely normal that women would leave their children in kindergartens. When I lived in Freiburg, I met a West German woman who served as the managing director of a well-known academic publishing house in Stuttgart.
 \par 
69
 \par 
70
 \par 
WHAT TO EXPECT WHEN YOU'RE EXPECTING EXPLOITATION: ON MOTHERHOOD
 \par 
Not everyone is a fan of half-hearted government-mandated paid maternity leave policies, especially those that are not enforced. Some feminists object to these policies because they fear they will disadvantage women in competitive labor markets. Employers will prefer to hire men who will not get pregnant, like my friend Jake’s boss. This is why some nations have instituted take-it-or-lose-it paternity leaves to try to equalize the expectation for men’s and women’s care responsibilities. Until 2017, Sweden required that new mothers and fathers take a mandatory sixty days of leave each in order to qualify for the state’s generous benefits. Free marketers argue that companies should be free to set their own priorities without interference from the federal government, but corporate self-regulation has had a pretty abysmal success rate. As of 2013, only an estimated {\color{blue}12} percent of American workers were covered by paid parental leave policies. And this is completely predictable in a free market scenario. No business wants to be known as the one with the generous maternity leave policies because it fears that the women most likely to have babies will flock to it over its competitors. But if the law requires that all companies must offer the same job-protected leave, and if the government picks up part of the tab, as in Braun’s maternity
 \par 
KRISTEN R. GHODSEE insurance plan, then many employers would be willing to support these policies. It would mean they could hire the most promising job candidates and invest in training them with a high degree of certainty that they would reap the benefits of that training. Thus, the only way to ensure that all women benefit from these policies (not just wealthier, professional women working in already enlightened companies) is to have the full weight of the federal, state, or local government behind them.”
 \par 
These same employers could count on workers continuing after childbirth if high quality and reasonably priced child care were readily accessible to all parents of young children. After all, Jake’s star employee did not leave after having her baby. She left, reluctantly, when the weight of an inflexible work life and a patchwork of complicated child care arrangements came crashing down on her exhausted head. The biggest help to working women would be the expansion of high-quality, federally funded child care, which would support women’s ability to combine motherhood with paid employment. The United States once came close to having a nationwide child care system: the Comprehensive Child Development Act passed by a bipartisan vote of Democrats and Republicans in 1971. The act would have funded a national network of child care centers providing high-quality educational, medical, and nutritional services, a crucial first step for universal child care. President Richard Nixon vetoed the act and criticized the “family weakening implications of the system” it envisioned. In his official veto, Nixon wrote: “For the Federal Government to plunge headlong financially into supporting child development would commit the vast moral authority of the
 \par 
11
 \par 
72
 \par 
WHAT TO EXPECT WHEN YOU'RE EXPECTING EXPLOITATION: ON MOTHERHOOD
 \par 
National Government to the side of communal approaches to child-rearing over against the family-centered approach.” This “family-centered” approach required the unpaid labor of women in the home, reinforcing the traditional gender roles of male breadwinner and female homemaker. In essence, Nixon asked Why should the government pay for something that we can get women to do for free?”°
 \par 
Although research shows that children are not harmed by quality center-based child care, and may even enjoy greater cognitive, linguistic, and socioemotional development than children cared for at home, American conservatives hate the idea of child care because it also challenges male authority in the family. One op-ed contributor for Fox News sees universal child care as part of an evil plot, arguing “totalitarian governments have gone to great lengths to indoctrinate children, and the biggest obstacles they faced was parents who contradicted what the government was telling their kids” In this view, everything that state socialist countries did to support women—increasing labor force participation, liberalizing divorce laws, creating kindergartens and creches, and supporting women’s economic independence—was aimed at brainwashing children. Even public schools served the primary purpose of indoctrination.”!
 \par 
Women’s rights and entitlements are thus painted as part of a coordinated plan to promote world communism, a threat spreading across the West. From this perspective, even democratic socialist Sweden has “aggressively instituted a very costly system of nursery school care” to “force women out of the home and into the labor force.” As if Swedish women wouldn't choose to work of their own
 \par 
KRISTEN R. GHODSEE accord. Behind the fear of government indoctrination of children is a real fear of women’s economic independence and the breakdown of the traditional family.”
 \par 
For now, it is still women who must gestate and deliver the actual babies (at least until scientists develop pathogenesis), but fathers can be just as involved in child care as mothers. The number of stay-at-home dads is growing, and it may be that one day employers will view male employees as potential caregivers in the same way they now view women. But until that time, competitive labor markets will continue to penalize women for their biology. The high cost of private child care—combined with the gender wage gap and social expectations that young children need mothers more than fathers—means that it is overwhelmingly still women who interrupt their work lives to stay home with small children. In the United States, these years out of the labor force hurt mothers in a variety of ways: lost income, being passed over for promotions, less money toward social security or retirement, and increased economic dependence on men. Of course some women want to stay at home, and this should remain a choice, as long as staying home to do care work does not entail financial dependence. Our goal should be that an equal number of men and women choose to act as stay-at-home parents. While this option should be open to all, I expect most men and women will not take it. With reasonable parental leaves and enough high-quality affordable child care to go around, we really can have our cake and eat it too.
 \par 
One of the most obvious problems with many state socialist countries was that while citizens were guaranteed employment by the state, they were often forced to work at
 \par 
73
 \par 
7h
 \par 
WHAT TO EXPECT WHEN YOU'RE EXPECTING EXPLOITATION: ON MOTHERHOOD jobs they didn’t like. Many routine jobs were monotonous and unsatisfying (not so unlike routine jobs in the West). But too many American women who want to work are forced to stay home because of the scarcity of quality child care, the high cost when it is available, and the lack of flexibility in the labor market. Other women need to work to survive, particularly since private health insurance in the United States binds employees to their workplaces if they don’t want to lose benefits. Not all women have the option of a man who can support her, and even those who do would be wise not to rely too heavily on that option. Women should not be compelled into romantic relations because it is their only chance to have a roof over their heads. Our system also places a massive burden on men, since those who cannot afford to support their spouses are shunned as romantic partners (something that is already happening in the United States, where marriage rates among the poor are at an all-time low).
 \par 
At the end of the day, differences in reproductive biology make it impossible to treat men and women as equals in labor markets, where employers endeavor to hire those they guess will be their most valuable workers. This is a sticky problem that lacks simple solutions, but policies like parental leaves and state-funded universal child care help alleviate the root causes of gender discrimination. These policies started as socialist propositions and had the explicit goal of gender equity at work and at home. Over the last century, such policies have begun to work their way into the legislation of almost every country around the globe. In 2016, the United States joined New Guinea, Suriname, and some
 \par 
KRISTEN R. GHODSEE islands in the South Pacific in being the only countries in the world lacking a national law on paid parental leave.
 \par 
When I think about the woman who quit Jake’s firm to stay home with her baby and my former colleague Leslie, who worked for seventy cents a month, I lament that motherhood—which should be such a source of joy—has devolved into a crushing burden for so many women. Nowhere in the developed world is it harder for ordinary people to start their families. Surely the richest countries on the planet can do better.
 \par 
73
 \par 
\begin{ figure }
	\centering
	\\includegraphics[width=1.\textwidth]{ temp_files/images/UP_logo.png }
	\caption{Flora Tristan (1803-1844): A French utopian socialist theorist and activist who argued that the liberation of the working classes could not be achieved without the concomitant emancipation of women. Her 1843 essay,.Jhe Worker's Union, is a foundational socialist feminist text in which Tristan envisioned a grand la- bor collective in which workers (both men and women) would pool their re- sources to provide social services for their own benefit. Courtesy of TASS.}
	\label{ }
\end{ figure }