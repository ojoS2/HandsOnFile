{\chapter{The Excons} } {\label{The Excons} }{\par}{\textit{	} } {\par}{\par} {\textbf{\textit{	} } } {\par} 
{\footnote{This chapter originally appeared as an article in Lingua Franca (February 2001): 24–33.} }	{\textbf{\textit{In the spring of 2000, Alex Star, editor of the now-defunct Lingua Franca, commissioned me to write a profile of John Gray and Edward Cutaway, two conservative intellectuals who had moved to the left. Throughout the summer and fall, I interviewed Gray and Cutaway as well as other conservatives such as William F. Buckley, Irving Bristol, and Nor-man Podhoretz. It was a difficult time for the right. Bill Clinton was still president; 9/11 had not yet occurred. Prosperity was a given, war was a distant memory, and learned people still spoke of the end of history. The moment had a vastly diff erent feel from today, and it affected how conservatives thought about their ideas and politics. While some of the references and statements in this article are now dated, and some of its claims I no longer believe, I have decided not to revise the piece in order to pre-serve the mood of that moment. In chapter 8, I revisit some of the issues discussed here in light of 9/11, the war on terror, and the Iraq War.} }} {\par} {\textbf{\textit{There is another reason I have not revised this article. Though I had read Burke, Makeshift, and Notice in college and graduate school, researching and writing this article was my first sustained encounter with the worldview of the right. (It remains an unfortunate reality of American higher education that social scientists and historians can get through} }} {\par} {\par} {\textbf{\textit{Their training with only the most passing acquaintance with conservatism.) This article became a kind of sentimental education for me, my introduction to the agony and the ecstasy of the conservative mind. While I would certainly revise much of it today—particularly the underlying premise that the conservatives I discuss here are different from the main-stream—the article nevertheless provides the reader with a glimpse of what first interested me about the right and how I came to write this book.} } }{\par} According to popular myth, it was Winston Churchill who said, “Any man under thirty who is not a liberal has no heart, and any man over thirty who is not a conservative has no brains.” He didn’t say it, but his imprimatur turned a clever quip of uncertain provenance into an axiom of political biography: Radicalism is a privilege of youth, conservatism a responsibility of age, and every thinking person eventually surrenders the first for the second. From Max Eastman to Eugene Genovese, Whittaker Chambers to Ronald Radish, intellectuals migrate from left to right as if obeying a law of nature.{\par} Or do they? After all, John Stuart Mill published The Subjection of Women when he was sixty-three. In the last ten years of his life, Diderot hailed the American Revolution and blasted France as the reincarnation of imperial Rome. And when George Bernard Shaw addressed the question of politics and aging, he suggested just the opposite of what Churchill is supposed to have said. “The most distinguished persons,” Shaw wrote in 1903, “become more revolutionary as they grow older.” {\color{blue} 1 } {\par} Since the end of the Cold War, several prominent conservatives have followed Shaw’s prescription and turned left. Michael Lind, once a top editor at Irving Bristol’s The National Interest, has denounced his previous allies for prosecuting a “class war against wage-earning Americans.” Their market-driven theories, he writes, are “unconvincing,” their economic policies “appalling.” Arianna huffington, erstwhile confederate of Newt Gingrich, now inveighs against a United States where the great majority is “left choking on the dust of Wall Street’s galloping bulls.” {\color{blue} 2 } Glenn Lowry, an economist and former neoconservative darling, sports the signature emblem of left membership: he has become one of Norman Podhoretz’s ex-friends. But today’s most flamboyant expatriates are an Englishman, John Gray, and a Jewish émigré from Transylvania, Edward Cutaway.{\par} In the 1970s, John Gray was a rising star of the British New Right. An Oxford-trained political philosopher, he penned prose poems to the free market, crisscrossed the Atlantic to fuel up on the high-octane libertarianism of American right-wing think tanks, and, says a longtime friend, enthralled his comrades late into the night with visions of the coming “anarcho-capitalist” Utopia. But after the Berlin Wall collapsed, Gray defected. First he criticized the Cold War triumphalism of Francis Fukuyama’s “end of history” thesis and counseled against scrapping Britain’s National Health Service. And then in 1998, from his newly established position as professor of European thought at the London School of Economics (LSE), he handed down False Dawn, a ferocious denunciation of economic globalization. Assailing the “shock troops of the free market,” Gray warned that global capitalism could “come to rival” the former Soviet Union “in the suffering that it inflicts.” {\color{blue} 3 } Now he is a regular contributor to The Guardian and New Statesman, Britain’s principal left venues. So profound is his conversion that no less a figure than Margaret Thatcher has reportedly wondered, “Whatever became of John Gray? He used to be one of us.” {\color{blue} 4 } {\par} And what of Edward Cutaway? Once, he was one of Ronald Reagan’s court intellectuals, a brilliant military hawk who mercilessly criticized liberal defense policies and provided the philosophical rationale for the American military buildup of the 1980s.{\par} Liberal critics called him “Crazy Eddie,” but cutting a figure that was part Dr. Strange love and part Dr. Zhivago, Cutaway effortlessly parried their arguments, pressing the Cold War toward its conclusion. {\color{blue} 5 } Today, he is disillusioned by victory. He finds the United States a capitalist nightmare, “a grim warning” to leaders seeking to unleash free-market forces in their own countries. Deploying the same acerbic wit he once lofted against liberal peacetime, he mocks the “Napoleonic pretensions” of American business leaders, challenges the conventional wisdom that capitalism and democracy are inevitable bedfellows (“ free markets and less free societies go hand in hand”), and decries the savage inequalities produced by “turbo-capitalism.” He excoriates European center-leftists like Tony Blair for abandoning their socialist roots and for their unwillingness “to risk any innovative action” on behalf of “ordinary workers.” With their “disdain for the poor and other losers” and “contempt for the broad masses of working people,” Cutaway writes, Clintonesque New Democrats and European Third Waters “can yield only right-wing policies.” {\color{blue} 6 } {\par} In their original incarnations, Gray and Cutaway thrilled to two of conservatism’s galvanizing passions—anticommunism and the free market. But since the fall of the Soviet Union, they have been posing questions about the market they once would never have dared ask.{\par} Yet for all their disgust with unbridled capitalism, Gray and Cutaway find it hard to embrace any of the alternatives: The furthest Gray will go is to characterize himself as “center-left.” Nor is the left too eager to claim either of them. One reviewer of False Dawn wrote in These Times that Gray was merely a standard-bearer for the old regime, driven less by “a genuine hatred of inequality, injustice or poverty” than by “a deep fear of political instability.” {\color{blue} 7 } With Communism in shambles and the market omnipotent, the agonistic passion that originally inspired Cutaway and Gray now fi nds itself without a home. They are today’s most poignant exiles, lost in a diaspora of their own making.{\par} Conservatives usually style themselves as chastened skeptics holding the line against political enthusiasm. Where radicals tilt toward the utopian, conservatives settle for world-weary realism. But, in reality, conservatives have been temperamentally antagonistic, politically insurgent, and utterly opposed to established moral convention. Ever since Edmund Burke, thinkers from Samuel Taylor Coleridge to Martin Heidegger have sought a more intense, almost ecstatic mode of experience in the spheres of religion, culture, and even the economy—all of which, they believe, are repositories of the mysterious and the in eff able. Indulging in political romanticism, they draw from the stock-in-trade of the counterenlightenment, celebrating the intoxicating vitality of struggle while denouncing the bloodless norms of reason and rights. As Isaiah Berlin observed of Joseph de Maistre:{\par} {\textbf{\textit{His violent preoccupation with blood and death belongs to a world different. . . From the slow, mature wisdom of the landed gentry, the deep peace of the country houses great and small. . . . The facade of Maître’s system may be classical, but behind it there is something terrifyingly modern, and violently opposed to sweetness and light. {\color{blue} 8 } } } }{\par} The battle in the twentieth century against Communism and social democracy provided the perfect vehicle for these conservative sensibilities. For figures like John Gray, the Soviet Union and the welfare state were the ultimate symbols of cold Enlightenment rationalism, and the free market was the embodiment of the romantic counterenlightenment. But revolutionary romantics ultimately suff her the fate of all romantics: disillusionment. And so today, with Communism in ruins and the free market triumphant, the dissident spirit that originally inspired Gray now fires an equally militant apostasy.{\par} Gray was born in 1948 and grew up outside Newcastle, a port city near the North Sea in a coal-mining region only fifty miles from Scotland. In a country where accent is destiny, one still hears faint traces of his northeastern working-class origins, about which he is slightly defensive. His father was a carpenter; his entire family voted Labor. Gray arrived at Oxford in 1968, the annus mirabilis for young leftists throughout Europe. Sporting the costume of the period—“my hair was long, but everybody’s hair was long”—he traveled to London to demonstrate against the Vietnam War. After receiving his degree in philosophy, politics, and economics, Gray stayed on at Oxford for graduate school, writing a thesis on John Stuart Mill and John Rawls, both sympathetic to a liberal socialism that Gray initially found attractive.{\par} But as he muddled through Rails’s A Theory of Justice, Gray grew weary of the effort to extract socialist policies from liberal formulas. Part of his malaise was induced by Rails’s congested prose. “It’s an almost unreadable book,” he says. Rails’s plodding style seemed to mirror the deeper political ennui of social democracy. His work, says Gray, was “a transcendental deduction of the Labour Party in 1963.” Like many New Leftists in the United States, Gray found the business of the welfare state dull and uninspired, the weak tea of colorless bureaucrats. As he would later describe it, the welfare state was the product of a “triangular collusion of employers, unions and government.” It was a “colossal apparatus” extracting resources and energy from an enervated citizenry. Tepid compromise was the rule of the day; political leaders tried to be all things to all people. They refused “to admit the reality of con- fl its,” that “one equality, one demand of justice, may compete with another.” {\color{blue} 9 } The welfare state, in short, was a far cry from the vital working-class radicalism that had produced it.{\par} In Thatcherism, Gray caught a glimpse of revolutionary eternity. “There was a revolutionary, indeed a Bolshevik, aspect to the Thatcher ite project at the start which I thought was both exciting and necessary,” he says. Thatcher assumed the leadership of the Conservative Party at just about the time of Gray’s conversion to capitalism. She promised to liberate Britain from the stifling routine of social democracy, and the free market from the chains of state planning. Though no egalitarian, Thatcher stoked the ambitions of middle- and working-class voters who saw the free market as a vehicle of upward mobility.{\par} Her most impressive moment came in 1980, after her first year in power, when her policies seemed to be pushing the economy toward disaster. Having denounced her predecessor Edward Heath for executing his notorious “U-turn,” when he capitulated to left-wing pressure after vowing a rollback of social democracy, Thatcher faced pressure from moderates within her own party—the Tory “Wets”—to reverse course. Instead of retreating, she defiantly faced down her temporizing critics, memorably declaring, “You turn if you want to. The lady’s not for turning.” {\color{blue} 10 } Conservatives were smitten. Norman Barry, another Thatcher ite and until recently a close friend of Gray’s, recalls, “I had thought she was just an election winner who wasn’t Labor. But when she lifted exchange controls, I thought, ‘This babe knows market economics.’ So then I thought, ‘Yeah!’ And then she began privatization and other things. And then she wouldn’t do a U-turn, I thought, ‘This is for real.’”Many Hatcheries thought of themselves as free-market revolutionaries, and Gray brought to their cause a romantic panache not often associated with neoclassical economics. In 1974, he began reading the work of Friedrich Hayek, the Austrian-born economist and fierce critic of state planning. Ten years later, Gray published Hayek on Liberty, which the master himself described as “the first survey of my work which not only fully understands but is able to carry on my ideas beyond the point at which I left off.” The Hayek that Gray depicted was no antiseptic defender of property rights and low taxes. He was an exotic explorer of the subterranean, quasi-rational currents of human life, a Viennese voice that had more in common with Sigmund Freud and Ludwig Wittgenstein than with Milton Friedman or Robert Nozick. If Hayek on Liberty was an impassioned ode to the market, Gray was its yearning Byron.{\par} Where many conservatives saw in Hayek the logical fulfillment of a calm, quintessentially British tradition of political economy extending back to Adam Smith, Gray detected an “uncompromising modernity” in Hayek’s vision of the free market. {\color{blue} 11 } Intellectual ferment, political extremism, and social decay characterized phi node- since Vienna, the milieu in which Hayek was born. Out of this whirlwind came psychoanalysis, fascism, and modern economics. Each challenged old orders of knowledge and politics. Hayek followed in the footsteps of the late-nineteenth-century Austrian school, claiming that “economic value—the value of an asset or resource—is conferred on it by the preferences or valuations of individuals and not by any of its objective properties.” {\color{blue} 12 } While classical economists from David Ricardo to Karl Marx believed there had to be something real — most important, physical labor—behind the mysterious veil of prices, Hayek argued that it was only the eccentric preferences of particular human beings that gave value to goods in the world. An almost hyperactive subjectivity—comparable to Freud’s anarchic id—haunted Gray’s Hayek, reflecting Vienna’s “experience of an apparently inexorable drift to dissolution.” {\color{blue} 13 } {\par} Against philosophers who elevated theoretical reason to the highest form of knowledge, Hayek, wrote Gray, believed that rational understanding was only the tip of the iceberg. Beneath it lay a murky stratum of thought “rarely expressible in theoretical or technical terms,” and it was the free market’s particular genius to harness these premonitions to everyday economic activity. {\color{blue} 14 } {\par} Entrepreneurs were the sublime mediums of such “tacit knowledge,” channeling its deep truths to other market actors. They were romantic heroes possessed by fl ashes of almost poetic vision. “Entrepreneurial insight or perception,” explained Gray, was a matter not of book learning but of “serendipity and fl air.” It was “a creative activity insusceptible of formulation in hard and fast rules.” Lying “beyond our powers of conscious control,” the “entrepreneurial perception” appeared only infrequently, striking suddenly and with-out warning. {\color{blue} 15 } When it did appear, it reordered the universe.{\par} The market, in short, provided a refuge for self-expression and creativity, a sanctuary for the rapturous counterenlightenment. Unimaginative writers were content to argue that markets “allocate scarce resources most efficiently” or that the market “allows for the motive of self-interest.” But such defenses missed a more elemental truth: markets allowed for the expression of a “whole variety of human motives, in all of their complexity and mixtures.” {\color{blue} 16 } The market supplied a theater for dramatic self-disclosure, a stage on which individuals could project their most irrepressible visions and strenuous desires.{\par} All love affairs come to an end, but Gray’s breakup with the market has been particularly venomous. He now denounces it as the scourge of civilization. In the United States, he writes, the free market has “generated a long economic boom from which the majority of Americans has hardly benefited.” Americans suff her from “levels of inequality” that “resemble those of Latin American countries.” The middle class enjoys the dubious charms of “asset-less economic insecurity that afflicted the nineteenth-century proletariat.” The United States stands perilously close to massive social disruption, which has been held at bay only “by a policy of mass incarceration” of African Americans and other people of color. “The prophet of today’s America,” Gray claims, “is not Jefferson or Madison. . . . It is Jeremy Bentham”—the man who dreamed of a society “reconstructed on the model of an ideal prison.” {\color{blue} 17 } {\par} Even more appalling, writes Gray, global elites have sought to make American capitalism the model for the world. Even though market regimes vary by culture and country, the high priests of globalization impose a one-size-fits-all American model—with its minimal welfare state, weak business and environmental regulations, and low taxes. “According to the ‘Washington consensus,’” writes Gray, “the manifold economic cultures and systems that the world has always contained will be redundant. They will be merged into a single universal free market” based on the “world’s last great Enlightenment regime, the United States.” {\color{blue} 18 } {\par} When Gray first uttered these heresies, many of his conservative friends were shocked. Like Gray, Norman Barry is a political theorist who has written on Hayek. A professor at the University of Bucking-ham, the only wholly private university in Britain, he was the best man at Gray’s second wedding but now rarely speaks with him. Barry cannot shake the suspicion that Gray’s political turn was motivated by pure opportunism. “I believe in a proposition of neoclassical economics: Everybody’s a utility maximizer,” he explains. “It might have been a good career move to detach himself from libertarianism. I am speculating but not wildly. Libertarians don’t get the best positions in universities.” When Gray was only a fellow at a small Oxford college, claims Barry, “he used to say, ‘Well, the way the world works I wouldn’t get a chair.’ . . . You don’t get professorships at LSE if you’re a free-market fanatic.” The only continuity in Gray’s position that Barry recognizes is his penchant for “philosophical promiscuity.” Gray, says Barry, “was always flitting from person to person, philosopher to philosopher. . . . He couldn’t form a steady relationship with any thinker. He tried a bit of Popper. Tried Hayek. Of course, he later dumped Hayek. Other writers he would try and dump.”Gray claims that he changed his mind for two reasons. During the late 1980s, he says, he began to suspect that political thinking on the right had stiff end into stale ideology—not unlike the dull Bayesianism he fl ed so long ago. Gray had once thought of Thatch- prism as tactically flexible and politically savvy, a movement sensitive to popular moods, its leader a Machiavellian virtuoso of political change. But he now believed that the movement had lost its artistry; supple thought had degenerated into rote incantation. Gray says, “What was striking about Bolshevism was that Lenin was so extraordinarily flexible. Then it hardened into Trotskyism. And similarly Thatcherism began to harden. . . . It was a habit of thought that I found deeply repugnant.”The collapse of the Soviet Union also forced Gray to question his free-market faith. Until 1989, Gray says, it made sense to think of the state as “the principal enemy of well-being,” which was the attitude within “the admittedly hothouse atmosphere of the right-wing think thanks.” But after the Soviet empire fell, the former Yugoslavia spiraled into genocidal civil war, and Western free-marketeers applied shock therapy to formerly Communist countries with disastrous results, Gray came to think that the state was a necessary evil, perhaps even a positive good. It was the only force that could prevent societies from sliding into total chaos, extreme inequality, and poverty.{\par} But there is a deeper reason for Gray’s turn: by itself, the market could not sustain his affections. Without the Soviet Union and the welfare state as diverting symbols of Enlightenment rationalism, Gray could no longer believe in the market as he once had. The market, he now had to admit, sponsors a “cult of reason and Effie - mainly.” It “snaps the threads of memory and scatters local knowledge.” He used to think that the free market arose spontaneously and that state control of the economy was unnatural. But watching Jeffrey Sachs and the International Monetary Fund in Russia, he could not help but see the free market as “a product of artifice, design and political coercion.” The market had to be created, often with the aid of ruthless state power. Today, he argues that Thatcher built the free market by crushing trade unions, hollowing out the Conservative Party, and disabling Parliament. She “set British society on a forced march into late modernity.” Gray believes “Marx- ismleninism and free-market economic rationalism have much in common.” Both, he writes, “exhibit scant sympathy for the casualties of economic progress.” {\color{blue} 19 } There is only one difference: Communism is dead.{\par} In an unguarded moment, Norman Barry confesses that he cannot fathom Gray’s shift. “Maybe I just misunderstood him,” he says, “but I thought that he did believe deeply. Nobody could have read that amount of stuff without believing some of it, anyway. I wonder whether he ever did.” Gray did believe, but his belief was different from Barry’s. Barry loves the market because it operates according to “the iron laws of economics.” As he puts it, these may “take a little longer than Newtonian laws. If I drop this disk, it’s down in a second. If I introduce rent control, it would take maybe six months to create homelessness.” But, he adds, “it’s just as decisive.” By contrast, Gray once believed in capitalism precisely because he sought an escape from the laws of Newton. Having realized that the market inhibits passionate self-expression, Gray has been forced to acknowledge the truth of Irving Bristol’s dictum: “Capitalism is the least romantic conception of a public order that the human mind has ever conceived.” {\color{blue} 20 } {\par} By the time Edward Cutaway was in his early forties, he had outrun Nazis, escaped Communists, and been shot at by leftist guerrillas in Central America. But to this day, he remembers his childhood move from Palermo to Milan as the most “traumatic” event of his life. Born in 1942 into a wealthy Jewish family in Romania, Cutaway grew up in southern Transylvania, which was briefly occupied by the Nazis in 1944. When he was five years old, his family fl ed an imminent Communist takeover and settled in Palermo. It was winter, Cutaway recalls, and “Paris and London were shivering. There was a fuel shortage. Milano was shivering. Things were pretty bleak.” But in Palermo “the opera was in full swing.” It was “the land of oranges and lemons,” he says, where people could swim and ski almost year-round. Five years later, Cutaway’s family moved again, this time to Milan, the industrial center of Italy. “Stuff y and fog-ridden,” Milan made Cutaway miserable. “There was nowhere to play. The parks were a disgrace. I lost all my friends from Palermo. I found myself. . . Amid a bunch of very bourgeois kids.” The good life on the Mediterranean had come to an end, done in by dour industrialists to the north.{\par} For most of his adult life, Cutaway waged a militant struggle against Communism. Inspired by a strategic military vision that connected the Gallic Wars to the civil wars of Central America, he worked closely with the U.S. Defense Department as a consultant, advising everyone from junior officers to the top brass. But Cutaway was more than a cold warrior. He was a warrior, or at least a fervent theorist of “the art of war.” Where generals thought victory depended on aping management styles from IBM, Cutaway made the case for ancient battlefield tactics and forgotten maneuvers from the Roman Empire. Cutaway urged the military to look to Hadrian, not Henry Ford, for guidance. It was an arduous struggle, with officers more often acting like organization men than soldiers. Once again, Cutaway found his preferred way of life threatened by the culture of capitalism.{\par} Cutaway first gained notoriety in Britain, where he settled after receiving his undergraduate degree in economics at the London school of Economics. In 1968, he published Coup d’Eat: A Practical Handbook. The twenty-six-year-old author dazzled his readers with this audacious how-to guide, prompting a delighted John le Carré to write, “Mr. Cutaway has composed an unholy gastronomic guide to political poison. Those brave enough to look into his kitchen will never eat quite as peacefully again.” In 1970, Cutaway published an equally mischievous piece in Esquire, “A Scenario for a Military Coup d’Eat in the United States.” Two years later, he moved to the United States to write a dissertation in political science and classical history at Johns Hopkins, conducting extensive research using original Latin, German, French, English, and Italian sources. The result was they widely praised The Grand Strategy of the Roman Empire. While in graduate school, Cutaway began to work as a consultant to various branches of the U.S. armed services, ultimately making recommendations on everything from how NATO should conduct tactical maneuvers to what kind of rifle soldiers in the El Salvadoran military should carry.{\par} When Ronald Reagan ran for president in 1980, Cutaway was at the top of his game. A fellow at Georgetown’s Center for Strategic and International Studies and a frequent contributor to Commentary, he argued that the United States should accelerate the high-tech arms race, forcing the Soviet Union into a contest it could not win. Reagan’s closest advisers eagerly welcomed Cutaway to their inner circle. Just after Reagan’s election, Cutaway attended a dinner party in Bethesda, along with Jean Kirkpatrick, Fred Idle, and other luminaries of the Republican defense establishment. Richard Allen, who would become Reagan’s first national security adviser, worked the crowd, pretending to dispense positions in the administration as if they were party favors. As the Washington Post reported, Cutaway declined, explaining over chocolate Tia Maria pie, “I don’t believe scribblers like myself should be involved in politics. It’s like caviar. Very nice, but only in small quantities.”When pressed by Allen, he joked, “I only want to be vice-consul in Florence.” Allen responded, “Don’t you mean proconsul?” {\color{blue} 21 } {\par} The prep-school gladiator bonhomie evaporated before the end of Reagan’s first term. Cutaway may have been an invaluable asset when pushing for more defense spending, but he made enemies with his loud—and ever more sarcastic—criticisms of Pentagon mismanagement. In 1984, he published The Pentagon and the Art of War, where, among other things, he depicted Defense Secretary Caspar Weinberger as more of a slick used-car salesman than a genuine statesman. Military politicos struck back, dropping Cutaway from a roster of pro bono Pentagon consultants (he continued to do contract work elsewhere in the defense establishment). In 1986, Weinberger explained to the Los Angeles Times that Cutaway “just lost consulting positions from total incompetence, that’s all.” {\color{blue} 22 } {\par} But it was more than Cutaway’s criticisms of Weinberger that got him in trouble with the Defense Department. His real mistake in The Pentagon and the Art of War was to go after the military’s conduct during the Vietnam War. Cutaway downplayed the armed forces’ favorite explanations for their defeat in Vietnam—weak-willed politicians, the treasonous press, a defeatist public. He argued instead that America’s warrior elite had simply lost the taste for blood. During the Vietnam War, he wrote, “deskbound officers” were always “far from combat.” Their penchant for “out-right luxury” had a devastating effect on troop morale. Although Julius Caesar “retained both concubines and calamities in his rear-ward headquarters, ate off gold plate, and drank his Damian wine from jeweled goblets,” when he was on the front lines with his soldiers he “ate only what they ate, and slept as they did—under a tent if the troops had tents, or merely wrapped in a blanket if they did not.” By contrast, American officers refused “to share in the hardships and deadly risks of war.” {\color{blue} 23 } {\par} Pointy-headed bureaucrats also sapped the military’s strength, according to Cutaway. Always looking to cut costs, Pentagon officials insisted that weapons, machinery, and research-and-development programs be standardized. But this only made the military vulnerable to enemy attack. Standardized weapons systems were easily overcome; having overwhelmed one, an enemy could overwhelm them all. When it came to the military, Cutaway concluded, “we need more ‘fraud, waste, and mismanagement.’” {\color{blue} 24 } {\par} Top generals were obsessed with efficiency partially because they learned the methods of business management instead of the art of war. For every officer with a degree in military history, there were a hundred more “whose greatest personal accomplishment is a graduate degree in business administration, management or economics.” “Why should fighter pilots receive a full-scale university education,” Cutaway asked in the Washington Quarterly, “instead of being taught how to hunt and kill with their machines?” {\color{blue} 25 } {\par} The ultimate source of the military’s dysfunction was its embrace of American corporate culture and business values. Like Robert Mc namara, whom President Kennedy transferred to the Pentagon from the Ford Motor Company, most defense secretaries were in thrall to “corporate-style goals.” They sought the least risky, most cost-effective means to a given end. They preferred gray suits, eschewing “personal eccentricities in dress, speech, manner, and style because any unusual trait may irritate a customer or a banker in the casual encounters common in business.” Officers were merely “managers in uniform,” Cutaway told Forbes. But, he noted, “what is good for business is not good for deadly conflict.” Although “safely conservative dress and inoffensively conventional style” might work in an off ice, they could be deadly on the battlefield; they squelched bold initiatives and idiosyncratic genius. {\color{blue} 26 } Intimating that capitalism had colonized—indeed destroyed—spheres of society that were not strictly economic, Cutaway came perilously close to identifying himself with leading voices from the Marxist tradition—Jürgen Habermas, Georg Lukács, even Marx himself.{\par} While the Soviet Union still existed, Cutaway was able to channel his contempt for managerial and corporate values into proposals for military reform. The struggle against Bolshevism fully captured his imagination, speaking to principles of individualism, independence, and personal dignity that he had learned as a child of Jewish atheists. Cutaway’s parents taught him, he says, that “you wanted your shoulders out walking down the street. The master of your fate. Not to walk hunched, afraid that God will punish you if you eat a ham sandwich.” He continues: “There was a certain contempt about piety. Piety was not seen as compatible with dignity.” Dignity, he goes on, “is what we were defending in the Cold War. It was ideological. It was very fitting for me to be in the United States, to become an American, because the Americans were and are the ideological people. They were perfectly cast to be enlisted in an ideological struggle.”But now that the battle against Communism has been won, Cutaway has lost interest in most military matters; he no longer sees any compelling ideological reason to care about strategy and tactics. “Security problems and such have become peripheral, for all countries and for people, for myself as well. I don’t engage my existence in something that is peripheral. . . . There was a compel-ling imperative to be involved. There isn’t now.”Cutaway does occasionally muster energy for a specific project. During one of our interviews, he speaks by phone with a State Department official about doing consulting work for the war against the Colombian guerrillas. But when I ask him if the Colombian government is worth defending, he is uncharacteristically hesitant, finally confessing, “I don’t know if anything is worth defending, but I think the guerrillas are worth fighting.” I ask him why, and he responds that the guerrillas are aligned with drug traffickers who “do everything from taking people’s places in restaurants in Medellin on a Saturday night—people are waiting to take seats and these guys come in, and they grab their tables—everything from that to murder.”Military struggle may no longer hold any ideological allure for Cutaway, but his disaffection affords him the time and intellectual space to confront the enemy he has been shadowboxing his entire life: capitalism itself. “The market,” he says, “invades every sphere of life,” producing a “hellish society.” In the same way that market values once threatened national security, they now threaten the economic and spiritual well-being of society. “An optimal production system is a completely inhuman production system,” he explains, “because. . . You are constantly changing the number of people you employ, you’re moving them around, you’re doing different things, and that is not compatible with somebody being able to organize an existence for himself.”Although Cutaway writes in his 1999 book Turbocapitalism, “I deeply believe. . . In the virtues of capitalism,” his opposition to the spread of market values is so acute that it puts him on the far end of today’s political spectrum—a position that Cutaway congenially enjoys. {\color{blue} 27 } “Edward is a very perverse guy, intellectually and in many other ways,” says former Commentary editor Norman Podhoretz, one of Cutaway’s early champions during the 1970s. “He’s a contrarian. He enjoys confounding expectations. But I frankly don’t even know how serious he is in this latest incarnation.” Cutaway insists that he is quite serious. He calls for socialized medicine. He advocates a strong welfare state, claiming, “If I had my druthers, I would prohibit any form of domestic charity.” Charity is a “cop-out,” he says: it takes dignity away from the poor.{\par} The only thing that arouses Cutaway’s ire more than untrammeled capitalism is its elite enthusiasts—the intellectuals, politicians, policymakers, and businessmen who claim that “just because the market is always more efficient, the market should always rule.” Alan Greenspan earns Cutaway’s special contempt: “Alan Greenspan is a Spencerian. That makes him an economic fascist.” Spencerian's like Greenspan believe that “the harshest economic pressures” will “stimulate some people to. . . Economically heroic deeds. They will become great entrepreneurs or whatever else, and as for the ones who fail, let them fail.” Cutaway’s other bête noire is “Chainsaw Al” Dunlap, the peripatetic CEO who reaps unimaginable returns for corporate shareholders by phi ring substantial numbers of employees from companies. “Chainsaw does it,” says Cutaway, referring to Dunlap’s downsizing measures, “because he’s simpleminded, harsh, and cruel.” It’s just “economic sadism.” Against Greenspan and Dunlap, Cutaway affirms, “I believe that one ought to have only as much market efficiency as one needs, because everything that we value in human life is within the realm of inefficiency—love, family, attachment, community, culture, old habits, comfortable old shoes.”The defections of Cutaway and Gray suggest just how unkind the end of the Cold War has been to the conservative movement. It is increasingly clear that the fragile coalition of libertarians, traditionalists, and free-market enthusiasts once held together by the glue of anticommunism will no longer stick. The end of the Soviet Union “deprived us of an enemy,” Irving Bristol, the intellectual godfather of neoconservatism, tells me. “In politics, being deprived of an enemy is a very serious matter. You tend to get relaxed and dispirited. Turn inward.” Notorious for his self- confidence, Bristol now confesses to a sad bewilderment in the postcommunist world. “That’s one of the reasons I really am not writing much these days,” he says. “I don’t know the answers.”One might think the triumph of the free market would thrill right-wing intellectuals. But even the most revered conservative patriarchs worry that the market alone cannot sustain the flag- going energies of the movement. After all, Reagan and Thatcher summoned conservatives to a political crusade, but the free-mar- KET ideology they unleashed is suspicious of all political faiths. The market’s logic glorifies private initiative, individual action, the brilliance of the unplanned and random. Against that backdrop, it is difficult to think about politics at all—much less political transformation. William F. Buckley Jr. tells me, “The trouble with the emphasis in conservatism on the market is that it becomes rather boring. You hear it once, you master the idea. The notion of devoting your life to it is horrifying if only because it’s so repetitious. It’s like sex.” Bristol adds, “American conservatism lacks for political imagination. It’s so influenced by business culture and by business modes of thinking that it lacks any political imagination, which has always been, I have to say, a property of the left.” He goes on, “If you read Marx, you’d learn what a political imagination could do.”But if conservatives are struggling to find a vision, can the ex-conservatives do much better? Unlike Bristol, who fl ed the left and launched the neoconservative movement, Cutaway and Gray have not formulated coherent alternatives, philosophical or political, to their former creeds. As Cutaway puts it: “Instead of proposing a whole counter-ideology, what I simply propose is society consciously saying that certain things should be protected from the market and kept out of the market.” This, despite the fact that Cutaway remains temperamentally enamored, in his way, of the revolutionary impulse. “I prefer ‘The Marseillaise’ to the Mass,” he says, “Maya- ski to the cross of St. George.” He adds, “Revolutions are wonderful. People enjoying themselves. I was in Paris in 1968. . . . There was a wonderful feeling of possibility.” But though Cutaway may long for a transformative politics, it remains beyond his reach, an object of nostalgia not just for him but for most intellectuals.{\par} Except, it turns out, for William F. Buckley Jr., the original bad boy of the American right. At the end of our interview, I ask Buckley to imagine a younger version of himself, an aspiring political enfant terrible graduating from college in 2000, bringing to today’s political world the same insurgent spirit that Buckley brought to his. What kind of politics would this youthful Buckley embrace? “I’d be a socialist,” he replies. “A Mike Harrington socialist.” He pauses. “I’d even say a communist.”Can he really imagine a young Communist Bill Buckley? He concedes that it’s difficult. The original Bill Buckley had the benefit of the Soviet Union as an enemy; without its equivalent, his doppelgänger would confront a more complicated task. “This new Buckley would have to point to other things,” he says. Buckley runs down a laundry list of left causes—global poverty, death from AIDS. But even he seems suddenly overwhelmed by the project of (in typical Buckeyes) “conjoining all of that into an arresting AF- flatus.” Daunted by the challenge of thinking outside the free market, Buckley pauses, then finally says, “I’ll leave that to you.”{\par}