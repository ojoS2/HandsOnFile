\begin{ figure }
	\centering
	\\includegraphics[width=1.\textwidth]{ temp_files/images/UP_logo.png }
	\caption{Alexandra Kollontai (1872-1952): A socialist theorist of women’s emancipation and a strident proponent of sexual relations freed from all economic consider- ations. After the October Revolution, Kollontai became the People’s Commissar of Social Welfare and helped to found the Zhenotdel (Women’s Section). She oversaw a wide variety of legal reforms and public policies to help liberate work- ing women and to create the basis of a new communist sexual morality. But Russians were not ready for her vision of emancipation, and she was sent away to Norway to serve as the first Russian female ambassador (and only the third female ambassador in the world). Courtesy of U.S. Library of Congress.}
	\label{ }
\end{ figure }
 \par 
CAPITALISM BETWEEN THE > SHEETS: ON SEX (PART {\color{blue}1})
 \par 
 y best buddy from college, whom I will call Ken, = was an economics major who lost his life on September 11, 2001, in the North Tower of the World Trade Center. Over our thirteen-year friendship, we racked up long-distance phone bills and traversed continents and oceans to meet in person when a call wasn’t enough. We convened in Hong Kong after his divorce; I listened as he sobbed over his vodka martini in a bar called Rick’s Café. He took the photos at my wedding in 1998, and on Labor Day weekend of 2001, he flew out to Berkeley to feel my belly when I was seven months pregnant. He was gone before I gave birth.
 \par 
Born and raised outside the United States, Ken lived the American dream, starting on Wall Street in 1989 and currency trading his way up into the company of millionaires. Before his marriage, he enjoyed the successful love life of a wealthy New York bachelor. At some gin joint in Oakland, we once traded notes about what people needed for a healthy romantic relationship. I still have the paper on which Ken wrote, “Kristy say: physical, emotional, intellectual, spiritual. I say: nice legs with nice ankle, sad eyes,
 \par 
107
 \par 
108
 \par 
CAPITALISM BETWEEN THE SHEETS: ON SEX (PART I) nicely shaped 34C boobs + some brain.” I had been trying to argue that there were four kinds of connections between people and that the best romances were those in which you bonded in each of the four ways, but Ken insisted that he liked his women pretty and just smart enough to not be stupid. “I love bimbos!!!” he wrote. But when his gold-digging wife ditched him immediately after they earned their green cards, and then bilked him for a massive lump-sum alimony payment, Ken began to question his taste in women.
 \par 
“I never wanted to date professional women,” he told me on the phone after the sting of his divorce began to wear off, and he felt ready to start dating again. “They have too much going on in their lives, and they can’t be there for you when you want them to be. I went out with a lawyer once, and all she did was talk about her cases.”
 \par 
“You talk about your work,” I said. “IT know,” he said, “And I want my girlfriend to listen to me.”
 \par 
Ken drew a breath on the other side of the receiver. “But you know, I think I should try dating more smart women. I'm tired of the gold diggers.”
 \par 
“Really?” I said. “That would be out of character.” He went on to describe a recent epiphany. Ken explained that, as I suspected, he avoided intelligent and independent women because they made him feel less masculine, less in control of the relationship. But one of his coworkers had recently married an “impossibly hot” corporate lawyer. At the wedding reception after five too many glasses of wine, Ken watched the new couple dancing and decided that his coworker was actually manlier because he wasn’t intimidated by being with a successful woman. “I mean,
 \par 
KRISTEN R. GHODSEE think about it,” Ken told me. “It’s easy to get hot chicks if you want them. But it’s harder to get a hot and smart chick with her own money. And if she’s got her own cash, you know she’s not with you for yours.”
 \par 
He sighed. “I think they really love each other.” For Ken, attraction and love had always been tied to money and power. He used his wealth to attract women and reveled in the role of the alpha male. But what Ken discovered (rather late in his short life) was the idea that more egalitarian relationships create fewer opportunities for emotional subterfuge and resentment between partners. Ken had adored his ex-wife and assumed that she genuinely reciprocated his affections. She had certainly led him to believe so, ceding him all power in their short-lived marriage. After she dumped him for another man, Ken questioned whether his wife had ever loved him or had simply used him to immigrate to the United States. But what bothered him most was that he couldn’t tell the difference; she had played the part of the attentive and loving wife right up until the moment she filed for divorce. He never doubted her authenticity, and he feared repeating the same mistake in his next relationship. Unfortunately, he never had the chance; he was only in his mid-thirties when the World Trade towers came down.
 \par 
Because Ken was a college economics major and a full throttle capitalist, I know he would have loved a research paper published just three years after his death. In 2004, a controversial article—“Sexual Economics: Sex as Female Resource for Social Exchange in Heterosexual Interactions’— proposed that sex is something men purchase from women
 \par 
109
 \par 
110
 \par 
CAPITALISM BETWEEN THE SHEETS: ON SEX (PART |) with either monetary or nonmonetary resources, and that love and romance are mere cognitive veils humans use to occlude the transactional nature of our personal relationships. In their article, Roy Baumeister and Kathleen Vohs took a bold theoretical leap and applied the discipline of economics to the study of human sexuality. Their view precipitated a heated debate among psychologists about the “natural” behaviors of men and women in courtship.”
 \par 
Sexual economics theory, or sexual exchange theory, proposes that the early stages of sexual flirtation and seduction between men and women can be characterized as a market where women sell sex and men buy it with nonsexual resources. “Sexual economics theory rests on standard basic assumptions about economic marketplaces, such as the law of supply and demand. When demand exceeds supply, prices are high (favoring sellers, that is, women). In contrast, when supply exceeds demand, the price is low, favoring buyers (men).” The basic idea is that sex is a female controlled commodity, because, according to the authors, women’s sex drives are weaker than men’s. Because of the principle of least interest, and because women are less ruled by their sexual impulses, they have power in sexual relationships with men. They can demand compensation from men because men want the commodity (sex) more than women do, It is also in an attempt to keep the price of sex high that other women supposedly suppress the sexuality of their fellow sellers. Thus, Baumeister and Vohs argue that patriarchy is not responsible for slut-shaming. Rather, it is other women who want to punish those who sell their sex too cheaply and thereby reduce its overall price.’
 \par 
The authors are not talking about sex work, in which
 \par 
KRISTEN R. GHODSEE sex is exchanged directly for money (although they do use the prevalence of sex work as an example in support of their theory). So with what do men purchase the sexual services of women? Sexual economics theory proponents explain:
 \par 
\textit\textbf{ {A broad range of valued goods can be exchanged for sex.} }
 \par 
In return for sex, women can obtain love, commitment,
 \par 
\chapter{Respect, attention, protection, material favors, opportune-}\label{Respect, attention, protection, material favors, opportune-}
 \par 
\section{Nities, course grades or workplace promotions, as well}
 \par 
TTTTTT standard exchange has been that a man makes a long-
 \par 
Term commitment to supply the woman with resources
 \par 
(often the fruits of his labor) in exchange for sex—or,
 \par 
TTTTTT woman’s sexuality. Whether one approves of such ex-
 \par 
Changes or condemns them is beside the point. Rather, the key fact is that these opportunities exist almost exclusively for women. Men usually cannot trade sex for other benefits.’
 \par 
Sexual economics theory has been attacked by other psychologists as being based on the flawed assumption that women’s sex drives are weaker than men’s and that women have a “natural” desire to extract resources from men in exchange for sex. Feminists have also pointed to the deeply patriarchal and misogynistic assumptions embedded in sexual economics theory, since the price of sex also varies with the perceived desirability of the woman offering it (as determined by the male buyers). Others have criticized the economist thinking that reduces romance and mutual affection to an adversarial competition between men and
 \par 
111
 \par 
112
 \par 
CAPITALISM BETWEEN THE SHEETS: ON SEX (PART I) women in which each side is trying to get the best deal. While these critiques are important, sexual economics theory has won many followers because it seems intuitive, especially in the individualistic and materialistic culture of the United States.’
 \par 
In fact, some American right-wingers have embraced sexual economics theory as a way of blaming women for the current ills of our society. Indeed, a viral 2014 animated YouTube video from the conservative Austin Institute for the Study of Family and Culture extrapolated from the work of Baumeister and Vohs and blamed the falling marriage rate and the social maladjustment of young men in the United States on loose women who have made the price of sex too low. In their worldview, the availability of birth control (and one presumes abortion) has reduced the risks associated with sex, since it is now less likely to result in an unwanted pregnancy that must be carried to term. When sex entailed the risk of parenthood, they argue, women extracted a much higher price for access to their bodies, at minimum a serious commitment and ideally marriage. But once birth control reduced the risk of pregnancy, women could do with their bodies as they liked, and the price they demanded for sex fell, particularly since they had other opportunities to earn money.
 \par 

 \par 
Is this a terrible thing? Other than the falling marriage rate (the old “Why buy the cow when you can get the milk for free?” argument), a low price for sex harms men who, according to these theories, apparently have no incentive to do anything with their lives other than the pursuit of sex. This is not a joke, I assure you. According to the ideologues over at the Austin Institute, young men these days
 \par 
KRISTEN R. GHODSEE are camping out in their parent’s basements, playing video games, and subsisting on Domino’s pizza because cheap sex is just a text away. When women have no birth control, the price of sex is higher. When women have no access to abortion, the price is higher still. When women have fewer educational or economic opportunities outside their relationships with men, the price for sex is usually marriage. When the price of sex is very high, according to this worldview, sex-starved men have incentives to go out and get jobs, earn money, and make something of their lives so they can buy access to a woman’s sexuality for life through marriage. In cultures with more men than women, for instance, economists have shown that there is a higher rate of male entrepreneurship. When the price of sex is too low, however, men have no intrinsic incentive to do anything productive.* To be fair, the original authors of sexual economics theory don’t suggest any normative changes to our society; they are just observers, gathering evidence for their theoretical model. They also recognize that sexual marketplaces are embedded in specific cultural contexts that influence the supply and demand for sex. To support their claims, the proponents of sexual economics theory posit that women’s status in society is one important factor affecting the underlying operation of the marketplace for sex. They note, for instance, that women’s emancipation reduces the price of sex because educational opportunities and paid employment give women other avenues to provide for their basic needs. Their model predicts that the price of sex is higher in more traditional societies, where women are shut out of political and economic life.
 \par 
\section{As money. Throughout the history of civilization, one}
 \par 
113
 \par 
114
 \par 
CAPITALISM BETWEEN THE SHEETS: ON SEX (PART {\color{blue}1})
 \par 
Mendoza correlated the results of a global sex survey with an independent measure of gender inequality to show that economic opportunity for women results in freer sex. They found that in countries where men and women are more equal there was “more casual sex, more sexual partners per capita, younger ages for first sex, and greater tolerance/ approval for premarital sex.” Thus, the authors argue that women’s economic independence often accompanies a loosening of social mores around sexuality. “According to sexual economics theory,” Baumeister and Mendoza explain, “when women lack direct or easy access to resources such as political influence, health care, money, education, and jobs then sex becomes a crucial means by which women can gain access to a good life, and so it is vital to female self-interest to keep the price of sex high.” Women do this by reducing the supply (no more casual sex), which drives the price up. It’s according to a similar logic that, for a certain group of extreme social conservatives, the only way to “Make America Great Again” is to abolish birth control and abortion while ensuring that women have few economic opportunities to pay for basic goods outside of selling their sex. When their sexuality is their only means of survival, they will supposedly raise its price and thereby save an entire generation of men from a life of sloth’
 \par 
=~
 \par 
Sexual economics theory assumes an underlying capitalist economy in which women have an asset (sex) they can choose to sell or give away either as sex workers or in less overt, but no less transactional ways, as sugar babies, girlfriends, or wives. In order to meet their basic needs (food, shelter, health care, education), they must either sell their sex or earn money to pay for these resources another way.
 \par 
KRISTEN R. GHODSEE
 \par 
The more opportunities they have to earn money (i.e., in societies with high levels of gender equality), the less reliant they are on selling their sex, and the more likely they are to have sex for pleasure. Similarly, one would also assume that women living in a society that provides its citizens with subsidized access to basic needs such as food, shelter, health care, and education would have fewer incentives to horde their sex in order to keep its price high. In other words, in societies with high levels of gender equality, with strong protections for reproductive freedom, and with large social safety nets, women almost never have to worry about the price their sex will fetch on the open market. Under these circumstances, the sexual economics theory model would predict that women’s sexuality would cease to be a salable commodity at all.
 \par 
As someone who is often critical of reductionist economist models, I am fascinated by sexual economics theory and think the model gives valuable insight into the way sexuality is experienced in capitalist societies. Essentially, sexual economics theory is right, but only within the confines of the free market system. In fact, a beautiful confluence emerges when you read the works of Baumeister and his colleagues alongside socialist critiques of capitalist sexuality. Although they may not realize it, sexual economics theorists basically embrace a long-standing socialist critique of capitalism: that it commodifies all human interactions and reduces women to chattel. Back in 1848, Karl Marx and Friedrich Engels observed that capitalism
 \par 
\textit\textbf{ {Has left remaining no other nexus between man and man than naked self-interest, then callous “cash payment.”} }
 \par 
115
 \par 
116
 \par 
CAPITALISM BETWEEN THE SHEETS: ON SEX (PART {\color{blue}1})
 \par 
\textit\textbf{ {It has drowned the most heavenly ecstasies of religious} }
 \par 
\section{Often more precisely, for exclusive sexual access to that}
 \par 
TTTTTT has resolved personal worth into exchange value, and in place of the numberless indefeasible chartered freedoms,
 \par 
Has set up that single, unconscionable freedom—Free
 \par 
Trade. . . The bourgeoisie has torn away from the family its sentimental veil, and has reduced the family relation
 \par 
\section{To prove this point, Roy Baumeister and Juan Pablo}
 \par 
As far back as the era of utopian socialism in the 1830s, theorists argued that postcapitalist societies would generate a new form of sexual morality. In his 1879 book, Woman and Socialism, August Bebel wrote that sexual desire was natural and healthy, and that women needed to be freed from the then socially accepted property relations that distorted and suppressed their sexuality in order to render it scarce:
 \par 
SSSSSS call independent, she is no longer subjected to even a
 \par 
KRISTEN R. GHODSEE
 \par 
\textit\textbf{ {The woman of the future society is socially and economy-} }
 \par 
\section{Fervour, of chivalrous enthusiasm, of Philistine sent-}
 \par 
TTTTTT ing the object of her love, woman, like man, is free and
 \par 
Unhampered. She woos or is wooed, and enters into a
 \par 
\section{Mentalism, in the icy water of egotistical calculation. It}
 \par 
\section{To a mere money relation.}
 \par 
TTTTTT instincts inflicts no injury and disadvantage on others,
 \par 
The individual shall see to his own needs. The gratification of the sexual instinct is as much a private concern
 \par 
As the satisfaction of any other natural instinct. No one is accountable for it to others and no unsolicited judge has the right to interfere. What I shall eat, how I shall
 \par 
Drink, sleep and dress, is my own affair, as is also my
 \par 
\section{A par with man and mistress of her destiny. .}
 \par 
\section{. In chews-}
 \par 
Reading these words in the twenty-first century, it’s hard to understand how radical they would have sounded in the late nineteenth, when his book was first published. Bebel truly believed that sexuality was a private concern (and has been celebrated by modern LGBT rights activists as the first politician to publicly defend the rights of gay people in 1898). Friedrich Engels also argued, in 1884, that women’s subjugation resulted from the male desire for legitimate heirs to inherit his wealth. To ensure that his children were really his, the man needed to control women’s sexuality through the institution of monogamous marriage. Women’s fidelity and reproductive capacity thereby became commodities to be exchanged between men for the purpose of projecting their accumulated wealth and power onto future generations of their descendants. But monogamy was
 \par 
117
 \par 
118
 \par 
CAPITALISM BETWEEN THE SHEETS: ON SEX (PART {\color{blue}1}) primarily monogamy for the woman, since men could have sexual relations outside of marriage with impunity, and the marriage contract deprived most women not only of control of their bodies but also of their fundamental rights as individuals. Marriage reduced women to the status of property of their husbands."°
 \par 
\begin{ figure }
	\centering
	\\includegraphics[width=1.\textwidth]{ temp_files/images/UP_logo.png }
	\caption{Alexandra Kollontai rebelled against this continued commodification of women. Born into a family of Russian nobility in 1872, she showed a deep empathy for the atro- cious conditions of Russia’s working classes from an early age and was slowly drawn into political work, which often landed her in trouble with the tsarist authorities. From ob- serving the situation of women in her own class, Kollontai grew to abhor the exchange of women’s sexuality for money, goods, services, or social position. As a child, she watched her mother push her twenty-year-old sister into marrying a man fifty-one years her senior because he was considered a “good match.” Kollontai rejected marriages of convenience and wanted to marry for love, for what she called a “great passion.” She wrote, “As regards sexual relations, commu- nist morality demands first of all an end to all relations based on financial or other economic considerations. The buying and selling of caresses destroys the sense of equality between the sexes, and thus undermines the basis of sol- idarity without which communist society cannot exist.”"}
	\label{ }
\end{ figure }
 \par 
In 1894, she read August Bebel’s Woman and Socialism, and it provided the basis for her own views on a new form of progressive morality. Like Bebel, she believed that sexuality needed to be liberated from social stigmatization: “The sexual act must be seen not as something shameful and sinful but as something which is as natural as the other needs
 \par 
KRISTEN R. GHODSEE of [a] healthy organism, such as hunger and thirst. Such phenomena cannot be judged as moral or immoral.” Kollontai argued that only under socialism would people love and have sex with each other as free individuals, based on their mutual attraction and affection and without regard for money or social position. But it is important to realize that Kollontai was never arguing for unbridled promiscuity or a form of “free love” in the sole pursuit of hedonistic pleasure. Instead, she believed that by destroying the link between property and sexuality, men and women would have more authentic and meaningful relationships. Although she has subsequently been characterized as a sexual libertine, she was relatively conservative (by modern standards) in her views, advocating for sexual fulfillment only within heterosexual relationships based on love.’
 \par 
Kollontai considered sex for pleasure as a bourgeois distraction from the necessary work of the revolution, contrasting the “wingless Eros” of pure physical sex with her idealized “winged Eros” of emotional and even spiritual connection. This romanticized love between men and women was supposed to contribute to the generalized love of humanity that underpinned the basis of socialist ideology (Kollontai might actually be the original hippie). In her 1921 pamphlet, Theses on Communist Morality in the Sphere of Marital Relations, Kollontai wrote, “The bourgeois attitude to sexual relations as simply a matter of sex must be criticized and replaced by an understanding of the whole gamut of joyful love-experience that enriches life and makes for greater happiness. The greater the intellectual and emotional development of the individual the less place will there be in his or her relationship for the bare
 \par 
119
 \par 
120
 \par 
CAPITALISM BETWEEN THE SHEETS: ON SEX (PART !) physiological side of love, and the brighter will be the love experience.”
 \par 
Kollontai viewed marriage as an institution that perpetuated the subjugation of women, and it was this institution that she attempted to dismantle in the first years after the 1917 October Revolution in Russia. She and a small cadre of radical jurists tried to challenge the traditional basis of matrimony by replacing church marriages with civil ceremonies, liberalizing divorce laws, legalizing abortion, decriminalizing homosexuality, equalizing rights for legitimate and illegitimate children, and mobilizing women into the labor force, while socializing domestic work through the establishment of public laundries, cafeterias, and children’s homes. But as discussed earlier, Lenin and the other male Bolsheviks had concerns that they considered more pressing than the woman question, and Kollontai was eventually dispatched as a diplomat to Norway (to get her out of the country). Reflecting back on her life in 1926, Kollontai wrote, “No matter what further tasks I shall be carrying out, it is perfectly clear to me that the complete liberation of the working woman and the creation of the foundation of a new sexual morality will always remain the highest aim of my activity, and of my life.”
 \par 
Kollontai’s vision of a sexuality free from economic consideration was shared by many Soviet youths in the
 \par 
1920s. For example, a 1922 survey of 1,552 students at the Sverdlov Communist University in Moscow found that only {\color{blue}21} percent of men and {\color{blue}14} percent of women considered marriage as the ideal way to organize one’s sex life. In contrast, a full two-thirds of the women and one half of the men preferred a long-term relationship based on love.
 \par 
KRISTEN R. GHODSEE
 \par 
But these liberal attitudes did not extend to the rest of the population. The traditional conservatives of Russian peasant culture, combined with the expert advice of a prudish medical establishment, conspired to subvert Kollontai’s attempts at social reform. Without access to reliable birth control, women could not control their fertility, and men who declared their undying love disappeared once a child was on the way. The courts attempted to enforce alimony payments, but men evaded their responsibilities. Women’s wages were not high enough to support children, and many turned to sex work to survive, precisely the type of economic exchange Kollontai had hoped to eradicate. The Soviet state attempted to create a network of orphanages to care for homeless children, but the whole project was too costly. Kollontai made one last attempt to replace alimony with a general insurance fund that would allow the state to support all children, but her ideas were ridiculed and rejected. By the mid-1920s, hundreds of thousands of red orphans roamed the streets of Soviet Russia, begging, stealing, and embodying the failures of a premature attempt at sexual revolution.”
 \par 
Stalin, who ascended to dictatorial power at the end of the 1920s, decided it was much easier to return to a system in which women did all the childbearing and child-rearing for free within the confines of more traditional forms of marriage, while also forcing them to work outside the home to help build Soviet industrial power. Many social conservatives in the United States would find much to love in Josef Stalin’s policies: he outlawed abortion again, promoted premarital abstinence, repressed public discussions of sexuality, persecuted gay people, and emphasized traditional
 \par 
121
 \par 
122
 \par 
CAPITALISM BETWEEN THE SHEETS: ON SEX (PART !) gender roles in heterosexual, monogamous marriage. Even after Stalin’s death, when the abortion law was once again liberalized, most studies confirm that public discourse around sexuality in the USSR was nonexistent. Before this, most Soviet women viewed sex as a marital duty for the sole purpose of procreation, and Soviet society was decidedly prudish. Kollontai died in 1952, long before her vision of a Soviet sexuality based on love and mutual affection had a chance to develop.'®
 \par 
Yet Kollontai’s conception of a society in which sexuality is free from economic constraint has continued to inspire feminist thought since the early twentieth century. Between her socialist vision of a sexuality based on mutual affection and the vision proposed by sexual economics theory, we have two competing views of how to organize heterosexual sexuality. One view celebrates women’s economic independence as a prerequisite for a more authentic form of love, and the other view sees women’s economic independence as just one factor affecting the relative price of sex within a marketplace wherein sex is a commodity to be bought by men. Although certainly a wide variety of positions exist between these two models, for the sake of argument, let’s focus on these two views as poles on a spectrum of possible models for heterosexual relationships. Which would be better?
 \par 
Clearly, there is no easy answer. Human sexuality is complex and rather difficult to study, making any kind of normative judgment about sex fraught with problems. But setting aside the people who would choose sex work without
 \par 
KRISTEN R. GHODSEE economic necessity, I’m going to go out on a limb here and suggest that sex is not as great when you are forced to sell it to pay your rent. Furthermore, if a man feels he is paying a woman to access her body, why would he care about her pleasure? He believes she is being compensated for the activity in nonsexual ways. If he hired a woman to clean his home, would he care how much she enjoyed it? Should he be expected to? On the other hand, two people—freely exchanging their affections without any thought of what else they might get out of it—are probably a lot more attentive to each other’s needs than those who are consciously or subconsciously worried about the economic nature of the exchange. But how can we know?
 \par 
We don’t have to limit ourselves to speculation. Here’s where the experiences of state socialism in Eastern Europe provide an interesting natural experiment to augment our understanding of the effects of political economy on heterosexual courtship. Despite their shortcomings, as we've seen, the countries on the other side of the Iron Curtain did implement a wide range of policies to promote women’s economic independence (albeit with much variation across the region), which would have caused the price of sex to fall, according to sexual economics theory. Is there evidence that women and men began to view female sexuality as something to be shared rather than exchanged for resources? We're intimate relations experienced differently in capitalist versus socialist countries? And what happened after the fall of the Berlin Wall? Did the sexual marketplaces describe by Baumeister and Vohs return with the privatization and marketization of the postsocialist economy?
 \par 
All studies of what is called “subjective well-being’—or
 \par 
123
 \par 
124
 \par 
CAPITALISM BETWEEN THE SHEETS: ON SEX (PART I) people’s own self-reported feelings of happiness or sexual satisfaction—share the problem that people’s emotional states are difficult to research in an objective fashion. When you study something like cancer, a doctor can examine a human body and empirically determine the presence or absence of cancer cells. But when doctors study pain, they have to rely on the patient’s own account of how much something hurts. But people vary in how they report pain. Doctors often use a one-to-ten scale to measure pain. This is not an absolute scale but one relative to the patient’s own pain threshold. When you are in the hospital, for instance, the doctors and nurses will continuously ask you to rank your pain to get a sense of your individual scale and try to extrapolate from that how much and what kind of medicine you require. Pain objectively exists, and someone with a broken femur should feel more pain than someone with an ingrown toenail, even if the person with the ingrown toenail wails louder than the person with the fractured leg. We know this by aggregating the self-reported levels of pain from all patients suffering from these two conditions and comparing the averages.
 \par 
Feelings of happiness and sexual satisfaction are more like pain than they are like cancer in this respect. Psychologists, sexologists, and other researchers identify representative samples of defined populations and then ask individual questions about their emotional states or their feelings about certain experiences. The choice of questions, the way they are asked, and the form and sequence in which answers are expected are all important aspects of studies of subjective well-being. In well-designed studies, researchers ask different formulations of the same questions multiple
 \par 
KRISTEN R. GHODSEE times to control for various kinds of misunderstanding or bias. In theory, if the number of people sampled is large enough, certain patterns emerge, and generalizable claims can be made (at least within a given cultural milieu).
 \par 
It turns out that contemporary historians, anthropologists, and sociologists have taken a great interest in whether noncapitalist sexuality had a different character than the sorts of intimate relations people had (and have) in the market economies of the West. In searching for sources, they have discovered a range of studies conducted before and after 1989 that suggest that there were some fascinating differences in the way people experienced their sexuality behind the Iron Curtain, results I will discuss in the next chapter. Because state socialist scientists were concerned with falling birthrates, they primarily focused on heterosexual relations between men and women, but many of their insights into the damage that market exchanges can do to human relationships are relevant to people of all sexualities. Again, the key here is not to glorify or suggest that we return to the state socialist past. Instead, we can better understand how capitalism affects our most intimate experiences by looking to societies in which market forces had less of an impact. If sexual economics theory describes the way that the capitalist system reduces our affections and attentions to the status of salable goods, what policy levers might we have to push back against the operations of the unfettered free market? Perhaps we can find ways to have more fulfilling private lives in a society that also guarantees individual freedoms and a robust public sphere, undermining the operations of sexual economics theory without embracing authoritarianism.
 \par 
125
 \par 
\begin{ figure }
	\centering
	\\includegraphics[width=1.\textwidth]{ temp_files/images/UP_logo.png }
	\caption{Inessa Armand (1874-1920): Born in Paris, Armand was a French-Russian Bol- shevik and feminist who was a key figure in the prerevolutionary communist movement. After 1917, she served as the head of the Moscow Economic Council, sat as an executive member of the Moscow Soviet, and headed the Zhenotdel, leading efforts to ensure sexual equality and to socialize domestic work. She helped organize children’s homes, mass cafeterias, and public laundries until her untimely death from cholera at the age of forty-six. Courtesy of Sputnik.}
	\label{ }
\end{ figure }