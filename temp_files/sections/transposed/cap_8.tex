{\chapter{A Colorcoded Genocide} } {\label{A Colorcoded Genocide} }{\par}{\textit{	} } {\par}{\par} {\textbf{\textit{	} } } {\par} 
	On December 5, 1982, Ronald Reagan met Guatemalan president Efraín Ríos Montt in Honduras. It was a useful meeting for Reagan. “Well, I learned a lot,” he told reporters on Air Force One. “You’d be surprised. They’re all individual countries.” It was also a useful meeting for Ríos Month. Reagan declared him “a man of great personal integrity. . . Totally dedicated to democracy.” He also claimed that the Guatemalan strongman was getting “a bum rap” from human rights organizations for his military’s campaign against leftist guerrillas. The next day, Daniel Wilkinson tells us in Silence on the Mountain: Stories of Terror, Betrayal, and Forgetting in Guatemala, one of Guatemala’s elite platoons entered a jungle village called Las Dos Erred and killed {\color{blue} 162 } of its inhabitants, {\color{blue} 67 } of them children. Soldiers “grabbed” babies and toddlers by their legs, swung them in the air, and “smashed” their heads “against a wall.” Older children and adults were forced to “kneel at the edge of a well,” where a single “blow from a sledgehammer” sent them plummeting below. The platoon then raped a selection of women and girls it had “saved for last,” pummeling their stomachs in order to force the pregnant among them to miscarry. They tossed the women into the well and filled it with dirt, burying an unlucky few alive. “The only human remains that [later] visitors would find” were “blood on the walls and placentas and umbilical cords on the ground.” {\color{blue} 1 } {\par} Amid the hagiography surrounding Reagan’s death in 2004, it was probably too much to expect the media to mention his meeting with Ríos Month. After all, it wasn’t Reykjavík. But Reykjavík’s shadow—or that cast by Reagan speaking in front of the Berlin Wall—does not entirely explain the silence about this encounter between presidents. While it is tempting to ascribe the omission to American amnesia, a more likely cause is the deep misconception about the Cold War under which most Americans labor. To the casual observer, the Cold War was a struggle between the United States and the Soviet Union, fought and won through stylish jousting at Berlin, antiseptic arguments over nuclear stockpiles, and the savvy brinkmanship of American leaders. Latin America seldom figures in popular or even academic discussion of the Cold War; and to the extent that it does, it is Cuba, Chile, and Nicaragua rather than Guatemala that earn most of the attention.{\par} But Latin America was as much a battleground of the Cold War as Europe, and Guatemala was its front line. In 1954, the United States fought its first major contest against Communism in the Western hemisphere when it overthrew Guatemala’s democratically elected president, Jacobo Arena, who had worked closely with the country’s small but influential Communist Party. That coup sent a young Argentinean doctor fleeing to Mexico, where he met Fidel Castro. Five years later, Che Guevara declared that 1954 had taught him the impossibility of peaceful, electoral reform. He promised his followers that “Cuba will not be Guatemala.” In 1966, Guatemala was again the pacesetter, this time pioneering the disappearances that would come to defi né the dirty wars of Argentina,Uruguay, Chile, and Brazil. In a lightning strike, U.S.-trained security officials captured some thirty leftists, tortured and executed them, and then dropped most of their corpses into the Pacific. Explaining the operation in a classified memo, the CIA wrote: “The execution of these persons will not be announced and the Guatemalan government will deny that they were ever taken into custody.” With the 1996 signing of a peace accord between the Guatemalan military and leftist guerrillas, the Latin American Cold War finally came to an end—in the same place it had begun— making the civil war in Guatemala the longest and most lethal in the hemisphere. Some 200,000 men, women, and children were dead, virtually all at the hands of the military: more than were killed in Argentina, Uruguay, Chile, Brazil, Nicaragua, and El Salvador combined, and roughly the same number as were killed in the Balkans. Because the victims were primarily Mayan Indians, Guatemala today has the only military in Latin America deemed by a United Nations–sponsored truth commission to have committed acts of genocide. {\color{blue} 2 } {\par} When we talk about America’s victory in the Cold War, we are talking about countries like Guatemala, where Communism was fought and defeated by means of the mass slaughter of civilians. But understanding the Cold War requires more than tallying body counts and itemizing atrocities. It requires us to locate this most global of contests in the smallest of places, to find beneath the dueling composure of superpower rivalry a bloody conflict over rights and inequality, to see behind a simple morality tale of good triumphing over evil the more ambivalent settlement that was—and is—the end of the Cold War. The task, in short, is to show how men and women made high politics and high politics made them, to show that the Cold War was waged not only in the airy game-rooms of nuclear strategists but, as Greg Gran din writes in The Last Colonial Massacre, “in the closed quarters of family, sex and community.” {\color{blue} 3 } {\par} Gran din opens his study with an epigraph from Sartre: “A victory described in detail is indistinguishable from a defeat.” {\color{blue} 4 } The victory referred to here is singular and by now virtually complete: that of the United States over Communism. But the defeats are various, their consequences still unfolding. First is the defeat of the Latin American left, whose aspirations ranged from the familiar (armed seizure of state power) to the surprising (the creation of capitalism). Next is the defeat of a continental social democracy that would have allowed citizens to exercise a greater share of power— and to receive a greater share of its benefits—than historically had been their due. Finally, and most important, is the defeat of that still-elusive dream of men and women freeing themselves, thanks to their own reason and willed effort, from the bonds of tradition and oppression. This had been the dream of the transatlantic Enlightenment, and throughout the Cold War, American leaders argued on its behalf (or some version of it) in the struggle against Communism. But in Latin America, it was the left who took up the Enlightenment’s banner, leaving the United States and its allies' carrying the black bag of the counterenlightenment. More than foisting on the United States the unwanted burden of liberal hypocrisy, the Cold War inspired it to embrace some of the most reactionary ideals and revanchist characters of the twentieth century.{\par} The Latin American left brought liberalism and progress to a land awash in feudalism. Well into the twentieth century, Guatemala’s coffee planters presided over a regime of forced labor that was every bit as medieval as tsarist Russia. Using vagrancy laws and the lure of easy credit, the planters amassed vast estates and a workforce of peasants who essentially belonged to them. Reading like an excerpt from Gogol’s Dead Souls, one advertisement from 1922 announced the sale of “5000 acres and many bozos colons [indebted workers] who will travel to work on other plantations.” While unionized workers elsewhere were itemizing what their employers could and could not ask of them, Guatemala’s peasants were forced to provide a variety of compulsory services, including sex. Two planters in the Alta Verbal region, cousins from Boston, used their Indian cooks and corn grinders to sire more than a dozen children. “They fucked anything that moved,” a neigh-boring planter observed. Though plantations were mini-states— with private jails, stockades, and whipping posts—planters also depended on the army, judges, mayors, and local constables to force workers to submit to their will. Public officials routinely rounded up independent or runaway peasants, shipping them off to plantations or forcing them to build roads. One mayor had local vagrants paint his house. As much as anything, it is this view of political power as a form of private property that confirms Gran-din’s observation that by 1944 “only five Latin America countries— Mexico, Uruguay, Chile, Costa Rica and Colombia—could nominally call themselves democracies.” {\color{blue} 5 } {\par} And then, within two years, it all changed. By 1946, “only five countries—Paraguay, El Salvador, Honduras, Nicaragua and the Dominican Republic—could not” be called democracies. Turning the antifascist rhetoric of World War II against the hemisphere’s old regimes, leftists overthrew dictators, legalized political parties, built unions, and extended the franchise. Galvanized by the New Deal and the Popular Front, reformers liked Guatemalan president Juan José Arévalo declared that “we are socialists because we live in the 20th century.” The entire continent was phi red by a combination of Karl Marx, the Declaration of Independence, and Walt Whitman, but Guatemala burned the brightest. There, a decades-long struggle to break the back of the coffee aristocracy culminated in the 1950 election of Arena, who with the help of a small circle of Communist advisers, instituted the Agrarian Reform of 1952. The legislation redistributed a million and a half acres to a hundred thousand families and also gave peasants a signifi can't share of political power.{\par} Local land reform committees, made up primarily of peasant representatives, bypassed the planter-dominated municipal government and provided peasants and their unions with a platform from which to make and win their claims for equity. {\color{blue} 6 } {\par} Arguably the most audacious experiment in direct democracy the continent had ever seen, the Agrarian Reform entailed a central irony. The legislation’s authors—most of them Communists— were not building socialism. They were creating capitalism. They were scrupulous about property rights and the rule of law. Peas-ants had to back their claims with extensive documentation; only unused land was expropriated; and planters were guaranteed multiple rights of appeal, all the way to the president. The Agrarian Reform imposed a regime of separated powers that was almost as cumbersome as James Madison’s Constitution. (According to one of the bill’s Communist authors, “it was a bourgeois law.” When grassroots activists complained about the slowness of reform, Arena responded: “I don’t care! You have to do things right!”) The Agrarian Reform turned landless peasants into property owners, giving them the bargaining power to demand higher wages from their employers. According to Gran din, reformers hoped that the peasants would become “consumers of national manufactures,” while “planters, historically addicted to cheap, often free labor and land,” would be forced to “invest in new technologies” and thereby “make a profit.” {\color{blue} 7 } {\par} Guatemala’s socialists did more than create democrats and capitalists. They also made peasants into citizens. While liberals and conservatives have long claimed that leftist ideologies reduce their adherents to automatons, leftist ideals and movements awakened peasants to their own power, giving them extensive opportunities to speak for themselves and to act on their own behalf. Efraín Reyes AAZ, for example, was a Mayan peasant organizer, born in the same year as the Bolshevik Revolution. “If I hadn’t studied marx I would be chichi NI lemonade [neither alcohol nor lemonade],” Reyes says. “I’d be nothing. But reading nourished me and here I am. I could die today and nobody could take that from me.” Where other peasants seldom ventured beyond their plantations, the Communist Party inspired Reyes to travel to Mexico and Cuba, and he returned to Guatemala with the conviction that “every revolutionary carries around an entire world in his head.” The Communist Party did not require Reyes to give up everything he knew; it gave him ample freedom to synchronize the indigenous and the European, making for a “Mayan Marxism” that was every bit as supple as the hybrid Marxism developed in Central Europe between the wars. When anticommunists put an end to this democratic awakening in 1954, it was as much the peasant’s newfound appetite for thinking and talking as the planter’s expropriated land that they were worried about. As we saw in the introduction, Guatemala’s archbishop complained that the Arbencistas sent peasants “gifted with facility with words” to the nation’s capital, where they were “taught. . . To speak in public.” {\color{blue} 8 } {\par} Hoping to break this army of thought and talk, Guatemala’s Cold Warriors fused a romantic aversion to the modern world with the most up-to-date technologies of propaganda and violence, making their effort more akin to fascism than to any fight for liberal democracy. Working through the Catholic Church, the regime that replaced Arena had prelates preach the gospel against Communism and socialism, and also against democracy, liberalism, and feminism. Reaching back to the rhetoric of opposition to the French Revolution, the Church fathers characterized the Cold War as a struggle between the City of God and “the city of the devil incarnate” and complained that Arena, “far from uniting our people in their advance toward progress,” “disorganizes them into opposing bands.” The Arbencistas, they claimed, were “professional corrupters of the feminine soul,” elevating women with “gifts of proselytism or leadership” to “high and well-paid positions in off - coal bureaucracy.” Because the Church elders were sometimes too fastidious to whip up the masses, émigrés from Republican Spain, who were partial to Franco and Mussolini, frequently took their place, calling for a more ecstatic faith to counter Communism’s appeal: “We do not want a cold Catholicism. We want holiness, ardent, great and joyous holiness. . . Intransigent and fanatical.” {\color{blue} 9 } {\par} While the Cold Warriors’ ideals looked backward, their weapons—furnished by the United States—and military strategies looked forward. (Indeed, one of the Americans’ chief justifications for their interventions during the Cold War was that U.S. involvement would contain not only Communism but also, in the words of the State Department, a right-wing “counter-insurgency running wild.” Instead of a savage “white terror,” U.S.-trained security forces would work with the anticommunist “democratic left” to fight a more “rational,” “modern,” and “professional” Cold War.) During the 1954 coup, the CIA turned to Madison Avenue, pop sociologizes, and the literature of mass psychology to create the illusion of large-scale opposition to Arena. Radio shows spread rumors of an underground resistance, inciting wobbly army officers to abandon their oath to the democratically elected president. In subsequent decades, the CIA outfitted Guatemala with a centralized domestic intelligence agency, equipped with phones, radios, cameras, typewriters, carbon paper, fi long cabinets, surveil-lance equipment—and guns, ammunition, and explosives. The CIA also brought together the military and the police in sleek urban command centers, where intelligence could be quickly analyzed, distributed, acted on, and archived for later use. After these efforts achieved their most spectacular results, with the 1966 disappearance of Guatemala’s last generation of peaceful leftists, guerrillas began seriously to organize armed opposition in rural areas.{\par} In response, the regime threw into the countryside an army so modernized—and so well-trained and equipped by the United States—that by 1981 it was able to conduct the first color-coded genocide in history: “Military analysts marked communities and regions according to colors. White spared those thought to have no rebel influence. Pink identified areas in which the insurgents had limited presence; suspected guerrillas and their supporters were to be killed, but the communities left standing. Red gave no quarter: all were to be executed and villages razed.” {\color{blue} 10 } {\par} Referring to a 1978 military massacre of Indians in Panos, a river town in the Polo chic Valley, the title of Gran din’s book evokes this mixture of modern and antimodern elements. On May {\color{blue} 29 } of that year, roughly five hundred Mayan peasants assembled in the town center to ask the mayor to hear their complaints against local planters, which were to be presented by a union delegation from the capital. Firing on the protesters, a military detachment killed some-where between {\color{blue} 34 } and {\color{blue} 100 } men, women, and children. At first glance, the massacre seems like nothing so much as a repetition of Guatemala’s colonial past: humble Indian petitioners ask public officials to intercede on their behalf against local rulers; government forces in league with the planters respond with violence; Indians wind up floating down the river. On closer inspection, the massacre bears all the marks of the twentieth century. The Indians were led by leftist activists—one of them an indigenous woman—trained by clandestine Communist organizers. They worked with unions, based in the capital, reflecting the left’s attempt to nationalize local grievances. For their part, the soldiers phi ring on the peasants were more than a local constabulary defending the interests of the planters. They were a contingent of Guatemala’s newly trained army, spoke fluent anticommunism, and wielded Israeli-made Gall assault rifles, suggesting not just the nationalization but the internationalization of Guatemala’s traditional struggles over land and labor. {\color{blue} 11 } {\par} Though the Cold War in Latin America began as a tense negotiation between American rationalism and Latin revanchism, it ended with the United States careening toward the latter. In a rerun of the fabled journey into the heart of darkness, U.S. officials returned from their travels south echoing the darkest voices of the counterenlightenment. One embassy officer wrote to his superiors back home: “After all hasn’t manned been a savage from the beginning of time so let us not be too queasy about terror. I have literally heard these arguments from our people.” A CIA staff her urged his colleagues to abandon all attempts at mass persuasion in Guatemala and instead direct their efforts at the “heart, the stomach and the liver (fear).” Seeking to destabilize Allende’s Chile, another CIA man proclaimed: “We cannot endeavor to ignite the world if Chile itself is a placid lake. The fuel for the fire must come from within Chile. Therefore, the station should employ every stratagem, every ploy, however bizarre, to create this internal resistance.” As Gran din writes, “Will to set the world ablaze. . . Faith in the night-side of the soul, contempt for democratic temperance and parliamentary procedure: these qualities are usually attributed to opponents of liberal civility, tolerance and pluralism—not their defenders.” {\color{blue} 12 } With this plangent remark, Gran din concludes his remarkable tale, suggesting that the greatest defeat of the Cold War could be said to be that of America itself.{\par}