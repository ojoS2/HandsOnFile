TTTTTT n high school, I was a certifiable Model United Nations ; (MUN) nerd. MUN is a kind of debate club in which students do research on the foreign policies of UN member nations and then represent those countries in mock sessions with fabricated political scenarios based on real current events. To excel at MUN, you needed to learn the ins and outs of international relations as well as understand the social, political, and economic contexts that informed the foreign policy decisions of different countries around the globe. The highest prize of a MUN competition was the gavel, awarded to the student who represented his or her country most convincingly in a mock session. Generally, the most prestigious gavels were awarded to members of the mock Security Council, the most powerful of the UN committees. Students worked their way up from the lesser committees like the General Assembly or the Economic and Social Council (ECOSOC) until they were ready to serve on the Security Council, where the brightest and most informed students discussed and decided the fate of the globe.
 \par 
To increase your chances of winning a gavel, you wanted to represent one of the five permanent members of
 \par 
77
 \par 
78
 \par 
PANTSUITS ARE NOT ENOUGH: ON LEADERSHIP the Security Council—the United States, France, the United Kingdom, the USSR, or China, the only countries with veto power. If you have veto power, you cannot be outvoted, and all other delegates need to secure your support for their resolution or at least guarantee your abstention. In a big competition, your school would be very lucky if it got assigned an allotment of countries that included one of the Big Five. But as a girl, I knew that the boys in my club wouldn't let me represent the United States, the United Kingdom, and France. This was still more than a decade before Madeleine Albright became the first American female Secretary of State, and the boys argued that it was less likely for one of the Western countries to have a woman as a Security Council representative. Even in the era of Margaret Thatcher and Jeane Kirkpatrick, men still dominated foreign affairs.
 \par 
It was, however, more plausible for China and the USSR. Zoya Mironova had been a Soviet deputy security council representative from 1959 to 1962 and served as the Ambassador to the UN in Geneva from 1966 to 1983. I didn’t want to represent China because they abstained on everything. So I became the Eastern Bloc specialist, hoping that one day if we got the USSR, the Security Council seat would be mine. The lesson I learned at fifteen was that while it was less plausible that a Western country would allow women to make crucial foreign policy decisions on the world stage, this was normal for the Soviet Union. But how could this be? Democracy was good, and communism was bad. Why did the bad guys allow girls to do more?
 \par 
Fast-forward thirty years later, to November 2016, as I was sitting on the couch with my own fifteen-year-old daughter. We were watching PBS and ready to pop the champagne to
 \par 
KRISTEN R. GHODSEE celebrate the election of the first female President of the United States. Whatever my personal feelings toward Hillary Clinton (you know me well enough now to suspect that I preferred Bernie Sanders), I was thrilled that this glass ceiling would finally be broken. Where I had struggled to find role models of women in power, I hoped my daughter would spend her remaining high school years with a woman in the Oval Office. The bitter disappointment of that night reflected two unpleasant realities in America: the racist backlash against the first black president and a persistent bias against women in positions of authority. During the Cold War, the rise of a large domestic women’s movement—combined with political fears about the perceived progress of women in the state socialist countries—forced Western countries to outlaw discrimination on the basis of sex and promote policies to support gender equality in the workplace. In the course of two short decades, women enjoyed opportunities for labor force participation in almost all sectors of the economy, entering many professions once considered the exclusive purview of men. Today, women make up the majority of college graduates in many advanced capitalist countries. But despite their experience and education, women still face barriers to the top positions in government and business. Over forty years of women’s activism has done little to break the male stranglehold on political and economic power.
 \par 
In the United States, there exists much hand wringing about the lack of women in leadership positions. Even though studies show that diversity in corporate leadership increases profitability, efforts to challenge the status quo find few proponents. Researchers look for explanations, often faulting women for not being ambitious enough or for not
 \par 
79
 \par 
80
 \par 
PANTSUITS ARE NOT ENOUGH: ON LEADERSHIP
 \par 
“Leaning in.” Some blame the challenges of combining work with family responsibilities and the frequent career interruptions for those who perform care work in the home. Others say that competition for top jobs is nasty and full of treachery, and that women aren't willing to join the fray. If they do, ambitious men will backstab them first, believing women less likely to retaliate. While all of these things may contribute to the problem, the underlying issue is the persistence of gender stereotypes in society, stereotypes internalized by girls from the earliest age. Just as I learned that it wasn’t plausible for me to represent my country on the Security Council because of my sex, my daughter learned that a well-qualified woman with years of relevant experience could lose an election to a celebrity businessman with no governmental experience.
 \par 
Two 2014 surveys by the Pew Research Center revealed that most Americans recognize the pervasiveness of this underlying gender discrimination. One poll asked Americans what held women back from moving into “top executive business positions” and “high political offices.” Whereas only {\color{blue}9} percent believed that women weren't “tough enough” for the business world, {\color{blue}43} percent claimed that “women are held to higher standards” and that businesses were simply not ready to hire women as leaders despite their equal qualifications with men. In terms of high political office, only {\color{blue}8} percent claimed that women weren't “tough enough,” but {\color{blue}38} percent believed that female candidates were held to higher standards, and {\color{blue}37} percent agreed that Americans were simply not ready to elect a woman for a position of power. When asked about prospects for the next decades, a majority of Americans believed that “men will continue to hold more top business positions than women in the future.”!
 \par 
KRISTEN R. GHODSEE
 \par 
This is not to deny that American culture has changed; it is just to note that it is changing at a far slower pace compared to many of our peers. In 1990, only {\color{blue}7} percent of the members of the US Congress were women. In 2019, a historic high for female members of Congress, this rose to {\color{blue}24} percent. Compared with some of the democratic socialist Scandinavian countries, the home of the brave looks like a laggard. The election of women members in the Swedish parliament grew from {\color{blue}38} percent in 1990 to {\color{blue}47} percent in 2018. In Norway, {\color{blue}36} percent of MPs were women in 1990 and {\color{blue}41} percent in 2017. The relevant figures are {\color{blue}31} percent (1990) to {\color{blue}37} percent (2017) for Denmark and {\color{blue}32} percent
 \par 
(1990) to {\color{blue}42} percent (2015) in Finland. Iceland wins the prize for almost complete gender parity; women’s percentage of seats in parliament grew from {\color{blue}21} percent in 1990 to {\color{blue}48} percent in 2015, although it declined to {\color{blue}38} percent by 2017. Why the difference? One word: quotas.’
 \par 
In terms of women in leadership positions in the corporate world, the United States falls even further behind. Although women made up {\color{blue}45} percent of the employees in the top Fortune {\color{blue}500} companies in 2016, they held only {\color{blue}21} percent of the board seats and represented only {\color{blue}11} percent of the top earners. Compare this to Norway, where strict quota laws on board representation mean that {\color{blue}42} percent of corporate board seats were filled by women. In Sweden, this number is {\color{blue}36} percent, and in Finland it is {\color{blue}31} percent. But even democratic socialist countries like Sweden struggle with getting women into the c-suite; the percentage of women in executive positions was still under {\color{blue}15} percent in 2012. And in 2014, the Wall Street Journal reported that out of {\color{blue}145} large Nordic companies, only {\color{blue}3} percent had female
 \par 
81
 \par 
82
 \par 
PANTSUITS ARE NOT ENCUGH: ON LEADERSHIP chief executive officers. Although women have the education and experience, top leadership positions in business everywhere continue to be gendered male. The only way to crack this continued dominance is through legislation that forces or strongly incentivizes the gender parity of positions at the top.’
 \par 
So what about the state socialist countries? Although there were important efforts made to promote women to the highest ranks, and they certainly supported the idea that women could and should be in positions of power, the story is complicated by the specific nature of twentieth century East European regimes. First, while there were official quotas for women in parliaments and in the Central Committees of the Communist Parties of most states, the composition of the elite Political Bureau (Politburo), where the real power lay, remained overwhelmingly male. Second, even when women’s political participation increased at the local and municipal level, their participation was limited by the centralized nature of the one-party state. In terms of managerial positions within the state-run economy, the picture was also mixed. Decision-making power rested in the hands of the central planners, who were largely (though not exclusively) male. But different countries had different priorities, and certain sectors of the economy were more amenable to women’s leadership than others. Women dominated the fields of medicine, law, academia, and banking, and on a symbolic level, at least, the state socialist countries did have an excellent record of promoting women into top positions compared to countries in the West.‘
 \par 
KRISTEN R. GHODSEE
 \par 
Unlike capitalism, which distributes society’s wealth on a competitive model based on ideals of meritocracy and survival of the fittest, socialism supports an egalitarian ideology. Social inequality is considered an inevitable by-product of the private ownership of the means of production: the factories, machines, technologies, intellectual property, and so forth. Capitalist economies create an ever-growing wealth gap between those who own the means of production and those who must sell their labor for less than the value it creates in order to meet their basic needs. Ongoing exploitation of those who work for a living increases the wealth of those at the top; the rich get richer at a faster and faster rate, which allows them to control more and more of the means of production. Socialist policies interrupt this trend toward growing inequality through a number of mechanisms, including the creation of public or collectively owned enterprises (co-ops) and/or redistribution of wealth through progressive taxation and the creation of publicly funded social safety nets to prevent destitution. Other than promoting the interests of the poor majority over those of the rich minority, however, nothing inherent in socialist ideology privileges any one social group over another. And women’s emancipation was fundamental to the socialist vision from its inception (even if women’s class identity was always privileged over their gendered identity).
 \par 
The idea that men and women would share political power had roots in the earliest incarnations of socialist ideals, which emerged after the French Revolution. In the 1820s and 1830s, the utopian socialist Saint-Simonians organized themselves into small religious communities in Paris, pooling their incomes and living collectively. An
 \par 
83
 \par 
84
 \par 
PANTSUITS ARE NOT ENOUGH: ON LEADERSHIP early leader, Prosper Enfantin, served as the community's “pope”; he proposed to share his position of authority with a woman who would serve as a “popes.” Unlike Mary Wollstonecraft and John Stuart Mill, who based their arguments for sexual equality on men’s and women’s innate rationality, the Saint-Simonians believed that men and women had different but complementary natures and that both spiritual and political authority required representation from each half of humanity. After internal debates, Enfantin’s views prevailed, and the larger Saint-Simonian community was to be ruled by a couple-pope who served as the living representatives of God’s masculine and feminine attributes. All positions of power were to be shared by a representative from each sex: each smaller community was headed by a male-female couple, their collective homes were led by a “brother” and “sister” pair, and each of their work syndicates was governed by a “director” and a “directness.”®
 \par 
Another prominent utopian socialist was the Frenchman Charles Fourier, who is believed to have coined the word “feminism” in 1837. Fourier was a fierce advocate for women’s rights and believed that all professions should be open to women based on their abilities as individuals. Fourier understood that European women were no better than chattel to their fathers and husbands, and he proposed that enlightened societies would demonstrate their moral progress by freeing women from the narrow gender roles that trapped them in conventional marriage. Fourier promoted the idea of collectively owned agricultural communities (called “phalanxes”) in which men and women would work side by side and share the fruits of their labors in common. Fourier wrote: “Social progress and historic changes occur
 \par 
KRISTEN R. GHODSEE by virtue of the progress of women toward liberty, and decadence of the social order occurs as the result of a decrease in the liberty of women.”
 \par 
The Saint-Simonians and Charles Fourier influenced the work of another important French utopian socialist, the fascinating Flora Tristan. She was the first theorist to connect women’s emancipation with the liberation of the working classes. She understood that the relationship of the wife to the husband was analogous to that of the proletariat to the bourgeoisie. Writing and lecturing in the late 1830s and early 1840s, Tristan saw feminism and socialism as mutually dependent movements that would bring about a total transformation of French society; the emancipation of women could not happen without the liberation of workers, and vice versa. Instead of a model in which sexual equality tricked downward from the legal gains and increased educational opportunities of wealthy women, Tristan believed that the creation of one large and diverse worker's union (composed of both men and women) would realize sexual equality first among the toiling classes.’
 \par 
Expanding on these ideas, the German socialists August Bebel and Friedrich Engels proposed a historical justification for women’s emancipation, arguing that hunters and gatherers had once lived in primitive communal matriarchies. According to their theories, early humans survived — in clans that consisted of men and women who practiced a form of group marriage and raised their children collectively. Since paternity could not be established, descent was traced through the mother, and women had an equal if not greater share in decision-making. Bebel and Engels argued that it was only after the advent of agriculture and
 \par 
85
 \par 
86
 \par 
PANTSUITS ARE NOT ENOUGH: ON LEADERSHIP private property that wealth could be accumulated. Hunters and gatherers did not horde resources; they consumed everything they hunted and gathered. But when some humans began fencing off large tracts of land to produce more food than they needed to survive and started selling the surplus, a new set of incentives destroyed old social structures. Landowners needed laborers to help them create greater surpluses, and it was at this moment in history that women’s bodies became machines for manufacturing more workers. (They argue that this era also coincided with the invention of slavery).*
 \par 
According to Bebel and Engels, once landowners began accumulating private fortunes, this class of men desired to pass their wealth on to legitimate heirs. This precipitated the invention of monogamous marriage and the enforced fidelity of the wife. The old matrilineal system was replaced by a patrilineal system whereby descent was traced through the father. (We can see the operation of this patrimony today, when women take the last names of their husbands upon marriage and children receive the surnames of their fathers. In a matrilineal system, it would be the reverse.) Engels postulated that this desire to accumulate wealth robbed women of their earlier autonomy: “The overthrow of the mother right was the world historical defeat of the female sex. The man took command in the home also; the woman was degraded and reduced to servitude, she became the slave of his lust and a mere instrument for the production of children.” For early socialists, therefore, the abolition of private property would inevitably lead to the restoration of women's “natural” role as men’s equal.’
 \par 
KRISTEN R. GHODSEE
 \par 
Socialist ideas about women’s emancipation would help fuel revolutionary impulses in Russia in 1917. The February revolution that toppled Tsar Nicholas II began on International Women’s Day, precipitated by women strikers. As a provisional government tried to stabilize Russia in the following months, these women demanded full suffrage. In July 1917, they won the right to vote and stand for public office. After the October Revolution, Lenin and the Bolsheviks allowed women to vote and run in the elections for the Constituent Assembly. Most people don’t realize that the Soviet Union did not become a one-party authoritarian state overnight. Because Lenin hoped to win a popular mandate, he allowed “the freest elections ever held in Russia until after the collapse of the Soviet Union in 1991,” according to historian Rochelle Ruthchild. Voting began in November 1917 and lasted for about a month. Voter participation in the Constituent Assembly elections was incredible given the chaos of the time, and women’s electoral turnout exceeded all expectations. However, Lenin dissolved the democratically elected Constituent Assembly once it became clear that his Bolshevik party would not have a majority. Soviet women’s right to vote became largely superfluous in the dictatorship of the proletariat.”
 \par 
Despite the institution of “war communism” and the centralization of political authority, Lenin did initially empower a group of activists to lay the groundwork for the full emancipation of women. Alexandra Kollontai served as the people’s commissar for social welfare and helped to found the Soviet women’s organization the Zhenotdel. As
 \par 
87
 \par 
88
 \par 
PANTSUITS ARE NOT ENOUGH: ON LEADERSHIP discussed earlier, she would be in charge of implementing a wide range of policies to support women’s full incorporation into the Soviet labor force. The American journalist Louise Bryant was awed by Kollontai’s commitment and lack of fear when dealing with the Bolshevik men. Bryant reported in 1923:
 \par 
SSSSSS bad. She has unlimited courage and on several occa-
 \par 
Sions has openly opposed Lenin. As for Lenin, he has crushed her with his usual unruffled frankness. Yet in spite of her fiery enthusiasm she understands “party dis-
 \par 
Cipline” and takes defeat like a good soldier. If she had left the revolution four months after it began she could
 \par 
Have rested forever on her laurels. She seized those rosy first moments of elation, just after the masses had cap-
 \par 
Tured the state, to incorporate into the Constitution laws for women which are far-reaching and unprecedented.
 \par 
And the Soviets are very proud of these laws which already have around them the halo of all things connected with the Constitution.”
 \par 
Kollontai would eventually be sent as the Soviet ambassador to Norway, the first Russian woman to hold such a high diplomatic post (and the third female ambassador in the world), but after the rise of Stalin she would fall into relative obscurity, with many of her original dreams for women’s emancipation either discredited or forgotten.
 \par 
Among the other prominent women who worked with the Zhenotdel in the 1920s was Nadezhda Krupskaya,
 \par 
KRISTEN R. GHODSEE
 \par 
Lenin's wife, a radical pedagogue who served as the deputy minister for education from 1929 to 1939. She worked to build new schools and libraries for a population in which six out of ten people could not read or write in 1917, and her educational ideals would go on to inspire leftist educational reformers like Paulo Freire in Brazil. Another prominent Bolshevik, Inessa Armand, worked as a leader in the Moscow Economic Council, served as a top member of
 \par 
The Moscow Soviet, and would eventually be the director of the Zhenotdel. Countless other Bolshevik women would take up positions of power in the early Soviet government as the country struggled to survive a civil war, a horrific famine, and Lenin’s early death.”
 \par 
The Stalinist era saw a relative return to traditional gender roles even as the Soviets encouraged women to engage in military training. The historian Anna Krylova has explored the slow integration of Soviet women into the military despite initial male resistance. By World War II, the USSR had squadrons of trained female fighter pilots. These included the infamous Nachthexen (night witches) of the 588th Night Bomber Regiment of the Soviet Air Forces, who flew in stealth mode at night and dropped precision bombs on German targets. The women pilots were all in their late teens and early twenties, and they flew over twenty thousand missions from 1941 to 1945. Although other countries had trained female pilots who flew in support roles, the Soviet Union was the first country in the world to allow women to fly combat missions. The Nazis feared these female pilots, and any German pilot who shot a “witch” out of the sky supposedly won himself an automatic Iron Cross.” Across Eastern Europe, World War II also inspired
 \par 
89
 \par 
90
 \par 
PANTSUITS ARE NOT ENOUGH: ON LEADERSHIP thousands of women to take up arms as anti-Nazi guerillas, and many would go on to have careers in national and international politics. For example, Vida Tom$i¢ was a Slovenian communist who fought as a partisan against the Italians and became her country’s minister for social policy after the war. She served in a wide variety of government posts and became a dedicated women’s activist both within Yugoslavia and internationally during the Cold War. A legal scholar and jurist, Tomai¢ was revered as a national heroine between 1945 and 1991, and represented Yugoslavia in several posts at the United Nations.”
 \par 
Neighboring Bulgaria also produced spirited antifascist women who would later enter politics. Elena Lagadinova was the youngest female partisan fighting against her country’s Nazi-allied monarchy. She later earned a PhD in astrobiology and worked for thirteen years as a research scientist before serving as the president of the Committee of the Bulgarian Women’s Movement for twenty-two years. Lagadinova was also a member of Parliament, a member of the Central Committee, and a passionate advocate for women’s rights on the international stage, particularly during the United Nations Decade for Women between 1975 and 1985. Another Bulgarian partisan was Tsola Dragoycheva, who fought against Bulgaria’s right-wing monarchist regime beginning in the 1920s. A heroine of the Bulgarian Communist Party, Dragoycheva served as Bulgaria’s first woman to hold a cabinet position as the minister of the Postal Service, Telegraph and Telephone after World War II. From 1944 to 1948, she also served as the general secretary of the National Committee of the Fatherland Front, headed the Council of Ministers, and wielded great influence over
 \par 
KRISTEN R. GHODSEE the development of Bulgaria’s newly planned economy. Later she would become a full member of the Bulgarian Politburo, one of the few women in the Eastern Bloc to rise to such a high position without being the wife or daughter of a communist leader.'°
 \par 

 \par 
Other socialist women in Eastern Europe had been in and out of prison for their political activities in the 1930s or spent time as exiles in the Soviet Union until they could return home after the end of World War II. In Romania, the rise of “Aunty Ana” Pauker showed the world that state socialism would allow women to take up the highest positions in government, shocking Western observers. Writing in the New York Times in 1948, journalist W. H. Lawrence reported, “Ana Pauker is both architect and builder of the new Rumanian [sic] Communist state. She not only plans, but she translates political, economic, and social blueprints into action as secretary of the Rumanian Communist Party and Minister for Foreign Affairs of the newly proclaimed republic—the first woman in the world to hold the title of. From the standpoint of international Foreign Minister. .
 \par 
Communism, Ana Pauker’s is a Horatio Alger success story—from political rags to political riches.” In September 1948, Time featured her portrait on its cover and labeled her “the most powerful woman alive.”
 \par 
The Eastern Bloc countries also excelled at strategic international demonstrations of their commitment to women’s rights, particularly in the case of Valentina Tereshkova. In June 1963, just five years after the launch of Sputnik, the front page of the New York Herald Tribune read: “Soviet Blonde Orbiting as First Woman in Space.” In the same year that Betty Friedan published The Feminine Mystique,
 \par 
91
 \par 
92
 \par 
PANTSUITS ARE NOT ENOUGH: ON LEADERSHIP the banner headline of the Massachusetts Springfield Union declared: “Soviet Orbits First Cosmonette.” The Soviets made Tereshkova a symbol of their progressive social policy, and she headed their delegations to the three UN world conferences on women in 1975, 1980, and 1985. In 1982, cosmonaut Svetlana Sevitskaya was the first woman to fly on a space station, a year before Sally Ride became the first American woman astronaut. Two years later, Sevitskaya completed the first space walk by a woman and became the first woman to complete two separate space missions."®
 \par 
Although Soviet women rarely ventured into the realm of high politics, there were some important exceptions. In 1917, Elena Stasova was the first woman to become a candidate member of the Soviet Politburo, the highest political body in the country, although her tenure was very brief. Decades later, in 1957, Ekaterina Furtseva was elected as a full member of the Politburo, serving for four years. She supported Khrushchev’s desalinization policies and eventually left the Politburo to become the minister of culture from 1960 to 1974. In September 1988, Alexandra Biryukova became a candidate member of the Politburo, which carried nonvoting status. Finally, in 1990, Galina Semyonova was the second woman to become a full voting member of the Politburo. Nominated by Gorbachev himself as a first step in his plan to put more women into positions of power, Galina Semyonova earned a doctoral degree in philosophy and spent thirty-one years as a working journalist. At age fifty-three she was a mother and grandmother. Her election signaled that the Soviets were ready to take domestic women’s issues more seriously. In a January 1991 interview with the Los Angeles Times, Semyonova was openly critical of the
 \par 
KRISTEN R. GHODSEE
 \par 
Soviet government's previous policies toward women’s leadership. “From the founding of our state,” she told the American journalist, “we have many very humane laws. Lenin personally signed many decisions and laws on the family, on marriage, the political rights of women, the liquidation of illiteracy among the female population. But these laws, in fact, were quite often counteracted by social-economic practice. The result was that women were not prepared to assume the leading role in society.” Using the new freedoms being granted under perestroika, Semyonova hoped that putting more Soviet women into leadership roles would make politics “more humane and prevent it from becoming too aggressive.”””
 \par 
Although these high-profile examples demonstrate the state socialist countries’ commitment to the ideal of women’s rights, actual practice did not always live up to the rhetoric. Between 2010 and 2017, I spent over a hundred and fifty hours interviewing the octogenarian Elena Lagadinova, the president of Bulgaria’s national women’s organization. Lagadinova admitted that the socialist states did not achieve as much as she had hoped. I once asked her why more women did not rise up to the highest positions of power given the general commitment to women’s rights. Lagadinova acknowledged that this had been an ongoing challenge for the Bulgarian women’s committee and claimed that East European countries did not have enough time to overcome the centuries-old idea that leaders should be men. It wasn’t just that men disliked women in power, Lagadinova argued; it was that women also felt uncomfortable with women’s leadership. As a result, they were less likely to support their female comrades and more reticent
 \par 
93
 \par 
94
 \par 
PANTSUITS ARE NOT ENOUGH: ON LEADERSHIP to pursue positions of authority. They preferred to work behind the scenes, she said. High politics in Eastern Europe, just like high politics elsewhere, was a treacherous place, infused with intrigues and betrayals. Lagadinova suggested that women were less inclined to engage in the necessary subterfuges. On the other hand, she believed that political life might have been more civilized if there had been more women at the top. Her organization tried to promote qualified candidates when they could, but the patriarchal culture of the Balkans, combined with the authoritarian nature of the state (ruled by the same man for thirty-five years), discouraged women from getting involved.
 \par 
To encourage more women to take a chance in politics, Bulgaria and other state socialist countries introduced quotas for women in parliament, and they did have higher percentages of women holding political office than most of the Western democracies throughout the Cold War. Women’s positions in the governing apparatus of a one-party state were largely symbolic, but the symbolism was important. After all, male members of parliament and of the Central Committee enjoyed no greater authority than their female comrades. Women fared better in white-collar jobs in the planned economy, often dominating banking, medicine, the academy, and the judiciary. Part of this trend reflected specific policies to promote women in the professions, but it was also the case that blue-collar, industrial jobs paid higher wages under state socialism, so men tended to concentrate their labor in those sectors of the economy. But as discussed in Chapter 1, female labor force participation rates were the highest in the world. Because the number of women in the workforce was greater, there were numerically more women
 \par 
KRISTEN R. GHODSEE in managerial positions. Furthermore, the Eastern Bloc countries did an excellent job at funneling women into the science, engineering, and technology sectors. A March 9, 2018, article in the Financial Times revealed that eight of the top ten European countries with the highest rates of women in the tech sector were in Eastern Europe, a legacy of the Soviet era, when women were encouraged to pursue these careers. Indeed, between 1979 and 1989, the percentage of women in the USSR working as “engineering and technical specialists” increased from {\color{blue}48} to {\color{blue}50} percent of all workers in those fields—exact parity. Also by 1989, {\color{blue}73} percent of all Soviet “scientific workers, teachers, and educators” were women.”
 \par 
State-mandated quotas for women in political office, on corporate boards, and in public enterprises have been implemented in democratic countries around the world, and studies show that they have been remarkably effective, if properly enforced, in increasing the number of women in positions of authority. Since 1991, over ninety countries have implemented some kind of quota system for women in national parliaments, and the percentage of women in positions of power has skyrocketed, creating role models for the next generation of girls aspiring to careers in politics. In 2017, out of the forty-six countries that have {\color{blue}30} percent or more women in their parliaments, forty of them have some form of quota system in place. But quotas work best in electoral systems based on proportional representation, in which citizens vote for parties rather than individuals. Quotas can legislate that a certain percentage of the names
 \par 
95
 \par 
96
 \par 
PANTSUITS ARE NOT ENOUGH: ON LEADERSHIP on an electoral slate are those of women. Because Americans vote for individual politicians in single-member constituencies, quotas would be difficult to enforce. If political parties had to run a certain number of women, they might concentrate them in constituencies where they know women will lose. But there could be quotas for appointed cabinet positions, for instance, or other creative ways to increase women’s participation without revamping the electoral system.”
 \par 
State-mandated quotas for women on the executive boards of corporations and public enterprises have successfully promoted women into leadership positions and are quite doable in the US context. Quotas were first introduced in Norway in 2003; companies faced dissolution if they did not diversify their boards. For large firms, a full {\color{blue}40} percent of board seats needed to go to women. After Norway, other European countries imposed quotas on corporations, albeit with softer penalties for noncompliance. Perhaps not surprisingly, the softer the mandate, the fewer the companies that complied. Although the percentage of women serving on the boards of large publicly traded companies rose from {\color{blue}11} percent in 2007 to {\color{blue}23} percent in 2016, this figure was significantly higher in countries with strict quotas in place. {\color{blue}44} percent in Iceland, {\color{blue}39} percent in Norway, and {\color{blue}36} percent in France. In Germany, where quotas are voluntary save for large companies, the percentage is only {\color{blue}26} percent. As a result, the European Commission tried in 2017 to push for an EU-wide law requiring that large companies in all member states impose a {\color{blue}40} percent quota for women on corporate boards.”°
 \par 
Of course, no woman wants to feel like a second-class citizen or occupy a position merely because she is female,
 \par 
KRISTEN R. GHODSEE so it is important to realize that the ongoing discrimination against women in leadership positions is not because Americans think that women are less capable or lack the necessary leadership attributes. A 2014 Pew survey on women and leadership found that most respondents saw no difference between men and women’s innate abilities. In some categories, such as honesty, ability to mentor employees, and willingness to compromise, Americans who believed there were differences between men and women thought that women were better than men. Discrimination against women in leadership has little basis in differential skill sets and more to do with social attitudes about women in power. So this is not about putting less qualified women into leadership positions because they are female; it’s about trying to counteract the deep, unconscious gender stereotypes about men as leaders and women as followers. Some people just feel weird with a girl boss.”
 \par 
We don’t associate women with positions of authority because we’ve seen so few of them. And because there are so few women in positions of authority, both men and women continue to associate leadership with male bodies, a vicious cycle that is hard to break out of. (A similar problem can be found for women in the sciences, engineering, and tech.) When asked about factors affecting their own ambition and willingness to stand for election or compete for top jobs, women often blame a lack of female role models for their reticence. For example, the consulting firm KPMG conducted a Women’s Leadership Study in 2015, surveying 3,014 US women between the ages of eighteen and sixty-four. In terms of learning about leadership, {\color{blue}67} percent claimed that their most important lessons came from other women. An
 \par 
97
 \par 
98
 \par 
PANTSUITS ARE NOT ENOUGH: ON LEADERSHIP additional {\color{blue}88} percent reported that they were encouraged by seeing women in leadership positions, and {\color{blue}86} percent agreed with the statement “When they see more women in leadership, they are encouraged they can get there themselves.” Finally, {\color{blue}69} percent of the women surveyed agreed “that having more women represented in senior leadership will help move more women into leadership roles in the future.” Because of the importance of role models, KPMG recommended the promotion of qualified women into high managerial positions, onto corporate boards, and into the c-suite. And a 2016 study by the Rockefeller Foundation found that {\color{blue}65} percent of Americans “say it is especially important for women starting their careers to have women in leadership positions as role models.” But we know from the experiences in Europe that this won’t happen without some sort of external intervention.”
 \par 
While women’s leadership is important, it’s also worth noting that quotas in business and politics may only benefit a small percentage of white, middle-class women. If we focus on promoting women into positions of power to the exclusion of other pressing issues that affect poor and working-class women, particularly women of color, we fall into the dangerous trap of corporate feminism a la Ivanka Trump. Yes, the glass ceiling needs to be broken, but that does not mean we should ignore the pressing problems of this nowhere near as high in the pecking order. Both executive men and women, as well as men and women in politics, often build their careers on the backs of poorer women: the nannies, au pairs, cooks, cleaners, home health aides, nurses, and personal assistants to whom they outsource their care work. Policies to help women get to the top always
 \par 
KRISTEN R. GHODSEE must be combined with practical steps to help those women struggling at the bottom, or they simply exacerbate existing inequalities.
 \par 
For example, if federal, state, or local governments ever embraced the idea of job programs for the unemployed, it would be wise to couple this policy with a mandatory quota stipulating that {\color{blue}50} percent of all jobs created would be reserved for women. It’s not at all unthinkable that politicians might decide to create a special jobs program for men, believing that women don’t need to work since they have responsibilities in the home. In the early years of the economic transition in some Eastern European countries, job creation policies targeted displaced men on the assumption that the male breadwinner/female homemaker model was more desirable than the reverse. Since job creation in the private sector paled in comparison to the job losses caused by the rapid privatization or liquidation of state-owned enterprises, there simply weren’t enough jobs in the economy to employ all the people made redundant by the economic transition. In order to control unemployment, women were forced back into the home by rehabilitation policies, and there was explicit job discrimination against women as the imposition of free markets came bundled with the return of traditional gender roles.
 \par 
But when most people talk about quotas, they are usually discussing quotas for elite positions of power, and it’s also important to realize that quotas alone can’t remove all barriers. There may be other ways to increase the number of women in leadership positions, but the core idea is to create more positive role models, which can start to reshape societal attitudes. All women and girls are harmed when society
 \par 
99
 \par 
100
 \par 
PANTSUITS ARE NOT ENOUGH: ON LEADERSHIP casts ambitious women as wicked or ugly, imagining power and authority as a naturally masculine character trait. Patriarchal culture permeates society, and both men and women feel uncomfortable with women in power. Strong and competent women are considered less feminine, if not downright unpleasant. Notice the language used in Time's 1948 description of “the most powerful woman alive,” Romania’s Ana Pauker: “Now she is fat and ugly; but once she was slim and (her friends remember) beautiful. Once she was warmhearted, shy and full of pity for the oppressed, of whom she was one. Now she is cold as the frozen Danube, bold as a Boyar on his own rich land and pitiless as a scythe in the Moldavian grain.” Pauker’s ugliness develops as her political authority expands; her shy, warmhearted nature is corrupted by her entry into the male-dominated corridors of power. Not surprisingly, the Time cover image of Pauker is an unflattering profile of an angry, middle-aged woman with short, gray hair.”
 \par 
This negative image of fat and ugly communist women was consciously produced and reproduced by the American media throughout the Cold War. Growing up in the Reagan era, I believed those awful stereotypes that circulated about unattractive Soviet women. I remember an advertisement for the Wendy’s hamburger chain in the mid-1980s—a fashion show, Soviet style. Playing on the worst American tropes, the commercial features a fat, middle-aged woman wearing a gray smock and a grandmotherly kerchief around her hair. She struts up and down a catwalk below a portrait of Lenin. Another fat, masculine woman in an olive green military uniform calls out, “Day wear,” “Evening wear,” and “Swim wear” as the first woman walks out wearing the
 \par 
KRISTEN R. GHODSEE exact same smock, only holding a flashlight for “evening wear” and a beach ball for “swim wear.” The voiceover of the ad informs viewers that they have a choice at Wendy’s (unlike people in the USSR), but it was the image of Soviet femininity (or lack thereof) that made the commercial so powerful. I was still in my teens when I first saw this ad, and it certainly occurred to me that wanting to wield veto power might somehow strip me of my femininity. When I finally got my shot at a seat on the Security Council, I wondered whether the boys thought they were punishing me by making me represent the “evil empire.”
 \par 
Of course, representing the East Bloc countries was also much harder than role playing the United States, the
 \par 
United Kingdom, or France. To be a Western country, all you had to do was peruse the newspaper or binge-read U.S. News and World Report. Figuring out the ideological and practical motivations for Soviet and Eastern Bloc foreign policy positions required savvy research skills. In those days, long before the internet, foreign policy research had to be done using print sources, usually available only in a library. And if you wanted to read actual records from the United Nations, you had to find a way to get to a university library. But to win the top prize, I had to read books and reports produced by the Eastern Bloc countries. I needed to understand their worldviews, so I could represent them more convincingly.
 \par 
It was 1987 when I stumbled upon a large, hardcover coffee-table book as I conducted background research for the MUN conference, where I was representing the Soviet Union on the Security Council. Published in 1975 to coincide with the United Nations’ International Year of Women,
 \par 
101
 \par 
102
 \par 
PANTSUITS ARE NOT ENOUGH: ON LEADERSHIP
 \par 
Women in Socialist Society was an elegant piece of East German propaganda, celebrating the gains of women in the Eastern Bloc. Although I was suspicious of the didactic English text, I was entranced by the images. The photos of Rosa Luxemburg and Alexandra Kollontai, the latter a strikingly beautiful young woman. The lovely twenty-six-year-old Valentina Tereshkova in her uniform. As if directly responding to the Western stereotype of Eastern Bloc women as tired, fat, and ugly, the East Germans included a whole chapter on “Women, Socialism, Beauty and Love,” complete with stylized black-and-white nude photographs of gorgeous models baring their perky breasts for the cause. Scattered across the glossy pages were svelte, pretty women working in factories, in labs, in classrooms, and sitting around conference tables with men. Women competing at the Olympics, women smiling at their children, and women laughing together as workmates.
 \par 
Later, as I learned about the command economy, I understood that the images in the book represented more of the communist ideal than the lived reality of state socialism in Eastern Europe. In the late 1990s, when I first lived in Bulgaria, street vendors sold women’s panties on every corner. In newspaper kiosks, you could buy a lace thong with your morning edition, because people were trying to make up for their relative deprivation before 1989. Under state socialism, the central planners ignored women’s desires, and there were persistent shortages of the feminine accoutrements women take for granted in the West, including basic hygiene products. Bulgarian women of a certain age still cringe when they think of the rough cotton batting they had to use once a month (if they could find it).
 \par 
KRISTEN R. GHODSEE
 \par 
Slavenka Drakulić captured this frustration when she traveled around Eastern Europe for Ms. in 1991, reporting the complaint she “heard repeatedly from women in Warsaw, Budapest, Prague, Sofia, East Berlin: “Look at us—we don’t even look like women. There are no deodorants, perfumes, sometimes even no soap or toothpaste. There is no fine underwear, no pantyhose, no nice lingerie. Worst of all, there are no sanitary napkins. What can one say except that it is humiliating?” While women in Eastern Europe may have had far more career paths open to them, they certainly lacked the consumer products available to women in the West.”{\color{blue}4}
 \par 
But as a high school student, I didn’t know any of this yet, and the images in that glossy East German book gave me the confidence I needed to fully embrace my role as a Soviet diplomat to the United Nations. Since it was the eighties, I bought myself a shiny red crushed satin suit with massive shoulder pads, slathered on the eye shadow, and hot-rolled and hair-sprayed my curly hair to precarious heights. Somehow, it helped to know that there were societies that imagined, even if only in an idealized world, that women could be both ambitious and beautiful. I could have my breasts and veto power, too.
 \par 
In the end, although patriarchal culture changes at a glacial pace, experts from politicians in the European Union to the consultants at KPMG believe that affirmative steps must be taken to promote women’s leadership. There is no one-size-fits-all solution, but quotas can be an important part of the process. States have a role in shaping societal attitudes to increase diversity and inclusivity, and it is essential that we use the tools of thoughtful legislation to
 \par 
103
 \par 
104
 \par 
PANTSUITS ARE NOT ENOUGH: ON LEADERSHIP create more opportunities for women to stand for elected office or serve on executive boards. Yes, popular attitudes have to change, but this change requires that little girls grow up seeing more women in positions of power. The only way for girls to see women in positions of power is to find a way to challenge the political and economic cultures that prevent their participation in the first place.
 \par 
