\chapter{Inside Out}\label{Inside Out}
 \par 
“The 1960s are rightly remembered as years of cultural dissent and political upheaval, but they are wrongly remembered as years stirred only from the left,” writes George Will in the foreword to a reissued edition of Barry Goldwater’s The Conscience of a Conservative. {\color{blue}1} Several decades ago, such a claim would have elicited puzzled looks, if not catcalls and jeers. But in the years since, the publication of a slew of books, each advancing the notion that most of the political innovation of the last half-century has come from the right, has led historians to revise the conventional wisdom about postwar America, including the 1960s. The new consensus is reflected in the opening sentence of Ronald Story and Bruce Laurie’s The Rise of Conservatism in America, 1945–2000 : “The central story of American politics since World War II is the emergence of the conservative movement.” {\color{blue}2} Yet for some reason Will still feel that his kinsmen are insufficiently appreciated and recognized.
 \par 
This chapter originally appeared as a review of Barry Goldwater’s The Conscience of a Conservative (Princeton, N.J.: Princeton University Press, 2007, 1960) ; Right-ward Bound: Making America Conservative in the 1970s, ed. Bruce J. Schulman and Julian E. Zelizer (Cambridge, Mass.: Harvard University Press, 2008) ; and Jacob Heilbrunn’s They Knew They Were Right: The Rise of the Neocons (New York: Doubleday, 2008) in The Nation (June 23, 2008): 25–33.
 \par 
Will is hardly the first conservative to believe himself an exile in his own country. A sense of exclusion has haunted the movement from the beginning, when émigrés fl ed the French Revolution and Edmund Burke and Joseph de Maistre took up their cause. Born in the shadow of loss—of property, standing, memory, inheritance, a place in the sun—conservatism remains a gathering of fugitives. Even when assured of his position, the conservative plays the truant. Whether instrumental or sincere, this fusion of pariah and power is one of the sources of his appeal. As William F. Buckley wrote in the founding statement of National Review, the conservative’s badge of exclusion has made him “just about the hottest thing in town.”{\color{blue}3}
 \par 
While David Hume and Adam Smith are often cited by the more genteel defenders of conservatism as the movement’s leading lights, their writings cannot account for, as we have seen, what is truly bizarre about conservatism: a ruling class resting its claim to power upon its sense of victimhood, arguably for the first time in history. Plato’s guardians were wise; Aquinas’s king was good; Hobbes’s sovereign was, well, sovereign. But the best defense of monarchy Maistre could muster was that his aspiring king had attended the “terrible school of misfortune” and suffered in the “hard school of adversity.” {\color{blue}4} Maistre had good reason to off her this defense: playing the plebe, we now know, is a critical weapon in the conservative arsenal. Still, it’s a confusing defense. After all, if the main off bring a prince brings to the table is that he’s really a pauper, why not seat the pauper instead?
 \par 
Conservatives have asked us not to obey them, but to feel sorry for them—or to obey them because we feel sorry for them. Rousseau was the first to articulate a political theory of pity, and for that he has been called the “Homer of the losers.” {\color{blue}5} But doesn’t Burke, with his overwrought account of Marie Antoinette that we saw in chapter 1—“this persecuted woman,” dragged “almost naked” by
 \par 
“The furies of hell” from her bedroom in Versailles and marched to “a Bastile for kings” in Paris—have some claim to the title, too? {\color{blue}6}
 \par 
Marie Antoinette was a particular kind of loser, a person with everything who found herself utterly and at once dispossessed. Burke saw in her fall an archetype of classical tragedy, the great person laid low by fortune. But in tragedy, the almost any hero can hope for is to understand his fate: the wheel of time cannot be reversed; suffering cannot be undone. Conservatives, however, are not content with illumination. They want restoration, an opportunity presented by the new forces of revolution and counterrevolution. Identifying as victims, they become the ultimate moderns, adept competitors in a political marketplace where rights and their divestiture are prized commodities.
 \par 
Reformers and radicals must convince the subordinated and disenfranchised that they have rights and power. Conservatives are different. They are aggrieved and entitled—aggrieved because entitled—and already convinced of the righteousness of their cause and the inevitability of its triumph. They thus can play victim and victor with a conviction and dexterity the subaltern can only imagine. This makes them formidable claimants on our allegiance and affection. Whether we are rich or poor or somewhere in between, the conservative is, as Hugo Young said of Maggie Thatcher, one of us.{\color{blue}7}
 \par 
But how do they convince us that we are one of them? By making privilege democratic and democracy aristocratic. The conservative does not defend the Old Regime; he speaks on behalf of old regimes—in the family, the factory, the field. There, ordinary men, and sometimes women, get to play the part of little lords and ladies, supervising their underlings as if they all belong to a feudal estate. Long before Huey Long cried, “Every man a king,” a more ambiguous species of democrat spoke virtually the same words, though to different effect: the promise of democracy is to govern
 \par 
Another human being as completely as a monarch governs his subjects. The task of this type of conservatism—democratic feudalism—becomes clear: surround these old regimes with fences and gates, protect them from meddlesome intruders like the state or a social movement, while descanting on mobility and innovation, freedom and the future.
 \par 
Making privilege palatable to the masses is a permanent project of conservatism; but each generation must tailor that project to fit the contour of its times. Goldwater’s challenge was set out in the title of his book: to show that conservatives had a conscience. Not a heart—he lambasted Eisenhower and Nixon for trying to prove that Republicans were compassionate {\color{blue}8} —or a brain, which liberals from John Stuart Mill to Lionel Trilling had doubted, but a con-science. Political movements often have to convince their followers that they can succeed, that their cause is just, and their leaders are savvy, but rarely must they prove that theirs is a march of inner lights. Goldwater thought otherwise: to attract new voters and rally the faithful, conservatism had to establish its idealism and integrity, its absolute independence from the beck and call of wealth, from privilege and materialism—from reality itself. If they were to change reality, conservatives would have to divorce them-selves, at least in their self-understanding, from reality. {\color{blue}9} (In this regard, he was not altogether different from Burke, who warned that while the ruling classes in Britain had “a vast interest to pre-serve” against the Jacobin threat and “great means of preserving it,” they were like an “artificer. . . Incumbered by his tools.” Possessing vast “resources,” Burke concluded, “may be among impediments” in the struggle against revolution.) {\color{blue}10} In recent years, it has become fashionable to dismiss today’s Republican as a true believer who betrayed conservatism by abandoning its native skepticism and spirit of mild adjustment. Goldwater was independent and
 \par 
Ornery, the argument goes, recoiling from anything so stultifying (and Soviet) as an ideology; Bush (or the neocon or Tea Partier) is rigid and doctrinaire, an enforcer of bright lines and gospel truths. But conservatism has always been a creedal movement—if for no other reason than to oppose the creeds of the left. “The other side have got an ideology,” declared Thatcher. “We must have one as well.” {\color{blue}11} To counter the left, the right has had to mimic the left. “As small as they are,” John C. Calhoun wrote admiringly of the abolitionists, they “have acquired so much influence by the course they have pursued.”{\color{blue}12}
 \par 
Goldwater understood that. During the Gilded Age, conservatives had opposed unions and government regulation by invoking the freedom of workers to contract with their employer. Liberals countered that this freedom was illusory: workers lacked the means to contract as they wished; real freedom required material means. Goldwater agreed, only he turned the same argument against the New Deal: high taxes robbed workers of their wages, rendering them less free and less able to be free. Channeling John Dewey, he asked, “How can a man be truly free if he is denied the means to exercise freedom?” {\color{blue}13} Franklin Delano Roosevelt claimed that conservatives cared more about money than men. Goldwater said the same about liberals. Focusing on welfare and wages, they “look only at the material side of man’s nature” and “subordinate all others considerations to man’s material well-being.” Conservatives, by contrast, take in “the whole man,” making his “spiritual nature” the “primary concern” of politics and putting “material things in their proper place.” {\color{blue}14} This romantic howl against the economist of the New Deal— similar to that of the New Left—was not a protest against politics or government; Goldwater was no libertarian. It was an attempt to elevate politics and government, to direct public discussion toward ends more noble and glorious than the management of creature comforts and material well-being. Unlike the New Left, however,
 \par 
Goldwater did not reject the affluent society. Instead, he transformed the acquisition of wealth into an act of self-definition through which the “uncommon” man could distinguish himself from the “undifferentiated mass.” {\color{blue}15} To amass wealth was not only to exercise freedom through material means, but also a way of lording oneself over others.
 \par 
In his essay on conservative thought, Karl Mannheim argued that conservatives have never been wild about the idea of freedom. It threatens the submission of subordinate to superior. Because freedom is the lingua franca of modern politics, however, conservatives have had “a sound enough instinct not to attack” it. Instead, they have made freedom the stalking horse of inequality, and inequality the stalking horse of submission. Men are naturally unequal, they argue. Freedom requires that they be allowed to develop their unequal gifts. A free society must be an unequal society, composed of radically distinct, and hierarchically arrayed, particulars.{\color{blue}16}
 \par 
Goldwater never rejected freedom; indeed, he celebrated it. But there is little doubt that he saw it as a proxy for inequality—or war, which he called “the price of freedom.” A free society protected each man’s “absolute diff erectness from every other human being,” with difference standing in for superiority or inferiority. It was the “initiative and ambition of uncommon men”—the most different and excellent of men—that made a nation great. A free society would identify such men at the earliest stages of life and give them the resources they needed to rise to preeminence. Against those who subscribed to “the egalitarian notion that every child must have the same education,” Goldwater argued for “an educational system which will tax the talents and stir the ambitions of our best students and. . . Thus insure us the kind of leaders we will need in the future.”{\color{blue}17}
 \par 
Mannheim also argued that conservatives often champion the group—races or nations—rather than the individual. Races and
 \par 
Nations have unique identities, which must, in the name of freedom, be preserved. They are the modern equivalents of feudal estates. They have distinctive, and unequal, characters and functions; they enjoy different, and unequal, privileges. Freedom is the protection of those privileges, which are the outward expression of the group’s unique inner genius.{\color{blue}18}
 \par 
Goldwater rejected racism (though not nationalism); but try as he might, when discussing freedom he could not resist the tug of feudalism. He called states’ rights “the cornerstone” of liberty, “our chief bulwark against the encroachment of individual freedom” by the federal government. In theory, states protected individuals rather than groups. But who in 1960 were these individuals? Goldwater claimed that they were anyone and everyone, that states’ rights had nothing to do with Jim Crow. Yet even he was forced to admit that segregation “is, today, the most conspicuous expression of the principle” of states’ rights. {\color{blue}19} The rhetoric of states’ rights threw up a cordon around white privilege. While surely the most noxious plank in the conservative platform—eventually, it was abandoned—Goldwater’s argument for states’ rights fits squarely within a tradition that sees freedom as a shield for inequality and a surrogate for mass feudalism.
 \par 
Goldwater lost big in the 1964 presidential election. His children and grandchildren went on to win big—by broadening the circle of discontent beyond Southern whites to include husbands and wives, evangelicals and white ethnics, and by continuing to absorb and transmute the idioms of the left. {\color{blue}20} Adapting to the left didn’t make American conservatism less reactionary—any more than Maistre’s or Burke’s recognition that the French Revolution had permanently changed Europe tempered conservatism there. Rather, it made conservatism suppler and more successful. The more it adapted, the more reactionary conservatism became.
 \par 
Evangelical Christians were ideal recruits to the cause, deftly playing the victim card as a way of rejuvenating the power of whites. “It’s time for God’s people to come out of the closet,” declared a Texas televangelist in 1980. But it wasn’t religion that made evangelicals queer; it was religion combined with racism. One of the main catalysts of the Christian right was the defense of Southern private schools that were created in response to desegregation. By 1970, 400,000 white children were attending these “segregation academies.” States like Mississippi gave students tuition grants, and until the Nixon administration overturned the practice, the IRS gave donors to these schools tax exemptions. {\color{blue}21} According to New Right and direct-mail pioneer Richard Viguerie, the attack on these public subsidies by civil rights activists and the courts “was the spark that ignited the religious right’s involvement in real politics.” Though the rise of segregation academies “was often timed exactly with the desegregation of formerly all-white public schools,” writes one historian, their advocates claimed to be defending religious minorities rather than white supremacy (initially nonsectarian, most of the schools became evangelical over time). Their cause was freedom, not inequality—not the freedom to associate with whites, as the previous generation of massive resisters had claimed, but the freedom to practice their own embattled religion. {\color{blue}22} It was a shrewd transposition. In one fell swoop, the heirs of slaveholders became the descendants of persecuted Baptists, and Jim Crow a heresy the First Amendment was meant to protect.
 \par 
The Christian right was equally galvanized by the backlash against the women’s movement. Antifeminism was a latecomer to the conservative cause. Through the early 1970s, advocates of the Equal Rights Amendment (ERA) could still count Richard Nixon, George Wallace, and Strom Thurmond as supporters; even Phyllis Schlafl y described the ERA as something “between innocuous and
 \par 
Mildly helpful.” But once feminism entered “the sensitive and intensely personal arena of relations between the sexes,” writes historian Margaret Spruill, the abstract phrases of legal equality took on a more intimate and concrete meaning. The ERA provoked a counter-revolution, as we saw in chapter 1, led by Schlafl y and other women, that was as grassroots and nearly as diverse as the movement it opposed. {\color{blue}23} So successful was this counterrevolution—not just at derailing the ERA, but at propelling the Republican Party to power—that it seemed to prove the feminist point. If women could be that effective as political agents, why shouldn’t they be in Congress or the White House?
 \par 
Schlafl y grasped the irony. She understood that the women’s movement had tapped into and unleashed a desire for power and autonomy among women that couldn’t simply be quelled. If women were to be sent back to the exile of their homes, they would have to view their retreat not as a defeat, but as one more victory in the long battle for women’s freedom and power. As we saw in chapter 1, she described herself as a defender, not an opponent, of women’s rights. The ERA was “a takeaway of women’s rights,” she insisted, the “right of the wife to be supported and to have her minor children supported” by her husband. By focusing her argument on “the right of the wife in an ongoing marriage, the wife in the home,” Schlafl y reinforced the notion that women were wives and mothers first; their only need was for the protection provided by their husbands. At the same time, she described that relationship in the liberal language of entitlement rights. “The wife has the right to support” from her spouse, she claimed, treating the woman as a feminist claimant and her husband as the welfare state.{\color{blue}24}
 \par 
Like their Catholic predecessors in eighteenth-century France, the Christian Right appropriated not just the ideas but the manners and mores of its opponents. Billy Graham issued an album
 \par 
Called Rap Session: Billy Graham and Students Rap on Questions of Today’s Youth. Evangelicals criticized the culture of narcissism— and then colonized it. James Dobson of the Focus on the Family got his start as a child psychologist at the University of Southern California, competing with Dr. Spock as the author of a bestselling child-rearing text. Evangelical bookstores, according to historian Paul Boyer, “promoted therapeutic and self-help books off bring advice on finances, dating, marriage, depression, and addiction from an evangelical perspective.” Most audacious of all was the film version of Hal Lindsey’s book The Late Great Planet Earth. While the book popularized Christian prophecies of the End of Days, the film was narrated by Orson Welles, the original bad boy of the Popular Front.{\color{blue}25}
 \par 
The most interesting cases of the right’s appropriation of the left, however, came from big business and the Nixon administration. The business class saw the student movement as a critical constituency. Using hip and informal language, writes historian Bethany Moreton, corporate spokesmen left “their plaid suits in the closet” in order to sell capitalism as the fulfillment of sixties-style liberation, participation, and authenticity. Reeling from pro-tests against the invasion of Cambodia (and the massacre of four students that ensued), students at Kent State formed a chapter of Students in Free Enterprise (SIFE), one of {\color{blue}150} across the country. They sponsored a “Battle of the Bands,” for which one contestant wrote the following lyrics:
 \par 
You know I could never be happy Just working some nine-to-five. I’d rather spend my life poor. Then living it as a lie. If I could just save my money Or maybe get a loan,
 \par 
I could start my own business And make it on my own.
 \par 
Small business institutes were set up on college campuses, casting “the businessman as a victim, not a bully.” Business brought its Gramscian tactics to secondary schools as well. In Arkansas, SIFE performed classroom skits of Milton Friedman’s PBS series Free to Choose. In 1971, Arizona passed a law requiring high school graduates to take a course in economics so that they would have “some foundation to stand on,” according to the bill’s sponsor, when they came up “against professors that are collectivists or Socialists.” Twenty states followed suit. Arizona students could place out of the course if they passed an exam that asked them, among other things, to match the phrase “government intervention in a free enterprise system” with “is detrimental to the free market.”{\color{blue}26}
 \par 
The most ambidextrous of politicians, Nixon was the master of talking left while walking right. Nixon understood that the best response to the civil rights movement was not to defend whites against blacks, but to make whites into white ethnics burdened with their own histories of oppression and requiring their own liberation movements. Where immigrants from Southern and Eastern Europe had jumped into the melting pot and turned white, Nixon and the ethnic revivalists of the 1970s “provided Americans of European descent a new vehicle for asserting citizenship rights at a moment when it grew increasingly illegitimate to make claims on the state on the basis of whiteness,” write historian Tom Sugrue and sociologist John Skrentny. Under Nixon’s leadership, the Republican Party was transformed into a right-wing version of the Democratic urban machine. Poles and Italians were appointed to high-profile off ices in his administration, and Nixon campaigned vigorously in white ethnic neighborhoods. He even told one crowd that “he felt like he had Italian blood.” Nixon’s efforts occasionally
 \par 
Went beyond the symbolic—a 1971 proposal would have extended affirmative action to “members of certain ethnic groups, primarily of Eastern, Middle, and Southern European ancestry, such as Italians, Greeks, and Slavic groups”—but most were rhetorical. That didn’t make them less potent: the new vocabulary of white ethnicity helped create “a romanticized past of hard work, discipline, well-defined gender roles, and tight-knit families,” providing a new language for a new age—and a very old regime.{\color{blue}27}
 \par 
Barry Goldwater’s mother was a descendant of Roger Williams. His father, who converted to Episcopalianism, was a descendant of Polish Jews. When Goldwater ran in 1964, Harry Golden quipped, “I always knew the first Jew to run for president would be an Episcopalian.” {\color{blue}28} If the history of conservatism is any guide, perhaps he should have run as a Jew.