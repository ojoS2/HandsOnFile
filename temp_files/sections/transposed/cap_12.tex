{\chapter{Easy to Be Hard} } {\label{Easy to Be Hard} }{\par}{\textit{	I enjoy wars. Any adventures better than sitting in an offi CE.} } {\par}{\par} {\textbf{\textit{	—Harold Macmillan} } } {\par} 
{\footnote{This chapter originally appeared as “Easy to Be Hard: Conservatism and Violence,” in Performances of Violence, ed. Austin Sarah, Carleen Baler, and Thomas L. Dump (Amherst: University of Massachusetts Press, 2011), 18–42.} }	Despite the support among self-identified conservative voters and politicians for the death penalty, torture, and war, intellectuals on the right often deny any affinity between conservatism and violence. {\color{blue} 1 } “Conservatives,” writes Andrew Sullivan, “hate war.”{\textbf{\textit{Their domestic politics is rooted in a loathing of civil wars and violence, and they know that freedom is always the first casualty of international warfare. When countries go to war, their governments invariably get bigger and stronger, individual liberties are whittled away, and societies which once enjoyed the pluralist cacophony of freedom have to be marshaled into a single, collective note to face down an external foe. A state of permanent warfare—as George Orwell saw—is a virtual invitation to domestic tyranny. {\color{blue} 2 } } }} {\par} Channeling a tradition of skepticism from Makeshift to Hume, the conservative identifies limited government as the extent of his faith, the rule of law his one requirement for the pursuit of happiness. Pragmatic and adaptive, disposed rather than committed, such a sensibility—and it is a sensibility, the conservative insists, not an ideology—is not interested in violence. His endorsements of war, such as they are, are the weariest of concessions to reality. Unlike his friends on the left—conservative that he is, he values friendship more than agreement—he knows we live and love in the midst of great evil. This evil must be resisted, sometimes by violent means. All things being equal, he would like to see a world without violence. But all things are not equal, and he is not in the business of seeing the world as he’d like it to be.{\par} The historical record of conservatism—not only as a political practice, which is not my primary concern here, but as a theoretical tradition—suggests otherwise. Far from being saddened, burdened, or vexed by violence, the conservative has been enlivened by it. I don’t mean in a personal sense, though many a conservative, like Harold Macmillan quoted above or Winston Churchill quoted below, has expressed an unanticipated enthusiasm for violence. My concern is with ideas and argument rather than character or psychology. Violence, the conservative intellectual has maintained, is one of the experiences in life that makes us feel the most alive, and violence is an activity that makes life, well, lively. {\color{blue} 3 } Such arguments can be made nimbly—“Only the dead have seen the end of war,” as Douglas Mac arthur once put it {\color{blue} 4 } —or laboriously, as in the case of Tranche:{\par} {\textbf{\textit{To the historian who lives in the world of will it is immediately clear that the demand for a perpetual peace is thoroughly reactionary; he sees that with war all movement, all growth, must be struck out of history. It has always been the tired,} } }{\par} {\par} {\textbf{\textit{Unintelligent, and enervated periods that have played with the dream of perpetual peace. . . . However, it is not worth the trouble to discuss this matter further; the living God will see to it that war constantly returns as a dreadful medicine for the human race. {\color{blue} 5 } } } }{\par} Pithy or prolix, the case boils down to this: war is life, peace is death.{\par} This belief can be traced back to Edmund Burke’s A Philosophical Inquiry into the Origin of Our Ideas of the Sublime and the Beautiful. There Burke develops a view of the self desperately in need of negative stimuli of the sort provided by pain and danger, which Burke associates with the sublime. The sublime is most readily found in two political forms: hierarchy and violence. But for reasons that shall become clear, the conservative—again, consistent with Burke’s arguments—often favors the latter over the former. Rule may be sublime, but violence is more sublime. Most sublime of all is when the two are fused, when violence is performed for the sake of creating, defending, or recovering a regime of domination and rule. But as Burke warned, it’s always best to enjoy pain and danger at a remove. Distance and obscurity enhance sublimity; nearness and illumination diminish it. Counterrevolutionary violence may be the Everest of conservative experience, but one should view it from afar. Get too close to the mountaintop, and the air becomes thin, the view clouded. At the end of every discourse on violence, then, lies a waiting disappointment.{\par} The Sublime and the Beautiful begin on a high note, with a discussion of curiosity, which Burke identifies as “the first and simplest emotion.” The curious race “from place to hunt out something new.” Their sights are fixed, their attention is rapt. Then the world turns gray. They begin to stumble across the same things, “with less and less of any agreeable effect.” Novelty diminishes: how much, really, is there new in the world? Curiosity “exhausts” itself. Enthusiasm and engagement give way to “loathing and weariness.” {\color{blue} 6 } Burke moves on to pleasure and pain, which are supposed to transform the quest for novelty into experiences more sustaining and profound. But rather than a genuine additive to curiosity, plea-sure off ERS more of the same: a moment’s enthusiasm, followed by dull malaise. “When it has run its career,” Burke says, pleasure “sets us down very nearly where it found us.” Any kind of pleasure “quickly satisfies; and when it is over, we relapse into indifference.” {\color{blue} 7 } Quieter enjoyments, less intense than pleasure, are equally solo- right c. They generate complacency; we “give ourselves over to indolence and inaction.” {\color{blue} 8 } Burke turns to imitation as another potential force of outward propulsion. Through imitation, we learn manners and mores, develop opinions, and are civilized. We bring our-selves to the world, and the world is brought to us. But imitation contains its own narcotic. Imitate others too much and we cease to better ourselves. We follow the person in front of us “and so on in an eternal circle.” In a world of imitators, “there never could be any improvement.” Such “men must remain as brutes do, the same at the end that they are at this day, and that they were in the beginning of the world.” {\color{blue} 9 } {\par} Curiosity leads to weariness, pleasure to indifference, enjoyment to torpor, and imitation to stagnation. So many doors of the psyche open onto this space of inertial gloom we might well conclude that it lurks not at the edge, but at the center of the human condition. Here, in this dark courtyard of the self, all action ceases, creating an ideal environment for “melancholy, dejection, despair, and self-murder.” {\color{blue} 10 } Even love, the most outward of raptures, carries the self back to a state of internal dissolution. {\color{blue} 11 } Suicide, it seems, is the inevitable fate awaiting anyone who takes pleasure in the world as it is.{\par} For a certain type of conservative theorist, passages like these pose something of a challenge. Here is the inventor of the conservative tradition articulating a vision of the self dramatically at odds with the imagined self of conservative thought. The conservative self, as we have repeatedly seen, claims to prefer “the familiar to the unknown. . . The tried to the untried, fact to mystery, the actual to the possible, the limited to the unbounded, the near to the distant, the sufficient to the superabundant, the convenient to the perfect, present laughter to utopian bliss.” {\color{blue} 12 } He is partial to things as they are not because he finds things just or good, but because he finds them familiar. He knows them and is attached to them. He wishes neither to lose them nor to have them taken away. Enjoying what he has, rather than acquiring something better, is his highest good. But should the self of The Sublime and the Beautiful be assured of his attachments and familiars, he would quickly find himself confronting the specter of his own extinction, more than likely at his own hand.{\par} Perhaps it is this lethal ennui, lurking just beneath the surface of conservative discourse, that explains the failure of the conservative politician to follow the lead of the conservative theorist. Far from embracing the cause of quiet enjoyments and secure attachments, the conservative politician has consistently opted for an activism of the not-yet and the will-be. Ronald Reagan’s first inaugural address was a paean to the power of dreams: not small dreams but big, heroic dreams, of progress and betterment, and not dreams for their own sake, but dreams as a necessary and vital prod to action. Three months later, in an address before Congress, Reagan drove the point home with a quote from Carl Sandburg: “Nothing happens unless first a dream.” And nothing happening, or too few things happening, or things not happening quickly enough, is what the conservative in politics dislikes. Reagan could scarcely contain his impatience with the dithering of politicians:“The old and comfortable way is to shave a little here and add a little there. Well, that’s not acceptable anymore.” Old and comfort-able was the indictment, no “half-measures” the verdict. {\color{blue} 13 } {\par} Reagan was hardly the first conservative to act for the sake of the invisible and the ideal as against the material and the real. In his acceptance speech to the 1964 Republican National Convention, Barry Goldwater could find no more potent charge to level at the welfare state than that it had made a great nation “becalmed.” Thanks to the New Deal, the United States had lost its “brisk pace” and was now “plodding along.” Calm, slow, and plodding are usu-ally welcomed by the conservative theorist as signs of present bliss. But to the conservative politician, they are evils. He must declare war, rallying his armies against the listless and the languid with talk of “causes,” “struggle,” “enthusiasm,” and “devotion.” {\color{blue} 14 } {\par} That crusading zeal is not peculiar to American conservatism. It is found in Europe as well, even in England, the land that made moderation the moniker of conservatism. “Whoever won a battle,” scoff ed Margaret Thatcher, “under the banner ‘I stand for Consensus’?” {\color{blue} 15 } And then there is Winston Churchill, traveling to Cuba in 1895 to report on the Spanish war against Cuban independence. {\color{blue} 16 } Ruminating on the disappointments of his generation— latecomers to the Empire, they were deprived of the opportunity for imperial conquest (as opposed to administration)—he arrived in Havana. This is what he had to say (looking back on the experience in 1930):{\par} {\textbf{\textit{The minds of this generation, exhausted, brutalized, mutilated and bored by War, may not understand the delicious yet tremulous sensations with which a young British Officer bred in the long peace approached for the first time an actual the- are of operations. When first in the dim light of early morning I saw the shores of Cuba rise and defi né themselves} } }{\par} {\par} {\textbf{\textit{From dark-blue horizons, I felt as if I sailed with Long John Silver and first gazed on Treasure Island. Here was a place where real things were going on. Here was a scene of vital action. Here was a place where anything might happen. Here was a place where something would certainly happen. Here I might leave my bones. {\color{blue} 17 } } } }{\par} Whatever the relationship between theory and practice in the conservative tradition, it is clear from The Sublime and the Beautiful that if the self is to survive and flourish it must be aroused by an experience more vital and bracing than pleasure or enjoyment. Pleasure and enjoyment act like beauty, “relaxing the solids of the whole system.” {\color{blue} 18 } That system, however, must be made taut and tense. The mind must be quickened, the body exerted. Otherwise, the system will soften and atrophy, and ultimately die.{\par} What most arouses this heightened state of being is the confrontation with non-being. Life and health are pleasurable and enjoyable, and that is what is wrong with them: “they make no such impression” on the self because “we were not made to acquiesce in life and health.” Pain and danger, by contrast, are “emissaries” of death, the “king of terrors.” They are sources of the sublime, “the strongest”—most powerful, most affecting—“emotion which the mind is capable of feeling.” {\color{blue} 19 } Pain and danger, in other words, are generative experiences of the self.{\par} Pain and danger are generative because they have the contradictory effect of minimizing and maximizing our sense of self. When sensing pain or danger, our mind “is so entirely filled with its object, that it cannot entertain any other.” The “motions” of our soul “are suspended,” as harm and the fears it arouses “rush in upon the mind.” In the face of these fears, “the mind is hurried out of itself.” When we experience the sublime, we feel ourselves evacuated, overwhelmed by an external object of tremendous power and threat. Everything that gave us a sense of internal being and vitality ceases to exist. The external is all, we are nothing. God is a good example, and the ultimate expression, of the sublime: “Whilst we contemplate so vast an object, under the arm, as it were, of almighty power, and invested upon every side with omnipresence, we shrink into the minuteness of our own nature, and are, in a manner, annihilated before him.” {\color{blue} 20 } {\par} Paradoxically, we also feel our existence to an extent we never have felt it before. Seized by terror, our “attention” is roused and our “faculties” are “driven forward, as it were, on their guard.” We are pulled out of ourselves. We are cognizant of the immediate terrain and our presence upon it. Before, we barely noticed our-selves or our surroundings. Now we spill out of ourselves, inhabiting not only our bodies and minds but the surrounding space. We feel “a sort of swelling”—a sense that we are greater, our perimeter extends further—that “is extremely grateful to the human mind.” But this “swelling,” Burke reminds us, “is never more perceived, nor operates with more force, than when without danger we are conversant with terrible objects.” {\color{blue} 21 } {\par} In the face of the sublime, the self is annihilated, occupied, crushed, overwhelmed; in the face of the sublime, the self is heightened, aggrandized, magnified. Whether the self can truly occupy such opposing, almost irreconcilable, poles of experience at the same time—it is this contradiction, the oscillation between wild extremes, that generates a strong and strenuous sense of self. As Burke writes elsewhere, intense light resembles intense darkness not only because it blinds the eye and thus approximates darkness, but also because both are extremes. And extremes, particularly opposing extremes, are sublime because sublimity “in all things abhors mediocrity.” {\color{blue} 22 } The extremity of opposing sensations, the savage swing from being to nothingness, makes for the most intense experience of selfhood.{\par} The question for us, which Burke neither poses nor answers, here nor in his other work, is: What kind of political form entails this simultaneity of—or oscillation between—self-aggrandizement and self-annihilation? One possibility would be hierarchy, with its twin requirements of submission and domination; the other is violence, particularly warfare, with its rigid injunction to kill or be killed. Perhaps not coincidentally, both are of great significance to conservatism as a theoretical tradition and a historical practice.{\par} Rousseau and John Adams are not usually thought of as ideological bedfellows, but on one point they agreed: social hierarchies persist because they ensure that everyone, save those at the very bottom and the very top, enjoys the opportunity to rule and be ruled in turn. Not, to be sure, in the Aristotelian sense of self-governance, but in the feudal sense of reciprocal governance: each person dominates someone below him in exchange for submitting to someone above him. “Citizens only allow themselves to be oppressed to the degree that they are carried away by blind ambition,” writes Rousseau. “Since they pay more attention to what is below them than to what is above, domination becomes dearer to them than independence, and they consent to wear chains so that they may in turn give them to others. It is very difficult to reduce to obedience any-one who does not seek to command.” {\color{blue} 23 } The aspirant and the authoritarian are not opposing types: the will to rise precedes the will to bow. More than thirty years later, Adams would write that every man longs “to be observed, considered, esteemed, praised, beloved, and admired.” {\color{blue} 24 } To be praised, one must be seen, and the best way to be seen is to elevate oneself above one’s circle. Even the American democrat, Adams reasoned, would rather rule over an inferior to dispossess a superior. His passion is for supremacy, not equality, and so long as he is assured an audience of lesser, he will be con-tent with his lowly status:{\par} {\textbf{\textit{Not only the poorest mechanic, but the man who lives upon common charity, nay the common beggars in the streets. . . Court a set of admirers, and plume themselves on that superiority which they have, or fancy they have, over some others. . . . When a wretch could no longer attract the notice of a man, woman or child, he must be respectable in the eyes of his dog. “Who will love me then?” was the pathetic reply of one, who starved him-self to feed his mastiff, to a charitable passenger who advised him to kill or sell the animal. {\color{blue} 25 } } } }{\par} One can see in these descriptions of social hierarchy lineaments of the sublime: annihilated from above, aggrandized from below, the self is magnified and miniaturized by its involvement in the practice of rule. But here’s the catch: once we actually are assured of our power over another being, says Burke, our inferior loses her capacity to harm or threaten us. She loses her sublimity. “Strip” a creature “of its ability to hurt,” and “you spoil it of every thing sublime.” {\color{blue} 26 } Lions, tigers, panthers, and rhinoceroses are sublime not because they are magnify cent specimens of strength but because they can and will kill us. Oxen, horses, and dogs are also strong but lack the instinct to kill or have had that instinct suppressed. They can be made to serve us and in the case of dogs even love us. Because such creatures, how-ever strong, cannot threaten or harm us, they are incapable of sublimity. They are objects of contempt, contempt being “the attendant on a strength that is subservient and in noxious.” 27{\par} {\textbf{\textit{We have continually about us animals of a strength that is considerable, but not pernicious. Amongst these we never look for the sublime: it comes upon us in the gloomy forest, and in the howling wilderness. . . . Whenever strength is only useful, and employed for our benefit or our pleasure, then it is never sub-lime; for nothing can act agreeably to us, that does not act in} } }{\par} {\par} {\textbf{\textit{Conformity to our will; but to act agreeably to our will, it must be subject to us; and therefore can never be the cause of a grand and commanding conception. {\color{blue} 28 } } } }{\par} At least one-half, then, of the experience of social hierarchy— not the experience of being ruled, which carries the possibility of being destroyed, humiliated, threatened, or harmed by one’s superior, but the experience of easily ruling another—is incompatible with, and indeed weakens, the sublime. Confirmed of our power, we are lulled into the same ease and comfort, undergo the same inward melting, we experience while in the throes of pleasure. The assurance of rule is as debilitating as the passion of love.{\par} Burke’s intimations about the perils of long-established rule reflect a surprising strain within conservatism: a persistent, if unacknowledged, discomfort with power that has ripened and matured, authority that has grown comfortable and secure. Beginning with Burke himself, conservatives have expressed a deep unease about ruling classes so assured of their place in the sun that they lose their capacity to rule: their will to power dissipates; the muscles and intelligence of their command attenuate.{\par} As we saw in chapter 1, Burke believes that the Old Regime is beautiful. For that reason, it is also “sluggish, inert, and timid.” It cannot defend itself “from the invasions of ability,” with ability standing in for the new men of power that the Revolution brings forth. The moneyed interest, also allied with the Revolution, is stronger than the landed interest because it is “more ready for any adventure” and “more disposed to new enterprises of any kind.” {\color{blue} 29 } The Old Regime is beautiful, static, weak; the Revolution is ugly, dynamic, strong. “It is a dreadful truth,” Burke admits in the second of his Letters on a Regicide Peace, “but it is a truth that cannot be concealed; in ability, in dexterity, in the distinctness of their views, the Jacobins are our superiors.” {\color{blue} 30 } {\par} Joseph de Maistre was less tactful than Burke in his condemnations of the Old Regime, perhaps because he took its failings more personally. Long before the Revolution, he claims, the leadership of the Old Regime had been confused and bewildered. Naturally, the ruling classes were unable to comprehend, much less resist, the onslaught unleashed against them. Impotence, physical and cognitive, was—and remains—the Old Regime’s great sin. The aristocracy cannot understand; it cannot act. Some portion of the nobility may be well-meaning, but they cannot see their projects through. They are foppish and foolish. They have virtue but not virtue. The aristocracy “fails ridiculously in everything it undertakes.” The clergy has been corrupted by wealth and luxury. The monarchy consistently has shown that it lacks the will “to punish” that is the hallmark of every real sovereign. {\color{blue} 31 } Faced with such decadence, the inevitable outgrowth of centuries in power, Maître concludes it is a good thing the counterrevolution has not yet triumphed (he is writing in 1797). The Old Regime needs several more years in the wilderness if it is to shed the corrupting influences of its once beautiful life:{\par} {\textbf{\textit{The restoration of the throne would mean a sudden relaxation of the driving force of the state. The black magic working at the moment would disappear like mist before the sun. Kindness, clemency, justice, all the gentle and peaceful virtues, would suddenly reappear and would bring with them a general meekness of character, a certain cheerfulness entirely opposed to the rigors of the revolutionary regime. {\color{blue} 32 } } } }{\par} A century later, a similar case will be made by Georges Sorel against the belle époque. Sore is not usually seen as an emblematic figure of the right—then again, even Burke’s conservatism remains a subject of dispute {\color{blue} 33 } —and, indeed, his greatest work, Reflections on violence, is often thought of as a contribution, albeit minor, to the Marxist tradition. Yet Sore’s beginnings are conservative and his endings proto-fascist, and even in his Marxist phase his primary worry is decadence and vitality rather than exploitation and justice. The criticisms he lodges against the French ruling classes at the end of the nineteenth century are not dissimilar to those made by Burke and Maître at the end of the eighteenth. He even makes the comparison explicit: the French bourgeoisie, Sore writes, “has become almost as stupid as the nobility of the eighteenth century.” They are “an ultra-civilized aristocracy that demands to be left in peace.” Once, the bourgeoisie was a race of warriors. “Bold captains,” they were “creators of new industries” and “discovers of unknown lands.” They “directed gigantic enterprises,” inspired by that “conquering, insatiable and pitiless spirit” that laid railroads, subdued continents, and made a world economy. Today, they are timid and cowardly, refusing to take the most elemental steps to defend their own interests against unions, socialists, and the left. Rather than unleash violence against striking workers, they surrender to the workers’ threat of violence. They lack the ardor, the fire in the belly, of their ancestors. It is difficult not to conclude that “the bourgeoisie is condemned to death and that its disappearance is only a matter of time.” {\color{blue} 34 } {\par} Carl Schmitt formalized Sore’s contempt for the weaknesses of the ruling classes into an entire theory of politics. According to Schmitt, the bourgeois was as he was—risk-averse, selfish, uninterested in bravery or violent death, desirous of peace and security— because capitalism was his calling and liberalism his faith. Neither provided him with a good reason for dying for the state. In fact, both gave him good reasons, indeed an entire vocabulary, not to die for the state. Interest, freedom, profit, rights, property, individualism, and other such words had created one of the most self-absorbed ruling classes in history, a class that enjoyed privilege but did not feel itself obliged to defend that privilege. After all, the premise of liberal democracy was the separation of politics from economics and culture. One could pursue profit, at someone else’s expense, and think freely, no matter how subversive the thoughts, without disrupting the balance of power. The bourgeoisie, however, were confronting an enemy that very much understood the connections between ideas, money, and power, that economic arrangements and intellectual arguments were the stuff of political combat. Marxists got the friend-enemy distinction, which is constitutive of politics; the bourgeoisie did not. {\color{blue} 35 } The spirit of Hegel used to reside in Berlin; it has long since “wandered to Moscow.” {\color{blue} 36 } {\par} Sore identified one exception to this rule of capitalist decadence: the robber barons of the United States. In the Carnegie's and the Gould's of American industry, Sore thought he saw “the indomitable energy, the audacity based on an accurate appreciation of strength, the cold calculation of interests, which are the qualities of great generals and great capitalists.” Unlike the pampered bourgeoisie of France and Britain, the millionaires of Pitts-burgh and Pitts ton “lead to the end of their lives a galley-slave existence without ever thinking of leading a nobleman’s life, as the Rothschild's do.” {\color{blue} 37 } {\par} Sore’s spiritual counterpart across the Atlantic, Teddy Roosevelt, was not so sanguine about American industrialists and phi Nan- years. (Burke an anxiety about the ruling classes is common to the European and American conservative.) The capitalist, Roosevelt declared, sees his country as a “till,” always weighing the “the honor of the nation and the glory of the flag” against a “temporary interruption of money-making.” He is not “willing to lay down his life for little things” like the defense of the nation. He cares “only whether shares rise or fall in value.” {\color{blue} 38 } He shows no interest in great affairs of state, domestic or international, unless they impinge upon his own. It was no accident, Roosevelt claimed,Perhaps with a nod to Carnegie, that such men opposed the great imperial expedition that was the Spanishamerican War. {\color{blue} 39 } Complacent and comfortable, assured of their riches by the success of the labor wars of previous decades and the election of 1896, these were not men who could be counted upon to defend the nation or even themselves. “We may some day have bitter cause,” Roosevelt declared, “to realize that a rich nation which is slothful, timid, or unwieldy is an easy prey” for other, more martial peoples. The danger facing a ruling class, and a ruling nation, that has grown “skilled in commerce and finance” is that it “loses the hard fighting virtues.” {\color{blue} 40 } {\par} Roosevelt was hardly the first American conservative to worry about ruling classes gone soft and hierarchies overripe with power. Nor would he be the last. Throughout the 1830s, we saw in chapter 1, as the abolitionists began pressing their cause, John C. Calhoun drove himself into a rage over the easy living and willed cluelessness of his comrades on the plantation. They had grown lazy, fat, and complacent, so roundly enjoying the privileges of their position that they could not see the coming catastrophe. Or, if they could, the Southern planters couldn’t do anything to fend it off, their political and ideological muscles having atrophied long ago. {\color{blue} 41 } Barry Goldwater likewise expressed contempt for the Republican Establishment. {\color{blue} 42 } And throughout the 1990s—to jump ahead by another three decades—one could hear Roosevelt’s heirs on the right direct the same venom against the American capitalist at the masters of the universe on Wall Street and the geeky entrepreneurs of Silicon Valley. {\color{blue} 43 } {\par} If the ruling class is to be vigorous and robust, the conservative has concluded, its members must be tested, exercised, and challenged. Not just their bodies, but also their minds, even their souls. Echoing Milton—“I cannot praise a fugitive and cloistered virtue, unexercised and unbreathed, that never sallies out and sees her adversary, but slinks out of the race. . . . That which purifies us is trial, and trial is by what is contrary” {\color{blue} 44 } —Burke believes that adversity and difficulty, the confrontation with affliction and suffering, make for stronger, more virtuous beings.{\par} {\textbf{\textit{The great virtues turn principally on dangers, punishments, and troubles, and are exercised rather in preventing mischiefs, than in dispensing favors; and are therefore not lovely, though highly venerable. The subordinate turn on reliefs, gratify cations, and indulgences; and are therefore more lovely, though inferior in dignity. Those persons who creep into the hearts of most people, who are chosen as the companions of their softer hours, and their reliefs from care and anxiety, are never persons of shining qualities, nor strong virtues. {\color{blue} 45 } } } }{\par} Perhaps we see here the origins of the conservative preference for warfare over the welfare state, but that is another topic for another day). But where Milton and other like-minded republicans believe that impurity and corruption await the complacent and the comfortable, Burke espies the more terrifying specter of dissipation, degeneration, and death. If the powerful are to remain powerful, if they are to remain alive at all, their power, indeed the credibility of their own existence, must be continuously challenged, threatened, and defended.{\par} One of the more arresting—though I hope by now intelligible— features of conservative discourse is the fascination, indeed appreciation, one finds for the conservative’s enemies, particularly for their use of violence against him and his allies. Maître’s most rapturous comments in his Considerations on France are reserved for the Jacobins, whose brutal will and penchant for violence—their “black magic”—he plainly envies. Thanks to their efforts, France has been purified and restored to its rightful pride of place among the family of nations. They have rallied the people against foreign invaders, a “prodigy” that “only the infernal genius of Robespierre could accomplish.” Unlike the monarchy, the Revolution has the will to punish. {\color{blue} 46 } {\par} From the perspective of the Burke an sublime, however, Moist- re’s argument only goes so far. The Revolution rejuvenates the Old Regime by forcing it from power and purifying the people through violence. It delivers a clarifying shock to the system. But Maître never contemplates, or at least never discusses, the revivifying effect that wresting power back from the Revolution might have on the leaders of the Old Regime. And indeed, once he gets around to describing how he thinks the counterrevolution will occur, the final battle turns out to be a stunningly anticlimactic AFF air, with scarcely a shot phi red at all. “How Will the Counterrevolution Happen if it Comes?” Maître asks. “Four or five persons, perhaps, will give France a king.” Not exactly the stuff of a virile, transformed ruling class, battling its way back to power. {\color{blue} 47 } {\par} Maître never contemplated the restorative possibilities of hand-to-hand combat between the Old Regime and the Revolution; for this one must turn to Sore. And while Sore’s allegiances in the war between the rulers and the ruled of the late nineteenth century are more ambiguous than Maître’s, his account of the effect of the violence of the ruled upon the rulers is not. The French bourgeoisie has lost its fighting spirit, Sore claims, but that spirit is alive and well among the workers. Their battlefield is the workplace, their weapon is the general strike, and their aim is the overthrow of the state. It is the last that most impresses Sore, for the desire to overthrow the state signals just how unconcerned the workers are about “the material prof its of conquest.” Not only do they not seek higher wages and other improvements in their well-being; instead they have set their sights on the most improbable of goals—overthrowing the state by a general strike. It is that improbability, the distance between means and ends, that makes the violence of the proletariat so glorious. The proletarians are like Homeric warriors, absorbed in the grandeur of the battle and indifferent to the aims of the war: Who really has ever overthrown a state by a general strike? Theirs is a violence for its own sake, without concern for costs, benefits, and the calculations in between. {\color{blue} 48 } As Ernst Jünger wrote a generation later, it “is not what we fight for but how we fight.” {\color{blue} 49 } {\par} But what grips Sore is not the proletariat but the rejuvenating effects it might have on the bourgeoisie. Can the violence of the general strike “give back to the bourgeoisie an ardor which is extinguished?” Certainly the vigor of the proletariat might reawaken the bourgeoisie to its own interests and the threats its withdrawal from politics has posed to those interests. More tantalizing to Sore, how-ever, is the possibility that the violence of workers will “restore to [the bourgeoisie] the warlike qualities it formerly possessed,” forcing the “capitalist class to remain ardent in the industrial struggle.” Through the struggle against the proletariat, in other words, the bourgeoisie may recover its ferocity and ardor. And ardor is every-thing. From ardor alone, that splendid indifference to reason and self-interest, an entire civilization, drowning in materialism and complacency, will be reawakened. A ruling class, threatened by violence from the ruled, roused to its own taste for violence—that is the promise of the civil war in France. {\color{blue} 50 } {\par} For the conservative, no matter how modulated or moderate, a renewed vigor has always been the promise of civil war. For between the easy cases of a Catholic reactionary like Maître and a proto-fascist like Sore stands the more difficult but ultimately more revealing example of Alexis de Tocqueville. His drift from the moderation of the July Monarchy to the revanchism of 1848 demonstrates how easily and inexorably the Burke an conservative will swing from the beautiful to the sublime, how the music of prudence and moderation gives way to the march of violence and vitriol. {\color{blue} 51 } {\par} Publicly presenting himself as the consummate realist, discriminating and judicious, with little patience for enthusiasm of any sort, Tocqueville was actually a closet romantic. He confessed to his brother that he shared their father’s “devouring impatience,” his “need for lively and recurring sensations.” Reason, he said, “has always been for me like a cage,” behind which he would “gnash [his] teeth.” He longed for “the sight of combat.” Looking back on the French Revolution, which he missed (he was born in 1805), he lamented the end of the Terror, claiming that “men thus crushed can not only no longer attain great virtues, but they seem to have become almost incapable of great crimes.” Even Napoleon, scourge of conservatives, moderates, and liberals everywhere, earned Tocqueville’s admiration as the “most extraordinary being who has appeared in the world for many centuries.” Who, by contrast, could find inspiration in the parliamentary politics of the July Monarchy, that “little democratic and bourgeois pot of soup”?Yet once he set upon a career in politics, it was into that little bourgeois pot of soup that Tocqueville jumped. Predictably, it was not to his taste. Tocqueville may have mouthed the words of moderation, compromise, and the rule of law, but they did not move him. Without the threat of revolutionary violence, politics was simply not the grand drama he imagined it had been between 1789 and 1815. “Our fathers observed such extraordinary things that compared with them all of our works seem commonplace.” The politics of moderation and compromise produced moderation and compromise; it did not produce politics, at least not as Tocqueville understood the term. During the 1830s and 1840s, “what was most wanting. . . Was political life itself.” There was “no battlefield for contending parties to meet upon.” Politics had been “deprived” of “all originality, of all reality, and therefore of all genuine passions.”Then came 1848. Tocqueville didn’t support the Revolution. Indeed, he was among its most vociferous opponents. He voted for the full suspension of civil liberties, which he happily announced was done “with even more energy than had been done under the Monarchy.” He welcomed talk of a dictatorship—to protect the very regime he had spent the better part of two decades disparaging. And he loved it all: the violence, the counterviolence, the battle. Defending moderation against radicalism, Tocqueville was given a chance to use radical means for moderate ends, and it is not entirely clear which of the two most stirred him.{\par} {\textbf{\textit{Let me say, then, that when I came to search carefully into the depths of my own heart, I discovered, with some surprise, a certain sense of relief, a sort of gladness mingled with all the griefs and fears to which the Revolution had given rise. I suffered from this terrible event for my country, but clearly not for myself; on the contrary, I seemed to breathe more freely than before the catastrophe. I had always felt myself stifled in the atmosphere of the parliamentary world which had just been destroyed: I had found it full of disappointments, both where others and where I myself was concerned.} } }{\par} A self-styled poet of the tentative, the subtle, and the complex, Tocqueville burned with enthusiasm upon waking up to a world divided into two camps. Timid parliaments sowed a gray confusion; civil war forced upon the nation a bracing clarity of black and white. “There was no field left for uncertainty of mind: on this side lay the salvation of the country; on that, its destruction. . . . The road seemed dangerous, it is true, but my mind is so constructed that it is less afraid of danger than of doubt.” For this member of the ruling class, sublimity welling up from the violence of the lower orders off and an opportunity to escape the stifling beauty of life on the bourgeois Parnassus.{\par} Francis Fukuyama is perhaps the most thoughtful of recent writers to pursue this conservative line of argument about violence. Unlike Maître, however, or Tocqueville and Sore—all of whom wrote in the midst of battle, when the outcome was unclear—Fukuyama writes from the vantage of victory. It is 1992, and the capitalist classes have beaten their socialist opponents in the long civil war of the short twentieth century. It is not a pretty sight, at least not for Fukuyama. For the revolutionary was one of the few thematic men of the twentieth century. Thematic man is like Sore’s worker: he who risks his life for the sake of an improbable principle, who is unconcerned with his own material interests and cares only for honor, glory, and the values for which he fights. After a strange but brief homage to the Bloods and the Crops as thematic men, Fukuyama looks back fondly to men of purpose and power like Lenin, Trotsky, and Stalin, “striving for something purer and higher” and possessed of “greater than usual hardness, vision, ruthlessness, and intelligence.” By virtue of their refusal to accommodate them-selves to the reality of their times, they were the “most free and therefore the most human of beings.” But somehow or other, these men and their successors lost the civil war of the twentieth century, almost inexplicably, to the forces of “Economic Man.” For Economic Man is “the true bourgeois.” Such a man would never be “willing to walk in front of a tank or confront a line of soldiers” for any cause, even his own. Yet Economic Man is the victor, and far from rejuvenating or restoring him to his primal powers, the war seems only to have made him more bourgeois. Conservative that he is, Fukuyama can only chafe at the triumph of Economic Man and “the life of rational consumption” he has brought about, a life that is “in the end, boring.” {\color{blue} 52 } {\par} Far from being exceptional, Fukuyama’s disappointment about the actual—as opposed to anticipated or fantasized—effect of violence on a dissipated ruling class is emblematic. “The aims of battle and the fruits of conquest are never the same,” E. M. Forster observed in A Passage to India. “The latter have their value and only the saint rejects them, but their hint of immortality vanishes as soon as they are held in the hand.” {\color{blue} 53 } Deep within the conservative discourse lurks an element of anticlimax that cannot be contained. While the conservative turns to violence as a way of liberating himself, or the ruling classes, from the deadening ennui and softening atrophy that comes with power, virtually every encounter in conservative discourse with actual violence entails disillusion and deflation.{\par} Recall Teddy Roosevelt, brooding on the materialism and weakness of America’s capitalist classes. Where, he wondered, could one find an example of the “strenuous life”—the thrill of difficulty and danger, the strife that made for progress—in contemporary America? Perhaps in the foreign wars and conquests America had undertaken at the end of the century. Yet even here Roosevelt encountered frustration. Though his reports from the Spanishamerican War were filled with bravery and bravado, a careful reading of his adventures in Cuba suggests that his exploits there were a fiasco. Each of the famous charges Roosevelt led up or down a hill was an anticlimax. The first culminated with him seeing exactly two Spanish soldiers felled by his men: “These were the only Spaniards I actually saw fall to aimed shots by any one of my men,” he wrote, “except two guerillas in trees.” The second found him leading an army that neither heard nor followed him. So it was with a grim appreciation that he recited the dyspeptic comments of one of the army’s leaders in Cuba, a certain General Wheeler, who “had been through too much heavy fighting in the Civil War to regard the present fight as very serious.” {\color{blue} 54 } {\par} In the bloody occupations that followed the Spanishamerican War, however, Roosevelt thought he saw the true bliss it was in that dawn to be alive. Roosevelt was sure that America’s occupations of the Philippines and elsewhere were as close to a replay of the Civil War—that noble crusade of unsullied virtue—as he and his countrymen were ever likely to see. “We of this generation do not have to face a task such as that our fathers faced,” he declared in 1899, “and woe to us if we fail to perform them! . . . We cannot avoid the responsibilities that confront us in Hawaii, Cuba, Porto [sic] Rico, and the Philippines.” Here—in the islands of the Caribbean and the Pacific—was the confluence of blood and purpose he had been searching for his entire life. The task of imperial uplift, of educating the natives in “the cause of civilization,” was arduous and violent, imposing a mission upon America that would take years, God willing, to fulfill. If the imperial mission succeeded—and even if it failed—it would create a genuine ruling class in America, hardened and made strenuous by battle, nobler and less grubby-minded than Carnegie’s minions. {\color{blue} 55 } {\par} It was a beautiful dream. But it, too, could not bear the weight of reality. Though Roosevelt hoped the men who ruled the Philip-pines would be “chosen for signal capacity and integrity,” running “the provinces on behalf of the entire nation from which they come, and for the sake of the entire people to which they go,” he worried that America’s colonial occupiers would come from the same class of selfish financiers and industrialists that had driven him abroad in the first place. And so his paeans to imperialism ended on a sour note of warning, even doom. “If we permit our public service in the Philippines to become the prey of the spoils politicians, if we fail to keep it up to the highest standard, we shall be guilty of an act, not only of wickedness, but of weak and short-sighted folly, and we shall have begun to tread the path which was trod by Spain to her own bitter humiliation.” {\color{blue} 56 } {\par} But if his dream ended badly, Roosevelt at least had the advantage of being able to say that he always suspected it would. The same could not be said of the Fascists of Italy, whose self-deception about the wresting of power from the left persisted for decades, testifying to an inability to confront their own disappointment. For years, the Fascists celebrated the 1922 March on Rome as the violent and glorious triumph of will over adversity. October 28, the day of the Blackshirts’ arrival in Rome, became a national holiday; it was declared the first day of the Fascist New Year upon the introduction of the new calendar in 1927. The story of Mussolini’s arrival in particular—wearing the proverbial black shirt—was repeated with awe. “Sire,” he supposedly said to King Victor Emmanuel III, “forgive my attire. I come from the battlefields.” In actual fact, Mussolini traveled by train overnight from Milan, where he had been conspicuously attending the theater, snoozing comfortably in the sleeping car. The only reason he even made it into Rome was that a timid establishment, led by the king, telephoned him in Milan with a request that he form a government. Barely a shot was phi red, on either side. {\color{blue} 57 } Maître could not have written it better.{\par} We can see a similar phenomenon at play in the war on terror. Though many view the Bush administration and neoconservatism as departures from proper conservatism—the most recent statement of this thesis being Sam Tannhauser’s The Death of Conservatism {\color{blue} 58 } —the neocon project of imperial adventurism traces the Burke an arc of violence from beginning to end. I have already dis-cussed, in chapter 8, how the neoconservatives saw 9/11 and the war on terror as a chance to escape from the decadent and deadening peace and prosperity of the Clinton years, which they believed had weakened American society. Oozing in comfort, Americans—and more important their leaders—had supposedly lost the will, the desire and ability, to govern the world. Then 9/11 happened, and suddenly it seemed as if they could.{\par} That dream, of course, now lies in tatters, but one of its more idiosyncratic aspects is worth noting, for it presents a wrinkle in the long saga of conservative violence. According to many conservatives, and not just the neocons, one of the recent sources of American decadence, traceable back to the Warren Court and the rights revolutions of the 1960s, is the liberal obsession with the rule of law. This obsession, in the eyes of the conservative, takes many forms: the insistence on due process in criminal procedure; a partiality to litigation over legislation; an emphasis on diplomacy and international law over war; attempts to restrain executive power through judicial and legislative oversight. However unrelated these symptoms may seem, conservatives see in them a single disease: a culture of rules and laws slowly disabling and devitalizing the blond beast of prey that is American power. These are signs of a Nietzsche an unhealthiness, and 9/11 was the inevitable result.{\par} If another 9/11 is to be prevented, that culture of rights and rules must be repudiated and reversed. As the reporting of Seymour Harsh and Jane Mayer makes clear, the war on terror—with its push for torture, for overturning the Geneva Conventions, for refusing the restrictions of international law, for illegal surveil-lance, and for seeing terrorism through the lens of war rather than of crime and punishment—reflects as much, if not more, these conservative sensibilities and sensitivities as it does the actual facts of 9/11 and the need to prevent another attack. {\color{blue} 59 } “She’s soft—too soft,” says now-retired Lieutenant General Jerry Bodkin about the United States, pre- and post-9/11. The way to make her hard is not merely to undertake difficult and strenuous military action but also to violate the rules—and the culture of rules—that made her soft in the first place. The United States must learn how to “live on the edge,” says former NSA director Michael Hayden. “There’s nothing we won’t do, nothing we won’t try,” former CIA director George Tenet helpfully adds. {\color{blue} 60 } {\par} The great irony of the war on terror is that far from emancipating the blond beast of prey, the war has made law, and lawyers,Far more critical than one might imagine. As Mayer reports, the push for torture, unbridled executive power, the overthrow of the Geneva Conventions, and so on came not from the CIA or the military; the driving forces were lawyers in the White House and the Justice Department like David Addington and John Yew. Far from Machiavellian virtuosos of transgressive violence, Addington and Yew are fanatics about the law and insist on justifying their violence through the law. Lawyers, moreover, consistently oversee the actual practice of torture. As Tenet wrote in his memoir, “Despite what Hollywood might have you believe, in situations like this [the capture, interrogation, and torture of Alqaeda logistics chief Abu Zubaydah] you don’t call in the tough guys; you call in the lawyers.” Every slap on the face, every punch in the gut, every shake of the body—and much, much worse—must first be approved by high- er-ups in the various intelligence agencies, inevitably in consultation with attorneys. Mayer compares the practice of torture to a game of “Mother, May I?” As one interrogator states, “Before you could lay a hand on him [the torture victim], you had to send a cable saying, ‘He’s uncooperative. Request permission to do X.’ And permission would come, saying ‘You’re allowed to slap him one time in the belly with an open hand.” {\color{blue} 61 } {\par} Rather than free the blond beast to roam and prey as he wishes, the removal of the ban on torture and the suspension of the Geneva Conventions have made him, or at least the lawyers who hold his leash, more anxious. How far can he go? What can he do? Every act of violence, as this exchange between two Pentagon lawyers reveals, becomes a law school seminar:{\par} {\textbf{\textit{What did “deprivation of light and auditory stimuli” mean? Could a prisoner be locked in a completely dark cell? If so, could he be kept there for a month? Longer? Until he went blind? What, precisely, did the authority to exploit phobias permit? Could a} } }{\par} {\par} {\textbf{\textit{Detainee be held in a coffin? What about using dogs? Rats? How far could an interrogator push this? Until a man went insane? 62} } }{\par} Then there is the question of combining approved techniques of torture. May an interrogator withhold food from the prisoner and turn down the temperature of his cell at the same time? Does the multiplying effect of pains doubled and tripled cross a never-defined line? {\color{blue} 63 } As Orwell taught, the possibilities for cruelty and violence are as limitless as the imagination that dreams them up. But the armies and agencies of today’s violence are vast bureaucracies, and vast bureaucracies need rules. Eliminating the rules does not Prometheus unbind; it just makes for more billable hours.“No yielding. No equivocation. No layering this thing to death.” That was George W. Bush’s vow after 9/11 and his description of how the war on terror would be conducted. Like so many of Bush’s other declarations, it turned out to be an empty promise. This thing was layered to death. But, and this is the critical point, far from minimizing state violence—which was the great fear of the neocons—layering has proven to be perfectly compatible with violence. In a war already swollen with disappointment and disillusion, the realization that inevitably follows—the rule of law can, in fact, authorize the greatest adventures of violence and death, thereby draining them of sublimity—must be, for the conservative, the greatest disillusion of all.{\par} Had they been closer readers of Burke, the neoconservatives—like Fukuyama, Roosevelt, Sore, Schmitt, Tocqueville, Maître, Tre- Ritchie, and so many more on the American and European right— could have seen this disillusion coming. Burke certainly did. Even as he wrote of the sublime effects of pain and danger, he was careful to insist that should those pains and dangers “press too nearly” or “too close”—that is, should they become realities rather than fantasies, should they become “conversant about the present destruction of the person”—their sublimity would disappear. They would cease to be “delightful” and restorative and become simply terrible. {\color{blue} 64 } Burke’s point was not merely that no one, in the end, really wants to die or that no one enjoys unwelcome, excruciating pain. It was that sublimity of whatever kind and source depends upon obscurity: get too close to anything, whether an object or experience, see and feel its full extent, and it loses its mystery and aura. It becomes familiar. A “great clearness” of the sort that comes from direct experience “is in some sort an enemy to all enthusiasms whatsoever.” {\color{blue} 65 } “It is our ignorance of things that causes all our admiration, and chiefly excites our passions. Knowledge and acquaintance make the most striking causes affect but little.” {\color{blue} 66 } “A clear idea,” Burke concludes, “is therefore another name for a little idea.” {\color{blue} 67 } Get to know anything, including violence, too well, and it loses whatever attribute—rejuvenation, transgression, excitement, awe—you ascribed to it when it was just an idea. Earlier than most, Burke understood that if violence were to retain its sublimity, it had to remain a possibility, an object of fantasy—a horror movie, a video game, an essay on war. For the actuality (as opposed to the representation) of violence was at odds with the requirements of sublimity. Real, as opposed to imagined, violence entailed objects getting too close, bodies pressing too near, flesh upon flesh. Violence stripped the body of its veils; violence made its antagonists familiar to each other in a way they had never been before. Violence dispelled illusion and mystery, making things drab and dreary. That is why, in his discussion in the Reflections of the revolutionaries’ abduction of Marie Antoinette, Burke takes such pains to emphasize her “almost naked” body and turns so effortlessly to the language of clothing—“the decent drapery of life,” the “ward-robe of the moral imagination,” “antiquated fashion,” and so on—to describe the event. {\color{blue} 68 } The disaster of the revolutionaries’ violence, for Burke, was not cruelty; it was the unsought enlightenment.{\par} Since 9/11, many have complained, and rightly so, about the failure of conservatives—or their sons and daughters—to fight the war on terror themselves. For those on the left, that failure is symptomatic of the class injustice of contemporary America. But there is an additional element to the story. So long as the war on terror remains an idea—a hot topic on the blogs, a provocative op-ed, an episode of {\color{blue} 24 } —it is sublime. As soon as the war on terror becomes a reality, it can be as cheerless as a discussion of the tax code and as tedious as a trip to the DMV.{\par}