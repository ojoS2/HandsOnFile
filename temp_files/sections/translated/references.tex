


\section{N O T E S}


\section{Introduction}


	{\color{blue}1}. At the turn of the twentieth century, 98% of the overwhelmingly Republican —and anti-union—federal judiciary came from “the very top of the nation’s class and status hierarchies.” William E. Forbath, Law and the Shaping of the American Labor Movement (Cambridge, Mass.: Harvard Univer- sity Press, 1991), 33 .


	{\color{blue}2}. Even today, marital rape is punished with less severity—and requires prose- cutors to mount greater obstacles—than nonmarital rape. According to one scholar, “The marital rape exemption survives in some substantial form in a majority of states.” Jill Elaine Hasday, “Contest and Consent: A Legal History of Marital Rape,” California Law Review 88 (October 2000): 1375, 1490 ; Rebecca M. Ryan, “The Sex Right: A Legal History of the Marital Rape Exemption,” Law & Social Inquiry 20 (Autumn 1995): 941–942, 992–995 ; Nancy F. Cott, Public Vows: A History of Marriage and the Nation (Cambridge, Mass.: Harvard University Press, 2000), 211 .


	{\color{blue}3}. It should be pointed out that before the marital rape exemption was elimi- nated, sexual violence had come to be considered one of the few legitimate grounds for divorce. Hasday, “Contest and Consent,” 1397–1398, 1475–1484; Ryan, “Sex Right,” 941; Cott, Public Vows, 195, 203.


	{\color{blue}4}. Karen Orren, Belated Feudalism: Labor, the Law, and Liberal Development in the United States (New York: Cambridge University Press, 1991) ; Robert J. Stein- feld, Coercion, Contract, and Free Labor in the Nineteenth Century (New York: Cambridge University Press, 2001) ; Forbath, Shaping of the American Labor Movement .


	{\color{blue}5}. Greg Grandin, The Last Colonial Massacre: Latin America in the Cold War (Chi- cago: University of Chicago Press, 2004), 56–57 . The outbreak of political speech among those without power was also, according to a disgruntled Democrat writing to liberal Senator Paul Douglas in the 1960s, the great evil of the Great Society: “I feel Mr. Johnson is much responsible for the present riot by his constant encouragement for the Negro to take any measure to assert himself & DEMAND his rights.” Rick Perlstein, Nixonland: The Rise of a President and the Fracturing of America (New York: Scribner, 2008), 117 . 249


	{\color{blue}6}. John C. Calhoun, “Speech on the Admission of California—and the General State of the Union” (March 4, 1850), in Union and Liberty: The Political Philos- ophy of John C. Calhoun , ed. Ross M. Lence (Indianapolis: Liberty Fund, 1992), 583–585 .


	{\color{blue}7}. Alexander Keyssar, The Right to Vote: The Contested History of Democracy in the United States (New York: Basic, 2000), 112 .


	{\color{blue}8}. Jeremy Brecher, Strike! (Cambridge, Mass.: South End Press, 1997), 34, 126 . Also see Kim Phillips-Fein, Invisible Hands: The Businessmen’s Crusade against the New Deal (New York: Norton, 2009), 87–114 .


	{color{blue}9}. Forbath, Shaping of the American Labor Movement, 65.


	{\color{blue}10}. James Boswell, Life of Johnson , ed. R. W. Chapman and J. D. Fleeman (New York: Oxford University Press, 1998), 1017 .


	{\color{blue}11}. Edmund Burke, Refl ections on the Revolution in France , ed. J. C. D. Clark (Stan- ford, Calif.: Stanford University Press, 2001), 205–206 .


	{\color{blue}12}. Ibid ., 217–218 .


	{color{blue}13}. Cited in Daniel T. Rodgers, Age of Fracture (Cambridge, Mass.: Harvard University Press, 2011), 207 .


	{\color{blue}14}. Friedrich Hayek, Law, Legislation and Liberty, vol. 2, The Mirage of Social Justice (Chicago: University of Chicago Press, 1976), 84–85 ; Robert Nozick, Anarchy, State, and Utopia (New York: Basic Books, 1974), 235–238 .


	{\color{blue}15}. G. A. Cohen, Self-Ownership, Freedom, and Equality (New York: Cambridge University Press, 1995), 28–32, 53–59, 98–115, 236–238 .


	{\color{blue}16}. Cited in Friedrich A. Hayek, The Constitution of Liberty (Chicago: University of Chicago Press, 1960), 424; also see 16–19 .


	{\color{blue}17}. Elizabeth Cady Stanton, “Home Life,” in The Elizabeth Cady Stanton–Susan B. Anthony Reader , ed. Ellen Carol DuBois (Boston: Northeastern University Press, 1981, 1992), 132 . Also see Cott, Public Vows , 67; Amy Dru Stanley, From Bondage to Contract: Wage Labor, Marriage, and the Market in the Age of Slave Emancipation (New York: Cambridge University Press, 1998), 177–178 .


	{\color{blue}18}. Sometimes the transcript is not so hidden. Point Four of the 1948 platform of Strom Thurmond’s States’ Rights Democratic Party—the Dixiecrats— weaves together the public and private in a seamless and visible whole: “We stand for the segregation of the races and the racial integrity of each race; the constitutional right to choose one’s associates; to accept private employ- ment without governmental interference, and to earn one’s living in any lawful way. We oppose the elimination of segregation, the repeal of misce- genation statutes, the control of private employment by Federal bureaucrats called for by the misnamed civil rights program. We favor home-rule, local self-government and a minimum interference with individual rights.” The Rise of Conservatism in America, 1945–2000: A Brief History with Documents , ed. Ronald Story and Bruce Laurie (Boston: Bedford/St. Martin’s, 2008), 39 .


	{color{blue}19}. James Baldwin, “They Can’t Turn Back,” in The Price of the Ticket: Collected Nonfi ction, 1948–1985 (New York: St. Martin’s Press, 1985), 215 . I am grateful to Jason Frank for bringing this essay to my attention. N O T E S T O P A G E S 6 – 1 1


	{\color{blue}20}. Peter Kolchin, American Slavery 1619–1877 (New York: Hill and Wang, 1993, 2003), 100–102, 105, 111, 115, 117 .


	{\color{blue}21}. Thomas Roderick Dew, Abolition of Negro Slavery, and William Harper, Mem- oir on Slavery, in The Ideology of Slavery: Proslavery Thought in the Antebellum South, 1830–1860 , ed. Drew Gilpin Faust (Baton Rouge: Louisiana State Uni- versity Press, 1981), 65, 100 .


	{\color{blue}22}. Neil R. McMillen, Dark Journey: Black Mississippians in the Age of Jim Crow (Urbana: University of Illinois Press, 1989), 7 .


	{\color{blue}23}. Kolchin, American Slavery, 118–120, 123–124, 126; Ira Berlin, Many Thousands Gone: The First Two Centuries of Slavery in North America (Cambridge, Mass.: Harvard University Press, 1998), 94–95, 112, 128–132, 149–150, 174–175, 188–189 .


	{color{blue}24}. Calhoun, “Speech on the Reception of Abolition Petitions” (February 6, 1837), in Union and Liberty , 473; also see Dew, “Abolition of Negro Slavery,” 23–24, 27; Kolchin, American Slavery, 170, 181–182, 184, 189.


	{\color{blue}25}. Cited in Kolchin, American Slavery, 198.


	{color{blue}26}. Steven Hahn, A Nation under Our Feet: Black Political Strugg les in the Rural South from Slavery to the Great Migration (Cambridge, Mass.: Harvard University Press, 2003), 218 ; McMillen, Dark Journey , 125.


	{\color{blue}27}. Patrick Allitt, The Conservatives: Ideas & Personalities throughout American His- tory (New Haven, Conn.: Yale University Press, 2009), 19 .


	{\color{blue}28}. Edmund Burke, “Speech on the Army Estimates” (February 9, 1790), in The Portable Edmund Burke , ed. Isaac Kramnick (New York: Penguin, 1999), 413–414 .


	{color{blue}29}. Edmund Burke, letter to Earl Fitzwilliam (1791), cited in Daniel L. O’Neill, The Burke-Wollstonecraft Debate: Savagery, Civilization, and Democracy (Univer- sity Park: Pennsylvania State University Press, 2007), 211 .


	{\color{blue}30}. Cited in Conor Cruise O’Brien , The Great Melody: A Thematic Biography of Edmund Burke (Chicago: University of Chicago Press, 1992), 418–419 .


	{\color{blue}31}. Edmund Burke, Letters on a Regicide Peace (Indianapolis: Liberty Fund, 1999), 127 .


	{\color{blue}32}. John Adams, letter to James Sullivan (May 26, 1776), in The Works of John Adams , vol. 9, ed. Charles Francis Adams (Boston: Little Brown, 1854), 375 .


	{color{blue}33}. Abigail Adams, letter to John Adams (March 31, 1776), in The Letters of John and Abigail Adams (New York: Penguin, 2004), 148–49 .


	{\color{blue}34}. John Adams, letter to Abigail Adams (April 14, 1776), in Letters , 154.


	{color{blue}35}. John Adams, letter to James Sullivan (May 26, 1776), in Works , 378.


	{color{blue}36}. John Adams, A Defense of the Constitutions of Government of the United States of America , and Discourses on Davila , in The Political Writings of John Adams , ed. George A. Peck Jr. (Indianapolis: Hackett, [1954] 2003,), 148–149, 190 .


	{\color{blue}37}. Cited in Susan Moller Okin, Justice, Gender, and the Family (New York: Basic Books, 1989), 18 .


	{\color{blue}38}. Keyssar, Right to Vote, xxi.


	{color{blue}39}. Linda K. Kerber, No Constitutional Right to be Ladies: Women and the Obligations of Citizenship (New York: Hill and Wang, 1998), 3–46, 124–220 ; Ira Berlin, Bar- bara J. Fields, Steven F. Miller, Joseph P. Rediy, and Leslie S. Rowland, Slaves


	{\color{blue}40}. “The ultimate operative unit in our society is the family, not the individual.” Milton Friedman, Capitalism and Freedom (Chicago: University of Chicago Press, 1962, 1982, 2002), 32 ; also see 13. “It would be a mistake of major pro- portions to assume that legal rules are a dominant force in shaping individual character; family, school, and church are much more likely to be powerful infl uences. The people who run these institutions will use their infl uence to advance whatever conception of the good they hold, no matter what the state of the law.” Richard A. Epstein, “Libertarianism and Character,” in Va- rieties of Conservatism in America , ed. Peter Berkowitz (Stanford, Calif.: Hoover Institution Press, 2004), 76 . For earlier statements, see William Gra- ham Sumner, “The Family Monopoly,” in On Liberty, Society, and Politics: The Essential Essays of William Graham Sumner , ed. Robert C. Bannister (India- napolis, Liberty Fund, 1929), 136 ; William Graham Sumner, What the Social Classes Owe to Each Other (Caldwell, Idaho: Caxton Press, 2003), 63 ; Ludwig von Mises, Socialism: An Economic and Sociological Analysis (Indianapolis: Lib- erty Fund, 1981), 74–91 . More generally, see Okin, Justice, Gender, 74–88.


	{color{blue}41}. Edmund Burke, letter to Earl Fitzwilliam (1791), in O’Neill, Burke- Wollstonecraft Debate, 211.


	{\color{blue}42}. James Fitzjames Stephen, Liberty, Equality, Fraternity , ed. Stuart D. Warner (Indianapolis: Liberty Fund, 1993), 173 .


	{\color{blue}43}. David Farber, The Rise and Fall of Modern American Conservatism: A Short His- tory (Princeton, N.J.: Princeton University Press, 2010), 10 .


	{\color{blue}44}. Thomas Paine, Rights of Man, Part I , in Political Writings , ed. Bruce Kuklick (New York: Cambridge University Press, 2000), 130 ; Lionel Trilling, The Lib- eral Imagination (Garden City, N.Y.: Doubleday Anchor, 1950), 5 ; Robert O. Paxton, The Anatomy of Fascism (New York: Knopf, 2004), 42 .


	{\color{blue}45}. Michael Freeden, Ideologies and Political Theory (New York: Oxford University Press, 1996), 318 .


	{\color{blue}46}. Cited in Russell Kirk, “Introduction,” in The Portable Conservative Reader , ed. Russell Kirk (New York: Penguin, 1982), xxiii .


	{\color{blue}47}. Mark F. Proudman, “‘The Stupid Party’: Intellectual Repute as a Category of Ideological Analysis,” Journal of Political Ideologies 10 ( June 2005): 201–202, 206–207 .


	{\color{blue}48}. George H. Nash, The Conservative Intellectual Movement in America since 1945 (Wilmington, Del.: Intercollegiate Studies Institute), xiv ; Roger Scruton, The Meaning of Conservatism (London: Macmillan, 1980, 1984), 11 .


	{\color{blue}49}. “Problem: How did the exhausted come to make the laws about values? Put diff erently: How did those come to power who are the last?” Friedrich Nietzsche, The Will to Power , trans. Walter Kaufmann and R. J. Hollingdale (New York: Vintage, 1968), 34 .


	{\color{blue}50}. Kevin Mattson, Rebels All! A Short History of the Conservative Mind in Postwar America (Newark, N.J.: Rutgers University Press, 2008), 121–125 .


	{\color{blue}51}. Burke, Refl ections , 243; Russell Kirk, “The Conservative Mind,” in Conserva- tism in America since 1930 , ed. Gregory L. Schneider (New York: New York University Press, 2003), 107 . More recently still, Harvey Mansfi eld has declared, “But I understand conservatism as a reaction to liberalism. It isn’t a position that one takes up from the beginning but only when one is threat- ened by people who want to take away or harm things that deserve to be conserved.” The Point (Fall 2010), http://www.thepointmag.com/archive/ an-interview-with-harvey-mansfi eld , accessed April 9, 2011.


	{\color{blue}52}. Burke, Regicide Peace , 73.


	{color{blue}53}. Cited in John Ramsden, An Appetite for Power: A History of the Conservative Party since 1830 (New York: Harper Collins, 1999), 5 .


	{\color{blue}54}. The Faber Book of Conservatism , ed. Keith Baker (London: Faber and Faber, 1993), 6; also see Hugh Cecil, Conservatism (London: Thornton Butterworth, 1912), 39–44, 241, 244 .


	{\color{blue}55}. Robert Peel, speech at Merchant Taylor Hall (May 13, 1838), in British Conser- vatism: Conservative Thought from Burke to Thatcher , ed. Frank O’Gorman (London: Longman, 1986), 125 .


	{\color{blue}56}. Nash, Conservative Intellectual Movement, xiv.


	{color{blue}57}. Michael Oakeshott, “Rationalism in Politics” and “On Being Conservative,” in Rationalism in Politics and Other Essays (Indianapolis: Liberty Press, 1991), 31, 408, 435 .


	{\color{blue}58}. At one point in his essay, Oakeshott himself entertains this notion, only to dismiss it: “What would be the appropriateness of this disposition in circum- stances other than our own, whether to be conservative in respect of govern- ment would have the same relevance in the circumstances of an unadventurous, a slothful or a spiritless people, is a question we need not try to answer: we are concerned with ourselves as we are. I myself think that it would occupy an important place in any set of circumstances.” Why that is so he does not say. Oakeshott, “On Being Conservative,” 435.


	{\color{blue}59}. Benjamin Disraeli, The Vindication of the English Constitution, in Whigs and Whigg ism: Political Writings by Benjamin Disraeli , ed. William Hutcheon (New York: Macmillan, 1914), 126 .


	{\color{blue}60}. Karl Mannheim, “Conservative Thought,” in Essays on Sociology and Social Psychology , ed. Paul Kesckemeti (London: Routledge & Kegan Paul, 1953), 95, 115 ; also see Freeden, Ideologies and Political Theory, 335ff . Evidence for this ar- gument from the conservative tradition can be found in Frank Meyer, “Free- dom, Tradition, Conservatism,” in In Defense of Freedom and Related Essays (Indianapolis: Liberty Fund, 1996), 17–20 ; Mark C. Henrie, “Understanding Traditionalist Conservatism,” in Varieties of Conservatism in America , 11; Nash, Conservative Intellectual Movement, 50; Scruton, Meaning of Conservatism, 11.


	{color{blue}61}. Thus, when Irving Kristol claims in his Refl ections of a Neoconservative that neoconservatism “aims to infuse American bourgeois orthodoxy with a new self-conscious intellectual vigor,” he is not departing from conservative norms; he is articulating them. As the conservative sociologist and theologian 253


	{\color{blue}62}. Quintin Hogg, The Case for Conservatism , in British Conservatism , 76.


	{color{blue}63}. Boswell, Life of Johnson, 1018.


	{color{blue}64}. Edmund Burke, An Appeal from the New to the Old Whigs , in Further Refl ections on the Revolution in France , ed. Daniel F. Ritchie (Indianapolis: Liberty Fund, 1992), 167 .


	{\color{blue}65}. Giuseppe di Lampedusa, The Leopard (New York: Pantheon, 2007), 28 .


	{color{blue}66}. Mattson, Rebels All! 23, 35–36, 62.


	{color{blue}67}. Burke, Regicide Peace , 142.


	{color{blue}68}. Kirk, “The Conservative Mind,” 109; Oakeshott, “On Being Conservative,” 414–415.


	{\color{blue}69}. Cited in Allitt, Conservatives , 242; also see Arthur Moeller van den Bruck, Ger- many’s Third Empire , in The Nazi Germany Sourcebook: An Anthology of Texts , ed. Roderick Stackelberg and Sally A. Winkle (New York: Routledge, 2002), 77–78 .


	{color{blue}70}. Edmund Burke, Letter to a Noble Lord , in On Empire, Liberty, and Reform: Speeches and Letters , ed. David Bromwich (New Haven, Conn.: Yale Univer- sity Press, 2000), 479 .


	{\color{blue}71}. Cecil is one of the few conservatives to acknowledge how diffi cult it is to distinguish between reform and revolution (Cecil, Conservatism, 221–222). For a useful critique, see Ted Honderich, Conservatism: Burke, Nozick, Bush, Blair? (London: Pluto, 2005), 6–31 .


	{\color{blue}72}. Peter Kolozi, “Conservatives against Capitalism: The Conservative Critique of Capitalism in American Political Thought,” Ph.D. dissertation, CUNY Graduate Center, 2010, 138–172 ; Clinton Rossiter, Conservatism in America: The Thankless Persuasion (New York: Vintage, 1955, 1962), 241–242 ; Sam Tanen- haus, Whittaker Chambers: A Biography (New York: Modern Library, 1997), 165, 466, 488 .


	{\color{blue}73}. Nash, Conservative Intellectual Movement, xiv.


	{color{blue}74}. Abraham Lincoln, address at Cooper Institute (February 27, 1860), in The Por- table Abraham Lincoln , ed. Andrew Delbanco (New York: Penguin 1992), 178–179 . The typical conservative vision of reform, notes one scholar, “can be part of other political ideologies on account of—at least on the surface—its sheer reasonableness. It is, by itself, purely relative or ‘positional,’” and can thus be applied to or invoked by any ideology.” Jan-Werner Müller, “Comprehending Conservatism: A New Framework for Analysis,” Journal of Political Ideologies 11 (October 2006): 362 . N O T E S T O P A G E S 2 3 – 2 6


	{\color{blue}75}. Ramsden, Appetite for Power, 28.


	{color{blue}76}. Cited in C. B. Macpherson, Burke (New York Hill and Wang, 1980), 22 ; also see Burke, Regicide Peace , 381.


	{\color{blue}77}. Ramsden, Appetite for Power, 46, 95. Carnavon’s was the minority position on the British right; under the leadership of Derby and Disraeli, the Conserva- tives presided over the Act’s passage. But that should not be taken as evi- dence of a deep Burkean impulse on the right. Disraeli’s North Star throughout the debate was simple opposition to Gladstone. If Gladstone was for it, Disraeli was against it, and vice versa. If there was any vision beyond that, it was partisan and tactical, involving decidedly non-Burkean tactics at that. Explaining his support for a series of measures more radical than anything initially countenanced by the Liberals, Disraeli said to Derby, “The bold line is the safer one.” See Ramsden, Appetite for Power, 91–99. For a dissenting view, see Gertrude Himmelfarb, “Politics and Ideology: The Reform Act of 1867,” in Victorian Minds (New York: Knopf, 1968), 333–392 .


	{color{blue}78}. Allitt, Conservatives, 48. For other examples, see Allan Bloom, The Closing of the American Mind (New York: Simon and Schuster, 1987), 101 ; Calhoun, “Speech on the Oregon Bill,” in Union and Liberty , 565; Adams, Discourses on Davila , in Political Writings , 190–192, 201; Theodore Roosevelt: An American Mind , ed. Mario R. DiNunzio (New York: Penguin, 1994), 116, 119 ; Phillips- Fein, Invisible Hands, 82.


	{\color{blue}79}. Michael J. Gerson, Heroic Conservatism: Why Republicans Need to Embrace America’s Ideals ( And Why They Deserve to Fail If They Don’t ) (New York: Harper Collins, 2007), 261, 264 .


	{\color{blue}80}. While Huntington is right to stress the “situational” or “positional” dimen- sions of conservatism—that it is called into being in response to systemic challenges to the established order—he is wrong to suggest that the conserva- tive defends the established order simply because it is the established order. The conservative defends a particular type of order—the hierarchical institu- tion of personal rule—because he sincerely believes that inequality is a necessary condition of excellence. At times, he is willing to contest the estab- lished order, if he believes it is too egalitarian; such was the case with the postwar conservative movement in America. Samuel Huntington, “Conser- vatism as an Ideology,” American Political Science Review 51 ( June 1957): 454–473 .


	{color{blue}81}. The defense of the free market “became stationary when it was most infl u- ential,” while it “often progressed when on the defensive” from attacks on the left (Hayek, Constitution of Liberty , 7). “It is ironic, although not histori- cally unprecedented, that such a burst of creative energy on the intellectual level [on the right] should occur simultaneously with a continuing spread of the infl uence of Liberalism in the practical political sphere” (Frank Meyer, “Freedom, Tradition, Conservatism,” in Defense of Freedom , 15). “In times of crisis,” observes Scruton, “conservatism does its best” (Scruton, Meaning of Conservatism, 11). On the “dialectical” relationship between left and right in recent American history, see Julian E. Zelizer, “Refl ections: Rethinking the 255


	{\color{blue}82}. Matthew Arnold, Culture and Anarchy , in Culture and Anarchy and Other Writ- ings , ed. Stefan Collini (New York: Cambridge University Press, 1993), 95 .


	{color{blue}83}. Joseph Schumpeter, “Social Classes in an Ethnically Homogenous Environ- ment,” in Conservatism: An Anthology , 227.


	{\color{blue}84}. Attaining and maintaining real economic power, Schumpeter adds, requires a continuous “departure from routine.” Schumpeter, “Social Classes,” 227. “We must make it clear to ourselves that there can be no standing still, no being satisfi ed for us, but only progress or retrogression, and that it is tanta- mount to retrogression when we are contented with our present place.” Friedrich von Bernhardi, Germany and the Next War , trans. Allen Powles (Lon- don: Edward Arnold, 1912), 103 .


	{\color{blue}85}. Burke, Refl ections , 207. Also see Justus Möser, “No Promotion According to Merit,” in Conservatism: An Anthology , 74–77.


	{\color{blue}86}. Burke, Letter to a Noble Lord , 484.


	{color{blue}87}. Fritz Lens, Psychological Diff erences between the Leading Races of Mankind , in Nazi Germany Sourcebook , 75.


	{color{blue}88}. Muller, Conservatism, 26–27, 210.


	{color{blue}89}. Sumner, What the Social Classes Owe to Each Other , 59–60, 66–67.


	{color{blue}90}. Sumner, “Liberty,” in On Liberty, Society, and Politics , 246.


	{color{blue}91}. “All ownership derives from occupation and violence . . . . That all rights derive from violence, all ownership from appropriation or robbery, we may freely admit.” Mises, Socialism, 32.


	{\color{blue}92}. Sumner, “The Absurd Eff ort to Make the World Over,” in On Liberty, Society, and Politics , 254.


	{\color{blue}93}. Burke, Letter to a Noble Lord , 484.


	{color{blue}94}. With every passing month, the number of books about American conserva- tism seems to increase. Among the more notable of the last decade are Rick Perlstein, Before the Storm: Barry Goldwater and the Unmaking of the American Consensus (New York: Hill & Wang, 2001) ; Lisa McGirr, Suburban Warriors: The Origins of the New American Right (Princeton, N.J.: Princeton University Press, 2001) ; Donald Critchlow, Phyllis Schlafl y and Grassroots Conservatism: A Woman’s Crusade (Princeton, N.J.: Princeton University Press, 2005) ; Kevin Kruse, White Flight: Atlanta and the Making of Modern Conservatism (Prince- ton, N.J.: Princeton University Press, 2005) ; Jason Sokol, There Goes My Every- thing: White Southerners in the Age of Civil Rights, 1945–1975 (New York: Vintage, 2006) ; Matthew Lassiter, The Silent Majority: Suburban Politics in the Sunbelt South (Princeton, N.J.: Princeton University Press, 2006) ; Joseph Lowndes, From the New Deal to the New Right: Race and the Southern Origins of Modern Conservatism (New Haven, Conn.: Yale University Press, 2008) ; Allan J. Licht- man, White Protestant Nation: The Rise of the American Conservative Movement (New York: Grove Press, 2008) ; Mattson, Rebels All! ; Steven Teles, The Rise of the Conservative Legal Movement: The Battle for Control of the Law (Princeton, N O T E S T O P A G E S 2 8 – 3 2


\section{Writings (Lincoln: University of Nebraska Press, 1965), 139 .}


	{\color{blue}96}. “‘Metaphysical pathos’ is exemplifi ed in any description of the nature of things, any characterization of the world to which one belongs, in terms which, like the words of a poem, awaken through their associations, and through a sort of empathy which they engender, a congenial mood or tone of feeling on the part of the philosopher or his readers.” Arthur O. Lovejoy, The Great Chain of Being: A Study of the History of an Idea (New York: Harper & Brothers, 1936), 11 . Cited in Joseph F. Femia, Against the Masses: Varieties of Anti-Democratic Thought since the French Revolution (New York: Oxford University Press, 2001), 13–14 .


	{\color{blue}97}. Cf. Bruce Frohnen, Virtue and the Promise of Conservatism: The Legacy of Burke and Tocqueville (Lawrence: University of Kansas Press, 1993) ; Nash, Conserva- tive Intellectual Movement ; Allitt, Conservatives ; Scruton, Meaning of Conserva- tism ; Berkowitz, Varieties of Conservatism . More useful treatments include Robert Nisbet, Conservatism: Dream and Reality (Minneapolis: University of Minnesota Press, 1986) ; Stephen Holmes, The Anatomy of Antiliberalism (Cambridge, Mass.: Harvard University Press, 1993) ; Albert O. Hirschman, The Rhetoric of Reaction: Perversity, Futility, Jeopardy (Cambridge, Mass.: Har- vard University Press, 1991) ; Mannheim, “Conservative Thought”; Muller, Conservatism ; Femia, Against the Masses .


	{\color{blue}98}. Mattson, Rebels All! , 3, 11–12, 42, 79. Also see Sam Tanenhaus, The Death of Conservatism (New York: Random House, 2009), 16–19, 49–51 .


	{\color{blue}99}. Cara Camcastle, The More Moderate Side of Joseph de Maistre: Views on Political Liberty and Political Economy (Montreal and Kingston: McGill-Queen’s Uni- versity Press, 2005) ; Isaiah Berlin, “Joseph de Maistre on the Origins of Mod- ern Fascism,” in The Crooked Timber of Humanity: Chapters in the History of Ideas , ed. Henry Hardy (New York: Vintage, 1992), 91–174 .


	{\color{blue}100}. Nash, Conservative Intellectual Movement , 69–70.


	{color{blue}101}. Published in June 2008, Lichtman’s White Protestant Nation appeared before the advent of the Tea Party—indeed, before the election of Barack Obama— but its analysis of the continuities between the conservatism that arose in the aftermath of World War I and the conservatism of George W. Bush can be extrapolated to today.


	{\color{blue}102}. Mattson, Rebels All! , 7, 15; Farber, Rise and Fall of Modern American Conserva- tism , 78; Donald T. Critchlow, The Conservative Ascendancy: How the GOP Right Made Political History (Cambridge, Mass.: Harvard University Press, 2007), 6–13 ; Tanenhaus, Death of Conservatism , 29, 32, 104, 109, 111, 114.


	{\color{blue}103}. “The right’s political philosophy, organizing strategy, and grassroots appeal transcend its hostility to liberalism. Modern conservatism has a life, history,


	{\color{blue}104}. Cf. Zelizer, “Rethinking the History of American Conservatism,” 371–374.


	{color{blue}105}. Cited in Mattson, Rebels All! , 112.


	{color{blue}106}. Händler und Helden. Patriotische Besinnungen , in Nazi Germany Sourcebook , 36.


	{color{blue}107}. Noberto Bobbio, Left & Right: The Significance of a Political Distinction (Chicago: University of Chicago Press, 1996) .


	{\color{blue}108}. Müller, “Comprehending Conservatism,” 359; Muller, Conservatism , 22–23; J. G. A. Pocock, introduction to Burke, Refl ections on the Revolution in France (Indianapolis: Hackett, 1987), xlix .


	{\color{blue}109}. Nash, Conservative Intellectual Movement, xiv–xv.


\section{Chapter 1}


	{\color{blue}1}. Michael Oakeshott, “On Being Conservative,” in Rationalism in Politics and Other Essays (Indianapolis: Liberty Press, 1991), 408 .


	{\color{blue}2}. Russell Kirk, “Introduction,” in The Portable Conservative Reader , ed. Russell Kirk (New York: Penguin, 1982), xi–xiv ; Robert Nisbet, Conservatism: Dream and Reality (Minneapolis: University of Minnesota Press, 1986) ; Peter Viereck, Conservatism: From John Adams to Churchill (Princeton, N.J.: D. Van Nostrand, 1956), 10–17 .


	{\color{blue}3}. Joseph de Maistre, Considerations on France , trans. and ed. Richard A. Lebrun (New York: Cambridge University Press, 1974, 1994), 10 . Also see Maistre’s crit- icism of Europe’s old regimes in Jean-Louis Darcel, “The Roads of Exile, 1792–1817,” and Darcel, “Joseph de Maistre and the House of Savoy: Some Aspects of his Career,” in Joseph de Maistre’s Life, Thought, and Infl uence: Selected Studies , ed. Richard A. Lebrun (Montreal: McGill-Queen’s University Press, 2001), 16, 19–20, 52 .


	{\color{blue}4}. Cf. Edmund Burke, Letter to a Noble Lord , in On Empire, Liberty, and Reform: Speeches and Letters , ed. David Bromwich (New Haven, Conn.: Yale University Press, 2000), 500–501 ; Burke, Letters on a Regicide Peace (Indianapolis: Liberty Fund, 1999), 69–70, 74–76, 106, 108–111, 158–160, 167, 184, 205, 218, 218, 222, 271, 304–305 .


	{\color{blue}5}. Edmund Burke, Refl ections on the Revolution in France , ed. J. C. D. Clark (Stan- ford, Calif.: Stanford University Press, 2001), 239 .


	{\color{blue}6}. Edmund Burke, A Philosophical Enquiry into the Origins of Our Ideas of the Sub- lime and the Beautiful , ed. David Womersley (New York: Penguin, 1998), 177 .


	{\color{blue}7}. Burke, Regicide Peace , 75.


	{color{blue}8}. Though sometimes it is the old regime itself. Cf. Burke, Regicide Peace , 384–385.


	{color{blue}9}. Edmund Burke, “Speech on American Taxation” (April 19, 1774), in Selected Works of Edmund Burke , vol. 1 (Indianapolis: Liberty Fund, 1999), 186 ; also see Burke, Regicide Peace , 69–70, 154–155, 184–185, 304–306, 384–385. This critique runs counter to Burke’s praise for the aristocrat as the man of the long view;


	{\color{blue}10}. Thomas Roderick Dew, Abolition of Negro Slavery , and William Harper, Mem- oir on Slavery , in The Ideology of Slavery: Proslavery Thought in the Antebellum South, 1830 – 1860 , ed. Drew Gilpin Faust (Baton Rouge: Louisiana State Press, 1981), 25, 123 . Also see John C. Calhoun, “Speech on the Force Bill,” “Speech on the Reception of Abolitionist Petitions,” and “Speech on the Oregon Bill,” in Union and Liberty: The Political Philosophy of John C. Calhoun , ed. Ross M. Lence (Indianapolis: Liberty Fund, 1992), 426, 465, 475, 562 ; Manisha Sinha, The Counterrevolution of Slavery: Politics and Ideology in Antebellum South Carolina (Chapel Hill: University of North Carolina Press, 2000), 33–93 .


	{\color{blue}11}. Barry Goldwater, The Conscience of a Conservative (Princeton, N.J.: Princeton University Press, 1960, 2007), 1 .


	{\color{blue}12}. Calhoun, “Speech on the Reception of Abolitionist Petitions,” 476.


	{color{blue}13}. Oakeshott, “On Being Conservative,” 407–408.


	{color{blue}14}. Charles Loyseau, A Treatise of Orders and Plain Dignities , ed. Howell A. Lloyd (New York: Cambridge University Press, 1994), 75 .


	{\color{blue}15}. Cited in Anne Norton, Leo Strauss and the Politics of American Empire (New Haven, Conn.: Yale University Press, 2004), 49 .


	{\color{blue}16}. Joseph de Maistre, St. Petersburg Dialogues or Conversations on the Temporal Gov- ernment of Providence , trans. and ed. Richard A. Lebrun (Montreal and Kings- ton: McGill-Queen’s University Press, 1993), 216 .


	{\color{blue}17}. Maistre, Considerations , 16–17. Also see Jean-Louis Darcel, “The Apprentice Years of a Counter-Revolutionary: Joseph de Maistre in Lausanne, 1793–1797,” in Joseph de Maistre’s Life, Thought, and Infl uence , 43–44 .


	{\color{blue}18}. Burke, Sublime and the Beautiful , 86, 96, 121, 165.


	{color{blue}19}. Burke, Refl ections , 207, 243, 275. Also see Burke, Regicide Peace , 66, 70, 107, 157, 207, 222.


	{\color{blue}20}. Burke, Regicide Peace , 184.


	{color{blue}21}. Darrin M. McMahon, Enemies of the Enlightenment: The French Counter- Enlightenment and the Making of Modernity (New York: Oxford University Press, 2001), 27–28 .


	{\color{blue}22}. Cited in Robert Perkinson, Texas Tough: The Rise of America’s Prison Empire (New York: Metropolitan, 2009), 297 .


	{\color{blue}23}. Cited in Alexander P. Lamis, “The Two-Party South: From the 1960s to the 1990s,” in Southern Politics in the 1990s , ed. Alexander P. Lamis (Baton Rouge: Louisiana State University Press, 1990), 8 .


	{\color{blue}24}. David Horowitz, “The Campus Blacklist,” FrontPage (April 18, 2003), http:// www.studentsforacademicfreedom.org/essays/blacklist.html ,accessed March 24, 2011 .


	{\color{blue}25}. Cited in Lamis, “Two-Party South,” 8.


	{color{blue}26}. Phyllis Schlafl y, The Power of the Positive Woman (New York: Harcourt Brace Jovanovich, 1977), 7–8 .


	{\color{blue}27}. “Interview with Phyllis Schlafl y,” Washington Star ( January 18, 1976), in The Rise of Conservatism in America, 1945 – 2000: A Brief History with Documents , ed. Ronald Story and Bruce Laurie (Boston: Bedford/St. Martin’s, 2008), 104 .


	{color{blue}28}. Susan Faludi, Backlash: The Undeclared War against American Women (New


\section{Conservatism in America , 53.}


	{\color{blue}31}. Gary Wills, Reagan’s America (New York: Penguin, 1988), 355 .


	{color{blue}32}. Cited in J. C. D. Clark, introduction to Burke, Refl ections , 104.


	{color{blue}33}. Alexander Stephens, “The Cornerstone Speech,” in Defending Slavery: Proslav- ery Thought in the Old South , ed. Paul Finkelman (Boston: Bedford/St. Mar- tin’s, 2003), 91 .


	{\color{blue}34}. Goldwater, Conscience of a Conservative, 70.


	{color{blue}35}. Maistre, Considerations , 89 .


	{color{blue}36}. Ibid. , 69, 74.


	{color{blue}37}. James Oakes, The Ruling Race: A History of American Slaveholders (New York: Vintage, 1982), 37, 42, 141–143, 230–232 .


	{color{blue}38}. Calhoun, “Speech on the Oregon Bill,” 564.


	{color{blue}39}. Cited in Peter Kolchin, American Slavery 1619 – 1877 (New York: Hill and Wang,


	{\color{blue}40}. Dew, Abolition of Negro Slavery , 66–67.


	{color{blue}41}. Cited in Jacob Heilbrunn, They Knew They Were Right: The Rise of the Neocons (New York: Random House, 2008), 6 .


	{\color{blue}42}. Burke, Refl ections , 229; William F. Buckley Jr., “Publisher’s Statement on Founding National Review ,” National Review (November 19, 1955), in Rise of Conservatism in America , 50.


	{\color{blue}43}. Andrew Sullivan, The Conservative Soul: Fundamentalism, Freedom, and the Future of the Right (New York: Harper Perennial, 2006), 9 .


	{\color{blue}44}. Burke, Regicide Peace , 138.


	{color{blue}45}. Maistre, Considerations , 77.


	{color{blue}46}. Corey Robin, “The Ex-Cons: Right-Wing Thinkers Go Left!” Lingua Franca (February 2001), 32. Reprinted in this volume as chapter 5.


\section{Chapter 2}


	{\color{blue}1}. Noel Malcolm, Aspects of Hobbes (New York: Oxford University Press, 2002), 15–16 ; Richard Tuck, Hobbes (New York: Oxford University Press, 1989), 24 ; Quentin Skinner, Visions of Politics, vol. 3, Hobbes and Civil Sciences (New York: Cambridge University Press, 2002), 8–9 ; A. P. Martinich, Hobbes (New York: Cambridge University Press, 1999), 161–162 .


	{\color{blue}2}. Skinner, Visions , 16.


	{color{blue}3}. Malcolm, Aspects of Hobbes, 20–21; Skinner, Visions , 22–23; Martinich, Hobbes, 209–210.


	{\color{blue}4}. T. S. Eliot, “John Bramhall,” in Selected Essays 1917 – 1932 (New York: Harcourt Brace, 1932), 302 .


	{\color{blue}5}. Perry Anderson, “The Intransigent Right,” in Spectrum: From Right to Left in the World of Ideas (New York: Verso, 2005), 3–28 .


	{\color{blue}6}. Michael Oakeshott, “On Being Conservative,” in Rationalism in Politics and Other Essays (Indianapolis: Liberty Press, 1991), 435 . Also see the useful remarks of Paul Franco in his foreword to Michael Oakeshott, Hobbes on Civil Associa- tion (Indianapolis: Liberty Fund, 2000) , v–vii; Paul Franco, Michael Oakeshott: An Introduction (New Haven, Conn.: Yale University Press, 2004), 10, 103, 106 .


	{color{blue}7}. Friedrich A. Hayek, The Constitution of Liberty (Chicago: University of Chi- cago Press, 1960), 56 ; Carl Schmitt, The Leviathan in the State Theory of Thomas Hobbes: Meaning and Failure of a Political Symbol (Chicago: University of Chi- cago Press, 2008), 42, 68–69 ; Leo Strauss, Natural Right and History (Chicago: University of Chicago Press, 1953), 165–202 ; Leo Strauss, “Comments on Carl Schmitt’s Der Begriff des Politischen ,” in Carl Schmitt, The Concept of the Political (New Brunswick, N.J.: Rutgers University Press, 1967), 89 .


	{\color{blue}8}. Hayek, Constitution of Liberty , 397–411.


	{color{blue}9}. Hobbes, Behemoth , ed. Ferdinand Tönnies (Chicago: University of Chicago Press, 1990), 204 .


	{\color{blue}10}. Benjamin Constant, The Liberty of the Ancients Compared with That of the Moderns, in Political Writings , ed. Biancamaria Fontana (New York: Cambridge University Press, 1988), 307–328 ; Karl Marx, The Eighteenth Brumaire of Louis Bonaparte , in The Marx- Engels Reader , ed. Robert C. Tucker (New York: Norton, 1978), 595 .


	{\color{blue}11}. Hobbes, Behemoth , 28.


	{color{blue}12}. Quentin Skinner, Hobbes and Republican Liberty (New York: Cambridge Uni- versity Press, 2008) .


	{color{blue}13}. Skinner, Hobbes , xiv .


	{color{blue}14}. David Wootton, Divine Right and Democracy (New York: Penguin, 1986), 28 .


	{color{blue}15}. Ibid. , 25–26 .


	{color{blue}16}. Skinner, Hobbes , 57ff .


	{color{blue}17}. Ibid. , 27 .


	{color{blue}18}. Hobbes, Leviathan , ed. Richard Tuck (New York: Cambridge, 1996), 149 .


	{color{blue}19}. Skinner, Hobbes , x–xi, 25–33, 68–72 .


	{color{blue}20}. Ibid. , xi, 215 .


	{color{blue}21}. Skinner, Hobbes , 211–212.


	{color{blue}22}. Hobbes, Leviathan , 44 .


	{color{blue}23}. Ibid. , 145–146 .


	{color{blue}24}. Cited in Skinner, Hobbes , 130 .


	{color{blue}25}. Ibid. , 116–123, 157, 162, 173 .


	{color{blue}26}. Hobbes, Leviathan , 146.


	{color{blue}27}. Hobbes, De Cive , in Man and Citizen , ed. Bernard Gert (Indianapolis: Hackett,


	{\color{blue}28}. Greg Grandin, Empire’s Workshop: Latin America, the United States, and the Rise of the New Imperialism (New York: Metropolitan Books, 2006), 173–174 ; Naomi


	{\color{blue}29}. Klein, Shock Doctrine, 117.


	{color{blue}30}. Hobbes, Leviathan , 148.


	{color{blue}31}. Klein, Shock Doctrine, 131, 138.


\section{Chapter 3}


	{\color{blue}1}. Anne C. Heller, Ayn Rand and the World She Made (New York: Knopf, 2009) , xii; http://www.randomhouse.com/modernlibrary/100bestnovels.html, accessed April 8, 2011


	{\color{blue}2}. Amy Wallace, “Farrah’s Brainy Side,” The Daily Beast ( June 25, 2009), http:// www.thedailybeast.com/blogs-and-stories/2009-06-25/farrahs-brainy-side , accessed April 8, 2011 ; Heller, Ayn Rand, 401.


	{\color{blue}3}. Heller, Ayn Rand, 167.


	{color{blue}4}. Ayn Rand, “The Objectivist Ethics,” in Rand, The Virtue of Selfi shness (New York: Penguin, 1961, 1964), 39 .


	{\color{blue}5}. Elizabeth Gettelman, “I’m With the Rand,”


	{\color{blue}6}. Ayn Rand, The Fountainhead (New York: Signet, 1996), 678 .


	{color{blue}7}. Heller, Ayn Rand, 155, 275, 292; Rand, Fountainhead , 24–25; http://en.wikipedia. org/wiki/1957_in_literature , accessed April 8, 2011; http://atlasshrugged. com/book/history.html#publication , accessed May 1, 2010 .


	{\color{blue}8}. Heller, Ayn Rand, 88, 186, 278.


	{color{blue}9}. Rand, Fountainhead , 675; Ayn Rand, Atlas Shrugg ed (New York: Plume, 1957, 1992), 1022 .


	{\color{blue}10}. Heller, Ayn Rand, 1–3 .


	{color{blue}11}. Ibid. , 5 .


	{color{blue}12}. Ibid. , 29 ; Jennifer Burns, Goddess of the Market: Ayn Rand and the American Right (New York: Oxford University Press, 2009), 14–15.


	{\color{blue}13}. Cited in Theodor Adorno, Prisms (Cambridge: MIT Press, 1967), 109.


	{color{blue}14}. Heller, Ayn Rand, 32, 35, 69, 159, 299, 395–396 .


	{color{blue}15}. Ibid. , 38–39, 44, 82–83, 114, 336, 371 .


	{color{blue}16}. Ibid. , 9, 11, 15 .


	{color{blue}17}. Burns, Goddess of the Market, 3, 229, 285 .


	{color{blue}18}. Ibid. , 16–17, 21, 27 .


	{color{blue}19}. Ayn Rand, For the New Intellectual (New York: Signet, 1961), 18 .


	{color{blue}20}. Burns, Goddess of the Market, 307.


	{color{blue}21}. Julian Sanchez, “An Interview with Robert Nozick” ( July 26, 2001), http:// www.trinity.edu/rjensen/NozickInterview.htm , accessed April 8, 2011 .


	{color{blue}22}. Sidney Hook, “Each Man for Himself,” New York Times, April 9, 1961, BR3 .


	{color{blue}23}. Rand, “The Cult of Moral Grayness,” in The Virtue of Selfi shness , 92.


	{color{blue}24}. Rand, “Objectivist Ethics,” 16. N O T E S T O P A G E S 7 4 – 8 6


	{\color{blue}25}. Tara Smith, Ayn Rand’s Normative Ethics: The Virtuous Egoist (New York: Cam- bridge University Press, 2006), 28–29 ; Rand, “Objectivist Ethics,” 25.


	{\color{blue}26}. Rand, “Objectivist Ethics,” 28.


	{color{blue}27}. The Nazi Germany Sourcebook , ed. Roderick Stackelberg and Sally Winkle (London: Routledge, 2002), 302–303 .


	{\color{blue}28}. Ibid. , 105 .


	{color{blue}29}. Rand, Capitalism: The Unknown Ideal (New York: Signet, 1967), 2, 6, 8, 11, 24 .


	{color{blue}30}. Nazi Germany Sourcebook , 131 .


	{color{blue}31}. Ibid. , 130 .


	{color{blue}32}. Rand, Capitalism , 18.


	{color{blue}33}. Nazi Germany Sourcebook , 105, 131.


	{color{blue}34}. Rand, Atlas Shrugg ed , 1065.


	{color{blue}35}. Rand, Fountainhead , 681.


	{color{blue}36}. Burns, Goddess of the Market, 16, 22, 25; Heller, Ayn Rand, 57.


	{color{blue}37}. Burns, Goddess of the Market, 28, 70 .


	{color{blue}38}. Ibid. , 42 .


	{color{blue}39}. Ibid. , 177 .


	{color{blue}40}. Ibid. , 43 .


	{color{blue}41}. Joseph de Maistre, St. Petersburg Dialogues , trans. and ed. Richard Lebrun (Montreal: McGill-Queen’s University Press, 1993), 335 . Burke also traced the French Revolution back to the Reformation. See Conor Cruise O’Brien, The Great Melody: A Thematic Biography and Commented Anthology of Edmund Burke (Chicago: University of Chicago Press, 1992), 452–453 .


	{\color{blue}42}. Joseph de Maistre, Considerations on France , ed. Richard Lebrun (New York: Cambridge University Press, 1974, 1994), 27 .


	{\color{blue}43}. Friedrich Nietzsche, On the Genealogy of Morals , trans. Walter Kaufmann (New York: Random House, 1967), 24–56 .


	{\color{blue}44}. Friedrich Nietzsche, The Will to Power , trans. Walter Kaufmann and R. J. Hollingdale (New York: Random House, 1967), 401 . Also see Nietzsche, Genealogy , 36, 54; Friedrich Nietzsche, Beyond Good and Evil (New York: Vintage, 1989), 116 .


	{\color{blue}45}. Burns, Goddess of the Market, 2, 4.


	{color{blue}46}. http://yglesias.thinkprogress.org/archives/2010/03/beck-vs-social-justice. php , accessed April 8, 2011; http://yglesias.thinkprogress.org/archives/2010/03/ lds-scholars-confi rm-mormon-commitment-to-social-justice.php , accessed April 8, 2011.


	{\color{blue}47}. Rand, Fountainhead , 606.


\section{Chapter 4}


	{\color{blue}1}. George Will, foreword to Barry Goldwater, The Conscience of a Conservative (Princeton, N,J.: Princeton University Press, 2007, 1960), xi .


	{\color{blue}2}. The Rise of Conservatism in America, 1945–2000: A Brief History with Documents , ed. Ronald Story and Bruce Laurie (Boston: Bedford/St. Martin’s, 2008), 1 .


	{\color{blue}3}. William F. Buckley Jr., “Publisher’s Statement on Founding National Review ,” National Review (November 19, 1955), in Rise of Conservatism in America , 51.


	{color{blue}4}. Joseph de Maistre, Considerations on France , trans. and ed. Richard A. Lebrun (New York: Cambridge University Press, 1974, 1994), 69, 74 .


	{\color{blue}5}. Judith N. Shklar, “Jean-Jacques Rousseau and Equality,” in Political Thought and Political Thinkers , ed. Stanley Hoff mann (Chicago: University of Chicago Press, 1998), 290 .


	{\color{blue}6}. Edmund Burke, Refl ections on the Revolution in France , ed. J. C. D. Clark (Stan- ford, Calif.: Stanford University Press, 2001), 232–233 .


	{\color{blue}7}. Hugo Young, One of Us: A Biography of Margaret Thatcher (London: Pan Books, 1989, 1991) .


	{\color{blue}8}. Goldwater, Conscience of a Conservative, 1 .


	{color{blue}9}. Ibid . , xxiii.


	{color{blue}10}. Edmund Burke, Letters on a Regicide Peace (Indianapolis: Liberty Fund, 1999), 69 .


	{color{blue}11}. Young, One of Us, 406.


	{color{blue}12}. “Speech at the Meeting of the Citizens of Charleston,” in Union and Liberty: The Political Philosophy of John C. Calhoun , ed. Ross M. Lence (Indianapolis: Liberty Fund, 1992), 536 .


	{\color{blue}13}. Goldwater, Conscience of a Conservative, 54 .


	{color{blue}14}. Ibid . , 2.


	{color{blue}15}. Ibid. , 3–4 .


	{color{blue}16}. Karl Mannheim, “Conservative Thought,” in Essays on Sociology and Social Psy- chology , ed. Paul Kesckemeti (London: Routledge & Kegan Paul, 1953), 106 .


	{\color{blue}17}. Goldwater, Conscience of a Conservative, 3, 78–79, 119.


	{color{blue}18}. Mannheim, “Conservative Thought,” 107.


	{color{blue}19}. Goldwater, Conscience of a Conservative, 17–18, 25.


	{color{blue}20}. “Introduction,” in Rightward Bound: Making America Conservative in the 1970s , ed. Bruce J. Schulman and Julian E. Zelizer (Cambridge, Mass.: Harvard Uni- versity Press, 2008), 4 .


	{\color{blue}21}. Matthew D. Lassiter, “Inventing Family Values,” and Joseph Crespino, “Civil Rights and the Religious Right,” in Rightward Bound , 14, 90–91, 93 .


	{\color{blue}22}. Crespino, “Civil Rights,” 91, 92–93, 97, 102–103.


	{color{blue}23}. Marjorie J. Spruill, “Gender and America’s Right Turn,” in Rightward Bound , 77–79 .


	{\color{blue}24}. “Interview with Phyllis Schlafl y,” Washington Star ( January 18, 1976), in The Rise of Conservatism in America , 104–105.


	{\color{blue}25}. Lassiter, “Inventing Family Values,” and Paul Boyer, “The Evangelical Resur- gence in 1970s American Protestantism,” in Rightward Bound , 19–20, 34, 37, 40–41.


	{\color{blue}26}. Bethany E. Moreton, “Make Payroll, Not War,” in Rightward Bound , 53, 55–57, 65, 69 .


	{\color{blue}27}. Thomas J. Sugrue and John D. Skrentny, “The White Ethnic Strategy,” in Rightward Bound , 174–175, 189, 191 .


	{\color{blue}28}. Rick Perlstein, Before the Storm: Barry Goldwater and the Unmaking of the Ameri- can Consensus (New York: Hill & Wang, 2001), 17 .


\section{Chapter 5}


	{\color{blue}1}. Denis Diderot, Extracts from the Histoire des Deux Indes, in Political Writings , ed. John Hope Mason and Robert Wokler (New York: Cambridge University Press, 1992), 202–203 ; George Bernard Shaw, Man and Superman (New York: Penguin, 2001), 213 .


	{\color{blue}2}. Michael Lind, Up from Conservatism: Why the Right Is Wrong for America (New York: Free Press, 1997), 235, 257 ; Arianna Huffi ngton, How to Overthrow the Gov- ernment (New York: Harper Collins, 2001), 8 .


	{\color{blue}3}. John Gray, False Dawn: The Delusions of Global Capitalism (New York: New Press, 2000), 3, 141 .


	{\color{blue}4}. Robert Skidelsky, “What’s Wrong with Global Capitalism?” Times Literary Supplement (March 27, 1998) .


	{\color{blue}5}. Michael Gordon, “Right-of-Center Defense Groups—The Pendulum Has Swung Their Way,” National Journal ( January 24, 1981): 128 .


	{\color{blue}6}. Edward N. Luttwak, Turbo-Capitalism: Winners and Losers in the Global Economy (New York: Harper Perennial, 2000), 15, 193, 195 .


	{\color{blue}7}. Kim Phillips-Fein, “Laissez-Faire No More,” In These Times ( July 11, 1999): 19 .


	{color{blue}8}. Isaiah Berlin, “Joseph de Maistre on the Origins of Modern Fascism,” in The Crooked Timber of Humanity: Chapters in the History of Ideas , ed. Henry Hardy (New York: Vintage, 1992), 126 .


	{\color{blue}9}. John Gray, “After Social Democracy,” Endgames: Questions in Late Modern Polit- ical Thought (Cambridge, U.K. : Polity, 1997), 23–24 .


	{\color{blue}10}. Hugo Young, One of Us: A Biography of Margaret Thatcher (London: Pan Books, 1989, 1991), 209 .


	{\color{blue}11}. John Gray, Hayek on Liberty (New York: Routledge, 1984, 1998), 2 .


	{color{blue}12}. John Gray, Liberalism (Minneapolis: University of Minnesota Press, 1995), 38 .


	{color{blue}13}. John Gray, “Hayek as a Conservative,” in Gray, Post-Liberalism: Studies in Polit- ical Thought (New York: Routledge, 1996), 33 .


	{\color{blue}14}. Gray, Hayek on Liberty , 37 .


	{color{blue}15}. Ibid ., 14, 37–38 .


	{color{blue}16}. John Gray, “Limited Government: A Positive Agenda,” in Gray, Beyond the New Right: Markets, Government, and the Common Environment (New York: Rout- ledge, 1995), 15 .


	{\color{blue}17}. Gray, False Dawn , 2, 111, 119 .


	{color{blue}18}. Ibid ., 2 .


	{color{blue}19}. Ibid ., 3, 17, 37, 35, 215.


	{color{blue}20}. Irving Kristol, Two Cheers for Capitalism (New York: Signet, 1979), x .


	{color{blue}21}. Mary Battiata, “Places of Honor,” Washington Post, November 15, 1980, F1 .


	{color{blue}22}. Los Angeles Times, July 20, 1986, 1.


	{\color{blue}23}. Edward Luttwak, The Pentagon and the Art of War (New York: Simon and Schuster, 1985), 33–34 .


	{\color{blue}24}. Ibid ., 134–135 .


	{color{blue}25}. Ibid ., 138–139 ; Washington Quarterly (Autumn 1982): 6–7.


	{color{blue}26}. Luttwak, Pentagon and the Art of War , 138, 140, 143–144; Forbes (May 26, 1980), 4.


	{color{blue}27}. Luttwak, Turbo-Capitalism , ix.


\section{Chapter 6}


	{\color{blue}1}. Joan Biskupic, American Original: The Life and Constitution of Supreme Court Jus- tice Antonin Scalia (New York: Farrar, Straus and Giroux, 2009), 340 .


	{\color{blue}2}. Nixon v. Missouri Municipal League , 541 U.S. 125, 141–142 (2004) (Scalia, concurring).


	{\color{blue}3}. Hamdi v. Rumsfeld , 542 U.S. 507, 576 (2004) (Scalia, dissenting).


	{color{blue}4}. Biskupic, American Original, 282.


	{color{blue}5}. Cited in Mark Tushnet, A Court Divided (New York: Norton, 2005), 149 .


	{color{blue}6}. Biskupic, American Original, 7, 11, 14, 346 .


	{color{blue}7}. Ibid. , 17, 19, 21, 25 .


	{color{blue}8}. Ibid. , 23, 40–41, 73 .


	{color{blue}9}. Ibid. , 41 .


	{color{blue}10}. Ibid. , 66–67 .


	{color{blue}11}. Antonin Scalia, A Matter of Interpretation: Federal Courts and the Law (Prince- ton, N.J.: Princeton University Press, 1997), 23, 145 .


	{\color{blue}12}. Ibid. , 23 .


	{color{blue}13}. Ibid. , 46 .


	{color{blue}14}. Remarks at Catholic University (October 18, 1996), http://www.joink.com/ homes/users/ninoville/cua10-18-96.asp , accessed April 8, 2011; Scalia, A Matter of Interpretation , 47, 149.


	{\color{blue}15}. Scalia, A Matter of Interpretation , 14.


	{color{blue}16}. Robert H. Bork, The Tempting of America (New York: Simon and Schuster, 1990), 133, 188 .


	{\color{blue}17}. Biskupic, American Original, 25, 209, 211.


	{color{blue}18}. PGA TOUR, Inc. v. Casey Martin , 532 U.S. 661 (2001) (Scalia, dissenting).


	{color{blue}19}. Alexis de Tocqueville, Democracy in America (New York: Harper, 1969), 150 .


	{color{blue}20}. Lawrence v. Texas , 539 U.S. 568, 590 (2003) (Scalia, dissenting).


	{color{blue}21}. Biskupic, American Original, 189.


	{color{blue}22}. Board of County Commissioners, Wabaunsee County, Kansas v. Umbehr , 518 U.S. 668, 711 (1996) (Scalia, dissenting).


	{\color{blue}23}. http://www.nytimes.com/2003/06/29/opinion/29DOWD.html?pagewanted=1 , accessed April 8, 2011.


	{\color{blue}24}. Biskupic, American Original, 362.


	{\color{blue}25}. William J. Brennan, “Speech to the Text and Teaching Symposium,” in Origi- nalism: A Quarter-Century of Debate , ed. Steven Calabresi (Washington, D.C.: Regnery, 2007), 59, 61 .


	{\color{blue}26}. Scalia, A Matter of Interpretation , 67 .


	{color{blue}27}. Citizens United v. Federal Election Commission , 558 U.S. 201, 209, 212 (2010) (Ste- vens, dissenting).


	{\color{blue}28}. Biskupic, American Original, 9, 134, 196.


	{color{blue}29}. Scalia Dissents: Writings of the Supreme Court’s Wittiest, Most Outspoken Justice , ed. Kevin A. Ring (Washington, D.C.: Regnery, 2004), 9 .


	{\color{blue}30}. http://www.law.yale.edu/news/5658.htm , accessed April 8, 2011.


	{color{blue}31}. Biskupic, American Original, 8.


	{color{blue}32}. Jeff rey Toobin, The Nine: Inside the Secret World of the Supreme Court (New York: Random House, 2008), 65 .


	{\color{blue}33}. Tara Trask and Ryan Malphurs, “‘Don’t Poke Scalia!’ Lessons for Trial Lawyers from the Nation’s Highest Court,” Jury Expert 21 (November 2009): 46 .


\section{Chapter 7}


	{\color{blue}1}. Daniel Wilkinson, Silence on the Mountain: Stories of Terror, Betrayal, and Forget- ting in Guatemala (Boston: Houghton Miffl in, 2002), 327–328 .


	{\color{blue}2}. Greg Grandin, The Last Colonial Massacre: Latin America in the Cold War (Chi- cago: University of Chicago Press, 2004), 5, 12, 100 .


	{\color{blue}3}. Ibid. , 16 .


	{color{blue}4}. Ibid. , vi .


	{color{blue}5}. Ibid. , 5, 26, 27, 32, 39 .


	{color{blue}6}. Ibid. , 5, 9, 47, 59, 90 .


	{color{blue}7}. Wilkinson, Silence on the Mountain, 165; Grandin, Last Colonial Massacre, 54.


	{color{blue}8}. Grandin, Last Colonial Massacre, 57, 80, 106, 108, 120 .


	{color{blue}9}. Ibid. , 80 .


	{color{blue}10}. Ibid. , 75, 77, 189–191 .


	{color{blue}11}. Ibid. , 1–3, 148 .


	{color{blue}12}. Ibid. , 190–191 .


\section{Chapter 8}


	{\color{blue}1}. Corey Robin, “The Ex-Cons: Right-Wing Thinkers Go Left!” Lingua Franca (February 2001): 32–33 ; Irving Kristol, interview with author (Washington, D.C., August 31, 2000).


	{\color{blue}2}. Ron Suskind, “Faith, Certainty and the Presidency of George W. Bush,” New York Times Magazine , October 17, 2004.


	{\color{blue}3}. Frank Rich, “The Day before Tuesday,” New York Times, September 15, 2001, A23; Maureen Dowd, “From Botox to Botulism,” New York Times, September


	{\color{blue}4}. Francis Fukuyama, “Francis Fukuyama Says Tuesday’s Attack Marks the End of ‘America’s Exceptionalism,’” Financial Times, September 15, 2001, 1 ; Nicho- las Lemann, “The Next World Order,” New Yorker, April 1, 2002, 48; David Brooks, “Facing Up to Our Fears,” Newsweek, October 22, 2001.


	{\color{blue}5}. Andrew Sullivan, “High Impact: The Dumb Idea of September 11,” New York Times Magazine, December 9, 2001; George Packer, “Recapturing the Flag,” New York Times Magazine, September 30, 2001, 15–16 ; Brooks, “Facing Up to Our Fears”; Brooks, “The Age of Confl ict.”


	{\color{blue}6}. Brooks, “Facing Up to Our Fears.” 7. Ibid.


	{color{blue}8}. On 9/11, trust in government, and the welfare state, see Jacob Weisberg, “Feds Up,” New York Times Magazine, October 21, 2001, 21–22 ; Michael Kelly, “The Left’s Great Divide,” Washington Post, November 7, 2001, A29 ; Robert Putnam, “Bowling Together,” American Prospect ( January 23, 2002) ; Bernard Weinraub, “The Moods They Are a’Changing in Films,” New York Times, October 10, 2001, E1 ; Nina Bernstein, “On Pier 94, a Welfare State That Works, and Possible Models for the Future,” New York Times, Septem- ber 6, 2001, B8 ; Michael Kazin, “The Nation: After the Attacks, Which Side Is the Left On?” New York Times, October 7, 2001, section 4, 4; Katrina van- den Heuvel and Joel Rogers, “What’s Left? A New Life for Progressivism,” Los Angeles Times, November 25, 2001, M2 ; Michael Kelly, “A Renaissance of Liberalism,” Atlantic Monthly ( January 2002): 18–19 . On 9/11 and the cul- ture wars, see Richard Posner, “Strong Fiber after All,” Atlantic Monthly ( January 2002): 22–23 ; Rick Lyman, “At Least for the Moment, a Cooling of the Culture Wars,” New York Times, November 13, 2001, E1 ; Maureen Dowd, “Hunks and Brutes,” New York Times, November 28, 2001, A25 ; Richard Posner, “Reflections on an America Transformed,” New York Times, September 8, 2002, Week in Review, 15. On 9/11, bipartisanship, and the new presidency, see “George Bush, G.O.P. Moderate,” New York Times, September 29, 2001, A18; Maureen Dowd, “Autumn of Fears,” New York Times, November 23, 2001, Week in Review, 17; Richard L. Berke, “Bush ‘Is My Commander,’ Gore Declares in Call for Unity,” New York Times, Sep- tember 30, 2001, A29 ; Frank Bruni, “For President, a Mission and a Role in History,” New York Times, September 21, 2001, A1 ; “Politics Is Adjourned,” New York Times, September 20, 2001, A30; Adam Clymer, “Disaster Forges a Spirit of Cooperation in a Usually Contentious Congress,” New York Times, September 20, 2001, B3 . For a general statement of these various themes, see “In for the Long Haul,” New York Times, September 16, 2001, Week in Review, 10.


	{\color{blue}9}. Judy Keen, “Same President, Diff erent Man in Oval Offi ce,” USA Today, Octo- ber 29, 2001, 6A; Christopher Hitchens, “Images in a Rearview Mirror,” The Nation (December 3, 2001): 9.


	{\color{blue}10}. Lemann, “Next World Order,” 44; Joseph S. Nye Jr., The Paradox of American Power: Why the World’s Only Superpower Can’t Go It Alone (New York: Oxford University Press, 2002), 168 ; Brooks, “The Age of Confl ict.”


	{\color{blue}11}. George Steiner, In Bluebeard’s Castle: Some Notes toward the Redefi nition of Cul- ture (New Haven, Conn.: Yale University Press, 1971), 11 .


	{\color{blue}12}. Cheney cited in Donald Kagan and Frederick W. Kagan, While America Sleeps: Self-Delusion, Military Weakness, and the Threat to Peace Today (New York: St. Martin’s Press, 2000), 294 ; Condoleezza Rice, “Promoting the National In- terest,” Foreign Aff airs ( June 2000): 45 ; Nye, Paradox of American Power, 139.


	{color{blue}13}. The Clinton Foreign Policy Reader: Presidential Speeches with Commentary, ed. Alvin Z. Rubinstein, Albina Shayevich, and Boris Zlotnikov (Armonk, N.Y.: M. E. Sharpe, 2000), 9, 20, 22–23 . It should be pointed out that after several years of reduced military spending, Clinton, in his second term, steadily began to increase military appropriations. Between 1998 and 2000, military expenditures went from $259 billion to $301 billion. This increase in spending coincided with a reconsideration of the dangers confronting the United States. In his last years in offi ce, Clinton began to sound the alarm more force- fully against the threat of terrorism and rogue states. See Clinton Foreign Policy Reader, 36–42; Paul-Marie de la Gorce, “Off ensive New Pentagon Defence Doctrine,” Le Monde Diplomatique, March 2002.


	{\color{blue}14}. David Halberstam, War in a Time of Peace (New York: Scribner, 2001), 22–23, 110–113, 152–153, 160–163, 193, 242 .


	{\color{blue}15}. Nye, Paradox of American Power, 8–11, 110. On occasion, Clinton even went so far as to suggest that pouring so much money into fi ghting the Cold War was, if not exactly a waste, then at least an unnecessary strain on the nation’s vital resources. “The Cold War,” he said at American University in 1993, “was a draining time. We devoted trillions of dollars to it, much more than many of our more visionary leaders thought we should have.” Clinton Foreign Policy Reader, 9.


	{\color{blue}16}. Brooks, “The Age of Confl ict”; Robert D. Kaplan, The Coming Anarchy: Shat- tering the Dreams of the Post Cold War (New York: Vintage, 2000), 23–24, 89 . Also see Francis Fukuyama, The End of History and the Last Man (New York: Harper Collins, 1992, 2002), 304–305, 311–312 .


	{\color{blue}17}. See Robert Putnam, Bowling Alone: The Collapse and Revival of American Com- munity (New York: Simon & Schuster, 2000) ; Dinesh D’Souza, The Virtue of Prosperity: Finding Values in an Age of Techno-Affl uence (New York: Simon & Schuster, 2000) ; John B. Judis, The Paradox of American Democracy: Elites, Special Interests, and the Betrayal of the Public Trust (New York: Pantheon, 2000) ; Kagan and Kagan, While America Sleeps.


	{\color{blue}18}. Indeed, the Clinton administration’s many pronouncements on the issue of multi- and unilateralism sound remarkably similar to those of the administra- tion of George W. Bush. In an address to the United Nations in 1993, Clinton stated, “We will often work in partnership with others and through multilat- eral institutions such as the United Nations. It is in our national interests to do


	{\color{blue}19}. Kagan and Kagan, While America Sleeps, 1–2, 4; Kaplan, Coming Anarchy, 157, 172, 176.


	{\color{blue}20}. Brooks, “Age of Confl ict”; Steven Mufson, “The Way Bush Sees the World,” Washington Post, February 17, 2002, B1: “Paul Wolfowitz, Velociraptor.”


	{\color{blue}21}. Lemann, “Next World Order,” 43, 47–48; Seymour M. Hersh, “The Iraq Hawks,” New Yorker (December 24 and 31, 2001), 61 ; Kagan, “Fightin’ Demo- crats”; Kagan and Kagan, While America Sleeps, 293, 295.


	{\color{blue}22}. Emily Eakin, “All Roads Lead to D.C.,” New York Times, March 31, 2002, Week in Review, 4; Lemann, “Next World Order,” 44. Also see Alexander Stille, “What Is America’s Place in the World Now?” New York Times, January 12, 2002, B7; Michael Ignatieff , “The American Empire (Get Used to It),” New York Times Magazine, January 5, 2003, 22ff ; Bill Keller, “The I-Can’t-Believe-I’m-a- Hawk Club,” New York Times February 8, 2003, A17; Lawrence Kaplan, “Regime Change,” New Republic (March 3, 2003).


	{\color{blue}23}. Lemann, “Next World Order,” 43–44; Hersh, “The Iraq Hawks,” 61; George W. Bush, “State of the Union Address,” New York Times, January 30, 2002, A22: Mufson, “Way Bush Sees the World,” B1.


	{\color{blue}24}. Eric Schmitt and Steve Lee Myers, “U.S. Steps Up Air Attack, While Defending Results of Campaign,” New York Times, October 26, 2001, B1; Susan Sachs, “U.S. Appears to Be Losing Public Relations War So Far,” New York Times, October 28, 2001, B8: Warren Hoge, “Public Apprehension Felt in Europe over the Goals of Afghanistan Bombings,” New York Times, November 1, 2001, B2; Dana Canedy, “Vietnam-Era G.I.’s Watch New War Warily,” New York Times, November 12, 2001, B9.


	{\color{blue}25}. Robin Wright, “Urgent Calls for Peace in Mideast Ring Hollow as Prospects Dwindle,” Los Angeles Times, March 31, 2002 .


	{\color{blue}26}. Ibid.


	{color{blue}27}. David E. Rosenbaum, “Senate Deletes Higher Mileage Standard in Energy Bill,” New York Times, March 14, 2002, A28.


	{\color{blue}28}. Diana B. Henriques and David Barstow, “Victim’s Fund Likely to Pay Average of $1.6 Million Each,” New York Times, December 21, 2001, A1. For an excellent critique, see Eve Weinbaum and Max Page, “Compensate All 9/11 Families Equally,” Christian Science Monitor, January 4, 2002, 11 .


	{\color{blue}29}. Tim Jones, “Military Sees No Rush to Enlist,” Chicago Tribune, March 24, 2002; David W. Chen, “Armed Forces Stress Careers, Not Current War,” New York Times, October 20, 2001, B10.


	{\color{blue}30}. Andrew Gumbel, “Pentagon Targets Latinos and Mexicans to Man the Front Lines in War on Terror,” The Independent, September 10, 2003.


	{\color{blue}31}. R. W. Apple Jr., “Nature of Foe Is Obstacle in Appealing for Sacrifi ce,” New York Times, October 15, 2001, B2; Frank Rich, “War Is Heck,” New York Times, November 10, 2001, A23; Alison Mitchell, “After Asking for Volunteers, Govern- ment Tries to Determine What They Will Do,” New York Times, November 10, 2001, B7. Also see Michael Lipsky, “The War at Home: Wartime Used to Entail National Unity and Sacrifi ce,” American Prospect ( January 28, 2002): 15–16 .


	{color{blue}32}. The NewsHour, October 29, 2001, http://www.pbs.org/newshour/bb/white_ house/july-dec01/historians_10-29.html , accessed April 8, 2011.


	{\color{blue}33}. Elisabeth Bumiller, “Bush Asks Volunteers to Join Fight on Terrorism,” New York Times, January 31, 2002, A20; Mitchell, “After Asking for Volunteers,” B7. Also see David Brooks, “Love the Service Around Here,” New York Times Mag- azine, November 25, 2001, 34 .


	{\color{blue}34}. This corporate abdication of government power applies even to those cases— like the fi rst Gulf War or the signing of NAFTA—where many had thought they’d seen the heavy footprints of corporate America. According to the best accounts of the Gulf War and NAFTA, it was political offi cials, particularly the fi rst President Bush, who pressed these policies, often persuading a reluctant business and military community to follow along. John R. MacArthur, The Selling of “ Free Trade”: NAFTA, Washington, and the Subversion of American Democracy (Berkeley: University of California Press, 2000), 137, 170, 174–175, 194 ; Halberstam, Peace in Time of War, 69–70; Kagan and Kagan, While America Sleeps, 244–250.


	{color{blue}35}. Thomas Friedman, The Lexus and the Olive Tree: Understanding Globalization (New York: Farrar, Straus, Giroux, 1999), 373 .


\section{Chapter 9}


	{\color{blue}1}. Speaking to senior SS offi cers in Posen on October 4, 1943, Himmler declared, “We had the moral right, we had the duty to our people, to kill this people which wanted to kill us.” The Nazi Germany Sourcebook: An Anthology of Texts , ed. Roderick Stackelberg and Sally A. Winkle (New York: Routledge, 2002), 370 . Also see J. Arch Getty and Oleg V. Naumov, The Road to Terror: Stalin and the Self-Destruction of the Bolsheviks, 1932–1939 (New Haven, Conn.: Yale Univer-


	{\color{blue}2}. Seymour M. Hersh, Chain of Command: The Road from 9/11 to Abu Ghraib (New York: Harper Collins, 2004), 38–39 ; Jane Mayer, The Dark Side: The Inside Story of How the War on Terror Turned into a War on American Ideals (New York: Dou- bleday, 2008), 167–168 .


	{\color{blue}3}. Joseph S. Nye Jr., The Paradox of American Power: Why the World’s Only Super- power Can’t Go It Alone (New York: Oxford University Press, 2002), 159, 163 .


	{\color{blue}4}. Nye, Paradox of American Power, 135, 139.


	{color{blue}5}. Michael Walzer, Arguing about War (New Haven, Conn.: Yale University Press, 2004), 33, 43 .


	{\color{blue}6}. Cited in Otto Kirchheimer, Political Justice: The Use of Legal Procedure for Polit- ical Ends (Princeton, N.J.: Princeton University Press, 1961), 29 .


	{\color{blue}7}. United States v. Dennis et al. v. United States , 183 F.2d 212 (1950).


	{color{blue}8}. Francis Bacon, Considerations Touching a War with Spain , in The Works of Fran- cis Bacon , vol. 2 (Philadelphia: A. Hart, 1850), 205 .


	{\color{blue}9}. Adolf Hitler, speech on the anniversary of the 1923 Putsch (November 8, 1942), in Nazi Germany Sourcebook , 295.


	{\color{blue}10}. Hersh, Chain of Command, 231.


	{color{blue}11}. http://frwebgate.access.gpo.gov/cgi-bin/getdoc.cgi?dbname=2003_presi- dential_documents&docid=pd03fe03_txt-6 , accessed April 8, 2011.


	{\color{blue}12}. http://www.pbs.org/wgbh/pages/frontline/shows/wmd/etc/script.html , accessed April 8, 2011.


	{\color{blue}13}. Edmund Burke, Reflections on the Revolution in France , ed. J. C. D. Clark (Stanford, Calif.: Stanford University Press, 2001), 154 .


	{\color{blue}14}. Walzer, Arguing about War, 88, 155, 160 .


	{color{blue}15}. Ibid. , 53 .


	{color{blue}16}. Cited in Avi Shlaim, The Iron Wall: Israel and the Arab World (New York: Norton, 2001), 501 .


	{\color{blue}17}. Cited in Robert O. Paxton, The Anatomy of Fascism (New York: Knopf, 2004), 156 .


	{\color{blue}18}. Thomas Carlyle, “Signs of the Times,” in A Carlyle Reader , ed. G. B. Tennyson (New York: Cambridge University Press, 1984), 34 ; Chateaubriand cited in Roger Boesche, The Strange Liberalism of Alexis de Tocqueville (Ithaca, N.Y.: Cornell University Press, 1987), 84 .


	{\color{blue}19}. Hersh, Chain of Command, 16, 209, 220, 267; David Brooks, “The Art of Intelli- gence,” New York Times, April 2, 2005.


	{\color{blue}20}. Hersh, Chain of Command, 13, 17, 62, 265, 271. For additional examples, see Mayer, Dark Side, 36, 41–43, 69, 80, 124–125, 132, 161, 241.


	{\color{blue}21}. Hersh, Chain of Command, 40; Christian Parenti, The Freedom: Shadows and Hallucinations in Occupied Iraq (New York: New Press, 2005), 141 .


	{\color{blue}22}. Jean Bethke Elshtain, “Refl ections on the Problem of ‘Dirty Hands,’” in Tor- ture , ed. Sanford Levinson (New York: Oxford University Press, 2004), 79 .


	{\color{blue}23}. Elshtain, “Refl ections on the Problem,” 80, 85–86; Hersh, 22; Mark Danner, Torture and Truth: America, Abu Ghraib, and the War on Terror (New York: New York Review Books, 2004), 4, 6, 13, 240, 248, 262, 292, 514, 538 .


	{\color{blue}24}. Sanford Levinson, editor of Torture , writes that all of the essays gathered in the book, including Elshtain’s, were written before the revelations of Abu Ghraib. Though he writes that “no doubt many of the authors would wish to rewrite some of their remarks,” none did. He also notes that “the brutal fact is that far less rewriting would be necessary than some might wish.” Seven years later, I can fi nd no evidence that Elshtain has rewritten or retracted any of her essay. Levinson, “Acknowledgments,” in Torture , 20.


	{\color{blue}25}. Levinson, “Contemplating Torture,” in Torture , 37.


	{color{blue}26}. Elshtain, “Refl ections on the Problem,” 83, 86.


	{color{blue}27}. Michael Walzer, “Political Action: The Problem of Dirty Hands,” in Torture , 62–63; Walzer, Arguing about War , 45.


	{\color{blue}28}. Isaiah Berlin, “The Counter-Enlightenment,” in Against the Current: Essays in the History of Ideas (Princeton, N.J.: Princeton University Press, 2001), 3 .


	{\color{blue}29}. Elshtain, “Refl ections on the Problem,” 83–84.


	{color{blue}30}. Isaiah Berlin, “The Apotheosis of the Romantic Will,” The Crooked Timber of


\section{58; Walzer, “Political Action,” 72.}


	{\color{blue}33}. Machiavelli, letter to Vettori (April 16, 1527), in The Letters of Machiavelli , ed. Allan Gilbert (New York: Capricorn, 1961), 249 .


\section{Chapter 10}


	{\color{blue}1}. David K. Johnson, The Lavender Scare: The Cold War Persecution of Gays and Lesbians in the Federal Government (Chicago: University of Chicago Press, 2004), 2, 19, 55, 138 .


	{\color{blue}2}. Ibid. , 1, 4–5, 102–103, 114 .


	{color{blue}3}. Ibid. , 9–10, 108–109 .


	{color{blue}4}. Ibid. , 16, 31–37 .


	{color{blue}5}. Aaron Belkin, personal communication, December 10, 2010.


	{color{blue}6}. Johnson, Lavender Scare, 70–72 .


	{color{blue}7}. Ibid. , 72 .


	{color{blue}8}. John Stuart Mill, Utilitarianism (New York: New American Library, 1974), 310 . Also see John Dunn, “Political Obligation,” in The History of Political Theory and Other Essays (New York: Cambridge University Press, 1996), 66–90 ; Ber- nard Williams, In the Beginning Was the Deed: Realism and Moralism in Political Argument , ed. Geoff rey Hawthorn (Princeton, N.J.: Princeton University Press, 2005), 3 . Mill was referring to the security of persons rather than of nations or states. But his argument about personal security is often extended to nations and states, which are conceived to be persons writ large. See Michael Walzer, 273


	{\color{blue}9}. Arnold Wolfers, “National Security as an Ambiguous Symbol,” Political Sci- ence Quarterly 67 (December 1952): 481 .


	{\color{blue}10}. David Cole and James. X. Dempsey, Terrorism and the Constitution: Sacrifi cing Civil Liberties in the Name of National Security (New York: New Press, 2002, 2006), x, 210 . Also see Jane Mayer, The Dark Side: The Inside Story of How the War on Terror Turned into a War on American Ideals (New York: Doubleday, 2008), 16–17, 34–36 .


	{\color{blue}11}. John Solomon, “Bureaucracy Impedes Bomb Detection Work,” Associated Press, August 12, 2006.


	{\color{blue}12}. Cole and Dempsey, Terrorism , 177–178, 234. Also see Mayer, Dark Side, 12–13, 105–106, 116, 119, 156, 166, 177.


	{\color{blue}13}. Nancy V. Baker, General Ashcroft: Attorney at War (Lawrence: University Press of Kansas, 2006), 5, 8, 36, 45, 54 .


	{\color{blue}14}. Baker, General Ashcroft, 67, 82, 106, 108.


	{color{blue}15}. Mayer, Dark Side, 34, 41, 47, 52, 55–67.


	{color{blue}16}. Johnson, Lavender Scare, 56 .


	{color{blue}17}. Ibid. , 42–56 .


	{color{blue}18}. Ibid. , 55, 90 .


	{color{blue}19}. Baker, General Ashcroft, 67.


	{color{blue}20}. Georg Lukács, The Historical Novel (Boston: Beacon, 1962), 22–23 .


	{color{blue}21}. James Risen, State of War: The Secret History of the CIA and the Bush Administra- tion (New York: Free Press, 2006), 39, 44 .


	{\color{blue}22}. Ibid. , 50–52 .


	{color{blue}23}. http://thinkprogress.org/cheney-teleconference , accessed April 8, 2011.


	{color{blue}24}. http://frwebgate.access.gpo.gov/cgi-bin/getdoc.cgi?dbname=107_cong_ public_laws&docid=f:publ056.107.pdf , accessed April 8, 2011.


	{\color{blue}25}. Nancy Chang, Silencing Political Dissent: How Post–September 11 Anti-Terrorism Measures Threaten Our Civil Liberties (New York: Seven Stories Press, 2002), 44–45 .


	{\color{blue}26}. http://www.leg.state.or.us/03reg/pdf/SB742.pdf , accessed April 8, 2011.


	{color{blue}27}. Randal C. Archibold, “Protesters Try to Get in Last Word before Curtain Falls,” New York Times, September 3, 2004.


	{\color{blue}28}. American Communications Assn. v. Douds , 339 U.S. 382 (1950).


	{color{blue}29}. Corey Robin, Fear: The History of a Political Idea (New York: Oxford University Press, 2004), 190 .


	{\color{blue}30}. Ibid.


	{color{blue}31}. Ralph S. Brown Jr., Loyalty and Security: Employment Tests in the United States (New Haven, Conn.: Yale University Press, 1958), 181 ; Griffi n Fariello, Red Scare: Memories of the American Inquisition (New York: Avon, 1995), 43 .


	{\color{blue}32}. Gary Younge, “Between a Crisis and a Panic,” Guardian, March 21, 2005.


	{color{blue}33}. Eric Boehlert, Lapdogs: How the Press Rolled Over for Bush (New York: Free Press, 2006), 17 .


	{\color{blue}34}. Ibid. , 210–211, 268, 278–279 .


	{color{blue}35}. Cole and Dempsey, Terrorism, 221–222.


	{color{blue}36}. David Cole, No Equal Justice: Race and Class in the American Criminal Justice System (New York: New Press, 1999), 7 .


	{\color{blue}37}. David Cole, Enemy Aliens: Double Standards and Constitutional Freedoms in the War on Terrorism (New York: New Press, 2003), 4–5 .


	{\color{blue}38}. Ibid. , 6, 18 .


	{color{blue}39}. Ibid. , 91–101 .


	{color{blue}40}. John Locke, A Letter Concerning Toleration , ed. James H. Tully (Indianapolis: Hackett, 1983), 46 ; J. S. Mill, On Liberty , in On Liberty and Other Writings , ed. Stefan Collini (New York: Cambridge University Press, 1989), 13 ; Schenck v. United States , 249 U.S. 47 (1919).


	{\color{blue}41}. Patrick Devlin, The Enforcement of Morals (London: Oxford University Press, 1965), 3, 9, 13–14 .


	{\color{blue}42}. Nicola Lacey, A Life of H. L. A. Hart: The Nightmare and the Noble Dream (New York: Oxford University Press, 2004), 220–221 .


	{\color{blue}43}. H. L. A. Hart, “Immorality and Treason,” The Listener ( July 30, 1959).


\section{Chapter 11}


	{\color{blue}1}. Jim Sidanius, Michael Mitchell, Hillary Haley, and Carlos David Navarrete, “Support for Harsh Criminal Sanctions and Social Dominance Beliefs,” Social Justice Research 19 (December 2006): 440 ; Tom Pyszczynski, Abdolhossein Abdollahi, Sheldon Solomon, Jeff Greenberg, Florette Cohen, and David Weise, “Mortality Salience, Martyrdom, and Military Might: The Great Satan Versus the Axis of Evil,” Personality and Social Psychology Bulletin 32 (April 2006): 525–537 ; http://www.gallup.com/poll/101863/Sixtynine-Percent-Americans- Support-Death-Penalty.aspx , accessed April 5, 2011; http://pewforum.org/Poli- tics-and-Elections/The-Torture-Debate-A-Closer-Look.aspx , accessed April 5, 2011; http://www.sourcewatch.org/index.php?title=McCain_Amendment_ No._1977 , accessed April 5, 2011; Sean Olson, “Senate Approves Abolishment of Death Penalty,” Albuquerque Journal (March 13, 2009). I am grateful to Shang Ha for providing me with these citations.


	{\color{blue}2}. Andrew Sullivan, The Conservative Soul: Fundamentalism, Freedom, and the Future of the Right (New York: Harper Perennial, 2006), 276–277 .


	{\color{blue}3}. Francis Fukuyama, The End of History and the Last Man (New York: Harper Collins, 1992), xxiii, 147, 150–151, 255–256, 318, 329 .


	{\color{blue}4}. This statement comes from MacArthur’s 1962 address at West Point, and he attributes it to Plato. No scholar has ever found such a statement in Plato, but it (and the Plato attribution) does appear on a wall in London’s Imperial War Museum and in Ridley Scott’s 2001 fi lm Black Hawk Down . The most likely


	{\color{blue}5}. Selections from Treitschke’s Lectures on Politics , trans. Adam L. Gowans (New York: Frederick A. Stokes, 1914), 24–25 .


	{\color{blue}6}. Edmund Burke, A Philosophical Enquiry into the Origin of Our Ideas of the Sublime and the Beautiful , ed. David Womersley (New York: Penguin, 1998, 2004), 79 .


	{\color{blue}7}. Ibid. , 82 .


	{color{blue}8}. Ibid. , 88 .


	{color{blue}9}. Ibid. , 96 .


	{color{blue}10}. Ibid. , 164 .


	{color{blue}11}. Ibid. , 177–178 .


	{color{blue}12}. Michael Oakeshott, “On Being Conservative,” in Rationalism in Politics and Other Essays (Indianapolis: Liberty Press, 1962), 408 . Also see Walter Bagehot, “Intellectual Conservatism,” in The Portable Conservative Reader , ed. Russell Kirk (New York: Penguin, 1982), 239–241 ; Russell Kirk, “What Is Conserva- tism?” in The Essential Russell Kirk , ed. George A. Panichas (Wilmington, Del.: ISI Books, 2007), 7 ; Roger Scruton, The Meaning of Conservatism (London: Macmillan, 1980, 1984), 21–22, 40–43 ; Robert Nisbet, Conservatism (Minneapo- lis: University of Minnesota Press, 1986), 26–27 .


	{\color{blue}13}. Ronald Reagan, First Inaugural Address and address before a Joint Session of the Congress (April 28, 1981), in Conservatism in America since 1930 , ed. Gregory L. Schneider (New York: New York University Press, 2003), 343, 344, 351, 352 .


	{color{blue}14}. Barry Goldwater, acceptance speech at 1964 Republican National Convention ( July 16, 1964), in Conservatism in America , 238–239.


	{\color{blue}15}. Hugo Young, One of Us (London: Macmillan, 1989, 1991), 224 .


	{color{blue}16}. William Manchester, The Last Lion: Winston Spencer Churchill: Visions of Glory 1874 – 1932 (Boston: Little, Brown, 1982), 222–231 .


	{\color{blue}17}. Winston Churchill, My Early Life: 1874 – 1904 (New York: Scribner, 1996), 77 .


	{color{blue}18}. Burke , Sublime and the Beautiful , 177 .


	{color{blue}19}. Ibid. , 86 .


	{color{blue}20}. Ibid. , 101, 106, 108, 111 .


	{color{blue}21}. Ibid. , 96, 123 .


	{color{blue}22}. Ibid. , 121 .


	{color{blue}23}. Jean-Jacques Rousseau, Discourse on the Origin and Foundations of Inequality among Men , in Rousseau’s Political Writings , ed. Alan Ritter and Julia Conaway Bondanella (New York: Norton, 1988), 54 .


	{\color{blue}24}. John Adams, Discourses on Davila , in The Political Writings of John Adams (India- napolis: Hackett, 2003), 176 .


	{\color{blue}25}. Ibid. , 183–184 .


	{color{blue}26}. Burke, Sublime and the Beautiful , 108 .


	{color{blue}27}. Ibid. , 109 .


	{color{blue}28}. Ibid.


	{\color{blue}29}. Burke, Refl ections on the Revolution in France , ed. J. C. D. Clark (Stanford, Calif.: Stanford University Press, 2001), 207–208, 275 .


	{\color{blue}30}. Burke, Letters on a Regicide Peace , ed. E. J. Payne (Indianapolis: Liberty Fund, 1999), 157 .


	{\color{blue}31}. Joseph de Maistre, Considerations on France , trans. and ed. Richard A. Lebrun (New York: Cambridge University Press, 1974, 1994), 4, 9–10, 13–14, 16–18, 100 .


	{color{blue}32}. Ibid. , 17 . For other examples, see Jean-Louis Darcel, “The Roads of Exile, 1792–1817,” and Darcel, “Joseph de Maistre and the House of Savoy: Some Aspects of His Career,” in Joseph de Maistre’s Life, Thought, and Infl uence: Selected Studies , ed. Richard A. Lebrun (Montreal: McGill-Queen’s University Press, 2001), 16, 19–20, 52 .


	{\color{blue}33}. Cf. David Bromwich, “Introduction,” in Edmund Burke, On Empire, Liberty, and Reform: Speeches and Letters , ed. David Bromwich (New Haven, Conn.: Yale Uni- versity Press, 2000), 10 ; Jan-Werner Müller, “Comprehending Conservatism: A New Framework for Analysis,” Journal of Political Ideologies 11 (October 2006): 360.


	{color{blue}34}. Georges Sorel, Refl ections on Violence , ed. Jeremy Jennings (New York: Cam- bridge University Press, 1999), 61–63, 72, 75–76 .


	{\color{blue}35}. Carl Schmitt, The Concept of the Political , trans. George Schwab (New Bruns- wick, N.J.: Rutgers University Press, 1976), 22, 48, 62–63, 65, 71–72, 74, 78 .


	{\color{blue}36}. Schmitt, Concept of the Political, 63.


	{color{blue}37}. Sorel, Refl ections on Violence, 75.


	{color{blue}38}. Theodore Roosevelt, address to Naval War College ( June, 2, 1897), in Theodore Roosevelt: An American Mind. Selected Writings , ed. Mario R. DiNunzio (New York: Penguin, 1994), 175–176, 179 .


	{\color{blue}39}. Roosevelt, address to Hamilton Club of Chicago (April 10, 1899), and An Auto- biography , in Theodore Roosevelt , 186, 194.


	{color{blue}40}. Roosevelt, Naval War College address, 174.


	{color{blue}41}. John C. Calhoun, “Speech on the Reception of Abolitionist Petitions” (Febru- ary 6, 1837), in Union and Liberty: The Political Philosophy of John C. Calhoun , ed. Ross M. Lence (Indianapolis: Liberty Fund, 1992), 476 .


	{\color{blue}42}. Barry Goldwater, The Conscience of a Conservative (Princeton, N.J.: Princeton University Press, 1960, 2007), 1 .


	{\color{blue}43}. Fukuyama, End of History, 315–318, 329; also see chapter 8.


	{color{blue}44}. John Milton, Aeropagitica , in Complete Poems and Major Prose , ed. Merritt Y. Hughes (New York: Macmillan, 1957), 728 .


	{\color{blue}45}. Burke, Sublime and the Beautiful , 145.


	{color{blue}46}. Maistre, Considerations on France , 14, 16, 18–19. Also see Darcel, “The Ap prentice Years of a Counter-Revolutionary: Joseph de Maistre in Lausanne, 1793–1797,” in Joseph de Maistre’s Life, Thought, and Infl uence , 43–44 .


	{\color{blue}47}. Maistre, Considerations on France , 77.


	{color{blue}48}. Sorel, Refl ections on Violence, 63, 160–161.


	{color{blue}49}. Cited in William Pfaff , The Bullet’s Song: Romantic Violence and Utopia (New York: Simon and Schuster, 2004), 97 .


	{color{blue}50}. Sorel, Refl ections on Violence, 76–78, 85.


	{\color{blue}51}. What follows is an abridged account of my discussion in Fear: The History of a Political Idea (New York: Oxford University Press, 2004), 88–94. Sources for all quotations cited here can be found there.


	{\color{blue}52}. Fukuyama, End of History , 148, 180, 304–305, 312, 314, 328–329.


	{color{blue}53}. E. M. Forster, A Passage to India (New York: Harcourt, 1924), 289 .


	{color{blue}54}. Roosevelt, The Rough Riders , in Theodore Roosevelt , 30–32, 37. One might also point to Roosevelt’s Naval War College address, where several thousand words in praise of manliness and military preparedness come to a climax in a call for the United States to build a modern navy that might well never be used. Theodore Roosevelt , 178.


	{\color{blue}55}. Roosevelt, Hamilton Club address, Theodore Roosevelt , 185, 188.


	{color{blue}56}. Roosevelt, Lincoln Club address of February 1899, and Hamilton Club ad- dress, ibid. , 182, 189.


	{\color{blue}57}. R. J. B. Bosworth, Mussolini (New York: Oxford University Press, 2002), 167– 169 ; Robert O. Paxton, The Anatomy of Fascism (New York: Knopf, 2004), 87–91 .


	{\color{blue}58}. Sam Tanenhaus, The Death of Conservatism (New York: Random House, 2009) .


	{color{blue}59}. Seymour Hersh, Chain of Command: The Road from 9/11 to Abu Ghraib (New York: Harper Collins, 2004) ; Jane Mayer, The Dark Side: The Inside Story of How the War on Terror Turned into a War on American Ideals (New York: Doubleday, 2008) .


	{\color{blue}60}. Mayer, Dark Side, 69, 132, 241 .


	{color{blue}61}. Ibid. , 55, 120, 150, 167, 231, 301 .


	{color{blue}62}. Ibid. , 223 .


	{color{blue}63}. Ibid.


	{color{blue}64}. Burke, Sublime and the Beautiful , 86, 92, 165 .


	{color{blue}65}. Ibid. , 104 .


	{color{blue}66}. Ibid. , 105 .


	{color{blue}67}. Ibid. , 106 .


	{color{blue}68}. Burke, Refl ections , 232, 239. Conclusion


	{\color{blue}1}. Frank Meyer, “Freedom, Tradition, Conservatism,” in In Defense of Freedom and Related Essays (Indianapolis: Liberty Fund, 1996), 15 ; Roger Scruton, The Meaning of Conservatism (London: Macmillan, 1980, 1984), 11 ; Friedrich A. Hayek, The Constitution of Liberty (Chicago: University of Chicago Press, 1960), 7 .


	{\color{blue}2}. David Frum, Comeback: Conservatism That Can Win Again (New York: Double- day, 2008) ; Ross Douthat and Reihan Salam, Grand New Party: How Republi- cans Can Win the Working Class and Save the American Dream (New York: Doubleday, 2008) ; Mickey Edwards, Reclaiming Conservatism: How a Great American Political Movement Got Lost—and How It Can Find Its Way Back (New York: Oxford University Press, 2008) ; John J. DiIulio Jr., Godly Republic: A Cen- trist Blueprint for America’s Faith-Based Future (Berkeley: University of Califor- nia Press, 2007) ; Michael J. Gerson, Heroic Conservatism: Why Republicans Need N O T E S T O P A G E S 2 3 5 – 2 4 7


	{\color{blue}2}. David Frum, Comeback: Conservatism That Can Win Again (New York: Double- day, 2008) ; Ross Douthat and Reihan Salam, Grand New Party: How Republi- cans Can Win the Working Class and Save the American Dream (New York: Doubleday, 2008) ; Mickey Edwards, Reclaiming Conservatism: How a Great American Political Movement Got Lost—and How It Can Find Its Way Back (New York: Oxford University Press, 2008) ; John J. DiIulio Jr., Godly Republic: A Cen- trist Blueprint for America’s Faith-Based Future (Berkeley: University of Califor- nia Press, 2007) ; Michael J. Gerson, Heroic Conservatism: Why Republicans Need N O T E S T O P A G E S 2 3 5 – 2 4 7


	{\color{blue}3}. Sullivan, Conservative Soul , 9 .


	{color{blue}4}. George Packer, “The Fall of Conservatism,” The New Yorker (May 26, 2008) .