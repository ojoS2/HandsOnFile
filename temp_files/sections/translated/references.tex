


\section{600 / Navarro}


\section{Table 21}


\section{No. of years of socialist rule}


\section{Union}


\section{Infant mortality, per 1,ooO}


\section{Sweden Norway Denmark United States}


\section{CONCLUSIONS}


\section{both in space and tim-pitalism}


\section{REFERENCES}


	{\color{blue}1}. Fukuyama, F. The end of history. The Nationallnterest, Summer 1990.


	{color{blue}2}. MacPherson, S. Five hundred million children. In Poverty and Child Weyare in the Third World. St. Martin’s Press, New York, 1987.


	{\color{blue}3}. Excerpts from the Pope’s Encyclicals: On Giving Capitalism a Human Face. New York Times, May 3, 1991, p. A10.


	{\color{blue}4}. Tabb, W. T. The Future of Socialism: Perspective from the Left. Monthly Review Press, New York, 1990.


	{\color{blue}5}. Pmworski, A. Could we feed everyone? The irrationality of capitalism and the unfeasibility of socialism. Politics andSociety, March 1991, p.


	{color{blue}14}.


	{\color{blue}6}. Roosevelt, F. D. Presidential Address to U.S. Congress, January 11, 1944.


	{color{blue}7}. McKeown, T. The Role of Medicine: Dream, Mirage, or Nemesis? Princeton University Press, Princeton, N.J., 1979.


\section{Has Socialism Failed? / 601}


	{\color{blue}8}. Baltimore City Health Department. Report on Infant Mortality. 1991.


	{color{blue}9}. Pan-American Health Organization. Health Conditions in the Americas, Vol. I. Washington, D.C., 1990.


	{\color{blue}10}. UNICEF. World Statistics on Children: UNICEF Statistical Pocketbook. New York, 1986.


	{color{blue}11}. UNICEF. Statistics on Children in UNICEF Assisted Countries. New York, 1987.


	{color{blue}12}. Nutrition survey of sixth graders of Cuba. J. Nutr. 64(3), March 1958.


	{color{blue}13}. UNICEF. The State of the World’s Children New York. 1989.


	{color{blue}14}. Dim-Briquets, S. The Health Revolution in Cuba. University of Texas Press, 1983.


	{color{blue}15}. Pan-American Health Organization. Health Conditions in the Americas 1953-56. Washing- ton, D.C.. 1956.


	{\color{blue}16}. Statistical Yearbook for Latin America and the Caribbean 1990.


	{color{blue}17}. Escudero, J. C. Malnutrition in Latin America. Unpublished manuscript. University of Buenos Aires, Argentina, 1986.


	{\color{blue}18}. Pan-American Health Organization. Cholera Report. Washington, D.C., 1991.


	{color{blue}19}. World Bank. China: The Health Sector. A World Bank Country Study. New York, November 1984.


	{\color{blue}20}. Eberstadt, N. The health crisis in the USSR. New YorkReview ofBmks, February 19,1981.


	{color{blue}21}. Szymanski, A. On the uses of disinfonnation to legitimize the revival of the cold war: Health in the USSR. Int. J. Healfh Serve. 1 2 481496,1982.


	{\color{blue}22}. Navarro, V. Social Security and Medicine in the USSR. Lexington Books, Lexington, Mass., 1976.


	{\color{blue}23}. Przeworski, A. Social democracy as a historical phenomenon. New LeJ Rev. 122 45,1980.


	{color{blue}24}. Kautsky, K. The Class Struggle, p.


	{color{blue}186}. Norton, New York, 1971.


	{color{blue}25}. Navarro, V. Production and the welfare state: The political context of reforms. Int. J. Health Sew. 21: 585414.1991.


	{\color{blue}26}. Himmelstrand, U. Sweden: Paradise in trouble. In Beyond fhe Weyare State, edited by I. Hare. Schocken Books, New York, 1982.


	{\color{blue}27}. Korpi, W. The Democratic Class Struggle. Routledge and Kegan Paul, Boston, 1983.


\section{Direct reprint requests to:}