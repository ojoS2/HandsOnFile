


\section{References}


\section{Arthur, C.J. (1992) Marx’s Capital: A Student Edition. London: Lawrence &}


\section{Wishart.}


\section{Publishers.}


\section{Volume Two of Capital. London: Macmillan.}


\section{Class Structure and Economic Development Cambridge: Cambridge University Press.}


\section{Banaji, J. (2010) Theory as History: Essays on Modes of Production and}


\section{Exploitation. Leiden and Boston: Brill.}


\section{Nature of Oil Rent’, Capital & Class 39, pp.82–112.}


\section{Blackledge, P. (2006) Reflections on the Marxist Theory of History. Manchester:}


\section{Manchester University Press.}


\section{Bleaney, M. (1976) Underconsumption Theories: A History and Critical Analysis.}


\section{London: Lawrence & Wishart.}


\section{Bottomore, T. (ed.) (1991) A Dictionary of Marxist Thought. Oxford: Basil}


\section{Blackwell.}


\section{Bowring, F. (2003) ‘Manufacturing Scarcity: Food Biotechnology and the}


\section{Life-Sciences Industry’, Capital & Class 79, pp.107–44.}


\section{Brenner, R. (1986) ‘The Social Basis of Economic Development’, in J. Roemer}


\section{(ed.) Analytical Marxism. Cambridge: Cambridge University Press.}


\section{Brenner, R. (1998) ‘The Economics of Global Turbulence’, New Left Review}


\section{London: Verso.}


\section{Brewer, A. (1989) Marxist Theories of Imperialism: A Critical Survey. London:}


\section{Routledge.}


\section{Brighton Labour Process Group (1977) ‘The Capitalist Labour Process’, Capital}


\section{Marxism. London, Routledge.}


\section{Burawoy, M. (1979) Manufacturing Consent: Changes in the Labor Process under}


\section{Monopoly Capitalism. Chicago: University of Chicago Press.}


\section{Martin’s Press.}


\section{Burkett, P. (2003) ‘Capitalism, Nature and the Class Struggle’, in A. Saad-Filho}


\section{Byres, T. (1996) Capitalism from above and Capitalism from below. London:}


\section{Macmillan.}


\section{Callinicos, A. (2014) Deciphering Capital: Marx’s Capital and Its Destiny.}


\section{London: Bookmarks.}


\section{London: Bookmark Publications.}


\section{Christophers, B. (2015a) ‘The Limits to Financialization’, Dialogues in Human}


\section{Christophers, B.}


\section{‘From Financialization}


\section{to Finance: For}


\section{“De-Financialization”’, Dialogues in Human Geography, 5(2), pp.229–32.}


\section{Davidson, N. (2010) How Revolutionary Were the Bourgeois Revolutions? Chicago:}


\section{Haymarket.}


\section{Revolution. Cambridge, Mass.: Harvard University Press.}


\section{Duménil, G. and Lévy, D. (2011) The Crisis of Neoliberalism. Cambridge, Mass.:}


\section{Harvard University Press, chs.15–20.}


\section{Elson, D. (1979a) Value: The Representation of Labour in Capitalism. London:}


\section{CSE Books, reproduced by Verso, 2015.}


\section{(CD-ROM). London: Electric Book Company.}


\section{Etherington, N. (1984) Theories of Imperialism: War, Conquest and Capital.}


\section{London: Croom Helm.}


\section{Conference of Socialist Economists 12, pp.82–96.}


\section{Fine, B. (1983b) ‘Marx on Economic Relations under Socialism’, in B. Matthews}


\section{Fine, B. (1986) (ed.) The Value Dimension: Marx versus Ricardo and Sraffa.}


\section{London: Routledge & Kegan Paul.}


\section{Orzech’, History of Political Economy 22(1), pp.149–55.}


\section{Fine, B. (1990b) The Coal Question: Political Economy and Industrial Change from}


\section{the Nineteenth Century to the Present Day. London: Routledge.}


\section{Fine, B. (1992b) Women’s Employment and the Capitalist Family. London:}


\section{Routledge.}


\section{Fine, B. (1998) Labour Market Theory: A Constructive Reassessment. London:}


\section{Routledge.}


\section{Fine, B. (2001) ‘The Continuing Imperative of Value Theory’, Capital & Class}


\section{Fine, B. (2002) The World of Consumption: The Material and Cultural Revisited}


\section{(2nd edn). London: Routledge.}


\section{Fine, B. (2003) ‘Contesting Labour Markets’, in A. Saad-Filho (ed.) Anti-Capi-}


\section{Journal of Political Economy 42(4), pp.47–66.}


\section{Economics as Social Theory. London: Routledge.}


\section{of Value: A Reply to Kincaid’, Historical Materialism 16(4), pp.167–80.}


\section{Once More’, Historical Materialism 17(3), pp.192–207.}


\section{Cheltenham: Edward Elgar.}


\section{and the World Economy’, Capital & Class 100, pp.69–83.}


\section{London: Routledge.}


\section{Fine, B., Lapavitsas, C. and Milonakis, D. (1999) ‘Addressing the World}


\section{Economy: Two Steps Back’, Capital & Class 67, pp.47–90.}


\section{Mass.: Harvard University Press.}


\section{Radical Political Economics 32(1), pp.1–39.}


\section{York: Monthly Review Press.}


\section{Revolution. New York: Little, Brown & Co.}


\section{Verso: London.}


\section{Press.}


\section{Left Books.}


\section{vols). London: Macmillan.}


\section{Crisis after 1973’, Review of Political Economy 2(3), pp.267–91.}


\section{Itoh, M. and Lapavitsas, C. (1999) Political Economy of Money and Finance.}


\section{London: Macmillan.}


\section{Jessop, B. (1982) The Capitalist State: Marxist Theories and Methods. Oxford:}


\section{Robertson.}


\section{Neoliberal Disorder. London: Pluto Press.}


\section{Kiely, R. (2005b) The Clash of Globalisations: Neo-liberalism, the Third Way and}


\section{Anti-globalisation. Leiden: Brill.}


\section{Saad-Filho on Value Theory’, Historical Materialism 15(4), pp.137–65.}


\section{Saad-Filho’, Historical Materialism 16(4), pp.181–203.}


\section{Kincaid, J. (2009) ‘The Logical Construction of Value Theory: More on Fine}


\section{and Saad-Filho’, Historical Materialism 17(3), pp.208–20.}


\section{Konings, M. and Panitch, L. (2008) ‘US Financial Power in Crisis’, Historical}


\section{Materialism 16(4), pp.3–34.}


\section{Lapavitsas, C. (2000b) ‘On Marx’s Analysis of Money Hoarding in the Turnover}


\section{of Capital’, Review of Political Economy 12(2), pp.219–35.}


\section{Lapavitsas, C. (2003b) Social Foundations of Markets, Money and Credit. London:}


\section{Routledge.}


\section{Lapavitsas, C. (2013) Profit without Producing: How Finance Exploits Us All.}


\section{London: Verso.}


\section{Lapides, K. (1998) Marx’s Wage Theory in Historical Perspective. Westport, Conn.:}


\section{Praeger.}


\section{Lebowitz, M. (2003a) Beyond Capital: Marx’s Political Economy of the Working}


\section{Class (2nd edn). London: Palgrave.}


\section{Historical Materialism 14(2), pp.29–47.}


\section{Historical Materialism 18(1), pp.131–49.}


\section{www.marxists.org/archive/lenin/works/1913/mar/x01.htm}


\section{vol.3. London: Lawrence & Wishart.}


\section{Levidow, L. (2003) ‘Technological Change as Class Struggle’, in A. Saad-Filho}


\section{Marot, J.E. (2012) The October Revolution in Prospect and Retrospect. Leiden:}


\section{Brill.}


\section{Marx, K. (1969, 1972, 1978a) Theories of Surplus Value (3 vols). London:}


\section{Lawrence & Wishart.}


\section{Marx, K. (1974) ‘Critique of the Gotha Programme’, in The First International}


\section{and After. Harmondsworth: Penguin.}


\section{Works, vol.29. London: Lawrence & Wishart.}


\section{Politics (CD-ROM). London: Electric Book Company.}


\section{London: Merlin Press.}


\section{McNally, D. (2011) Global Slump: The Economics and Politics of Crisis and}


\section{Resistance. Oakland, Calif.: PM Press.}


\section{Medio, A. (1977) ‘Neoclassicals, Neo-Ricardians, and Marx’, in J. Schwartz (ed.)}


\section{Moseley, F. (ed.) (1993) Marx’s Method in Capital: A Re examination. Atlantic}


\section{Highlands, N.J.: Humanities Press.}


\section{Kegan Paul.}


\section{Oakley, A. (1984, 1985) Marx’s Critique of Political Economy: Intellectual Sources}


\section{Okishio, N. (1961) ‘Technical Change and the Rate of Profit’, Kobe University}


\section{Economic Review 7, pp.85–99.}


\section{Okishio, N. (2000) ‘Competition and Production Prices’, Cambridge Journal of}


\section{Panitch, L. and Konings, M. (2008) (eds) American Empire and the Political}


\section{Economy of Global Finance. London: Palgrave.}


\section{Conn.: Praeger.}


\section{Perelman, M. (2000) Transcending the Economy: On the Potential of Passionate}


\section{Perelman, M. (2003) ‘The History of Capitalism’, in A. Saad-Filho (ed.)}


\section{Pilling, G. (1980) Marx’s Capital: Philosophy and Political Economy. London:}


\section{Routledge & Kegan Paul.}


\section{Postone, M. (1993) Time, Labour and Social Domination: A Re-examination of}


\section{Robertson, M. (2014) ‘The Financialisation of British Housing: A Systems of}


\section{Provision Approach’, Ph.D. thesis, University of London.}


\section{Wishart.}


\section{Capital’, Capital & Class 50, pp.127–46.}


\section{Saad-Filho, A. (1997a) ‘Concrete and Abstract Labour in Marx’s Theory of}


\section{Value’, Review of Political Economy 9(4), pp.457–77.}


\section{Saad-Filho, A. (1997b) ‘An Alternative Reading of the Transformation of Values}


\section{into Prices of Production’, Capital & Class 63, pp.115–36.}


\section{Research in Political Economy 19, pp.69–85.}


\section{Saad-Filho, A. (2002) The Value of Marx: Political Economy for Contemporary}


\section{Capitalism. London: Routledge.}


\section{Saad-Filho, A. (2003a) ‘Introduction’, in A. Saad-Filho (ed.) Anti-Capitalism: A}


\section{Marxist Introduction. London: Pluto Press.}


\section{Saad-Filho, A. (2003b) ‘Value, Capital and Exploitation’, in A. Saad-Filho (ed.)}


\section{Saad-Filho, A. (ed.) (2003c) Anti-Capitalism: A Marxist Introduction. London:}


\section{Pluto Press.}


\section{Saad-Filho, A. and Johnston, D. (2005) (eds) Neoliberalism: A Critical Reader.}


\section{London: Pluto Press.}


\section{Goodyear.}


\section{Controversy. London: Verso.}


\section{Shaikh, A. (1982) ‘Neo-Ricardian Economics: A Wealth of Algebra, a Poverty}


\section{of Theory’, Review of Radical Political Economics 14(2), pp.67–83.}


\section{N.J.: Humanities Press.}


\section{Modern’, New Left Review 146, pp.94–111.}


\section{Wajcman, J. (2002) ‘Addressing Technological Change: The Challenge to Social}


\section{Theory’, Current Sociology 50(3), pp.347–64.}


\section{Weeks, J. (1982a) ‘Equilibrium, Uneven Development and the Tendency of the}


\section{Rate of Profit to Fall’, Capital & Class 16, pp.62–77.}


\section{Weeks, J. (1982b) ‘A Note on Underconsumptionist Theory and the Labor}


\section{Theory of Value’, Science & Society 46(1), pp.60–76.}


\section{Weeks, J. (1983) ‘On the Issue of Capitalist Circulation and the Concepts}


\section{Appropriate to Its Analysis’, Science & Society 48(2), pp.214–25.}


\section{Weeks, J. (1985–6) ‘Epochs of Capitalism and the Progressiveness of Capital’s}


\section{Expansion’, Science & Society 49(4), pp.414–35.}


\section{Weeks, J. (1990) ‘Abstract Labor and Commodity Production’, Research in}


\section{Political Economy 12, pp.3–19.}


\section{Weeks, J. (2001) ‘The Expansion of Capital and Uneven Development on a}


\section{World Scale’, Capital & Class 74, pp.9–30.}


\section{Oxford: Oxford University Press.}


\section{Capitalism’, New Left Review 127, May–June, pp.66–95.}


\section{Wood, E.M. (1984) ‘Marxism and the Course of History’, New Left Review}


\section{Materialism. Cambridge: Cambridge University Press.}


\section{Wood, E.M. (1998) The Retreat from Class: A New ‘True’ Socialism. London:}


\section{Verso.}