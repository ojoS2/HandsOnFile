\chapter{Conservadorismo e Contrarrevolução}\label{Conservadorismo e Contrarrevolução}
 \par 
Quem quer que consiga monstros deve cuidar para que no processo não se torne um monstro.
 \par 
Quando John McCain anunciou Sarah Palin como sua companheira de chapa durante a campanha presidencial de 2008, vozes no movimento conservador expressaram surpresa, até mesmo choque. Não foi apenas o facto de McCain ter escolhido um novato político, um ingénuo e estranho às formas e meios de governação nos quarenta e oito estados mais baixos. Foi assim que ele a escolheu: com pouca ou nenhuma verificação e muita fé na superioridade da intuição e do impulso (dele e dela) sobre a razão e a reflexão. Além disso, parecia ser uma decisão nada condenável: impetuosa, mal pensada, imprudente.
 \par 
Esta não foi a primeira vez que um porta-estandarte do conservadorismo não conseguiu corresponder à autoimagem do conservador. Na Primavera de 2003, vários conservadores manifestaram preocupação com a audácia da decisão de George W. Bush de travar o que era essencialmente uma guerra de escolha. Também salientaram o pedigree liberal de uma das justificações da Guerra do Iraque: a difusão da democracia e dos direitos humanos. Aqui estava um líder conservador, mais uma vez parecia, agindo da maneira mais
 \par 
Este capítulo apareceu originalmente em Raritan 30, no. {\color{blue}1} (verão de 2010): 1–17.
 \par 
Não conservador em termos de conduta: abandonou o realismo de seu pai e de seu partido em favor de um internacionalismo há muito considerado propriedade exclusiva da esquerda, impulsionando a marcha da história contra o status quo do Oriente Médio.
 \par 
Desde que Edmund Burke inventou o conservadorismo como ideia, o conservador tem-se autodenominado um homem de prudência e moderação, e a sua causa é um reconhecimento sóbrio – e sério – dos limites. “Ser conservador”, ouvimos Michael Oakeshott declarar na introdução, “é preferir o familiar ao desconhecido. . . O tentado ao não experimentado, o fato ao mistério, o real ao possível, o limitado ao ilimitado, o próximo ao distante.” {\color{blue}1} No entanto, os esforços políticos que levaram o conservador às suas reacções mais profundas – as reacções contra as revoluções francesa e bolchevique; a defesa da escravidão e de Jim Crow; o ataque à social-democracia e ao Estado-providência; e as reações violentas em série contra o New Deal, a Grande Sociedade, os direitos civis, o feminismo e os direitos dos homossexuais – têm sido tudo menos isso. Seja na Europa ou nos Estados Unidos, neste século ou nos anteriores, o conservadorismo tem sido um movimento progressivo de mudança incansável e implacável, parcial à assunção de riscos e ao aventureirismo ideológico, militante na sua postura e populista nas suas orientações, amigável com os novatos e os insurgentes. , estrangeiros e recém-chegados. Embora o teórico conservador reivindique para a sua tradição o manto da prudência e da moderação, há uma tensão não tão subterrânea de imprudência e moderação que atravessa essa tradição – uma tensão que, por mais contra-intuitiva que pareça, liga Sarah Palin a Edmund Burke.
 \par 
Uma consideração desta vertente mais profunda do conservadorismo dá-nos uma noção mais clara do que é o conservadorismo. Embora o conservadorismo seja uma ideologia de reacção – originalmente contra a Revolução Francesa, mais recentemente contra os movimentos de libertação dos anos sessenta e setenta – essa reacção não foi bem compreendida. Longe de apresentar uma defesa instintiva de um velho regime imutável ou
 \par 
Tradicionalismo ponderado, o imperativo reacionário pressiona o conservadorismo em duas direções bastante diferentes: primeiro, para uma crítica e reconfiguração do antigo regime; e segundo, a uma absorção das ideias e tácticas da própria revolução ou reforma à qual se opõe. O que o conservadorismo procura realizar através dessa reconfiguração do antigo e da absorção do novo é tornar popular o privilégio, transformar um velho regime cambaleante num movimento de massas dinâmico e ideologicamente coerente. Um novo e velho regime, poder-se-ia dizer, que traz a energia e o dinamismo da rua às antigas desigualdades de uma propriedade dilapidada.
 \par 
À medida que o domínio de quarenta anos da direita começa a desaparecer, por mais adequado que seja, escritores como Sam Tanenhaus, Andrew Sullivan, Jeff Rey Hart, Sidney Blumenthal e John Dean afirmam que o conservadorismo entrou em declínio quando Palin, ou Bush, ou Reagan, ou Goldwater, ou Buckley, ou alguém tirou isso dos trilhos. Originalmente, prossegue o argumento, o conservadorismo era uma disciplina responsável das classes governantes, mas algures entre Joseph de Maistre e Joe, o Encanador, ele deixou-se levar por si mesmo. Tornou-se aventureiro, fanático, populista, ideológico. O que esta história de declínio ignora – quer emane da direita ou da esquerda – é que todos estes supostos vícios do conservadorismo contemporâneo estavam presentes no início, nos escritos de Burke e Maistre, só que não estavam presentes. vistos como vícios. Eles eram vistos como virtudes. O conservadorismo sempre foi um movimento mais selvagem e extravagante do que muitos imaginam – e é precisamente esta selvageria e extravagância que tem sido uma das fontes do seu apelo contínuo.
 \par 
Não é provocativo dizer que o conservadorismo surgiu em reacção à Revolução Francesa. A maioria dos conservadores com mentalidade histórica concordaria. {\color{blue}2} Mas se olharmos mais atentamente para duas vozes emblemáticas dessa reacção – Burke e Maistre – encontraremos vários elementos surpreendentes e raramente notados. A primeira é uma antipatia,
 \par 
Beirando o desprezo pelo antigo regime que reivindicam como causa. Os capítulos iniciais das Considerações sobre a França de Maistre são um ataque implacável aos três pilares do Antigo Regime: a aristocracia, a Igreja e a monarquia. Maistre divide a nobreza em duas categorias: os traiçoeiros e os sem noção. O clero é corrupto, enfraquecido pela sua riqueza e pela sua moral negligente. A monarquia é branda e não tem vontade de punir. Maistre descarta todos os três com uma frase de Racine: “Agora vejam os tristes frutos que suas falhas produziram, / Sintam os golpes que vocês mesmos induziram”.{\color{blue}3}
 \par 
No caso de Burke, a crítica é mais sutil, mas mais profunda. (Embora no final de sua vida ele falasse no mesmo tom modulado de Maistre.) {\color{blue}4} Isso ocorre durante seu relato, em Reflexões sobre a Revolução na França, sobre a tomada do palácio de Versalhes e a captura da família real. . Lá, Burke descreve Maria Antonieta como uma “visão encantadora. . . Brilhando como a estrela da manhã, cheia de vida, esplendor e alegria.” Burke considera a sua beleza um símbolo da beleza do antigo regime, onde os costumes e costumes feudais “tornaram o poder suave” e “por uma assimilação branda, incorporaram na política os sentimentos que embelezam e suavizam a sociedade privada”.{\color{blue}5}
 \par 
Desde que escreveu essas linhas, Burke tem sido ridicularizado por seu sentimentalismo. Mas os leitores do trabalho anterior de Burke sobre estética, Uma Investigação Filosófica sobre as Origens das Nossas Ideias do Sublime e do Belo, saberão que a beleza, para Burke, nunca é um sinal de vitalidade do poder; é sempre um sinal de decadência. A beleza desperta o prazer, que dá lugar à indiferença ou leva à dissolução total do eu. “A beleza atua”, escreve Burke, “relaxando os sólidos de todo o sistema”. {\color{blue}6} É este relaxamento e dissolução dos corpos – corpos físicos, sociais, políticos – que torna a beleza um símbolo e agente tão potente de degeneração e morte. “Nossas instituições mais salutares e mais belas não produzem nada além de poeira e sujeira.”{\color{blue}7}
 \par 
O que estas duas declarações iniciais de persuasão conservadora sugerem é que o maior inimigo do antigo regime é
 \par 
Nem o revolucionário nem o reformador; é o próprio antigo regime ou, para ser mais preciso, os defensores do antigo regime. {\color{blue}8} Falta-lhes simplesmente os meios ideológicos para pressionar a causa do antigo regime com o vigor, a clareza e o propósito necessários. Como Burke declarou sobre George Grenville, no contexto muito diferente da relação da Grã-Bretanha com as suas colónias americanas:
 \par 
Mas pode-se dizer com verdade que os homens muito versados ​​em assuntos externos raramente têm mentes de notável expansão. . . . As pessoas que são criadas fora do gelo se saem admiravelmente bem, desde que as coisas sigam sua ordem normal; mas quando as estradas são rompidas e as águas se esgotam, quando uma cena nova e conturbada é aberta, e o arquivo não oferece precedentes, então é que um maior conhecimento da humanidade e uma compreensão muito mais ampla das coisas são necessários. requisito, do que o gelo já deu, ou do que o gelo pode dar.{\color{blue}9}
 \par 
Mais tarde, os conservadores farão esta afirmação de várias maneiras. Por vezes acusarão os defensores do antigo regime de terem sido intimidados pelo desafio revolucionário ou reformista. De acordo com Thomas Dew, um dos primeiros e mais agressivos apologistas da escravatura americana, a rebelião de Nat Turner destruiu “todo o sentimento de segurança e confiança” entre a classe dominante. Eles ficaram tão assustados que “a razão foi quase banida da mente”. Não foi apenas a violência dos escravos que os assustou. Foi a acusação moral levantada pelos escravos e pelos abolicionistas, que de alguma forma se insinuou nas mentes dos proprietários de escravos e os deixou inseguros quanto à sua própria posição. “Nós próprios”, escreveu William Harper, outro defensor da escravatura, “até certo ponto nos declaramos culpados do impeachment”.{\color{blue}10}
 \par 
Mais de um século depois, Barry Goldwater abordaria o mesmo tema. O primeiro parágrafo de A Consciência de um Conservador não dirige o seu fogo contra os liberais ou os Democratas, nem mesmo contra os
 \par 
Estado de bem-estar; visa a timidez moral do que mais tarde será chamado de “Estabelecimento Republicano”.
 \par 
Tenho estado muito preocupado com o facto de tantas pessoas hoje com instintos conservadores se sentirem compelidas a pedir desculpa por eles. Ou, se não for para pedir desculpa diretamente, para qualificar o seu compromisso de uma forma que equivale a bater no peito. “Os candidatos republicanos”, disse o vice-presidente Nixon, “deveriam ser conservadores económicos, mas conservadores com coração”. O Presidente Eisenhower anunciou durante o seu primeiro mandato: “Sou conservador quando se trata de problemas económicos, mas liberal quando se trata de problemas humanos”. . . . Estas formulações equivalem a uma admissão de que o conservadorismo é uma teoria económica estreita e mecanicista que pode funcionar muito bem como um guia para o contabilista, mas não pode ser considerada uma filosofia política abrangente.{\color{blue}11}
 \par 
Mais frequentemente, os conservadores argumentam que o defensor do antigo regime é simplesmente obtuso. Ele se tornou preguiçoso, gordo e complacente, desfrutando tão plenamente dos privilégios de sua posição que não consegue ver a catástrofe que se aproxima. Ou, se o consegue ver, não pode fazer nada para se defender, uma vez que os seus músculos políticos se atrofiaram há muito tempo. John C. Calhoun era um desses conservadores e, ao longo da década de 1830, quando os abolicionistas começaram a pressionar a sua causa, ele ficou furioso com a vida fácil e a ignorância obstinada dos seus camaradas na plantação. A sua fúria atingiu o auge em 1837, quando, num discurso no plenário do Senado, instou o Congresso a não receber uma petição abolicionista – um momento, como vimos na introdução, que ele recordaria até ao dia da sua morte. “Tudo o que queremos é um concerto”, implorou ele aos seus colegas sulistas, para “nos unirmos com zelo e energia para repelir os perigos que se aproximam”. Mas, continuou ele, “não ouso esperar que qualquer coisa que eu possa dizer desperte o Sul para a devida sensação de perigo. Temo que esteja além do poder do
 \par 
Voz mortal para despertá-lo a tempo da segurança fatal em que caiu.”{\color{blue}12}
 \par 
Em seu influente ensaio, Oakeshott argumentou que o conservadorismo “não é um credo ou uma doutrina, mas uma disposição”. Especificamente, ele pensou, é uma disposição para aproveitar o presente. Não porque o presente seja melhor do que as alternativas ou mesmo porque seja bom em seus próprios termos. Isso implicaria um nível de reflexão consciente e escolha ideológica que Oakeshott acredita ser estranho ao conservador. Não, a razão pela qual o conservador aproveita o presente é simples e meramente porque ele é familiar, porque está lá, porque está à mão.{\color{blue}13}
 \par 
A visão de Oakeshott do conservador – e esta visão é amplamente partilhada tanto pela esquerda como pela direita – não é uma visão; é uma presunção. Ignora o facto de que o conservadorismo surge invariavelmente em resposta a uma ameaça ao antigo regime ou depois de o antigo regime ter sido destruído. (Oakeshott admite abertamente que a perda ou a ameaça de perda nos faz valorizar o presente, como argumentei na introdução, mas ele não permite que esse insight penetre ou desaloje sua compreensão geral do conservadorismo.) Oakeshott está descrevendo o antigo regime numa poltrona. , quando a sua mortalidade é uma noção distante e o tempo é um meio de aquecimento e não um solvente acre. Este é o antigo regime de Charles Loyseau, que escreveu quase dois séculos antes da Revolução Francesa que a nobreza não tem “começo” e, portanto, não tem fim. Ele “existe no tempo fora da mente”, sem consciência ou percepção da passagem da história.{\color{blue}14}
 \par 
O conservadorismo surge em cena precisamente quando – e precisamente porque – tais declarações já não podem ser feitas. Walter Berns, um dos muitos futuros neoconservadores em Cornell que ficaram traumatizados em 1969 pela tomada de Willard Straight Hall pelos estudantes negros, afirmou no seu discurso de despedida quando se demitiu da universidade: “Tínhamos um mundo demasiado bom; não poderia durar. {\color{blue}15} Nada perturba tanto o idílio da herança como a substituição súbita e muitas vezes brutal de um mundo por outro. Tendo testemunhado
 \par 
Com a morte daquilo que deveria viver para sempre, o conservador não pode mais considerar o tempo como o aliado natural ou habitat do poder. O tempo agora é o inimigo. A mudança, e não a permanência, é o governante universal, e a mudança não significa nem progresso nem melhoria, mas morte, e ainda por cima uma morte precoce e não natural. “O decreto da morte violenta”, diz Maistre, “está escrito nas próprias fronteiras da vida”. {\color{blue}16} O problema do defensor do antigo regime, diz o conservador, é que ele não conhece esta verdade ou, se conhece, não tem vontade de fazer algo a respeito.
 \par 
O segundo elemento que encontramos nessas primeiras vozes de reação é uma admiração surpreendente pela própria revolução contra a qual estão escrevendo. Os comentários mais arrebatadores de Maistre são reservados aos jacobinos, cuja vontade brutal e propensão à violência — sua “magia negra” — ele claramente inveja. Os revolucionários têm fé, em sua causa e em si mesmos, o que transforma um movimento de mediocridades na força mais implacável que a Europa já viu. Graças aos seus esforços, a França foi purificada e restaurada ao seu legítimo orgulho de lugar entre a família das nações. “O governo revolucionário”, conclui Maistre, “endureceu a alma da França ao temperá-la em sangue”.{\color{blue}17}
 \par 
Burke, novamente, é mais sutil, mas corta mais profundamente. O grande poder, sugere ele em O Sublime e o Belo, nunca deveria aspirar a ser – e nunca poderá realmente ser – bonito. O que um grande poder precisa é de sublimidade. O sublime é a sensação que experimentamos diante de extrema dor, perigo ou terror. É algo parecido com admiração, mas tingido de medo e pavor. Burke chama isso de “horror delicioso”. O grande poder deveria aspirar à sublimidade em vez da beleza, porque a sublimidade produz “a emoção mais forte que a mente é capaz de sentir”. É uma emoção cativante, mas revigorante, que tem o efeito simultâneo, mas contraditório, de nos diminuir e de nos engrandecer. Sentimo-nos aniquilados por um grande poder; no mesmo
 \par 
O tempo, nosso senso de identidade “incha” quando “estamos familiarizados com objetos terríveis”. O grande poder alcança a sublimidade quando é, entre outras coisas, obscuro e misterioso, e quando é extremo. “Em todas as coisas”, escreve Burke, o sublime “abomina a mediocridade”.{\color{blue}18}
 \par 
Nas Reflexões, Burke sugere que o problema em França é que o antigo regime é belo enquanto a revolução é sublime. O interesse fundiário, a pedra angular do antigo regime, é “lento, inerte e tímido”. Não pode defender-se “das invasões de capacidade”, sendo que a capacidade representa aqui os novos homens de poder que a revolução traz à tona. Noutra parte das Reflexões, Burke diz que o interesse monetário, que está aliado à revolução, é mais forte do que o interesse aristocrático porque está “mais pronto para qualquer aventura” e “mais disposto a novos empreendimentos de qualquer tipo”. O antigo regime, por outras palavras, é belo, estático e fraco; a revolução é feia, dinâmica e forte. E nos horrores que a revolução perpetra – a turba invadindo o quarto da rainha, arrastando-a seminua para a rua e levando-a e à sua família para Paris – a revolução atinge uma espécie de sublimidade: “Estamos alarmados verdadeira exão”, escreve Burke sobre as ações dos revolucionários. "Nossas mentes. . . São purificados pelo terror e pela piedade; nosso orgulho fraco e irrefletido é humilhado, sob as dispensações de uma sabedoria misteriosa.”{\color{blue}19}
 \par 
Para além destas simples declarações de inveja ou admiração, o conservador na verdade copia e aprende com a revolução a que se opõe. “Para destruir esse inimigo”, escreveu Burke sobre os jacobinos, “de uma forma ou de outra, a força que se opõe a ele deveria ter alguma analogia e semelhança com a força e o espírito que esse sistema exerce”. {\color{blue}20} Este é um dos aspectos mais interessantes e menos compreendidos da ideologia conservadora. Embora os conservadores sejam hostis aos objectivos da esquerda, particularmente ao empoderamento das castas e classes mais baixas da sociedade, são frequentemente os melhores alunos da esquerda. Às vezes, seus estudos são autoconscientes e estratégicos,
 \par 
À medida que olham para a esquerda em busca de formas de adaptar os novos vernáculos, ou os novos meios de comunicação, aos seus objectivos subitamente legitimados. Temendo que os philosophes tivessem assumido o controlo da opinião popular em França, os teólogos reaccionários de meados do século XVIII olharam para o exemplo dos seus inimigos. Eles pararam de escrever dissertações obscuras uns para os outros e começaram a produzir agitprop católico, que seria distribuído através das mesmas redes que levaram o esclarecimento ao povo francês. Eles gastaram vastas somas financiando concursos de redação, como aqueles em que Rousseau fez seu nome, para recompensar escritores que escreveram defesas acessíveis e populares da religião. Os tratados de fé anteriores, declarou Charles-Louis Richard, eram “inúteis para a multidão que, sem armas e sem defesas, sucumbe rapidamente à Filosofia”. Sua obra, ao contrário, foi escrita “com o objetivo de colocar nas mãos de todos aqueles que sabem ler uma arma vitoriosa contra os assaltos desta turbulenta Filosofia”.{\color{blue}21}
 \par 
Os pioneiros da Estratégia do Sul na administração Nixon, para citar um exemplo mais recente, compreenderam que depois das revoluções pelos direitos dos anos {\color{blue}60} já não podiam fazer simples apelos ao racismo branco. De agora em diante, eles teriam que falar em código, de preferência um código palatável para a nova dispensação dos daltônicos. Como observou o chefe de gabinete da Casa Branca, H. R. Haldeman, no seu diário, Nixon “enfatizou que é preciso encarar o facto de que todo o problema são realmente os negros. A chave é criar um sistema que reconheça isso, embora não pareça.” {\color{blue}22} Relembrando esta estratégia em 1981, o estratega republicano Lee Atwater expôs os seus elementos de forma mais clara:
 \par 
Você começa em 1954 dizendo: “Nigger, nigger, nigger”. Em 1968 você não pode dizer “negro” – isso machuca você. Backfi res. Então você diz coisas como ônibus forçado, direitos dos estados e todas essas coisas. Você está ficando tão abstrato agora que está falando em cortar impostos, e
 \par 
Todas essas coisas de que você está falando são coisas totalmente econômicas e um subproduto delas é que os negros são mais prejudicados do que os brancos. E subconscientemente talvez isso faça parte disso.{\color{blue}23}
 \par 
Mais recentemente ainda, David Horowitz encorajou os estudantes conservadores “a usarem a linguagem que a esquerda utilizou de forma tão eficaz em nome das suas próprias agendas. Professores radicais criaram um “ambiente de aprendizagem hostil” para estudantes conservadores. Há uma falta de “diversidade intelectual” nas faculdades e nas salas de aula acadêmicas. O ponto de vista conservador está “sub-representado” no currículo e nas suas listas de leitura. A universidade deve ser uma comunidade ‘inclusiva’ e intelectualmente ‘diversa’”.{\color{blue}24}
 \par 
Outras vezes, a educação do conservador é inconsciente, acontecendo, por assim dizer, pelas suas costas. Ao resistir e, assim, envolver-se com o argumento progressista dia após dia, ele acaba por ser influenciado, muitas vezes contra sua vontade, pelo próprio movimento ao qual se opõe. Ao tentar dobrar um vernáculo à sua vontade, ele descobre que sua vontade é dobrada pelo vernáculo. Atwater afirma que foi precisamente isso que ocorreu dentro do Partido Republicano; depois de sugerir “inconscientemente, talvez isso faça parte”. Ele adiciona:
 \par 
Eu não estou dizendo isso. Mas estou dizendo que se isso for tão abstrato e codificado, estaremos eliminando o problema racial de uma forma ou de outra. Você me segue - porque obviamente ficar sentado dizendo: “Queremos cortar isso” é muito mais abstrato do que até mesmo a coisa do ônibus, e muito mais abstrato do que “Nigger, nigger”.{\color{blue}25}
 \par 
Os republicanos aprenderam a disfarçar tão bem as suas intenções, argumenta Atwater, que o disfarce penetrou e transformou a intenção. Supondo que tal transformação tenha realmente ocorrido,
 \par 
Poderíamos muito bem perguntar se o conservador deixou de ser o que pretendia ser. Mas essa é uma questão para outro dia.
 \par 
Mesmo sem envolverem-se directamente no argumento progressista, os conservadores podem absorver, por alguma osmose indescritível, as categorias e expressões mais profundas da esquerda, mesmo quando essas expressões vão directamente contra a sua posição oficial. Depois de anos de oposição ao movimento das mulheres, por exemplo, Phyllis Schlafly parecia genuinamente incapaz de evocar a visão pré-feminista das mulheres como esposas e mães respeitosas. Em vez disso, ela celebrou o “poder da mulher positiva” ativista. E então, como se pegasse emprestada uma página de The Feminine Mystique, ela criticou a falta de sentido e de realização entre as mulheres americanas; só que ela culpou o feminismo por esses males, e não o sexismo. {\color{blue}26} Quando se manifestou contra a Emenda sobre a Igualdade de Direitos (ERA), não afirmou que esta introduzia uma nova linguagem radical de direitos. Seu argumento foi o oposto. A ERA, disse ela ao Washington Star, “é uma lição dos direitos das mulheres”. Irá “tirar o direito da esposa num casamento contínuo, da esposa no lar”. {\color{blue}27} Schlafly estava obviamente a utilizar a linguagem dos direitos de uma forma que se opunha aos objectivos do movimento feminista; ela estava usando o discurso sobre direitos para colocar as mulheres de volta em casa, para mantê-las como esposas e mães. Mas a questão é essa: o conservadorismo adapta e adopta, muitas vezes inconscientemente, a linguagem da reforma democrática à causa da hierarquia.
 \par 
Também se pode detectar uma certa franqueza sexual – até mesmo preocupação feminista – nas primeiras conversas da direita cristã que teria sido impensável antes do movimento das mulheres. Em 1976, Beverly e Tim LaHaye escreveram um livro, The Act of Marriage, que Susan Faludi chamou corretamente de “o equivalente evangélico de The Joy of Sex”. Lá, os LaHayes afirmaram que “as mulheres são passivas demais no ato sexual”. Deus, disseram os LaHayes às suas leitoras, “colocou [seu clitóris] lá para sua diversão”. Eles também reclamaram que “alguns maridos são heranças do
 \par 
Idade das Trevas, como aquele que disse à sua esposa frustrada: ‘Garotas legais não devem chegar ao clímax’.{\color{blue}28}
 \par 
O que o conservador aprende em última análise com os seus oponentes, intencionalmente ou inconscientemente, é o poder da agência política e a potência das massas. Do trauma da revolução, os conservadores aprendem que homens e mulheres, seja através de actos de força voluntários ou de algum outro exercício de acção humana, podem ordenar as relações sociais e o tempo político. Em cada movimento social ou momento revolucionário, os reformadores e radicais têm de inventar – ou redescobrir – a ideia de que a desigualdade e a hierarquia social não são fenómenos naturais, mas criações humanas. Se a hierarquia pode ser criada por homens e mulheres, pode ser descriada por homens e mulheres, e é isso que um movimento social ou uma revolução se propõe fazer. A partir destes esforços, os conservadores aprendem uma versão da mesma lição. Enquanto os seus antecessores no antigo regime pensavam na desigualdade como um fenómeno que ocorre naturalmente, uma herança transmitida de geração em geração, o encontro dos conservadores com a revolução ensina-lhes que, afinal de contas, os revolucionários tinham razão: a desigualdade é uma criação humana. E se pode ser incriado por homens e mulheres, pode ser recriado por homens e mulheres.
 \par 
“Cidadãos!” exclama Maistre no final de Considerações sobre a França. “É assim que as contra-revoluções são feitas.” {\color{blue}29} Sob o antigo regime, a monarquia – tal como o patriarcado ou Jim Crow – não é feita. Apenas isso. Seria difícil imaginar um Loyseau ou Bossuet declarando: “Homens” – muito menos cidadãos – “é assim que se faz uma monarquia”. Mas uma vez ameaçado ou derrubado o antigo regime, o conservador é forçado a perceber que é uma agência humana, a imposição voluntária do intelecto e da imaginação ao mundo, que gera e mantém a desigualdade ao longo do tempo. Saindo do seu confronto com a revolução, o conservador expressa o tipo de afirmação de agência política que se encontra neste editorial de 1957
 \par 
Da Revisão Nacional de William F. Buckley: “A questão central que emerge” do movimento pelos direitos civis “é se a comunidade branca no Sul tem o direito de tomar as medidas necessárias para prevalecer, política e culturalmente, em áreas em que não predomina numericamente? A resposta séria é sim – a comunidade branca tem esse direito porque, por enquanto, é a raça avançada.”{\color{blue}30}
 \par 
O revolucionário declara o Ano I e, em resposta, o conservador declara o Ano Negativo I. A partir da revolução, o conservador desenvolve uma atitude particular em relação ao tempo político, uma crença no poder dos homens e das mulheres para moldar a história, para impulsioná-la para frente ou para frente. para trás; e em virtude dessa crença, ele passa a adotar o futuro como seu tempo verbal preferido. Ronald Reagan e a destilação perfeita deste fenómeno quando invocou, repetidamente, a máxima de Thomas Paine de que “temos em nosso poder começar o mundo de novo”. {\color{blue}31} Mesmo quando o conservador afirma estar a preservar um presente que está ameaçado ou a recuperar um passado que está perdido, ele é impelido pelo seu próprio activismo e agência a confessar que está a fazer um novo começo e a criar o futuro.
 \par 
Burke estava especialmente sintonizado com este problema e por isso muitas vezes se esforçava para lembrar aos seus camaradas na batalha contra a Revolução que tudo o que fosse reconstruído na França após a restauração inevitavelmente, como ele disse numa carta a um emigrado, “estaria em algum lugar”. medir uma coisa nova.” {\color{blue}32} Outros conservadores têm sido menos ambivalentes, afirmando alegremente as virtudes da criatividade política e da originalidade moral. Alexander Stephens, vice-presidente da Confederação dos EUA, declarou orgulhosamente que “nosso novo governo é o primeiro, na história do mundo” a ser fundado na “grande verdade física, filosófica e moral” de que “o negro não é igual para o homem branco; que a escravidão – subordinação à raça superior – é sua condição natural e normal”. {\color{blue}33} Barry Goldwater disse simplesmente: “O nosso futuro, tal como o nosso passado, será aquilo que fizermos dele”.{\color{blue}34}
 \par 
A partir das revoluções, os conservadores também desenvolvem um gosto e talento pelas massas, mobilizando as ruas para exibições espetaculares de poder, ao mesmo tempo em que garantem que o poder nunca seja verdadeiramente compartilhado ou redistribuído. Essa é a tarefa do populismo de direita: apelar para a massa sem interromper o poder das elites ou, mais precisamente, aproveitar a energia da massa para reforçar ou restaurar o poder das elites. Longe de ser uma inovação recente da direita cristã ou do movimento Tea Party, o populismo reacionário corre como um fio vermelho por todo o discurso conservador desde o início.
 \par 
Maistre foi um pioneiro no teatro do poder de massa, imaginando cenas e encenando dramas em que os mais baixos dos mais baixos podiam ver-se refletidos nos mais altos dos mais altos. “A monarquia”, escreve ele, “é, sem contradição, a forma de governo que confere maior distinção ao maior número de pessoas”. As pessoas comuns “partilham” do seu “brilho” e brilho, embora não, Maistre tem o cuidado de acrescentar, nas suas decisões e deliberações: “o homem é honrado não como um agente, mas como uma porção da soberania”. {\color{blue}35} Arquimonarquista como era, Maistre compreendeu que o rei nunca poderia retornar ao poder se não tivesse um toque de plebeu. Assim, quando Maistre imagina o triunfo da contrarrevolução, tem o cuidado de enfatizar as credenciais populistas do monarca que regressa. O povo deveria identificar-se com este novo rei, diz Maistre, porque, tal como eles, ele frequentou a “terrível escola do infortúnio” e sofreu na “dura escola da adversidade”. Ele é “humano”, com humanidade aqui conotando uma capacidade de erro quase pedestre e reconfortante. Ele será como eles. Ao contrário dos seus antecessores, ele saberá disso, o que “é muito”.{\color{blue}36}
 \par 
Mas para apreciar completamente a inventividade do populismo de direita, temos que nos voltar para a classe dominante do Velho Sul. O senhor de escravos criou uma forma quintessencial de feudalismo democrático,
 \par 
Transformando a maioria branca em uma classe senhorial, compartilhando os privilégios e prerrogativas de governar a classe escrava. Embora os membros dessa classe dominante soubessem que não eram iguais uns aos outros, eles eram compensados ​​pela ilusão de superioridade — e a realidade do governo — sobre a população negra abaixo deles.
 \par 
Uma escola de pensamento – chamemos-lhe escola de oportunidades iguais – localizou a promessa democrática da escravatura no facto de colocar a possibilidade de domínio pessoal ao alcance de todos os homens brancos. A genialidade dos proprietários de escravos, escreveu Daniel Hundley em seu livro Social Relations in Our Southern States, é que eles “não são uma aristocracia exclusiva. Todo homem branco livre em toda a União tem o mesmo direito de se tornar um oligarca.” Isto não era apenas propaganda: em 1860, havia {\color{blue}400} mil proprietários de escravos no Sul, tornando a classe dominante americana uma das mais democráticas do mundo. Os proprietários de escravos tentaram repetidamente aprovar leis que encorajassem os brancos a possuir pelo menos um escravo e até consideraram conceder incentivos fiscais para facilitar essa propriedade. O pensamento deles, nas palavras de um agricultor do Tennessee, era que “no minuto em que se tira do poder dos agricultores comuns comprar um homem ou uma mulher negra. . . Você faz dele um abolicionista imediatamente.”{\color{blue}37}
 \par 
Essa escola de pensamento lutou com uma segunda escola, possivelmente mais influente. A escravidão americana não era democrática, de acordo com esta linha de pensamento, porque representava uma oportunidade de domínio pessoal para os homens brancos: a escravidão americana era democrática porque tornava cada homem branco, proprietário de escravos ou não, um membro da classe dominante em virtude de a cor de sua pele. Nas palavras de Calhoun: “Conosco, as duas grandes divisões da sociedade não são os ricos e os pobres, mas os brancos e os negros; e todos os primeiros, tanto os pobres como os ricos, pertencem à classe alta e são respeitados e tratados como iguais.” {\color{blue}38} Ou, como disse o seu colega júnior James Henry Hammond: “Num país escravista, todo homem livre é um aristocrata”. {\color{blue}39} Mesmo sem escravos ou os pré-requisitos materiais para
 \par 
Liberdade, um homem branco pobre poderia se autodenominar membro da nobreza e, portanto, ser confiável para tomar as medidas necessárias em sua defesa.
 \par 
Quer se subscrevesse a primeira ou a segunda escola de pensamento, a classe dominante acreditava que o feudalismo democrático era um potente contra-ataque aos movimentos igualitários que então agitavam a Europa e a América jacksoniana. Os radicais europeus, declarou Dew, “desejam que toda a humanidade seja levada a um nível comum. Acreditamos que a escravidão, nos Estados Unidos, conseguiu isso.” Ao libertar os brancos de “regiões servis e inferiores”, a escravatura eliminou “a maior causa de distinção e separação das classes da sociedade”. {\color{blue}40} Enquanto as classes dominantes do século XIX enfrentavam desafio após desafio ao seu poder, a classe dominante aumentava a dominação racial como forma de aproveitar a energia das massas brancas, em apoio, e não em oposição, aos privilégios e poderes das elites estabelecidas. Este programa encontraria o seu cumprimento final um século mais tarde e a um continente de distância.
 \par 
Estas correntes populistas podem ajudar-nos a compreender um elemento final do conservadorismo. Desde o início, o conservadorismo apelou e contou com pessoas de fora. Maistre era da Sabóia, Burke da Irlanda. Alexander Hamilton nasceu fora do casamento em Nevis e, segundo rumores, era parte negro. Disraeli era judeu, tal como muitos dos neoconservadores que ajudaram a transformar o Partido Republicano de um cocktail em Darien no partido de Scalia, d’Souza, Gonzalez e Yoo. (Foi Irving Kristol quem primeiro identificou “a tarefa histórica e o propósito político do neoconservadorismo” como a conversão “do Partido Republicano, e do conservadorismo americano em geral, contra as suas respectivas vontades, num novo tipo de política conservadora adequada para governar uma economia moderna”. democracia.”) {\color{blue}41} Allan Bloom era judeu e homossexual. E como ela nunca se cansou de nos lembrar durante a campanha de 2008, Sarah Palin é uma mulher num mundo de
 \par 
Homens, uma do Alasca que disse não a Washington (embora ela realmente não o tenha feito), um dissidente que atacou outro dissidente.
 \par 
O conservadorismo não dependeu apenas de pessoas de fora; também se viu como a voz do estranho. Desde o grito de Burke de que “a galeria está no lugar da casa” até à queixa de Buckley de que o conservador moderno está “fora do lugar”, o conservador tem servido de tribuna para os deslocados, sendo o seu movimento uma transmissão das suas queixas. {\color{blue}42} Longe de ser uma invenção do politicamente correto, a vitimização tem sido um tema de discussão da direita desde que Burke condenou o tratamento dado pela multidão a Maria Antonieta. O conservador, sem dúvida, fala em nome de um tipo especial de vítima: aquela que perdeu algo de valor, em oposição aos miseráveis ​​da terra, cuja principal queixa é que nunca tiveram nada a perder. O seu eleitorado é constituído pelos contingentemente despossuídos – o “homem esquecido” de William Graham Sumner – e não pelos sobrenaturalmente oprimidos. Longe de diminuir o seu apelo, este tipo de vitimização confere à queixa conservadora um significado mais universal. Liga a sua deserdação a uma experiência que todos partilhamos – nomeadamente, a perda – e entrelaça os fios dessa experiência numa ideologia que promete que essa perda, ou pelo menos uma parte dela, pode ser reparada.
 \par 
As pessoas de esquerda muitas vezes não conseguem perceber isto, mas o conservadorismo realmente fala para e para as pessoas que perderam alguma coisa. Pode ser uma propriedade fundiária ou os privilégios da pele branca, a autoridade inquestionável de um marido ou os direitos irrestritos de um proprietário de fábrica. A perda pode ser tão material quanto o dinheiro ou tão etérea quanto a sensação de posição. Pode ser a perda de algo que nunca foi legitimamente possuído; pode, quando comparado com o que o conservador retém, ser pequeno. Mesmo assim, é uma perda, e nada é tão valorizado como aquilo que já não possuímos. Costumava ser uma das grandes virtudes da esquerda o fato de ser a única a compreender a natureza muitas vezes de soma zero da política, onde os ganhos de um
 \par 
A classe implica necessariamente as perdas do outro. Mas à medida que esse sentimento de conflito diminui na esquerda, cabe à direita lembrar aos eleitores que existem realmente perdedores na política e que são eles – e só eles – que falam por eles. “Todo o conservadorismo começa com a perda”, observa Andrew Sullivan, com razão, o que faz do conservadorismo não o Partido da Ordem, como Mill e outros afirmaram, mas o partido do perdedor.{\color{blue}43}
 \par 
O principal objectivo do perdedor não é – e na verdade não pode ser – a preservação ou a protecção. É recuperação e restauração. Acredito que esse seja um dos segredos do sucesso do conservadorismo. Apesar de todo o seu frisson demótico e grandiosidade ideológica, apesar de toda a sua insistência no triunfo e na vontade, no movimento e na mobilização, o conservadorismo pode ser, em última análise, um assunto pedestre. Dado que as suas perdas são recentes – a direita agita contra a reforma em tempo real, e não milénios depois do facto – o conservador pode afirmar de forma credível perante o seu eleitorado, na verdade, perante o sistema político em geral, que os seus objectivos são práticos e alcançáveis. Ele apenas procura recuperar o que é seu, e o fato de que já o teve — na verdade, provavelmente já o teve há algum tempo — sugere que ele é capaz de possuí-lo novamente. “Não é uma estrutura antiga”, declarou Burke sobre a França jacobina, mas “um erro recente”. {\color{blue}44} Enquanto o programa de redistribuição da esquerda levanta a questão de saber se os seus beneficiários estão realmente preparados para exercer os poderes que procuram, o projecto conservador de restauração não sofre tal desafio. Além disso, ao contrário do reformador ou do revolucionário, que enfrenta a tarefa quase impossível de dar poder aos que não têm poder – isto é, de transformar as pessoas daquilo que são naquilo que não são – o conservador limita-se a pedir aos seus seguidores que façam mais daquilo que sempre fizeram. feito (embora melhor e de forma diferente). Como resultado, a sua contrarrevolução não exigirá a mesma perturbação que a revolução causou no país. “Quatro ou cinco pessoas, talvez”, escreve Maistre, “darão um rei à França”.{\color{blue}45}
 \par 
Para alguns, talvez muitos, no movimento conservador, este conhecimento é uma fonte de alívio: o seu sacrifício será pequeno, a sua recompensa grande. Para outros, é uma fonte de amarga decepção. Para este subconjunto de activistas e militantes, a batalha é toda. Saber que tudo acabará em breve e não exigirá tanto deles é suficiente para desencadear um complexo de desespero: repulsa pela mesquinhez do seu esforço, tristeza pelo desaparecimento do seu inimigo, ansiedade pela reforma antecipada em que se encontram. foi forçado. Como reclamou Irving Kristol após o fim da Guerra Fria, a derrota da União Soviética e da esquerda em geral “privou” conservadores como ele “de um inimigo” e “na política, ser privado de um inimigo é um assunto muito sério”. . Você tende a ficar relaxado e desanimado. Volte-se para dentro.” {\color{blue}46} A depressão assombra o conservadorismo tão certamente como a grande riqueza. Mas, mais uma vez, longe de diminuir o apelo do conservadorismo, esta dimensão mais sombria apenas o aumenta. No palco, o conservador enfeita Byronic, examinando melancolicamente a soma de suas perdas diante de uma plateia de apaixonados e fascinados. Fora do palco e fora da vista, os seus gestores compilam silenciosamente a soma dos seus ganhos.
 \par 
A revolução enviou Thomas Hobbes para o exílio; a contrarrevolução o enviou de volta. Em 1640, oponentes parlamentares de Carlos I, como John Pym, estavam denunciando qualquer um que “pregasse a favor da monarquia absoluta para que o rei pudesse fazer o que quisesse”. Hobbes havia terminado recentemente de escrever The Elements of Law, que fez exatamente isso. Depois que o principal conselheiro do rei e um teólogo que defendia o poder real ilimitado foram presos, Hobbes decidiu que era hora de ir. Sem esperar que suas malas fossem feitas, ele fugiu da Inglaterra para a França.{\color{blue}1}
 \par 
Onze anos e uma guerra civil depois, Hobbes fugiu da França para a Inglaterra. Desta vez, ele estava fugindo dos monarquistas. Como antes, Hobbes acabara de terminar um livro. O Leviatã, explicaria mais tarde, “luta em nome de todos os reis e de todos aqueles que, sob qualquer nome, detêm os direitos dos reis”. {\color{blue}2} Era a segunda metade dessa afirmação, com a sua aparente indiferença relativamente à identidade do soberano, que agora o colocava em apuros. O Leviatã justificou, não, exigiu, que os homens se submetessem a qualquer pessoa ou pessoas capazes de protegê-los de ataques estrangeiros e agitação civil. Com a monarquia abolida e as forças de Oliver Cromwell no controle de