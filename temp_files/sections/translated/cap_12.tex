\chapter{Fácil de ser difícil}\label{Fácil de ser difícil}
 \par 
Eu gosto de guerras. Qualquer aventura é melhor do que ficar sentado em um escritório.
 \par 
Apesar do apoio entre eleitores e políticos autoidentificados como conservadores à pena de morte, tortura e guerra, os intelectuais da direita frequentemente negam qualquer afinidade entre conservadorismo e violência. {\color{blue}1} “Conservadores”, escreve Andrew Sullivan, “odeiam a guerra”.
 \par 
A sua política interna está enraizada numa aversão às guerras civis e à violência, e eles sabem que a liberdade é sempre a primeira vítima da guerra internacional. Quando os países entram em guerra, os seus governos tornam-se invariavelmente maiores e mais fortes, as liberdades individuais são reduzidas e as sociedades que outrora desfrutaram da cacofonia pluralista da liberdade têm de ser reunidas numa nota única e colectiva para enfrentar um inimigo externo. Um estado de guerra permanente – como viu George Orwell – é um convite virtual à tirania interna.{\color{blue}2}
 \par 
Este capítulo apareceu originalmente como “Fácil de ser difícil: conservadorismo e violência”, em Performances of Violence, ed. Austin Sarat, Carleen Basler e Thomas L. Dumm (Amherst: University of Massachusetts Press, 2011), 18–42.
 \par 
Canalizando uma tradição de cepticismo de Oakeshott a Hume, o conservador identifica o governo limitado como a extensão da sua fé, e o Estado de direito como o seu único requisito para a busca da felicidade. Pragmática e adaptativa, mais disposta do que comprometida, tal sensibilidade – e é uma sensibilidade, insiste o conservador, não uma ideologia – não está interessada na violência. O seu endosso à guerra, tal como é, é a mais cansativa das concessões à realidade. Ao contrário dos seus amigos de esquerda – por mais conservador que seja, ele valoriza mais a amizade do que o acordo – ele sabe que vivemos e amamos no meio de um grande mal. Este mal deve ser combatido, por vezes por meios violentos. Se todas as coisas fossem iguais, ele gostaria de ver um mundo sem violência. Mas nem todas as coisas são iguais e ele não pretende ver o mundo como gostaria que fosse.
 \par 
O registo histórico do conservadorismo – não apenas como prática política, que não é a minha principal preocupação aqui, mas como tradição teórica – sugere o contrário. Longe de ficar entristecido, oprimido ou irritado pela violência, o conservador foi animado por ela. Não me refiro no sentido pessoal, embora muitos conservadores, como Harold Macmillan citado acima ou Winston Churchill citado abaixo, tenham expressado um entusiasmo inesperado pela violência. Minha preocupação é com ideias e argumentos, e não com caráter ou psicologia. A violência, afirmou o intelectual conservador, é uma das experiências da vida que nos faz sentir mais vivos, e a violência é uma atividade que torna a vida, bem, viva. {\color{blue}3} Tais argumentos podem ser apresentados de forma ágil – “Só os mortos viram o fim da guerra”, como disse certa vez Douglas MacArthur {\color{blue}4} – ou laboriosamente, como no caso de Treitschke:
 \par 
Para o historiador que vive no mundo da vontade, torna-se imediatamente claro que a exigência de uma paz perpétua é totalmente reacionária; ele vê que com a guerra todo movimento, todo crescimento deve ser eliminado da história. Sempre foi o cansado,
 \par 
Períodos pouco inteligentes e enervados que brincaram com o sonho da paz perpétua. . . . Contudo, não vale a pena discutir mais este assunto; o Deus vivo fará com que a guerra retorne constantemente como um remédio terrível para a raça humana.{\color{blue}5}
 \par 
Enérgico ou prolixo, o caso se resume a isto: guerra é vida, paz é morte.
 \par 
Essa crença pode ser rastreada até A Philosophical Inquiry into the Origin of Our Ideas of the Sublime and the Beautiful, de Edmund Burke. Lá, Burke desenvolve uma visão do eu que precisa desesperadamente de estímulos negativos do tipo fornecido pela dor e pelo perigo, que Burke associa ao sublime. O sublime é mais facilmente encontrado em duas formas políticas: hierarquia e violência. Mas, por razões que ficarão claras, o conservador – mais uma vez, consistente com os argumentos de Burke – favorece frequentemente o último em detrimento do primeiro. O governo pode ser sublime, mas a violência é mais sublime. O mais sublime de tudo é quando os dois se fundem, quando a violência é praticada com o objetivo de criar, defender ou recuperar um regime de dominação e governo. Mas, como alertou Burke, é sempre melhor aproveitar a dor e o perigo à distância. A distância e a obscuridade aumentam a sublimidade; a proximidade e a iluminação diminuem-no. A violência contra-revolucionária pode ser o Everest da experiência conservadora, mas deve-se vê-la de longe. Chegue muito perto do topo da montanha e o ar ficará rarefeito e a vista ficará turva. No final de cada discurso sobre a violência, portanto, existe uma decepção que aguarda.
 \par 
O Sublime e o Belo começa com uma nota alta, com uma discussão sobre a curiosidade, que Burke identifica como “a primeira e mais simples emoção”. A curiosa corrida “de um lugar para caçar algo novo”. Sua visão está fixa, sua atenção está extasiada. Então o mundo fica cinza. Eles começam a tropeçar nas mesmas coisas, “com menos
 \par 
E menos efeito agradável.” A novidade diminui: quanto, realmente, há de novo no mundo? A curiosidade “se esgota”. O entusiasmo e o envolvimento dão lugar à “aversão e ao cansaço”. {\color{blue}6} Burke passa para o prazer e a dor, que deveriam transformar a busca pela novidade em experiências mais sustentadas e profundas. Mas, em vez de ser um verdadeiro aditivo à curiosidade, o prazer oferece mais do mesmo: um momento de entusiasmo, seguido de um enfadonho mal-estar. “Quando termina a sua carreira”, diz Burke, o prazer “nos coloca quase onde nos encontrou”. Qualquer tipo de prazer “satisfaz rapidamente; e quando acaba, recaímos na indiferença.” {\color{blue}7} Prazeres mais tranquilos, menos intensos que o prazer, são igualmente individuais c. Geram complacência; nós “nos entregamos à indolência e à inação”. {\color{blue}8} Burke recorre à imitação como outra força potencial de propulsão externa. Através da imitação, aprendemos boas maneiras e costumes, desenvolvemos opiniões e somos civilizados. Nós nos trazemos ao mundo, e o mundo é trazido até nós. Mas a imitação contém o seu próprio narcótico. Imitar demais os outros e deixaremos de nos melhorar. Seguimos a pessoa à nossa frente “e assim por diante em um círculo eterno”. Num mundo de imitadores, “nunca poderia haver qualquer melhoria”. Tais “homens devem permanecer como os brutos, os mesmos no final que são hoje e que eram no começo do mundo”.{\color{blue}9}
 \par 
A curiosidade leva ao cansaço, o prazer à indiferença, o prazer ao torpor e a imitação à estagnação. Tantas portas da psique se abrem para este espaço de escuridão inercial que poderíamos muito bem concluir que ele se esconde não nos limites, mas no centro da condição humana. Aqui, neste pátio escuro do eu, toda ação cessa, criando um ambiente ideal para “melancolia, desânimo, desespero e autoassassinato”. {\color{blue}10} Até o amor, o mais exterior dos êxtases, leva o eu de volta a um estado de dissolução interna. {\color{blue}11} O suicídio, ao que parece, é o destino inevitável que aguarda qualquer pessoa que tenha prazer no mundo tal como ele é.
 \par 
Para um certo tipo de teórico conservador, passagens como estas representam um certo desafio. Aqui está o inventor da tradição conservadora articulando uma visão do eu dramaticamente em desacordo com o eu imaginado do pensamento conservador. O eu conservador, como vimos repetidamente, afirma preferir “o familiar ao desconhecido. . . O tentado ao não experimentado, o fato ao mistério, o real ao possível, o limitado ao ilimitado, o próximo ao distante, o suficiente ao superabundante, o conveniente ao perfeito, o riso presente à felicidade utópica.” {\color{blue}12} Ele gosta das coisas como elas são, não porque as considere justas ou boas, mas porque as considera familiares. Ele os conhece e está apegado a eles. Ele não deseja perdê-los nem que sejam levados embora. Desfrutar do que tem, em vez de adquirir algo melhor, é o seu bem maior. Mas se o eu do Sublime e do Belo tivesse certeza de seus apegos e familiares, ele rapidamente se veria confrontando o espectro de sua própria extinção, mais do que provavelmente por suas próprias mãos.
 \par 
Talvez seja esse tédio letal, espreitando logo abaixo da superfície do discurso conservador, que explique o fracasso do político conservador em seguir a liderança do teórico conservador. Longe de abraçar a causa de prazeres tranquilos e apegos seguros, o político conservador tem consistentemente optado por um ativismo do ainda não e do que será. O primeiro discurso de posse de Ronald Reagan foi um hino ao poder dos sonhos: não sonhos pequenos, mas grandes sonhos heróicos, de progresso e melhoria, e não sonhos por si só, mas sonhos como um estímulo necessário e vital para a ação. Três meses depois, em um discurso perante o Congresso, Reagan enfatizou o ponto com uma citação de Carl Sandburg: "Nada acontece a menos que primeiro um sonho". E nada acontecendo, ou poucas coisas acontecendo, ou coisas que não acontecem rápido o suficiente, é o que o conservador na política não gosta. Reagan mal conseguia conter sua impaciência com a hesitação dos políticos:
 \par 
“O jeito antigo e confortável é raspar um pouco aqui e acrescentar um pouco ali. Bem, isso não é mais aceitável.” Antiga e confortável era a acusação, sem “meias medidas” o veredicto.{\color{blue}13}
 \par 
Reagan não foi o primeiro conservador a agir em prol do invisível e do ideal, em oposição ao material e ao real. No seu discurso de aceitação da Convenção Nacional Republicana de 1964, Barry Goldwater não conseguiu encontrar nenhuma acusação mais poderosa contra o Estado-Providência do que a de que este tinha feito uma grande nação “calmar”. Graças ao New Deal, os Estados Unidos tinham perdido o seu “ritmo acelerado” e estavam agora a “avançar lentamente”. Calma, lentidão e trabalho árduo são geralmente bem recebidos pelo teórico conservador como sinais de felicidade presente. Mas para o político conservador, são males. Ele deve declarar guerra, reunindo os seus exércitos contra os apáticos e lânguidos com conversas sobre “causas”, “luta”, “entusiasmo” e “devoção”.{\color{blue}14}
 \par 
Esse zelo cruzado não é peculiar ao conservadorismo americano. Também se encontra na Europa, até mesmo na Inglaterra, o país que fez da moderação o apelido de conservadorismo. “Quem ganhou uma batalha”, zombou Margaret Thatcher, “sob a bandeira ‘Eu defendo o Consenso’?” {\color{blue}15} E depois há Winston Churchill, que viajou para Cuba em 1895 para fazer uma reportagem sobre a guerra espanhola contra a independência cubana. {\color{blue}16} Ruminando sobre as decepções de sua geração – retardatários que chegaram ao Império, foram privados da oportunidade de conquista imperial (em oposição à administração) – ele chegou a Havana. Isto é o que ele tinha a dizer (relembrando a experiência de 1930):
 \par 
As mentes desta geração, exaustas, brutalizadas, mutiladas e entediadas pela guerra, podem não compreender as sensações deliciosas, mas trêmulas, com que um jovem oficial britânico, criado na longa paz, aproximou-se pela primeira vez de um verdadeiro teatro de operações. Quando pela primeira vez, na penumbra da manhã, vi as costas de Cuba erguerem-se e definirem-se
 \par 
Dos horizontes azul-escuros, senti como se tivesse navegado com Long John Silver e contemplado pela primeira vez a Ilha do Tesouro. Aqui estava um lugar onde coisas reais estavam acontecendo. Aqui estava uma cena de ação vital. Aqui era um lugar onde tudo poderia acontecer. Aqui estava um lugar onde algo certamente aconteceria. Aqui eu poderia deixar meus ossos.{\color{blue}17}
 \par 
Qualquer que seja a relação entre teoria e prática na tradição conservadora, fica claro em O Sublime e o Belo que, para que o eu sobreviva e floresça, ele deve ser despertado por uma experiência mais vital e revigorante do que o prazer ou o prazer. Prazer e prazer agem como beleza, “relaxando os sólidos de todo o sistema”. {\color{blue}18} Esse sistema, no entanto, deve ser tornado rígido e tenso. A mente deve ser acelerada, o corpo exercitado. Caso contrário, o sistema irá amolecer e atrofiar e, por fim, morrer.
 \par 
O que mais desperta esse estado elevado de ser é o confronto com o não-ser. A vida e a saúde são prazerosas e divertidas, e é isso que há de errado com elas: “elas não causam tal impressão” em si mesmo porque “não fomos feitos para concordar com a vida e a saúde”. A dor e o perigo, pelo contrário, são “emissários” da morte, o “rei dos terrores”. Eles são fontes da sublime, “a mais forte” – mais poderosa, mais comovente – “emoção que a mente é capaz de sentir”. {\color{blue}19} A dor e o perigo, por outras palavras, são experiências geradoras do self.
 \par 
A dor e o perigo são geradores porque têm o efeito contraditório de minimizar e maximizar o nosso sentido de identidade. Ao sentir dor ou perigo, nossa mente “está tão completamente preenchida com seu objeto que não consegue entreter nenhum outro”. Os “movimentos” da nossa alma “ficam suspensos”, à medida que o dano e os medos que ela desperta “invadem a mente”. Diante desses medos, “a mente sai apressada de si mesma”. Quando experimentamos o sublime, sentimo-nos evacuados, oprimidos por um objeto externo de tremendo poder
 \par 
E ameaça. Tudo o que nos deu uma sensação de ser interno e de vitalidade deixa de existir. O externo é tudo, não somos nada. Deus é um bom exemplo e a expressão máxima do sublime: “Enquanto contemplamos um objeto tão vasto, debaixo do braço, por assim dizer, do poder onipotente, e investido por todos os lados com onipresença, nos encolhemos na pequenez de nossa própria natureza, e somos, de certa forma, aniquilados diante dele”.{\color{blue}20}
 \par 
Paradoxalmente, também sentimos a nossa existência de uma forma que nunca sentimos antes. Tomadas pelo terror, a nossa “atenção” é despertada e as nossas “faculdades” são “impulsionadas, por assim dizer, em guarda”. Somos arrancados de nós mesmos. Estamos cientes do terreno imediato e da nossa presença nele. Antes, mal percebíamos a nós mesmos ou ao nosso redor. Agora saímos de nós mesmos, habitando não apenas nossos corpos e mentes, mas também o espaço circundante. Sentimos “uma espécie de inchaço” – uma sensação de que somos maiores, que o nosso perímetro se estende ainda mais – que “é extremamente grato à mente humana”. Mas esse “inchaço”, lembra-nos Burke, “nunca é mais percebido, nem opera com mais força, do que quando, sem perigo, estamos familiarizados com objetos terríveis”.{\color{blue}21}
 \par 
Diante do sublime, o eu é aniquilado, ocupado, esmagado, oprimido; diante do sublime, o eu é elevado, engrandecido, ampliado. Se o self pode realmente ocupar esses pólos de experiência opostos, quase irreconciliáveis, ao mesmo tempo - é esta contradição, a oscilação entre extremos selvagens, que gera um sentimento forte e extenuante de self. Como Burke escreve em outro lugar, a luz intensa se assemelha à escuridão intensa não apenas porque cega os olhos e, portanto, se aproxima da escuridão, mas também porque ambas são extremos. E os extremos, especialmente os extremos opostos, são sublimes porque a sublimidade “em todas as coisas abomina a mediocridade”. {\color{blue}22} O extremo das sensações opostas, o balanço selvagem do ser ao nada, contribui para a experiência mais intensa da individualidade.
 \par 
A questão para nós, que Burke não coloca nem responde, nem aqui nem no seu outro trabalho, é: que tipo de forma política implica esta simultaneidade – ou oscilação entre – auto-engrandecimento e auto-aniquilação? Uma possibilidade seria a hierarquia, com os seus requisitos gémeos de submissão e dominação; a outra é a violência, particularmente a guerra, com a sua rígida injunção de matar ou ser morto. Talvez não por coincidência, ambos são de grande importância para o conservadorismo como tradição teórica e prática histórica.
 \par 
Rousseau e John Adams geralmente não são vistos como companheiros ideológicos, mas num ponto eles concordaram: as hierarquias sociais persistem porque garantem que todos, exceto aqueles que estão na base e no topo, desfrutem da oportunidade de governar e ser governados, por sua vez. . Não, certamente, no sentido aristotélico de autogoverno, mas no sentido feudal de governo recíproco: cada pessoa domina alguém abaixo dele em troca de se submeter a alguém acima dele. “Os cidadãos só se deixam oprimir na medida em que são levados pela ambição cega”, escreve Rousseau. “Como prestam mais atenção ao que está abaixo deles do que ao que está acima, a dominação torna-se-lhes mais cara do que a independência, e eles consentem em usar correntes para que, por sua vez, possam dá-las aos outros. É muito difícil reduzir à obediência alguém que não procura comandar.” {\color{blue}23} O aspirante e o autoritário não são tipos opostos: a vontade de ascender precede a vontade de curvar-se. Mais de trinta anos depois, Adams escreveria que todo homem deseja “ser observado, considerado, estimado, elogiado, amado e admirado”. {\color{blue}24} Para ser elogiado é preciso ser visto, e a melhor maneira de ser visto é elevar-se acima do seu círculo. Até mesmo o democrata americano, argumentou Adams, preferiria governar um inferior a desapropriar um superior. Sua paixão é pela supremacia, não pela igualdade, e enquanto lhe for garantida uma audiência inferior, ele ficará satisfeito com seu status inferior:
 \par 
Não apenas o mecânico mais pobre, mas o homem que vive da caridade comum, ou melhor, os mendigos comuns nas ruas. . . Corteje um conjunto de admiradores e vanglorie-se da superioridade que eles têm, ou imaginam ter, sobre alguns outros. . . . Quando um desgraçado não consegue mais atrair a atenção de um homem, mulher ou criança, ele deve ser respeitável aos olhos de seu cachorro. “Quem vai me amar então?” foi a resposta patética de alguém que passou fome para alimentar seu mastim, a um passageiro caridoso que o aconselhou a matar ou vender o animal.{\color{blue}25}
 \par 
Pode-se ver nestas descrições da hierarquia social os contornos do sublime: aniquilado a partir de cima, engrandecido a partir de baixo, o eu é ampliado e miniaturizado pelo seu envolvimento na prática do governo. Mas aqui está o problema: uma vez que temos realmente a certeza do nosso poder sobre outro ser, diz Burke, o nosso inferior perde a sua capacidade de nos prejudicar ou ameaçar. Ela perde sua sublimidade. “Tire” uma criatura “de sua capacidade de ferir” e “você a estragará com tudo que é sublime”. {\color{blue}26} Leões, tigres, panteras e rinocerontes são sublimes não porque sejam exemplares magníficos de força, mas porque podem e irão nos matar. Bois, cavalos e cães também são fortes, mas não têm o instinto de matar ou tiveram esse instinto suprimido. Eles podem ser feitos para nos servir e no caso dos cães até nos amar. Como tais criaturas, por mais fortes que sejam, não podem nos ameaçar ou prejudicar, elas são incapazes de sublimidade. Eles são objetos de desprezo, sendo o desprezo “o atendente de uma força que é subserviente e nociva”.{\color{blue}27}
 \par 
Temos continuamente ao nosso redor animais de uma força considerável, mas não perniciosa. Entre estes, nunca procuramos o sublime: ele vem até nós na floresta sombria e no deserto uivante. . . . Sempre que a força é apenas útil e empregada para nosso benefício ou prazer, ela nunca é sublime; pois nada pode agir de acordo conosco, se não agir de acordo
 \par 
Conformidade com a nossa vontade; mas para agir de acordo com a nossa vontade, ela deve estar sujeita a nós; e, portanto, nunca pode ser a causa de uma concepção grandiosa e dominante.{\color{blue}28}
 \par 
Portanto, pelo menos metade da experiência da hierarquia social – não a experiência de ser governado, que traz a possibilidade de ser destruído, humilhado, ameaçado ou prejudicado por um superior, mas a experiência de governar facilmente outro – é incompatível. com, e na verdade enfraquece, o sublime. Confi rmados em nosso poder, somos embalados pela mesma facilidade e conforto, passamos pela mesma fusão interior que experimentamos enquanto estamos no auge do prazer. A segurança do governo é tão debilitante quanto a paixão do amor.
 \par 
As insinuações de Burke sobre os perigos de um governo há muito estabelecido refletem uma tensão surpreendente dentro do conservadorismo: um desconforto persistente, embora não reconhecido, com o poder que amadureceu e amadureceu, autoridade que se tornou confortável e segura. Começando com o próprio Burke, os conservadores expressaram um profundo desconforto sobre as classes dominantes tão seguras de seu lugar ao sol que perdem sua capacidade de governar: sua vontade de poder se dissipa; os músculos e a inteligência de seu comando se atenuam.
 \par 
Como vimos no capítulo 1, Burke acredita que o Antigo Regime é lindo. Por isso também é “lento, inerte e tímido”. Não pode defender-se “das invasões de capacidade”, com a capacidade a substituir os novos homens de poder que a Revolução traz à tona. O interesse monetário, também aliado à Revolução, é mais forte do que o interesse fundiário porque está “mais pronto para qualquer aventura” e “mais disposto a novos empreendimentos de qualquer tipo”. {\color{blue}29} O Antigo Regime é belo, estático, fraco; a Revolução é feia, dinâmica, forte. “É uma verdade terrível”, admite Burke na segunda das suas Cartas sobre uma Paz Regicida, “mas é uma verdade que não pode ser escondida; em habilidade, em destreza, na distinção de seus pontos de vista, os jacobinos são nossos superiores”.{\color{blue}30}
 \par 
Joseph de Maistre foi menos diplomático do que Burke nas suas condenações do Antigo Regime, talvez porque encarasse as suas falhas de forma mais pessoal. Muito antes da Revolução, afirma ele, a liderança do Antigo Regime estava confusa e desnorteada. Naturalmente, as classes dominantes foram incapazes de compreender, e muito menos de resistir, ao ataque desencadeado contra elas. A impotência, física e cognitiva, foi – e continua sendo – o grande pecado do Antigo Regime. A aristocracia não consegue compreender; não pode agir. Alguma parcela da nobreza pode ser bem-intencionada, mas não consegue levar seus projetos até o fim. Eles são arrogantes e tolos. Eles têm virtude, mas não virtude. A aristocracia “falha ridiculamente em tudo o que empreende”. O clero foi corrompido pela riqueza e pelo luxo. A monarquia tem demonstrado consistentemente que lhe falta a vontade de “punir”, que é a marca registrada de todo verdadeiro soberano. {\color{blue}31} Confrontado com tal decadência, o inevitável crescimento de séculos no poder, Maistre conclui que é bom que a contra-revolução ainda não tenha triunfado (escreveu em 1797). O Antigo Regime precisa de mais alguns anos no deserto se quiser livrar-se das influências corruptoras da sua outrora bela vida:
 \par 
A restauração do trono significaria um súbito relaxamento da força motriz do Estado. A magia negra em ação no momento desapareceria como névoa diante do sol. A bondade, a clemência, a justiça, todas as virtudes gentis e pacíficas, reapareceriam subitamente e trariam consigo uma mansidão geral de caráter, uma certa alegria inteiramente oposta aos rigores do regime revolucionário.{\color{blue}32}
 \par 
Um século mais tarde, um caso semelhante será apresentado por Georges Sorel contra a belle époque. Sorel não é normalmente visto como uma figura emblemática da direita – por outro lado, até o conservadorismo de Burke continua a ser um tema de disputa {\color{blue}33} – e, de facto, a sua maior obra, Reflexões sobre
 \par 
A violência é frequentemente considerada uma contribuição, ainda que menor, para a tradição marxista. No entanto, o início de Sorel é conservador e o seu final proto-fascista, e mesmo na sua fase marxista a sua principal preocupação é a decadência e a vitalidade, em vez da exploração e da justiça. As críticas que ele faz às classes dominantes francesas no final do século XIX não são diferentes daquelas feitas por Burke e Maistre no final do século XVIII. Ele até torna a comparação explícita: a burguesia francesa, escreve Sorel, “tornou-se quase tão estúpida quanto a nobreza do século XVIII”. Eles são “uma aristocracia ultracivilizada que exige ser deixada em paz”. Antigamente, a burguesia era uma raça de guerreiros. “Capitães ousados”, eram “criadores de novas indústrias” e “descobridores de terras desconhecidas”. Eles “dirigiram empresas gigantescas”, inspirados por aquele “espírito conquistador, insaciável e impiedoso” que construiu ferrovias, subjugou continentes e criou uma economia mundial. Hoje, são tímidos e cobardes, recusando-se a tomar as medidas mais elementares para defender os seus próprios interesses contra os sindicatos, os socialistas e a esquerda. Em vez de desencadearem violência contra os trabalhadores em greve, rendem-se à ameaça de violência dos trabalhadores. Falta-lhes o ardor, o fogo na barriga, dos seus antepassados. É difícil não concluir que “a burguesia está condenada à morte e que o seu desaparecimento é apenas uma questão de tempo”.{\color{blue}34}
 \par 
Carl Schmitt formalizou o desprezo de Sorel pelas fraquezas das classes dominantes numa teoria inteira da política. De acordo com Schmitt, o burguês era como ele era – avesso ao risco, egoísta, desinteressado na bravura ou na morte violenta, desejoso de paz e segurança – porque o capitalismo era a sua vocação e o liberalismo a sua fé. Nenhum dos dois lhe forneceu um bom motivo para morrer pelo Estado. Na verdade, ambos lhe deram boas razões, na verdade todo um vocabulário, para não morrer pelo Estado. Juros, liberdade, lucro, direitos, propriedade, individualismo e outras palavras semelhantes criaram uma das classes dominantes mais egocêntricas da história, uma classe que gozava de privilégios, mas não se sentia
 \par 
Ela própria obrigada a defender esse privilégio. Afinal, a premissa da democracia liberal era a separação da política da economia e da cultura. Alguém poderia buscar lucro, às custas de outra pessoa, e pensar livremente, não importa quão subversivos fossem os pensamentos, sem perturbar o equilíbrio de poder. A burguesia, no entanto, estava enfrentando um inimigo que entendia muito bem as conexões entre ideias, dinheiro e poder, que arranjos econômicos e argumentos intelectuais eram o material do combate político. Os marxistas obtiveram a distinção amigo-inimigo, que é constitutiva da política; a burguesia não. {\color{blue}35} O espírito de Hegel costumava residir em Berlim; há muito tempo ele “vagou para Moscou”.{\color{blue}36}
 \par 
Sorel identificou uma excepção a esta regra da decadência capitalista: os barões ladrões dos Estados Unidos. Nos Carnegies e nos Goulds da indústria americana, Sorel pensou ter visto “a energia indomável, a audácia baseada numa avaliação precisa da força, o cálculo frio dos interesses, que são as qualidades dos grandes generais e dos grandes capitalistas”. Ao contrário da burguesia mimada de França e da Grã-Bretanha, os milionários de Pittsburgh e Pittston “levam até ao fim das suas vidas uma existência de escravos nas galés, sem nunca pensarem em levar uma vida de nobre, como fazem os Rothschilds”.{\color{blue}37}
 \par 
O homólogo espiritual de Sorel do outro lado do Atlântico, Teddy Roosevelt, não era tão otimista em relação aos industriais e anos de finança americanos. (A ansiedade burkeana em relação às classes dominantes é comum aos conservadores europeus e americanos.) O capitalista, declarou Roosevelt, vê o seu país como uma “caixa registadora”, sempre pesando a “honra da nação e a glória da bandeira” contra uma “interrupção temporária de ganhar dinheiro.” Ele não está “disposto a dar a vida por pequenas coisas” como a defesa da nação. Ele se preocupa “apenas se o valor das ações sobe ou desce”. {\color{blue}38} Ele não demonstra interesse em grandes assuntos de Estado, nacionais ou internacionais, a menos que estes interfiram com os seus próprios. Não foi por acaso, afirmou Roosevelt,
 \par 
Talvez com um aceno a Carnegie, que tais homens se opuseram à grande expedição imperial que foi a Guerra Hispano-Americana. {\color{blue}39} Complacentes e confortáveis, seguros das suas riquezas pelo sucesso das guerras laborais das décadas anteriores e da eleição de 1896, estes não eram homens com quem se pudesse contar para defender a nação ou mesmo a si próprios. “Poderemos algum dia ter uma causa amarga”, declarou Roosevelt, “para perceber que uma nação rica que é preguiçosa, tímida ou pesada é uma presa fácil” para outros povos mais marciais. O perigo que enfrenta uma classe dominante, e uma nação dominante, que se tornou “qualificada no comércio e nas finanças” é que “perda as duras virtudes de combate”.{\color{blue}40}
 \par 
Roosevelt não foi o primeiro conservador americano a preocupar-se com o abrandamento das classes dominantes e com as hierarquias repletas de poder. Nem ele seria o último. Ao longo da década de 1830, vimos no capítulo 1, à medida que os abolicionistas começaram a pressionar a sua causa, John C. Calhoun ficou furioso com a vida fácil e a ignorância voluntária dos seus camaradas na plantação. Eles haviam se tornado preguiçosos, gordos e complacentes, desfrutando tão plenamente dos privilégios de sua posição que não conseguiam prever a catástrofe que se aproximava. Ou, se pudessem, os proprietários do Sul não poderiam fazer nada para a defender, uma vez que os seus músculos políticos e ideológicos já tinham atrofiado há muito tempo. {\color{blue}41} Barry Goldwater também expressou desprezo pelo establishment republicano. {\color{blue}42} E ao longo da década de 1990 – para avançar mais três décadas – podíamos ouvir os herdeiros de Roosevelt à direita dirigirem o mesmo veneno contra o capitalista americano contra os mestres do universo em Wall Street e os empresários geeks de Silicon Valley.{\color{blue}43}
 \par 
Para que a classe dominante seja vigorosa e robusta, concluiu o conservador, os seus membros devem ser testados, exercitados e desafiados. Não apenas seus corpos, mas também suas mentes, até mesmo suas almas. Ecoando Milton: “Não posso elogiar uma virtude fugitiva e enclausurada, não exercida e não respirada, que nunca sai e a vê
 \par 
Adversário, mas foge da corrida. . . . Aquilo que nos purifica é a provação, e a provação ocorre pelo que é contrário” {\color{blue}44} —Burke acredita que a adversidade e a dificuldade, o confronto com a aflição e o sofrimento, criam seres mais fortes e mais virtuosos.
 \par 
As grandes virtudes giram principalmente em torno de perigos, punições e problemas, e são exercidas mais na prevenção de danos do que na distribuição de favores; e, portanto, não são amáveis, embora altamente veneráveis. Os subordinados recorrem a alívios, gratificações e indulgências; e são, portanto, mais amáveis, embora inferiores em dignidade. Aquelas pessoas que penetram nos corações da maioria das pessoas, que são escolhidas como companheiras de suas horas mais tranquilas e como alívio de cuidados e ansiedade, nunca são pessoas de qualidades brilhantes, nem virtudes fortes.{\color{blue}45}
 \par 
Talvez vejamos aqui as origens da preferência conservadora pela guerra em detrimento do Estado de bem-estar social, mas isso é outro assunto para outro dia). Mas enquanto Milton e outros republicanos com ideias semelhantes acreditam que a impureza e a corrupção aguardam os complacentes e confortáveis, Burke espia o espectro mais aterrorizante da dissipação, da degeneração e da morte. Se quisermos que os poderosos permaneçam poderosos, se quisermos continuar vivos, o seu poder, e na verdade a credibilidade da sua própria existência, deve ser continuamente desafiado, ameaçado e defendido.
 \par 
Uma das características mais impressionantes – embora espero que já sejam inteligíveis – do discurso conservador é o fascínio, e na verdade apreço, que se encontra pelos inimigos do conservador, particularmente pelo uso da violência contra ele e os seus aliados. Os comentários mais arrebatadores de Maistre nas suas Considerações sobre a França são reservados aos jacobinos, cuja vontade brutal e propensão para a violência – a sua “magia negra” – ele claramente inveja. Graças aos seus esforços, a França
 \par 
Foi purificado e restaurado ao seu legítimo lugar de honra entre a família das nações. Eles reuniram o povo contra invasores estrangeiros, um “prodígio” que “somente o gênio infernal de Robespierre poderia realizar”. Ao contrário da monarquia, a Revolução tem a vontade de punir.{\color{blue}46}
 \par 
Da perspectiva do sublime burkeano, contudo, o argumento de Maistre só vai até certo ponto. A Revolução rejuvenesce o Antigo Regime, forçando-o a sair do poder e purificando o povo através da violência. Ele proporciona um choque esclarecedor ao sistema. Mas Maistre nunca contempla, ou pelo menos nunca discute, o efeito revivificador que a recuperação do poder da Revolução poderia ter sobre os líderes do Antigo Regime. E, de facto, quando ele consegue descrever como pensa que a contra-revolução ocorrerá, a batalha final revela-se um caso surpreendentemente anticlimático, quase sem qualquer tiro. “Como acontecerá a contra-revolução se vier?” Maistre pergunta. “Quatro ou cinco pessoas, talvez, darão um rei à França.” Não é exatamente o material de uma classe dominante viril e transformada, lutando para voltar ao poder.{\color{blue}47}
 \par 
Maistre nunca contemplou as possibilidades restaurativas do combate corpo a corpo entre o Antigo Regime e a Revolução; para isso é preciso recorrer a Sorel. E embora as lealdades de Sorel na guerra entre os governantes e os governados do final do século XIX sejam mais ambíguas do que as de Maistre, a sua descrição do efeito da violência dos governados sobre os governantes não o é. A burguesia francesa perdeu o seu espírito de luta, afirma Sorel, mas esse espírito está vivo e bem entre os trabalhadores. O seu campo de batalha é o local de trabalho, a sua arma é a greve geral e o seu objectivo é a derrubada do Estado. É este último aspecto que mais impressiona Sorel, pois o desejo de derrubar o Estado sinaliza o quão despreocupados os trabalhadores estão com “os lucros materiais da conquista”. Não só não procuram salários mais elevados e outras melhorias no seu bem-estar; em vez disso, eles se concentraram no máximo
 \par 
Objectivos improváveis ​​– derrubar o Estado através de uma greve geral. É essa improbabilidade, a distância entre meios e fins, que torna a violência do proletariado tão gloriosa. Os proletários são como guerreiros homéricos, absortos na grandeza da batalha e indiferentes aos objectivos da guerra: Quem realmente alguma vez derrubou um Estado através de uma greve geral? A violência deles é uma violência por si só, sem preocupação com custos, benefícios e cálculos intermediários. {\color{blue}48} Como Ernst Jünger escreveu uma geração mais tarde, “não é aquilo por que lutamos, mas como lutamos”.{\color{blue}49}
 \par 
Mas o que domina Sorel não é o proletariado, mas os efeitos rejuvenescedores que poderá ter sobre a burguesia. Poderá a violência da greve geral “devolver à burguesia um ardor que se extinguiu?” Certamente o vigor do proletariado poderá despertar novamente a burguesia para os seus próprios interesses e para as ameaças que a sua retirada da política representa para esses interesses. Mais tentadora para Sorel, no entanto, é a possibilidade de a violência dos trabalhadores “restaurar [à burguesia] as qualidades guerreiras que anteriormente possuía”, forçando a “classe capitalista a permanecer ardente na luta industrial”. Através da luta contra o proletariado, por outras palavras, a burguesia pode recuperar a sua ferocidade e o seu ardor. E o ardor é tudo. Somente do ardor, daquela esplêndida indiferença à razão e ao interesse próprio, uma civilização inteira, afogada no materialismo e na complacência, será despertada. Uma classe dominante, ameaçada pela violência dos governados, despertada pelo seu próprio gosto pela violência – essa é a promessa da guerra civil em França.{\color{blue}50}
 \par 
Para o conservador, por mais modulado ou moderado que seja, um vigor renovado sempre foi a promessa da guerra civil. Pois entre os casos fáceis de um reacionário católico como Maistre e de um protofascista como Sorel está o exemplo mais difícil, mas em última análise mais revelador, de Alexis de Tocqueville. A sua passagem da moderação da Monarquia de Julho para o revanchismo de 1848 demonstra quão fácil e inexoravelmente o conservador burkeano irá
 \par 
Balanço do belo ao sublime, como a música da prudência e da moderação dá lugar à marcha da violência e do vitríolo.{\color{blue}51}
 \par 
Apresentando-se publicamente como o realista consumado, criterioso e judicioso, com pouca paciência para entusiasmo de qualquer tipo, Tocqueville era na verdade um romântico enrustido. Ele confessou ao irmão que compartilhava da "impaciência devoradora" do pai, sua "necessidade de sensações vivas e recorrentes". A razão, ele disse, "sempre foi para mim como uma gaiola", atrás da qual ele "rangeria [seus] dentes". Ele ansiava pela "visão do combate". Olhando para trás, para a Revolução Francesa, da qual ele perdeu (ele nasceu em 1805), ele lamentou o fim do Terror, alegando que "homens assim esmagados não só não podem mais atingir grandes virtudes, mas parecem ter se tornado quase incapazes de grandes crimes". Até Napoleão, flagelo dos conservadores, moderados e liberais em todos os lugares, ganhou a admiração de Tocqueville como o "ser mais extraordinário que apareceu no mundo por muitos séculos". Quem, por outro lado, poderia encontrar inspiração na política parlamentar da Monarquia de Julho, aquela “pequena panela de sopa democrática e burguesa”?
 \par 
No entanto, assim que iniciou uma carreira na política, foi naquela pequena panela burguesa de sopa que Tocqueville saltou. Previsivelmente, não era do seu gosto. Tocqueville pode ter proferido palavras de moderação, compromisso e Estado de direito, mas elas não o comoveram. Sem a ameaça da violência revolucionária, a política simplesmente não era o grande drama que ele imaginava que tivesse sido entre 1789 e 1815. “Os nossos pais observaram coisas tão extraordinárias que, comparadas com elas, todas as nossas obras parecem comuns.” A política de moderação e compromisso produziu moderação e compromisso; não produziu política, pelo menos não na forma como Tocqueville entendia o termo. Durante as décadas de 1830 e 1840, “o que mais faltava. . . Foi a própria vida política.” Não havia “nenhum campo de batalha para as partes em conflito se encontrarem”. A política foi “privada” de “toda originalidade, de toda realidade e, portanto, de todas as paixões genuínas”.
 \par 
Então veio 1848. Tocqueville não apoiou a Revolução. Na verdade, ele estava entre os seus oponentes mais veementes. Votou a favor da suspensão total das liberdades civis, o que anunciou com alegria que foi feito “com ainda mais energia do que tinha sido feito sob a Monarquia”. Ele acolheu bem os rumores de uma ditadura – para proteger o mesmo regime que passou a maior parte de duas décadas menosprezando. E ele adorou tudo: a violência, a contraviolência, a batalha. Defendendo a moderação contra o radicalismo, foi dada a Tocqueville a oportunidade de utilizar meios radicais para fins moderados, e não está totalmente claro qual dos dois mais o comoveu.
 \par 
Deixe-me dizer, então, que quando procurei cuidadosamente nas profundezas do meu próprio coração, descobri, com alguma surpresa, uma certa sensação de alívio, uma espécie de alegria misturada com todas as tristezas e medos aos quais a Revolução havia causado. deu origem. Sofri com este acontecimento terrível para o meu país, mas claramente não para mim; pelo contrário, parecia-me respirar mais livremente do que antes da catástrofe. Sempre me senti sufocado na atmosfera do mundo parlamentar que acabava de ser destruído: achei-o cheio de desilusões, tanto no que diz respeito aos outros como no que me dizia respeito.
 \par 
Autoproclamado poeta do hesitante, do sutil e do complexo, Tocqueville ardeu de entusiasmo ao acordar para um mundo dividido em dois campos. Parlamentos tímidos semearam uma confusão cinzenta; a guerra civil forçou à nação uma clareza revigorante de preto e branco. “Não sobrou campo para a incerteza mental: deste lado estava a salvação do país; nisso, sua destruição. . . . A estrada parecia perigosa, é verdade, mas minha mente está construída de tal forma que tem menos medo do perigo do que da dúvida.” Para este membro da classe dominante, a sublimidade brota da violência das classes inferiores e uma oportunidade de escapar à beleza sufocante da vida no Parnaso burguês.
 \par 
Francis Fukuyama é talvez o mais ponderado dos escritores recentes a seguir esta linha conservadora de argumento sobre a violência. Ao contrário de Maistre, porém, ou de Tocqueville e Sorel – todos os quais escreveram no meio da batalha, quando o resultado não era claro – Fukuyama escreve da perspectiva da vitória. Estamos em 1992 e as classes capitalistas derrotaram os seus oponentes socialistas na longa guerra civil do curto século XX. Não é uma visão bonita, pelo menos não para Fukuyama. Pois o revolucionário foi um dos poucos homens temáticos do século XX. O homem timótico é como o trabalhador de Sorel: aquele que arrisca a vida em prol de um princípio improvável, que não se preocupa com seus próprios interesses materiais e se preocupa apenas com a honra, a glória e os valores pelos quais luta. Depois de uma estranha, mas breve homenagem aos Bloods e aos Crips como homens temáticos, Fukuyama olha com carinho para homens de propósito e poder como Lenin, Trotsky e Stalin, “lutando por algo mais puro e superior” e possuidores de “dureza maior do que o normal”. , visão, crueldade e inteligência. Em virtude da sua recusa em acomodar-se à realidade do seu tempo, eles eram os “mais livres e, portanto, os mais humanos dos seres”. Mas, de uma forma ou de outra, estes homens e os seus sucessores perderam a guerra civil do século XX, quase inexplicavelmente, para as forças do “Homem Económico”. Pois o Homem Económico é “o verdadeiro burguês”. Tal homem nunca estaria “disposto a andar na frente de um tanque ou confrontar uma fila de soldados” por qualquer causa, mesmo a sua própria. No entanto, o Homem Económico é o vencedor e, longe de o rejuvenescer ou restaurar os seus poderes primordiais, a guerra parece apenas tê-lo tornado mais burguês. Conservador como é, Fukuyama só pode irritar-se com o triunfo do Homem Económico e com “a vida de consumo racional” que ele trouxe, uma vida que é “no final das contas, chata”.{\color{blue}52}
 \par 
Longe de ser excepcional, a decepção de Fukuyama sobre o efeito real – em oposição ao efeito antecipado ou fantasiado – da violência sobre uma classe dominante dissipada é emblemática. “Os objetivos
 \par 
A batalha e os frutos da conquista nunca são os mesmos”, observou E. M. Forster em A Passage to India. “Estes últimos têm o seu valor e só o santo os rejeita, mas o seu indício de imortalidade desaparece assim que são segurados nas mãos.” {\color{blue}53} Nas profundezas do discurso conservador esconde-se um elemento de anticlímax que não pode ser contido. Embora o conservador se volte para a violência como forma de se libertar, ou das classes dominantes, do tédio mortal e da atrofia suavizante que acompanha o poder, praticamente todos os encontros no discurso conservador com a violência real implicam desilusão e deflação.
 \par 
Lembremo-nos de Teddy Roosevelt, meditando sobre o materialismo e a fraqueza das classes capitalistas da América. Onde, perguntou-se ele, se poderia encontrar um exemplo da “vida extenuante” – a emoção da dificuldade e do perigo, a luta que levou ao progresso – na América contemporânea? Talvez nas guerras e conquistas estrangeiras que a América empreendeu no final do século. No entanto, mesmo aqui Roosevelt encontrou frustração. Embora os seus relatórios sobre a Guerra Hispano-Americana estivessem repletos de bravura e bravata, uma leitura cuidadosa das suas aventuras em Cuba sugere que as suas façanhas foram um fiasco. Cada uma das famosas investidas que Roosevelt conduziu para cima ou para baixo de uma colina foi um anticlímax. A primeira culminou com ele vendo exatamente dois soldados espanhóis abatidos pelos seus homens: “Estes foram os únicos espanhóis que realmente vi serem atingidos por tiros certeiros de qualquer um dos meus homens”, escreveu ele, “exceto dois guerrilheiros nas árvores”. A segunda o encontrou liderando um exército que não o ouviu nem o seguiu. Por isso, foi com uma apreciação sombria que ele recitou os comentários dispépticos de um dos líderes do exército em Cuba, um certo General Wheeler, que “tinha passado por demasiados combates pesados ​​na Guerra Civil para considerar a luta actual como muito séria”.{\color{blue}54}
 \par 
Nas ocupações sangrentas que se seguiram à Guerra Hispano-Americana, porém, Roosevelt pensou ter visto a verdadeira felicidade que era estar vivo naquela madrugada. Roosevelt tinha certeza de que as ocupações de
 \par 
As Filipinas e outros lugares estavam tão perto de uma repetição da Guerra Civil — aquela nobre cruzada de virtude imaculada — quanto ele e seus compatriotas provavelmente veriam. “Nós, desta geração, não temos que enfrentar uma tarefa como a que nossos pais enfrentaram”, ele declarou em 1899, “e ai de nós se não conseguirmos realizá-la! . . . Não podemos evitar as responsabilidades que nos confrontam no Havaí, Cuba, Porto [sic] Rico e Filipinas.” Aqui — nas ilhas do Caribe e do Pacífico — estava a confluência de sangue e propósito que ele vinha buscando por toda a sua vida. A tarefa de elevação imperial, de educar os nativos na “causa da civilização”, era árdua e violenta, impondo uma missão à América que levaria anos, se Deus quisesse, para ser cumprida. Se a missão imperial tivesse sucesso — e mesmo se falhasse — ela criaria uma classe dominante genuína na América, endurecida e extenuada pela batalha, mais nobre e menos suja do que os asseclas de Carnegie.{\color{blue}55}
 \par 
Foi um sonho lindo. Mas também não suportou o peso da realidade. Embora Roosevelt esperasse que os homens que governavam as Filipinas fossem “escolhidos pela sua capacidade e integridade de sinal”, governando “as províncias em nome de toda a nação de onde vêm, e pelo bem de todo o povo para onde vão, ” ele temia que os ocupantes coloniais da América viessem da mesma classe de financistas e industriais egoístas que o levaram para o estrangeiro em primeiro lugar. E assim os seus louvores ao imperialismo terminaram com uma nota amarga de advertência, até mesmo de condenação. “Se permitirmos que o nosso serviço público nas Filipinas se torne presa dos despojos políticos, se não conseguirmos mantê-lo ao mais alto padrão, seremos culpados de um acto, não só de maldade, mas de fraqueza e falta de poder. loucura avistada, e teremos começado a trilhar o caminho que foi trilhado pela Espanha para sua própria humilhação amarga.{\color{blue}56}
 \par 
Mas se seu sonho terminou mal, Roosevelt pelo menos teve a vantagem de poder dizer que sempre suspeitou que isso aconteceria. O mesmo não poderia ser dito dos fascistas da Itália, cujo autoengano
 \par 
A arrancada do poder à esquerda persistiu durante décadas, testemunhando uma incapacidade de enfrentar a sua própria decepção. Durante anos, os fascistas celebraram a Marcha sobre Roma de 1922 como o triunfo violento e glorioso da vontade sobre a adversidade. {\color{blue}28} de outubro, dia da chegada dos Camisas Negras a Roma, tornou-se feriado nacional; foi declarado o primeiro dia do Ano Novo Fascista após a introdução do novo calendário em 1927. A história da chegada de Mussolini em particular – vestindo a proverbial camisa preta – foi repetida com admiração. “Senhor”, ele supostamente disse ao rei Victor Emmanuel III, “perdoe meu traje. Eu venho dos campos de batalha.” Na verdade, Mussolini viajou durante a noite de trem desde Milão, onde frequentava conspicuamente o teatro, cochilando confortavelmente no vagão-leito. A única razão pela qual conseguiu chegar a Roma foi que um tímido establishment, liderado pelo rei, telefonou-lhe para Milão com um pedido para que formasse um governo. Quase nenhum tiro foi vermelho, de cada lado. {\color{blue}57} Maistre não poderia ter escrito melhor.
 \par 
Podemos ver um fenômeno semelhante em jogo na guerra contra o terror. Embora muitos vejam o governo Bush e o neoconservadorismo como afastamentos do conservadorismo adequado — a declaração mais recente dessa tese sendo The Death of Conservatism 58, de Sam Tanenhaus — o projeto neocon de aventureirismo imperial traça o arco burkeano de violência do começo ao fim. Já discuti, no capítulo 8, como os neoconservadores viram o {\color{blue}11} de setembro e a guerra contra o terror como uma chance de escapar da paz e prosperidade decadentes e mortíferas dos anos Clinton, que eles acreditavam ter enfraquecido a sociedade americana. Exalando conforto, os americanos — e mais importante, seus líderes — supostamente perderam a vontade, o desejo e a capacidade de governar o mundo. Então o {\color{blue}11} de setembro aconteceu, e de repente parecia que eles podiam.
 \par 
Esse sonho, claro, está agora em frangalhos, mas vale a pena notar um dos seus aspectos mais idiossincráticos, pois apresenta uma ruga no
 \par 
A longa saga da violência conservadora. De acordo com muitos conservadores, e não apenas os neoconservadores, uma das fontes recentes da decadência americana, que remonta ao Tribunal Warren e às revoluções pelos direitos da década de 1960, é a obsessão liberal com o Estado de direito. Esta obsessão, aos olhos dos conservadores, assume muitas formas: a insistência no devido processo no processo penal; uma parcialidade em litígios sobre legislação; uma ênfase na diplomacia e no direito internacional sobre a guerra; tentativas de restringir o poder executivo através da supervisão judicial e legislativa. Por mais não relacionados que estes sintomas possam parecer, os conservadores vêem neles uma única doença: uma cultura de regras e leis que lentamente incapacita e desvitaliza a fera loira de rapina que é o poder americano. Estes são sinais de uma insalubridade nietzschiana, e o {\color{blue}11} de Setembro foi o resultado inevitável.
 \par 
Se quisermos evitar outro {\color{blue}11} de Setembro, essa cultura de direitos e regras deve ser repudiada e revertida. Como deixam claro os relatórios de Seymour Hersh e Jane Mayer, a guerra ao terror – com o seu impulso à tortura, à derrubada das Convenções de Genebra, à recusa das restrições do direito internacional, à vigilância ilegal e à visão do terrorismo através das lentes de guerra em vez de crime e punição – reflecte tanto, se não mais, estas sensibilidades e sensibilidades conservadoras como os factos reais do {\color{blue}11} de Setembro e a necessidade de evitar outro ataque. {\color{blue}59} “Ela é mole – demasiado mole”, diz o agora reformado tenente-general Jerry Boykin sobre os Estados Unidos, antes e depois do {\color{blue}11} de Setembro. A forma de a tornar dura não é apenas empreender acções militares difíceis e extenuantes, mas também violar as regras – e a cultura de regras – que a tornaram branda em primeiro lugar. Os Estados Unidos têm de aprender a “viver no limite”, afirma o ex-diretor da NSA Michael Hayden. “Não há nada que não façamos, nada que não tentemos”, acrescenta o ex-diretor da CIA George Tenet.{\color{blue}60}
 \par 
A grande ironia da guerra ao terror é que, longe de emancipar a fera loura de rapina, a guerra criou a lei, e os advogados,
 \par 
Muito mais crítico do que se possa imaginar. Como relata Mayer, o impulso à tortura, ao poder executivo desenfreado, à derrubada das Convenções de Genebra, e assim por diante, não veio da CIA ou dos militares; as forças motrizes eram advogados da Casa Branca e do Departamento de Justiça, como David Addington e John Yoo. Longe de virtuosos maquiavélicos da violência transgressora, Addington e Yoo são fanáticos pela lei e insistem em justificar a sua violência através da lei. Além disso, os advogados supervisionam consistentemente a prática real da tortura. Como Tenet escreveu em suas memórias: “Apesar do que Hollywood possa fazer você acreditar, em situações como esta [a captura, interrogatório e tortura do chefe de logística da Al-Qaeda, Abu Zubayda], você não chama os durões; você chama os advogados. Cada tapa na cara, cada soco no estômago, cada sacudida do corpo – e muito, muito pior – devem primeiro ser aprovados pelos superiores das diversas agências de inteligência, inevitavelmente em consulta com advogados. Mayer compara a prática da tortura a um jogo de “Mãe, posso?” Como afirma um interrogador: “Antes que você pudesse colocar a mão nele [a vítima de tortura], você tinha que enviar um telegrama dizendo: ‘Ele não coopera. Solicite permissão para fazer X.’ E a permissão viria, dizendo ‘Você pode dar um tapa na barriga dele uma vez com a mão aberta.{\color{blue}61}
 \par 
Em vez de libertar a fera loira para vaguear e atacar como quiser, a remoção da proibição da tortura e a suspensão das Convenções de Genebra deixaram-no, ou pelo menos aos advogados que o controlam, mais ansiosos. Até onde ele pode ir? O que ele pode fazer? Cada ato de violência, como revela esta conversa entre dois advogados do Pentágono, torna-se um seminário de faculdade de direito:
 \par 
O que significa “privação de estímulos luminosos e auditivos”? Um prisioneiro poderia ser trancado em uma cela completamente escura? Se sim, ele poderia ficar lá por um mês? Mais longo? Até ele ficar cego? O que, precisamente, permitiu a autoridade para explorar fobias? Poderia um
 \par 
Detido será mantido em caixão? Que tal usar cães? Ratos? Até onde um interrogador poderia levar isso? Até um homem enlouquecer? {\color{blue}62}
 \par 
Depois, há a questão da combinação de técnicas de tortura aprovadas. Um interrogador pode negar comida ao prisioneiro e ao mesmo tempo diminuir a temperatura de sua cela? O efeito multiplicador das dores duplicadas e triplicadas cruza uma linha nunca definida? {\color{blue}63} Como ensinou Orwell, as possibilidades de crueldade e violência são tão ilimitadas quanto a imaginação que as imagina. Mas os exércitos e agências da violência de hoje são vastas burocracias, e vastas burocracias precisam de regras. Eliminar as regras não desvincula Prometeu; isso apenas gera mais horas faturáveis.
 \par 
“Não ceda. Sem equívocos. Nada de colocar essa coisa em camadas até a morte. Essa foi a promessa de George W. Bush após o {\color{blue}11} de Setembro e a sua descrição de como a guerra contra o terrorismo seria conduzida. Tal como tantas outras declarações de Bush, acabou por ser uma promessa vazia. Essa coisa foi mergulhada até a morte. Mas, e este é o ponto crítico, longe de minimizar a violência estatal – que era o grande medo dos neoconservadores – a estratificação provou ser perfeitamente compatível com a violência. Numa guerra já repleta de decepções e desilusões, a constatação que inevitavelmente se segue – o Estado de direito pode, de facto, autorizar as maiores aventuras de violência e morte, drenando-as assim de sublimidade – deve ser, para o conservador, a maior desilusão. de tudo.
 \par 
Se tivessem sido leitores mais atentos de Burke, os neoconservadores – como Fukuyama, Roosevelt, Sorel, Schmitt, Tocqueville, Maistre, Treitschke e tantos outros na direita americana e europeia – poderiam ter previsto esta desilusão. Burke certamente fez isso. Mesmo enquanto escrevia sobre os sublimes efeitos da dor e do perigo, ele teve o cuidado de insistir que se essas dores e perigos “se aproximassem demais” ou “muito perto” - isto é, se se tornassem realidades em vez de
 \par 
Do que fantasias, se elas se tornassem “conversas sobre a destruição presente da pessoa” — sua sublimidade desapareceria. Elas deixariam de ser “deliciosas” e restauradoras e se tornariam simplesmente terríveis. {\color{blue}64} O ponto de Burke não era meramente que ninguém, no final, realmente quer morrer ou que ninguém gosta de dor indesejada e excruciante. Era que a sublimidade de qualquer tipo e fonte depende da obscuridade: chegue muito perto de qualquer coisa, seja um objeto ou experiência, veja e sinta sua extensão total, e ela perde seu mistério e aura. Ela se torna familiar. Uma “grande clareza” do tipo que vem da experiência direta “é de alguma forma inimiga de todos os entusiasmos, sejam eles quais forem”. {\color{blue}65} “É nossa ignorância das coisas que causa toda a nossa admiração e principalmente excita nossas paixões. Conhecimento e familiaridade fazem com que as causas mais marcantes afetem pouco”. {\color{blue}66} “Uma ideia clara”, Burke conclui, “é, portanto, outro nome para uma pequena ideia”. {\color{blue}67} Conheça qualquer coisa, incluindo a violência, muito bem, e ela perde qualquer atributo — rejuvenescimento, transgressão, excitação, admiração — que você atribuiu a ela quando era apenas uma ideia. Antes da maioria, Burke entendeu que, se a violência mantivesse sua sublimidade, ela teria que permanecer uma possibilidade, um objeto de fantasia — um filme de terror, um videogame, um ensaio sobre a guerra. Pois a realidade (em oposição à representação) da violência estava em desacordo com os requisitos da sublimidade. A violência real, em oposição à imaginada, envolvia objetos chegando muito perto, corpos pressionando muito perto, carne sobre carne. A violência despojava o corpo de seus véus; a violência tornava seus antagonistas familiares uns aos outros de uma forma que nunca tinham sido antes. A violência dissipava a ilusão e o mistério, tornando as coisas monótonas e sombrias. É por isso que, na sua discussão nas Reflexões sobre o rapto de Maria Antonieta pelos revolucionários, Burke se esforça tanto para enfatizar o seu corpo “quase nu” e recorre tão facilmente à linguagem do vestuário — “o traje decente da vida”, o “guarda-roupa da imaginação moral”, a “moda antiquada” e assim por diante — para
 \par 
Descreva o evento. {\color{blue}68} O desastre da violência dos revolucionários, para Burke, não foi crueldade; foi a iluminação não desejada.
 \par 
Desde o {\color{blue}11} de Setembro, muitos se queixaram, e com razão, do fracasso dos conservadores – ou dos seus filhos e filhas – em lutarem eles próprios na guerra contra o terrorismo. Para os da esquerda, esse fracasso é sintomático da injustiça de classe da América contemporânea. Mas há um elemento adicional na história. Enquanto a guerra contra o terrorismo continuar a ser uma ideia – um tema quente nos blogues, um artigo de opinião provocativo, um episódio de {\color{blue}24} Horas – será sublime. Assim que a guerra contra o terrorismo se tornar realidade, pode ser tão triste como uma discussão sobre o código fiscal e tão tediosa como uma visita ao DMV.