 
 \chapter{Easy to Be Hard}  

 \label{Easy to Be Hard}  
 
 
\par
 
 
 \textit{	I enjoy wars. Any adventures better than sitting in an offi CE.}  

 
\par
 
 
 
\par
 

 \textbf{\textit{	—Harold Macmillan} }  

 
\par
 

 \footnote{This chapter originally appeared as “Easy to Be Hard: Conservatism and Violence,” in Performances of Violence, ed. Austin Sarah, Carleen Baler, and Thomas L. Dump (Amherst: University of Massachusetts Press, 2011), 18–42.}  
Apesar do apoio de eleitores e políticos autodenominados conservadores à pena de morte, à tortura e à guerra, intelectuais de direita frequentemente negam qualquer afinidade entre conservadorismo e violência.
 {\color{blue} 1}  
“Os conservadores”, escreve Andrew Sullivan, “podiam a guerra”.
 \textbf{\textit{Their domestic politics is rooted in a loathing of civil wars and violence, and they know that freedom is always the first casualty of international warfare. When countries go to war, their governments invariably get bigger and stronger, individual liberties are whittled away, and societies which once enjoyed the pluralist cacophony of freedom have to be marshaled into a single, collective note to face down an external foe. A state of permanent warfare—as George Orwell saw—is a virtual invitation to domestic tyranny. {{\color{blue} 2} } } } 

 
\par
 
Canalizando uma tradição de cepticismo desde Improvisado até Hume, o conservador identifica o governo limitado como a extensão da sua fé, e o Estado de direito como o seu único requisito para a busca da felicidade. Pragmática e adaptativa, mais disposta do que comprometida, tal sensibilidade – e é uma sensibilidade, insiste o conservador, não uma ideologia – não está interessada na violência. O seu endosso à guerra, tal como é, é a mais cansativa das concessões à realidade. Ao contrário dos seus amigos de esquerda – por mais conservador que seja, ele valoriza mais a amizade do que o acordo – ele sabe que vivemos e amamos no meio de um grande mal. Este mal deve ser combatido, por vezes por meios violentos. Se todas as coisas fossem iguais, ele gostaria de ver um mundo sem violência. Mas nem todas as coisas são iguais e ele não pretende ver o mundo como gostaria que fosse.
 
\par
 
O registo histórico do conservadorismo – não apenas como prática política, que não é a minha principal preocupação aqui, mas como tradição teórica – sugere o contrário. Longe de ficar entristecido, oprimido ou irritado pela violência, o conservador foi animado por ela. Não me refiro no sentido pessoal, embora muitos conservadores, como Harold Macmillan citado acima ou Winston Churchill citado abaixo, tenham expressado um entusiasmo inesperado pela violência. Minha preocupação é com ideias e argumentos, e não com caráter ou psicologia. A violência, afirmou o intelectual conservador, é uma das experiências da vida que nos faz sentir mais vivos, e a violência é uma atividade que torna a vida, bem, viva.
 {\color{blue} 3}  
Tais argumentos podem ser apresentados com agilidade: “Apenas os mortos viram o fim da guerra”, como disse certa vez Douglas Mac Arthur.
 {\color{blue} 4}  
— ou laboriosamente, como na Fatia:
 
\par
 

 \textbf{\textit{To the historian who lives in the world of will it is immediately clear that the demand for a perpetual peace is thoroughly reactionary; he sees that with war all movement, all growth, must be struck out of history. It has always been the tired,} }  
 
 
\par
 

 
\par
 

 \textbf{\textit{Unintelligent, and enervated periods that have played with the dream of perpetual peace. . . . However, it is not worth the trouble to discuss this matter further; the living God will see to it that war constantly returns as a dreadful medicine for the human race. {{\color{blue} 5} } } }  
 
 
\par
 
Enérgico ou prolixo, o caso se resume a isto: guerra é vida, paz é morte.
 
\par
 
Essa crença pode ser rastreada até A Philosophical Inquiry isto lhe Origin of. Our Ideas of. lhe Sublime ano lhe Beautiful, de Edmund Burke. Lá, Burke desenvolve uma visão do eu que precisa desesperadamente de estímulos negativos do tipo fornecido pela dor e pelo perigo, que Burke associa ao sublime. O sublime é mais facilmente encontrado em duas formas políticas: hierarquia e violência. Mas, por razões que ficarão claras, o conservador – mais uma vez, consistente com os argumentos de Burke – favorece frequentemente o último em detrimento do primeiro. O governo pode ser sublime, mas a violência é mais sublime. O mais sublime de tudo é quando os dois se fundem, quando a violência é praticada para criar, defender ou recuperar um regime de dominação e governo. Mas, como alertou Burke, é sempre melhor aproveitar a dor e o perigo à distância. A distância e a obscuridade aumentam a sublimidade; a proximidade e a iluminação diminuem-no. A violência contra-revolucionária pode ser o Everest da experiência conservadora, mas deve-se vê-la de longe. Chegue muito perto do topo da montanha e o ar ficará rarefeito e a vista ficará turva. No final de cada discurso sobre a violência, portanto, existe uma decepção que aguarda.
 
\par
 
O Sublime e o Belo começa com uma nota alta, com uma discussão sobre a curiosidade, que Burke identifica como “a primeira e mais simples emoção”. A curiosa corrida “de um lugar para caçar algo novo”. Sua visão está fixa, sua atenção está extasiada. Então o mundo fica cinza. Eles começam a tropeçar nas mesmas coisas, “com cada vez menos efeitos agradáveis”. A novidade diminui: quanto, realmente, há de novo no mundo? A curiosidade “se esgota”. O entusiasmo e o envolvimento dão lugar à “aversão e ao cansaço”.
 {\color{blue} 6}  
Burke passa para o prazer e a dor, que deveriam transformar a busca pela novidade em experiências mais sustentadas e profundas. Mas, em vez de um verdadeiro aditivo à curiosidade, o prazer do ERS é mais do mesmo: um momento de entusiasmo, seguido de um mal-estar monótono. “Quando termina a sua carreira”, diz Burke, o prazer “nos coloca quase onde nos encontrou”. Qualquer tipo de prazer “satisfaz rapidamente; e quando acaba, recaímos na indiferença.”
 {\color{blue} 7}  
Prazeres mais tranquilos, menos intensos que o prazer, são igualmente individuais, certo c. Geram complacência; nós “nos entregamos à indolência e à inação”.
 {\color{blue} 8}  
Burke recorre à imitação como outra força potencial de propulsão externa. Através da imitação, aprendemos boas maneiras e costumes, desenvolvemos opiniões e somos civilizados. Nós nos trazemos ao mundo, e o mundo é trazido até nós. Mas a imitação contém o seu próprio narcótico. Imitar demais os outros e deixaremos de nos melhorar. Seguimos a pessoa à nossa frente “e assim por diante em um círculo eterno”. Num mundo de imitadores, “nunca poderia haver qualquer melhoria”. Tais “homens devem permanecer como os brutos, os mesmos no final que são hoje e que eram no começo do mundo”.
 {\color{blue} 9}  

 
\par
 
A curiosidade leva ao cansaço, o prazer à indiferença, o prazer ao torpor e a imitação à estagnação. Tantas portas da psique se abrem para este espaço de escuridão inercial que poderíamos muito bem concluir que ele se esconde não nos limites, mas no centro da condição humana. Aqui, neste pátio escuro do eu, toda ação cessa, criando um ambiente ideal para “melancolia, desânimo, desespero e auto assassinato”.
 {\color{blue} 10}  
Até mesmo o amor, o mais exterior dos êxtases, leva o eu de volta a um estado de dissolução interna.
 {\color{blue} 11}  
O suicídio, ao que parece, é o destino inevitável que aguarda qualquer pessoa que tenha prazer no mundo tal como ele é.
 
\par
 
Para um certo tipo de teórico conservador, passagens como estas representam um certo desafio. Aqui está o inventor da tradição conservadora articulando uma visão do eu dramaticamente em desacordo com o eu imaginado do pensamento conservador. O eu conservador, como vimos repetidamente, afirma preferir “o familiar ao desconhecido. . . O tentado ao não experimentado, o fato ao mistério, o real ao possível, o limitado ao ilimitado, o próximo ao distante, o suficiente ao superabundante, o conveniente ao perfeito, o riso presente à felicidade utópica.”
 {\color{blue} 12}  
Ele gosta das coisas como elas são, não porque as considere justas ou boas, mas porque as considera familiares. Ele os conhece e está apegado a eles. Ele não deseja perdê-los nem que sejam levados embora. Desfrutar do que tem, em vez de adquirir algo melhor, é o seu bem maior. Mas se o eu do Sublime e do Belo tivesse certeza de seus apegos e familiares, ele rapidamente se veria confrontando o espectro de sua própria extinção, mais do que provavelmente por suas próprias mãos.
 
\par
 
Talvez seja este tédio letal, que se esconde logo abaixo da superfície do discurso conservador, que explica o fracasso do político conservador em seguir o exemplo do teórico conservador. Longe de abraçar a causa dos prazeres tranquilos e dos apegos seguros, o político conservador tem optado consistentemente por um ativismo do que ainda não é e do que será. O primeiro discurso inaugural de Ronald Reagan foi um hino ao poder dos sonhos: não sonhos pequenos, mas sonhos grandes e heroicos, de progresso e melhoria, e não sonhos por si só, mas sonhos como um estímulo necessário e vital para a ação. Três meses mais tarde, num discurso perante o Congresso, Reagan deixou claro a questão com uma citação de Carl Sandburg: “Nada acontece a menos que primeiro seja um sonho”. E nada que aconteça, ou muito poucas coisas que aconteçam, ou que as coisas não aconteçam com rapidez suficiente, é o que os conservadores na política não gostam. Reagan mal conseguia conter a sua impaciência com a hesitação dos políticos: “A maneira antiga e confortável é raspar um pouco aqui e acrescentar um pouco ali. Bem, isso não é mais aceitável.” Antiga e confortável era a acusação, sem “meias-medidas” o veredicto.
 {\color{blue} 13}  

 
\par
 
Reagan foi dificilmente o primeiro conservador a agir em prol do invisível e do ideal em vez do material e do real. Em seu discurso de aceitação na Convenção Nacional Republicana de 1964, Barry Goldwater não conseguiu encontrar nenhuma acusação mais potente para fazer ao estado de bem-estar social do que a de que ele havia deixado uma grande nação “calma”. Graças ao New Deal, os Estados Unidos perderam seu “ritmo acelerado” e agora estavam “se arrastando”. Calmo, lento e lento são geralmente bem-vindos pelo teórico conservador como sinais de felicidade presente. Mas para o político conservador, eles são males. Ele deve declarar guerra, reunindo seus exércitos contra os apáticos e lânguidos com conversas sobre “causas”, “luta”, “entusiasmo” e “devoção”.
 {\color{blue} 14}  

 
\par
 
Esse zelo cruzado não é peculiar ao conservadorismo americano. Também se encontra na Europa, até mesmo na Inglaterra, o país que fez da moderação o apelido de conservadorismo. “Quem ganhou uma batalha”, zombou Margaret Thatcher, “sob a bandeira ‘Eu defendo o Consenso’?”
 {\color{blue} 15}  
E há ainda Winston Churchill, que viajou para Cuba em 1895 para fazer uma reportagem sobre a guerra espanhola contra a independência cubana.
 {\color{blue} 16}  
Ruminando sobre as decepções de sua geração — retardatários que chegaram ao Império, foram privados da oportunidade de conquista imperial (em oposição à administração) — ele chegou a Havana. Isto é o que ele tinha a dizer (relembrando a experiência de 1930):
 
\par
 

 \textbf{\textit{The minds of this generation, exhausted, brutalized, mutilated and bored by War, may not understand the delicious yet tremulous sensations with which a young British Officer bred in the long peace approached for the first time an actual the- are of operations. When first in the dim light of early morning I saw the shores of Cuba rise and defi né themselves} }  
 
 
\par
 

 
\par
 

 \textbf{\textit{From dark-blue horizons, I felt as if I sailed with Long John Silver and first gazed on Treasure Island. Here was a place where real things were going on. Here was a scene of vital action. Here was a place where anything might happen. Here was a place where something would certainly happen. Here I might leave my bones. {{\color{blue} 17} } } }  
 
 
\par
 
Independentemente da relação entre teoria e prática na tradição conservadora, fica claro em O Sublime e o Belo que, para que o eu sobreviva e floresça, ele deve ser despertado por uma experiência mais vital e revigorante do que o prazer ou o prazer. Prazer e prazer agem como beleza, “relaxando os sólidos de todo o sistema”.
 {\color{blue} 18}  
Esse sistema, no entanto, deve ser tornado rígido e tenso. A mente deve ser acelerada, o corpo exercitado. Caso contrário, o sistema irá amolecer e atrofiar e, por fim, morrer.
 
\par
 
O que mais desperta esse estado elevado de ser é o confronto com o não-ser. A vida e a saúde são prazerosas e agradáveis, e é isso que está errado com elas: “elas não causam tal impressão” no eu porque “não fomos feitos para concordar com a vida e a saúde”. Dor e perigo, por outro lado, são “emissários” da morte, o “rei dos terrores”. Eles são fontes do sublime, “a mais forte” — mais poderosa, mais comovente — “emoção que a mente é capaz de sentir”.
 {\color{blue} 19}  
Dor e perigo, em outras palavras, são experiências geradoras do seu.
 
\par
 
Dor e perigo são geradores porque têm o efeito contraditório de minimizar e maximizar nosso senso de identidade. Ao sentir dor ou perigo, nossa mente “está tão completamente preenchida com seu objeto, que não consegue entreter nenhum outro”. Os “movimentos” de nossa alma “são suspensos”, enquanto o dano e os medos que ele desperta “invadem a mente”. Diante desses medos, “a mente é apressada para fora de si mesma”. Quando experimentamos o sublime, nos sentimos evacuados, sobrecarregados por um objeto externo de tremendo poder e ameaça. Tudo o que nos dava uma sensação de ser interno e vitalidade deixa de existir. O externo é tudo, nós não somos nada. Deus é um bom exemplo, e a expressão máxima, do sublime: “Enquanto contemplamos um objeto tão vasto, debaixo do braço, por assim dizer, de poder todo-poderoso, e investido de todos os lados com onipresença, encolhemos na pequenez de nossa própria natureza, e somos, de certa forma, aniquilados diante dele”.
 {\color{blue} 20}  

 
\par
 
Paradoxalmente, também sentimos a nossa existência de uma forma que nunca sentimos antes. Tomadas pelo terror, a nossa “atenção” é despertada e as nossas “faculdades” são “impulsionadas, por assim dizer, em guarda”. Somos arrancados de nós mesmos. Estamos cientes do terreno imediato e da nossa presença nele. Antes, mal percebíamos a nós mesmos ou ao nosso redor. Agora saímos de nós mesmos, habitando não apenas nossos corpos e mentes, mas também o espaço circundante. Sentimos “uma espécie de inchaço” – uma sensação de que somos maiores, que o nosso perímetro se estende ainda mais – que “é extremamente grato à mente humana”. Mas esse “inchaço”, lembra-nos Burke, “nunca é mais percebido, nem opera com mais força, do que quando, sem perigo, estamos familiarizados com objetos terríveis”.
 {\color{blue} 21}  

 
\par
 
Diante do sublime, o eu sou aniquilado, ocupado, esmagado, oprimido; diante do sublime, o eu sou elevado, engrandecido, ampliado. Se o seu pode realmente ocupar esses polos de experiência opostos, quase irreconciliáveis, ao mesmo tempo - é esta contradição, a oscilação entre extremos selvagens, que gera um sentimento forte e extenuante de seu. Como Burke escreve em outro lugar, a luz intensa se assemelha à escuridão intensa não apenas porque cega os olhos e, portanto, se aproxima da escuridão, mas também porque ambas são extremos. E os extremos, especialmente os extremos opostos, são sublimes porque a sublimidade “em todas as coisas abominam a mediocridade”.
 {\color{blue} 22}  
O extremo das sensações opostas, o balanço selvagem do ser ao nada, contribui para a experiência mais intensa da individualidade.
 
\par
 
A questão para nós, que Burke não coloca nem responde, nem aqui, nem no seu outro trabalho, é: que tipo de forma política implica esta simultaneidade – ou oscilação entre – autoengrandecimento e autoaniquilação? Uma possibilidade seria a hierarquia, com os seus requisitos gêmeos de submissão e dominação; a outra é a violência, particularmente a guerra, com a sua rígida injunção de matar ou ser morto. Talvez não por coincidência, ambos são de grande importância para o conservadorismo como tradição teórica e prática histórica.
 
\par
 
Rousseau e John Adams não costumam ser considerados companheiros ideológicos, mas em um ponto eles concordavam: hierarquias sociais persistem porque garantem que todos, exceto aqueles no fundo, e no topo, aproveitem a oportunidade de governar e serem governados por sua vez. Não, com certeza, no sentido aristotélico de autogoverno, mas no sentido feudal de governança recíproca: cada pessoa domina alguém abaixo dela em troca de se submeter a alguém acima dela. “Os cidadãos só se permitem ser oprimidos enquanto são levados pela ambição cega”, escreve Rousseau. “Como prestam mais atenção ao que está abaixo deles do que ao que está acima, a dominação se torna mais cara para eles do que a independência, e eles consentem em usar correntes para poderem, por sua vez, dá-las a outros. É muito difícil reduzir à obediência qualquer um que não busque comandar.”
 {\color{blue} 23}  
O aspirante e o autoritário não são tipos opostos: a vontade de ascender precede a vontade de curvar-se. Mais de trinta anos depois, Adams escreveria que todo homem deseja “ser observado, considerado, estimado, elogiado, amado e admirado”.
 {\color{blue} 24}  
Para ser elogiado é preciso ser visto, e a melhor forma de ser visto é elevar-se acima do seu círculo. Até mesmo o democrata americano, argumentou Adams, preferiria governar um inferior a desapropriar um superior. Sua paixão é pela supremacia, não pela igualdade, e enquanto lhe for garantida uma audiência inferior, ele ficará satisfeito com seu posição inferior:
 
\par
 

 \textbf{\textit{Not only the poorest mechanic, but the man who lives upon common charity, nay the common beggars in the streets. . . Court a set of admirers, and plume themselves on that superiority which they have, or fancy they have, over some others. . . . When a wretch could no longer attract the notice of a man, woman or child, he must be respectable in the eyes of his dog. “Who will love me then?” was the pathetic reply of one, who starved him-self to feed his mastiff, to a charitable passenger who advised him to kill or sell the animal. {{\color{blue} 25} } } }  
 
 
\par
 
Pode-se ver nestas descrições da hierarquia social os contornos do sublime: aniquilado a partir de cima, engrandecido a partir de baixo, o eu sou ampliado e miniaturizado pelo seu envolvimento, na prática do governo. Mas aqui está o problema: uma vez que temos realmente a certeza do nosso poder sobre outro ser, diz Burke, o nosso inferior perde a sua capacidade de nos prejudicar ou ameaçar. Ela perde sua sublimidade. “Tire” uma criatura “de sua capacidade de ferir” e “você a estragará com tudo que é sublime”.
 {\color{blue} 26}  
Leões, tigres, panteras e rinocerontes são sublimes não porque sejam exemplares magníficos de força, mas porque podem e irão nos matar. Bois, cavalos e cães também são fortes, mas não têm o instinto de matar ou tiveram esse instinto suprimido. Eles podem ser feitos para nos servir e no caso dos cães até nos amar. Como tais criaturas, por mais fortes que sejam, não podem nos ameaçar ou prejudicar, elas são incapazes de sublimidade. Eles são objetos de desprezo, sendo o desprezo “o atendente de uma força que é subserviente e nociva”. 27
 
\par
 

 \textbf{\textit{We have continually about us animals of a strength that is considerable, but not pernicious. Amongst these we never look for the sublime: it comes upon us in the gloomy forest, and in the howling wilderness. . . . Whenever strength is only useful, and employed for our benefit or our pleasure, then it is never sub-lime; for nothing can act agreeably to us, that does not act in} }  
 
 
\par
 

 
\par
 

 \textbf{\textit{Conformity to our will; but to act agreeably to our will, it must be subject to us; and therefore can never be the cause of a grand and commanding conception. {{\color{blue} 28} } } }  
 
 
\par
 
Portanto, pelo menos metade da experiência da hierarquia social – não a experiência de ser governado, que traz a possibilidade de ser destruído, humilhado, ameaçado ou prejudicado por um superior, mas a experiência de governar facilmente outro – é incompatível. Com, e, na verdade enfraquece, o sublime. Confirmados o nosso poder, somos embalados pela mesma facilidade e conforto, passamos pela mesma fusão interior que experimentamos enquanto estamos no auge do prazer. A segurança do governo é tão debilitante quanto a paixão do amor.
 
\par
 
As insinuações de Burke sobre os perigos de um governo há muito estabelecido refletem uma tensão surpreendente dentro do conservadorismo: um desconforto persistente, embora não reconhecido, com o poder que amadureceu e amadureceu, com a autoridade que se tornou confortável e segura. Começando pelo próprio Burke, os conservadores expressaram um profundo desconforto relativamente às classes dominantes tão seguras do seu lugar ao sol que perdem a sua capacidade de governar: a sua vontade de poder dissipa-se; os músculos e a inteligência de seu comando se atenuam.
 
\par
 
Como vimos no capítulo 1, Burke acredita que o Antigo Regime é lindo. Por isso também é “lento, inerte e tímido”. Não pode defender-se “das invasões de capacidade”, com a capacidade a substituir os novos homens de poder que a Revolução revisita. O interesse monetário, também aliado à Revolução, é mais forte do que o interesse fundiário porque está “mais pronto para qualquer aventura” e “mais disposto a novos empreendimentos de qualquer tipo”.
 {\color{blue} 29}  
O Antigo Regime é belo, estático, fraco; a Revolução é feia, dinâmica, forte. “É uma verdade terrível”, admite Burke na segunda das suas Cartas sobre uma Paz Regicida, “mas é uma verdade que não pode ser escondida; em habilidade, em destreza, na distinção de seus pontos de vista, os jacobinos são nossos superiores”.
 {\color{blue} 30}  

 
\par
 
Joseph de Maistre foi menos diplomático do que Burke nas suas condenações do Antigo Regime, talvez porque encarasse as suas falhas de forma mais pessoal. Muito antes da Revolução, afirma ele, a liderança do Antigo Regime estava confusa e desnorteada. Naturalmente, as classes dominantes foram incapazes de compreender, e muito menos de resistir, ao ataque desencadeado contra elas. A impotência, física e cognitiva, foi – e continua sendo – o grande pecado do Antigo Regime. A aristocracia não consegue compreender; não pode agir. Alguma parcela da nobreza pode ser bem-intencionada, mas não consegue levar seus projetos até o fim. Eles são arrogantes e tolos. Eles têm virtude, mas não virtude. A aristocracia “falha ridiculamente em tudo o que empreende”. O clero foi corrompido pela riqueza e pelo luxo. A monarquia tem demonstrado consistentemente que lhe falta a vontade de “punir”, que é a marca registrada de todo verdadeiro soberano.
 {\color{blue} 31}  
Confrontado com tal decadência, o inevitável crescimento de séculos no poder, maître conclui que é bom que a contra-revolução ainda não tenha triunfado (escreveu em 1797). O Antigo Regime precisa de mais alguns anos no deserto se quiser livrar-se das influências corruptoras da sua outrora bela vida:
 
\par
 

 \textbf{\textit{The restoration of the throne would mean a sudden relaxation of the driving force of the state. The black magic working at the moment would disappear like mist before the sun. Kindness, clemency, justice, all the gentle and peaceful virtues, would suddenly reappear and would bring with them a general meekness of character, a certain cheerfulness entirely opposed to the rigors of the revolutionary regime. {{\color{blue} 32} } } }  
 
 
\par
 
Um século mais tarde, um caso semelhante será apresentado por Georges Sorel contra a belle epoquê. Sore geralmente não é visto como uma figura emblemática da direita – por outro lado, até mesmo o conservadorismo de Burke permanece um assunto de disputa
 {\color{blue} 33}  
– E, de facto, a sua maior obra, reflexões sobre a violência, é frequentemente considerada uma contribuição, ainda que menor, para a tradição marxista. No entanto, o início de Sore é conservador e o seu final proto-fascista, e mesmo na sua fase marxista a sua principal preocupação é a decadência e a vitalidade, em vez da exploração e da justiça. As críticas que ele faz às classes dominantes francesas no final do século XIX não são diferentes daquelas feitas por Burke e Maître no final do século XVIII. Ele até torna a comparação explícita: a burguesia francesa, escreve Sore, “tornou-se quase tão estúpida quanto a nobreza do século XVIII”. Eles são “uma aristocracia ultracivilizada que exige ser deixada em paz”. Antigamente, a burguesia era uma raça de guerreiros. “Capitães ousados”, eram “criadores de novas indústrias” e “descobridores de terras desconhecidas”. Eles “dirigiram empresas gigantescas”, inspirados por aquele “espírito conquistador, insaciável e impiedoso” que construiu ferrovias, subjugou continentes e criou uma economia mundial. Hoje, são tímidos e cobardes, recusando-se a tomar as medidas mais elementares para defender os seus próprios interesses contra os sindicatos, os socialistas e a esquerda. Em vez de desencadearem violência contra os trabalhadores em greve, rendem-se à ameaça de violência dos trabalhadores. Falta-lhes o ardor, o fogo na barriga, dos seus antepassados. É difícil não concluir que “a burguesia está condenada à morte e o seu desaparecimento é apenas uma questão de tempo”.
 {\color{blue} 34}  

 
\par
 
Carl Schmitt formalizou o desprezo de Sore pelas fraquezas das classes dominantes numa teoria inteira da política. De acordo com Schmitt, o burguês era como ele era – avesso ao risco, egoísta, desinteressado na bravura ou na morte violenta, desejoso de paz e segurança – porque o capitalismo era a sua vocação e o liberalismo a sua fé. Nenhum dos dois lhe deu um bom motivo para morrer pelo Estado. Na verdade, ambos lhe deram boas razões, na verdade, todo um vocabulário, para não morrer pelo Estado. Juros, liberdade, lucro, direitos, propriedade, individualismo e outras palavras semelhantes criaram uma das classes dominantes mais egocêntricas da história, uma classe que gozava de privilégios, mas não se sentia obrigada a defender esse privilégio. Afinal, a premissa da democracia liberal era a separação entre a política, a economia e a cultura. Poderíamos buscar o lucro, às custas de outrem, e pensar livremente, por mais subversivos que sejam os pensamentos, sem perturbar o equilíbrio de poder. A burguesia, no entanto, enfrentava um inimigo que compreendia perfeitamente as ligações entre ideias, dinheiro e poder, que os arranjos econômicos e os argumentos intelectuais eram a matéria-prima do combate político. Os marxistas obtiveram a distinção amigo-inimigo, que é constitutiva da política; a burguesia não.
 {\color{blue} 35}  
O espírito de Hegel residia em Berlim; há muito que “vagou para Moscou”.
 {\color{blue} 36}  

 
\par
 
Sore identificou uma excepção a esta regra da decadência capitalista: os barões ladrões dos Estados Unidos. Nos Carnegie's e nos Gould's da indústria americana, sore pensou ter visto “a energia indomável, a audácia baseada numa avaliação precisa da força, o cálculo frio dos interesses, as qualidades dos grandes generais e dos grandes capitalistas”. Ao contrário da burguesia mimada da França e da Grã-Bretanha, os milionários de Pitts-burgo e Pittston “levam até ao fim das suas vidas uma existência de escravos nas galés, sem nunca pensarem em levar uma vida de nobre, como fazem os Rothschild”.
 {\color{blue} 37}  

 
\par
 
O homólogo espiritual de Sore do outro lado do Atlântico, Teddy Roosevelt, não estava tão optimista em relação aos industriais americanos e aos anos fi Nan. (A ansiedade de Burke em relação às classes dominantes é comum aos conservadores europeus e americanos.) O capitalista, declarou Roosevelt, vê o seu país como uma “caixa”, sempre pesando a “honra da nação e a glória da bandeira” contra uma “interrupção temporária de ganhar dinheiro”. Ele não está “disposto a dar a vida por pequenas coisas” como a defesa da nação. Ele se preocupa “apenas se o valor das ações sobe ou desce”.
 {\color{blue} 38}  
Ele não demonstra interesse em grandes assuntos de Estado, nacionais ou internacionais, a menos que estes incidam sobre os seus. Não foi por acaso, afirmou Roosevelt, talvez com um aceno para Carnegie, que tais homens se opuseram à grande expedição imperial a Guerra Hispano-Americana.
 {\color{blue} 39}  
Complacentes e confortáveis, seguros das suas riquezas pelo sucesso das guerras laborais das décadas anteriores e da eleição de 1896, estes não eram homens com quem se pudesse contar para defender a nação ou mesmo a si próprios. “Poderemos algum dia ter uma causa amarga”, declarou Roosevelt, “para perceber que uma nação rica que é preguiçosa, tímida ou pesada é uma presa fácil” para outros povos mais marciais. O perigo que enfrenta uma classe dominante, e uma nação dominante, que se tornou “qualificada no comércio e nas finanças” é que “perda as duras virtudes de combate”.
 {\color{blue} 40}  

 
\par
 
Roosevelt não foi o primeiro conservador americano a preocupar-se com o abrandamento das classes dominantes e com as hierarquias repletas de poder. Nem ele seria o último. Ao longo da década de 1830, vimos no capítulo 1, à medida que os abolicionistas começaram a pressionar a sua causa, John C. Calhoun ficou furioso com a vida fácil e a ignorância voluntária dos seus camaradas na plantação. Eles haviam se tornado preguiçosos, gordos e complacentes, desfrutando tão plenamente dos privilégios de sua posição que não conseguiam prever a catástrofe que se aproximava. Ou, se pudessem, os proprietários do Sul não poderiam fazer nada para a defender, uma vez que os seus músculos políticos e ideológicos já tinham atrofiado há muito tempo.
 {\color{blue} 41}  
Barry Goldwater também expressou desprezo pelo establishment republicano.
 {\color{blue} 42}  
E ao longo da década de 1990 – para avançar mais três décadas – podíamos ouvir os herdeiros de Roosevelt à direita dirigirem o mesmo veneno contra o capitalista americano contra os mestres do universo em Wall Street e os empresários geeks de Silicon Valley.
 {\color{blue} 43}  

 
\par
 
Para que a classe dominante seja vigorosa e robusta, concluiu o conservador, os seus membros devem ser testados, exercitados e desafiados. Não apenas seus corpos, mas também suas mentes, até mesmo suas almas. Ecoando Milton: “Não posso elogiar uma virtude fugitiva e enclausurada, não exercida e sem fôlego, que nunca sai e vê seu adversário, mas foge da corrida. . . . Aquilo que nos purifica é a prova, e a prova é pelo que é contrário”
 {\color{blue} 44}  
— Burke acredita que a adversidade e a dificuldade, o confronto com a aflição e o sofrimento, tornam seres mais fortes e virtuosos.
 
\par
 

 \textbf{\textit{The great virtues turn principally on dangers, punishments, and troubles, and are exercised rather in preventing mischiefs, than in dispensing favors; and are therefore not lovely, though highly venerable. The subordinate turn on reliefs, gratify cations, and indulgences; and are therefore more lovely, though inferior in dignity. Those persons who creep into the hearts of most people, who are chosen as the companions of their softer hours, and their reliefs from care and anxiety, are never persons of shining qualities, nor strong virtues. {{\color{blue} 45} } } }  
 
 
\par
 
Talvez vejamos aqui as origens da preferência conservadora pela guerra em detrimento do Estado de bem-estar social, mas isso é outro assunto para outro dia). Mas enquanto Milton e outros republicanos com ideias semelhantes acreditam que a impureza e a corrupção aguardam os complacentes e confortáveis, Burke espia o espectro mais aterrorizante da dissipação, da degeneração e da morte. Se quisermos que os poderosos permaneçam poderosos, se quisermos continuar vivos, o seu poder, e, na verdade a credibilidade da sua própria existência, deve ser continuamente desafiado, ameaçado e defendido.
 
\par
 
Uma das características mais impressionantes – embora espero que já sejam inteligíveis – do discurso conservador é o fascínio, e, na verdade apreço, que se encontra pelos inimigos do conservador, particularmente pelo uso da violência contra ele e os seus aliados. Os comentários mais arrebatadores de Maître nas suas Considerações sobre a França são reservados aos jacobinos, cuja vontade brutal e propensão para a violência – a sua “magia negra” – ele claramente inveja. Graças aos seus esforços, a França foi purificada e restaurada no seu legítimo lugar de destaque entre a família das nações. Eles reuniram o povo contra os invasores estrangeiros, um “prodígio” que “só o gênio infernal de Robespierre poderia realizar”. Ao contrário da monarquia, a Revolução tem vontade de punir.
 {\color{blue} 46}  

 
\par
 
Da perspectiva do sublime de Burke, contudo, o argumento de Moistre só vai até certo ponto. A Revolução rejuvenesce o Antigo Regime, forçando-o a sair do poder e purificando o povo através da violência. Ele proporciona um choque esclarecedor ao sistema. Mas Maître nunca contempla, ou pelo menos nunca discute, o efeito revivificador que a recuperação do poder da Revolução poderia ter sobre os líderes do Antigo Regime. E, de facto, quando ele consegue descrever como pensa que a contra-revolução irá ocorrer, a batalha final revela-se um ar AFF espantosamente anticlimático, com quase nenhum tiro disparado. “Como acontecerá a contra-revolução se vier?” Maître pergunta. “Quatro ou cinco pessoas, talvez, darão um rei à França.” Não é exatamente o material de uma classe dominante viril e transformada, lutando para voltar ao poder.
 {\color{blue} 47}  

 
\par
 
Maître nunca contemplou as possibilidades restaurativas do combate corpo a corpo entre o Antigo Regime e a Revolução; para isso é preciso recorrer a Sore. E embora as lealdades de Sore na guerra entre os governantes e os governados do final do século XIX sejam mais ambíguas do que as de Maître, o seu relato do efeito da violência dos governados sobre os governantes não o é. A burguesia francesa perdeu o seu espírito de luta, afirma Sore, mas esse espírito está vivo e bem entre os trabalhadores. O seu campo de batalha é o local de trabalho, a sua arma é a greve geral e o seu objectivo é a derrubada do Estado. É o último que mais impressiona Sore, pois o desejo de derrubar o Estado sinaliza quão despreocupados os trabalhadores estão com “os lucros materiais da conquista”. Não só não procuram salários mais elevados e outras melhorias no seu bem-estar; em vez disso, concentraram-se no objectivo mais improvável: derrubar o Estado por uma greve geral. É essa improbabilidade, a distância entre meios e fins, que torna a violência do proletariado tão gloriosa. Os proletários são como guerreiros homéricos, absortos na grandeza da batalha e indiferentes aos objectivo da guerra: Quem realmente alguma vez derrubou um Estado por uma greve geral? A violência deles é uma violência por si só, sem preocupação com custos, benefícios e cálculos intermediários.
 {\color{blue} 48}  
Como Ernst Jünger escreveu uma geração mais tarde, “não é aquilo por que lutamos, mas como lutamos”.
 {\color{blue} 49}  

 
\par
 
Mas o que prende Sore não é o proletariado, mas os efeitos rejuvenescedores que este poderá ter sobre a burguesia. Poderá a violência da greve geral “devolver à burguesia um ardor que se extinguiu?” Certamente o vigor do proletariado poderá despertar novamente a burguesia para os seus próprios interesses e para as ameaças que a sua retirada da política representa para esses interesses. Mais tentadora para Sore, no entanto, é a possibilidade de a violência dos trabalhadores “restaurar [à burguesia] as qualidades bélicas que anteriormente possuía”, forçando a “classe capitalista a permanecer ardente na luta industrial”. Através da luta contra o proletariado, por outras palavras, a burguesia pode recuperar a sua ferocidade e o seu ardor. E o ardor é tudo. Somente do ardor, daquela esplêndida indiferença à razão e ao interesse próprio, uma civilização inteira, afogada no materialismo e na complacência, será despertada. Uma classe dominante, ameaçada pela violência dos governados, despertada pelo seu próprio gosto pela violência – essa é a promessa da guerra civil em França.
 {\color{blue} 50}  

 
\par
 
Para o conservador, por mais modulado ou moderado que seja, um vigor renovado sempre foi a promessa da guerra civil. Pois entre os casos fáceis de um reacionário católico como Maître e de um protofascista como Sore esta o exemplo mais difícil, mas em última análise mais revelador, de Alexis de Tocqueville. A sua passagem da moderação da Monarquia de julho para o revanchismo de 1848 demonstra quão fácil e inexoravelmente o Burke, um conservador, oscilará do belo para o sublime, como a música da prudência e da moderação dá lugar à marcha da violência e do vitríolo.
 {\color{blue} 51}  

 
\par
 
Apresentando-se publicamente como um realista consumado, discriminador e criterioso, com pouca paciência para qualquer tipo de entusiasmo, Tocqueville era, na verdade um romântico enrustido. Ele confessou ao irmão que compartilhava da “impaciência devoradora” do pai, de sua “necessidade de sensações vivas e recorrentes”. A razão, disse ele, “sempre foi para mim como uma gaiola”, atrás da qual ele “rangeria os dentes”. Ele ansiava “pela visão do combate”. Relembrando a Revolução Francesa, da qual perdeu (nasceu em 1805), lamentou o fim do Terror, afirmando que “os homens assim esmagados não só já não conseguem atingir grandes virtudes, mas parecem ter-se tornado quase incapazes de grandes crimes.” Até Napoleão, flagelo dos conservadores, moderados e liberais em todo o mundo, conquistou a admiração de Tocqueville como o “ser mais extraordinário que apareceu no mundo durante muitos séculos”. Quem, pelo contrário, poderia encontrar inspiração na política parlamentar da Monarquia de julho, aquela “pequena panela de sopa democrática e burguesa”? No entanto, uma vez iniciado uma carreira na política, foi nessa pequena panela de sopa burguesa que Tocqueville saltou. . Previsivelmente, não era do seu gosto. Tocqueville pode ter proferido palavras de moderação, compromisso e Estado de Direito, mas elas não o comoveram. Sem a ameaça da violência revolucionária, a política simplesmente não era o grande drama que ele imaginava que tivesse sido entre 1789 e 1815. “Os nossos pais observaram coisas tão extraordinárias que, comparadas com elas, todas as nossas obras parecem comuns.” A política de moderação e compromisso produziu moderação e compromisso; não produziu política, pelo menos não na forma como Tocqueville entendia o termo. Durante as décadas de 1830 e 1840, “o que mais faltava. . . Foi a própria vida política.” Não havia “nenhum campo de batalha para as partes em conflito se encontrarem”. A política tinha sido “privada” de “toda originalidade, de toda realidade e, portanto, de todas as paixões genuínas”. Então veio 1848. Tocqueville não apoiou a Revolução. Na verdade, ele estava entre os seus oponentes mais veementes. Votou a favor da suspensão total das liberdades civis, o que anunciou com alegria feito “com ainda mais energia do que tinha sido feito sob a Monarquia”. Ele acolheu bem os rumores de uma ditadura – para proteger o mesmo regime que passou a maioria de duas décadas menosprezando. E ele adorou tudo: a violência, a contra violência, a batalha. Defendendo a moderação contra o radicalismo, foi dada a Tocqueville a oportunidade de utilizar meios radicais para fins moderados, e não está totalmente claro qual dos dois mais o comoveu.
 
\par
 

 \textbf{\textit{Let me say, then, that when I came to search carefully into the depths of my own heart, I discovered, with some surprise, a certain sense of relief, a sort of gladness mingled with all the griefs and fears to which the Revolution had given rise. I suffered from this terrible event for my country, but clearly not for myself; on the contrary, I seemed to breathe more freely than before the catastrophe. I had always felt myself stifled in the atmosphere of the parliamentary world which had just been destroyed: I had found it full of disappointments, both where others and where I myself was concerned.} }  
 
 
\par
 
Autoproclamado poeta do hesitante, do sutil e do complexo, Tocqueville ardeu de entusiasmo ao acordar para um mundo dividido em dois campos. Parlamentos tímidos semearam uma confusão cinzenta; a guerra civil forçou à nação uma clareza revigorante de preto e branco. “Não sobrou campo para a incerteza mental: deste lado estava a salvação do país; nisso, sua destruição. . . . A estrada parecia perigosa, é verdade, mas minha mente está construída de tal forma que tem menos medo do perigo do que da dúvida.” Para este membro da classe dominante, a sublimidade brota da violência das classes inferiores e uma oportunidade de escapar à beleza sufocante da vida no Parnaso burguês.
 
\par
 
Francis Fukuyama é talvez o mais ponderado dos escritores recentes a seguir esta linha conservadora de argumento sobre a violência. Ao contrário de Maître, porém de Tocqueville e Sore – todos os quais escreveram no meio da batalha, quando o resultado não era claro – Fukuyama escreve da perspectiva da vitória. Estamos em 1992 e as classes capitalistas derrotaram os seus oponentes socialistas na longa guerra civil do curto século XX. Não é uma visão bonita, pelo menos não para Fukuyama. Pois o revolucionário foi um dos poucos homens temáticos do século XX. O homem temático é como o trabalhador de Sore: aquele que arrisca a vida em prol de um princípio improvável, que não se preocupa com os próprios interesses materiais e se preocupa apenas com a honra, a glória e os valores pelos quais luta. Depois de uma estranha, mas breve homenagem aos Bloods ano lhe Crops como homens temáticos, Fukuyama olha com carinho para homens de propósito e poder como Lenin, Trotsky e Stalin, “lutando por algo mais puro e superior” e possuidores de “dureza maior do que o normal”.  visão, crueldade e inteligência. Em virtude da sua recusa em acomodar-se à realidade do seu tempo, eles eram os “mais livres e, portanto, os mais humanos dos seres”. Mas, garantidamente, estes homens e os seus sucessores perderam a guerra civil do século XX, quase inexplicavelmente, para as forças do “Homem Econômico”. Pois o Homem Econômico é “o verdadeiro burguês”. Tal homem nunca estaria “disposto a andar na frente de um tanque ou confrontar uma fila de soldados” por qualquer causa, mesmo a sua própria. No entanto, o Homem Econômico é o vencedor e, longe de o rejuvenescer ou restaurar os seus poderes primordiais, a guerra parece apenas tê-lo tornado mais burguês. Conservador como é, Fukuyama só pode irritar-se com o triunfo do Homem Econômico e com “a vida de consumo racional” que ele trouxe, uma vida que é “no final das contas, chata”.
 {\color{blue} 52}  

 
\par
 
Longe de ser excepcional, a decepção de Fukuyama sobre o efeito real – em oposição ao efeito antecipado ou fantasiado – da violência sobre uma classe dominante dissipada é emblemática. “Os objetivos da batalha e os frutos da conquista nunca são os mesmos”, observou E. M. Forster em A Passage to India. “Estes últimos têm o seu valor e só o santo os rejeita, mas o seu indício de imortalidade desaparece assim que são segurados nas mãos.”
 {\color{blue} 53}  
Nas profundezas do discurso conservador esconde-se um elemento de anticlímax que não pode ser contido. Embora o conservador se volte para a violência como forma de se libertar, ou das classes dominantes, do tédio mortal e da atrofia suavizante que acompanha o poder, praticamente todos os encontros no discurso conservador com a violência real implicam desilusão e deflação.
 
\par
 
Lembremo-nos de Teddy Roosevelt, meditando sobre o materialismo e a fraqueza das classes capitalistas da América. Onde, perguntou-se ele, se poderia encontrar um exemplo da “vida extenuante” – a emoção da dificuldade e do perigo, a luta que levou ao progresso – na América contemporânea? Talvez nas guerras e conquistas estrangeiras que a América empreendeu no final do século. No entanto, mesmo aqui Roosevelt encontrou frustração. Embora os seus relatórios sobre a Guerra Hispano-Americana estivessem repletos de bravura e bravata, uma leitura cuidadosa das suas aventuras em Cuba sugere que as suas façanhas naquele país foram um fiasco. Cada uma das famosas investidas que Roosevelt conduziu para cima ou para baixo de uma colina foi um anticlímax. A primeira culminou com ele vendo exatamente dois soldados espanhóis abatidos pelos seus homens: “Estes foram os únicos espanhóis que realmente vi serem atingidos por tiros certeiros de qualquer um dos meus homens”, escreveu ele, “exceto dois guerrilheiros nas árvores”. A segunda o encontrou liderando um exército que não o ouviu nem o seguiu. Por isso, foi com uma apreciação sombria que ele recitou os comentários dispépticos de um dos líderes do exército em Cuba, um certo General Wheeler, que “tinha passado por demasiados combates pesados ​​na Guerra Civil para considerar a luta atual como muito séria”.
 {\color{blue} 54}  

 
\par
 
Nas ocupações sangrentas que se seguiram à Guerra Hispano-Americana, porém, Roosevelt pensou ter visto a verdadeira felicidade que era estar vivo naquela madrugada. Roosevelt tinha a certeza de que as ocupações americanas nas Filipinas e noutros locais estavam tão próximas de uma repetição da Guerra Civil – aquela nobre cruzada de virtude imaculada – como ele e os seus compatriotas provavelmente veriam. “Nós, desta geração, não temos de enfrentar uma tarefa como a que nossos pais enfrentaram”, declarou ele em 1899, “e aí de nós se não conseguirmos realizá-la! . . . Não podemos evitar as responsabilidades que enfrentamos no Havaí, em Cuba, em Porto [assim usado] Rico e nas Filipinas.” Aqui – nas ilhas do Caribe e do Pacífico – estava a confluência de sangue e propósito que ele buscou durante toda a sua vida. A tarefa de elevação imperial, de educar os nativos na “causa da civilização”, foi árdua e violenta, impondo à América uma missão que levaria anos, se Deus quisesse, para ser cumprida. Se a missão imperial fosse bem sucedida – e mesmo que fracassasse – criaria uma classe dominante genuína na América, endurecida e tornada extenuante pela batalha, mais nobre e menos suja de espírito do que os asseclas de Carnegie.
 {\color{blue} 55}  

 
\par
 
Foi um sonho lindo. Mas também não suportou o peso da realidade. Embora Roosevelt esperasse que os homens que governavam as Filipinas fossem “escolhidos pela sua capacidade e integridade de sinal”, governando “as províncias em nome de toda a nação de onde vêm, e pelo bem de todo o povo para onde vão,” ele temia que os ocupantes coloniais da América viessem da mesma classe de financistas e industriais egoístas que o levaram para o estrangeiro em primeiro lugar. E assim os seus louvores ao imperialismo terminaram com uma nota amarga de advertência, até mesmo de condenação. “Se permitirmos que o nosso serviço público nas Filipinas se torne presa dos despojos políticos, se não conseguirmos mantê-lo ao mais alto padrão, seremos culpados de um alto, não só de maldade, mas de fraqueza e falta de poder. Loucura avistada, e teremos começado a trilhar o caminho trilhado pela Espanha para sua própria humilhação amarga.
 {\color{blue} 56}  

 
\par
 
Mas se o seu sonho terminasse mal, Roosevelt pelo menos tinha a vantagem de poder dizer que sempre suspeitou que isso aconteceria. O mesmo não se pode dizer dos fascistas de Itália, cujo autoengano sobre a arrancada do poder à esquerda persistiu durante décadas, testemunhando uma incapacidade de enfrentar a sua própria decepção. Durante anos, os fascistas celebraram a Marcha sobre Roma de 1922 como o violento e glorioso triunfo da vontade sobre a adversidade. 28 de outubro, dia da chegada dos Camisas Negras a Roma, tornou-se feriado nacional; foi declarado o primeiro dia do Ano Novo Fascista após a introdução do novo calendário em 1927. A história da chegada de Mussolini em particular – vestindo a proverbial camisa preta – foi repetida com admiração. “Senhor”, ele supostamente disse ao rei Victor Emmanuel III, “perdoe meu traje. Eu venho dos campos de batalha.” Na verdade, Mussolini viajou durante a noite de trem desde Milão, onde frequentava continuamente o teatro, cochilando confortavelmente no vagão-leito. A única razão pela qual conseguiu chegar a Roma foi que um tímido establishment, liderado pelo rei, telefonou-lhe para Milão com um pedido para que formasse um governo. Quase nenhum tiro foi vermelho, de cada lado.
 {\color{blue} 57}  
Maître não poderia ter escrito melhor.
 
\par
 
Podemos ver um fenômeno semelhante em jogo na guerra contra o terrorismo. Embora muitos vejam a administração Bush e o neoconservadorismo como desvios do conservadorismo propriamente dito – a declaração mais recente desta tese é The Death of. Conservatism, de Sam Tannhauser.
 {\color{blue} 58}  
— o projeto neoconservador de aventureirismo imperial traça no Burke um arco de violência do princípio ao fim. Já discuti, no capítulo 8, como os neoconservadores encararam o 11 de setembro e a guerra contra o terrorismo como uma oportunidade para escapar à paz e prosperidade decadentes e mortíferas dos anos Clinton, que eles acreditavam terem enfraquecido a sociedade americana. Esbanjando conforto, os americanos – e mais importante, os seus líderes – supostamente perderam a vontade, o desejo e a capacidade de governar o mundo. Então aconteceu o 11 de setembro e, de repente, parecia que sim.
 
\par
 
Esse sonho, claro, está agora em frangalhos, mas vale a pena notar um dos seus aspectos mais idiossincráticos, pois apresenta uma ruga na longa saga da violência conservadora. De acordo com muitos conservadores, e não apenas os neoconservadores, uma das fontes recentes da decadência americana, que remonta ao Tribunal Warren e às revoluções pelos direitos da década de 1960, é a obsessão liberal com o Estado de direito. Esta obsessão, aos olhos dos conservadores, assume muitas formas: a insistência no devido processo no processo penal; uma parcialidade em litígios sobre legislação; uma ênfase na diplomacia e no direito internacional sobre a guerra; tentativas de restringir o poder executivo através da supervisão judicial e legislativa. Por mais não relacionados que estes sintomas possam parecer, os conservadores beém neles uma única doença: uma cultura de regras e leis que lentamente incapacita e desvitaliza a fera loira de rapina que é o poder americano. Estes são sinais de uma insalubridade de Nietzsche, e o 11 de setembro foi o resultado inevitável.
 
\par
 
Se quisermos evitar outro 11 de setembro, essa cultura de direitos e regras deve ser repudiada e revertida. Como deixam claro os relatórios de Seymour Hersh e Jane Mayer, a guerra ao terror – com o seu impulso à tortura, à derrubada das Convenções de Genebra, à recusa das restrições do direito internacional, à vigilância ilegal e à visão do terrorismo através das lentes de guerra em vez de crime e punição – reflete tanto, se não mais, estas sensibilidades e sensibilidades conservadoras como os factos do 11 de setembro e a necessidade de evitar outro ataque.
 {\color{blue} 59}  
“Ela é mole – mole demais”, diz o agora aposentado tenente-general Jerry Bodkin sobre os Estados Unidos, antes e depois do 11 de setembro. A forma de a tornar dura não é apenas empreender ações militares difíceis e extenuantes, mas também violar as regras – e a cultura de regras – que a tornaram branda em primeiro lugar. Os Estados Unidos têm de aprender a “viver no limite”, afirma o ex-diretor da NSA Michael Hayden. “Não há nada que não façamos, nada que não tentemos”, acrescenta o ex-diretor da CIA, George Tenet.
 {\color{blue} 60}  

 
\par
 
A grande ironia da guerra ao terror é que, longe de emancipar a fera loura de rapina, a guerra tornou a lei e os advogados muito mais críticos do que se poderia imaginar. Como relata Mayer, o impulso à tortura, ao poder executivo desenfreado, à derrubada das Convenções de Genebra, e assim por diante, não veio da CIA ou dos militares; as forças motrizes eram advogados da Casa Branca e do Departamento de Justiça, como David Addington e John Yew. Longe de virtuosos maquiavélicos da violência transgressora, Addington e Yew são fanáticos pela lei e insistem em justificar a sua violência através da lei. Além disso, os advogados supervisionam consistentemente a prática real da tortura. Como escreveu Tenet nas suas memórias: “Apesar do que Hollywood possa fazer-nos acreditar, em situações como esta [a captura, interrogatório e tortura do chefe de logística da Al-Qaeda, Abu Zubaydah], não se chamam os valentões; você chama os advogados. Cada tapa na cara, cada soco no estômago, cada sacudida do corpo – e muito, muito pior – devem primeiro ser aprovados pelos superiores das diversas agências de inteligência, inevitavelmente em consulta com advogados. Mayer compara a prática da tortura a um jogo de “Mãe, posso?” Como afirma um interrogador: “Antes que você pudesse colocar a mão nele [a vítima de tortura], você tinha que enviar um telegrama dizendo: ‘Ele não coopera. Solicite permissão para fazer X.’ E a permissão viria, dizendo ‘Você pode dar um tapa na barriga dele uma vez com a mão aberta.
 {\color{blue} 61}  

 
\par
 
Em vez de libertar a fera loira para vaguear e atacar como quiser, a remoção da proibição da tortura e a suspensão das Convenções de Genebra deixaram-no, ou pelo menos aos advogados que o controlam, mais ansiosos. Até onde ele pode ir? O que ele pode fazer? Cada ato de violência, como revela esta conversa entre dois advogados do Pentágono, torna-se um seminário de faculdade de direito:
 
\par
 

 \textbf{\textit{What did “deprivation of light and auditory stimuli” mean? Could a prisoner be locked in a completely dark cell? If so, could he be kept there for a month? Longer? Until he went blind? What, precisely, did the authority to exploit phobias permit? Could a} }  
 
 
\par
 

 
\par
 

 \textbf{\textit{Detainee be held in a coffin? What about using dogs? Rats? How far could an interrogator push this? Until a man went insane? 62} }  
 
 
\par
 
Depois, há a questão da combinação de técnicas de tortura aprovadas. Um interrogador pode negar comida ao prisioneiro e, ao mesmo tempo diminuir a temperatura de sua cela? O efeito multiplicador das dores duplicadas e triplicadas cruza uma linha nunca definida?
 {\color{blue} 63}  
Como ensinou Orwell, as possibilidades de crueldade e violência são tão ilimitadas quanto a imaginação que as imagina. Mas os exércitos e agências da violência de hoje são vastas burocracias, e vastas burocracias precisam de regras. Eliminar as regras não desvincula Prometeu; isso apenas gera mais horas faturáveis. “Sem cedência. Sem equívocos. Nada de colocar essa coisa em camadas até a morte. Essa foi a promessa de George W. Bush após o 11 de setembro e a sua descrição de como a guerra contra o terrorismo seria conduzida. Tal como tantas outras declarações de Bush, acabou por ser uma promessa vazia. Essa coisa foi mergulhada até a morte. Mas, e este é o ponto crítico, longe de minimizar a violência estatal – que era o grande medo dos neoconservadores – a estratificação provou ser perfeitamente compatível com a violência. Numa guerra já repleta de decepções e desilusões, a constatação que inevitavelmente se segue – o Estado de direito pode, de facto, autorizar as maiores aventuras de violência e morte, drenando-as assim de sublimidade – deve ser, para o conservador, a maior desilusão. de tudo.
 
\par
 
Se tivessem sido leitores mais atentos de Burke, os neoconservadores – como Fukuyama, Roosevelt, Sore, Schmitt, Tocqueville, Maître, Tre-Ritchie e tantos outros na direita americana e europeia – poderiam ter previsto esta desilusão. Burke certamente fez isso. Mesmo enquanto escrevia sobre os efeitos sublimes da dor e do perigo, ele teve o cuidado de insistir que se essas dores e perigos “se aproximassem demais” ou “muito perto” - isto é, se se tornassem realidades em vez de fantasias, deveriam se tornar“ familiarizados com a atual destruição da pessoa” – sua sublimidade desapareceria. Eles deixariam de ser “encantadores” e restauradores e se tornariam simplesmente terríveis.
 {\color{blue} 64}  
O argumento de Burke não era apenas que, no final das contas, ninguém realmente quer morrer ou que ninguém desfruta de uma dor indesejável e excruciante. Foi aquela sublimidade de qualquer tipo e fonte que depende da obscuridade: chegue muito perto de qualquer coisa, seja um objeto ou uma experiência, veja e sinta toda a sua extensão, e isso perderá seu mistério e sua aura. Torna-se familiar. Uma “grande clareza” do tipo que vem da experiência direta “é de alguma forma inimiga de todos os entusiasmos”.
 {\color{blue} 65}  
“É a nossa ignorância das coisas que causa toda a nossa admiração e excita principalmente as nossas paixões. O conhecimento e a familiaridade fazem com que as causas mais marcantes afetem pouco.”
 {\color{blue} 66}  
“Uma ideia clara”, conclui Burke, “é, portanto, outro nome para uma pequena ideia”.
 {\color{blue} 67}  
Conheça muito bem qualquer coisa, inclusive a violência, e ela perderá qualquer atributo – rejuvenescimento, transgressão, excitação, admiração – que você atribuiu a ela quando era apenas uma ideia. Mais cedo do que a maioria, Burke compreendeu que, se a violência quisesse manter a sua sublimidade, teria de continuar a ser uma possibilidade, um objeto de fantasia – um filme de terror, um videogame, um ensaio sobre a guerra. Pois a realidade (em oposição à representação) da violência estava em desacordo com as exigências da sublimidade. A violência real, em oposição à imaginária, implicava que os objetos se aproximassem demasiado, os corpos se aproximassem demasiado, carne sobre carne. A violência despojou o corpo dos véus; a violência tornou seus antagonistas familiares entre si de uma forma que nunca haviam acontecido antes. A violência dissipou a ilusão e o mistério, tornando as coisas monótonas e sombrias. É por isso que, em sua discussão nas Reflexões sobre o sequestro de Maria Antonieta pelos revolucionários, Burke se esforça tanto para enfatizar seu corpo “quase nu” e se volta tão facilmente para a linguagem do vestuário – “o drapeado decente da vida”, o “guarda-roupa da imaginação moral”, “moda antiquada” e assim por diante – para descrever o evento.
 {\color{blue} 68}  
O desastre da violência dos revolucionários, para Burke, não foi crueldade; foi a iluminação não desejada.
 
\par
 
Desde o 11 de setembro, muitos se queixaram, e com razão, do fracasso dos conservadores – ou dos seus filhos e filhas – em lutarem eles próprios na guerra contra o terrorismo. Para os da esquerda, esse fracasso é sintomático da injustiça de classe da América contemporânea. Mas há um elemento adicional na história. Enquanto a guerra contra o terrorismo continuar a ser uma ideia – um tema quente nos blogs, um artigo de opinião provocativo, um episódio de
 {\color{blue} 24}  
– É sublime. Assim que a guerra contra o terrorismo se tornar realidade, pode ser tão triste como uma discussão sobre o código fiscal e tão tediosa como uma visita ao DMV.
 
\par
  
 
999999
