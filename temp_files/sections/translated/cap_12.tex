\chapter{11 Capital do comerciante}\label{11 Capital do comerciante}
 \par 
Este capítulo e o próximo descrevem a teoria do capital de Marx dentro da esfera da troca. Nos capítulos anteriores, o foco tem sido principalmente no papel do capital na produção de mais-valia, com a troca como um complemento necessário, mas dificilmente explorado. No entanto, a análise de lucros, juros e crises requer estudo detalhado da atividade capitalista além da produção, mas em estreita relação com os tópicos de estudo anteriores. Este capítulo explica a categoria de capital mercantil. O Capítulo {\color{blue}12} investiga o capital portador de juros.
 \par 
\section{A categoria de capital mercantil de Marx}
 \par 
Um dos temas que permeiam o tratamento que Marx dá ao capital em troca é que há uma distinção crucial a ser feita entre dinheiro como dinheiro e dinheiro como capital (ver Capítulos {\color{blue}4} e {\color{blue}12}). O dinheiro funciona como dinheiro quando actua como meio de troca entre dois agentes, mediando a troca de mercadorias, independentemente da posição desses agentes na circulação do capital - sejam eles capitalistas envolvidos na produção ou capitalistas e trabalhadores envolvidos no consumo. Conseqüentemente, o papel do dinheiro como dinheiro é entendido por referência à simples circulação de mercadorias, C - D - C. Por outro lado, o dinheiro como capital é entendido por referência ao circuito do capital, D - C… P … C' - D' , onde o dinheiro é empregado com o propósito específico de produzir mais-valia.
 \par 
Existe uma relação definida entre as duas funções do dinheiro no capitalismo, uma vez que a simples circulação de mercadorias e a produção industrial estão intimamente ligadas. Por exemplo, um trabalhador vende força de trabalho e compra uma bicicleta. Isto tem a forma de circulação simples de mercadorias, C - M - C. Ambas as fases de C - M - C, nomeadamente C - M e M - C, estão presentes se vistas do ponto de vista do trabalhador. Mas do ponto de vista dos capitalistas, C - M - C é o contrário, com primeiro a venda da bicicleta, C - M, e depois a compra da força de trabalho, D - C. O que é C - M para um? agente é M - C para outro. Além disso, a utilização do dinheiro tanto como dinheiro como como capital pode envolver relações de crédito, uma vez que o dinheiro é emprestado e emprestado para facilitar os actos de troca. No seu tratamento do capital mercantil, Marx analisa detalhadamente a operação do dinheiro como dinheiro.
 \par 
O tratamento dado por Marx ao capital mercantil é abstrato. Embora a produção e o comércio capitalistas estejam intimamente interligados, são estruturalmente distintos, e Marx identifica uma tendência para a separação destas atividades na economia. Esta tendência real deve ser captada na teoria, a fim de compreender a natureza específica do capital mercantil, que se dirige apenas à realização de trocas.
 \par 
Além de distinguir entre o capital industrial, que produz mais-valia, e o capital mercantil, que o faz circular e facilita a transição entre as formas de capital mercadoria e dinheiro (aumentando indiretamente a massa de mais-valia produzida pelo capital industrial), Marx aponta que o capital mercantil o próprio capital tende a ser dividido em duas formas: capital comercial (compra e venda de mercadorias) e capital de negociação de dinheiro, ou MDC (manuseio de dinheiro).
 \par 
Com o desenvolvimento da produção, os atos de comprar e vender tornam-se tarefas especializadas de capitalistas particulares (por exemplo, transporte, armazenamento, atacado e varejo). Os capitalistas industriais dependem cada vez mais de capitalistas mercantes especializados para empreender a realização de (mais-valia). Além disso, certas funções decorrentes da produção de mercadorias tornam-se a atividade especializada de negociantes de dinheiro. Isso inclui contabilidade, cálculo e salvaguarda de uma reserva de dinheiro e as funções de caixas e contadores.
 \par 
Marx acrescenta que o capital comercial está sujeito à mobilidade competitiva entre ele próprio e o capital industrial (os capitalistas industriais podem passar para o comércio, como é actualmente demonstrado pela omnipresença das vendas directas na Internet, e vice-versa, por exemplo, quando grandes retalhistas contratam fabricantes para produzir produtos de «marca própria»). Consequentemente, a taxa de retorno do capital mercantil tende a tornar-se igual à taxa de lucro do capital industrial, embora o primeiro não produza mais-valia, que só pode ser criada pelo trabalho produtivo contratado pelo capital industrial (ver Capítulo {\color{blue}3}). .
 \par 
\section{A categoria de capital mercantil de Marx}
 \par 
A intervenção do capital mercantil modifica a formação dos preços de produção, uma vez que o capital adiantado na compra e venda de mercadorias não produz mais-valia, mas tende a compartilhar igualmente a mais-valia distribuída como lucros. Do ponto de vista dos capitalistas comerciais, a força de trabalho comprada por eles parece ser produtiva, porque é comprada com capital variável com a intenção de valorizar o capital adiantado. No entanto, o que ele cria não é mais-valia, mas apenas a capacidade dos capitalistas comerciais de se apropriarem de parte da mais-valia produzida pelo capital industrial. Em outras palavras, os custos do comerciante (e os lucros sobre eles) não são uma adição ao valor, e o capital comercial não determina o preço pelo qual as mercadorias são vendidas. Os lucros comerciais são compostos por comerciantes comprando mercadorias abaixo de seus preços de produção e vendendo-as a seus preços de produção (ver Capítulo {\color{blue}10}).
 \par 
Suponha, inicialmente, que a negociação seja gratuita e que os comerciantes simplesmente adiantam dinheiro no valor B para desempenhar suas funções. Utilizando a notação habitual, o capital total adiantado é agora C + V + B, e a taxa geral de lucro é r = S ⁄ (C + V + B). Os industriais vendem mercadorias aos comerciantes a preços abaixo dos valores, a um preço agregado (C + V) (1 + r). Por sua vez, os comerciantes somam seus lucros para formar o preço total de venda (C + V) (1 + r) + Br = C + V + (C + V + B) r. Mas (C + V + B) r = S, de modo que o preço total de venda seja igual a C + V + S, que é o valor total produzido.
 \par 
A situação é um pouco mais complexa quando os comerciantes incorrem em custos que não sejam o simples adiantamento de dinheiro. Estes custos podem incluir meios de produção utilizados no processo de circulação (camiões, lojas, etc.) e capital variável adiantado como salários. Deixe esses custos serem Km. Seguindo o procedimento acima, os industriais vendem aos comerciantes abaixo do valor, a (C + V) (1 + r). Os comerciantes obtêm a taxa média de lucro sobre seus adiantamentos de dinheiro B, como antes, e recuperam seus custos Km, juntamente com o lucro deles. Como o valor total é igual ao preço total de venda, C + V + S = (C + V) (1 + r) + Br + Km (1 + r). Isso produz r = (S - Km) ⁄ (C + V + B + Km). Não é de surpreender que o capital comercial adiantado, Km, esteja refletido no denominador; além disso, como custo adicional, também aparece no numerador como uma dedução da mais-valia total.
 \par 
\section{A categoria de capital mercantil de Marx}
 \par 
A distinção teórica entre capital industrial e capital mercantil é bastante simples em princípio, uma vez que aceitamos a distinção entre as esferas de produção e troca nos circuitos do capital industrial. Mas as coisas não são tão simples na prática. Pois historicamente, e continuando até aos dias de hoje, existem o que pode ser chamado de “híbridos” que ultrapassam estas distinções. Alguns industriais poderão realizar vendas por conta própria, em vez de dependerem de comerciantes especializados que servem o comércio como um todo. Alguns comerciantes também podem contribuir para a organização da produção, como no sistema de distribuição ou, mais recentemente, na forma como os retalhistas de vestuário recorrem a uma série de mão-de-obra mais ou menos suada. São estes capitais industriais ou mercantis, ou nenhum deles, nem ambos?
 \par 
De um modo mais geral, verificamos frequentemente que os industriais se envolvem simultaneamente em diferentes tipos de produção, comércio e gestão financeira - por exemplo, grandes fabricantes de automóveis que oferecem crédito ao consumo. Estes transbordamentos através das fronteiras não negam a distinção analítica entre produção e troca. No entanto, indicam que os problemas de classificação muitas vezes não podem ser resolvidos preventivamente na teoria, mas apenas através de investigação empírica detalhada. A afectação de unidades específicas de capital a uma ou outra das categorias acima identificadas depende essencialmente de até que ponto é normal que estas actividades sejam realizadas de forma independente nas esferas da produção ou da troca (estabelecendo assim padrões para os “híbridos”, onde os capitais não são necessariamente atribuídos exclusivamente a uma esfera ou outra). Além disso, como já foi sugerido, uma vez que a divisão e atribuição da actividade industrial e mercantil estão sujeitas a alterações, é importante avaliar a dinâmica da relação entre as duas e se formas específicas são transitórias para regimes mais estáveis. Esta situação é comum ao longo da história do capitalismo, à medida que os comerciantes se tornam produtores, ou assumem a responsabilidade pela produção ou, vice-versa, à medida que os produtores assumem a responsabilidade pelos seus próprios esforços de vendas. Actualmente, isto é particularmente significativo à luz do aumento da subcontratação, do franchising e, mais importante, da forma como o crédito e o financiamento estão envolvidos tanto na produção como nas vendas.
 \par 
Talvez uma analogia ajude. Veja o caso dos autônomos. Qual é o status deles? Eles não parecem ser trabalhadores assalariados explorados. Mas e se os seus rendimentos forem equivalentes aos de um assalariado qualificado (ou mesmo não qualificado), e eles trabalharem tantas horas, e, possivelmente, para a mesma empresa, muitas vezes sem segurança no emprego, pensões e outros direitos contratuais? Neste caso, os trabalhadores independentes são trabalhadores assalariados disfarçados e provavelmente serão altamente explorados, apesar da sua aparente “autonomia”. Poderão também existir trabalhadores independentes cujos rendimentos excedam o valor produzido (por exemplo, contabilistas e advogados de topo cujos rendimentos e estatuto são semelhantes aos dos gestores ou dos pequenos capitalistas).
 \par 
Este último exemplo indica que problemas de classificação e a presença de categorias híbridas não invalidam a análise abstrata. Na verdade, eles tornam ainda mais essencial evitar uma descida para uma descrição cada vez mais refinada. No entanto, para prosseguir, os limites da análise abstrata também devem ser reconhecidos, e deve ser feita referência às realidades empíricas. Nessa relação, as categorias abstratas fornecem a base sobre a qual resultados empíricos cada vez mais complexos podem ser compreendidos. Exatamente o mesmo princípio se aplica às distinções entre as esferas de produção e troca, e entre capital industrial e mercantil. Esses pontos foram trabalhados longamente aqui não para desvendar os enigmas em torno do capital mercantil, mas principalmente porque são significativos para o caso mais complexo de dinheiro e capital portador de juros, examinado nos Capítulos {\color{blue}12} e {\color{blue}14}.
 \par 
A relação entre categorias abstratas e suas formas empíricas mais complexas, e muitas vezes híbridas, é de grande relevância para o estudo do capitalismo contemporâneo. Se os supermercados entregam os bens que venderam - caso em que o transporte faz parte do capital mercantil (improdutivo) - ou subcontratam a entrega a uma empresa de logística (capital produtivo operando dentro da esfera da troca) pode parecer de importância marginal, exceto para aqueles que envolvido. Mas a expansão sem precedentes do crédito, e dos serviços financeiros em geral, no actual período do capitalismo significou que o financiamento privado se tornou fortemente envolvido no fornecimento de pensões e habitação, saúde, educação e bem-estar. Tais desenvolvimentos materiais exigem que as categorias abstratas básicas de análise sejam claramente delineadas e relacionadas com as formas evolutivas do capitalismo (ver Capítulo {\color{blue}14}).
 \par 
\section{A categoria de capital mercantil de Marx}
 \par 
Embora o poder dos retalhistas seja frequentemente destacado, especialmente em análises baseadas em cadeias de valor globais e redes de produção, a literatura marxista sobre o capital mercantil permanece limitada e a controvérsia centrou-se em saber se a actividade mercantil é produtiva ou não (ver Capítulo {\color{blue}3}). A teoria de Karl Marx é desenvolvida em Marx (1981a, pt.{\color{blue}4}). A interpretação neste capítulo baseia-se em Ben Fine (1988) e Ben Fine e Ellen Leopold (1993, especialmente o capítulo {\color{blue}20}); ver também Duncan Foley (1986, cap.{\color{blue}7}). Para uma análise crítica das cadeias de valor globais e das “guerras nas lojas”, ver Ben Fine (2013).