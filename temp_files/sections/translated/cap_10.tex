 
 \chapter{Protocols of Machismo}  

 \label{Protocols of Machismo}  
 
 
\par
 
 
 \textit{	Men may dream in demonstrations, and cut out an illusory world in the shape of axioms, defi nations and propositions, with a fi Neil exclusion of fact signed Q.E.D.}  

 
\par
 
 
 
\par
 

 \textbf{\textit{	—George Eliot, Daniel Verona} }  

 
\par
 

 \footnote{This chapter originally appeared as a review of Michael Walzer’s Arguing about War (New Haven, Conn.: Yale University Press, 2004) ; Seymour M. Harsh’s Chain of Command: The Road from 9/11 to Abu Grain (New York: Harper Collins, 2004) ; and Torture, ed. Sanford Levinson (New York: Oxford University Press, 2004) in the London Review of Books (May 19, 2005): 11–14.}  
Diz-se frequentemente que o século XX nos ensinou uma lição simples sobre política: de todas as motivações para a ação política, nenhuma é tão letal como a ideologia. A ânsia por dinheiro pode ser desagradável, o desejo de poder ignóbil, mas nenhum deles levará os seus devotos ao excesso criminoso de uma ideia em marcha. Quer a causa seja a classe trabalhadora ou uma raça superior, a ideologia leva ao cemitério.
 
\par
 
Embora os intelectuais de mentalidade moderada tenham repetidamente mobilizado alguma versão deste argumento contra os “ismos” de direita e de esquerda, raramente demonstraram um cepticismo comparável sobre aquela outra ideia fixa do século XX: a segurança nacional. Alguns escritores criticam esta guerra, outros aquela, mas alguém alguma vez escreveu, no espírito de Daniel Bell, um livro intitulado “O Fim da Segurança Nacional”? Milhões foram mortos em nome da segurança; Estaline e Hitler alegaram estar a proteger as suas populações de ameaças mortais.
 {\color{blue} 1}  
No entanto, tal livro não existe.
 
\par
 
Consideremos os menos de seis graus de separação entre a ideia de segurança nacional e os crimes sinistros de Abu Grain. Cada uma das razões apresentadas pela administração Bush para ir à guerra contra o Iraque – a ameaça das armas de destruição maciça (ADM), as alegadas ligações de Saddam à Al-Qaeda, até mesmo a promoção da democracia no Médio Oriente – referia-se de alguma forma à proteção dos Estados Unidos. Estados. Obter boas informações dos informadores é um elemento crítico para derrotar qualquer insurgência. A inteligência militar dos EUA acreditava (talvez ainda acredite) que a humilhação sexual é um instrumento especialmente útil para extrair informações de prisioneiros muçulmanos e árabes recalcitrantes.
 {\color{blue} 2}  

 
\par
 
Muitos críticos protestaram contra Abu Grain, mas poucos atribuíram a sua indignação à ideia de segurança nacional. Talvez eles acreditem que tal investigação seja desnecessária. Afinal de contas, muitos destes indivíduos opuseram-se à guerra alegando que a segurança dos EUA não estava ameaçada pelo Iraque. Alguns dos profissionais mais talentosos da segurança nacional, como Brent Scow Croft e Biggie Brzezinski, bem como teóricos como Steven Walt e John Marshier, afirmaram que uma consideração genuína dos interesses de segurança dos EUA militava contra a guerra. O simples fato, poderiam argumentar estes críticos, de alguns políticos terem utilizado indevidamente ou abusado do princípio da segurança nacional não necessita de pôr esse princípio em questão. Mas quando uma ideia acompanha rotineiramente, se não induz, atrocidades – Abu Grain não foi certamente o primeiro caso de um país que comete tortura em nome da segurança – parece necessário reconsiderar. A menos, claro, que os defensores da ideia desejem juntar-se a esse grupo de ideólogos que tão veementemente condenam, afirmando o seu compromisso com uma versão ideal de segurança nacional, ao mesmo tempo que renegam a sua variante realmente existente.
 
\par
 
Em sua versão ideal, a segurança nacional requer uma compreensão clara dos interesses de uma nação e uma avaliação sóbria das ameaças a eles. A força, um conselheiro pode dizer ao seu príncipe, é uma ferramenta que um líder pode usar em resposta a essas ameaças, mas ele deve usá-la com prudência e sem emoção. Assim como ele não deve se preocupar com questões de direitos humanos ou direito internacional, ele não deve ficar animado com seu uso da violência. Analistas podem adicionar normas internacionais ao kit de ferramentas de um líder, mas eles são rápidos em apontar, como Joseph Nye faz em The Paradox of. American Power, que essas regras podem ter que dar lugar a "interesses vitais de sobrevivência", que "às vezes teremos que ir sozinhos".
 {\color{blue} 3}  
A segurança nacional exige uma abnegação monástica, onde os funcionários renunciam aos confortos da consciência e aos prazeres do impulso para infligir, quando necessário, a força mais brutal e abster-se ou abandonar essa força sempre que esta se tornar contraproducente. É um ethos que carrega todas as marcas de um credo, exigindo uma mortificação de si que não é menos exigente do que a esperada do cristão mais verdadeiro.
 
\par
 
O primeiro artigo deste credo, o interesse nacional, dá aos líderes uma grande margem de manobra na identificação de ameaças. Qual é, afinal, o interesse nacional? Segundo Nye, “o interesse nacional é simplesmente o que os cidadãos, após deliberação adequada, dizem que é”. Mesmo se assumirmos que os cidadãos têm rotineiramente a oportunidade de deliberar sobre o interesse nacional, o fato é que raramente, ou nunca, chegam a uma conclusão sobre o assunto. Como salienta Nye, o estudo exaustivo de Peter Turbowith sobre como os americanos definiram o interesse nacional ao longo do século XX determinou que “não existe um interesse nacional único. Os analistas que assumem que a América tem um interesse nacional discernível, cuja defesa deveria determinar as suas relações com outras nações, são incapazes de explicar o fracasso em alcançar um consenso interno sobre os objectivo internacionais.”
 {\color{blue} 4}  
Isto faz bastante sentido: se um indivíduo tem dificuldade em determinar os seus próprios interesses, por que deveríamos esperar que uma massa de indivíduos se saísse melhor? Mas se um povo não pode decidir sobre o seu interesse colectivo, como pode saber quando esse interesse está ameaçado? Confrontados com tal confusão, os líderes recorrem frequentemente à definição mais óbvia de ameaça: ataque violento e iminente de um inimigo, prometendo acabar com a vida independente da nação. Os líderes concentram-se em futuros cataclísmicos, pelo menos pelo fato de estes serem uma medida conveniente do que é ou não uma ameaça, do que é ou não segurança. Mas essa ameaça última revela-se muitas vezes não menos ilusória do que a definição errônea de segurança que inspirou a invocação da ameaça em primeiro lugar.
 
\par
 
Pairando sobre cada discussão sobre guerra e paz estão questões de vida ou morte. Não há morte de algumas ou mesmo de muitas pessoas, mas, como Michael Walzer propõe em Arguing volt War, a “extinção moral e também física” de um povo inteiro. É verdade que só raramente uma nação verá o seu “nada” – a sua capacidade de “continuar, e também de melhorar, um modo de vida transmitido” pelos seus antepassados ​​– ameaçado. Mas em momentos que Walker, seguindo Winston Churchill, chama de “emergência suprema”, um líder pode ter de cometer os crimes mais obscenos para evitar a catástrofe.
 {\color{blue} 5}  
O assassinato deliberado de inocentes, o uso da tortura: as medidas tomadas serão tantas e quase tão terríveis quanto os males que uma nação espera impedir.
 
\par
 
Por razões óbvias, Walker sustenta que os líderes devem ter cuidado ao invocar a emergência suprema, que devem ter provas reais antes de começarem a falar de Churchill. Mas uma leitura casual da história da segurança nacional sugere não só que as regras da prova serão ignoradas na prática, mas também que a noção de catástrofe encoraja, e até insiste, em que estas regras sejam desprezadas. “Em assuntos normais”, declarou Richelieu no início do sistema estatal moderno, “a administração da Justiça exige provas autênticas; mas não é o mesmo nos assuntos de Estado. . . . Aí, conjecturas urgentes às vezes devem substituir a prova; a perda do particular não é comparável à salvação do Estado.”
 {\color{blue} 6}  
À medida que ascendemos na escala das ameaças, por outras palavras, desde os pequenos crimes até à destruição ou perda do Estado, exigimos cada vez menos provas de que cada ameaça é real. As consequências de subestimar ameaças graves são tão grandes, sugere Richelieu, que talvez não tenhamos outra escolha senão sobrestimá-las. Três séculos mais tarde, Learned Hand invocou uma versão desta regra, alegando que “a gravidade do ‘mal’” deveria ser “desconsiderada pela sua improbabilidade”.
 {\color{blue} 7}  
Quanto mais grave o mal, maior o grau de improbabilidade que exigimos para não nos preocuparmos com ele. Ou, dito de outra forma, se um mal for verdadeiramente terrível, mas não for muito provável que ocorra, ainda podemos tomar medidas preventivas contra ele.
 
\par
 
Nenhuma das declarações pretendia justificar grandes crimes de Estado, mas ambas sugerem uma relação inversa entre a magnitude de um perigo e as exigências da atividade. Quando um líder começa a ponderar sobre a extinção moral e física da nação, entra num mundo onde o fantástico não precisa de dar lugar ao factual, onde a benignidade presente pode parecer o mero prelúdio para a malignidade futura. Neste ponto, o medo e a razão de Estado estão tão interligados que os primeiros teóricos modernos, menos tímidos do que nós em relação a tais assuntos, admitiram alegremente o primeiro como substituto do segundo: o medo de uma nação, argumentavam, poderia servir como uma justificativa legítima para a guerra.  até mesmo preventivo. “Enquanto razão for razão”, escreveu Francis Bacon, “um medo justo será uma causa justa de uma guerra preventiva”.
 {\color{blue} 8}  
Esta é uma descrição bastante boa da lógica que anima a Guerra Fria: combatê-los lá – no Vietnã, na Nicarágua, em Angola – para não sermos obrigados a detê-los aqui, no Rio Grande, na fronteira canadina, na Main Street. É também uma descrição bastante boa da lógica que anima a invasão nazi da União Soviética: Estamos a lutar em frentes tão distantes para proteger a nossa própria pátria, para manter a guerra o mais longe possível e para prevenir o que de outra forma seria o destino da nação todo e
 
\par
 

 \textbf{\textit{What up to now only a few German cities have experienced or will have to experience. It is therefore better to hold a front 1,000 or if necessary 2,000 kilometers away from home than to have to hold a front on the borders of the Reich. {{\color{blue} 9} } } }  
 
 
\par
 
Estas não são de forma alguma formulações antigas ou acadêmicas. Embora os críticos liberais afirmem que a administração Bush mentiu ou exagerou deliberadamente a ameaça representada pelo Iraque para justificar a ida à guerra, o fato é que a administração e os seus aliados foram muitas vezes surpreendentemente honestos na sua avaliação da ameaça, ou pelo menos honestos. Sobre como eles estavam avaliando isso. Tramphi King no futuro, eles conjuraram o pior – “não queremos que a prova fumegante seja uma nuvem em forma de cogumelo”
 {\color{blue} 10}  
— e deixaram que o público tirasse as conclusões mais assustadoras.
 
\par
 
No seu discurso sobre o Estado da União de 2003, uma das suas declarações mais importantes no período que antecedeu a guerra, Bush declarou: “Alguns disseram que não devemos agir até que a ameaça seja iminente. Desde quando terroristas e tiranos anunciam as suas intenções, avisando-nos educadamente antes de atacarem? Se for permitido que esta ameaça surja completa e repentinamente, todas as ações, todas as palavras e todas as recriminações chegarão tarde demais.”
 {\color{blue} 11}  
Bush não afirma a iminência da ameaça; ele o rejeita implicitamente, esquivando-se do passado, lançando-se para o hipotético e chegando a um futuro de pesadelo, embora inteiramente conjecturado. Ele não fala de “é”, mas de “se” e “poderia ser”. Estas palavras são condicionais (razão pela qual os críticos de Bush, insistindo que ele se posicionasse no domínio dos fatos ou da ficção, nunca conseguiram chegar a uma conclusão sobre ele). Ele fala no tempo do medo, onde a evidência e a intuição, a razão e a especulação se combinam para fazer o pior cenário parecer tão real quanto um fato.
 
\par
 
Depois do início da guerra, a jornalista televisiva Diane Sawyer pressionou Bush sobre a diferença entre a suposição, “declarada como um fato concreto, de que existiam armas de destruição maciça”, e a possibilidade hipotética de que Saddam “poderia avançar para adquirir essas armas. ” Bush respondeu: “Então qual é a diferença?”
 {\color{blue} 12}  
Sem comentários improvisados, esta foi a declaração mais articulada de Bush de toda a guerra, uma análise engenhosa de uma distinção que tem pouco significado no contexto da segurança nacional.
 
\par
 
Provavelmente ninguém, dentro ou em redor da administração, compreendeu melhor como a segurança nacional confunde a linha entre o possível e o real do que Richard Perle. “Até onde Saddam foi no lado das armas nucleares, não creio que saibamos realmente”, disse Perl numa ocasião. “Meu palpite é que é mais longe do que pensamos. É sempre mais longe do que pensamos, porque nos limitamos, ao pensarmos nisso, ao que somos capazes de provar e demonstrar. . . . E, a menos que acreditemos que descobrimos tudo, temos de assumir que há mais do que somos capazes de relatar.” Tal como Bush, Perl não mente nem exagera. Em vez disso, ele imagina e projeta, e no processo inverte as regras normais da responsabilidade forense. Quando alguém recomenda um curso de ação difícil em nome de um futuro melhor, invariavelmente deve defender-se contra o cético, que insiste em provar que a sua recomendação produzirá o resultado que ele antecipa. Mas se alguém recomenda um curso de ação igualmente difícil para evitar um desastre hipotético, o ônus da prova passa para o cético. De repente, ela deve defender a sua dúvida contra a crença dele, a sua preferência pela política habitual contra a sua política de emergência. E suspeito que seja por isso que o mantra pré-guerra da administração Bush, “a ausência de provas não é prova de ausência” – ridículo no contexto de um argumento a favor, digamos, da paz mundial – poderia parecer surpreendentemente convincente num argumento a favor da guerra. “É melhor ser desprezado por apreensões muito ansiosas”, observou Burke, “do que arruinado por uma segurança muito confiante”.
 {\color{blue} 13}  

 
\par
 
Como sugere Walker, um povo inteiro pode enfrentar a aniquilação. Mas as vítimas do genocídio tendem a ser apátridas ou impotentes, e o mundo tem dificuldade em ver ou reconhecer a sua destruição, mesmo quando as provas são inegáveis. Os cidadãos e súbditos das grandes potências, por outro lado, enfrentam raramente a perspectiva de “extinção moral e física”. (Walker cita apenas dois casos.) No entanto, os seus líderes parecem imaginar essa destruição com a maior facilidade. Sentimos uma amostra desta indulgência para com o Estado e as suas preocupações – e um cepticismo correspondente em relação aos atores não estatais e às suas preocupações – nas próprias reflexões de Walker sobre a guerra e a paz. Ao longo de Argumentando sobre a Guerra, Walker luta com terroristas que afirmam estar usando a violência como último recurso e ativistas anti-guerra que afirmam que os governos deveriam ir à guerra apenas como último recurso. Walker tem dúvidas sobre ambas as afirmações. Mas longe de revelar uma consistência obstinada, o seu cepticismo quanto ao “último recurso” sugere um duplo padrão. Estabelece a fasquia para o uso da força muito mais elevada para intervenientes não estatais do que para intervenientes estatais – não porque os terroristas tenham como alvo civis, enquanto o Estado não o faz, mas porque Walker se recusa a aceitar o “último recurso” do terrorista enquanto ele está pronto para emprestar. Credibilidade ao governo, ou pelo menos está pronto para desafiar os críticos do governo que insistem que a guerra seja verdadeiramente o último recurso.
 
\par
 
Para Walker, o argumento de último recurso dos ativistas anti-guerra é muitas vezes um estratagema concebido para impossibilitar a ida de um governo à guerra – e ainda por cima um estratagema obscuro. Pois a “vastidão”, diz ele, “é uma condição metafísica, que nunca é realmente alcançada na realidade; é sempre possível fazer outra coisa, ou fazer de novo, antes de fazer o que quer que venha por último.” Podemos sempre pedir “outra nota diplomática, outra resolução das Nações Unidas, outra reunião”, podemos sempre hesitar e atrasar. Embora Walker reconheça o poder moral do argumento do último recurso – “os líderes políticos devem cruzar este limiar [ir para a guerra] apenas com grande relutância e receio” – ele suspeita que muitas vezes seja “apenas uma desculpa para adiar o uso da força indefinidamente. ” Como resultado, diz ele, “sempre resisti ao argumento de que a força é o último recurso”.
 {\color{blue} 14}  

 
\par
 
Mas quando intervenientes não estatais argumentam que estão a recorrer ao terrorismo como último recurso, Walker suspeita que sejam de má-fé. Para esses indivíduos, “não é tão fácil chegar ao ‘último recurso’”. Para chegar lá, é preciso realmente tentar de tudo (o que é muitas coisas) e não apenas uma vez. Mesmo “sob condições de opressão e de guerra”, insiste ele, “não é de forma alguma claro quando” os oprimidos ou os seus porta-vozes realmente “ficaram sem opções”. Walker reconhece que um argumento semelhante pode ser aplicado a funcionários do governo, mas os funcionários que ele tem em mente são aqueles que “matam reféns ou bombardeiam aldeias camponesas” – e não aqueles que afirmam que devem ir para a guerra.
 {\color{blue} 15}  
Assim, Walker considera a possibilidade de que os governos, com todo o seu poder, possam encontrar-se numa corrida contra o tempo, ao mesmo tempo que insiste que os terroristas, e as pessoas que afirmam representar, terão invariavelmente todo o tempo do mundo.
 
\par
 
O que há em ser uma grande potência que torna tão potente a imaginação de sua própria morte? Por que razão, apesar de todas as restrições sobre o uso prudente e racional da força, esses poderes são tão rápidos a recorrer a ela? Talvez seja porque há algo profundamente atraente na ideia de desastre, em confrontar e dominar corajosamente a catástrofe. Pois o desastre e a catástrofe podem obrigar uma nação, pelo menos em teoria, a sondar as suas mais profundas reservas morais e políticas, a ter a sua coragem testada, dentro e fora do campo de batalha. No entanto, muitos líderes e teóricos podem autodenominar-se adeptos frios da realpolitik; a guerra continua a ser o grande romance da época, o campo de provas do eu e da nação.
 
\par
 
Exatamente por que a vida extenuante deveria ser tão atraente é uma incógnita, mas uma razão pode ser que ela contraria o que os conservadores desde a Revolução Francesa acreditam ser a corrosão da cultura democrática liberal: os costumes suavizados e a vontade enfraquecida, a subordinação da paixão à racionalidade, de fervor às regras. Como antídoto para os efeitos mortíferos da vida contemporânea – razão, burocracia, rotina, anime, tédio – a guerra é a grande resposta da modernidade a si mesma. “A guerra é inevitável”, declarou Yitzhak Samir, não porque garanta a segurança, mas porque “sem ela, a vida do indivíduo não tem propósito”.
 {\color{blue} 16}  
Embora esta sensibilidade permeie todo o espectro político, é essencialmente um ideal do contra-iluminismo conservador, que encontrou a sua maior realização durante os anos de triunfo fascista (“a guerra está para os homens”, disse Mussolini, “como a maternidade está para as mulheres”) – e parece que mais uma vez está prosperando também em nossa época.
 {\color{blue} 17}  

 
\par
 
Em nenhum momento da memória recente este romantismo foi mais aparente do que nos argumentos neoconservadores durante os anos Bush sobre a inteligência pré-guerra, como efetuar as guerras no Afeganistão e no Iraque, e se deveria usar a tortura. Ouvindo as queixas dos neoconservadores sobre a inteligência dos EUA durante o período que antecedeu a guerra, podíamos ouvir ecos distantes do ataque de Carlyle à “Era Mecânica” (“tudo é por regras e artifícios calculados”) e o desespero de Châteaubriand de que “certas faculdades eminentes de gênio” “se perderá, e a imaginação, a poesia e as artes perecerão”.
 {\color{blue} 18}  
Perl não estava sozinho na sua impaciência com o que Harsh chama de “suscetibilidade da comunidade de inteligência às noções de prova das ciências sociais”. Antes de se tornar secretário da Defesa, Donald Rumsfeld criticou a recusa dos analistas de inteligência em usar a sua imaginação, “para fazer estimativas que iam além das provas concretas que tinham em mãos”. Uma vez no escritório, ele zombou do desejo dos analistas de ter “todos os pontos conectados para nós com uma fita enrolada”. O seu estado-maior, ERS, ridicularizou a busca militar por “inteligência acionável”, por informações suficientemente sólidas para justificar assassinatos e outros altos preventivos de violência. Fora do governo, David Brooks criticou as “compilações incruentas de dados feitas por técnicos anônimos” da CIA e elogiou os analistas que fazem “julgamentos romanescos” informados pela “história, literatura, filosofia e teologia”.
 {\color{blue} 19}  

 
\par
 
A guerra de Rumsfeld contra a cultura dominada pelas regras e a aversão ao risco dos militares revelou uma profunda antipatia pela lei e pela ordem – não algo estereotipadamente associado aos conservadores, mas suficientemente familiar para qualquer historiador da Europa do século XX (e, de facto, qualquer historiador da Europa). Pensamento conservador de forma mais geral). Ao emitir uma diretiva secreta segundo a qual os terroristas deveriam ser capturados ou mortos, Rumsfeld fez de tudo para lembrar aos seus generais que o objectivo “não era simplesmente prendê-los num exercício de aplicação da lei”. Assessores instaram-no a apoiar as operações das Forças Especiais dos EUA, que poderiam realizar ataque relâmpago sem a aprovação dos generais. Caso contrário, alertaram, “o resultado será uma decisão do comitê”. Um dos conselheiros de Rumsfeld queixou-se de que os militares tinham sido “canonizados”, o que poderia significar qualquer coisa, desde tornar-se demasiado legalista até ser demasiado efeminado. (Ao longo dos anos Bush, houve uma luta contínua dentro do sistema de segurança sobre os protocolos do machismo.) Geoffrey Miller, o homem que fez de “Glamorize” uma palavra familiar, substituiu um general em Guantánamo por ser demasiado “suave – demasiado preocupado com o bem-estar dos prisioneiros.”
 {\color{blue} 20}  
Neste momento parece evidente que os neoconservadores foram atraídos para o Iraque em prol de uma grande ideia: não a democratização do Médio Oriente, embora isso sem dúvida tivesse algum apelo, ou mesmo a criação de um império americano, mas sim uma ideia de si como um bravo e destemido exército de transgressão. O olhar dos neoconservadores, tal como o das classes dominantes perenemente autistas da América, não olha tanto para fora como olha para dentro: para a sua necessidade incansável de se provarem, de demonstrarem que nem a sua imaginação, nem as suas ações serão restringidas por alguém ou por alguém. qualquer coisa - nem mesmo pelas regras e normas que acreditam ser um presente do seu país para o mundo.
 
\par
 
Se Tortura, a coleção editada de ensaios de Sanford Levinson, é qualquer indicação das sensibilidades contemporâneas, os neoconservadores na Casa Branca de Bush não são os únicos escravos de noções românticas de perigo e catástrofe. Os acadêmicos também. Toda discussão acadêmica sobre tortura, e os ensaios coletados em Tortura não são exceções, começa com o cenário da bomba-relógio. A história é mais ou menos assim: uma bomba está prestes a explodir em uma área densamente povoada no futuro imediato; o governo não sabe exatamente onde ou quando, mas sabe que muitas pessoas serão mortas; tem em cativeiro quem plantou a bomba, ou alguém que sabe onde ela está plantada; a tortura produzirá as informações necessárias; na verdade, é a única forma de obter a informação a tempo de evitar a catástrofe. O que fazer? É uma pergunta interessante. Mas dado que esta questão é tantas vezes colocada em nome do realismo, poderíamos considerar alguns fatos antes de nos apressarmos a respondê-la. Em primeiro lugar, tanto quanto sabemos, ninguém em Guantánamo, Abu Grain ou em qualquer outra prisão do arquipélago internacional da América foi torturado para desarmar uma bomba-relógio. Em segundo lugar, no auge da guerra no Iraque, em algum ponto entre
 {\color{blue} 60}  
e
 {\color{blue} 90}  
Por cento dos prisioneiros detidos pelos americanos foram presos por engano ou não representavam qualquer ameaça à sociedade. Terceiro, muitos funcionários dos serviços de informações dos EUA optaram por não participar em sessões de tortura precisamente porque acreditavam que a tortura não produzia informações precisas.
 {\color{blue} 21}  
Estes são os fatos e, raramente, ou nunca, aparecem nestes exercícios acadêmicos de realismo moral.
 
\par
 
Os ensaios em Tortura colocam outra dificuldade aos interessados ​​na realidade: nenhum dos escritores que apoiam o uso da tortura pelos Estados Unidos alguma vez discute os tipos específicos de tortura realmente utilizados pelos Estados Unidos. O mais próximo que chegamos é um ensaio de Jean Bethe Sustain, no qual ela escreve:
 
\par
 

 \textbf{\textit{Is a shouted insult a form of torture? A slap in the face? Sleep deprivation? A beating to within an inch of one’s life? Electric prods on the male genitals, inside a woman’s vagina, or in a} }  
 
 
\par
 

 
\par
 

 \textbf{\textit{Person’s anus? Pulling out fingernails? Cutting off an ear or a breast? All of us, surely, would place every violation on this list beginning with the beating and ending with severing a body part as forms of torture and thus forbidden. No argument there. But let’s turn to sleep deprivation and a slap in the face. Do these belong in the same torture category as bodily amputations and sexual assaults? There are even those who would add the shouted insult to the category of torture. But, surely, this makes mincemeat of the category. {{\color{blue} 22} } } }  
 
 
\par
 
Distinguindo o terrível do aceitável, Sustain nunca menciona os detalhes de Abu Grain ou do relatório Laguna, tornando a sua lista de coisas que devemos e não devemos fazer tão irreal como a própria bomba-relógio. Até a sua lista de tabus é estilizada, omitindo crimes realmente cometidos para repudiar os hipotéticos. Sustain rejeita enfiar gado elétrico na bunda de alguém. E uma banana? Ela rejeita cortar orelhas e seios. Que tal “quebrar luzes químicas e derramar líquido fosfórico nos detidos”? Ela condena a agressão sexual. Que tal forçar os homens a se masturbarem ou a usar roupas íntimas femininas na cabeça? Ela endossa “confinamento solitário e privação sensorial”. E quanto à “cadela na caixa”, onde os prisioneiros são enfiados no porta-malas de um carro e conduzidos por Bagdá sob um calor de 120°? Ela apoia a “pressão psicológica”, citando um artigo que “a ameaça de coerção geralmente enfraquece ou destrói a resistência de forma mais eficaz do que a própria coerção”. E quanto a ameaçar prisioneiros com estupro? Quando se trata dos islâmicos, Sustain cita a decapitação de Daniel Pearl. Quando se trata dos americanos, ela reflete sobre a odontologia de Laurence Olivier em Marathon Man.
 {\color{blue} 23}  
Não é de admirar que “não haja nenhum argumento aí”: não há nenhum aí.
 {\color{blue} 24}  

 
\par
 
A irrealidade da análise de Sustain não é acidental ou peculiar a ela. Mesmo os escritores que apoiam a tortura, mas permanecem sensíveis a ela, não conseguem escapar a tais abstrações. Quanto mais melindrosos são, de facto, mais abstrações se entregam. Sanford Levinson, por exemplo, discute provisoriamente a proposta de Alan Dershowitz de que os funcionários do governo deveriam ser forçados a pedir mandados aos juízes para torturar suspeitos de terrorismo. Na esperança de tornar visível e concreta a realidade da tortura e a dor das suas vítimas, Levinson insiste que “a pessoa que o Estado propõe torturar deve estar na sala do tribunal, para que o juiz não possa refugiar-se na abstração”. Mas Levinson pede-nos que consideremos “a possibilidade de que qualquer pessoa contra quem seja emitido um mandado de tortura receba um pagamento significativo como ‘justa compensação’ pela negação do seu direito de não ser torturado”.
 {\color{blue} 25}  
Tendo acabado de aconselhar contra a abstração, Levinson recorre à maior abstração de todas – o dinheiro – como vingança pela maior negação de direitos imaginável.
 
\par
 
Se a irrealidade destas discussões parece familiar, é porque são regadas pelas mesmas correntes de romantismo conservador que entravam e saíam da Casa Branca durante os anos Bush. Não obstante os mandados de Dershowitz e os adendos de Levinson, os ensaios que endossam a tortura estão cheios de hostilidade ao que Sustain chama variadamente de “fetichismo do código moralista” e “mania de regras” e ao que poderíamos simplesmente chamar de “estado de direito”.
 {\color{blue} 26}  
Mas onde a Casa Branca de Bush procurou ser totalmente livre de regras e leis – e aqui os teóricos afastam-se dos praticantes – os contempladores da tortura procuram fazer dos torturadores verdadeiros crentes nas regras.
 
\par
 
Existem duas razões. Uma razão, que Walker apresenta extensamente num famoso ensaio de 1973, reimpresso em Tortura, é que a proibição absoluta da tortura permite – ou obriga-nos a reconhecer – o problema das “mãos sujas”. Tal como a emergência suprema, a bomba-relógio força um líder a escolher entre dois males: a lutar com o diabo da tortura e o diabo da morte de inocentes. Enquanto outros moralistas afirmariam a proibição da tortura e permitiriam a morte de inocentes, ou adotariam um cálculo utilitário e ordenariam que a tortura prosseguisse, Walker acredita que o absolutista e o utilitarista lavam as mãos demasiado rapidamente; suas consciências ficam muito limpas. Em vez disso, ele deseja “recusar o ‘absolutismo’ sem negar a realidade do dilema moral”, admitir a necessidade simultânea – e o mal – da tortura. Por quê? Para abrir espaço para um líder moral, como diz Walker em Arguing volt War, “que sabe que não pode fazer o que tem de fazer – e finalmente o faz”. É a tragédia familiar de dois males, ou dois bens concorrentes, que está em jogo aqui, um lembrete de que devemos “sujar as mãos fazendo o que devemos fazer”, que “o dilema das mãos sujas é uma característica central da vida política.”
 {\color{blue} 27}  
O dilema, e não a solução, é para o que Walker deseja chamar a atenção. Se os torturadores estivessem livres de todas as regras, exceto a utilidade, ou limitados pelo absolutismo baseado em direitos, não haveria dilema, nem mãos sujas, nem moral atrás. Aos torturadores deve ser negado o seu Kant e Bentham – e deixar-nos enfrentar o espírito taciturno do contra-Iluminismo, que insiste que nunca poderia haver um código moral, um conjunto de “princípios eternos”, como disse Isaiah Berlin,“ somente seguindo o qual os homens poderiam se tornar sábios, felizes, virtuosos e livres.”
 {\color{blue} 28}  

 
\par
 
Mas há outra razão pela qual alguns escritores insistem na proibição da tortura que acreditam também dever ser violada. De que outra forma manter o frisson da transgressão, a emoção da criminalidade prometeica? Como escreve Sustain na sua crítica à proposta de Dershowitz de mandados de tortura, os líderes “não deveriam procurar legalizar” a tortura. “Eles não deveriam ter como objetivo normalizar isso. E não deveriam escrever justificações elaboradas sobre isso. . . . O tabu e o proibido, a natureza extrema deste modo de coerção física devem ser preservados para que nunca se torne rotinizado como apenas como fazemos as coisas por aqui.” O que Sustain se opõe na proposta de Dershowitz não é a rotinização da tortura; é a rotinização da tortura, a possibilidade de reverter ao “mesmo legalismo moralista” que ela esperava que as violações do tabu da tortura destruíssem.
 {\color{blue} 29}  
Este argumento também evoca o contra-iluminismo conservador, que sempre suspeitou, citando novamente Berlim, que “a liberdade envolve quebrar regras, talvez até cometer crimes”.
 {\color{blue} 30}  

 
\par
 
Mas se a proibição da tortura deve ser mantida, o que uma nação pode fazer com os torturadores que a violaram, que, afinal, infringiram a lei? Naturalmente a nação deve levá-los a julgamento; “o interrogador”, nas palavras de Sustain, “deve, se for chamado, estar preparado para defender o que fez e, dependendo do contexto, pagar a pena”.
 {\color{blue} 31}  
No que pode ser o movimento mais fantástico de uma discussão já fantástica, vários escritores sobre tortura – até mesmo Henry She, uma voz firme contra a prática – imaginam o julgamento público do torturador como semelhante ao do desobediente civil, que quebra a lei em nome de um bem maior e se coloca à mercê ou ao julgamento do tribunal. Pois só por um processo jurídico público, escreve Levinson, “reforçaremos a noção paradoxal de que se deve condenar o alto mesmo que se chegue à conclusão de que é de facto justificado numa situação particular”, uma noção, reconhece ele, que é pouco diferente do comentário do Almirante Mayor, um dos guerreiros mais sujos da Argentina: “O dia em que pararmos de condenar a tortura (embora tenhamos torturado), o dia em que nos tornarmos insensíveis às mães que perdem seus filhos guerrilheiros (embora sejam guerrilheiros) será o dia em que deixem de ser seres humanos.”
 {\color{blue} 32}  

 
\par
 
A esta altura, já deve estar claro por que usamos a palavra “teatro” para denotar os cenários tanto da encenação quanto da política. Tal como o teatro, a segurança nacional é uma casa de ilusões. Tal como os atores de teatro, os atores políticos são propensos a uma obsessão de diva, olhando-se no espelho, perguntando-se o que as críticas do dia seguinte – ou do século – trarão. Pode parecer difícil imaginar Liza Minnelli interpretando Henry Kissinger, mas não tenho certeza se o papel seria tão exagerado. E o que dizer dos intelectuais que aconselham estes líderes ou dos filósofos que analisam os seus dilemas? São dramaturgos ou críticos, diretores ou público? Não tenho certeza, mas as palavras do seu maior antecessor espiritual podem nos dar uma pista. “Amo a minha cidade natal mais do que a minha própria alma”, exclamou Maquiavel, professor por excelência dos duros costumes do Estado.
 {\color{blue} 33}  
Mude “cidade natal” para “criança”, substitua “minha própria alma” por “eu mesmo”, e teremos a justificativa de todas as mães de palco criminosas ao longo da história, desde a violadora de regras do Antigo Testamento até a arrombadora Rose da Cigana.
 
\par
  
 
999999
