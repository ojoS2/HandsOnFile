\chapter{9 A queda da taxa de lucro}\label{9 A queda da taxa de lucro}
 \par 
A teoria de Marx da lei da tendência de queda da taxa de lucro (LTRPF) tem sido extremamente controversa em termos da sua validade, interpretação e significado. Este capítulo descreve a lei de Marx e responde a algumas das críticas que lhe foram feitas. Duas interpretações equivocadas do LTRPF são frequentemente encontradas na literatura. Por um lado, a contribuição de Marx é removida para o âmbito da alta filosofia, com a LTRPF assumindo o caráter de uma verdade abstrata, algo derivado da própria lógica do capital e, portanto, irrefutável, mas também desprovido de qualquer significado empírico. Por outro lado, a análise de Marx foi tratada como se representasse um conjunto de proposições empíricas que são corretas, incorretas ou algo entre elas, dependendo das inclinações do analista e das implicações do modelo de economia escolhido.
 \par 
A posição aqui adotada difere desses dois extremos, reconhecidamente parodiados. No entanto, o argumento é complexo, dependendo de considerações conceituais e não algébricas. Como resultado, a estrutura da análise é resumida primeiro, seguida por um relato mais detalhado contendo elaboração e justificativa.
 \par 
\section{Resumo do argumento}
 \par 
O LTRPF de Marx baseia-se na distinção conceptual entre as composições orgânicas e de valor do capital (OCC e VCC), sendo que a literatura raramente distingue entre as duas e geralmente utiliza o termo OCC quando se refere ao VCC (ver Capítulo {\color{blue}8}). Mostrámos que o OCC mede os resultados da acumulação por referência exclusiva à esfera da produção, ou seja, a criação de (excedente) valor, enquanto o VCC mede e reflete o processo de acumulação na esfera da troca, ou seja, a realização (excedente) de valor, que se centra, mas não deve se limitar, ao problema da venda.
 \par 
O OCC tende a aumentar ao longo do tempo devido à adopção de métodos de produção especificamente capitalistas, especialmente a utilização de maquinaria, no contexto da concorrência dentro dos sectores e da tentativa sistemática de extrair mais-valia relativa. Esta tendência de aumento do OCC é a fonte da lei como tal, enquanto a formação do VCC está associada às tendências contrárias (CTs) ao LTRPF. A interação entre a lei e as TCs é um aspecto essencial do processo de acumulação. Esta interacção forma fenómenos económicos mais complexos, mas apenas para aquela fase de desenvolvimento do capitalismo em que a produção mecanizada é predominante. Isto implica que a LTRPF não é uma lei empírica no sentido estritamente preditivo - é, antes, uma lei abstrata. Não fornece indicações prospectivas (quantitativas) sobre movimentos na taxa de lucro, mas fornece a base sobre a qual podem ser estudados fenómenos económicos mais complexos (ver Capítulo {\color{blue}1}).
 \par 
Esta apresentação da LTRPF de Marx contrasta fortemente com a compreensão e crítica da mesma associada ao economista japonês Nobuo Okishio, que foi adoptada pela escola de economia Sraffiana, bem como por alguns marxistas. Dado que esta abordagem está limitada ao que é denominado estática comparativa (ou seja, a comparação do equilíbrio antes e depois da mudança técnica), ela trata o processo de acumulação como um processo que necessariamente engendra a integração harmoniosa entre produção e circulação, em vez de analisar as suas dinâmicas contraditórias. Consequentemente, a análise de Okishio pode ser caracterizada como o oposto dialético da de Marx.
 \par 
\section{Resumo do argumento}
 \par 
O tratamento dado por Marx à LTRPF ocupa três capítulos na terceira parte do Volume {\color{blue}3} de O Capital. O primeiro deles intitula-se “A Lei como Tal” e contém o que parece ser uma simples demonstração algébrica da queda da rentabilidade no capitalismo. Como a taxa de lucro pode, em termos de valor, ser escrita como r = s ⁄ (c + v) = e ⁄ (OCC + {\color{blue}1}), onde e é a taxa de mais-valia (s ⁄ v) e o OCC é c ⁄ v, uma queda em r é a consequência direta de um OCC crescente, desde que não haja aumento em e.
 \par 
Esta interpretação mecanicista é, contudo, incorrecta e o LTRPF não consegue prever movimentos empíricos na taxa de lucro por duas razões. Primeiro, as leis marxistas não são a expressão teórica de regularidades empíricas. Aqui, uma analogia com a lei da gravidade pode ajudar: esta lei física baseia-se na ideia de que os corpos se atraem mutuamente, como na maçã de Newton que cai na terra. Mas, empiricamente, a lei da gravidade também pode explicar resultados que parecem contradizê-la - os planetas têm órbitas elípticas estáveis ​​em torno do Sol, os aviões voam e os edifícios permanecem de pé. Da mesma forma, as leis marxistas expressam as principais forças materiais constituídas pelas relações sociais capitalistas, o que Marx chama de tendências. É por isso que a LTRPF é aparentemente estranhamente chamada de “lei da tendência”. Embora as leis e tendências marxistas surjam das relações sociais que definem o modo de produção, e sejam, portanto, necessárias (por outras palavras, são inevitáveis ​​nesse tipo de sociedade), elas não determinam diretamente os resultados empíricos. Por exemplo, a tendência para a mecanização e (consequentemente) o aumento do OCC não implica que a taxa de lucro deva cair continuamente; inversamente, as flutuações da taxa de lucro não anulam o LTRPF. Da mesma forma, a tendência para a equalização das taxas de lucro entre sectores como resultado da maximização dos lucros e da mobilidade do capital não implica que estas taxas serão efectivamente equalizadas num ponto específico no futuro (é apenas no mundo de faz-de-conta da economia dominante que esta tendência apoia um equilíbrio real em que todas as taxas de lucro são equalizadas).
 \par 
Para Marx, as leis e tendências devem ser localizadas analiticamente no contexto das suas fontes e das formas (relativamente mais complexas) pelas quais essas leis e tendências se manifestam. Por exemplo, as tendências interagem sempre com contratendências no contexto de circunstâncias históricas específicas, conduzindo a resultados que são indeterminados ex ante mas, em princípio, compreensíveis ex post (ver Capítulo {\color{blue}1}). No caso de capitais concorrentes, por exemplo, a tendência para a equalização das suas taxas de lucro tem de ser contraposta à concorrência entre capitais do mesmo sector, que diferencia as suas taxas de lucro, seja através da acumulação para aumentar a produtividade, da pagamento de salários mais baixos, ou qualquer outra coisa (ver Capítulo {\color{blue}6}).
 \par 
A segunda razão pela qual a LTRPF não permite previsões empíricas é que qualquer consideração da composição orgânica (e não de valor) do capital, como é o caso desta lei, é restrita a mudanças na produção, sem qualquer referência ao reflexo dessas mudanças. mudanças de valor em circulação. Isto explica por que o valor constante de e não é uma suposição arbitrária, mas, antes, uma expressão dos valores imutáveis ​​das mercadorias (incluindo a força de trabalho) durante a produção.
 \par 
O segundo capítulo de Marx, intitulado “Fatores Contrariantes”, trata dos TCs. Eles se enquadram em duas categorias. Há aqueles que decorrem diretamente das alterações de valores resultantes do aumento do OCC. Se escrevermos r = s ⁄ (c + v), segue-se que qualquer coisa que reduza c ou v, e qualquer coisa que aumente s, tende a aumentar r. A produção de mais-valia relativa faz tudo isto, porque o aumento da produtividade reduz o valor de c e v (seja directamente no sector dos bens assalariados ou indirectamente através da sua utilização de matérias-primas de menor valor) e aumenta s, através da declínio em v (dado o salário real). Essas mudanças de valores são sinônimos da formação do VCC, destacando a importância desse conceito e sua diferença em relação ao OCC.
 \par 
Marx também considera os TCs de uma variedade menos sistemática. Por exemplo, ele lista a superexploração da força de trabalho, especialmente dos desempregados e dos desorganizados (produzindo mais-valia absoluta), o barateamento de matérias-primas e bens salariais através do comércio exterior, e a formação de sociedades por ações (que podem assumir uma taxa de lucro mais segura mas mais baixa nas suas actividades em grande escala). Este grupo de TC não resulta necessariamente da acumulação de capital ou da crescente OCC, embora sejam resultados prováveis ​​do desenvolvimento capitalista. Marx parece agrupá-los com os outros sem separá-los analiticamente. Isto pode ser explicado pela falta de preparação final do Volume {\color{blue}3} para publicação. Além disso, a lista de TCs de Marx segue de perto a de J.S. Mill, sugerindo que ele ainda não havia retrabalhado esse material. No entanto, uma diferença importante entre Marx e Mill é que o tratamento que este último dá à lei segue o de Ricardo e se baseia no declínio da produtividade da agricultura, e não, como aconteceu com Marx, no aumento da produtividade na indústria.
 \par 
O tratamento que Marx dá aos TC também faz parecer que ele está a lidar com movimentos imediatos em r como um contrapeso numérico à lei como tal. Contudo, os CTs estão necessariamente localizados num nível de análise mais complexo do que o direito, pois, como vimos, eles envolvem a formação do VCC, que incorpora mudanças tanto na produção quanto na troca (enquanto o próprio direito envolve apenas mudanças na produção e a formação do OCC). No entanto, tal como a lei, as CT não devem ser vistas como factores de peso empírico que governam directamente a taxa de lucro, mas como incorporando aqueles processos de acumulação e reestruturação que transformam mudanças nas condições de produção em movimentos de troca.
 \par 
\section{Resumo do argumento}
 \par 
Na secção anterior, interpretámos tanto o LTRPF como os CT como capturando processos e relações relativamente abstractos, em vez de preverem movimentos imediatos na taxa de lucro. Esta é a base para examinar o terceiro capítulo de Marx sobre a LTRPF, apropriadamente denominado “Desenvolvimento das Contradições Internas do Direito”. Neste capítulo, Marx examina o direito e as TC como uma unidade contraditória de processos subjacentes que dão origem a fenómenos empíricos mais complexos. Mesmo nesta fase relativamente concreta, Marx está mais preocupado com a coexistência antagónica da lei e dos seus CT do que com a previsão de movimentos na taxa de lucro. Isto acontece porque a lei e os TC não podem ser somados algebricamente para originar um aumento ou uma descida na taxa de lucro, dependendo de qual dos dois é o mais forte, tal como os efeitos da concorrência dentro e entre sectores não podem ser somados. até sugerir que as taxas de lucro irão divergir no sentido do monopólio ou, em vez disso, igualar-se-ão em todos os capitais no tempo histórico (ver Capítulo {\color{blue}6}). Pelo contrário, Marx está preocupado com as contradições entre a produção e a circulação de (mais) valor à medida que o processo de criação de valor prossegue, com base em valores que são constantemente perturbados pela acumulação de capital.
 \par 
Que a LTRPF diz respeito à interacção de tendências abstractas, em vez de antecipar um declínio inevitável nas taxas de lucro reais das empresas ou economias capitalistas, é implicitamente confirmado pela análise de Marx das contradições internas da lei. Há pouca ou nenhuma discussão sobre movimentos na taxa de lucro na terceira parte do Volume {\color{blue}3} de O Capital, e uma preocupação muito maior com a capacidade da economia de acumular a massa de mais-valia que foi capaz de produzir, e que precisa fazê-lo para continuar a se expandir. Por outras palavras, há um foco maior na questão de saber se a acumulação pode ser sustentada do que em saber se ela gera uma taxa de lucro mais alta ou mais baixa. Por exemplo, se o progresso técnico reduz os valores do capital constante e variável, como tende a fazer, isso é indicativo da tradução de mudanças nas condições de produção para a esfera da troca, o que gera uma tendência à queda das taxas de lucro (para o medida em que o valor da força de trabalho seja sustentado e os salários reais aumentem em linha com a acumulação e a produtividade). Em contraste, a formação de sociedades por acções, a super-exploração dos trabalhadores e a abertura do comércio externo conduzem a uma acumulação contínua, independentemente da taxa de lucro a que ocorrem.
 \par 
\section{Resumo do argumento}
 \par 
A consideração da LTRPF como uma lei abstrata não nega o seu significado empírico. A principal conclusão de Marx nesta parte de O capital é que a lei e as TC não podem existir lado a lado em harmonia indefinidamente, mas devem por vezes dar origem a crises. Isto requer uma interpretação cuidadosa, pois não existe uma derivação axiomática da necessidade de crises, tal como não existe uma derivação axiomática de uma taxa de lucro decrescente. Pelo contrário, Marx está a apontar para a possibilidade imanente de crises, tal como fez no Volume {\color{blue}2} de O Capital, como resultado da potencial disjunção entre venda e compra com base em valores imutáveis ​​(ver Capítulo {\color{blue}7}). Isto pode ser estabelecido, como na teoria keynesiana da procura ineficaz, sem referência ao capitalismo, a não ser como um sistema de ofertas e procuras coordenado pelo dinheiro. Mas no Volume {\color{blue}1} de O Capital, Marx estabeleceu não só que a acumulação é um imperativo para o capitalismo, mas também que envolve processos de reestruturação económica e social que devem ter no seu cerne a reprodução económica simples e expandida, tal como estabelecido no Volume {\color{blue}2}. Por outras palavras, a troca não é simplesmente nem principalmente uma coordenação de mercados, mas é a expressão mais aberta das contradições da acumulação de mais-valia.
 \par 
Para o LTRPF, uma fonte potencial de disjunção na circulação do valor (excedente) é a acomodação em troca tanto da expulsão relativa do trabalho como da mudança de valores devido à reestruturação do capital. Estes processos estão sujeitos a interrupções incessantes devido às mudanças técnicas em toda a economia. Por exemplo, a redução dos valores à medida que a acumulação prossegue mina a preservação dos valores do capital, enquanto a expulsão do trabalho perturba os equilíbrios entre a oferta e a procura, a extracção de mais-valia e a reprodução da força de trabalho.
 \par 
Estas perturbações demonstram que o LTRPF e os CTs têm uma ligação direta com fenómenos observáveis, embora não envolvam simples previsões de tendências. Em vez disso, fornecem um quadro para a compreensão das tensões e deslocamentos devidos à acumulação de capital, apoiando a conclusão de que a lei e as TC não podem coexistir lado a lado em repouso: os capitais são desvalorizados mesmo quando são preservados e expandidos. Estas contradições dão origem a crises, booms e ciclos de produção e troca. Além disso, o desenvolvimento da possibilidade imanente de crise aponta para a probabilidade de crise quando estes processos já não podem ser acomodados, especialmente (mas não exclusivamente) devido a desproporções, investimentos equivocados e bolhas especulativas. Estas crises, e o resultante desemprego, concentração e centralização do capital, e assim por diante, são as “previsões” que decorrem da análise de Marx do LTRPF. Os ciclos correspondentes estão associados a movimentos observáveis ​​na taxa de lucro. Estes movimentos não são arbitrários, mas baseiam-se nas tendências abstratas e nas suas contradições.
 \par 
Esta análise leva a implicações empíricas adicionais do LTRPF, pois sugere que as crises que devem as suas origens aos desenvolvimentos na esfera da produção irão, no entanto, romper a esfera da circulação, e podem fazê-lo de formas surpreendentes, dependendo da situação relativa. pontos fortes e facilidades dos participantes na circulação do capital como um todo. Esta é uma das razões pelas quais o LTRPF é susceptível de conduzir empiricamente a quedas reais na taxa de lucro: à medida que o processo de acumulação vacila, a massa de lucro realizada é colocada contra uma massa imutável de capital fixo e a rentabilidade tende a diminuir. Mas não precisa ser assim. Se, por exemplo, como resultado da estagnação económica ou de falências, grandes massas de capital forem depreciadas ou compradas pelos capitalistas sobreviventes a preços baixíssimos, a taxa de lucro poderá aumentar, um factor que muitas vezes desempenha um papel importante na recuperação económica.
 \par 
\section{Resumo do argumento}
 \par 
O ponto anterior ilustra que a queda da taxa de lucro tem sido uma espécie de fetiche na literatura. Muitas vezes, o foco tem sido saber se a teoria pode produzir uma queda na taxa de lucro, por qualquer mecanismo, seja um aumento do OCC, do VCC ou dos salários (à custa dos lucros). Quando a taxa de lucro cai, presume-se que a economia entra em crise devido ao investimento deficiente, o que, por sua vez, conduz a uma procura deficiente do produto potencial, como na teoria keynesiana. Nesta perspectiva, existe uma separação completa entre a teoria que dá origem à queda da rentabilidade e os resultados dessa queda, ou seja, entre a causa e o curso da crise (e, ainda mais longe, o mecanismo de recuperação - que, em A análise de Keynes depende de um deus ex machina, dos gastos deficitários do Estado e do seu impacto sobre as expectativas capitalistas). Contudo, não se pode presumir que uma queda na rentabilidade resulte automaticamente numa crise. Poderá haver um incentivo e uma capacidade de acumulação reduzidos; mas alguma recompensa é melhor que nenhuma. A acumulação contínua pode ser necessária para preservar o capital (fixo) existente e pagar dívidas existentes; e, mais importante ainda, a queda da rentabilidade é uma poderosa força competitiva. Consequentemente, à medida que os capitalistas tentam restaurar a rentabilidade, poderão até acumular mais rapidamente do que anteriormente!
 \par 
Para Marx, as quedas na taxa de lucro podem desencadear crises económicas (por exemplo, as falências industriais podem levar à falência de bancos e a uma crise de crédito), mas isto oferece mais uma descrição do que uma análise penetrante da causa última e do curso das crises. Mais importante ainda, não demonstra a relação orgânica entre a crise e a acumulação de capital, excepto de forma trivial, ao implicar que uma economia de mercado descoordenada é incapaz de alcançar um crescimento equilibrado a longo prazo. Em contraste, se o LTRPF for entendido como a combinação de tendências contraditórias que operam na produção e na troca, as crises podem ser analisadas com base nas características fundamentais do processo de acumulação de capital.
 \par 
Isto requer uma análise da produção de valor e da sua expressão na troca num contexto muito mais amplo do que aquele apresentado nos capítulos iniciais do Volume {\color{blue}1} de O Capital. Ali, o valor é entendido como uma relação social que expressa a equivalência entre diferentes tipos de trabalho, através da categoria de trabalho abstrato. Em cada economia capitalista, existirão diferentes competências e tipos de trabalho. Dentro de cada setor, também existirão empresas concorrentes com diferentes níveis de produtividade. O imperativo do lucro, o controlo capitalista sobre o processo de trabalho, a concorrência dentro e entre sectores e a equivalência de mercadorias na troca reduzem estes trabalhos ao denominador comum de valor (ver Capítulos {\color{blue}2} e {\color{blue}3}). Com a acumulação e a competição para reduzir os valores das mercadorias, o tempo de trabalho socialmente necessário (SNLT) em cada sector torna-se o centro em torno do qual giram os processos individuais de trabalho e acumulação.
 \par 
O reconhecimento da interação entre a lei e as TC levanta problemas difíceis para a teoria do valor, que só podem ser resolvidos através de uma compreensão cada vez mais complexa e concreta do valor. Por exemplo, uma vez que a acumulação leva à redução contínua do SNLT, o conceito de valor parece estar em risco, pois a sua quantificação é perturbada assim que é estabelecida. A única forma de resolver esta dificuldade é através do reconhecimento de que a equivalência entre diferentes tipos de trabalho se estende a trabalhos de produtividade diferente. Já ilustramos dois exemplos desse processo. Primeiro, os insumos fabricados em diferentes momentos e com diferentes tecnologias são transformados pelo trabalho vivo em novos produtos que, por sua vez, são frequentemente consumidos de forma produtiva como insumos noutro processo de produção. Consequentemente, a equivalência material entre diferentes tipos de trabalho e entre trabalhos de produtividade diferente é geralmente estabelecida na produção e não na troca. Em segundo lugar, o OCC é determinado com base na equivalência baseada em valores previamente estabelecidos, enquanto o VCC é formado através do surgimento de novos valores determinados pelas mudanças nas condições de produção associadas ao aumento do OCC.
 \par 
Isto é tudo o que se pode dizer sobre a dinâmica da taxa de lucro geral a este nível de análise, e nenhum progresso adicional pode ser feito sem especificar a natureza da interacção entre a lei e as CT. Isto pode ser feito teoricamente, através da análise dos mecanismos pelos quais as relações de valor são expressas na troca, ou empiricamente, especificando as condições em que a acumulação ocorre historicamente. Dois factores importantes em ambos os aspectos da análise da rentabilidade são o papel das finanças e o papel do capital fixo. À sua maneira, ambos são extremamente influentes e diretamente afetados pelo estabelecimento da equivalência de valor na troca, à medida que os capitais procuram preservar e transmitir as mudanças de valores durante um período prolongado, durante o qual são suscetíveis de serem confrontados competitivamente por substitutos mais baratos. e concorrentes mais produtivos. Esses tópicos não podem ser abordados aqui, mas veja o Capítulo {\color{blue}3} e outras leituras.
 \par 
\section{Resumo do argumento}
 \par 
A crítica mais conhecida da teoria de Marx sobre o LTRPF toma como ponto de partida um teorema apresentado e reproduzido em forma matemática pelo economista japonês Nobuo Okishio. De forma breve e informal, Okishio argumenta que, dada uma disponibilidade mais ampla de técnicas de produção, a taxa de lucro não pode cair a menos que os salários reais aumentem. Em outras palavras, uma taxa de lucro em queda depende do aumento dos salários, em vez de ser o resultado de contradições internas ao processo de acumulação de capital, como Marx considera ser o caso. Na análise de Okishio, os capitalistas adotarão novas técnicas de produção somente se estas forem mais lucrativas do que as técnicas existentes, dados os preços das commodities e o nível de salários prevalecentes. Uma vez que essas novas técnicas sejam generalizadas, isso resultará em um novo conjunto (menor) de preços e uma nova taxa de lucro, equalizada entre os setores. Os preços mudarão não apenas nos setores onde houve inovação, porque esses preços mais baixos serão repassados ​​aos setores nos quais essas commodities são usadas como insumos ou como parte do salário. Neste caso, a questão de Okishio é a seguinte: poderiam os capitalistas, agindo cegamente para aumentar a lucratividade individual introduzindo novas técnicas, paradoxalmente levar o sistema a uma taxa de lucro menor? Sem surpresa, ele chega a uma resposta negativa, a menos que os salários reais aumentem essencialmente proporcionalmente mais do que o aumento da produtividade, e conclui que a análise de Marx do LTRPF está incorreta.
 \par 
O teorema de Okishio é um exercício de estática comparativa, ou seja, compara uma posição de equilíbrio económico com outra, embora a estática comparativa seja inadequada para a análise de alterações na taxa de lucro como fonte de crises. Por outras palavras, se passarmos de uma posição de equilíbrio (estático) para outra, não podemos analisar as crises independentemente do que acontece à taxa de lucro, uma vez que estamos apenas a comparar o que consideramos ser um equilíbrio com outro. No entanto, Okishio chega à sua conclusão com base nos pressupostos, em primeiro lugar, de que a economia se move de uma posição de equilíbrio estático para outra; e em segundo lugar, implicitamente, que se a taxa de lucro cair (devido a aumentos salariais demasiado elevados) teremos uma crise, mas, caso contrário, não teremos. No entanto, não fica claro por que razão uma taxa de lucro de equilíbrio mais baixa entraria em colapso numa crise, especialmente porque mesmo uma taxa mais baixa é preferível a um colapso económico.
 \par 
Isto levanta a questão muito mais interessante do movimento entre os dois equilíbrios. Ao examinar este processo, torna-se evidente que, longe de interpretar a LTRPF de Marx, a abordagem associada a Okishio é o seu oposto. Pois, na abordagem de Okishio, um capitalista individual adopta inicialmente uma técnica de produção mais vantajosa através de um acesso superior ao financiamento ou à tecnologia e, aos preços iniciais, este capitalista obtém uma taxa de lucro mais elevada. Esta abordagem contrasta fortemente com a análise de Marx sobre o crescente OCC. Para Marx, como foi demonstrado acima, a tendência para a queda da rentabilidade deve-se à avaliação dos factores de produção e dos produtos em valores antigos, o que se aplica ao capital como um todo.
 \par 
Consideremos agora, no contexto do teorema de Okishio, as consequências da generalização da nova técnica para todos os capitais do sector, e a formação de novos preços e taxas de lucro de equilíbrio. Pode ser demonstrado matematicamente que a taxa de lucro de curto prazo do capitalista inovador é maior do que a nova taxa de “equilíbrio” de longo prazo (após a difusão da mudança técnica) que, por sua vez, é maior do que a taxa de “equilíbrio original”. taxa de lucro (antes da mudança técnica). Isto implica que o capitalista que adquiriu uma vantagem através da inovação técnica descobre que essa vantagem se desgasta à medida que a inovação se generaliza. Isto é, a redução dos preços através da introdução da nova técnica acaba por reduzir a taxa de lucro do capitalista inovador. Portanto, para Okishio, a formação de preços a partir da mudança técnica actua para o capitalista inovador individual como uma pressão que reduz a taxa de lucro em direcção à média (nova e mais elevada do que antes). Em contraste, para Marx, o processo de formação de preços (e VCC) resultante da mudança técnica é uma tendência contrária à queda da rentabilidade do capital como um todo, uma vez que leva a uma redução no valor do capital constante e variável.
 \par 
Introduzir nova tecnologia e generalizá-la a outros produtores para formar novos preços. Para Okishio, esses processos são fenômenos de equilíbrio empírico imediato. Eles não interagem entre si para produzir resultados mais complexos e concretos; em vez disso, são simplesmente somados algebricamente para mostrar um aumento na rentabilidade da economia como um todo, de um equilíbrio para o seguinte. Além disso, os dois processos de desequilíbrio cancelam
 \par 
Agora junte os dois processos, um ao outro como processos de mudança e deixe o sistema em equilíbrio harmonioso. Por causa disso, a abordagem Okishio não consegue distinguir entre o VCC e o OCC. Em vez disso, baseia-se exclusivamente numa noção de equilíbrio do VCC que, no entanto, recebe o nome de composição orgânica. Em contraste, para Marx, a lei e as TC são tendências abstratas cuja interação não é uma soma algébrica, mas um caminho de acumulação assolado por crises que pode ser compreendido, mas nem sempre antecipado. O resultado de Okishio é poderoso apenas no sentido limitado de que a taxa de lucro pode cair se os salários aumentarem suficientemente (mais do que o suficiente para compensar o impacto dos aumentos de produtividade na rentabilidade). Contudo, a taxa de lucro pode cair por outras razões não relacionadas com os movimentos dos salários; por exemplo, se a economia sofrer um choque externo adverso (uma deterioração nos termos de comércio, por exemplo, devido ao aumento dos preços das importações), uma crise financeira (atualmente pertinente à luz da estagnação dos salários nas últimas três décadas ou mesmo mais) ou qualquer perda de confiança empresarial. Isto sugere que temos de localizar o impacto dos salários como (no máximo) uma influência próxima na rentabilidade, bem como na acumulação (lembrando que as análises do tipo Okishio são inteiramente estáticas). Pois os salários, em Marx, são uma consequência do processo de acumulação e não algum tipo de influência independente. Especificamente, embora salários mais elevados possam precipitar uma crise, a acumulação de capital também pode prosperar com o aumento dos salários reais, porque conduzem a níveis mais elevados de consumo e vendas. Em contraste, se os salários reais permanecerem os mesmos apesar do progresso técnico, haverá uma redução no valor da força de trabalho e um aumento na taxa de mais-valia. Estes são CTs para Marx. O facto de existirem, como resultado da acumulação, não garante a ausência de crise. Embora estes resultados sejam sempre possíveis no contexto da análise de Marx do LTRPF e dos CT, eles são excluídos pelo estreito interesse de Okishio no trade-off lucro-salário.
 \par 
A actual crise financeira global demonstra como a queda da rentabilidade e a crise podem resultar independentemente, ou mesmo apesar, da estagnação dos salários reais (ver Capítulo {\color{blue}14}). Assim, o teorema de Okishio, na melhor das hipóteses, só pode ser resgatado aceitando que não se aplica nestas circunstâncias. Em contraste, a LTRPF e as CT de Marx aplicam-se, são diferentes em método, âmbito e conteúdo, e não são invalidadas por Okishio. Porque visam as contradições (e a possibilidade de crise) inerentes à acumulação e circulação do capital como um todo, para as quais o aumento dos salários reais é apenas uma parte que precisa de ser adequadamente localizada analiticamente, em vez de ser tomada como um factor exógeno e fator independente.
 \par 
\section{Resumo do argumento}
 \par 
As questões em torno do LTRPF foram abordadas no texto. Marx desenvolve sua análise em Marx (1981a, pt.{\color{blue}3}). A exposição neste livro baseia-se em Ben Fine (1982, cap.{\color{blue}8} e, especialmente, 1992a) e Ben Fine e Laurence Harris (1979, cap.{\color{blue}4}). Para interpretações semelhantes, ver Duncan Foley (1986, cap.{\color{blue}8}), Geert Reuten (1997), Roman Rosdolsky (1977, cap.{\color{blue}26}) e John Weeks (1982a). A crítica de Nobuo Okishio (1961) a Marx atraiu enorme atenção - ver, por exemplo, Research in Political Economy (vol. {\color{blue} 18 } {\par} , 2000); mas veja também o reconhecimento de Okishio (2000) das limitações no seu artigo original (incluindo alterações propostas, que não conseguem, no entanto, resolver os problemas identificados neste capítulo). Para uma revisão mais ampla do LTRPF em Marx, ver Reuten e Thomas (2011).