\chapter{15 O marxismo e o século XXI}\label{15 O marxismo e o século XXI}
 \par 
A popularidade e a proeminência do marxismo aumentam e diminuem com as modas intelectuais e com o ritmo dos acontecimentos mundiais. Estas duas influências estão longe de serem independentes uma da outra; e, além disso, o que se entende ser o conteúdo e a ênfase do marxismo é igualmente variável ao longo do tempo, lugar e contexto. O marxismo pode ser visto, por um lado, como uma crítica do capitalismo (uma posição que está actualmente em primeiro plano na suposta era da globalização); ou, por outro lado, pode ser interpretado como fornecendo alternativas ao capitalismo, como nos casos dos (anteriormente) “países socialistas” ou nas lutas actuais de alguns países pós-coloniais. O marxismo também tem estado fortemente envolvido em todos os principais debates académicos nas ciências sociais, embora, mais uma vez, o peso e o conteúdo da sua presença tenham sido diversos e desiguais ao longo do tempo, tópico e disciplina.
 \par 
O objetivo deste capítulo final é defender a importância contínua da economia política de Marx para o estudo de questões contemporâneas. Necessariamente, só pode ser sugestivo e limitado na cobertura, bem como direcionado para tópicos que desempenharam um papel significativo no desenvolvimento do marxismo. Um ponto de partida apropriado é o grande ataque académico feito contra o marxismo no Ocidente desde o seu último pico de popularidade durante as décadas de 1960 e 1970. Para além de promover a ideia mítica de que o Keynesianismo tinha resolvido mais ou menos o problema das crises capitalistas (que foi mais tarde descartada, pelo menos em teoria, sob o neoliberalismo), o anti-Marxismo floresceu ao sugerir que o Marxismo é rude e doutrinário. Duas questões intimamente ligadas vêm à tona - uma diz respeito à natureza da classe e a outra à natureza do Estado (capitalista). As preocupações com o meio ambiente e as consequências do capitalismo também são examinadas abaixo.
 \par 
\section{Aula}
 \par 
A principal crítica feita ao marxismo no que diz respeito à classe é a sua suposta incapacidade de lidar com a complexidade e a diversidade das relações de classe dentro da sociedade capitalista avançada, diversamente apelidada de pós-industrial, democrática, de bem-estar social, essencialmente de "classe média", meritocrática e assim por diante. sobre. A crítica tem dois componentes separados, um relativo à estrutura de classes e outro relativo às implicações dessa estrutura. Em suma, e em parte porque Marx alegadamente previu uma crescente polarização na estrutura de classes (incluindo, erradamente, a presunção da pauperização “absoluta” dos trabalhadores), argumenta-se que a divisão entre a burguesia e o proletariado é demasiado grosseira e, não menos importante, devido às aspirações revolucionárias de Marx para a classe trabalhadora, a acção de classe e a ideologia presumivelmente não conseguiram corresponder às suas expectativas baseadas nesta estrutura de classe postulada. Por exemplo, porque é que os trabalhadores assalariados votam em governos de direita e porque é que os governos conservadores introduzem reformas que beneficiam os trabalhadores? Essas questões são abordadas abaixo. A nível metodológico, são expressas preocupações tanto sobre a estrutura da teoria de Marx como sobre o seu conteúdo causal. Por exemplo, é considerado demasiado determinista e reducionista - supostamente implica que tudo flui do económico, com o próprio económico identificado principalmente com a produção e as relações de classe e com a evolução do capitalismo conduzindo inevitavelmente à supremacia numérica e à hegemonia política do o proletariado, ele próprio dominado pela coorte de trabalhadores industriais (predominantemente masculinos).
 \par 
Não há dúvida de que muitos marxistas têm sido culpados destes pecados analíticos de simplificação excessiva e de omissão de outros factores, ainda que em parte na tentativa de expor as falácias da “liberdade”, da “eficiência” e da “igualdade” que são muito facilmente apresentadas como virtudes de capitalismo. Esperemos, porém, que já tenha sido apresentado neste livro o suficiente da economia política e do método de Marx para mostrar que o próprio Marx não poderia ser acusado destas deficiências. Na verdade, Marx certa vez declarou-se não marxista, tendo em conta a forma como o seu método tinha sido abusado durante a sua vida!
 \par 
Mais especificamente, no caso da classe, a economia política de Marx revela a componente crucial e central da estrutura de classes do capitalismo: que o capital e o trabalho se confrontam necessariamente durante a compra e venda da força de trabalho. Além disso, tal como apresentado neste livro, a economia política de Marx preocupa-se com as consequências desta estrutura de classes para a acumulação, reprodução, desenvolvimento desigual, crises, e assim por diante. Assim, longe de reduzir todos os outros fenómenos económicos e sociais a tal análise, a economia política de Marx abre o caminho para uma investigação mais ampla, sistemática e mais complexa da estrutura, das relações, dos processos e das consequências do capitalismo - e o que isto consegue é muito e de crucial importância.
 \par 
Assim, a economia política de Marx não reduz a estrutura de classes à do capital e do trabalho. Pelo contrário, localiza outras classes em relação ao capital e ao trabalho, seja como partes essenciais ou contingentes do modo de produção capitalista. Dentro do próprio capitalismo, por exemplo, o marxismo mostra como é criado espaço para o surgimento de trabalhadores independentes e para os “profissionais” prosperarem porque, por diferentes razões, eles podem reter todos os frutos do seu trabalho apesar de receberem um salário ou, mais exatamente, um salário - embora este possa assumir diferentes formas, incluindo taxas, comissões e assim por diante. Formalmente, isto pode ser representado pela ideia de que tais estratos recebem a recompensa total pelo seu trabalho vivo, l = v + s, em vez de uma remuneração baseada no valor da força de trabalho, v. , e as actividades e condições de trabalho que lhes estão associadas, não são apropriadas pelo capital e reduzidas em termos de competências ou estatuto social ao nível do trabalho assalariado.
 \par 
Podem ser apresentados vários argumentos gerais, alguns estruturais e outros contingentes. Por exemplo, uma pré-condição para o capitalismo avançado é a emergência de sistemas sofisticados de crédito e comerciais, nos quais podem ser acumuladas recompensas consideráveis ​​para aqueles que mobilizam e alocam fundos e mercadorias em nome de outros. O mesmo se aplica às profissões necessárias para facilitar ou salvaguardar a circulação do capital e a sua reprodução social em geral, embora estas actividades variem em peso e significado ao longo do tempo e do lugar e, onde as associações profissionais se revelam ineficazes, estão sujeitas à proletarização. Afinal, existem enormes diferenças entre o trabalhador ocasional da construção civil “autônomo” ou o faxineiro contratado e o médico especialista ou consultor de gestão.
 \par 
Finalmente, e com base no acima exposto, o que é percebido como o maior desafio para a economia política de classe é a ascensão da classe média, ela própria um estrato altamente diversificado em termos da sua composição e características. O capitalismo avançado testemunhou o declínio dos trabalhadores industriais e a ascensão dos serviços, significativamente aqueles empregados pelo Estado e, portanto, potencialmente afastados da motivação e do cálculo comercial directo. Em suma, será que o crescente exército de trabalhadores da saúde, da educação e de outros trabalhadores empregados pelo Estado mina a análise baseada numa estrutura de classes baseada no capital e no trabalho?
 \par 
Colocar o problema nestes termos aponta para a relevância contínua da classe económica no capitalismo contemporâneo, com o trabalho definido em termos da sua dependência de um salário. Isto não significa negar que a classe de trabalho é fortemente diferenciada dentro de si mesma - por sector, habilidade (manual e mental), processo de trabalho; entre indústria e comércio; entre os setores público e privado, e assim por diante. Tais diferenciações não invalidam o conceito de classe, mas realçam que os interesses e acções de classe não podem sempre, ou mesmo predominantemente, existir como consequências imediatas da estrutura de classe. Pelo contrário, os interesses de classe são formados económica, política e ideologicamente através de relações económicas concretas e de circunstâncias históricas. Assim, não se trata de enquadrar um ou outro indivíduo nesta ou naquela classe com base nas suas características individuais - trabalhadores manuais, sindicalistas, membros de partidos operários, etc. - mas sim de traçar as relações através das quais o a classe trabalhadora é reproduzida concretamente e representada nas relações materiais e ideológicas. Nesta base, não pode haver nenhuma presunção de uma correspondência clara ou fixa entre características económicas e outras características sociais, mas estas também não são independentes umas das outras. O facto de a classe trabalhadora (isto é, os assalariados em geral, e não o subconjunto muito mais restrito de trabalhadores industriais operários) depender dos salários para a sua reprodução condiciona todos os aspectos da vida social contemporânea, mesmo quando parece ser o contrário; mas os salários e as condições sociais também não estão sujeitos a uma determinação férrea em termos de incidência e conteúdo.
 \par 
\section{Aula}
 \par 
Estas observações gerais sobre classe têm relevância para a teoria do Estado capitalista. Mais uma vez, o marxismo foi sujeito a críticas sob a forma de paródia, com a sua teoria do Estado percebida como reduzida à simples proposição de que o Estado serve apenas a classe dominante e, portanto, os interesses capitalistas. Isto abre-se imediatamente à objecção de que o Estado implementa frequentemente políticas que beneficiam os trabalhadores, especialmente através da prestação de assistência social. O marxismo é então grosseiramente retratado como defendendo-se através da compreensão da reforma como uma estratégia tortuosa por parte da classe dominante para antecipar a revolução - onde de outra forma não está a garantir uma classe trabalhadora mais capaz de produzir (e travar guerras) em seu nome.
 \par 
Tal como anteriormente, os registos históricos não conseguem confirmar motivos tão simples para o calendário e o conteúdo da reforma, nem são suficientes para explicar a prestação de cuidados de saúde, educação, pensões, etc. produtividade do trabalho. Outra deturpação popular da teoria marxista é considerar o Estado («relativamente autónomo») como essencial na mediação entre interesses conflituantes dentro da classe capitalista, e não entre capital e trabalho. Neste caso, a principal função do Estado é evitar que os capitalistas enganem uns aos outros e que a intensidade da concorrência seja indevidamente disfuncional. Tal como a teoria do Estado como instrumento de uma classe contra outra, esta abordagem lança apenas uma luz limitada sobre a complexidade e a diversidade do papel e das ações do Estado.
 \par 
O problema em cada um desses casos é que o estado é visto como uma instituição internamente homogênea, claramente separada do "mercado", e um instrumento que serve a interesses prontamente identificáveis ​​- do capital contra o trabalho, ou para o capital como um todo contra as inclinações destrutivas de seus elementos individuais, ou mesmo para "a nação" contra nações e capitais rivais. Mas tais interesses não existem e nem sempre podem existir em formas tão altamente abstratas e ainda assim prontamente reconhecíveis. Em vez disso, classes e interesses de classe são formados por meio de ações econômicas, políticas e ideológicas, condicionadas, mas não rigidamente determinadas pela acumulação e reestruturação do capital e pelos padrões de reprodução social dos quais a formação de classes depende em maior ou menor grau e de diversas maneiras. (Esses padrões incluem estruturas de emprego, condições de trabalho, sindicatos e outras formas de atividade, e reprodução diária em casa, no local de trabalho e em outros lugares.)
 \par 
Em cada uma destas áreas, o Estado capitalista ocupa um papel cada vez mais central. A circulação de capital cria uma esfera de actividade económica que é estruturalmente separada da não-económica, mas simultaneamente dependente dela e que a apoia. A observância complacente das relações de propriedade por parte dos trabalhadores e a legitimação das desigualdades económicas e outras precisam ser reproduzidas pelo menos tanto quanto as relações de valor imediatas. Assim, a necessidade estrutural do Estado capitalista é criada em grande parte pelo seu papel não económico, na reprodução social e não na reprodução económica, mas em conjunto com ela. Mesmo assim, o Estado está sempre fortemente e directamente envolvido na vida económica do capitalismo - apropriando-se e desembolsando valor (excedente) através de impostos e despesas, regulando a acumulação, reestruturando o capital à medida que este atravessa os seus padrões cíclicos, manipulando as taxas de câmbio através de medidas monetárias e outras. políticas macroeconómicas e influenciar as relações distributivas através de políticas fiscais, de despesas e de rendimentos.
 \par 
Infelizmente, estas percepções extremamente importantes do marxismo têm sido frequentemente ignoradas, mesmo quando Marx tem sido elogiado pela sua visão ao antecipar a globalização ou por reconhecer processos semelhantes numa fase histórica anterior. Certamente Marx enfatiza o carácter internacional do capitalismo e a sua busca incessante de lucros onde quer que estes possam ser encontrados. Isto cria afinidades com aqueles que entendem a globalização em termos do desaparecimento do Estado-nação, à medida que este supostamente se torna cada vez mais impotente contra um capital internacionalmente móvel que percorre o mundo sem esforço através do comércio electrónico (e impõe globalmente os valores culturais dos EUA através dos meios de comunicação social).
 \par 
Qualquer que seja o nível de internacionalização do capital nas suas três formas (dinheiro, mercadorias e produção), a reprodução não económica do capitalismo requer inevitavelmente e até fortalece o papel do Estado-nação, embora a pressão para se conformar aos imperativos unidimensionais do comércio não leva à uniformidade. Num certo sentido, isto foi reconhecido por aqueles que se opõem à “globalização”, apontando e apresentando alternativas às suas manifestações deletérias. No entanto, tais opiniões permanecem limitadas, sendo o capitalismo muitas vezes entendido apenas como globalização - a partir da qual todas as suas consequências nefastas podem ser facilmente deduzidas e, em princípio, corrigidas através da implementação de políticas “adequadas”. No entanto, a globalização, seja qual for o aspecto e como for entendida, deve ser vista como o efeito da reprodução internacional do capitalismo e, consequentemente, como a forma assumida pelas leis da economia política no período actual. Em suma, qualquer que seja o significado atribuído à globalização na sua aplicação nos aspectos económicos, políticos e ideológicos, a sua ligação fundamental à produção e apropriação de mais-valia precisa de ser sustentada analiticamente.
 \par 
\section{Aula}
 \par 
Consideremos agora o problema da degradação ambiental. Aqui o marxismo foi acusado de privilegiar o social em detrimento do natural, de subestimar o potencial de reforma e até de excluir a consideração do natural devido à preocupação excessiva com o económico. Embora Marx tivesse muito a dizer sobre o que hoje chamaríamos de “meio ambiente”, ele raramente o abordava diretamente. Mas as suas teorias do fetichismo da mercadoria e do processo de trabalho oferecem excelentes insights sobre a sua ênfase nos factores sociais e materiais, uma vez que a produção de valor é sempre, simultaneamente, a produção de valores de uso com um conteúdo físico e ambiental.
 \par 
Isto oferece a base para uma abordagem apropriada do ambiente, que deve ser entendido em termos de relações ambientais (e estruturas e conflitos correspondentes) características do capitalismo. Isto contrasta com a ideia de um conflito trans-histórico entre os humanos e os sistemas ecológicos, ou entre o ambiente e a economia. As relações ambientais do capitalismo são impulsionadas pelas relações dominantes de produção. Assim, como é facilmente reconhecido, o impulso para a rentabilidade leva, através da crescente composição orgânica do capital, à transformação de cada vez mais matérias-primas em mercadorias e à correspondente extracção e utilização de energia e minerais, sem ter em conta de imediato as consequências ambientais resultantes. impacto.
 \par 
No entanto, o capitalismo também é capaz, sobretudo através do desenvolvimento de novos materiais e da regulação estatal, de moderar ou mesmo reverter, pelo menos em parte, tal degradação ambiental. A este respeito, é importante reconhecer a natureza multidimensional do ambiente e a diversidade de questões e resultados envolvidos: poluição, biotecnologia, medicamentos, vacinas, e assim por diante. Mais uma vez, as lições a retirar do fetichismo da mercadoria são significativas. Marx argumenta que as relações mercantis são relações sociais expressas como relações entre coisas, aparecendo a um nível superficial puramente como magnitudes monetárias, ocultando assim tanto quanto é revelado. O que não é aparente são as relações de classe de exploração subjacentes, a dinâmica a que dão origem e as razões para elas. Da mesma forma, a forma como as mercadorias foram criadas como valores de uso, com a sua correspondente ligação ao ambiente, não nos é revelada mais do que as origens geográficas da mercadoria ou a sua dependência (ou não) do trabalho suado ou infantil - a menos que sejam são abertamente implantados, legitimamente ou não, como um argumento de venda.
 \par 
Não é de surpreender que esses aspectos ‘ocultos’ da mercadoria e seus sistemas de produção, distribuição e troca sejam inevitavelmente trazidos à nossa atenção de tempos em tempos, induzindo reações contra eles. Lutas contra o trabalho infantil, a fim de revelar sua incidência e fazer campanha contra ele do ponto de produção até o ponto de venda, são, afinal, direcionadas à natureza da humanidade e sua reprodução em aspectos materiais e culturais. Da mesma forma, a reprodução das relações ambientais, otimistamente apelidada de ‘sustentabilidade’, é inevitavelmente um confronto mutável com uma série de aspectos das relações capitalistas de mercadorias. Enquanto essas relações persistirem, o mesmo acontecerá com o sistema de produção ao qual estão vinculadas, com as tendências correspondentes de apropriação, transformação e degradação do meio ambiente - por mais que isso possa ser temperado pela regulamentação, que tende a ser obstruída ou evitada por pressões competitivas.
 \par 
\section{Aula}
 \par 
O que é o socialismo e oferece melhores perspectivas em aspectos sociais, ambientais e outros? As experiências socialistas no século XX associaram-se intimamente a Marx (ismo) e foram vistas como marxistas no entendimento popular. No entanto, muito antes do colapso do bloco da Europa de Leste, a controvérsia já existia entre os marxistas sobre a natureza da União Soviética, com posições que iam desde o apoio acrítico à condenação como capitalismo (de Estado).
 \par 
No caso, a União Soviética, durante o que é, em termos relativos, um breve período histórico, passou por uma transformação notável, bem captada na noção de acumulação primitiva de Marx. Pois o que era em grande parte uma sociedade semifeudal, com uma grande proporção da sua força de trabalho na agricultura, conseguiu criar, a uma velocidade vertiginosa, um mercado de trabalho assalariado e uma base industrial relativamente avançada e bem integrada. O período desde o colapso da URSS testemunhou a conclusão desta transição através do ressurgimento de uma classe de capitalistas e da propriedade privada da maior parte dos meios de produção. Alguns argumentaram que tal resultado final era inevitável, dada a baixa base produtiva inicial e a implacável hostilidade internacional enfrentada pela União Soviética ao longo da sua história. Mesmo assim, o ritmo, a direcção e as consequências de tal transição para o capitalismo estavam longe de ser predeterminados, como é evidente pela adopção menos cataclísmica, embora igualmente dramática, de um mal denominado “socialismo com características chinesas” na segunda maior economia do mundo.
 \par 
Embora Marx seja bem conhecido pelas suas críticas ao capitalismo como um sistema explorador, é provavelmente também frequentemente considerado como tendo inspirado tentativas falhadas de construção do socialismo. Embora haja pouco trabalho de Marx que trate direta e exclusivamente da economia do socialismo, Marx, contrariamente a muitas opiniões, tem muito a dizer sobre o tema, nomeadamente na Crítica do Programa de Gotha. Geralmente, ele está menos interessado em conceber planos utópicos do que em basear-se e extrapolar os desenvolvimentos dentro do próprio capitalismo, procedendo de duas maneiras distintas, mas intimamente relacionadas.
 \par 
Primeiro, ele vê o capitalismo como uma vida cada vez mais socializante - por meio da organização da produção, da economia de modo mais geral e por meio do poder estatal - mas de maneiras que são fundamentalmente restringidas pela natureza privada do mercado, propriedade privada e o imperativo da lucratividade. A competição tende a socializar a produção capitalista por meio da divisão cada vez mais intrincada do trabalho no chão de fábrica e na sociedade como um todo. Além disso, o papel crescente do Estado na provisão de bem-estar, redistribuição e produção em si, por meio do planejamento ou indústrias nacionalizadas, por exemplo, todos antecipam algumas das formas econômicas e sociais de um futuro socialismo. O mesmo se aplica à formação de coisas como cooperativas de trabalhadores, com ou sem apoio estatal.
 \par 
No entanto, estas formas embrionárias são inevitavelmente limitadas no conteúdo, na forma e até na sobrevivência pelo seu confinamento na sociedade capitalista, pelo impulso directo ou indirecto à rentabilidade e pelo sistema económico e social que impõe imperativos comerciais a todos. Algumas formas de socialização - o planeamento da produção dentro de empresas de grande escala, com exclusão do mercado, ou o papel mais amplo e profundo do dinheiro através do sistema financeiro - têm uma afinidade muito diferente com o socialismo do que a provisão de saúde, educação e bem-estar por parte do Estado. A este respeito, o slogan popular “o povo antes do lucro” expressa valores socialistas dentro de uma aceitação do capitalismo, uma vez que o lucro é permitido desde que não seja privilegiado. Aqui há uma correspondência nítida com a crítica de Marx à noção de Proudhon de que “propriedade é roubo”, pois Proudhon condena e aceita a propriedade (sem a qual não pode haver roubo).
 \par 
Em segundo lugar, então, a antecipação do socialismo por Marx deriva das contradições dentro do capitalismo, independentemente de estas terem evoluído para formas socialistas embrionárias. O mais notável é o papel revolucionário a ser desempenhado pela classe trabalhadora, com o capitalismo a criar, expandir, fortalecer e organizar o trabalho para fins de produção, mas necessariamente explorando a maioria trabalhadora e não conseguindo satisfazer as suas aspirações e potencial. Na frase reveladora do Manifesto Comunista, “o que a burguesia… produz, acima de tudo, são os seus próprios coveiros. A sua queda e a vitória do proletariado são igualmente inevitáveis.’
 \par 
Esses são os meios para a revolução socialista. A motivação surge dos vários aspectos da exploração, alienação e degradação humana característicos do capitalismo, e da forma como podem ser superados. Sob o capitalismo, a classe trabalhadora está privada do controlo do processo de produção, dos seus resultados nos próprios produtos, e do conhecimento abrangente e da influência sobre o funcionamento da sociedade e o seu desenvolvimento. Os trabalhadores também estão sujeitos a severas limitações nas suas perspectivas e realizações potenciais, e a convulsões contínuas nas suas condições de vida, cuja sorte muda com o fluxo e refluxo do imperativo do lucro e da sorte da economia. Isto representa um grande desperdício em termos económicos e, mais importante, em termos humanos. Isto levou à resistência no local de trabalho e ao confronto político e, historicamente, proporcionou um poderoso estímulo às reformas sociais e à rebelião anticapitalista.
 \par 
Para Marx, a abolição do capitalismo marca o fim da pré-história da sociedade humana. No entanto, a transição para o comunismo não é inexorável nem inevitável. As relações sociais no centro do capitalismo só mudarão se uma pressão esmagadora for aplicada pela maioria. Caso contrário, o capitalismo poderá persistir indefinidamente, apesar dos seus crescentes custos humanos e ambientais. Em todos os casos, a passagem para o socialismo só pode ser alcançada por etapas, em vez de ser magicamente concluída a pedido. A sua primeira fase será inevitavelmente marcada pela influência contínua da pesada bagagem histórica do capitalismo. Marx argumenta que, numa fase posterior, quando a divisão do trabalho e a oposição entre trabalho mental e manual tiverem sido superadas, e o desenvolvimento das forças produtivas tiver atingido um nível suficientemente elevado para permitir o desenvolvimento integral dos indivíduos , a fase avançada do socialismo (comunismo) pode ser alcançada. Como ele disse na Crítica ao Programa de Gotha, “de cada um de acordo com a sua capacidade, a cada um de acordo com as suas necessidades!”
 \par 
\section{Aula}
 \par 
Excelentes estudos marxistas de classe incluem Geoffrey de Ste. Croix (1984) e Ellen Meiksins Wood (1998); ver também os ensaios em Socialist Register (2001, 2014, 2015) e Sam Gindin (2015). As teorias marxistas do Estado são revisadas por Ben Fine e Laurence Harris (1979, capítulos 6, {\color{blue}9}); ver também Simon Clarke (1991), Bob Jessop (1982, 2012) e Ellen Meiksins Wood (1981, 1991, 2003).
 \par 
A “globalização” capitalista é discutida numa vasta literatura. Esta seção baseia-se em Ben Fine (2002, cap.{\color{blue}2}), Alfredo Saad-Filho (2003a) e Alfredo Saad-Filho e Deborah Johnston (2005); ver também Peter Gowan (1999), Hugo Radice (1999, 2000) e John Weeks (2001). Outro conjunto de estudos marxistas refere-se especificamente ao imperialismo; ver, por exemplo, Anthony Brewer (1989), Norman Etherington (1984), Eric Hobsbawm (1987), Socialist Register (2004, 2005) e números recentes de Historical Materialism, Monthly Review e New Left Review. A relação entre neoliberalismo e globalização também é discutida em Gerard Duménil e Dominique Lévy (2004, 2011), David Harvey (2005), Ray Kiely (2005a, 2005b, 2012) e Alfredo Saad-Filho (2003c, 2007).
 \par 
Há uma literatura crescente sobre meio ambiente e crise ambiental. Ver, por exemplo, Ted Benton (1996), Finn Bowring (2003), Paul Burkett (1999, 2003), John Bellamy Foster (1999, 2000, 2002, 2009), Les Levidow (2003), Tony Weis (2007, 2013). ) e Registro Socialista (2007). As revistas Capitalism, Nature, Socialism e Monthly Review incluem uma riqueza de material.
 \par 
Os comentários de Marx sobre socialismo e comunismo podem ser encontrados principalmente em Karl Marx (1974) e Karl Marx e Friedrich Engels
 \par 
(1998); ver também Friedrich Engels (1998, pt.{\color{blue}3}). Este capítulo baseia-se em Ben Fine (1983b). Os debates atuais sobre o socialismo são revistos por Al Campbell (2012), Makoto Itoh (2012), Michael Lebowitz (2003b, 2013), David McNally (2006); ver também Michael Perelman (2000), Socialist Register (2000, 2013) e números recentes da New Left Review e Science & Society. A revista Critique publicou extensivamente sobre a experiência soviética; ver também John Marot (2012) e Marcel van der Linden (2007).