\chapter{12 Capital Bancário e a Teoria dos Juros}\label{12 Capital Bancário e a Teoria dos Juros}
 \par 
A análise de Marx do capital mercantil, explicada no capítulo anterior, baseia-se no papel do dinheiro como meio de troca, isto é, do dinheiro como dinheiro (mesmo que seja utilizado na circulação de mercadorias para obter lucro). Em contraste, a teoria do capital portador de juros (IBC) de Marx baseia-se no papel do dinheiro como capital. Esta teoria diz respeito aos empréstimos e empréstimos que ocorrem entre os capitalistas monetários e os capitalistas industriais ou mercantis. Para Marx, não é o ato de contrair um empréstimo bancário ou o pagamento de juros que caracteriza o IBC, mas o uso que o empréstimo é dado. O empréstimo deve ser utilizado para embarcar num circuito de capital industrial, ou seja, deve ser adiantado como capital monetário. Portanto, poder utilizar o IBC é ser capaz de ser capitalista e não simplesmente ser capaz de contrair empréstimos.
 \par 
Como sujeito de empréstimos e empréstimos nesta relação, o capital monetário torna-se um tipo especial de mercadoria. Fornece o valor de uso da auto-expansão para o credor e para o devedor simultaneamente, o primeiro realizando os juros e o segundo o lucro da empresa que resta ao capital industrial, após o pagamento dos juros, a partir da mais-valia produzida através do uso de o capital monetário emprestado. Marx sublinha que o preço desta mercadoria única (a taxa de juro) é “irracional”, uma vez que não está relacionado com as condições de produção. Depende inteiramente das relações competitivas entre mutuários e credores. Essas questões são exploradas abaixo.
 \par 
\section{Capital com Juros}
 \par 
Duas características distinguem o IBC do capital industrial e mercantil. A primeira diz respeito à utilização de empréstimos e empréstimos (isto é, relações de crédito) especificamente com o propósito de adiantar capital monetário para a apropriação de mais-valia. Estas relações de crédito envolvem as duas frações mais importantes da classe capitalista: os capitalistas monetários, que controlam a oferta do IBC, e os capitalistas industriais, que tomam emprestado o IBC para usar como capital na produção e são responsáveis ​​pelo funcionamento do capital sobre o setor industrial. circuito, supervisionando a produção e, muitas vezes, a venda. A esta divisão da classe capitalista corresponde uma divisão da mais-valia que ela extrai. Tal como explicado acima, enquanto os capitalistas monetários recebem juros, os capitalistas industriais apropriam-se do lucro da empresa que sobra após o pagamento dos juros (a determinação da taxa de juro é discutida abaixo).
 \par 
Em segundo lugar, para a sua existência, o IBC recorre ao capital monetário acumulado através da venda de capital mercadoria, bem como às reservas de dinheiro temporariamente ocioso dos capitalistas industriais e comerciais, dos trabalhadores, do Estado ou de qualquer outra pessoa. Estas reservas e poupanças são recolhidas e centralizadas nas instituições financeiras e transformadas em capital monetário potencial disponível para o capital industrial. A IBC, portanto, desempenha as funções de propriedade e controle do capital monetário em nome do capital como um todo. O IBC não é, no entanto, propriedade jurídica destas instituições e os depositantes têm o direito de levantar os seus fundos (embora diferentes tipos de investimento financeiro possam incorrer em restrições temporárias à capacidade de fazer levantamentos). Os bancos normalmente concedem crédito para além dos seus níveis de depósitos, e esse crédito pode ser utilizado para iniciar novos circuitos de capital.
 \par 
As diferenças entre o capital industrial e o IBC são claramente ilustradas pelos seus respectivos circuitos. Foi demonstrado no Capítulo {\color{blue}4} que o capital industrial é expresso por D - D - D', para o qual o dinheiro intervém nos processos de produção e troca. Em contraste, o IBC é representado por M - M', onde o dinheiro se destaca destes processos.
 \par 
É um tema constante ao longo dos três volumes de O Capital que o acesso ao IBC detém a chave para a acumulação rápida. O aumento no tamanho do capital, frequentemente alcançado por meio de empréstimos, é um dos meios mais importantes de acumulação competitiva. Por exemplo, o processo de centralização pode ser financiado por empréstimos bancários, como em fusões e aquisições, e o tamanho do capital desempenha um papel crítico na busca por aumentos de produtividade por meio da introdução de maquinário mais avançado. É por meio da análise detalhada dessas relações e processos que Marx explica a estrutura do sistema financeiro e sua relação com o capital industrial.
 \par 
\section{Capital com Juros}
 \par 
A distinção de Marx entre capital industrial e IBC nem sempre se traduz perfeitamente na análise empírica, como exemplificado no capítulo anterior para os “híbridos” ligados ao capital mercantil. Isto acontece por duas razões principais.
 \par 
Por um lado, as funções do dinheiro enquanto dinheiro podem ser assumidas por vários instrumentos financeiros - um cartão de crédito, por exemplo, pode servir como meio de pagamento, mas não pode liquidar todas as contas de uma vez por todas. Como resultado, existe uma cascata complexa e sobreposta de instrumentos monetários que servem todas as funções e circunstâncias, com o “dinheiro propriamente dito” - seja o dólar dos EUA ou qualquer outra coisa que seja tão boa como o ouro - no seu auge. Da mesma forma, as atividades associadas ao capital de negociação de dinheiro (MDC), tais como a escrituração contábil, o cálculo e a salvaguarda de uma reserva monetária e o papel de caixa, podem ser realizadas de várias maneiras; por exemplo, internamente (quando as empresas contratam pessoal especializado para especular sobre activos de risco ou movimentos cambiais, ou em mercados de futuros e opções), por empresas especializadas fora do sistema bancário, ou por instituições financeiras. Em termos analíticos, mesmo que estas actividades sejam realizadas internamente pelo capital industrial, são uma função do capital mercantil e atraem a taxa normal de lucro, embora não produzam mais-valia (ver Capítulo {\color{blue}11}).
 \par 
Três distinções analíticas separam o MDC do IBC. Primeiro, o MDC adianta crédito em geral (por exemplo, crédito ao consumo, incluindo cartões de crédito), enquanto o IBC adianta capital monetário, para que o mutuário possa apropriar-se da mais-valia. Em segundo lugar, o MDC baseia-se no lucro industrial (da mesma forma que o capital comercial), enquanto o IBC induz a divisão estrutural da mais-valia em juros e lucro da empresa. Terceiro, o retorno do MDC tende a igualar a taxa de lucro geral. Em contraste, a taxa de retorno do IBC não envolve esta tendência, uma vez que surge da divisão da mais-valia entre juros e lucro da empresa (ver abaixo). Apesar destas diferenças, na sociedade contemporânea as funções do MDC (por exemplo, emissão de cartões de crédito) são normalmente assumidas pelo sistema bancário, pelo que os recursos envolvidos passam a fazer parte do IBC. Consequentemente, pode ser difícil classificar as empresas e os recursos que elas controlam como pertencentes a uma ou outra das categorias de capital industrial, comercial, de negociação de dinheiro ou de juros, e há uma margem considerável para a existência de “híbridos” em prática.
 \par 
O IBC pode participar de diversas operações destinadas à produção ou apropriação de mais-valia, de forma independente ou em associação com o capital industrial. O sistema de crédito amplia os limites do processo de reprodução e acelera o desenvolvimento das forças produtivas e do mercado mundial. Os retornos destas operações podem variar de acordo com o destino de investimentos específicos, bem como da macroeconomia capitalista, como acontece com ações, derivativos ou capital de risco; ou esses retornos podem ser designados antecipadamente. Qualquer que seja a forma e as condições adoptadas por estas transacções, a IBC liga-se através delas à reprodução do capital como um todo, representando um direito sobre a mais-valia que ainda não foi produzida. Esta reivindicação pode ser expressa através de transações que envolvem pagamentos ainda a serem feitos, ou a transformação dessas reivindicações em ativos negociáveis ​​de diversas maneiras, desde a cobrança de dívidas de agressores até títulos do governo, contratos futuros, obrigações de dívida garantidas e assim por diante. . Por sua vez, estes mercados reproduzem-se uns sobre os outros, com os serviços financeiros a serem vendidos em ligação com carteiras de activos, como em fundos de pensões e fundos de investimento. Cada um deles é uma reivindicação de propriedade em papel que pode ou não incluir capital produtivo que, por sua vez, pode ou não gerar ou apropriar-se de mais-valia. Isto é o que Marx chama de “capital fictício”: reivindicações de mais-valia no papel que podem ou não ser realizadas, mas que não são necessariamente fraudulentas.
 \par 
Nesta perspectiva, não surpreende que o sector financeiro seja capaz de financiar a superprodução e de gerar bolhas especulativas e quebras espectaculares. Também não é surpreendente que a fraude esteja sempre presente. A distinção entre finanças e indústria e a mudança no equilíbrio entre elas são dramaticamente ilustradas pelos desenvolvimentos nas finanças mundiais e nos sistemas financeiros nacionais ao longo dos últimos {\color{blue}40} anos. O inchado e fortemente recompensado sistema financeiro internacional beneficiou à custa da acumulação real e, ao longo das últimas décadas, tem estado sujeito a grave instabilidade e crises dispendiosas. No Volume {\color{blue}3} de O Capital, Marx investiga as circunstâncias em que a acumulação do IBC e dos activos e mercados construídos sobre ele pode ser validada pela acumulação de capital real. Conclui que nenhuma resposta pode ser dada antecipadamente, porque não pode haver garantia de produção e apropriação futuras de mais-valia (ver Capítulo {\color{blue}7}). Por exemplo, o proprietário do IBC pode adiantar-se a um industrial que seja corrupto, incompetente ou frustrado pela concorrência nacional ou estrangeira, ou a um consumidor que seja, ou se torne, incapaz de pagar, ou que, em última análise, se recuse a fazê-lo. Em qualquer dos casos, o circuito do IBC pode ser interrompido, com implicações potencialmente graves para a reprodução tanto do capital remunerado como do capital industrial.
 \par 
Em conclusão, a relação entre o capital industrial e o IBC baseia-se numa mistura de circuitos de capital sem resultados predeterminados em termos de acumulação real. Por esta razão básica, nem o funcionamento do sistema financeiro nem a sua interacção com a acumulação real podem ser sujeitos a controlo no sentido dominante de fixar a oferta de dinheiro ou de vinculá-la (ou o seu custo) ao nível de actividade económica. No caso de capital fictício, por exemplo, isto pode corresponder a uma acumulação real de capital, apoiada por investimentos bem-sucedidos. Mas poderá igualmente reflectir a titularização de um fluxo de rendimento não relacionado com a acumulação como tal - como acontece com os pagamentos de hipotecas, que depois se tornam sujeitos à especulação através de derivados formados a partir deles (como é universalmente reconhecido que ocorreu com o subprime dos EUA).
 \par 
Isto não significa sugerir que a regulação privada ou pública do sistema financeiro, incluindo a política monetária, não possa ter efeito sobre os resultados. Mas a ideia de que o capital fictício pode ser totalmente alinhado com a acumulação real através da regulação é equivocada, porque o capital fictício tornou-se cada vez mais necessário para a acumulação real, mas não pode garanti-la. Da mesma forma, a natureza e a estrutura do sistema financeiro e as modalidades da sua interacção com a acumulação real não podem ser determinadas por análise abstracta. Pelo contrário, evoluem em conjunto, estabelecendo estruturas específicas de actividade financeira e industrial, bem como resultados específicos durante o curso das crises.
 \par 
\section{Capital com Juros}
 \par 
Com base na análise acima, é possível identificar as características distintivas da teoria das finanças e dos juros de Marx. Marx divide o capital que funciona dentro da troca em capital mercantil (comercial) e capital que rende juros. O capital mercantil normalmente envolve comércio, como varejo e atacado, e, além de sua localização na esfera da troca, é logicamente definido por não produzir valor (excedente), embora esteja sujeito à entrada e saída competitiva, assim como o capital industrial. Consequentemente, o capital mercantil está sujeito à tendência para taxas de lucro equalizadas. O capital mercantil também envolve uma variedade de crédito não comercial e outras relações e funções monetárias, que, por conveniência, chamamos de capital de negociação de dinheiro, juntamente com Marx. O MDC é uma categoria geral definida pela necessidade da circulação monetária para a reprodução capitalista. As tarefas correspondentes de gestão de reservas, e assim por diante, podem ser atribuídas a capitalistas especializados ou retidas em empresas individuais.
 \par 
Em contraste, o IBC envolve a contração e o empréstimo de capital monetário, quer para produzir mais-valia, quer para apropriar-se dela através do capital mercantil. Como resultado, o IBC potencialmente ganha juros, levando a uma divisão da mais-valia entre esses juros e o lucro da empresa, sendo este último distribuído entre capitais industriais concorrentes e sujeito à taxa de equalização do lucro. O funcionamento do IBC mostra que a acumulação de capital é mediada pelo acesso diferenciado dos capitalistas concorrentes ao capital monetário.
 \par 
A divisão entre o lucro da empresa e os juros não é predeterminada pelo sistema de valores. Pelo contrário, é o resultado do processo de acumulação, tanto em termos de quanto de mais-valia é realizada (já que o avanço do capital monetário é uma pré-condição, mas não uma garantia de rentabilidade), e como é dividido entre IBC, industriais e mercantis. capitalistas. Esta divisão não tem relação exata com a taxa de juros. No entanto, as diferenças entre as taxas de juro contraídas e emprestadas, as taxas bancárias e outros encargos são mecanismos significativos através dos quais o IBC se apropria de parte da mais-valia produzida.
 \par 
Isto não significa que a divisão entre interesse e lucro da empresa não esteja sujeita a forças e determinações sistemáticas. Mas a capacidade de se apropriar da mais-valia sob a forma de juros deriva do papel do IBC como alavanca da concorrência na acumulação de capital, onde o IBC está situado de forma diferenciada em relação ao capital industrial e mercantil. Por exemplo, um banco pode estar disposto a emprestar a um industrial para competir com outro no mesmo sector, mas é menos provável que empreste para apoiar o crescimento de uma instituição financeira rival. É claro que isto não significa que não haja concorrência no sector financeiro, ou que o crédito interbancário esteja ausente, apenas que essas relações competitivas (e outras) são de natureza diferente das do resto da economia. É precisamente por isso que os juros associados ao IBC não são reduzidos a nada, ou à taxa normal de lucro sobre a utilização do capital próprio financeiro adiantado. Na verdade, embora seja um exemplo extremo, consideremos um banco que contrai e empresta dinheiro (com margem), utilizando ao mesmo tempo um capital próprio mínimo. A sua taxa de retorno sobre o seu próprio capital seria extremamente elevada!
 \par 
Crucial, então, para a teoria de Marx é a separação simples e abstrata entre o IBC e outras formas de capital e a apropriação de juros no IBC a partir da mais-valia. Mas, na acumulação e circulação de capital como um todo, o papel dos pagamentos de juros e dos mercados monetários é muito mais complexo e confuso na prática, com o recebimento de juros, dividendos ou outras formas de receitas constituindo os mecanismos pelos quais o lucro é equalizado entre alguns capitais (industriais e comerciais) ou a mais-valia é apropriada pelo IBC. Isto é ainda mais complicado pela medida em que o próprio IBC está incorporado noutros tipos de actividades comerciais de forma híbrida, por analogia com os híbridos de capital industrial e comercial.
 \par 
Nem isto é de interesse puramente acadêmico. Pois a actual era de financeirização é precisamente aquela em que tem havido uma expansão desproporcional do capital em troca, não só através da proliferação de derivados financeiros, mas também através da extensão do financiamento a cada vez mais áreas de reprodução económica e social, de das quais as finanças pessoais são um exemplo importante (juntamente com hipotecas, pensões e cuidados de saúde). Estes processos podem ser compreendidos através da aplicação do método de Marx e das categorias delineadas acima, que sugerem que tem havido uma mudança crescente da actividade capitalista ao longo do continuum produtivo, comercial, de negociação de dinheiro e de produção de juros, bem como uma hibridação entre estes processos. categorias. Por outras palavras, uma gama crescente de actividades tem estado sob os auspícios do IBC - nomeadamente o financiamento habitacional, como foi dramaticamente ilustrado pela ascensão e colapso das hipotecas sub-prime nos Estados Unidos - e não tanto na venda de hipotecas. aos proprietários de casas como a venda de obrigações de pagamento de hipotecas como derivados financeiros (ou seja, capital fictício). Mas isso é para antecipar nosso capítulo final.
 \par 
A capacidade de Marx de construir uma teoria do interesse em oposição ao lucro é uma característica distintiva da sua análise. Na economia política clássica, por exemplo, o juro é uma categoria introduzida com pouca ou nenhuma explicação, e a taxa de juro oscila em torno de uma taxa “natural” arbitrária para a qual não existem outros determinantes para além da oferta e da procura de moeda. Da mesma forma, na economia neoclássica, mais notavelmente na teoria fisheriana do consumo e da produção intertemporal, as taxas de juro e de lucro são conceptualmente idênticas e quantitativamente iguais em equilíbrio. Mesmo na economia keynesiana (e para o próprio Keynes), onde os factores monetários são especificamente introduzidos, a taxa de lucro - representada pela eficiência marginal do capital - é igual à taxa de juro. Embora as expectativas de curto prazo possam levar a um valor de desequilíbrio da taxa de juro, subjacente ao keynesianismo está a ideia de que existe uma taxa de juro de pleno emprego natural ou de equilíbrio. Esta divergência significativa da teoria de Marx está intimamente ligada ao fracasso da teoria keynesiana em diferenciar entre a procura, e portanto o crédito, para a acumulação e para o consumo, excepto no que diz respeito ao impacto dos multiplicadores sobre a procura efectiva.
 \par 
Em contraste, Marx não só categoriza o interesse de forma distinta, como também o localiza dentro da estrutura analítica do seu pensamento económico, derivando o interesse das relações competitivas entre duas frações claramente distintas da classe capitalista. Fá-lo com referência às tendências e estruturas abstratas que identificou para a economia capitalista; por exemplo, para que a taxa de lucro seja equalizada entre capitais industriais e de negociação de dinheiro concorrentes, para que o sistema de crédito se torne um mecanismo-chave de competição e alavanca de acumulação, para que o dinheiro como capital se destaque de outras mercadorias, para reservas ociosas ser centralizado no sistema bancário, e assim por diante. Estas considerações abstratas podem ser aplicadas nas análises históricas e empíricas marxistas do IBC e das estruturas financeiras específicas nas quais está inserido. Marx tinha muito a dizer sobre estas questões, especialmente no seu estudo sobre o sistema financeiro britânico no Volume {\color{blue}3} de O Capital, mas este material complexo não pode ser revisto aqui.
 \par 
\section{Capital com Juros}
 \par 
Apesar da sua enorme importância para o capitalismo contemporâneo, os estudos marxistas sobre dinheiro e finanças têm progredido de forma relativamente lenta, com pouco a ser dito sobre as questões mais fundamentais da natureza das finanças e da relação entre capital financeiro e industrial (excepto no que diz respeito ao crescente proeminência do primeiro, especialmente na época histórica do neoliberalismo).
 \par 
Os marxistas têm debatido frequentemente se o dinheiro mercadoria é uma abstracção para Marx, e se esta é uma abstracção legítima, ou - algo bastante mais forte - uma necessidade para o capitalismo. A nossa opinião é que a teoria do dinheiro de Marx demonstra como a sua presença material é cada vez mais substituída por símbolos, nomeadamente papel e dinheiro de crédito (ver, por exemplo, Marx 1981b, 1987). Mais importantes do que o papel residual do ouro enquanto moeda mundial, para efeitos de acumulação, por exemplo, são as relações monetárias associadas à acumulação e a forma como estas evoluem ao longo do tempo (que é examinado no Capítulo {\color{blue}14}). Ver o debate entre Jim Kincaid (2007, 2008, 2009) e Ben Fine e Alfredo Saad-Filho (2008, 2009) para a questão relacionada, mas separada, do papel do dinheiro no desenvolvimento e apresentação da teoria do valor de Marx.
 \par 
A teoria do IBC e dos juros de Marx é delineada em Marx (1981b e, especialmente, 1981a, pt.{\color{blue}5}). Este capítulo baseia-se em Ben Fine (1985-6). Diferentes aspectos da teoria do dinheiro e do crédito de Marx são explicados por Suzanne de Brunhoff (1976 e 2003), Duncan Foley (1986, cap.{\color{blue}7}), David Harvey (1999, caps 9-10), Rudolf Hilferding (1981), Makoto Itoh e Costas Lapavitsas (1999), Costas Lapavitsas (2000a, 2000b, 2003a, 2003b, 2013), Costas Lapavitsas e Alfredo Saad-Filho (2000), Roman Rosdolsky (1977, cap.{\color{blue}27}) e John Weeks (2010, cap.{\color{blue}5} ).