\chapter{Crédito da foto Yakubova Alina:}\label{Crédito da foto Yakubova Alina:}
 \par 
\begin{figure}
	\centering
	\includegraphics[width=1.\textwidth]{temp\_files/images/UP\_logo.png }
	\caption{Quando o Muro de Berlim caiu em 1989, Kristen R. Ghodsee estava viajando pela Europa e passou o verão de 1990 testemunhando em primeira mão a esperança e a euforia iniciais que se seguiram ao colapso repentino e inesperado do socialismo de estado no antigo Bloco Oriental. O caos político e econômico que se seguiu inspirou Ghodsee a seguir uma carreira acadêmica estudando essa reviravolta, focando em como a vida das pessoas comuns — e das mulheres em particular — mudou quando o socialismo de estado deu lugar ao capitalismo. Nas últimas duas décadas, ela visitou a região regularmente e viveu por mais de três anos na Bulgária e nas partes orientais da Alemanha reunificada. Agora professora de Estudos Russos e do Leste Europeu na Universidade da Pensilvânia, ela ganhou muitos prêmios por seu trabalho, incluindo uma bolsa Guggenheim, e escreveu seis livros sobre gênero, socialismo e pós-socialismo, examinando as experiências cotidianas de reviravolta e deslocamento que continuam a assombrar a região até hoje. Seus artigos e ensaios foram traduzidos para mais de uma dúzia de idiomas e apareceram em publicações como Dissent, Foreign Affairs, Jacobin, The Baffler, The New Republic, The Lancet, Washington Post e New York Times.}
	\label{ }
\end{figure}