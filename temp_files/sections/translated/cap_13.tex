\chapter{Conclusão}\label{Conclusão}
 \par 
O conservadorismo dominou a política americana nos últimos quarenta anos. Tal como as administrações republicanas de Dwight Eisenhower e Richard Nixon demonstraram a resiliência do New Deal, também as administrações democratas de Bill Clinton e Barack Obama demonstraram a resiliência do reaganismo. A aceitação conservadora do capitalismo não regulamentado e do poder imperial ainda envolve os nossos dois partidos. Consistente com o argumento deste livro sobre a vida privada do poder, o esforço mais visível do Partido Republicano desde as eleições intercalares de 2010 tem sido restringir os direitos dos trabalhadores e os direitos das mulheres. Embora o sucesso da direita nestas campanhas não esteja de forma alguma garantido, o facto de os republicanos terem visado o último reduto do movimento laboral e toda a Planned Parenthood dá alguma indicação do quão longe chegaram. O fim (em ambos os sentidos da palavra) da longa marcha da direita contra o século XX pode estar à vista.
 \par 
O sucesso da direita, contudo, não é uma bênção absoluta. Como os conservadores há muito notaram, existe uma sinergia dialética entre a esquerda e a direita, na qual o progresso da primeira estimula as inovações da segunda. “É irônico, embora não seja historicamente inédito”, escreveu Frank Meyer, o arquiteto intelectual da estratégia fusionista que reuniu os movimentos libertário e tradicionalista.
 \par 
Asas do conservadorismo moderno, “que tal explosão de energia criativa no nível intelectual” na direita “deveria ocorrer simultaneamente com uma propagação contínua da influência do liberalismo na esfera política prática”. Do outro lado do Atlântico, Roger Scruton, um tipo mais tradicional de conservador britânico, escreveu que “em tempos de crise. . . O conservadorismo faz o seu melhor”, enquanto Friedrich Hayek observou que a defesa do mercado livre “tornou-se estacionária quando era mais influente” e “progrediu” quando estava “na defensiva”. {\color{blue}1} É verdade que eram intelectuais que escreviam sobre ideias; os agentes conservadores podem estar menos otimistas quanto à perspectiva de negociar mais quatro anos por alguns bons livros. Mesmo assim, se o destino final de um partido está ligado à força das suas ideias – não à verdade das suas ideias, mas à ressonância e pertinência dessas ideias, à sua aquisição cultural e à capacidade de viajar através da paisagem política – deveria ser é motivo de preocupação para a direita o facto de as suas ideias terem tido um sucesso tão absoluto. Como Burke alertou há muito tempo, a vitória pode ser simplesmente um caminho para a morte.
 \par 
Vários livros recentes de introspecção conservadora sugerem que muitos na direita estão de facto preocupados com o estado das ideias conservadoras. {\color{blue}2} Mas a maior parte destas tentativas de autocrítica parecem motivadas por um simples medo de derrota nas urnas. Orientados como estão para o ciclo eleitoral ou para os prós e contras de políticas específicas, não vêem que o conservadorismo, como qualquer partido, possa perder eleições e ainda assim controlar o debate público. Mais importante ainda, estes escritores não compreendem que o fracasso é a fonte da renovação conservadora. Eles imaginam que o conservadorismo pode simplesmente ser reinventado ou reformulado para atender às necessidades de um eleitorado em mudança ou aos cavalos de pau dos seus teóricos. Mas não é assim que funciona o conservadorismo. O conservadorismo exige derrota; o fracasso é sua fonte de inspiração mais potente. Não o fracasso no sentido taciturno e romântico que Andrew Sullivan articula em seu hino à perda, mas o fracasso no sentido simultaneamente ameaçador e galvanizador. {\color{blue}3} A perda – perda social real, de poder e posição, privilégio e prestígio – é a
 \par 
Semente de mostarda da inovação conservadora. O que a direita sofre hoje não é perda, mas sucesso, e até que um grupo significativamente dominante na sociedade seja forçado a sofrer a sua perda – do tipo vivido pelos empregadores durante a década de 1930, pelos supremacistas brancos durante a década de 1960, ou pelos maridos na década de 1960. década de 1970 – continuará a ser um movimento filosoficamente flácido. Politicamente poderoso, mas intelectualmente moribundo.
 \par 
O que me leva a questionar-me sobre as perspectivas a longo prazo do Tea Party, a mais recente variante do populismo de direita. O Tea Party deu ao conservadorismo um novo sopro de vida? Ou será o Tea Party, tal como a Nova Política do final da década de 1960 e início da década de 1970, a última centelha de uma força esgotada, sendo as suas energias frenéticas uma máscara para o declínio do movimento mais vasto do qual faz parte? É impossível dizer, mas isto é claro: enquanto existirem movimentos sociais que exijam maior liberdade e igualdade, haverá o direito de os contrariar. Exceto o movimento pelos direitos dos homossexuais, não existem hoje movimentos sociais de esquerda ameaçadores. Assim que surgirem, surgirá com eles uma nova direita – não uma direita que precise de inventar papões como o socialismo de Obama, mas uma direita com verdadeiros monstros para destruir. Até então, podemos atribuir o estado actual da direita não às suas falhas de imaginação ou ao excesso de melancolia – como alguns fizeram4 – mas ao seu sucesso esmagador.
 \par 
O conservadorismo moderno entrou em cena no século XX para derrotar os grandes movimentos sociais de esquerda. Até onde a vista alcança, ele alcançou seu propósito. Feito isso, agora ele pode partir. Resta saber se isso acontecerá e quanto levará consigo ao sair.