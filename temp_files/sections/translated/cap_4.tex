\chapter{Lixo e Gravitas}\label{Lixo e Gravitas}
 \par 
A revolta de São Petersburgo deu-nos Vladimir Nabokov, Isaiah Berlin e Ayn Rand. O primeiro foi um romancista, o segundo um filósofo. A terceira não era nenhuma das duas coisas, mas pensava que ela era as duas coisas. Muitas outras pessoas também pensaram assim. Em 1998, os leitores que responderam a uma pesquisa da Modern Library identificaram Atlas Shrugg ed e The Fountainhead como os dois maiores romances em inglês do século XX – superando Ulysses, To the Lighthouse e Invisible Man. Em 1991, uma pesquisa realizada pela Biblioteca do Congresso e pelo Clube do Livro do Mês descobriu que, exceto a Bíblia, nenhum livro influenciou mais leitores americanos do que Atlas Shrugg ed.{\color{blue}1}
 \par 
Uma dessas leitoras pode muito bem ter sido Farrah Fawcett. Pouco antes de morrer, a atriz chamou Rand de “gênio literário” cuja recusa em fazer sua arte “como a de todo mundo” inspirou os próprios experimentos de Fawcett em pintura e escultura. A admiração, ao que parece, era mútua. Rand assistia Charlie’s Angels toda semana e, de acordo com Fawcett, “via algo” no show “que os críticos não viam”.
 \par 
Ela descreveu o show como um “triunfo de conceito e elenco”. Ayn disse que, embora Angels fosse exclusivamente americano, também era uma exceção à televisão americana, pois era o único programa a capturar o verdadeiro “romantismo” – retratava intencionalmente o mundo não como era, mas como deveria ser. Aaron Spelling foi provavelmente a única pessoa a ver os Anjos dessa forma, embora se referisse a isso como “televisão de conforto”.
 \par 
Rand ficou tão encantada com Fawcett que esperava que a atriz (ou se não ela, Raquel Welch) fizesse o papel de Dagny Taggart em uma versão para TV de Atlas Shrugg editada na NBC. Infelizmente, o chefe da rede, Fred Silverman, encerrou o projeto em 1978. “Sempre pensarei em ‘Dagny Taggart’ como o melhor papel que eu deveria interpretar, mas nunca fiz”, disse Fawcett.{\color{blue}2}
 \par 
Os seguidores de Rand em Hollywood sempre foram fortes. Barbara Stanwyck e Veronica Lake lutaram para fazer o papel de Dominique Francon na versão cinematográfica de The Fountainhead. Para nunca ficar atrás nesse departamento, Joan Crawford deu um jantar para Rand, no qual ela se vestiu como Francon, usando um vestido branco esvoaçante pontilhado com pedras preciosas água-marinha. {\color{blue}3} Mais recentemente, o autor de A Virtude do Egoísmo e da afirmação “se a civilização quiser sobreviver, é a moralidade altruísta que os homens têm de rejeitar” encontrou um par improvável de fãs no conjunto humanitário de Hollywood. {\color{blue}4} Rand “tem uma filosofia muito interessante”, diz Angelina Jolie. “Você reavalia sua própria vida e o que é importante para você.” A Nascente “é tão denso e complexo”, maravilha-se Brad Pitt, “que teria que ser um filme de seis horas”. (A versão cinematográfica de 1949 tem uma duração de {\color{blue}113} minutos e parece longa.) Christina Ricci afirma que A Nascente é seu livro favorito porque lhe ensinou que “você não é uma pessoa má se não ama. todos." Rob Lowe se gaba de que Atlas Shrugg ed é “uma conquista estupenda, e eu simplesmente adoro isso”. E qualquer namorado de Eva Mendes, diz a atriz, “tem que ser fã de Ayn Rand”.{\color{blue}5}
 \par 
Mas Rand, pelo menos de acordo com sua ficção, não deveria ter atraído nenhum fã. O enredo central de seus romances é o conflito entre o indivíduo criativo e a massa hostil. Quanto maior a conquista do indivíduo, maior a resistência da massa. Como diz Howard Roark, herói arquiteto de The Fountainhead:
 \par 
Os grandes criadores – os pensadores, os artistas, os cientistas, os inventores – enfrentaram sozinhos os homens do seu tempo. Todo grande pensamento novo foi contestado. Cada grande nova invenção foi denunciada. O primeiro motor foi considerado tolo. O avião foi considerado impossível. O tear mecânico foi considerado cruel. A anestesia era considerada pecaminosa. Mas os homens de visão emprestada seguiram em frente. Eles lutaram, sofreram e pagaram.{\color{blue}6}
 \par 
Rand claramente se considerava uma dessas criadoras. Numa entrevista com Mike Wallace, ela se declarou “a pensadora mais criativa do mundo”. Isso foi em 1957, quando Arendt, Quine, Sartre, Camus, Lukács, Adorno, Murdoch, Heidegger, Beauvoir, Rawls, Anscombe e Popper estavam todos trabalhando. Foi também o ano da primeira apresentação de Endgame e da publicação de Pnin, Doutor Jivago e O Gato do Chapéu. Dois anos depois, Rand disse a Wallace que “o único filósofo que me influenciou” foi Aristóteles. Caso contrário, tudo saiu “da minha mente”. Ela se gabou para seus amigos e para seu editor na Random House, Bennett Cerf, de que estava “desafiando a tradição cultural de dois mil e quinhentos anos”. Ela se viu como viu Roark, que disse: “Não herdo nada. Estou no fim de nenhuma tradição. Posso, talvez, estar no início de um.” Mesmo assim, dezenas de milhares de fãs já estavam ao lado dela. Em 1945, apenas dois anos após sua publicação, The Fountainhead vendeu {\color{blue}100}.000 exemplares. Em 1957, ano em que Atlas Shrugg ed foi publicado, ficou na lista dos mais vendidos do New York Times por {\color{blue}21} semanas.{\color{blue}7}
 \par 
Rand pode ter ficado desconfortável com o desafio que sua popularidade representava para sua visão de mundo, pois ela passou boa parte de sua vida posterior contando histórias sobre a resposta fria que ela e seu trabalho receberam. Ela falsamente alegou que doze editoras rejeitaram The Fountainhead antes que ele encontrasse um lar. Ela se autodenominou vítima de um isolamento terrível, mas necessário, alegando que "todas as conquistas e progressos foram realizados, não apenas por homens de habilidade e certamente não por grupos de homens, mas por uma luta entre o homem e a multidão". Mas quantos escritores solitários emergem de seus estudos, tendo acabado de escrever "The End" na última página de seu romance, para serem recebidos por um coro de parabéns de um círculo de fãs em espera? {\color{blue}8}
 \par 
Se ela tivesse sido uma leitora mais cuidadosa de seu trabalho, Rand poderia ter previsto essa ironia. Por mais que ela gostasse de colocar o gênio contra a massa, sua ficção sempre traía uma comunhão secreta entre os dois. Cada um de seus dois romances mais famosos dá ao seu herói distante a oportunidade de se defender em um longo discurso diante dos incultos e iletrados. Roark declama perante um júri das “caras mais duras” que inclui “um motorista de caminhão, um pedreiro, um eletricista, um jardineiro e três operários de fábrica”. John Galt vai ao ar em Atlas Shrugg ed, dirigindo-se a milhões de ouvintes por horas a fio. Em cada caso, o herói é compreendido, o seu génio aclamado, a sua alienação resolvida. E isso porque, como explica Galt, “não há conflitos de interesses entre homens racionais” – o que é apenas uma forma randiana de dizer que toda história tem um final feliz.{\color{blue}9}
 \par 
O principal conflito nos romances de Rand, então, não é entre o indivíduo e as massas. É entre o semideus criador e todos os elementos improdutivos da sociedade – os intelectuais, burocratas e intermediários – que se interpõem entre ele e as massas. Esteticamente, isso é kitsch; politicamente, inclina-se para o fascismo. É certo que o argumento de que existe uma ligação entre o fascismo e o kitsch sofreu uma derrota ao longo dos anos.
 \par 
No entanto, certamente o exemplo de Rand é suficientemente sugestivo para colocar a questão dessa ligação de volta na mesa.
 \par 
Ela nasceu em {\color{blue}2} de fevereiro, três semanas após a revolução fracassada de 1905. Seus pais eram judeus. Eles viviam em São Petersburgo, uma cidade governada por muito tempo pelo ódio aos judeus. Em 1914, seu registro de restrições antissemitas tinha quase {\color{blue}1}.{\color{blue}000} páginas, incluindo um estatuto limitando os judeus a não mais do que 2% da população. Eles a chamaram de Alissa Zinovievna Rosenbaum.{\color{blue}10}
 \par 
Quando tinha quatro ou cinco anos perguntou à mãe se poderia comprar uma blusa igual à que as primas usavam. A mãe dela disse que não. Ela pediu uma xícara de chá igual à que é servida aos adultos. Novamente sua mãe disse não. Ela se perguntou por que não poderia ter o que queria. Algum dia, ela jurou, ela o faria. Mais tarde na vida, Rand aproveitaria muito essa experiência. O seu biógrafo também o faz: “O elaborado e controverso sistema filosófico que ela criou aos quarenta e cinquenta anos era, no fundo, uma resposta a esta questão.”{\color{blue}11}
 \par 
A história, como contada, é pura Rand. Há o foco em um único incidente como presságio ou precipitante de um destino dramático. Há a elevação de um lugar-comum da infância a uma grande filosofia. Que criança, afinal, não se irritou por ter negado o que quer? Embora Rand pareça ter levado o egoísmo juvenil aos seus limites extremos — quando criança, ela não gostava de Robin Hood; quando adolescente, ela viu sua família quase morrer de fome enquanto ela se presenteava com o teatro — seu solipsismo não era nem tão raro nem tão precioso a ponto de justificar mais do que a quantidade usual de autoabsorção adolescente. {\color{blue}12} Há, finalmente, a revelação inadvertida de que a visão de mundo de alguém constitui pouco mais do que um caso de desenvolvimento interrompido. “Não é que a goma de mascar enfraqueça a metafísica”, Max Horkheimer escreveu certa vez sobre a cultura de massa, “mas que é metafísica — é isso que deve ficar claro”. {\color{blue}13} Rand deixou isso muito, muito claro.
 \par 
Mas a anedota sugere algo ainda mais distinto sobre Rand. Não suas opiniões ou gostos, que eram medianos e convencionais. Rand reivindicou Victor Hugo como sua principal inspiração em questões de ficção; Cyrano de Bergerac, de Edmond Rostand, foi outra pedra de toque. Ela considerou Rachmaninoff superior a Bach, Mozart e Beethoven. Ela ficou surpresa com a comparação reconhecidamente tola de um crítico de The Fountainhead com The Magic Mountain. Mann, pensou Rand, era o autor inferior, assim como Solzhenitsyn. {\color{blue}14} Nem foi seu senso de identidade que diferenciou Rand dos outros. É verdade que ela tendia para o desenho animado e o grandioso. Ela disse a Nathaniel Branden, seu amante muito mais jovem e discípulo de muitos anos, que ele deveria desejá-la mesmo que ela tivesse oitenta anos e estivesse em uma cadeira de rodas. Seus ensaios costumam citar os discursos de Galt como se o personagem fosse uma pessoa real, um filósofo da ordem de Platão ou Kant. Ela alegou ter se criado sem a ajuda de ninguém, embora tenha sido beneficiária vitalícia da generosidade social-democrata. Ela obteve educação universitária graças à Revolução Russa, que abriu as universidades para mulheres e judeus e, depois que os bolcheviques tomaram o poder, tornou-as gratuitas. Subsidiando o teatro para as massas, os bolcheviques também possibilitaram que Rand assistisse semanalmente operetas cafonas. Depois que a primeira peça de Rand foi encerrada na cidade de Nova York, em abril de 1936, a Works Progress Administration levou-a para os cinemas de todo o país, dando a Rand uma bela renda de US$ {\color{blue}10} por apresentação no final da década de 1930. Bibliotecários da Biblioteca Pública de Nova York a ajudaram na pesquisa para The Fountainhead. {\color{blue}15} Ainda assim, o seu narcisismo provavelmente não era maior – e certamente não menos sustentador – do que o do seu autor comum e esforçado.
 \par 
Não, o que realmente distinguiu Rand foi sua capacidade de traduzir seu senso de identidade em realidade, de transformar sua identidade imaginada em fatos materiais. Não por ser ótima, mas por persuadir outros, até mesmo biógrafos astutos, de que ela era ótima. Anne Heller, por exemplo, autora de Ayn
 \par 
Rand e o mundo que ela criou elogia repetidamente a “mente original e afiada” e a “lógica extremamente rápida” de Rand, fazendo com que nos perguntemos se ela leu algum trabalho de Rand. Ela afirma que Rand foi capaz de “escrever de forma mais persuasiva do ponto de vista masculino do que qualquer escritora desde George Eliot”. {\color{blue}16} Heller realmente acredita que Roark ou Galt são mais credíveis ou persuasivos do que Lawrence Selden ou New-land Archer? Ou o pequeno James Ramsay, que parece ter adquirido mais profundidade psíquica em seus seis anos do que qualquer um dos protagonistas de Rand, homem ou mulher, demonstrou ao longo de toda a sua vida? Jennifer Burns, historiadora intelectual e autora de Goddess of the Market: Ayn Rand and the American Right, escreve que Rand foi “um dos primeiros a identificar o poder muitas vezes aterrorizante do Estado moderno e a torná-lo uma questão de preocupação popular”, o que é verdade apenas se deixarmos de lado Montesquieu, Godwin, Constant, Tocqueville, Proudhon, Bakunin, Spencer, Kropotkin, Malatesta e Emma Goldman. Ela afirma que Rand não gostava da “confusão dos estudantes manifestantes da Boémia” dos anos {\color{blue}60} porque foi “criada na alta tradição europeia”. Mas que tipo de alta tradição europeia inclui operetas e Rachmaninoff, melodrama e filmes? Ela conclui que “o que resta” de valor duradouro em Rand é sua injunção de “ser fiel a si mesmo”. No entanto, dificilmente foi necessário que Rand nos ensinasse isso; na verdade, a mesma noção aparece numa peça sobre um príncipe dinamarquês escrita cerca de cinco séculos antes do nascimento de Rand.{\color{blue}17}
 \par 
Para compreender como Alissa Rosenbaum criou Ayn Rand, precisamos de traçar o seu itinerário não até à Rússia pré-revolucionária, que é a presunção errada dos seus biógrafos, mas até ao seu destino ao deixar a Rússia Soviética em 1926: Hollywood. Pois onde mais, senão na fábrica de sonhos, Rand poderia ter aprendido a realizar sonhos – sobre a América, o capitalismo e sobre si mesma?
 \par 
Mesmo antes de estar em Hollywood, Rand já era de Hollywood. Só em 1925, ela viu {\color{blue}117} filmes. Foi nos filmes, diz Burns, que Rand “vislumbrou a América” – e, poderíamos acrescentar, desenvolveu seu
 \par 
Sentido duradouro da forma narrativa. Uma vez lá, ela se tornou o tema de sua própria história em Hollywood. Ela foi descoberta por Cecil B. DeMille, que a viu vagando pelo estúdio em busca de trabalho. Intrigado com seu olhar intenso, ele lhe deu uma carona em seu carro e um trabalho como figurante, que ela rapidamente transformou em um trabalho de roteirista. Em poucos anos, seus roteiros estavam atraindo a atenção de grandes atores, o que levou um jornal a publicar uma matéria com a manchete “Garota russa encontra o fim do arco-íris em Hollywood”. {\color{blue}18} Rand, é claro, não foi o único europeu que veio para Hollywood durante os anos entre guerras. Mas, ao contrário de Fritz Lang, Hanns Eisler e todos os outros exilados no paraíso, Rand não fugiu para Hollywood; ela foi lá de boa vontade, ansiosamente. Billy Wilder chegou e encolheu os ombros; Rand veio de joelhos. Sua missão era aprender, e não refinar ou melhorar, a arte da fábrica de sonhos: como transformar uma boa história em um enredo de suspense, uma pessoa comum em um herói (ou vilão) descomunal - todos os truques da narrativa melodramática projetados para convencer milhões de telespectadores de que a vida é realmente vivida de forma febril. Mais importante ainda, ela aprendeu como realizar essa alquimia em si mesma. Ayan Rand era Norma Desmond ao contrário: ela era pequena; foram as fotos que ficaram grandes.
 \par 
Ao desempenhar o papel de Filósofo, Rand gostava de reivindicar Aristóteles como seu tutor. “Nunca tantos” – estranhamente, ela se incluiu aqui – “deveram tanto a um homem”. {\color{blue}19} Não está claro quanto Rand realmente leu da obra de Aristóteles: quando não estava citando Galt, ela tinha o hábito de atribuir ao filósofo grego afirmações e ideias que não aparecem em nenhum de seus escritos. Um alegado aristotelismo que Rand gostava de citar apareceu, completo com falsas atribuições, na autobiografia de Albert Jay Nock, um influente libertário da era do New Deal. Na cópia de Rand das memórias de Nock, Burns observa numa nota final, a passagem está marcada “com seis linhas verticais”.{\color{blue}20}
 \par 
Rand também gostava de citar a lei de identidade ou não contradição de Aristóteles — a noção de que tudo é idêntico a si mesmo, capturada pela abreviação “A é A” — como a base de sua defesa do egoísmo, do livre mercado e do estado limitado. Esse transporte em particular deixou os admiradores de Rand em êxtase e levou seus críticos, mesmo os mais amigáveis, à distração. Vários meses antes de sua morte em 2002, o filósofo de Harvard Robert Nozick, o mais sofisticado analiticamente dos libertários do século XX, disse que “o uso que é feito por pessoas na tradição randiana desse princípio da lógica... É completamente injustificado, até onde posso ver; é ilegítimo.” {\color{blue}21} Em 1961, Sidney Hook escreveu no New York Times:
 \par 
Desde o seu batismo na época medieval, Aristóteles serviu a muitos propósitos estranhos. Nada foi mais estranho do que esta aliança sacramental, por assim dizer, de Aristóteles com Adam Smith. As virtudes extraordinárias que Miss Rand encontra na lei de que A é A sugerem que ela não tem consciência de que os princípios lógicos por si só só podem testar a consistência. Eles não podem estabelecer a verdade. . . . Jurando fidelidade a Aristóteles, a senhorita Rand afirma deduzir não apenas questões de fato da lógica, mas, com a mesma pouca garantia, também regras éticas e verdades econômicas. Tal como ela as entende, as leis da lógica permitem-na proclamar que “a existência existe”, o que é muito parecido com dizer que a lei da gravitação é pesada e a fórmula do açúcar é doce.{\color{blue}22}
 \par 
Quer Rand tenha lido Aristóteles ou não, está claro que ele causou pouca impressão nela, especialmente quando se tratava de ética. Aristóteles tinha uma abordagem distinta da moralidade, totalmente fora de sintonia com as sensibilidades modernas; e embora Rand tivesse alguma consciência de seu caráter distintivo, sua substância parece ter se perdido para ela. Como um conjunto de clássicos de couro sintético na prateleira da sala de estar, Aristóteles estava lá para impressionar a empresa – e, no caso de Rand, desviar a atenção do verdadeiro negócio em questão.
 \par 
Ao contrário de Kant, o moderno emblemático que afirmava que a justeza dos nossos actos é determinada unicamente pela razão, imaculada pela necessidade, desejo ou interesse, Aristóteles enraizou a sua ética na natureza humana, nos hábitos e práticas, nas disposições e tendências, que tornam nos felizes e possibilitar nosso florescimento. E enquanto Kant acreditava que a moralidade consiste em regras austeras, impondo-nos deveres incondicionais e exigindo o nosso mais extenuante sariphi ce, Aristóteles situou a vida ética nas virtudes. Estas são qualidades ou estados, algures entre a razão e a emoção, mas combinando elementos de ambas, que nos transportam e conduzem, pelos meios mais suaves e subtis, às colinas exteriores da boa conduta. Uma vez lá, somos inspirados e equipados para escalar essas alturas mais baixas, de onde passamos para os níveis mais elevados. Uma pessoa que age virtuosamente desenvolve uma natureza que quer e é capaz de agir virtuosamente e que encontra felicidade na virtude. Essa coincidência de pensamento e sentimento, razão e desejo, é alcançada ao longo de uma vida inteira de atos virtuosos. A virtude, em outras palavras, é menos um códice de regras, que deve ser observado diante da oposição mais violenta do eu, do que o alimento e a fibra, a graxa e a gasolina, de uma alma que funciona adequadamente.
 \par 
Se Kant é um atleta da vida moral, Aristóteles é o seu virtuoso. Rand, por outro lado, é um melodramático da vida moral. Aprendiz em Hollywood e não em Atenas, ela tem pouca paciência para a silenciosa habituação às virtudes que a ética aristotélica acarreta. Em vez disso, ela retorna à sua imagem favorita de um indivíduo heróico que enfrenta um caminho difícil. A dificuldade nunca é o resultado de confusão ou ambiguidade; Rand detestava “o culto da obscuridade moral”, insistindo que a moralidade é antes de tudo e sempre “um código preto e branco”. {\color{blue}23} O que torna o caminho traiçoeiro – não para o herói, que parece ter nascido totalmente equipado para isso, mas para o resto de nós – são os obstáculos ao longo do caminho. Fazer a coisa certa traz dificuldades, penúria e exílio, enquanto fazer a coisa errada traz riqueza, status e
 \par 
Aclamação. Por se recusar a submeter-se às convenções arquitetônicas, Roark acaba dividindo rochas em uma pedreira. Peter Keating, o sósia de Roark, trai a todos, inclusive a si mesmo, e é o brinde da cidade. No final das contas, é claro, a distribuição de recompensas e punições será revertida: Roark está feliz, Keating, infeliz. Mas, em última análise, está sempre e inevitavelmente muito longe.
 \par 
Em seus ensaios, Rand procura aplicar a essas imagens um brilho aristotélico superficial. Ela também enraíza a sua ética na natureza humana e recusa-se a fazer uma distinção entre o interesse próprio e o bem, entre a conduta ética e o desejo ou a necessidade. Mas a métrica de Rand para o bem e o mal, a virtude e o vício, não é a felicidade ou o florescimento. São as exigências severas e severas da vida e da morte. Como ela escreve em “A Ética Objetivista”:
 \par 
Cito o discurso de Galt: “Existe apenas uma alternativa fundamental no universo: existência ou inexistência – e ela pertence a uma única classe de entidades: aos organismos vivos. A existência da matéria inanimada é incondicional, a existência da vida não: depende de um curso de ação específico. A matéria é indestrutível, muda de forma, mas não pode deixar de existir. É apenas um organismo vivo que enfrenta uma alternativa constante: a questão da vida ou da morte. A vida é um processo de ação autossustentável e autogerada. Se um organismo falhar nessa ação, ele morre; seus elementos químicos permanecem, mas sua vida deixa de existir. É apenas o conceito de ‘Vida’ que torna possível o conceito de ‘Valor’. É somente para uma entidade viva que as coisas podem ser boas ou más.”{\color{blue}24}
 \par 
Os defensores de Rand gostam de afirmar que o que Rand tem em mente por “vida” não é simplesmente a preservação biológica, mas a boa vida do grande homem sujo de Aristóteles, o que Rand caracteriza como “a sobrevivência do homem enquanto homem”. {\color{blue}25} E é verdade que Rand não se interessa muito pela mera vida
 \par 
Ou a vida pela vida. Isso seria muito pedestre. Mas o naturalismo de Rand está muito distante do de Aristóteles. Para ele a vida é um fato, para ela é uma questão, e essa mesma questão é o que faz da vida, por si só, tal objeto e fonte de reflexão.
 \par 
O que dá valor à vida é a possibilidade sempre presente de que ela possa (e um dia irá) acabar. Rand nunca fala da vida como um dado ou fundamento. É uma escolha condicional que devemos fazer, não uma vez, mas repetidamente. A morte lança uma sombra, conferindo aos nossos dias uma urgência e um peso que de outra forma não teriam. Exige vigilância, um estado de alerta para a fatalidade de cada momento. “Nunca se deve agir como um zumbi”, recomenda Rand. {\color{blue}26} Em suma, a morte torna a vida dramática. Faz com que as nossas escolhas – não apenas as grandes, mas as pequenas que fazemos todos os dias, a cada segundo – sejam importantes. No universo Randiano, é meio-dia o tempo todo. Longe de ser exaustiva ou enervante, tal existência, pelo menos para Rand e seus personagens, é animadora e excitante.
 \par 
Se esta ideia tiver alguma ressonância moral, ela será ouvida não nos escritos de Aristóteles nem no existencialismo superficialmente semelhante de Sartre, mas na marcha do fascismo. A noção de que a vida é uma luta contra e até à morte, de cada momento carregado de destruição, de cada escolha prenhe de destino, de cada acção oprimida pela aniquilação, da sua pressão letal gerando significado moral - estas são as palavras de ordem da noite europeia. No seu discurso no Sportpalest de Berlim, de Fevereiro de 1943, Goebbels declarou: “Tudo o que lhe serve e à sua luta pela existência é bom e deve ser sustentado e nutrido. Tudo o que lhe é prejudicial e à sua luta pela existência é mau e deve ser removido e eliminado.” {\color{blue}27} O “isso” em questão é a nação alemã, não o indivíduo Randiano. Mas se retirarmos o pronome do seu antecedente – e ouvirmos o zumbido de fundo de Sein oder Nichtsein, preservação versus eliminação – as semelhanças entre a sintaxe moral do Randianismo e do fascismo tornam-se claras. Bondade
 \par 
É medida pela vida, a vida é uma luta contra a morte, e só a nossa vigilância diária garante que uma não prevaleça sobre a outra.
 \par 
Rand, sem dúvida, se oporia à comparação. Afinal, existe uma diferença entre o individual e o coletivo. Rand considerava o primeiro um fundamento existencial, o segundo – quer assumisse a forma de classe, raça ou nação – uma monstruosidade moral. E enquanto Goebbels falava de violência e guerra, Rand falava de comércio e comércio, produção e economia. Mas o fascismo dificilmente é hostil ao indivíduo heróico. Além disso, esse indivíduo encontra muitas vezes a sua vocação mais profunda na actividade económica. Longe de demonstrar uma divergência em relação ao fascismo, os escritos económicos de Rand registam a sua presença de forma indelével.
 \par 
Aqui está Hitler falando com um grupo de industriais em Düsseldorf-
 \par 
Os senhores afirmam, senhores, que a economia alemã deve ser construída com base na propriedade privada. Ora, tal concepção de propriedade privada só pode ser mantida na prática se, de alguma forma, parecer ter um fundamento lógico. Esta concepção deve derivar a sua justificação ética da compreensão de que isto é o que a natureza dita.{\color{blue}28}
 \par 
Rand também acredita que o capitalismo é vulnerável a ataques porque lhe falta “uma base filosófica”. Para sobreviver, deve ser racionalmente justificado. Devemos “começar do início”, com a própria natureza. “Para sustentar a sua vida, cada espécie viva tem de seguir um certo curso de ação exigido pela sua natureza.” Como a razão é o “meio de sobrevivência” do homem, a natureza dita que “os homens prosperam ou fracassam, sobrevivem ou perecem em proporção ao grau da sua racionalidade”. (Observe o deslizamento entre o sucesso e o fracasso e a vida e a morte.) O capitalismo é o único sistema que reconhece e incorpora este ditame da natureza. “É o básico, metafísico
 \par 
Um facto da natureza do homem – a ligação entre a sua sobrevivência e o uso da razão – que o capitalismo reconhece e protege.” {\color{blue}29} Tal como Hitler, Rand encontra na natureza, na luta do homem pela sobrevivência, um “fundamento lógico” para o capitalismo.
 \par 
Longe de privilegiar o colectivo em detrimento do individual ou de subsumir este último ao primeiro, Hitler acreditava que era a “força e o poder da personalidade individual” que determinava o destino económico (e cultural) da raça e da nação. {\color{blue}30} Aqui está ele, em 1933, dirigindo-se a outro grupo de industriais:
 \par 
Tudo o que de positivo, bom e valioso foi alcançado no mundo no campo da economia ou da cultura pode ser atribuído exclusivamente à importância da personalidade. . . . Todos os bens materiais que possuímos devemos à luta de uns poucos eleitos.{\color{blue}31}
 \par 
Os homens excepcionais, os inovadores, os gigantes intelectuais. . . . São os membros desta minoria excepcional que elevam toda uma sociedade livre ao nível das suas próprias realizações, ao mesmo tempo que ascendem cada vez mais.{\color{blue}32}
 \par 
Se a primeira metade das opiniões económicas de Hitler celebra o génio romântico do industrial individual, a segunda explicita as implicações igualitárias da primeira. Assim que reconhecermos “as realizações notáveis ​​dos indivíduos”, diz Hitler em Düsseldorf, devemos concluir que “as pessoas não têm o mesmo valor ou a mesma importância”. A propriedade privada “só pode ser moral e eticamente justificada se admitirmos que as realizações dos homens são diferentes”. Uma compreensão da natureza promove o respeito pelo indivíduo heróico, o que promove uma apreciação da desigualdade na sua forma mais cruel. “As forças criativas e de decomposição de um povo sempre lutam umas contra as outras.”{\color{blue}33}
 \par 
A apreciação de Rand sobre a desigualdade é igualmente pungente. Cito
 \par 
Do discurso de Galt:
 \par 
O homem no topo da pirâmide intelectual é o que mais contribui para todos os que estão abaixo dele, mas não recebe nada exceto o seu pagamento material, não recebendo nenhum bônus intelectual de outros que acrescente valor ao seu tempo. O homem que está na base e que, abandonado a si mesmo, morreria de fome em sua desesperada inépcia, não contribui em nada para aqueles que estão acima dele, mas recebe o bônus de todos os seus cérebros. Tal é a natureza da “competição” entre os fortes e os fracos do intelecto. Este é o padrão de “exploração” pelo qual vocês condenaram os fortes.{\color{blue}34}
 \par 
O caminho de Rand da natureza ao individualismo e à desigualdade também termina num mundo dividido entre “as forças criativas e de decomposição”. Em cada sociedade, diz Roark, existe um “criador” e um parasita “de segunda mão”, cada um com a sua própria natureza e código. A primeira “permite ao homem sobreviver”. O segundo é “incapaz de sobreviver”. {\color{blue}35} Um produz vida, o outro induz a morte. Em Atlas Shrugg ed, a batalha é entre o produtor e os “saqueadores” e “vagabundos”. Também deve terminar em vida ou morte.
 \par 
Encontrar Rand em tal companhia não deveria ser surpresa, pois ela e os nazis partilham um património no nietzscheanismo vulgar que tem perseguido a direita radical, seja nas suas variantes libertárias ou fascistas, desde o início do século XX. Como mostram seus dois biógrafos, Nietzsche exerceu desde cedo um controle sobre Rand que nunca foi realmente afrouxado. Seu primo brincou com Rand dizendo que Nietzsche “venciou você em todas as suas ideias”. Quando Rand chegou aos Estados Unidos, Assim Falou Zaratustra foi o primeiro livro em inglês que ela comprou. Com Nietzsche em mente, ela se inspirou a escrever em seus diários que “o segredo da vida” é “você deve ser
 \par 
Nada além de vontade. Saiba o que você quer e faça. Saiba o que você está fazendo e por que está fazendo isso, a cada minuto do dia. Toda vontade e todo controle. Mande todo o resto para o inferno! Suas entradas frequentemente incluem frases como “Nietzsche e eu pensamos” e “como Nietzsche disse”.{\color{blue}36}
 \par 
Rand ficou muito impressionado com a ideia do criminoso violento como herói moral, um tradutor nietzschiano de todos os valores; de acordo com Burns, ela “considerou a criminalidade uma metáfora irresistível para o individualismo”. Leopoldo e Loeb literário, ela planejou uma novela baseada no caso real de um assassino que estrangulou uma menina de {\color{blue}12} anos. O assassino, disse Rand, “nasce com uma consciência maravilhosa, livre e leve – resultante da absoluta falta de instinto social ou sentimento de rebanho. Ele não entende, porque não tem órgão para compreender, a necessidade, o significado ou a importância das outras pessoas.” {\color{blue}37} Esta não é uma má descrição da master class de Nietzsche em A Genealogia da Moral.
 \par 
Embora os defensores de Rand afirmem que ela mais tarde abandonou a sua paixão por Nietzsche, há demasiadas provas da sua persistência. Há a figura do próprio Roark: “Enquanto ela fazia anotações sobre a personalidade de Roark”, escreve Burns, “ela disse a si mesma: 'Veja Nietzsche sobre o riso.' '” {\color{blue} 38 } {\par} E há ainda a Revolta de Atlas, que Ludwig von Mises, uma das eminências presidentes da economia neoclássica, elogiou assim:
 \par 
Você tem a coragem de dizer às massas o que nenhum político lhes disse: você é inferior e todas as melhorias nas suas condições, que você simplesmente considera um dado adquirido, devem-se ao esforço de homens que são melhores do que você.{\color{blue}39}
 \par 
Mas a influência de Nietzsche saturou a escrita de Rand de uma forma mais profunda, emblemática da trajetória geral da direita desde a sua criação.
 \par 
Nascimento no cadinho da Revolução Francesa. Rand foi ateia ao longo da vida, com uma animosidade especial pelo cristianismo, que ela chamou de “o melhor jardim de infância possível do comunismo”. {\color{blue}40} Longe de representar uma tendência herética dentro do conservadorismo, a declaração de Rand canaliza uma tradição de suspeita da direita sobre os efeitos insidiosos da religião, particularmente do Cristianismo, no mundo moderno. Enquanto muitos conservadores desde 1789 se uniram ao cristianismo e à religião como um antídoto para as revoluções democráticas dos séculos XVIII e XIX, um subconjunto não insignificante entre eles viu a religião, ou pelo menos algum aspecto da religião, como o ajudante de revolução.
 \par 
Joseph de Maistre foi um dos primeiros. Arquicatólico, ele atribuiu a Revolução Francesa aos solventes acre da Reforma. Com a sua celebração da “interpretação privada” das Escrituras, o protestantismo pavimentou o caminho, século após século, de regicídio e revolta originados nas classes mais baixas.{\color{blue}41}
 \par 
É da sombra de um claustro que emerge um dos maiores flagelos da humanidade. Lutero aparece; Calvin o segue. A Revolta Camponesa; a Guerra dos Trinta Anos; a guerra civil na França. . . Os assassinatos de Henrique II, Henrique IV, Maria Stuart e Carlos I; e finalmente, nos nossos dias, da mesma fonte, a Revolução Francesa.{\color{blue}42}
 \par 
Nietzsche, filho de um pastor luterano, radicalizou este argumento, pintando todo o cristianismo – na verdade, toda a religião ocidental, remontando ao judaísmo – como uma moralidade escrava, a revolta psíquica das classes inferiores contra os seus superiores. Antes de haver religião ou mesmo moralidade, havia o sentido e a sensibilidade da classe dominante. O mestre olhou para seu corpo – sua força e beleza, sua excelência demonstrada e suas reservas de poder – e viu e disse que era bom. Pensando melhor, ele olhou para o escravo,
 \par 
E viu e disse que era ruim. O escravo nunca olhava para si mesmo: era consumido pela inveja e pelo ressentimento em relação ao seu senhor. Fraco demais para agir de acordo com sua raiva e se vingar, ele lançou uma revolta mental silenciosa, mas letal. Ele chamou todos os atributos do mestre – poder, indiferença ao sofrimento, crueldade impensada – de maus. Ele falou de seus próprios atributos – mansidão, humildade, tolerância – como bons. Além disso, ele criou uma religião que fazia do egoísmo e da preocupação consigo mesmo um pecado, e da compaixão e da preocupação pelos outros o caminho para a salvação. Além disso, ele imaginou uma irmandade universal de crentes, iguais perante Deus, e amaldiçoou a ordem do mestre de excelência distribuída de forma desigual. {\color{blue}43} O resíduo moderno dessa revolta de escravos, Nietzsche deixa claro, não é encontrado no cristianismo, ou mesmo na religião, mas nos movimentos do século XIX pela democracia e pelo socialismo:
 \par 
Outro conceito cristão, não menos louco, penetrou ainda mais profundamente no tecido da modernidade: o conceito da “igualdade das almas perante Deus”. Este conceito fornece o protótipo de todas as teorias de igualdade de direitos: a humanidade foi primeiro ensinada a gaguejar a proposição de igualdade num contexto religioso, e só mais tarde ela foi transformada em moralidade: não é de admirar que o homem tenha acabado por levá-la a sério, levando-a na prática! – isto é, politicamente, democraticamente, socialmente.{\color{blue}44}
 \par 
Quando Rand investe contra o cristianismo como o antepassado do socialismo, quando critica o altruísmo e o sacrifício como inversões da verdadeira hierarquia de valores, ela está cultivando a tensão dentro do conservadorismo que vê a religião não como um remédio, mas como uma ajudante, da esquerda. . E quando ela olha, ainda que ineptamente, para Aristóteles em busca de uma moralidade alternativa, ela está recapitulando a viagem de Nietzsche de volta à antiguidade, onde ele esperava encontrar uma moralidade de classe superior não contaminada pelos valores igualitários das classes inferiores.
 \par 
Embora a defesa anti-religiosa do capitalismo por Rand possa parecer deslocada no firmamento político de hoje, faríamos bem em recordar o recente renascimento do interesse pelos seus livros. Mais de {\color{blue}800} mil cópias de seus romances foram vendidas somente em 2008; como Burns observa corretamente, “Rand é uma presença mais ativa na cultura americana agora do que durante sua vida”. Na verdade, Rand é regularmente citado como uma influência formativa sobre toda uma nova geração de líderes republicanos; Burns a chama de “a droga definitiva para a vida da direita”. {\color{blue}45} Quer ela seja invocada pelo nome, a presença de Rand é palpável na preocupação, ouvida cada vez mais na direita, de que há algo sinistro em andamento nas instituições e nos ensinamentos do Cristianismo.
 \par 
Eu imploro, procure as palavras “justiça social” ou “justiça econômica” no site da sua igreja. Se você encontrar, corra o mais rápido que puder. Justiça social e justiça económica são palavras-código. Agora, estou aconselhando as pessoas a deixarem suas igrejas? Sim.
 \par 
Esse foi Glenn Beck em seu programa de rádio de {\color{blue}2} de março de 2010, posicionando-se contra, bem, praticamente todas as igrejas da fé cristã: católica, episcopal, metodista, batista - até mesmo sua própria Igreja de Jesus Cristo dos Últimos Dias. Santos.{\color{blue}46}
 \par 
Sozinha, Rand tem pouca importância. É apenas a sua ressonância na cultura americana – e as associações desagradáveis ​​que a sua ressonância evoca – que a tornam de algum interesse. Ela não é diferente da “segunda mão” descrita por Roark: “A realidade deles não está dentro deles, mas em algum lugar naquele espaço que divide um corpo humano de outro. Não uma entidade, mas uma relação. . . . A pessoa age de segunda mão, mas a fonte de suas ações está espalhada em todas as outras pessoas vivas.” {\color{blue}47} Pela primeira vez, ao que parece, ele sabia de onde falava.
 \par 
Mas depois de tudo o que Nietzsche foi dito e Aristóteles terminou, ainda nos resta um enigma sobre Rand: como poderia tal mediocridade, não apenas de segunda mão, mas de segunda mão, exercer uma influência tão contínua na cultura em geral?
 \par 
Possuímos uma literatura inteira, de Melville a Mamet, dedicada ao vigarista e ao traficante, e é tentador ver Rand como uma das muitas falsificações e fraudes que periodicamente iluminam a paisagem americana. Mas essa tentação deve ser resistida. Rand representa algo diferente, mais perturbador. O vigarista é um mentiroso que pode averiguar a verdade das coisas, muitas vezes melhor do que o resto de nós. Ele tem que fazer isso: se ele vai roubar sua marca, ele tem que saber quem é a marca e quem ela gostaria de ser. Trabalhando naquele submundo entre o fato e a fantasia, o vigarista só poderá dourar o lírio se vir o lírio como ele realmente é. Mas Rand não tinha vontade de dourar nada. O lírio dourado era realidade. O que havia para acrescentar? Ela até usava um distintivo de lapela para deixar claro: feito de ouro e em forma de cifrão, era um enfeite do tipo mais literal.
 \par 
Desde o século XIX, tem sido tarefa da esquerda apresentar à civilização liberal um espelho dos seus valores mais elevados e dizer: “Você não tem esta aparência”. Você afirma acreditar nos direitos do homem, mas defende apenas os direitos de propriedade. Você afirma defender a liberdade, mas é apenas a liberdade dos fortes dominar os fracos. Se você deseja viver de acordo com seus princípios, deve ceder ao demiurgo deles. Permita que os despossuídos assumam o poder e o ideal se tornará real, a metáfora se tornará material.
 \par 
Rand acreditava que esse encontro do céu e da terra poderia ser organizado por outros meios. Em vez de refazer o mundo à imagem do paraíso, ela procurou o paraíso à imagem do mundo. A transformação política não era necessária. A transubstanciação foi suficiente. Diga algumas palavras, acene com as mãos e o ideal é real, a metáfora é material. Idealista do tipo mais primitivo, Rand pegou um século de dicotomias socialistas e as destruiu. Pequena maravilha
 \par 
Muitos a acusaram de intolerância: quando o céu e a terra estão tão próximos, onde há espaço para dissidência?
 \par 
Longe de precisar de explicação, seu sucesso se explica por si só. Rand trabalhou naquele campo de provas americano por excelência – ao lado de nomes como Richard Nixon, Ronald Reagan e Glenn Beck – onde o lixo ganha seriedade e a besteira é abençoada. Lá ela aprendeu que sonhos não se realizam. Eles são verdadeiros. Transforme sua metafísica em chiclete, e seu chiclete será metafísica. A é A.