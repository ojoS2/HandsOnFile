\chapter{3 Capital e Exploração}\label{3 Capital e Exploração}
 \par 
No capítulo anterior foi demonstrado que a produção de valores de uso como mercadorias, típica do capitalismo, tende a ocultar as relações sociais de produção como uma relação entre produtores. Este tipo de produção concentra a atenção na troca como relação entre coisas. No entanto, como a simples produção de mercadorias demonstra logicamente e a história do comércio demonstra na realidade, a própria troca pode existir e existe sem o capitalismo. Só quando a força de trabalho se torna uma mercadoria e os trabalhadores assalariados são regularmente contratados para produzir mercadorias para venda com lucro é que o capitalismo se torna o modo de produção típico de uma determinada sociedade. Neste capítulo, ao examinar a troca a partir da perspectiva primeiro dos trabalhadores e depois dos capitalistas, veremos por que o capitalismo não é apenas um sistema de produção de mercadorias, mas é também, mais crucialmente, um sistema de trabalho assalariado.
 \par 
\section{Intercâmbio}
 \par 
Além da simples troca, que é um fenômeno histórico muito limitado, o dinheiro é essencial para a troca. As funções do dinheiro foram bem exploradas na literatura: é uma medida de valor, um padrão de preços (ou seja, uma unidade de conta), um meio de circulação e uma reserva de valor. Como meio de circulação, medeia o processo de troca. Quando as mercadorias são compradas a crédito e a dívida é posteriormente liquidada, o dinheiro funciona como meio de pagamento. A utilização do dinheiro como meio de pagamento pode entrar em conflito com a utilização do dinheiro como reserva de valor, e isto é importante em crises, quando o crédito é concedido com menos facilidade e os pagamentos reais são exigidos.
 \par 
Consideremos inicialmente um problema geral: um indivíduo possui uma mercadoria mas, por qualquer razão, preferiria trocá-la por outra. Primeiro, a mercadoria (C) deve ser trocada por dinheiro (M). Esta venda é representada por C - M. Em segundo lugar, o dinheiro obtido é trocado pela mercadoria desejada, D - C. Em ambos os casos, C - M e D - C, os valores das mercadorias são realizados no mercado; o vendedor obtém dinheiro e o comprador adquire um valor de uso, que pode ser utilizado tanto no consumo quanto na produção. Em geral, então, as mercadorias são vendidas para comprar outras mercadorias, e isto pode ser representado por C - M - C: a circulação de mercadorias. Os dois extremos da circulação de mercadorias são denotados por C porque estão na forma de mercadoria e têm o mesmo valor, não porque sejam a mesma coisa - na verdade, não podem ser a mesma coisa, caso contrário todo o propósito da troca será derrotado, a actividade especulativa deixando de lado as commodities.
 \par 
Presumimos que ambas as mercadorias têm o mesmo valor, porque a circulação de mercadorias (troca) como tal não pode acrescentar valor aos bens ou serviços trocados. Embora alguns vendedores possam lucrar com a venda de mercadorias acima do valor (troca desigual), como acontece com comerciantes e especuladores inescrupulosos, por exemplo, isto não é possível para todos os vendedores porque qualquer valor que uma parte ganhe em troca deve ser perdido para a outra. Sob esta luz, as bolsas simples de mercadorias estão resumidas na Figura {\color{blue}3}.1. Normalmente, sob o capitalismo, as simples trocas de mercadorias podem começar com um trabalhador ou um capitalista. Para o trabalhador, a única mercadoria disponível para vender é a sua força de trabalho, e esta é trocada por salários (M) e, eventualmente, por bens salariais (C). Alternativamente, a venda de mercadorias C - M também poderia ser realizada por um capitalista, seja para comprar bens para consumo pessoal
 \par 
\section{Intercâmbio}
 \par 
\section{Intercâmbio}
 \par 
Figura {\color{blue}3}. {\color{blue} 1 } {\par} Troca simples de mercadorias: venda para comprar ou renovar a produção, por exemplo, através da compra subsequente de força de trabalho, matérias-primas, máquinas, etc.
 \par 
\section{Intercâmbio}
 \par 
Em contraste com as trocas simples, que começam com a venda de mercadorias, a produção capitalista deve começar com a compra de dois tipos de mercadorias. Estas mercadorias são os meios de produção (insumos para processamento posterior, máquinas, peças sobressalentes, combustível, electricidade, etc.) e a força de trabalho. Uma condição necessária para este último é a vontade por parte dos trabalhadores de vender esta mercadoria. Esta vontade, um exercício da “liberdade” de troca, é imposta aos trabalhadores: por um lado, a venda da força de trabalho é uma condição de trabalho, pois caso contrário os trabalhadores não poderão ter acesso aos meios de produção, que são monopolizados pelos trabalhadores. capitalistas. Por outro lado, é um requisito para o consumo, pois é a única mercadoria que os trabalhadores conseguem vender de forma consistente (ver Capítulos {\color{blue}2} e {\color{blue}6}).
 \par 
Tendo reunido meios de produção e força de trabalho (M - C), os capitalistas organizam e supervisionam o processo de produção e vendem a produção resultante (C - M). Neste último caso, o travessão esconde a intervenção da produção na transformação dos factores de produção em dinheiro (ver Capítulo {\color{blue}4}). Por enquanto, podemos representar a atividade de troca de um capitalista por D - M - D'. Em contraste com a simples troca de mercadorias, C - M - C, discutida na secção anterior, a circulação capitalista de mercadorias começa e termina com dinheiro, não com mercadorias. Isto implica que nos dois extremos se encontra a mesma coisa, dinheiro, em vez de coisas diferentes, mercadorias com valores de uso distintos. Claramente, o único objectivo de empreender esta actividade de troca numa base sistemática é obter mais valor, em vez de valores de uso diferentes (M' deve ser maior que M). A diferença entre D' e M é s, ou mais-valia. As trocas capitalistas estão resumidas na Figura {\color{blue}3}.2.
 \par 
Marx aponta que o capital é um valor autoexpansível. O dinheiro age como capital somente quando é usado para gerar mais dinheiro ou, mais precisamente, quando é empregado na produção de mais-valia. Essa compreensão básica da natureza do capital permite que ele seja
 \par 
\section{Intercâmbio}
 \par 
\section{Intercâmbio}
 \par 
Figura {\color{blue}3}. {\color{blue} 2 } {\par} Troca capitalista: comprar para vender mais caro distingue-se das várias formas específicas que assume e das funções desempenhadas por essas formas, seja como dinheiro, factor de produção ou mercadoria. Cada um deles só é capital na medida em que contribui diretamente para a expansão do valor adiantado. Como tal, o dinheiro funciona como capital, bem como desempenha as suas tarefas específicas como meio de pagamento, depositário de valor de troca ou meio de produção.
 \par 
Caracterizamos o capital por meio da atividade dos capitalistas industriais (que inclui não apenas o capital de fabricação, mas também a prestação de serviços e outras atividades produtivas de mais-valia). Existem outras formas de capital, no entanto, a saber, o capital mercantil e o capital de empréstimo. Ambos também expandem o valor comprando (mercadorias ou ativos financeiros em vez de meios de produção) para vender mais caro. Ambos aparecem historicamente antes do capital industrial. Foi a percepção de Marx reverter sua ordem histórica de aparência, a fim de analisar o capitalismo abstratamente e em sua forma pura como um sistema social de produção. Isso permite que ele se concentre na relação salarial e na produção de (mais-valia) sem as complicações introduzidas por formas e relações de troca, incluindo mercantilismo ou usura, que apenas transferem valor (para uma análise mais detalhada dessas formas de capital, veja os Capítulos {\color{blue}11} e {\color{blue}12}).
 \par 
\section{Intercâmbio}
 \par 
A maioria dos economistas pode achar essa caracterização do capital como valor autoexpansível incontroversa, mesmo que um pouco estranha. Olhando para a Figura {\color{blue}3}. {\color{blue} 2 } {\par} , e com referência à Figura {\color{blue}3}. {\color{blue} 1 } {\par} , é evidente que, embora M e M' tenham valores diferentes, M e C têm o mesmo valor. Isso implica que valor extra foi criado no movimento C - M'. Esse valor adicionado (excedente) é a diferença entre os valores de saídas e entradas. A existência de mais-valia (lucro em sua forma monetária) é incontroversa, pois esta é obviamente a força motriz da produção capitalista, e
 \par 
M - C - M' é claramente a sua forma geral. O problema é fornecer uma explicação para a origem da mais-valia.
 \par 
Isto já foi localizado na produção, acima, ao mostrar que a troca não cria valor. Portanto, entre as mercadorias compradas pelo capitalista deve haver uma ou mais que criem mais valor do que custam. Por outras palavras, para a produção de mais-valia, pelo menos uma mercadoria deve contribuir com mais tempo de trabalho (valor) para os produtos do que custa para produzir como insumo; portanto, um dos seus valores de uso é a produção de valor (mais-valia). Como já foi indicado, isto deixa apenas um candidato - a força de trabalho.
 \par 
Primeiro, considere as outras entradas. Embora contribuam com valor para a produção como resultado do tempo de trabalho socialmente necessário para produzi-los no passado, a quantidade de valor que acrescentam à produção não é nem mais nem menos do que o seu próprio valor ou custo - pois de outra forma o dinheiro seria crescendo magicamente em árvores ou, pelo menos, em máquinas. Por outras palavras, os insumos não famosos não podem transferir mais valor para o produto do que custam como insumos, pois, como foi mostrado acima, trocas iguais não criam valor, e trocas desiguais não podem criar mais-valia, apenas alterar a sua distribuição onde ela já existe. .
 \par 
Agora consideremos a força de trabalho. O seu valor é representado pelo seu custo ou, mais precisamente, pelo valor obtido pelos trabalhadores com a venda da sua força de trabalho. Isto normalmente corresponde ao tempo de trabalho socialmente necessário para produzir os bens salariais regularmente adquiridos pela classe trabalhadora. Em contraste, o valor criado pela força de trabalho na produção é o tempo de trabalho exercido pelos trabalhadores em troca desse salário. Ao contrário dos outros factores de produção, não há razão para que a contribuição dos trabalhadores para o valor do produto (digamos, dez horas por dia por trabalhador) seja igual ao custo da força de trabalho (cujo valor pode ser produzido em, digamos, cinco horas). Na verdade, só pode ser porque o valor da força de trabalho é inferior ao tempo de trabalho social contribuído que é criada mais-valia.
 \par 
Utilizando o tempo de trabalho social como unidade de conta, foi demonstrado que o capital só pode expandir-se se o valor contribuído pelos trabalhadores exceder a remuneração recebida pela sua força de trabalho - a mais-valia é criada pelo excesso do tempo de trabalho sobre o valor do trabalho. poder. Portanto, a força de trabalho não cria apenas valores de uso: quando exercida como trabalho, também cria valor e, potencialmente, mais-valia. A força deste argumento é vista pela sua breve comparação com teorias alternativas de valor.
 \par 
As teorias da abstinência, da espera ou da preferência intertemporal dependem do sacrifício, por parte dos capitalistas, do consumo presente como fonte de lucros. Ninguém poderia negar que estes “sacrifícios” (geralmente feitos com conforto luxuoso) são uma condição para o lucro, mas, como milhares de outras condições, não são uma causa de lucros. Pessoas sem capital podiam abster-se, esperar e fazer escolhas intertemporais até ficarem com a cara azul, sem gerar lucros para si próprias. Não é a abstinência que cria capital, mas o capital que exige abstinência. A espera existiu em todas as sociedades; pode ser encontrado até mesmo entre esquilos, sem que eles obtenham qualquer lucro. Conclusões semelhantes aplicam-se à visão do risco como uma fonte de lucro. Deve-se sempre ter em mente que não são as coisas, abstratas ou não, que criam categorias económicas, por exemplo lucros ou salários, mas sim relações sociais definidas entre as pessoas.
 \par 
As teorias da produtividade marginal, no cerne da economia dominante, explicam o aumento do valor entre C e D' pelas contribuições tecnicamente (ou fisicamente) determinadas do trabalho e dos bens de capital para a produção. Uma tal abordagem não pode ter qualquer conteúdo social e não oferece nenhuma visão específica sobre a natureza do trabalho e do “capital” quando ligados ao capitalismo. Pois o trabalho e a força de trabalho (nunca claramente distinguidos um do outro) são tratados em pé de igualdade com as coisas, enquanto a teoria não tem interesse em explicar, nem capacidade de explicar, a organização social da produção. Apenas as quantidades de meios de produção e de força de trabalho importam, como se a produção fosse principalmente um processo tecnológico e não social. Contudo, factores de produção existiram em todas as sociedades; o mesmo não se pode dizer dos lucros, dos salários, das rendas ou mesmo dos preços, que, na sua actual difusão, são novos, historicamente falando. A explicação da forma do processo de produção, do modo de interacção social e de reprodução nele baseado, e das categorias a que eles dão origem, exige mais do que a teoria económica dominante é capaz de oferecer.
 \par 
Marx argumenta que todo valor (incluindo mais-valia ou lucro) é criado pelo trabalho, e que mais-valia é provocada pela exploração do trabalho direto ou vivo. Suponhamos que a jornada média de trabalho seja de dez horas e que o salário corresponda à metade do valor criado nesse tempo de trabalho. Então, durante cinco horas por dia, o trabalho é “gratuito” para a classe capitalista. Neste caso, a taxa de exploração, definida como o rácio entre o excedente e o tempo de trabalho necessário, é de cinco horas divididas por cinco horas, ou uma (100 por cento). Embora Marx se refira à taxa de mais-valia quando é específico sobre a exploração no capitalismo, este conceito poderia ser aplicado de forma semelhante a outros modos de produção, por exemplo, o feudalismo com taxas feudais ou a escravatura. A diferença é que, nestes dois últimos casos, o facto da exploração e a sua medida são aparentes, enquanto, sob o capitalismo, a exploração na produção é disfarçada pela liberdade de troca.
 \par 
Denote o tempo de trabalho excedente por s e o tempo de trabalho necessário por v. Juntos s e v constituem o trabalho vivo, l. Com s conhecido como mais-valia, v é chamado de capital variável e l é o valor recém-produzido:
 \par 
\section{Intercâmbio}
 \par 
A taxa de exploração é e = s ⁄v. Marx chama v de capital variável porque a quantidade de valor que será agregada pelos trabalhadores, l, não é fixada antecipadamente, quando são contratados, mas depende da quantidade de trabalho que pode ser extraída na linha de produção, na fazenda. , ou no escritório. É variável, em contraste com o capital constante, c. Não se trata de capital fixo (como é, por exemplo, uma fábrica que dura vários ciclos de produção), mas sim das matérias-primas e do desgaste do capital fixo, na medida em que são consumidos durante o período de produção. Por exemplo, um edifício ou máquina que tenha um valor de {\color{blue}100}.000 unidades de tempo de trabalho, mas que dure dez anos, contribuirá com {\color{blue}10}.{\color{blue}000} unidades de valor por ano para o capital constante. O valor do capital constante não varia durante a produção (uma vez que apenas o trabalho cria valor), mas é preservado (ou, noutros trabalhos, transferido para) a produção pelo trabalho do trabalhador, um serviço prestado livre e involuntariamente ao capitalista. Claramente, c e v são ambos capital porque representam o valor adiantado pelos capitalistas para obter lucro. Portanto, o valor λ de uma mercadoria é composto por capital constante e variável mais mais-valia, λ = c + v + s. Alternativamente, pode-se considerar que λ inclui o capital constante c mais o trabalho vivo v + s; finalmente, o custo da mercadoria é c + v, com a(s) mais-valia(s) formando os lucros do capitalista.
 \par 
\section{Intercâmbio}
 \par 
A mais-valia produzida depende da taxa de exploração e da quantidade de trabalho empregado (que pode ser aumentada pela acumulação de capital; ver Capítulo {\color{blue}6}). Suponhamos agora que os salários reais permanecem inalterados. A taxa de exploração pode ser aumentada de duas maneiras, e serão feitas tentativas para aumentá-la. Pois a natureza do capital como valor autoexpansível impõe um objetivo qualitativo a todo capitalista sob pena de extinção (isto é, falência empresarial e, potencialmente, degradação na classe dos trabalhadores assalariados): a maximização do lucro, ou pelo menos que o crescimento da lucratividade deveria ter alta prioridade.
 \par 
Primeiro, e pode ser aumentado através do que Marx chama de produção de mais-valia absoluta. Com base nos métodos de produção existentes - isto é, com os valores das mercadorias permanecendo os mesmos - a maneira mais simples de fazer isso é através da extensão da jornada de trabalho. Se, no exemplo dado acima, a jornada de trabalho for aumentada de dez para onze horas, com todo o resto constante, incluindo os salários, a taxa de exploração sobe de {\color{blue}5} ⁄ {\color{blue}5} para {\color{blue}6} ⁄ 5, um aumento de {\color{blue}20} por cento. A produção de mais-valia absoluta (s') é ilustrada na Figura {\color{blue}3}.3 (a mais-valia total é s + s').
 \par 
\section{Intercâmbio}
 \par 
\section{Intercâmbio}
 \par 
Existem outras maneiras de produzir mais-valia absoluta. Por exemplo, se o trabalho se torna mais intenso durante um determinado dia de trabalho, mais trabalho será realizado no mesmo período, e mais-valia absoluta será produzida. O mesmo resultado pode ser alcançado tornando o trabalho contínuo, sem pausas nem mesmo para descanso e refresco. A produção de mais-valia absoluta é frequentemente um subproduto da mudança técnica, porque a introdução de novas máquinas, como transportadores e, mais tarde, robôs na linha de produção, também permite a reorganização do processo de trabalho. Isso oferece uma desculpa para a eliminação de pausas ou "poros" no dia de trabalho que são fontes de ineficiência para os capitalistas e, simultaneamente, leva a um maior controle sobre o processo de trabalho (bem como maior intensidade de trabalho) e maior lucratividade, independentemente das mudanças de valor trazidas pelo novo maquinário.
 \par 
O ritmo de trabalho desejado também poderia ser obtido através de uma disciplina aplicada de forma grosseira. Pode haver supervisão constante por parte da gestão intermédia e sanções, até mesmo despedimento, ou recompensas por trabalho mais árduo (ou seja, produção de mais valor). Mas métodos mais indiretos também podem ser empregados. Um sistema de salários baseado em taxas por peça, por exemplo, destina-se a encorajar os trabalhadores a estabelecerem um ritmo de trabalho elevado, enquanto um prémio por horas extraordinárias é um incentivo para trabalhar para além do horário normal (embora o prémio não deva absorver a totalidade do excedente extra valor criado nesse tempo adicional, caso contrário não haveria lucro extra para o capitalista).
 \par 
Ainda outra forma de produzir mais-valia absoluta é a extensão do trabalho a toda a família da classe trabalhadora. À primeira vista, filhos, esposa e marido recebem salários separados. Mas o papel estrutural desempenhado socialmente por esses salários é fornecer os meios para reproduzir a família da classe trabalhadora e, portanto, a classe trabalhadora como um todo. Com a extensão do trabalho assalariado a toda a família, a pressão do mercado de trabalho (salários mais baixos devido a mais trabalhadores à procura de emprego) poderá até resultar na disponibilização de mais mão-de-obra com pouco ou nenhum aumento no valor dos salários como um todo.
 \par 
Existem limites até que ponto o capitalismo pode depender da produção de mais-valia absoluta. Independentemente dos limites naturais do número de horas do dia e das exigências fisiológicas de reprodução dos trabalhadores, a resistência da classe trabalhadora e, como resultado disso, as leis trabalhistas e as regras de saúde e segurança podem oferecer barreiras à extração de mais-valia absoluta. No entanto, a mais-valia absoluta é sempre importante nas fases iniciais do desenvolvimento capitalista, quando as cargas de trabalho tendem a aumentar rapidamente, e em qualquer altura é um remédio para a baixa rentabilidade (mesmo para os países capitalistas desenvolvidos contemporâneos) - na medida em que a medicina pode ser administrado. A mais-valia relativa não sofre das mesmas limitações e tende a tornar-se o método dominante de aumento à medida que o capitalismo se desenvolve (ver Capítulo {\color{blue}6}). A mais-valia relativa é produzida através da redução do valor da força de trabalho (v) através de melhorias na produção de bens salariais (com um salário real constante) ou, mais geralmente, através da apropriação de ganhos de produtividade pela classe capitalista. Neste caso, a jornada de trabalho permanece a mesma, por exemplo em dez horas, mas, devido aos ganhos de produtividade (directa ou indirectamente através do capital constante utilizado) na produção de bens salariais, v cai de cinco para quatro horas, deixando uma mais-valia de seis horas (e sobe de {\color{blue}5} ⁄ {\color{blue}5} para {\color{blue}6} ⁄ 4, ou seja, em {\color{blue}50} por cento). Existem várias formas de alcançar este resultado, incluindo o aumento da cooperação e uma divisão mais precisa do trabalho, a utilização de melhores máquinas e a descoberta científica e a inovação em toda a economia. A produção de mais-valia relativa é ilustrada na Figura {\color{blue}3}.4. Como resultado da mudança técnica, v cai para v', e a mais-valia relativa é produzida além da antiga mais-valia. (Este valor deve ser comparado com a Figura {\color{blue}3}.3.)
 \par 
\section{Intercâmbio}
 \par 
\section{Intercâmbio}
 \par 
\section{Intercâmbio}
 \par 
\section{Intercâmbio}
 \par 
A produção de mais-valia absoluta pode basear-se na determinação sombria de capitalistas individuais, recorrendo à ameaça de punição, despedimento e bloqueios, com uma intervenção estatal de apoio raramente considerada insuficiente quando necessária; por exemplo, para proteger a viabilidade da empresa, a competitividade da indústria ou para promover o “interesse nacional”. Em contraste, a produção de mais-valia relativa depende de todos os capitalistas, uma vez que nenhum produz sozinho uma proporção significativa das mercadorias necessárias para a reprodução da classe trabalhadora. Em particular, depende da concorrência e da acumulação na economia como um todo, induzindo mudanças técnicas que reduzem o valor da força de trabalho.
 \par 
\section{Intercâmbio}
 \par 
Marx atribui grande importância à análise da forma como a produção se desenvolve no capitalismo. Ele dedica considerável atenção tanto às relações de poder entre trabalhadores e capitalistas, como às relações técnicas sob as quais a produção ocorre, ao mesmo tempo que mostra que não devem ser tratadas separadamente: as tecnologias de produção incorporam relações de poder. Em particular, para o capitalismo desenvolvido, Marx argumenta que o sistema fabril predomina necessariamente (em vez de, por exemplo, a produção artesanal independente ou o sistema de distribuição, em que os capitalistas fornecem insumos aos trabalhadores artesanais e, mais tarde, recolhem as mercadorias produzidas). Dentro da fábrica, a produção de mais-valia relativa é perseguida sistematicamente através da introdução de novas máquinas, que podem trazer, pelo menos temporariamente, lucros extraordinários ao capitalista inovador. Novas máquinas aumentam a produtividade porque permitem que maiores quantidades de matérias-primas sejam transformadas em produtos finais num determinado tempo de trabalho. Inicialmente, a força física do trabalhador é substituída pela força das máquinas. Mais tarde, as ferramentas dos trabalhadores são incorporadas nas máquinas, transformando os trabalhadores em vigilantes ou apêndices das máquinas - para alimentar as máquinas e vigiá-las, e para se tornarem seus servos e não vice-versa (o que pode, no entanto, exigir níveis elevados de perícia técnica).
 \par 
A introdução de maquinaria aumenta a intensidade do trabalho de uma forma que difere daquela experimentada sob a produção de mais-valia absoluta, pois a nova maquinaria reestrutura inevitavelmente o processo de trabalho. Isto tem efeitos contraditórios sobre a classe trabalhadora. Eles são desqualificados pelas máquinas que os substituem e simplificam suas tarefas no trabalho, mas também são obrigados a adquirir novas habilidades à medida que várias dessas tarefas simplificadas são combinadas, muitas vezes simplesmente para que máquinas mais complexas possam ser operadas em níveis mais elevados de produtividade. . Da mesma forma, embora a carga física do trabalho seja aliviada pela potência das máquinas, também é aumentada através do maior ritmo, intensidade e reestruturação do trabalho, e pela necessidade de adaptar os corpos humanos às exigências impostas pelas novas tecnologias.
 \par 
Em grande medida, esta análise pressupõe um determinado conjunto de produtos e processos de produção que são sistematicamente transformados através do uso crescente de máquinas. Marx não negligencia, na verdade sublinha, o papel da ciência e da tecnologia na produção de inovação tanto em produtos como em processos. Mas tais desenvolvimentos não podem ser objecto de uma teoria geral, uma vez que a sua extensão e ritmo não ocorrem geralmente sob o comando da produção capitalista e dependem do progresso da descoberta científica, da tradução das descobertas em tecnologias mais produtivas, e do seu sucesso. introdução no local de trabalho. No entanto, Marx conclui que o sistema fabril leva a um aumento maciço na proporção entre capital físico e trabalho - o que ele chamou de composição técnica do capital (ver Capítulo {\color{blue}8}). Por um lado, isto decorre da definição de aumento de produtividade, uma vez que cada trabalhador transforma mais matérias-primas em produtos finais (caso contrário a produtividade não teria aumentado). Por outro lado, esta é uma condição para o crescimento da produtividade, uma vez que a massa de capital fixo sob a forma de máquinas e fábricas também deve aumentar.
 \par 
\section{Intercâmbio}
 \par 
A distinção de Marx entre trabalho produtivo e improdutivo é um corolário do seu conceito de mais-valia. O trabalho assalariado é produtivo se produzir diretamente mais-valia. Isto implica que o trabalho produtivo é o trabalho assalariado realizado para (e sob o controlo do) capital, na esfera da produção, e produzindo directamente mercadorias para venda com lucro. As mercadorias produzidas e o tipo de trabalho realizado - desde a construção naval, passando pela colheita, até à programação de computadores, ou ao ensino ou canto - são irrelevantes; além disso, as mercadorias evidentemente não precisam ser materiais.
 \par 
Todos os outros tipos de trabalho “assalariado” são improdutivos: por exemplo, trabalho que não é contratado pelo capital (como os produtores independentes de mercadorias, os trabalhadores por conta própria e a maioria dos funcionários públicos), trabalho que não é diretamente empregado na produção (como como gestores ou trabalhadores empregados em actividades de câmbio, incluindo os sectores retalhista e financeiro, bem como contabilistas, vendedores e caixas, mesmo que sejam empregados pelo capital industrial), e trabalhadores que não produzem mercadorias para venda (tais como empregadas domésticas e outros fornecedores independentes de serviços pessoais). A distinção produtivo-improdutivo é específica do trabalho assalariado capitalista. Não é determinado pelo produto da atividade, pela sua utilidade ou pela sua importância social, mas pelas relações sociais sob as quais o trabalho é realizado. Por exemplo, médicos e enfermeiros podem realizar trabalho produtivo ou improdutivo, dependendo da sua forma de emprego - numa clínica privada ou num hospital público, por exemplo. Embora as suas actividades sejam as mesmas, e possivelmente igualmente valiosas para a sociedade em certo sentido, num caso o seu emprego depende da rentabilidade da empresa, enquanto no outro caso prestam um serviço público que é potencialmente gratuito no ponto de entrega. .
 \par 
É importante sublinhar que, embora os trabalhadores improdutivos não produzam directamente mais-valia, são explorados se trabalharem mais tempo do que o valor representado pelo seu salário - ser improdutivos não é obstáculo à exploração capitalista! Do ponto de vista do capital, os setores improdutivos - o comércio retalhista, a banca ou o sistema público de saúde, por exemplo - são um entrave à acumulação porque absorvem parte da mais-valia produzida na economia para obter os meios para pagar os salários. , outras despesas e lucros próprios. Isto é feito através de transferências dos sectores produtores de valor através do mecanismo de preços. Por exemplo, o capital comercial compra mercadorias abaixo do valor e vende-as pelo valor, enquanto o capital que rende juros (incluindo bancos e outras empresas financeiras) obtém receitas principalmente através do pagamento de taxas de cliente e juros sobre empréstimos (ver Capítulos {\color{blue}11} e {\color{blue}12}). Finalmente, os serviços públicos são financiados por impostos gerais e, em alguns casos, por taxas de utilização. Nada disto sugere que os sectores improdutivos sejam “inúteis” em geral ou simplesmente um obstáculo à prosperidade social: estes sectores e os seus trabalhadores podem prestar serviços úteis à sociedade e/ou à acumulação de capital como um todo - mas não produzem directamente mais-valia.
 \par 
\section{Intercâmbio}
 \par 
O Volume {\color{blue}1} de O Capital está em parte preocupado com a questão: como o lucro é compatível com a liberdade de troca? A resposta dada transforma a questão na questão de como a mais-valia é produzida. Isto, por sua vez, é respondido com referência às propriedades únicas da força de trabalho como mercadoria e à extracção de mais-valia relativa e absoluta. Marx aborda estas questões nos termos teóricos aqui abordados, mas também com algum detalhe empírico, centrando-se nas mudanças nos próprios métodos de produção, especialmente na mudança da produção (literalmente produção manual) para o sistema fabril.
 \par 
É importante para Marx explicar a fonte da mais-valia antes de examinar como esta é distribuída como lucro (e juros) e renda - algo que ele faz no Volume {\color{blue}3} de O Capital (abordado em capítulos posteriores). Ele designa lucro, aluguel e salários como a “Fórmula da Trindade”. Estas formas de receita, que têm origens e lugares muito diferentes nas estruturas e processos do sistema capitalista, estão sujeitas à ilusão de que são colocadas simetricamente como os preços do capital, da terra e do trabalho, respectivamente, em vez de serem a base para um sistema de exploração.
 \par 
A teoria de Marx sobre capital e exploração é explicada em várias de suas obras, especialmente Marx (1976, pts 2-6). A interpretação neste capítulo se baseia em Ben Fine (1998) e Alfredo Saad-Filho (2002, caps. 3-5, 2003b). Para abordagens semelhantes, veja Chris Arthur (2001), Duncan Foley (1986, caps. 3-4), David Harvey (1999, caps. 1-2), Roman Rosdolsky (1977, pt.{\color{blue}3}), John Weeks (2010, cap.{\color{blue}3}) e as referências citadas no Capítulo {\color{blue}2}.
 \par 
Uma vez aceite a especificidade da teoria do valor de Marx e a sua ênfase na singularidade da força de trabalho como mercadoria, a sua teoria da exploração para explicar a mais-valia e o lucro é relativamente incontroversa. É necessário, porém, ver a mais-valia como o resultado da coerção para trabalhar além do valor da força de trabalho, e não como uma dedução daquilo que o trabalhador produz ou como uma parte tomada na divisão do produto líquido (como no que é denominado Abordagens Sraffianas ou Neo-Riparianas). Sobre isso, ver Ben Fine, Costas Lapavitsas e Alfredo Saad-Filho (2004), Alfredo Medio (1977) e Bob Rowthorn (1980, especialmente cap.{\color{blue}1}). A teoria da exploração de Marx inspirou uma rica veia de análises complementares do processo de trabalho, tanto nos seus aspectos técnicos como organizacionais. A ciência e a tecnologia não melhoram simplesmente a técnica; são governados, se não determinados, pelo imperativo da rentabilidade, com a correspondente necessidade de controlar e disciplinar o trabalho (influenciando assim o que é inventado e como, e, da mesma forma, o que é adotado na produção e como); ver Brighton Labor Process Group (1977), Les Levidow (2003), Les Levidow e Bob Young (1981, 1985), Phil Slater (1980), Bruno Tinel (2012) e Judy Wajcman (2002). Além disso, existe um imperativo de garantir a venda (com lucro), com um correspondente afastamento dos produtos e métodos de venda das necessidades sociais dos consumidores (por mais definidas e determinadas que sejam) na procura da rentabilidade privada dos produtores; veja Ben Fine (2002), por exemplo. Em contraste com a economia dominante, para a qual o trabalho passou a ser tratado como uma desutilidade por necessidade, em oposição a uma consequência da sua organização sob o capitalismo, ver David Spencer (2008).
 \par 
Finalmente, a distinção entre trabalho produtivo e improdutivo é importante como ponto de partida para examinar os diferentes papéis desempenhados pelos trabalhadores industriais, financeiros, do sector público e outros trabalhadores na reprodução económica e social (ver Capítulo {\color{blue}5}). Um debate tem sido sobre se a distinção é válida ou válida - por exemplo, com base no facto de todo o trabalho explorado (assalariado) dever ser agrupado como fontes de mais-valia. Outro debate, entre aqueles que aceitam a distinção, diz respeito a quem deve ser considerado produtivo: isto pode ser definido de forma restrita, como incluindo apenas o trabalho assalariado manual, ou mais amplamente, como incluindo todos os trabalhadores assalariados. Marx explica suas categorias de trabalho produtivo e improdutivo em Marx (1976, app., 1978a, cap.{\color{blue}4}). Essas categorias são discutidas por Ben Fine e Laurence Harris (1979, cap.{\color{blue}3}), Simon Mohun (2003, 2012), Isaak I. Rubin (1975, cap. {\color{blue} 19 } {\par} , 1979, cap.{\color{blue}24}) e Sungur Savran e Ahmet Tonak ( 1999).