\begin{figure}
	\centering
	\includegraphics[width=1.\textwidth]{temp\_files/images/UP\_logo.png }
	\caption{Lily Braun (1865-1916): Escritora feminista e política dentro do Partido Social-Democrata Alemão. Seu livro de 1901, The Women’s Question: Historical Development and Economic Aspect, propôs muitas soluções inovadoras para os desafios enfrentados pelas mães trabalhadoras, incluindo propostas para o que ela chamou de “seguro-maternidade”. Braun era moderada e reformadora e não acreditava que a revolução fosse necessária para atingir o socialismo. Cortesia do Lebendiges Museum On-line (Deutsches Historisches Museum).}
	\label{ }
\end{figure}
 \par 
\chapter{O QUE ESPERAR QUANDO VOCÊ ESTÁ ESPERANDO EXPLORAÇÃO: SOBRE A MATERNIDADE}\label{O QUE ESPERAR QUANDO VOCÊ ESTÁ ESPERANDO EXPLORAÇÃO: SOBRE A MATERNIDADE}
 \par 
Né dos meus amigos de infância, a quem chamarei de Jake, © ansiava por sucesso financeiro numa sociedade onde o sucesso financeiro refletia uma espécie de superioridade moral. Jake valorizou a ideia do sonho americano. Ele viu a bondade no tipo de Horatio Alger, o trabalho árduo necessário para “fazer algo” de si. Naquela época, eu já era uma feminista preocupada com a desigualdade econômica, enquanto Jake, fiel ao espírito da década de 1980, acreditava que quem morre com mais brinquedos vence. Passamos horas a debater os prós e os contras do capitalismo e as formas como o Thatcherismo e a Reaganomics foram ou não uma droga. Jake abraçou o espírito da época de Gordon Gekko: “A ganância é boa”. Eu não estava acreditando. Mas naquela época em que a política interna não era tão polarizada, conseguimos manter a nossa amizade durante os anos de faculdade. Na década de 1990, enquanto eu estava ensinando inglês e lendo Karl Polanyi no Japão, Jake estava subindo na hierarquia corporativa em uma start-up de tecnologia.
 \par 
49
 \par 
50
 \par 
O QUE ESPERAR QUANDO VOCÊ ESTÁ ESPERANDO EXPLORAÇÃO: SOBRE A MATERNIDADE
 \par 
Um dia, em 1997, Jake me informou com grande prazer que havia contratado uma jovem promissora para uma posição estratégica em sua empresa. Ela havia sido finalista com outros dois homens e, com minha voz ecoando em seus ouvidos, ele decidiu arriscar nela. "Eles eram todos igualmente qualificados no papel", ele me disse, "mas após anos ouvindo seus discursos feministas, convenci minha chefe de que, como as mulheres enfrentam tantas barreiras na tecnologia, ela havia realmente trabalhado mais para chegar onde estava do que os homens na piscina". Eu estava lutando no meu primeiro ano de pós-graduação na época, e as notícias de Jake aqueceram meu coração; eu havia feito uma pequena diferença no mundo.
 \par 
Nos anos seguintes, a mulher provou ser inteligente, competente e trabalhadora. A empresa de Jake deu a ela um período sabático remunerado de três meses para treinamento adicional, preparando-a para uma promoção. Então ela anunciou que estava grávida. A startup não tinha uma política formal de licença-maternidade, mas Jake pediu ao seu chefe que lhe desse doze semanas remuneradas para ficar em casa com seu bebê e fazer arranjos para cuidar das crianças. Jake argumentou que eles já tinham investido tanto dinheiro em seu treinamento que uma licença de doze semanas se pagaria a longo prazo. Seu chefe concordou relutantemente.
 \par 
\textit\textbf{ {A mulher voltou a trabalhar após o nascimento do bebê e tentou o máximo para atender às demandas de uma pequena start-up. Mas ela estava amamentando. E o bebê a mantinha acordada à noite. Ela comparecia às reuniões com os olhos turvos e despreparada. Ela ligava dizendo que estava doente quando a babá não aparecia. Além disso, ela encontrou uma vaga em uma boa creche, mas se seu filho adoecesse, eles o mandavam para casa. Seu marido viajava a negócios e ela não tinha família na área. Jake, sempre otimista, acreditava que as coisas melhorariam quando a criança crescesse. Ele} }
 \par 
KRISTEN R. GHODSEE até se ofereceu para ser babá em uma emergência. Sua funcionária estrela conseguiu segurar por seis meses. Então ela pediu demissão.
 \par 
Naquela noite, Jake me ligou para compartilhar a notícia. Abatido e frustrado, ele me disse: “Nunca mais vou contratar uma mulher.
 \par 
“Mas ela é apenas uma mulher”, eu disse. “Nem toda mulher vai fazer sua escolha.”
 \par 

 \par 
Acho que desliguei na cara dele. Mas não foi culpa do Jake. O que ele poderia fazer em um sistema que não dá suporte para mulheres quando elas se tornam mães, que força as mulheres a escolher entre suas carreiras e suas famílias? Os economistas chamam isso de "discriminação estatística". A ideia básica é que, como os empregadores não podem observar diretamente a produtividade de cada trabalhador, eles podem fazer observações sobre características demográficas correlacionadas com a produtividade do trabalhador. Eles tomam decisões com base nas médias: se as mulheres são mais propensas a pedir demissão do que os homens por motivos pessoais, os empregadores presumem que qualquer mulher tem mais probabilidade de pedir demissão do que um homem. Os economistas observam que a teoria da discriminação estatística pode criar um ciclo vicioso. Se as mulheres são (ou costumavam ser) mais propensas a pedir demissão, elas receberão menos. Se elas recebem menos, elas têm mais probabilidade de pedir demissão. Esse ciclo vicioso fornece uma justificativa ótima para a intervenção do governo.
 \par 
51
 \par 
52
 \par 
O QUE ESPERAR QUANDO VOCÊ ESTÁ ESPERANDO EXPLORAÇÃO: SOBRE A MATERNIDADE
 \par 
A percepção da inferioridade comparativa das mulheres como trabalhadoras está ligada à sua capacidade biológica para gerar filhos e amamentar, e à concomitante expectativa social de que as mulheres serão as principais cuidadoras de bebês e crianças pequenas. E em algum mundo de fantasia patriarcal, nossa natureza supostamente inata de cuidado também nos torna perfeitamente adequados para cuidar de outros parentes doentes, fracos ou idosos. E como as mulheres estão em casa de qualquer maneira, assim continua o argumento, poderíamos muito bem-fazer todas as compras, cozinhar, limpar e todo o trabalho emocional necessário para manter uma casa, certo? Alguém tem que fazer isso, e esse alguém é quase sempre uma mulher, em parte porque a localização das tarefas está alinhada, mas também porque ela foi socializada desde a infância para acreditar que esse é o seu papel natural. Bonecas, fornos EZ Bake e aspiradores de pó de brinquedo permitem que as meninas brinquem e pratiquem os trabalhos que realizarão quando crescerem.
 \par 
Os empregadores discriminam aqueles cujos corpos podem gerar filhos porque a sociedade atribui certas características aos donos desses corpos. Quando os acadêmicos falam sobre homens e mulheres, eles geralmente fazem uma distinção entre os termos "sexo" e "gênero". A palavra "sexo" significa a diferença biológica entre homens e mulheres e a palavra "gênero" conota os papéis sociais que as culturas esperam que correspondam à biologia. Por exemplo, por sexo, sou uma mulher porque tenho o equipamento fisiológico necessário para a fabricação de bebês, mas meu gênero também é feminino porque, de muitas maneiras, me conformo com a imaginação da sociedade americana contemporânea sobre o que uma mulher deve ser: tenho cabelo comprido; uso saias, joias e maquiagem; gosto de comédias românticas e bons produtos de banho; e embora eu possa alegar que é para minha saúde geral, faço uma hora diária na elíptica
 \par 
KRISTEN R. GHODSEE treinadora porque me preocupo com meu peso (ok, bem, talvez sejam apenas quarenta e cinco minutos, e não é todo dia, mas você entendeu a ideia). De outras maneiras, no entanto, minha identidade de gênero é mais masculina: sempre trabalhei em período integral e ganhei meu próprio dinheiro; gosto de assistir futebol, ficção científica e filmes de ação; adoro uma boa cerveja; e embora eu tente ser educada sobre isso, sempre falo o que penso, mesmo que meus pensamentos e opiniões possam ofender. Não tolero tolos, enquanto, de acordo com alguns, mulheres de verdade toleram apalpadores, explicadores e idiotas comuns com um sorriso.
 \par 
A discriminação de gênero surge porque a sociedade constrói arquétipos do homem ideal e da mulher ideal com base em suas supostas diferenças biológicas naturais. Isso não quer dizer que homens e mulheres sejam iguais — eles não são — mas apenas que nossas crenças sobre como homens e mulheres se comportam são uma invenção de nossa imaginação coletiva — uma invenção poderosa, sim, mas uma invenção mesmo assim. Quando um aluno classifica um professor com um nome feminino abaixo de um professor com um nome masculino, o aluno pode presumir que o professor tem mais tempo e energia para se dedicar ao seu ensino porque ele não está distraído com suas obrigações de cuidado fora do trabalho. Quando empregadores como o chefe do meu amigo Jake veem o nome de uma mulher em um formulário de emprego, eles imediatamente pensam que "mulher" é igual a uma mãe em potencial com prioridades na vida que têm precedência sobre suas carreiras. Os empregadores também presumem que os homens colocarão suas carreiras acima de suas famílias porque eles são supostamente menos apegados biologicamente aos filhos. Não importa se os homens decidem ficar em casa com os filhos ou se as mulheres se esterilizam para superar os desafios do equilíbrio entre trabalho e família; os nossos estereótipos de gênero sobre como os homens e as mulheres
 \par 
53
 \par 
54
 \par 
O QUE ESPERAR QUANDO ESTÁ ESPERANDO EXPLORAÇÃO: NA MATERNIDADE o comportamento está enraizado em nossas ideias sobre a ligação “natural” entre o sexo biológico e como isso influência nossas escolhas de vida.
 \par 
Eu costumava fazer um exercício em sala de aula com meus alunos para fazê-los pensar sobre a relação entre sexo e gênero. Peguei emprestado um cenário do clássico romance de ficção científica de Ursula Le Guin, The Left Hand of. Darkness, onde um homem da Terra é enviado para trabalhar em um planeta de "hermafroditas bissexuais". Isso significa que todas as pessoas têm órgãos sexuais e hormônios masculinos e femininos. Ao longo do mês, há períodos de sete dias em que uma parte da população experimenta uma forma de calor: um desejo irresistível de copular. No início do contato sexual, um dos membros do par se torna o macho, e a outra pessoa se torna a fêmea. Em qualquer encontro sexual, um indivíduo se tornará aleatoriamente o macho ou a fêmea. O membro do par que se torna fêmea pode engravidar e terá um período de gestação de nove meses antes de dar à luz. Quando um indivíduo não está copulando ou grávido, ele reverte para um estado neutro até seu próximo encontro sexual, quando o processo se repete. Qualquer indivíduo pode, portanto, ser pai e mãe, e todos estão igualmente “em risco” de gravidez e parto.
 \par 
Pedi aos meus alunos que tentassem imaginar como a sociedade neste planeta fictício seria organizada em comparação com a nossa sociedade nos Estados Unidos. A primeira coisa a desaparecer seria a discriminação sexual, já que todos seriam biologicamente idênticos. Todas as pessoas são “hermafroditas”, então você não poderia usar o sexo biológico para criar hierarquias. Claro, “hermafroditas bissexuais” mais atraentes poderiam desfrutar de mais privilégios do que os feios, e os velhos poderiam ter mais poder sobre os jovens, mas a discriminação não
 \par 
KRISTEN R. GHODSEE seria baseado em se você pode fazer bebês. Da mesma forma, os papéis sociais ligados à biologia seriam os mesmos para todos, já que a maioria dos membros desta sociedade seriam mães e pais de vários filhos. Meus alunos também imaginaram que a sociedade neste planeta fictício seria organizada para acomodar as demandas da gravidez e do parto, já que cada membro daquela sociedade se beneficiaria de formas coletivamente organizadas de apoio.
 \par 
Os socialistas há muito entenderam que criar equidade entre homens e mulheres, apesar de suas diferenças biológicas de sexo, requer formas coletivas de apoio à criação dos filhos. Em meados do século XIX, quando as mulheres inundaram a força de trabalho industrial da Europa, os socialistas teorizaram que não era possível construir movimentos fortes de trabalhadores sem a participação das mulheres. A feminista alemã Lily Braun promoveu a ideia de um "seguro-maternidade" financiado pelo estado já em 1897. Nesse esquema, as trabalhadoras desfrutariam de licenças remuneradas de seus empregos antes e depois do parto, com garantias de que seus empregos seriam mantidos em sua ausência. É importante lembrar que, ainda em 1891, na Alemanha, as trabalhadoras industriais trabalhavam por um mínimo de sessenta e cinco horas por semana, mesmo se estivessem grávidas. Nessas circunstâncias, mulheres grávidas e meninas ficavam na linha de montagem até darem à luz e, se não tivessem marido ou família para sustentá-las, voltavam ao trabalho logo depois. A taxa de mortalidade infantil e materna para mulheres trabalhadoras era mais do que o dobro da das mulheres de classe média devido às duras condições.
 \par 
Embora as feministas britânicas e americanas quisessem
 \par 
55
 \par 
56
 \par 
O QUE ESPERAR QUANDO VOCÊ ESTÁ ESPERANDO EXPLORAÇÃO: SOBRE A MATERNIDADE apoiar mães trabalhadoras por meio de instituições de caridade não estatais, Braun propôs que os fundos para o seguro-maternidade fossem levantados por meio de um imposto de renda progressivo. O governo alemão poderia então pagar o salário de uma mulher por um período fixo antes e depois do nascimento de seu filho. Todos contribuiriam para um pote especial de dinheiro do qual as novas mães poderiam sacar, muito parecido com o seguro-desemprego ou uma pensão estatal. Braun afirmou que, uma vez que a sociedade se beneficiava das crianças, ela deveria ajudar a arcar com os custos de criá-las. As crianças são futuros soldados, trabalhadores e contribuintes. Elas são um benefício para todos, não apenas para os pais que as trazem ao mundo (e alguns pais de adolescentes podem argumentar que elas são mais benéficas para a sociedade do que para seus pais). Isso é especialmente verdadeiro em estados etnicamente homogêneos, onde as sociedades dão um prêmio à preservação de uma identidade nacional específica.
 \par 
Mas a proposta de Braun era cara. Exigia novos impostos e redistribuiria a riqueza pelas classes trabalhadoras, uma ideia à qual muitos homens e mulheres da classe média se opunham. As ideias de Braun também enfrentaram oposição inicial da esquerda. Como Braun era uma reformadora e acreditava que o seu esquema de maternidade poderia ser implementado sob o capitalismo, socialistas alemães mais radicais como Clara Zetkin rejeitaram inicialmente as suas ideias, alegando que só poderiam ser realizadas sob uma economia socialista. Braun também favoreceu arranjos de vida comunitária (comunas) em vez de creches e jardins de infância financiados pelo Estado, enquanto Zetkin acreditava que o trabalho doméstico e o cuidado dos filhos deveriam ser socializados. No entanto, as propostas de Braun, pelo menos de forma diluída, foram transformadas em lei já em 1899. E pela Segunda Conferência Internacional de Mulheres Socialistas em 1910, a proposta de Braun
 \par 
As ideias de KRISTEN R. GHODSEE foram incorporadas à plataforma socialista oficial com o apoio de Clara Zetkin e da russa Alexandra Kollontai.
 \par 
O quarto ponto na plataforma socialista de 1910 estabeleceu a base para todas as políticas socialistas subsequentes sobre as responsabilidades do Estado em relação às trabalhadoras. Sob o título “Proteção Social e Provisão para Maternidade e Bebês”, as mulheres da Segunda Internacional exigiram uma jornada de trabalho de oito horas. Elas propuseram que as mulheres grávidas parassem de trabalhar (sem aviso) por oito semanas antes da data prevista para o parto, e que as mulheres recebessem um “seguro de maternidade” pago de oito semanas se a criança sobrevivesse, o que poderia ser estendido para treze semanas se a mãe estivesse disposta e fosse capaz de amamentar o bebê. As mulheres teriam uma licença de seis semanas para crianças natimortas, e todas as mulheres trabalhadoras desfrutariam desses benefícios, “incluindo trabalhadoras agrícolas, trabalhadoras domésticas e empregadas domésticas”. Essas políticas seriam pagas pelo estabelecimento permanente de um fundo especial de maternidade com receitas fiscais”.
 \par 
Sete anos depois, Kollontai tentou implementar algumas dessas políticas na União Soviética após a revolução bolchevique. Em vez de sobrecarregar mulheres individuais com tarefas domésticas e cuidados infantis, além de seu trabalho industrial, o jovem estado soviético propôs construir jardins de infância, creches, lares para crianças e refeitórios e lavanderias públicas. Em 1919, o Oitavo Congresso do Partido Comunista deu a Kollontai um mandato para expandir seu trabalho para as mulheres soviéticas, e ela garantiu compromissos estatais para gastar os fundos necessários para construir uma ampla rede de serviços sociais. O ano de 1919 também viu a criação
 \par 
57
 \par 
58
 \par 
O QUE ESPERAR QUANDO VOCÊ ESTÁ ESPERANDO EXPLORAÇÃO: SOBRE A MATERNIDADE de uma organização chamada Zhenotdel, a Seção Feminina, que supervisionaria o trabalho de implementação do programa radical de reforma social que levaria à emancipação total das mulheres.*
 \par 
Mas o entusiasmo soviético pela emancipação das mulheres logo evaporou diante de preocupações demográficas, econômicas e políticas mais urgentes. Depois que o país foi devastado pelos anos brutais da Primeira Guerra Mundial, seguidos pela Guerra Civil e pela fome horrenda de 1921 e 1922, Lenin e os bolcheviques não tinham fundos para apoiar o plano de Kollontai. Centenas de milhares de órfãos de guerra vagavam pelas principais cidades, atormentando os moradores com pequenos crimes e roubos. O estado não tinha recursos para cuidar deles; os lares para crianças estavam sobrecarregados e com falta de pessoal. A liberalização das leis de divórcio significava que os pais abandonavam suas esposas grávidas, e a fraca aplicação das leis de pensão alimentícia e pensão alimentícia significava que os homens que sobreviveram à Primeira Guerra Mundial, à Guerra Civil e à fome rotineiramente se esquivavam de suas responsabilidades. As mulheres trabalhadoras não podiam cuidar de seus filhos e esperavam que o estado interviesse e ajudasse, como Kollontai e as outras ativistas femininas haviam prometido. Em 1920, a União Soviética também se tornou o primeiro país da Europa a legalizar o aborto sob demanda durante as primeiras doze semanas de gravidez. As taxas de natalidade despencaram quando as mulheres buscaram limitar o tamanho de suas famílias. Eventualmente, houve o medo de que a queda na taxa de natalidade combinada com a devastação da guerra e da fome descarrilaria os planos do país para uma rápida modernização.
 \par 
Ninguém jamais quis que a independência econômica das mulheres fosse alcançada à custa da maternidade, mas é isso que
 \par 
KRISTEN R. GHODSEE aconteceu. À medida que as demandas sobre o tempo das mulheres soviéticas aumentavam, elas escolheram atrasar ou limitar a gravidez. Eventualmente, Stalin dissolveu o Zhenotdel, declarando que o
 \par 
A “questão da mulher” havia sido resolvida. Em 1936, ele reverteu a maioria das políticas liberais, proibiu o aborto e restabeleceu a família tradicional, além de seu programa sustentado de terror de estado e expurgos arbitrários. O estado soviético em rápida industrialização precisava de mulheres para trabalhar, ter filhos e fazer todo o trabalho de cuidado que o primeiro estado socialista do mundo ainda não tinha condições de pagar. As mulheres soviéticas estavam longe de serem emancipadas, e Alexandra Kollontai passou a maioria de seus anos restantes em exílio diplomático.
 \par 
Embora o experimento soviético tenha falhado, as ideias de Braun e o programa das mulheres socialistas em 1910 encontraram solo fértil nas democracias sociais escandinavas. Os dinamarqueses introduziram uma licença de duas semanas para mulheres trabalhadoras já em 1901, e em 1960 uma licença-maternidade universal, paga e financiada pelo estado foi estendida a todas as mulheres trabalhadoras. Em 1919, a Finlândia aprovou disposições de licença-maternidade para operárias de fábrica e mulheres profissionais, e adicionou proteções de emprego em 1922. A Suécia introduziu uma licença-maternidade não remunerada de quatro semanas já em 1901, e em 1963, o governo garantiu às mulheres {\color{blue}180} dias de licença-maternidade protegida pelo emprego a {\color{blue}80} por cento de seus salários. Compare isso com os Estados Unidos, que nem sequer aprovaram uma lei proibindo a discriminação contra mulheres grávidas até 1978. E as mulheres americanas não tinham uma lei federal para licença não remunerada protegida pelo emprego até 1993. Ainda não temos licença-maternidade remunerada obrigatória (mas, por outro lado, também não temos licença médica remunerada obrigatória).
 \par 
59
 \par 
60
 \par 
O QUE ESPERAR QUANDO VOCÊ ESTÁ ESPERANDO EXPLORAÇÃO: SOBRE A MATERNIDADE
 \par 
Os países do Leste Europeu também fizeram uso precoce de disposições de licença-maternidade. A Polônia concedeu doze semanas de licença-maternidade totalmente remunerada em 1924, mas a maioria dos países introduziu essas disposições após a Segunda Guerra Mundial. Essas nações precisavam de mulheres para trabalhar porque havia escassez de mão de obra masculina, mas também investiram pesadamente na educação e treinamento profissional das mulheres e não queriam perder sua especialização (pense no raciocínio de Jake no início deste capítulo). Por exemplo, os tchecoslovacos introduziram as primeiras políticas de apoio à maternidade em 1948 e, em 1956, o Código do Trabalho garantiu às mulheres dezoito semanas de licença remunerada e protegida pelo emprego. Na Bulgária, a constituição de 1971 garantiu às mulheres o direito à licença-maternidade. Em 1973, as mulheres búlgaras desfrutavam de uma licença-maternidade totalmente remunerada de {\color{blue}120} dias antes e depois do nascimento do primeiro filho, bem como seis meses extras de licença paga ao salário mínimo nacional. As novas mães também podiam tirar licença não remunerada até que seus filhos completassem três anos, quando uma vaga em um jardim de infância público seria disponibilizada. O tempo em licença-maternidade contava como serviço de trabalho para a pensão da mulher, e todas as licenças eram protegidas pelo emprego. Mais tarde, uma lei alterada permitiu que pais e avós tirassem licença parental no lugar da mãe. Os búlgaros cobriram aqueles em licença parental com o trabalho de novos graduados universitários. (Na Bulgária, a educação pós-secundária era gratuita para "estudantes que concordassem em completar um período de serviço nacional obrigatório após obterem seus diplomas. Esses estágios permitiam que os jovens ganhassem experiência de trabalho e garantiam que o emprego de um pai ou mãe estaria esperando quando ele ou ela retornasse da licença.)" A decisão do Politburo búlgaro de 1973 também incluiu linguagem sobre reeducar os homens para serem mais ativos no
 \par 
TTTTTT home: “A redução e o alívio do trabalho doméstico da mulher dependem muito da participação comum dos dois cônjuges na organização da vida familiar. É, portanto, imperativo: a) combater visões, hábitos e atitudes ultrapassadas quanto à distribuição do trabalho dentro da família; b) preparar os jovens para o desempenho das tarefas domésticas desde a infância e adolescência, tanto pela escola e sociedade como pela família.”*
 \par 
Nas páginas da revista feminina búlgara The Woman Today, os editores publicaram artigos sobre homens fazendo sua parte justa do trabalho doméstico e encorajando-os a serem pais mais ativos para seus filhos. Nos Jovens Pioneiros e no Komsomol, duas organizações juvenis integradas por gênero, meninos e meninas eram socializados para tratar reciprocamente como iguais, ambos com papéis importantes (embora diferentes) a desempenhar na construção de uma sociedade socialista. Onde os homens faziam serviço militar obrigatório após o ensino médio, os trabalhos reprodutivos das mulheres contavam como uma forma equivalente de serviço nacional. No final, essas políticas falharam em desafiar os papéis tradicionais de gênero, mas é importante reconhecer que houve pelo menos tentativas de redefinir ideias sobre masculinidade e feminilidade. De fato, esforços estatais específicos para encorajar os homens a serem pais mais ativos e participar mais do trabalho doméstico podem ser encontrados já na década de 1950 na Alemanha Oriental e na Tchecoslováquia. Entretanto, diante da recalcitrância masculina, os governos concentraram seus esforços na socialização das tarefas domésticas e do cuidado das crianças, na esperança de expandir a rede de cozinhas comunitárias e lavanderias públicas por todo o país.
 \par 
Já em 1817, o socialista utópico britânico Robert Owen sugeriu que as crianças com mais de três anos
 \par 
61
 \par 
62 deve ser criado por comunidades locais em vez de famílias nucleares, e essa ideia de provisão pública de cuidados infantis influenciou todos os experimentos do século XX com o socialismo estatal. Além das licenças-maternidade, países como Polônia, Hungria, Tchecoslováquia, Bulgária, Alemanha Oriental e Iugoslávia investiram fundos estatais para expandir a rede de creches (para crianças do nascimento aos três anos) e jardins de infância (para crianças de três a seis anos) para apoiar a participação contínua das mulheres na força de trabalho. Claro, a qualidade dessas creches era desigual em toda a região e frequentemente deixava muito a desejar; as crianças adoeciam com mais doenças transmissíveis, e os cuidadores eram frequentemente sobrecarregados pelas demandas de muitas crianças (problemas comuns em creches hoje em dia). Mas, como acontece com tantas coisas na economia de comando, os planejadores alocavam recursos de forma ineficiente, e a demanda sempre excedia a oferta. Em minha pesquisa nos arquivos do Comitê de Mulheres Búlgaras, por exemplo, descobri muitas cartas aos ministérios relevantes reclamando da falta de fundos alocados para creches e jardins de infância. Aqui, novamente, os países do norte da Europa, Suécia, Noruega, Dinamarca e Finlândia, se saíram muito melhor. Eles investiram fundos estatais para construir creches para promover o pleno emprego das mulheres. No final da Guerra Fria, as taxas de participação da força de trabalho feminina escandinava eram superadas apenas pelas das mulheres do Bloco Oriental.
 \par 
Após a publicação do meu artigo de opinião no New York Times, recebi inúmeras mensagens de leitores ocidentais que discutiam suas próprias frustrações. Muitas mulheres que cresceram no Bloco Oriental também me escreveram para relatar suas memórias
 \par 
\[KRISTEN R. GHODSEE\]
 \par 
TTTTTT e opiniões sobre a vida sob o socialismo, confirmando com suas anedotas pessoais que nem tudo era tão sombrio por trás da Cortina de Ferro. Minha carta favorita veio de uma mulher que morava na Suíça, nascida em uma família de classe média na Tchecoslováquia em 1943. Ela detalhou suas próprias lembranças da vida sob o socialismo de estado:
 \par 
A licença-maternidade SSSSSS durou oito meses e depois voltei a trabalhar. Eu tinha que acordar nossa filhinha gentilmente todas as manhãs às 5h30, pois a creche abria às 6h, e levava {\color{blue}15} minutos de bonde para chegar lá. Uma vez na creche, eu tinha que vesti-la com um uniforme e correr para pegar o ônibus às 6h30 para ir ao trabalho. Muitas vezes eu mal conseguia pegar o ônibus, e não era incomum que as portas do ônibus se fechassem atrás de mim com parte do meu casaco ainda pendurado do lado de fora. Na época, meu marido saía do trabalho às 14h, o que significava que ele poderia pegar nossa filha, comprar algumas compras e preparar o jantar a tempo para meu retorno por volta das 17h. Logo depois disso, colocávamos nossa filha para dormir, pois o dia seguinte prometia a mesma rotina apressada do dia anterior. Meu marido e eu estávamos cansados ​​depois de um dia assim.”
 \par 
A mulher suíço-checoslovaca realmente quis dizer essa descrição de sua vida anterior como uma crítica à versão alemã do artigo de opinião. Ela sentiu que sua vida era muito apressada
 \par 
63
 \par 
\[NC ANN UMA\]
 \par 
64 para sexo com o marido. Como mãe trabalhadora, entendo certamente o quão difícil é administrar o equilíbrio entre trabalho e família, mas não acho que essa mulher (com setenta e quatro anos quando me escreveu em 2017) tenha percebido a extensão de seu privilégio na Tchecoslováquia socialista estatal em comparação à situação das mulheres trabalhadoras hoje. Em sua crítica, ela menciona que ela e o marido tinham seu próprio apartamento particular, ela tinha oito meses de licença-maternidade, seu filho tinha uma vaga em uma creche financiada pelo estado a quinze minutos de casa, e seu marido saía do trabalho às duas da tarde e pegava a filha, comprava mantimentos e preparava o jantar antes que ela voltasse para casa às cinco. Ela me conta que ela e o marido estavam exaustos com essa "rotina apressada", mas suspeito que ela não tenha ideia de quão luxuosa essa rotina pode soar para as mulheres, mesmo as europeias, que tentam equilibrar trabalho e família hoje. Na verdade, a Cambridge Women’s Pornography Cooperative publica um livro chamado Porn for Women, que apresenta homens que buscam seus filhos, compram mantimentos e cozinham o jantar antes que suas esposas cheguem do trabalho.
 \par 
Para muitas mulheres, o acesso a cuidados infantis de qualidade e a preços acessíveis é mais importante do que a licença de maternidade, especialmente se esta última não estiver protegida pelo emprego. Quando comecei como professor assistente, estava longe da família e coloquei minha filha pequena na creche do campus em tempo integral, cinco dias por semana. Um dos meus colegas tinha três filhos com menos de quatro anos: duas meninas gêmeas de três anos e um filho de um ano. Este colega, que eu
 \par 
KRISTEN R. GHODSEE ligará para Leslie, era uma profissional estabelecida antes da maternidade e não tinha desejo de renunciar a sua carreira. Ela aceitou um emprego de três quartos de tempo bem abaixo de suas qualificações, e seu marido também providenciou uma redução para uma semana de quatro dias. Leslie pagou pelos três dias inteiros restantes de creche para seus três filhos diretamente por meio de um desconto na folha de pagamento. No final de cada mês, ela entrava no meu escritório com seu contracheque. Após impostos, pagamentos de seguro e o custo da creche, Leslie ganhava cerca de setenta centavos por mês. Ela trabalhava trinta horas por semana e frequentemente fazia horas extras não remuneradas para eventos noturnos, por menos de US$ 9,00 de salário líquido por ano. E ela fez isso por três anos!
 \par 
Certa vez perguntei a Leslie por que ela simplesmente não ficava em casa com as crianças, e ela admitiu que frequentemente fantasiava sobre isso. Mas ela se recusou a desistir de sua vida profissional e temia ter uma lacuna em seu currículo. “Já vi muitas mulheres profissionais ficarem completamente desanimadas após tirar um tempo da força de trabalho”, ela explicou. “Estou trabalhando de graça agora, mas vai valer a pena quando meus filhos tiverem idade suficiente para ir à escola e eu puder simplesmente sair e conseguir outro emprego de tempo integral.”
 \par 
Considere a situação de Leslie comparada à de Ilse, uma mulher composta com base em pesquisas sobre as experiências de uma típica mulher da Alemanha Oriental crescendo na década de 1980. Imediatamente após a Segunda Guerra Mundial, os alemães orientais mobilizaram as mulheres para a força de trabalho. O estado da Alemanha Oriental apoiava totalmente as mulheres no local de trabalho e, embora encorajasse o casamento, ser esposa não era considerada um precursor da maternidade. Como não havia homens suficientes para todos, o estado investiu pesadamente no apoio às mulheres solteiras.
 \par 
65
 \par 
66
 \par 
O QUE ESPERAR QUANDO VOCÊ ESTÁ ESPERANDO EXPLORAÇÃO: SOBRE A MATERNIDADE mães. Em particular, o governo da Alemanha Oriental idealizou a maternidade precoce e construiu moradias especiais "mãe e filho" nas universidades, onde as alunas podiam viver com seus bebês. Se Ilse fosse uma mulher média da Alemanha Oriental, ela teve seu primeiro filho aos {\color{blue}24} anos, provavelmente antes de se formar na faculdade, o que significava que ela evitou o declínio da fertilidade associado ao atraso na gravidez. O governo subsidiou fortemente moradia, roupas infantis, alimentos básicos e outras despesas associadas à criação dos filhos, além de fornecer a mulheres como Ilse acesso a cuidados infantis sempre que precisassem. Em 1989, os nascimentos fora do casamento representavam cerca de 34% de todos os nascimentos (em comparação com apenas 10% na Alemanha Ocidental), mas, ao contrário da maioria dos lugares no Ocidente capitalista, a maternidade solteira não levava à miséria. Um dos meus amigos búlgaros se formou em Leipzig na década de 1990. Ele se lembra de conhecer duas alunas por três anos antes de perceber que elas eram mães de crianças pequenas. Nada sobre a maternidade interferia na educação delas, porque seus bebês eram cuidados em creches do campus."”
 \par 
Em contraste, as mulheres na Alemanha Ocidental, assim como as mulheres nos Estados Unidos, voltaram para casa para serem donas de casa e mães dependentes após a Segunda Guerra Mundial, confinadas ao Kinder, Kiiche, Kirche (crianças, cozinha, igreja). Conforme observado anteriormente, a lei da Alemanha Ocidental exigia o consentimento do marido antes que uma mulher pudesse trabalhar fora de casa até 1957, e até 1977 a lei da família insistia que as mulheres casadas não deveriam deixar seus empregos interferirem em suas responsabilidades domésticas. Em um nível prático, os horários escolares e a falta de cuidados após a escola tornavam quase impossível para as mulheres da Alemanha Ocidental trabalhar em tempo integral. Mães casadas
 \par 
KRISTEN R. GHODSEE trabalhava principalmente em empregos de meio período, com uma diferença salarial de gênero maior do que a encontrada no Leste."
 \par 
Claro, nem todos os países socialistas apoiaram a independência econômica das mulheres na medida dos alemães orientais (que estavam presos em sua própria rivalidade da Guerra Fria com os alemães ocidentais). Os soviéticos relegaram o aborto em 1955, mas permaneceram decididamente pró-natalistas, e até mesmo a educação sexual mais básica estava ausente do discurso público. Romênia e Albânia eram terríveis em termos de liberdades reprodutivas das mulheres, com o estado forçando as mulheres a terem filhos restringindo o acesso ao controle de natalidade, educação sexual e aborto. Embora inicialmente legal na Romênia, o infame Decreto {\color{blue}770} de 1966 proibiu o aborto em um esforço para reverter o declínio populacional, e a lei foi fortalecida na década de 1980 para incluir exames ginecológicos obrigatórios para mulheres em idade reprodutiva. O estado romeno nacionalizou essencialmente os corpos das mulheres, e muitas mulheres buscaram abortos perigosos e ilegais, como dramatizado no brilhante filme de 2007 {\color{blue}4} Meses, {\color{blue}3} Semanas e {\color{blue}2} Dias.”
 \par 
A mensagem-chave aqui é que você não precisa ter um regime autoritário para implementar políticas que aliviem o conflito entre fertilidade e emprego. Hoje, quase todos os países do mundo têm alguma forma de licença-maternidade remunerada garantida para mulheres, e muitos estão instituindo licenças parentais com componentes obrigatórios de licença-paternidade. Na Islândia, o país com maior igualdade de gênero no mundo, conforme o Fórum Econômico Mundial, os pais têm noventa dias de licença, e 90% deles a tiram. O estado apoia ambos os pais a combinarem seu trabalho e
 \par 
67
 \par 
68
 \par 
O QUE ESPERAR QUANDO VOCÊ ESTÁ ESPERANDO EXPLORAÇÃO: SOBRE A MATERNIDADE responsabilidades familiares, proporcionando o caminho para a plena igualdade de gênero no lar e no local de trabalho.”
 \par 
Embora o socialismo de estado tivesse suas desvantagens, a mudança repentina na sorte das mulheres do Leste Europeu após 1989 demonstra amplamente como os mercados livres corroem rapidamente o potencial das mulheres para a autonomia econômica. Na Europa Central, por exemplo, os governos pós-1989 buscaram políticas conscientes de "reabilitação" para apoiar a transição do socialismo de estado para o capitalismo neoliberal. À medida que as empresas estatais fechavam ou eram vendidas a investidores privados, as taxas de desemprego disparavam. Muitos trabalhadores competiam por poucos empregos. Ao mesmo tempo, os novos estados democráticos reduziram seus gastos públicos ao refinanciar creches e jardins de infância. Estabelecimentos públicos de assistência à infância fecharam, e novas instalações privadas exigiram taxas substanciais. Alguns governos compensaram o fechamento de jardins de infância estendendo as licenças parentais por até quatro anos, mas com taxas muito mais baixas de compensação salarial e sem proteções de emprego.'®
 \par 
Essas políticas conspiraram para forçar as mulheres a voltarem para casa. Sem creches financiadas pelo estado ou licença-maternidade bem paga, e em um novo clima econômico em que os empregadores tinham um grande exército de desempregados para escolher, muitas mulheres foram expulsas do mercado de trabalho. De uma perspectiva macroeconômica, isso provou ser uma bênção para os estados em transição. As taxas de desemprego caíram (e, portanto, a necessidade de benefícios sociais), e as mulheres agora realizavam gratuitamente o trabalho de cuidado que o estado antes subsidiava para promover a igualdade de gênero. Mais tarde, quando cortes orçamentários mais profundos atingiram os aposentados e o sistema de saúde, as mulheres que já estavam em casa cuidando dos filhos agora podiam cuidar dos doentes e dos idosos — com grandes economias para o orçamento do estado.””
 \par 
KRISTEN R. GHODSEE
 \par 
Dado que muitas mulheres preferiam o emprego formal ao trabalho doméstico não remunerado, não deveria ser surpreendente que as taxas de natalidade pós-1989 tenham despencado. Embora as taxas de natalidade na Europa Oriental fossem maiores do que as da Europa Ocidental antes de 1989, elas começaram a cair assim que o processo de reabilitação começou. A instituição de mercados livres, na verdade dificultou em vez de ajudar a formação de novas famílias. Em nenhum lugar isso foi mais profundo do que na Alemanha Oriental, onde o desemprego disparado e o colapso do apoio ao cuidado infantil contribuíram para uma queda sem precedentes e descoordenada na fertilidade, o que a imprensa da Alemanha Ocidental chamou de "greve de nascimento". Em um período de cinco anos, a taxa de natalidade nos estados da Alemanha Oriental da Alemanha reunificada caiu 60%. Embora as taxas de fertilidade tenham saído dos buracos da década de 1990 em alguns países, as antigas nações socialistas estatais da Europa Oriental têm algumas das menores taxas de natalidade do mundo hoje. Em 2017, a Bulgária teve a população que mais encolheu no mundo, e dezesseis das vinte principais nações que enfrentam os declínios populacionais mais acentuados esperados até 2030 eram antigas nações socialistas estatais.” A ironia é que, enquanto as mulheres eram forçadas a voltar para casa na Alemanha Oriental, muitas mulheres da Alemanha Oriental se mudaram para o Ocidente em busca de empregos mais bem pagos, e essas mulheres trouxeram consigo um conjunto de expectativas que ajudaram as mulheres da Alemanha Ocidental a encontrar seu caminho no mercado de trabalho. Os jovens alemães orientais que inundaram a Alemanha Ocidental depois de 1989 eram filhos de mães trabalhadoras, e achavam absolutamente normal que as mulheres deixassem seus filhos em jardins de infância. Quando morei em Freiburg, conheci uma mulher da Alemanha Ocidental que atuou como diretora administrativa de uma conhecida editora acadêmica em Stuttgart.
 \par 
69
 \par 
70
 \par 
O QUE ESPERAR QUANDO VOCÊ ESTÁ ESPERANDO EXPLORAÇÃO: SOBRE A MATERNIDADE
 \par 
Nem todo mundo é fã de políticas de licença-maternidade remunerada impostas pelo governo, especialmente aquelas que não são aplicadas. Algumas feministas se opõem a essas políticas porque temem que elas prejudiquem as mulheres em mercados de trabalho competitivos. Os empregadores preferem contratar homens que não engravidam, como o chefe do meu amigo Jake. É por isso que algumas nações instituíram licenças-paternidade do tipo "pegue ou perca" para tentar igualar a expectativa das responsabilidades de cuidados de homens e mulheres. Até 2017, a Suécia exigia que novas mães e pais tirassem uma licença obrigatória de sessenta dias cada para se qualificarem para os generosos benefícios do estado. Os defensores do livre mercado argumentam que as empresas devem ser livres para definir suas próprias prioridades sem interferência do governo federal, mas a autorregulamentação corporativa tem tido uma taxa de sucesso bastante abismal. Em 2013, estima-se que apenas 12% dos trabalhadores americanos eram cobertos por políticas de licença-maternidade remunerada. E isso é completamente previsível em um cenário de livre mercado. Nenhuma empresa quer ser conhecida como aquela com políticas generosas de licença-maternidade porque teme que as mulheres com maior probabilidade de ter filhos migrem para ela em vez de suas concorrentes. Mas se a lei exigir que todas as empresas ofereçam a mesma licença protegida pelo emprego, e se o governo pagar parte da conta, como na licença-maternidade da Braun,
 \par 
KRISTEN R. GHODSEE plano de seguro, então muitos empregadores estariam dispostos a apoiar essas políticas. Isso significaria que eles poderiam contratar os candidatos mais promissores e investir em treiná-los com um alto grau de certeza de que eles colheriam os benefícios desse treinamento. Assim, a única maneira de garantir que todas as mulheres se beneficiem dessas políticas (não apenas as mulheres mais ricas e profissionais que trabalham em empresas já esclarecidas) é ter todo o peso do governo federal, estadual ou local por trás delas.”
 \par 
Esses mesmos empregadores poderiam contar com a continuidade dos trabalhadores após o parto se cuidados infantis de alta qualidade e preços razoáveis ​​estivessem prontamente acessíveis a todos os pais de crianças pequenas. Afinal, a funcionária estrela de Jake não saiu após ter seu bebê. Ela saiu, relutantemente, quando o peso de uma vida profissional inflexível e uma colcha de retalhos de arranjos complicados de cuidados infantis desabaram sobre sua cabeça exausta. A maior ajuda para as mulheres trabalhadoras seria a expansão de cuidados infantis de alta qualidade e financiados pelo governo federal, o que apoiaria a capacidade das mulheres de combinar a maternidade com o emprego remunerado. Os Estados Unidos chegaram perto de ter um sistema nacional de cuidados infantis: o Comprehensive Child Development Act aprovado por um voto bipartidário de democratas e republicanos em 1971. O ato teria financiado uma rede nacional de centros de cuidados infantis fornecendo serviços educacionais, médicos e nutricionais de alta qualidade, um primeiro passo crucial para o cuidado infantil universal. O presidente Richard Nixon vetou o ato e criticou as "implicações de enfraquecimento da família do sistema" que ele previa. No seu veto oficial, Nixon escreveu: “O Governo Federal, ao mergulhar de cabeça financeiramente no apoio ao desenvolvimento infantil, comprometeria a vasta autoridade moral do
 \par 
11
 \par 
72
 \par 
O QUE ESPERAR QUANDO VOCÊ ESTÁ ESPERANDO EXPLORAÇÃO: SOBRE A MATERNIDADE
 \par 
Governo Nacional para o lado de abordagens comunitárias para a criação de filhos em detrimento da abordagem centrada na família.” Essa abordagem “centrada na família” exigia o trabalho não remunerado das mulheres em casa, reforçando os papéis tradicionais de gênero de ganha-pão masculino e dona de casa feminina. Em essência, Nixon perguntou: “Por que o governo deveria pagar por algo que podemos fazer as mulheres fazerem de graça?”°
 \par 
Embora pesquisas mostrem que as crianças não são prejudicadas por creches de qualidade, e podem até mesmo desfrutar de maior desenvolvimento cognitivo, linguístico e socioemocional do que crianças cuidadas em casa, os conservadores americanos odeiam a ideia de creches porque ela também desafia a autoridade masculina na família. Um colaborador de um artigo de opinião da Fox News vê a creche universal como parte de uma trama maligna, argumentando que "governos totalitários fizeram grandes esforços para doutrinar crianças, e os maiores obstáculos que enfrentaram foram pais que contradiziam o que o governo estava dizendo a seus filhos". Nessa visão, tudo o que os países socialistas estatais fizeram para apoiar as mulheres — aumentando a participação na força de trabalho, liberalizando as leis de divórcio, criando jardins de infância e creches e apoiando a independência econômica das mulheres — tinha como objetivo fazer lavagem cerebral nas crianças. Até mesmo as escolas públicas serviam ao propósito principal de doutrinação."!
 \par 
Os direitos e prerrogativas das mulheres são, portanto, pintados como parte de um plano coordenado para promover o comunismo mundial, uma ameaça que se espalha pelo ocidente. Dessa perspectiva, até mesmo a Suécia socialista democrática “instituiu agressivamente um sistema muito custoso de creche” para “forçar as mulheres a saírem de casa e entrarem na força de trabalho”. Como se as mulheres suecas não escolhessem trabalhar por conta própria.
 \par 
Acordo de KRISTEN R. GHODSEE. Por trás do medo da doutrinação das crianças pelo governo está um medo real da independência econômica das mulheres e do colapso da família tradicional.”
 \par 
Por enquanto, ainda são as mulheres que devem gestar e dar à luz os bebês de fato (pelo menos até que os cientistas desenvolvam a patogênese), mas os pais podem estar tão envolvidos no cuidado das crianças quanto as mães. O número de pais que ficam em casa está crescendo, e pode ser que um dia os empregadores vejam os funcionários homens como potenciais cuidadores da mesma forma que veem as mulheres agora. Mas até lá, os mercados de trabalho competitivos continuarão a penalizar as mulheres por sua biologia. O alto custo do cuidado infantil privado — combinado com a disparidade salarial de gênero e as expectativas sociais de que crianças pequenas precisam mais das mães do que dos pais — significa que são esmagadoramente ainda as mulheres que interrompem suas vidas profissionais para ficar em casa com crianças pequenas. Nos Estados Unidos, esses anos fora da força de trabalho prejudicam as mães de várias maneiras: perda de renda, preterição de promoções, menos dinheiro para a previdência social ou aposentadoria e maior dependência econômica dos homens. Claro que algumas mulheres querem ficar em casa, e isso deve continuar sendo uma escolha, desde que ficar em casa para fazer o trabalho de cuidado não implique dependência financeira. Nossa meta deve ser que um número igual de homens e mulheres escolham atuar como pais que ficam em casa. Embora essa opção deva ser aberta a todos, espero que a maioria dos homens e mulheres não a aceite. Com licenças parentais razoáveis ​​e creches acessíveis e de alta qualidade o suficiente para todos, realmente podemos ter nosso bolo e comê-lo também.
 \par 
Um dos problemas mais óbvios de muitos países socialistas de Estado era que, embora os cidadãos tivessem emprego garantido pelo Estado, eram frequentemente forçados a trabalhar
 \par 
73
 \par 
7h
 \par 
O QUE ESPERAR QUANDO VOCÊ ESTÁ ESPERANDO EXPLORAÇÃO: EM MATERNIDADE empregos que elas não gostavam. Muitos empregos de rotina eram monótonos e insatisfatórios (não muito diferentes dos empregos de rotina no Ocidente). Mas muitas mulheres americanas que querem trabalhar são forçadas a ficar em casa devido à escassez de creches de qualidade, do alto custo quando estão disponíveis e da falta de flexibilidade no mercado de trabalho. Outras mulheres precisam trabalhar para sobreviver, principalmente porque o seguro saúde privado nos Estados Unidos vincula as funcionárias aos seus locais de trabalho se elas não quiserem perder os benefícios. Nem todas as mulheres têm a opção de um homem que possa sustentá-la, e mesmo aquelas que têm seriam sensatas em não depender muito dessa opção. As mulheres não devem ser compelidas a relacionamentos românticos porque é sua única chance de ter um teto sobre suas cabeças. Nosso sistema também coloca um fardo enorme sobre os homens, já que aqueles que não podem sustentar suas esposas são rejeitados como parceiros românticos (algo que já está acontecendo nos Estados Unidos, onde as taxas de casamento entre os pobres estão em um nível mais baixo de todos os tempos).
 \par 
No final das contas, as diferenças na biologia reprodutiva impossibilitam tratar homens e mulheres como iguais nos mercados de trabalho, onde os empregadores se esforçam para contratar aqueles que eles acham que serão seus trabalhadores mais valiosos. Este é um problema complicado que carece de soluções simples, mas políticas como licenças parentais e creches universais financiadas pelo estado ajudam a aliviar as causas básicas da discriminação de gênero. Essas políticas começaram como propostas socialistas e tinham o objetivo explícito de igualdade de gênero no trabalho e em casa. Ao longo do último século, essas políticas começaram a abrir caminho na legislação de quase todos os países ao redor do mundo. Em 2016, os Estados Unidos se juntaram à Nova Guiné, Suriname e alguns
 \par 
KRISTEN R. GHODSEE ilhas no Pacífico Sul são os únicos países do mundo sem uma lei nacional sobre licença parental remunerada.
 \par 
Quando penso na mulher que deixou a empresa de Jake para ficar em casa com seu bebê e na minha ex-colega Leslie, que trabalhava por setenta centavos por mês, lamento que a maternidade — que deveria ser uma fonte de alegria — tenha se transformado em um fardo esmagador para tantas mulheres. Em nenhum lugar do mundo desenvolvido é mais difícil para pessoas comuns começarem suas famílias. Certamente os países mais ricos do planeta podem fazer melhor.
 \par 
73
 \par 
\begin{figure}
	\centering
	\includegraphics[width=1.\textwidth]{temp\_files/images/UP\_logo.png }
	\caption{Flora Tristan (1803-1844): Uma teórica e ativista socialista utópica francesa que argumentou que a libertação das classes trabalhadoras não poderia ser alcançada sem a concomitante emancipação das mulheres. O seu ensaio de 1843, O Sindicato dos Trabalhadores, é um texto feminista socialista fundamental no qual Tristan imaginou um grande colectivo laboral no qual os trabalhadores (tanto homens como mulheres) reuniriam os seus recursos para fornecer serviços sociais em seu próprio benefício. Cortesia da TASS.}
	\label{ }
\end{figure}