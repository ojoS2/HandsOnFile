\chapter{13 Teoria da Renda Agrícola de Marx}\label{13 Teoria da Renda Agrícola de Marx}
 \par 
A teoria da renda agrícola de Marx contém dois componentes importantes e intimamente ligados: uma teoria da renda diferencial e uma teoria da renda absoluta. Para Marx, a propriedade privada da terra funciona como um obstáculo à acumulação de capital, porque os proprietários de terras capturam parte da mais-valia produzida na economia. Até certo ponto, o mesmo se aplica à teoria ortodoxa da renda, seja ela ricardiana ou neoclássica (embora Ricardo tenha tentado distinguir entre renda e lucro, enquanto a teoria neoclássica funde estas categorias, como é mostrado abaixo).
 \par 
Na teoria neoclássica, os produtores agrícolas pagam aluguel por causa de uma combinação de propriedade privada e restrições naturais ou técnicas - por exemplo, uma escassez de terra, seja na oferta geral ou na oferta de terra de melhor qualidade ou localização. Em contas mais sofisticadas, a demanda pelos diferentes produtos da terra também pode ser levada em conta. Em ambos os casos, o aluguel serve em parte para alocar recursos "eficientemente" entre diferentes terras, levando a taxas iguais de retorno em toda a economia. A visão dominante implica que, primeiro, a propriedade da terra apenas determina quem deve receber o aluguel, não seu nível. Segundo, o nível de aluguel é determinado pelas condições técnicas de produção (e demanda). Essas implicações podem ser usadas para destacar as características distintivas da abordagem de Marx.
 \par 
O ponto de partida de Marx são as condições sociais sob as quais parte da mais-valia é apropriada pelos proprietários de terras como renda.
 \par 
Por outras palavras, a teoria da renda deriva da relação entre a produção capitalista e a propriedade fundiária, e estas são historicamente específicas, e não tecnicamente dadas. Consequentemente, não pode haver uma teoria geral da renda, e as conclusões alcançadas num caso não podem ser automaticamente aplicadas a outros. Segue-se que a renda não pode ser analisada com base num efeito geral, por exemplo, de entrave à produção capitalista. Caso contrário, qualquer obstáculo ao investimento capitalista poderia ser descrito como renda, que é a essência da noção marshalliana de quase-rendas no curto prazo, quando um capitalista lucra temporariamente com um método de produção superior. Neste caso, o acesso privilegiado ao financiamento, aos mercados ou aos favores burocráticos e uma série de outras condições também poderiam ser abrangidos pela teoria da renda, como na teoria neoclássica da “busca de renda”, eliminando o âmbito para uma teoria específica do papel social. da propriedade fundiária. Em suma, a renda deve ser examinada em conjunto com as condições históricas em que existe, particularmente porque o capitalismo tende a eliminar as barreiras ao seu imperativo de acumulação. Por que e como é que a propriedade da terra limita a acumulação de capital ao longo do tempo e extrai uma parte da mais-valia bombeada pelo capital industrial?
 \par 
Este é o capítulo mais exigente deste livro. Está incluído aqui por duas razões: primeiro, porque ilustra uma aplicação importante do método de Marx e confronta uma questão que alegadamente contradiz a sua teoria do valor; em segundo lugar, devido à relevância contínua da renda para questões tão diversas como o petróleo, a mineração, o desenvolvimento agrícola, a regeneração urbana e a habitação.
 \par 
\section{Aluguel Diferencial 1}
 \par 
A teoria de Marx sobre renda diferencial (RD) pode ser entendida apenas examinando como a propriedade fundiária intervém na operação do capital dentro da agricultura. Como é que a competição deixa o valor excedente para ser apropriado na forma de renda, e quais são as implicações disso? Para confrontar esse problema, uma ligeira digressão é necessária para examinar como os capitais competem entre si dentro de um setor na ausência do efeito distorcido da propriedade fundiária.
 \par 
Foi demonstrado nos Capítulos {\color{blue}6} e {\color{blue}8} que os capitais dentro do mesmo sector competem entre si principalmente através do aumento da produtividade através de aumentos na composição orgânica do capital (OCC). Isto não ocorre de forma uniforme em todo o sector, pelo que tenderá a haver diferenças significativas de produtividade entre estas capitais. Marx argumenta que os valores das mercadorias são formados a partir dessas diferentes produtividades individuais. Significativamente, ele não insiste que os valores devem ser iguais ao tempo médio de trabalho para o sector (mesmo assumindo que os trabalhadores são idênticos em toda a economia). Por exemplo, se a técnica mais favorável ou a menos favorável tiver peso suficiente em comparação com a média, esta técnica, e não a média aritmética, regula o valor de mercado do sector. Em qualquer dos casos, os lucros excedentários ou excedentários reverterão para os capitais que produzem mais valor do que a média sectorial.
 \par 
A explicação de Marx sobre a renda diferencial começa por dividi-la em dois tipos, a renda diferencial um (denotada por DR1) e a renda diferencial dois (DR2, abordada na próxima secção). O DR1 centra-se na existência de lucros excedentários na agricultura resultantes exclusivamente de diferenças de fertilidade (ignorando os custos de transporte e outros custos de comercialização por conveniência). Isso geralmente está associado à extensa margem de Ricardo. Em resumo, o capital não pode fluir uniformemente para terras de fertilidade igual, uma vez que tais terras não estão naturalmente disponíveis. Os capitais que fluem para as terras melhores enfrentam a barreira da propriedade fundiária e são forçados pelos proprietários a renunciar a parte do seu lucro excedente sob a forma de renda.
 \par 
O resultado não é simplesmente a criação de renda, mas também uma distorção na formação do valor de mercado na agricultura. Na indústria, os piores métodos de produção predominam apenas onde são excepcionalmente pesados, e os capitais que empregam métodos mais produtivos capturam lucros excedentes. Em contraste, na agricultura os piores métodos podem predominar devido à propriedade fundiária, e os capitais investidos em terras melhores podem ter de entregar os seus lucros excedentes aos proprietários de terras sob a forma de DR1. Para Ricardo, isto acontecerá independentemente da propriedade da terra (que, para ele, apenas determina quem recebe as rendas determinadas pela fertilidade). Em contraste, para Marx, a renda depende sempre da capacidade dos proprietários de terras de se apropriarem do excedente diferencial associado a terras de qualidade distinta.
 \par 
A existência de diferenças de rentabilidade na agricultura é uma condição necessária mas insuficiente para a existência do DR1. Esses lucros excedentes também devem ser permanentes e apropriados por proprietários suficientemente poderosos, caso contrário (como nas quase-rendas de Marshall) o DR1 não só existiria em todos os setores da economia, mas também seria corroído como os lucros excedentes na indústria (que tendem a a serem eliminados pela concorrência devido aos movimentos de capitais e à difusão de inovações tecnológicas dentro de cada sector).
 \par 
Deve-se notar que diferentes condições naturais como tais não são a fonte da DR1. Podem contribuir para diferenças de produtividade, mas não criam nem as categorias de lucro excedente nem de renda diferencial. Em vez disso, a DR1 depende da utilização das condições naturais (e das diferenças de produtividade) nas relações de produção capitalistas, bem como da intervenção da propriedade fundiária. Por outras palavras, a renda existe não porque existam lucros excedentários, mas porque eles são apropriados pelo proprietário da terra e não pelo capitalista.
 \par 
\section{Aluguel Diferencial 1}
 \par 
A teoria da DR1 de Marx é construída com base em aplicações iguais de capital a diferentes terras, caso em que os lucros excedentários (e a renda) surgem das diferenças de fertilidade mais ou menos permanentes entre essas terras. A renda diferencial do segundo tipo (DR2) também se preocupa com a concorrência no sector agrícola. Contudo, a DR2 deve-se à apropriação de lucros excedentários criados por diferenças temporárias de produtividade decorrentes da aplicação de capitais desiguais a terras com fertilidade igual. Neste caso, os proprietários beneficiam do progresso da sociedade na introdução de inovações técnicas e na organização da produção em grande escala, permitindo-lhes apropriar-se de uma parte do excedente acrescentado.
 \par 
É evidente, porém, que a totalidade dos lucros excedentários produzidos na agricultura e que constituem a base potencial da DR2 podem não reverter para os proprietários de terras; além disso, estes lucros excedentários tendem a diminuir à medida que investimentos de capital inicialmente anormais se tornam normais em todo o sector. Independentemente destas limitações, a DR2 reduz o incentivo aos agricultores capitalistas para investirem intensivamente (mais capital e melhor tecnologia na mesma terra) em vez de extensivamente (mesma tecnologia em mais terras), o que prejudica o desenvolvimento tecnológico da agricultura. É por isso que Marx argumenta que a agricultura tende a apresentar um ritmo de progresso técnico mais lento do que a indústria. Esta é uma das conclusões mais importantes a retirar da teoria da RD2 de Marx: a sua preocupação dinâmica com os obstáculos ao desenvolvimento da acumulação de capital, em vez da formulação estática da distribuição da mais-valia sob a forma de renda.
 \par 
Se DR1 e DR2 fossem independentes um do outro, a análise de DR, como a simples adição de DR1 e DR2, estaria agora completa. Pois então o DR1 teria o efeito de equalizar os lucros entre terras de qualidade diferente para aplicação de quantidades iguais de capital, de modo que o DR2 pudesse ser calculado a partir das diferenças de rentabilidade decorrentes da aplicação de capitais desiguais. Alternativamente, o DR2 equalizaria os efeitos de diferentes aplicações de capital, de modo que o DR1 pudesse ser calculado a partir das diferentes festividades entre as terras. Este procedimento é, no entanto, inválido. Na verdade, no Volume {\color{blue}3} de O Capital, Marx nunca examina a DR2 na forma pura de aplicações desiguais de capital a terras iguais. Ele sempre discute DR2 na presença de DR1 - ou seja, de terras de qualidade desigual. A razão de Marx para o fazer é analisar a determinação quantitativa do DR2, tendo estabelecido a base qualitativa para a sua existência.
 \par 
Neste capítulo, DR1 e DR2 foram determinados com base em certas abstrações sobre a distribuição de capitais e diferentes festividades de terra. Isso foi feito para clareza expositiva; mas uma análise mais complexa está necessariamente envolvida sobre a coexistência de terras desiguais e capitais desiguais nessas terras, bem como questões de qualidade diferencial e localização de produção e venda, que podem mudar ao longo do tempo. Por exemplo, para DR1 há o problema de determinar qual é a pior terra na presença de aplicações desiguais de capital (DR2), uma vez que algumas terras podem ser piores para um nível ou tipo de investimento (tratores, digamos), mas não para outros (fertilizantes). Para DR2, há o problema de determinar o nível normal de investimento na presença de terras diferentes (DR1). Alguns capitais podem ser normais para alguns tipos de terras (aquelas que exigem a construção de canais de irrigação, digamos), outros capitais normais para outras terras (aquelas localizadas em encostas de montanhas).
 \par 
Existe uma dificuldade adicional para o DR2, uma vez que a produtividade decrescente de investimentos adicionais não permitiria lucros excedentários para capitais anormalmente grandes, a menos que o valor de mercado do produto agrícola aumentasse. Isto levanta se o valor de mercado deve ser determinado pela produtividade individual de alguma parcela de terra, ou se pode ser determinado por alguma parte do capital investido nessa terra. Por outras palavras, a dimensão do “capital normal” é sempre o capital total aplicado a alguma terra, ou pode ser alguma parte desse capital? Mesmo o termo “capital normal” pode ser inadequado, pois o investimento de capital num determinado terreno é sempre específico e não geral.
 \par 
Esses problemas dizem respeito à determinação simultânea da pior terra e do capital normal na agricultura, cada um dos quais influencia a formação de valor na presença de propriedade fundiária. A interação dos dois dá origem ao valor de mercado do produto agrícola, a partir do qual rendas diferenciais podem ser calculadas. Esse problema não surge para o capital industrial, porque a determinação do capital normal é sinônimo da determinação do valor. Foi mostrado acima que o mesmo é verdadeiro para cada um de DR1 e DR2 na ausência do outro. Para DR1 em sua forma pura (capitais iguais), a determinação da pior terra é sinônimo da determinação do valor, enquanto para DR2 em sua forma pura (terras iguais), a determinação do capital normal impulsiona a determinação do valor.
 \par 
Este problema da determinação conjunta do capital normal e da terra normal não pode ser resolvido abstratamente; correspondentemente, DR1 e DR2 não podem ser determinados puramente teoricamente. Tal como discutido anteriormente, dependem de condições historicamente contingentes: de como a agricultura se desenvolveu no passado e como se relaciona com a acumulação de capital em termos de acesso dos capitalistas à terra, que pode ser afectada por condições legais, financeiras e outras. Além disso, as mudanças nas culturas e nas tecnologias de produção modificam tanto a procura de terras como as definições das melhores e piores terras. Em suma, a teoria da RD não conduz a uma análise determinada da renda, mas revela alguns dos processos pelos quais esta pode ser examinada concretamente.
 \par 
\section{Aluguel Diferencial 1}
 \par 
Se a chave para a formação da renda diferencial é a determinação do valor e a presença de lucros excedentes na agricultura, a base para a formação da renda absoluta (AR) é a transformação dos valores de mercado em preços de produção (ver Capítulo {\color{blue}10}). Nesse sentido, AR se afasta de DR. Ambas as formas de renda dizem respeito ao obstáculo ao investimento de capital colocado pela propriedade fundiária, e ambas dão origem à apropriação de lucro excedente sob a forma de renda. No entanto, a RD e a AR estão localizadas em diferentes níveis de complexidade e as suas fontes são correspondentemente diferentes: a RD deriva das diferenças de produtividade dentro da agricultura, enquanto a AR deriva das diferentes taxas de variação do aumento da produtividade entre a agricultura e outros sectores da economia como um resultado da barreira à acumulação que é colocada pela propriedade fundiária.
 \par 
Em termos formais, a teoria da AR de Marx é a seguinte: devido às barreiras impostas pela propriedade fundiária, explicadas na análise da DR2, a agricultura tende a ter um OCC inferior ao da indústria. Portanto, há uma maior proporção de mão-de-obra viva empregada na agricultura e este sector produz mais-valia adicional. Na ausência de renda, o seu preço de produção estaria abaixo do valor. Este é, no entanto, um relato totalmente estático. Em termos dinâmicos (os detalhes algébricos são retomados abaixo), a formação dos preços de produção depende da concorrência e da possibilidade de fluxos de capitais entre setores. Contudo, os fluxos para a agricultura e a formação dos preços de produção neste sector são obstruídos pela propriedade fundiária (não se pode simplesmente investir no sector, é preciso pagar renda como condição de acesso à terra). Devido a este obstáculo, os proprietários de terras podem cobrar uma AR pelos fluxos de capital para novas terras (em alternativa, podem cobrar DR2 pelos fluxos para terras já em uso que subsequentemente se tornam mais intensivas em capital). Esta renda pode aumentar o preço dos produtos agrícolas acima do seu preço de produção. No limite, essas mercadorias podem ser vendidas pelo seu valor, sendo a diferença entre o seu preço de venda e o preço de produção capturada como AR. Nestas circunstâncias, a AR desapareceria na conjunção de duas condições: (a) se o ritmo de desenvolvimento da agricultura fosse igual ao da indústria, e o OCC da agricultura fosse igual (ou superior) à média social; e (b) se todas as terras tivessem sido utilizadas para cultivo, uma vez que a AR depende de movimentos de capitais para novas terras.
 \par 
Na literatura, encontra-se frequentemente uma interpretação diferente da teoria da AR de Marx, na qual os proprietários de terras obtêm uma renda porque podem impedir o fluxo de capital para a agricultura. No entanto, isto é simplesmente AR como uma renda de monopólio. Considerações semelhantes seriam aplicáveis ​​na ausência de propriedade fundiária - por exemplo, se houvesse uma patente essencial envolvida no processo de produção. Este paralelo é insuficiente por duas razões. Primeiro, o argumento baseia-se numa teoria estática da distribuição de mais-valia. Em segundo lugar, nesta interpretação as condições de Marx para a existência da AR tornam-se arbitrárias, uma vez que os OCC diferem entre sectores industriais sem que a AR seja formada. Além disso, mesmo na agricultura não haveria razão para que a AR se limitasse à diferença entre o valor e o preço de produção. Se a AR fosse uma renda monopolista, o preço de mercado dos produtos agrícolas poderia subir acima do seu valor de acordo com a capacidade dos proprietários de terras para impor tais preços.
 \par 
No entanto, a discussão de Marx sobre as condições sob as quais a AR desapareceria sugere que não está envolvida uma teoria estática. O que importa, como foi explicado acima, é o ritmo de desenvolvimento da agricultura em relação à indústria, e o movimento potencial de capital para novas terras durante o processo de acumulação. É claro que estas condições podem ser interpretadas estaticamente (por exemplo, assumindo que todas as terras estão arrendadas e que todos os sectores têm níveis iguais de desenvolvimento); mas, pelo contrário, os outros conceitos utilizados, em particular o OCC, devem ser interpretados na dinâmica da teoria da acumulação de Marx. Ao empreender esta tarefa, será demonstrado abaixo que a teoria da AR de Marx é totalmente consistente com a sua análise da acumulação de capital.
 \par 
Suponha inicialmente que o OCC em toda a economia seja dado por c ⁄ v, e que possa ser aumentado em qualquer setor (incluindo a agricultura) por um fator b > 1, de modo que uma determinada quantidade de trabalho converteria bc capital constante em bens finais , em vez de c. Para a agricultura, antes deste aumento no OCC, a diferença entre o valor e o preço de produção é onde r é a taxa de lucro. Com as mudanças técnicas em toda a economia, exceto na agricultura, a taxa geral de lucro, r, muda de s ⁄ (c + v) para s ⁄ (bc + v). Na agricultura, na medida em que o cultivo intensivo é obstruído, c continua a ser o
 \par 
Quantidade de valor trabalhada por v em vez de aumentar para bc como para outros setores. Portanto, a diferença entre valor e preço de produção na agricultura se torna, a partir da expressão acima para d e a nova taxa de lucro, r:
 \par 
\section{Aluguel Diferencial 1}
 \par 
\section{Aluguel Diferencial 1}
 \par 
Esta diferença, d, é igual à taxa de lucro, r, multiplicada pelo capital constante adicional posto em movimento ou, alternativamente, igual aos lucros excedentes decorrentes do OCC mais elevado. Estes lucros excedentários poderiam ser capturados como DR2 se o OCC tivesse aumentado nas terras actualmente em uso, com o excedente a reverter para os proprietários em vez de ser capturado pelos capitalistas como noutros sectores.
 \par 
Em suma, a AR está limitada pelo encargo máximo para o cultivo extensivo em novas terras, conforme permitido pela possibilidade alternativa de investimento no cultivo intensivo. Isto corresponde à diferença entre valor e preço da produção na agricultura. Por outras palavras, a escolha é entre investir intensamente nas terras existentes, mas desistir de alguns, possivelmente todos, dos lucros excedentes aos proprietários; ou investir em novas terras e enfrentar um encargo da mesma magnitude potencial. O ponto importante não é tanto que o preço da produção tenda a exceder o seu valor na agricultura (ou, mais genericamente, quando há terra envolvida); em vez disso, a presença de propriedade fundiária pode impedir a acumulação de capital (e certamente influencia a sua natureza), com a formação potencial de AR como consequência, ela própria limitada aos lucros extras que poderiam ser obtidos se o capital fosse investido intensivamente em terras existentes em uso. .
 \par 
Foi demonstrado, então, que a teoria da renda de Marx estende a sua teoria da acumulação de capital para examinar a barreira da propriedade fundiária. Para ele, a renda é a forma económica das relações de classe na agricultura e só pode ser compreendida examinando a relação entre capital e terra. A própria renda depende da produção e apropriação de mais-valia através da intervenção da propriedade fundiária. A RD deriva de lucros excedentes formados através da concorrência na agricultura. O DR1 resulta de diferenças de produtividade devido a condições “naturais”, levando a capitais iguais que obtêm diferentes taxas de lucro na agricultura. A DR2 deve-se aos diferentes retornos das aplicações desiguais de capital (capitais de diferentes dimensões) na agricultura. Na indústria, os lucros excedentes revertem para o capital mais produtivo. Em contraste, na agricultura podem ser apropriados como renda. Finalmente, a AR surge da diferença entre o valor e o preço da produção na agricultura, devido ao seu OCC inferior à média, caso a propriedade da terra obstrua a acumulação. Onde os capitalistas possuem a sua própria terra ou onde são encorajados ou mesmo facilitados a acumular pelos proprietários, tais obstáculos podem não prevalecer. De forma ainda mais geral, sempre que surge um excedente na presença de propriedade fundiária (seja devido a uma melhor fertilidade ou ao cultivo intensivo, melhor localização, ou por qualquer outra razão), proporciona o potencial de renda que pode ser apropriado pelos proprietários ou outros agentes. Esta não é apenas uma questão distributiva, mas tem o efeito de potencialmente obstruir o ritmo e as formas de acumulação, ou mesmo acelerá-la, caso seja o capitalista quem consiga apropriar-se desse excedente como proprietário de terras.
 \par 
A teoria da renda de Marx baseia-se nas suas teorias de produção, acumulação, formação de valor e na teoria dos preços de produção. Como tal, é provavelmente a aplicação mais complexa da sua compreensão da economia capitalista. Ao mesmo tempo, revela os seus próprios limites ao mostrar como uma análise mais aprofundada depende da forma como a propriedade fundiária se desenvolveu e interage com o desenvolvimento capitalista.
 \par 
\section{Aluguel Diferencial 1}
 \par 
Os aspectos mais controversos da teoria da renda de Marx são como ele difere de Ricardo na compreensão da renda diferencial, se a renda absoluta é renda monopolista e se o OCC mais baixo na agricultura é arbitrário (juntamente com se AR é limitado à diferença entre valor e preço). A importância da teoria de Marx reside menos no facto de fornecer uma teoria quantitativa da renda e do preço e mais no facto de chamar a atenção para as formas historicamente específicas pelas quais a propriedade fundiária influencia o ritmo, o ritmo e a direcção da acumulação de capital - quer no contexto da agricultura, quer no contexto da agricultura. , petróleo ou “regeneração urbana”.
 \par 
A teoria da renda de Marx é desenvolvida especialmente em Marx (1969, capítulos 1-14, 1981a, pt.{\color{blue}6}). Este capítulo baseia-se em Ben Fine (1982, capítulos 4, 7, 1986, 1990b). Para abordagens semelhantes, ver Cyrus Bina (1989), David Harvey (1999, cap.{\color{blue}11}) e Isaak I. Rubin (1979, cap.{\color{blue}29}); ver também o debate em Science & Society (70(3), 2006). Na sua tese de doutoramento, Mary Robertson (2014) associou a noção de renda monopolista (urbana) à noção mais antiga e amplamente utilizada de ganhos de desenvolvimento, não confinada ao marxismo como tal. Na teoria das rendas diferenciais e absolutas de Marx, tais rendas derivam da produtividade associada a terras individuais como condição de acesso dos capitalistas a essas terras. Nisto, Marx abstrai da “produtividade” mais geral, embora desigual, que se acumula nas terras à medida que a acumulação prossegue - pense, por exemplo, nos benefícios de aglomeração do desenvolvimento urbano ou na chegada de uma estação ferroviária. Estas podem ser interpretadas como rendas de monopólio e são importantes, por exemplo, no contexto da financiarização, uma vez que as rendas surgem dos preços inflacionados da habitação num boom especulativo. De um modo mais geral, essas rendas monopolistas são percebidas como a consequência da apropriação de valor, e não apenas de mais-valia, à medida que as condições de reprodução económica e social, e não apenas de produção, evoluem e são contestadas.