\chapter{Os ex-presidiários}\label{Os ex-presidiários}
 \par 
Na primavera de 2000, Alex Star, editor do agora extinto Lingua Franca, contratou-me para escrever um perfil de John Gray e Edward Luttwak, dois intelectuais conservadores que se tinham deslocado para a esquerda. Durante todo o verão e outono, entrevistei Gray e Luttwak, bem como outros conservadores como William F. Buckley, Irving Kristol e Norman Podhoretz. Foi um momento difícil para a direita. Bill Clinton ainda era presidente; O {\color{blue}11} de setembro ainda não havia ocorrido. A prosperidade era um dado adquirido, a guerra era uma memória distante e as pessoas instruídas ainda falavam do fim da história. O momento teve uma sensação muito diferente da de hoje e afetou a forma como os conservadores pensavam sobre as suas ideias e a sua política. Embora algumas das referências e declarações neste artigo estejam agora desatualizadas, e algumas de suas afirmações eu não acredite mais, decidi não revisar a peça para preservar o clima daquele momento. No capítulo 8, revisito algumas das questões aqui discutidas à luz do {\color{blue}11} de Setembro, da guerra ao terrorismo e da Guerra do Iraque.
 \par 
Há outra razão pela qual não revisei este artigo. Embora eu tivesse lido Burke, Oakeshott e Nozick na faculdade e na pós-graduação, pesquisar e escrever este artigo foi meu primeiro encontro sustentado com a visão de mundo da direita. (Continua sendo uma triste realidade do ensino superior americano que os cientistas sociais e os historiadores possam superar
 \par 
Este capítulo apareceu originalmente como um artigo na Lingua Franca (fevereiro de 2001): 24–33.
 \par 
Seu treinamento apenas com um conhecimento passageiro do conservadorismo.) Este artigo tornou-se para mim uma espécie de educação sentimental, minha introdução à agonia e ao êxtase da mente conservadora. Embora eu certamente revisasse grande parte dele hoje – particularmente a premissa subjacente de que os conservadores que discuto aqui são diferentes da corrente dominante – o artigo, no entanto, fornece ao leitor um vislumbre do que primeiro me interessou na direita e como cheguei a escreva este livro.
 \par 
De acordo com o mito popular, foi Winston Churchill quem disse: “Qualquer homem com menos de trinta anos que não seja liberal não tem coração, e qualquer homem com mais de trinta anos que não seja conservador não tem cérebro”. Ele não o disse, mas o seu imprimatur transformou uma piada inteligente de proveniência incerta num axioma da biografia política: o radicalismo é um privilégio da juventude, o conservadorismo uma responsabilidade da idade, e cada pessoa pensante acaba por renunciar ao primeiro pelo segundo. De Max Eastman a Eugene Genovese, de Whittaker Chambers a Ronald Radosh, os intelectuais migram da esquerda para a direita como se obedecessem a uma lei da natureza.
 \par 
Ou não? Afinal, John Stuart Mill publicou The Subjection of Women quando tinha sessenta e três anos. Nos últimos dez anos da sua vida, Diderot saudou a Revolução Americana e criticou a França como a reencarnação da Roma imperial. E quando George Bernard Shaw abordou a questão da política e do envelhecimento, sugeriu exactamente o oposto do que Churchill supostamente teria dito. “As pessoas mais ilustres”, escreveu Shaw em 1903, “tornam-se mais revolucionárias à medida que envelhecem”.{\color{blue}1}
 \par 
Desde o fim da Guerra Fria, vários conservadores proeminentes seguiram a prescrição de Shaw e viraram à esquerda. Michael Lind, que já foi editor-chefe do The National Interest, de Irving Kristol, denunciou seus anteriores aliados por promoverem uma “guerra de classes contra os americanos assalariados”. As suas teorias orientadas para o mercado, escreve ele, são “pouco convincentes” e as suas políticas económicas são “terríveis”. Arianna
 \par 
Huffi ngton, outrora confederado de Newt Gingrich, agora investe contra os Estados Unidos onde a grande maioria está “sufocada com a poeira dos touros galopantes de Wall Street”. {\color{blue}2} Glenn Loury, economista e antigo querido neoconservador, ostenta o emblema característico da pertença à esquerda: tornou-se um dos ex-amigos de Norman Podhoretz. Mas os expatriados mais extravagantes de hoje são um inglês, John Gray, e um judeu emigrado da Transilvânia, Edward Luttwak.
 \par 
Na década de 1970, John Gray era uma estrela em ascensão da Nova Direita Britânica. Filósofo político formado em Oxford, ele escreveu poemas em prosa para o mercado livre, cruzou o Atlântico para se alimentar do libertarianismo de alta octanagem dos think tanks de direita norte-americanos e, diz um amigo de longa data, encantou seus camaradas até altas horas da noite. com visões da vindoura Utopia “anarco-capitalista”. Mas depois da queda do Muro de Berlim, Gray desertou. Primeiro, criticou o triunfalismo da Guerra Fria da tese do “fim da história” de Francis Fukuyama e aconselhou contra o desmantelamento do Serviço Nacional de Saúde britânico. E então, em 1998, a partir da sua recém-criada posição como professor de pensamento europeu na London School of Economics (LSE), proferiu False Dawn, uma denúncia feroz da globalização económica. Atacando as “tropas de choque do mercado livre”, Gray advertiu que o capitalismo global poderia “vir a rivalizar” com a antiga União Soviética “no sofrimento que inflige”. {\color{blue}3} Agora ele é um colaborador regular do The Guardian e do New Statesman, os principais veículos de esquerda da Grã-Bretanha. Sua conversão é tão profunda que ninguém menos que Margaret Thatcher teria se perguntado: “O que aconteceu com John Gray? Ele costumava ser um de nós.{\color{blue}4}
 \par 
E quanto a Edward Luttwak? Antigamente, ele foi um dos intelectuais da corte de Ronald Reagan, um falcão militar brilhante que criticou impiedosamente as políticas de defesa liberais e forneceu a fundamentação filosófica para a escalada militar americana na década de 1980.
 \par 
Os críticos liberais chamavam-no de “Eddie louco”, mas representando uma figura que era parte Dr. Strangelove e parte Dr. Jivago, Luttwak rejeitou sem esforço os seus argumentos, pressionando a Guerra Fria para a sua conclusão. {\color{blue}5} Hoje ele está desiludido com a vitória. Ele considera os Estados Unidos um pesadelo capitalista, “um aviso sombrio” aos líderes que procuram libertar forças de mercado livre nos seus próprios países. Utilizando o mesmo humor ácido que outrora usou contra os tempos de paz liberais, ele zomba das “pretensões napoleónicas” dos líderes empresariais americanos, desafia a sabedoria convencional de que o capitalismo e a democracia são companheiros inevitáveis ​​(“mercados livres e sociedades menos livres andam de mãos dadas”), e condena as desigualdades selvagens produzidas pelo “turbocapitalismo”. Ele critica os europeus de centro-esquerda, como Tony Blair, por abandonarem as suas raízes socialistas e pela sua relutância em “arriscar qualquer acção inovadora” em nome dos “trabalhadores comuns”. Com o seu “desdém pelos pobres e outros perdedores” e “desprezo pelas amplas massas da classe trabalhadora”, escreve Luttwak, os Novos Democratas Clintonistas e os Europeus da Terceira Via “só podem produzir políticas de direita”.{\color{blue}6}
 \par 
Em suas encarnações originais, Gray e Luttwak se emocionaram com duas das paixões galvanizadoras do conservadorismo — o anticomunismo e o livre mercado. Mas desde a queda da União Soviética, eles têm levantado questões sobre o mercado que antes nunca ousariam perguntar.
 \par 
No entanto, apesar de toda a sua repulsa pelo capitalismo desenfreado, Gray e Luttwak têm dificuldade em abraçar qualquer uma das alternativas: o máximo que Gray irá é caracterizar-se como “centro-esquerda”. Nem a esquerda está muito ansiosa para reivindicar qualquer um deles. Um crítico de False Dawn escreveu no These Times que Gray era apenas um porta-estandarte do antigo regime, movido menos por “um ódio genuíno à desigualdade, à injustiça ou à pobreza” do que por “um profundo medo da instabilidade política”. {\color{blue}7} Com o comunismo em ruínas e o mercado omnipotente, a paixão agonística que originalmente inspirou Luttwak e Gray agora
 \par 
Encontra-se sem casa. São os exilados mais comoventes da atualidade, perdidos numa diáspora que eles próprios criaram.
 \par 
Os conservadores geralmente se autodenominam céticos castigados que defendem a linha contra o entusiasmo político. Enquanto os radicais se inclinam para o utópico, os conservadores contentam-se com um realismo cansado do mundo. Mas, na realidade, os conservadores têm sido temperamentalmente antagónicos, politicamente insurgentes e totalmente opostos às convenções morais estabelecidas. Desde Edmund Burke, pensadores, de Samuel Taylor Coleridge a Martin Heidegger, têm procurado um modo de experiência mais intenso e quase extático nas esferas da religião, da cultura e até mesmo da economia – todas as quais, eles acreditam, são repositórios do misterioso e o inefável. Entregando-se ao romantismo político, eles recorrem ao património do contra-Iluminismo, celebrando a vitalidade inebriante da luta enquanto denunciam as normas exangues da razão e dos direitos. Como Isaiah Berlin observou sobre Joseph de Maistre:
 \par 
Sua violenta preocupação com sangue e morte pertence a um mundo diferente. . . Da sabedoria lenta e madura da nobreza rural, a paz profunda do país abriga grandes e pequenos. . . . A fachada do sistema de Maistre pode ser clássica, mas por trás dela há algo terrivelmente moderno e violentamente oposto à doçura e à luz.{\color{blue}8}
 \par 
A batalha do século XX contra o comunismo e a social-democracia proporcionou o veículo perfeito para estas sensibilidades conservadoras. Para figuras como John Gray, a União Soviética e o Estado-providência eram os símbolos máximos do frio racionalismo iluminista, e o mercado livre era a personificação do contra-iluminismo romântico. Mas os românticos revolucionários
 \par 
Em última análise, sofrerá com ela o destino de todos os românticos: a desilusão. E assim hoje, com o comunismo em ruínas e o mercado livre triunfante, o espírito dissidente que originalmente inspirou Gray desencadeia agora uma apostasia igualmente militante.
 \par 
Gray nasceu em 1948 e cresceu nos arredores de Newcastle, uma cidade portuária perto do Mar do Norte, numa região de mineração de carvão a apenas {\color{blue}80} quilômetros da Escócia. Num país onde o sotaque é o destino, ainda se ouvem tênues vestígios de suas origens operárias nordestinas, sobre as quais ele se posiciona ligeiramente na defensiva. Seu pai era carpinteiro; toda a sua família votou no Partido Trabalhista. Gray chegou a Oxford em 1968, o annus mirabilis para jovens esquerdistas em toda a Europa. Ostentando o traje da época – “meu cabelo era comprido, mas o cabelo de todo mundo era comprido” – ele viajou para Londres para se manifestar contra a Guerra do Vietnã. Depois de se formar em filosofia, política e economia, Gray permaneceu em Oxford para fazer pós-graduação, escrevendo uma tese sobre John Stuart Mill e John Rawls, ambos simpáticos a um socialismo liberal que Gray inicialmente achou atraente.
 \par 
Mas à medida que se debruçava sobre Uma Teoria da Justiça de Rawls, Gray cansou-se do esforço para extrair políticas socialistas de fórmulas liberais. Parte do seu mal-estar foi induzido pela prosa congestionada de Rawls. “É um livro quase ilegível”, diz ele. O estilo penoso de Rawls parecia reflectir o tédio político mais profundo da social-democracia. O seu trabalho, diz Gray, foi “uma dedução transcendental do Partido Trabalhista em 1963”. Tal como muitos novos esquerdistas nos Estados Unidos, Gray considerava o negócio do Estado-providência enfadonho e sem inspiração, o chá fraco de burocratas incolores. Como ele descreveria mais tarde, o Estado-Providência foi o produto de um “conluio triangular entre empregadores, sindicatos e governo”. Era um “aparelho colossal” que extraía recursos e energia de uma população enervada. O compromisso morno era a regra do dia; os líderes políticos tentaram ser tudo para todas as pessoas. Recusaram-se a “admitir a realidade dos conflitos”, que “uma igualdade, uma exigência de justiça, possa competir
 \par 
Com outro." {\color{blue}9} O Estado-Providência, em suma, estava muito longe do radicalismo vital da classe trabalhadora que o produziu.
 \par 
No thatcherismo, Gray teve um vislumbre da eternidade revolucionária. “Havia um aspecto revolucionário, na verdade bolchevique, no projecto thatcherista no início, que considerei ao mesmo tempo excitante e necessário”, diz ele. Thatcher assumiu a liderança do Partido Conservador quase na época da conversão de Gray ao capitalismo. Ela prometeu libertar a Grã-Bretanha da rotina sufocante da social-democracia e o mercado livre das cadeias do planeamento estatal. Embora não fosse igualitária, Thatcher alimentou as ambições dos eleitores da classe média e trabalhadora que viam o mercado livre como um veículo de mobilidade ascendente.
 \par 
O seu momento mais impressionante ocorreu em 1980, após o seu primeiro ano no poder, quando as suas políticas pareciam estar a empurrar a economia para o desastre. Depois de ter denunciado o seu antecessor, Edward Heath, por ter executado a sua notória “reviravolta”, quando capitulou à pressão da esquerda depois de prometer um retrocesso da social-democracia, Thatcher enfrentou pressão dos moderados do seu próprio partido – os Conservadores “Wets” – para inverter a situação. curso. Em vez de recuar, ela enfrentou desafiadoramente seus críticos contemporizadores, declarando de forma memorável: “Você vira se quiser. A senhora não gosta de virar. {\color{blue}10} conservadores foram feridos. Norman Barry, outro thatcherista e até recentemente amigo próximo de Gray, recorda: “Pensei que ela era apenas uma vencedora das eleições que não era trabalhista. Mas quando ela levantou os controlos cambiais, pensei: ‘Esta miúda conhece economia de mercado.’ Então pensei: ‘Sim!’ E então ela começou a privatização e outras coisas. E então ela não quis dar meia-volta, pensei: ‘Isso é real’”.
 \par 
Muitos thatcheristas consideravam-se revolucionários do livre mercado, e Gray trouxe para a sua causa um brio romântico nem sempre associado à economia neoclássica. Em 1974, começou a ler a obra de Friedrich Hayek, economista nascido na Áustria e crítico ferrenho do planeamento estatal. Dez anos depois, Gray
 \par 
Publicou Hayek sobre a Liberdade, que o próprio mestre descreveu como “o primeiro levantamento do meu trabalho que não apenas compreende completamente, mas é capaz de levar adiante minhas ideias além do ponto em que parei”. O Hayek retratado por Gray não era um defensor anti-séptico dos direitos de propriedade e de impostos baixos. Ele era um explorador exótico das correntes subterrâneas e quase racionais da vida humana, uma voz vienense que tinha mais em comum com Sigmund Freud e Ludwig Wittgenstein do que com Milton Friedman ou Robert Nozick. Se Hayek em Liberty era uma ode apaixonada ao mercado, Gray era o seu anseio, Byron.
 \par 
Enquanto muitos conservadores viam em Hayek o cumprimento lógico de uma tradição calma e quintessencialmente britânica de economia política que remontava a Adam Smith, Gray detectou uma “modernidade intransigente” na visão de Hayek do mercado livre. {\color{blue}11} Fermentação intelectual, extremismo político e decadência social caracterizaram os nós phi desde Viena, o meio em que Hayek nasceu. Desse turbilhão surgiram a psicanálise, o fascismo e a economia moderna. Cada um desafiou antigas ordens de conhecimento e política. Hayek seguiu os passos da escola austríaca do final do século XIX, afirmando que “o valor económico – o valor de um activo ou recurso – é-lhe conferido pelas preferências ou avaliações dos indivíduos e não por qualquer uma das suas propriedades objectivas”. {\color{blue}12} Enquanto os economistas clássicos, de David Ricardo a Karl Marx, acreditavam que tinha de haver algo real – o mais importante, o trabalho físico – por trás do misterioso véu dos preços, Hayek argumentava que eram apenas as preferências excêntricas de determinados seres humanos que davam valor aos bens em o mundo. Uma subjetividade quase hiperativa – comparável ao id anárquico de Freud – assombrava Hayek de Gray, refletindo a “experiência de Viena de uma tendência aparentemente inexorável para a dissolução”.{\color{blue}13}
 \par 
Contra os filósofos que elevaram a razão teórica à forma mais elevada de conhecimento, Hayek, escreveu Gray, acreditava que a compreensão racional era apenas a ponta do iceberg. Abaixo dele
 \par 
Estabeleceu-se uma camada obscura de pensamento “raramente exprimível em termos teóricos ou técnicos”, e foi a genialidade particular do mercado livre aproveitar estas premonições para a actividade económica quotidiana.{\color{blue}14}
 \par 
Os empreendedores foram os meios sublimes desse “conhecimento tácito”, canalizando as suas verdades profundas para outros intervenientes no mercado. Eram heróis românticos possuídos por lampejos de visão quase poética. “Insight ou percepção empreendedora”, explicou Gray, não era uma questão de aprendizado de livro, mas de “acaso e talento”. Foi “uma atividade criativa insuscetível de formulação em regras rígidas e rápidas”. Situada “além dos nossos poderes de controlo consciente”, a “percepção empreendedora” apareceu apenas raramente, atingindo-a subitamente e sem aviso. {\color{blue}15} Quando apareceu, reordenou o universo.
 \par 
Em suma, o mercado proporcionou um refúgio para a auto-expressão e a criatividade, um santuário para o arrebatador contra-iluminismo. Os escritores sem imaginação contentaram-se em argumentar que os mercados “alocam os recursos escassos de forma mais eficiente” ou que o mercado “permite a motivação do interesse próprio”. Mas tais defesas ignoravam uma verdade mais elementar: os mercados permitiam a expressão de “toda a variedade de motivos humanos, em toda a sua complexidade e misturas”. {\color{blue}16} O mercado fornecia um teatro para a auto-revelação dramática, um palco onde os indivíduos podiam projectar as suas visões mais irreprimíveis e os seus desejos mais vigorosos.
 \par 
Todos os casos amorosos chegam ao fim, mas o rompimento de Gray com o mercado foi particularmente venenoso. Ele agora denuncia isso como o flagelo da civilização. Nos Estados Unidos, escreve ele, o mercado livre “gerou um longo boom económico do qual a maioria dos americanos quase não beneficiou”. Os americanos sofrem com “níveis de desigualdade” que “se assemelham aos dos países latino-americanos”. A classe média desfruta dos encantos duvidosos da “insegurança económica sem bens que afligiu o proletariado do século XIX”. Os Estados Unidos estão perigosamente perto de enormes
 \par 
Disrupção, que só foi controlada “por uma política de encarceramento em massa” de afro-americanos e outras pessoas de cor. “O profeta da América de hoje”, afirma Gray, “não é Jefferson ou Madison. . . . É Jeremy Bentham” – o homem que sonhou com uma sociedade “reconstruída segundo o modelo de uma prisão ideal”.{\color{blue}17}
 \par 
Ainda mais assustador, escreve Gray, as elites globais têm procurado fazer do capitalismo americano o modelo para o mundo. Embora os regimes de mercado variem de acordo com a cultura e o país, os sumos sacerdotes da globalização impõem um modelo americano de tamanho único — com seu estado de bem-estar social mínimo, regulamentações comerciais e ambientais fracas e impostos baixos. “De acordo com o ‘consenso de Washington’”, escreve Gray, “as múltiplas culturas e sistemas econômicos que o mundo sempre conteve serão redundantes. Eles serão fundidos em um único mercado livre universal” com base no “último grande regime iluminista do mundo, os Estados Unidos”.{\color{blue}18}
 \par 
Quando Gray pronunciou pela primeira vez estas heresias, muitos dos seus amigos conservadores ficaram chocados. Assim como Gray, Norman Barry é um teórico político que escreveu sobre Hayek. Professor da Universidade de Buckingham, a única universidade totalmente privada da Grã-Bretanha, ele foi o padrinho do segundo casamento de Gray, mas agora raramente fala com ele. Barry não consegue afastar a suspeita de que a viragem política de Gray foi motivada por puro oportunismo. “Acredito numa proposta da economia neoclássica: todos são maximizadores de utilidade”, explica ele. “Pode ter sido um bom passo na carreira se distanciar do libertarianismo. Estou especulando, mas não descontroladamente. Os libertários não conseguem as melhores posições nas universidades.” Quando Gray era apenas um membro de uma pequena faculdade de Oxford, afirma Barry, “ele costumava dizer: ‘Bem, da maneira como o mundo funciona, eu não conseguiria uma cadeira’. . . Você não consegue um cargo de professor na LSE se for um fanático pelo livre mercado.” A única continuidade na posição de Gray que Barry reconhece é a sua propensão para a “promiscuidade filosófica”. Gray, diz Barry, “estava sempre passando de pessoa para pessoa, de filósofo para filósofo. . . . Ele não conseguia formar um relacionamento estável com
 \par 
Qualquer pensador. Ele experimentou um pouco de Popper. Tentei Hayek. Claro, mais tarde ele largou Hayek. Outros escritores ele tentaria descartar.”
 \par 
Gray afirma que mudou de ideia por dois motivos. Durante o final da década de 1980, diz ele, começou a suspeitar que o pensamento político da direita tinha um fim rígido numa ideologia obsoleta – não muito diferente do monótono Rawlsianismo do qual ele fugiu há tanto tempo. Gray já havia pensado no Thatchprismo como taticamente flexível e politicamente experiente, um movimento sensível aos sentimentos populares, cujo líder era um virtuoso maquiavélico da mudança política. Mas ele agora acreditava que o movimento havia perdido o seu talento artístico; o pensamento flexível degenerou em encantamento mecânico. Gray diz: “O que foi surpreendente no bolchevismo foi que Lenin era tão extraordinariamente flexível. Então endureceu no trotskismo. E da mesma forma o thatcherismo começou a endurecer. . . . Era um hábito de pensamento que achei profundamente repugnante.”
 \par 
O colapso da União Soviética também forçou Gray a questionar sua fé no livre mercado. Até 1989, diz Gray, fazia sentido pensar no estado como "o principal inimigo do bem-estar", que era a atitude dentro da "atmosfera reconhecidamente de estufa do think thanks de direita". Mas depois que o império soviético caiu, a antiga Iugoslávia entrou em uma espiral de guerra civil genocida e os defensores do livre mercado ocidentais aplicaram terapia de choque a países anteriormente comunistas com resultados desastrosos, Gray passou a pensar que o estado era um mal necessário, talvez até mesmo um bem positivo. Era a única força que poderia impedir que as sociedades deslizassem para o caos total, extrema desigualdade e pobreza.
 \par 
Mas há uma razão mais profunda para a mudança de Gray: por si só, o mercado não conseguiu sustentar a sua afeição. Sem a União Soviética e o Estado-providência como símbolos divertidos do racionalismo iluminista, Gray já não podia acreditar no mercado como antes. O mercado, tinha agora de admitir, patrocina um “culto à razão e à eficiência – principalmente”. Ele “rompe os fios da memória e espalha o conhecimento local”. Ele costumava pensar que o livre mercado surgiu espontaneamente
 \par 
E esse controle estatal da economia não era natural. Mas observando Jeff Rey Sachs e o Fundo Monetário Internacional na Rússia, ele não pôde deixar de ver o mercado livre como “um produto de artifício, design e coerção política”. O mercado teve de ser criado, muitas vezes com a ajuda do poder estatal implacável. Hoje, ele argumenta que Thatcher construiu o mercado livre esmagando os sindicatos, esvaziando o Partido Conservador e incapacitando o Parlamento. Ela “colocou a sociedade britânica numa marcha forçada para a modernidade tardia”. Gray acredita que “o marxismo-leninismo e o racionalismo económico de livre mercado têm muito em comum”. Ambos, escreve ele, “demonstram pouca simpatia pelas vítimas do progresso económico”. {\color{blue}19} Há apenas uma diferença: o comunismo está morto.
 \par 
Em um momento de descuido, Norman Barry confessa que não consegue imaginar a mudança de Gray. “Talvez eu o tenha entendido mal”, diz ele, “mas pensei que ele acreditava profundamente. De qualquer forma, ninguém poderia ter lido tanta coisa sem acreditar em alguma coisa. Eu me pergunto se ele alguma vez fez isso. Gray acreditava, mas sua crença era diferente da de Barry. Barry adora o mercado porque ele opera de acordo com “as leis férreas da economia”. Como ele diz, estas podem “levar um pouco mais de tempo do que as leis newtonianas. Se eu deixar cair este disco, ele cairá em um segundo. Se eu introduzir o controle dos aluguéis, levaria talvez seis meses para criar a situação de sem-teto.” Mas, acrescenta, “é igualmente decisivo”. Em contraste, Gray já acreditou no capitalismo precisamente porque procurava escapar às leis de Newton. Tendo percebido que o mercado inibe a auto-expressão apaixonada, Gray foi forçado a reconhecer a verdade da máxima de Irving Kristol: “O capitalismo é a concepção menos romântica de uma ordem pública que a mente humana alguma vez concebeu”.{\color{blue}20}
 \par 
Quando Edward Luttwak tinha quarenta e poucos anos, já havia ultrapassado os nazistas, escapado dos comunistas e sido alvejado por guerrilheiros esquerdistas na América Central. Mas até hoje ele se lembra de sua infância
 \par 
A mudança de Palermo para Milão foi o acontecimento mais “traumático” de sua vida. Nascido em 1942 numa rica família judia na Roménia, Luttwak cresceu no sul da Transilvânia, que foi brevemente ocupada pelos nazis em 1944. Quando tinha cinco anos, a sua família fugiu de uma iminente tomada comunista e estabeleceu-se em Palermo. Era inverno, lembra Luttwak, e “Paris e Londres tremiam. Houve falta de combustível. Milano estava tremendo. As coisas estavam muito sombrias. Mas em Palermo “a ópera estava em pleno andamento”. Era “a terra das laranjas e dos limões”, diz ele, onde as pessoas podiam nadar e esquiar quase o ano todo. Cinco anos depois, a família de Luttwak mudou-se novamente, desta vez para Milão, o centro industrial da Itália. “Entupido e cheio de neblina”, Milan deixou Luttwak infeliz. “Não havia onde brincar. Os parques eram uma vergonha. Perdi todos os meus amigos de Palermo. Eu me achei. . . No meio de um bando de garotos muito burgueses.” A boa vida no Mediterrâneo havia chegado ao fim, destruída pelos severos industriais do norte.
 \par 
Durante a maior parte da sua vida adulta, Luttwak travou uma luta militante contra o comunismo. Inspirado por uma visão militar estratégica que ligava as Guerras Gálicas às guerras civis da América Central, trabalhou em estreita colaboração com o Departamento de Defesa dos EUA como consultor, aconselhando todos, desde oficiais subalternos até ao alto escalão. Mas Luttwak era mais do que um guerreiro frio. Ele era um guerreiro, ou pelo menos um fervoroso teórico da “arte da guerra”. Enquanto os generais pensavam que a vitória dependia da imitação dos estilos de gestão da IBM, Luttwak defendeu antigas tácticas de campo de batalha e manobras esquecidas do Império Romano. Luttwak instou os militares a buscarem orientação em Adriano, e não em Henry Ford. Foi uma luta árdua, com os oficiais agindo mais frequentemente como homens de organização do que como soldados. Mais uma vez, Luttwak viu o seu modo de vida preferido ameaçado pela cultura do capitalismo.
 \par 
Luttwak ganhou notoriedade pela primeira vez na Grã-Bretanha, onde se estabeleceu após receber seu diploma de graduação em economia na Universidade de Londres.
 \par 
Escola de Economia. Em 1968, ele publicou Golpe de Estado: Um Manual Prático. O autor de {\color{blue}26} anos deslumbrou seus leitores com este audacioso guia prático, levando um encantado John le Carré a escrever: “Sr. Luttwak compôs um guia gastronômico profano sobre o veneno político. Aqueles que forem corajosos o suficiente para olhar para sua cozinha nunca mais comerão tão pacificamente.” Em 1970, Luttwak publicou um artigo igualmente malicioso na Esquire, “Um Cenário para um Golpe de Estado Militar nos Estados Unidos”. Dois anos depois, mudou-se para os Estados Unidos para escrever uma dissertação em ciência política e história clássica na Johns Hopkins, conduzindo uma extensa pesquisa usando fontes originais em latim, alemão, francês, inglês e italiano. O resultado foi que elogiaram amplamente a Grande Estratégia do Império Romano. Enquanto cursava a pós-graduação, Luttwak começou a trabalhar como consultor para vários ramos das forças armadas dos EUA, fazendo recomendações sobre tudo, desde como a OTAN deveria conduzir manobras táticas até que tipo de rifle os soldados do exército salvadorenho deveriam portar.
 \par 
Quando Ronald Reagan concorreu à presidência em 1980, Luttwak estava no auge. Membro do Centro de Estudos Estratégicos e Internacionais de Georgetown e colaborador frequente da Commentary, argumentou que os Estados Unidos deveriam acelerar a corrida armamentista de alta tecnologia, forçando a União Soviética a participar numa competição que não poderia vencer. Os conselheiros mais próximos de Reagan acolheram Luttwak com entusiasmo no seu círculo íntimo. Logo após a eleição de Reagan, Luttwak participou num jantar em Bethesda, juntamente com Jeane Kirkpatrick, Fred Iklé e outros luminares do establishment republicano da defesa. Richard Allen, que se tornaria o primeiro conselheiro de segurança nacional de Reagan, agitou a multidão, fingindo distribuir cargos na administração como se fossem favores partidários. Como noticiou o Washington Post, Luttwak recusou, explicando enquanto comia a torta de chocolate Tia Maria: “Não acredito que escribas como eu devam se envolver em política. É como caviar. Muito bom, mas apenas em pequenas quantidades.”
 \par 
Quando pressionado por Allen, ele brincou: “Só quero ser vice-cônsul em Florença”. Allen respondeu: “Você não quer dizer procônsul?”{\color{blue}21}
 \par 
A camaradagem dos gladiadores da escola preparatória evaporou antes do fim do primeiro mandato de Reagan. Luttwak pode ter sido um trunfo inestimável ao pressionar por mais gastos com defesa, mas ele fez inimigos com suas críticas altas — e cada vez mais sarcásticas — à má gestão do Pentágono. Em 1984, ele publicou O Pentágono e a Arte da Guerra, onde, entre outras coisas, ele descreveu o Secretário de Defesa Caspar Weinberger mais como um vendedor de carros usados ​​do que um estadista genuíno. Os políticos militares revidaram, tirando Luttwak de uma lista de consultores pro bono do Pentágono (ele continuou a fazer trabalho contratado em outras partes do establishment de defesa). Em 1986, Weinberger explicou ao Los Angeles Times que Luttwak "simplesmente perdeu cargos de consultoria por total incompetência, só isso".{\color{blue}22}
 \par 
Mas foram mais do que as críticas de Luttwak a Weinberger que o colocaram em problemas com o Departamento de Defesa. O seu verdadeiro erro em O Pentágono e a Arte da Guerra foi perseguir a conduta dos militares durante a Guerra do Vietname. Luttwak minimizou as explicações favoritas das forças armadas para a sua derrota no Vietname – políticos de vontade fraca, a imprensa traiçoeira, um público derrotista. Em vez disso, argumentou que a elite guerreira da América tinha simplesmente perdido o gosto pelo sangue. Durante a Guerra do Vietname, escreveu ele, “oficiais de escritório” estavam sempre “longe do combate”. A sua propensão para o “luxo absoluto” teve um efeito devastador sobre o moral das tropas. Embora Júlio César “mantivesse concubinas e calamidades em seu quartel-general da retaguarda, comesse em pratos de ouro e bebesse seu vinho de Sâmia em taças de joias”, quando estava na linha de frente com seus soldados, ele “comia apenas o que eles comiam, e dormiam como eles - debaixo de uma tenda, se as tropas tivessem tendas, ou simplesmente enroladas em um cobertor, se não as tivessem. Em contraste, os oficiais americanos recusaram “partilhar as dificuldades e os riscos mortais da guerra”.{\color{blue}23}
 \par 
Burocratas de cabeça pontiaguda também minaram a força dos militares, segundo Luttwak. Sempre procurando cortar custos, os responsáveis ​​do Pentágono insistiram que as armas, a maquinaria e os programas de investigação e desenvolvimento fossem padronizados. Mas isto apenas tornou os militares vulneráveis ​​ao ataque inimigo. Os sistemas de armas padronizados foram facilmente superados; tendo subjugado um, um inimigo poderia subjugar todos eles. No que diz respeito aos militares, concluiu Luttwak, “precisamos de mais ‘fraude, desperdício e má gestão’”.
 \par 
Os principais generais eram obcecados pela eficiência, em parte porque aprenderam os métodos de gestão empresarial em vez da arte da guerra. Para cada oficial formado em história militar, havia mais cem “cuja maior realização pessoal é uma pós-graduação em administração de empresas, gestão ou economia”. “Por que deveriam os pilotos de caça receber uma educação universitária completa”, perguntou Luttwak no Washington Quarterly, “em vez de serem ensinados a caçar e matar com as suas máquinas?”{\color{blue}25}
 \par 
A fonte final da disfunção militar foi sua adoção da cultura corporativa e dos valores empresariais americanos. Como Robert McNamara, que o presidente Kennedy transferiu da Ford Motor Company para o Pentágono, a maioria dos secretários de defesa estava cativa por “objetivos de estilo corporativo”. Eles buscavam os meios menos arriscados e mais econômicos para um determinado fim. Eles preferiam ternos cinza, evitando “excentricidades pessoais em vestimenta, discurso, maneiras e estilo porque qualquer característica incomum pode irritar um cliente ou um banqueiro nos encontros casuais comuns nos negócios”. Os oficiais eram meramente “gerentes uniformizados”, Luttwak disse à Forbes. Mas, ele observou, “o que é bom para os negócios não é bom para conflitos mortais”. Embora “vestimentas conservadoras seguras e estilo inofensivamente convencional” pudessem funcionar em um escritório, eles poderiam ser mortais no campo de batalha; eles sufocavam iniciativas ousadas e gênios idiossincráticos. {\color{blue}26} Insinuando que o capitalismo havia colonizado — na verdade, destruído — esferas da sociedade que não eram estritamente econômicas, Luttwak chegou perigosamente
 \par 
Perto de se identificar com as principais vozes da tradição marxista – Jürgen Habermas, Georg Lukács e até o próprio Marx.
 \par 
Enquanto a União Soviética ainda existia, Luttwak conseguiu canalizar o seu desprezo pelos valores empresariais e empresariais em propostas de reforma militar. A luta contra o bolchevismo capturou totalmente a sua imaginação, falando de princípios de individualismo, independência e dignidade pessoal que ele aprendeu quando era filho de judeus ateus. Os pais de Luttwak lhe ensinaram, diz ele, que “você queria ter os ombros abertos andando pela rua. O mestre do seu destino. Não andar curvado, com medo de que Deus te castigue se você comer um sanduíche de presunto”. Ele continua: “Havia um certo desprezo pela piedade. A piedade não era vista como compatível com a dignidade.” Dignidade, prossegue ele, “é o que defendíamos na Guerra Fria. Foi ideológico. Foi muito apropriado para mim estar nos Estados Unidos, tornar-me americano, porque os americanos eram e são o povo ideológico. Eles foram perfeitamente escalados para serem alistados em uma luta ideológica.”
 \par 
Mas agora que a batalha contra o comunismo foi vencida, Luttwak perdeu o interesse na maioria dos assuntos militares; ele não vê mais nenhuma razão ideológica convincente para se preocupar com estratégia e tática. “Os problemas de segurança e outros tornaram-se periféricos, para todos os países e para as pessoas, e para mim também. Não envolvo minha existência em algo periférico. . . . Havia um imperativo imperativo de estar envolvido. Não há agora.
 \par 
Luttwak ocasionalmente reúne energia para um projeto específico. Durante uma de nossas entrevistas, ele fala por telefone com um funcionário do Departamento de Estado sobre fazer um trabalho de consultoria para a guerra contra as guerrilhas colombianas. Mas quando lhe pergunto se vale a pena defender o governo colombiano, ele mostra-se estranhamente hesitante, confessando finalmente: “Não sei se vale a pena defender alguma coisa, mas penso que vale a pena lutar contra as guerrilhas”. Eu pergunto por que e ele responde que
 \par 
As guerrilhas estão alinhadas com os traficantes de droga que “fazem de tudo, desde ocupar os lugares das pessoas nos restaurantes de Medillín num sábado à noite – as pessoas estão à espera para ocupar lugares e estes tipos entram, e agarram as suas mesas – tudo, desde isso até homicídio”.
 \par 
A luta militar pode já não ter qualquer fascínio ideológico para Luttwak, mas o seu descontentamento proporciona-lhe o tempo e o espaço intelectual para confrontar o inimigo que tem perseguido durante toda a sua vida: o próprio capitalismo. “O mercado”, diz ele, “invade todas as esferas da vida”, produzindo uma “sociedade infernal”. Da mesma forma que os valores de mercado antes ameaçavam a segurança nacional, ameaçam agora o bem-estar económico e espiritual da sociedade. “Um sistema de produção ideal é um sistema de produção completamente desumano”, explica ele, “porque. . . Você muda constantemente o número de pessoas que emprega, muda-as de lugar, faz coisas diferentes, e isso não é compatível com alguém ser capaz de organizar uma existência para si mesmo.”
 \par 
Embora Luttwak escreva no seu livro Turbo-Capitalism de 1999: “Acredito profundamente. . . Nas virtudes do capitalismo”, a sua oposição à difusão dos valores de mercado é tão aguda que o coloca no extremo oposto do espectro político actual – uma posição que Luttwak desfruta congenialmente. {\color{blue}27} “Edward é um cara muito perverso, intelectualmente e em muitos outros aspectos”, diz o ex-editor da Commentary Norman Podhoretz, um dos primeiros defensores de Luttwak durante a década de 1970. “Ele é um contraditório. Ele gosta de confundir expectativas. Mas, francamente, nem sei o quão sério ele está falando nesta última encarnação.” Luttwak insiste que está falando sério. Ele clama pela medicina socializada. Ele defende um estado de bem-estar forte, afirmando: “Se eu pudesse, proibiria qualquer forma de caridade doméstica”. A caridade é uma “escuta”, diz ele: tira a dignidade dos pobres.
 \par 
A única coisa que desperta a ira de Luttwak mais do que o capitalismo desenfreado são os seus entusiastas de elite – os intelectuais, políticos, decisores políticos e empresários que afirmam que “só porque
 \par 
O mercado é sempre mais eficiente, o mercado deve sempre governar.” Alan Greenspan merece o desprezo especial de Luttwak: “Alan Greenspan é um spenceriano. Isso faz dele um fascista econômico.” Spencerianos como Greenspan acreditam que “as pressões económicas mais duras” irão “estimular algumas pessoas a fazê-lo. . . Atos economicamente heróicos. Eles se tornarão grandes empreendedores ou qualquer outra coisa, e quanto aos que fracassarem, deixem-nos fracassar.” A outra bête noire de Luttwak é “Chainsaw Al” Dunlap, o CEO peripatético que colhe retornos inimagináveis ​​para os acionistas corporativos ao contratar um número substancial de funcionários das empresas. “A motosserra faz isso”, diz Luttwak, referindo-se às medidas de redução de pessoal de Dunlap, “porque ele é simplório, duro e cruel”. É apenas “sadismo económico”. Contra Greenspan e Dunlap, Luttwak afirma: “Acredito que se deve ter apenas a eficiência de mercado necessária, porque tudo o que valorizamos na vida humana está dentro do reino da ineficiência – amor, família, apego, comunidade, cultura, velhos hábitos, sapatos velhos e confortáveis.”
 \par 
As deserções de Luttwak e Gray sugerem quão cruel foi o fim da Guerra Fria para o movimento conservador. É cada vez mais claro que a frágil coligação de libertários, tradicionalistas e entusiastas do mercado livre, outrora mantida unida pela cola do anticomunismo, já não irá aderir. O fim da União Soviética “privou-nos de um inimigo”, diz-me Irving Kristol, o padrinho intelectual do neoconservadorismo. “Na política, ser privado de um inimigo é um assunto muito sério. Você tende a ficar relaxado e desanimado. Volte-se para dentro.” Famoso pela sua autoconfiança, Kristol confessa agora uma triste perplexidade no mundo pós-comunista. “Essa é uma das razões pelas quais não escrevo muito atualmente”, diz ele. “Não sei as respostas.”
 \par 
Poderíamos pensar que o triunfo do mercado livre emocionaria os intelectuais de direita. Mas mesmo o conservador mais reverenciado
 \par 
Os patriarcas preocupam-se com o facto de o mercado por si só não conseguir sustentar as energias enfraquecedoras do movimento. Afinal, Reagan e Thatcher convocaram os conservadores para uma cruzada política, mas a ideologia do mercado livre que desencadearam é suspeita para todas as crenças políticas. A lógica do mercado glorifica a iniciativa privada, a acção individual, o brilho do não planeado e do aleatório. Neste contexto, é difícil pensar em política – muito menos em transformação política. William F. Buckley Jr. me diz: “O problema com a ênfase no conservadorismo no mercado é que isso se torna um tanto enfadonho. Você ouve uma vez e domina a ideia. A ideia de dedicar sua vida a isso é horrível, apenas porque é muito repetitivo. É como sexo. Kristol acrescenta: “O conservadorismo americano carece de imaginação política. É tão influenciado pela cultura empresarial e pelos modos de pensar empresariais que lhe falta qualquer imaginação política, o que sempre foi, devo dizer, uma propriedade da esquerda.” Ele continua: “Se você ler Marx, aprenderá o que uma imaginação política poderia fazer”.
 \par 
Mas se os conservadores estão a lutar para encontrar uma visão, poderão os ex-conservadores fazer muito melhor? Ao contrário de Kristol, que fugiu da esquerda e lançou o movimento neoconservador, Luttwak e Gray não formularam alternativas coerentes, filosóficas ou políticas, aos seus antigos credos. Como diz Luttwak: “Em vez de propor toda uma contra-ideologia, o que proponho simplesmente é que a sociedade diga conscientemente que certas coisas devem ser protegidas do mercado e mantidas fora do mercado”. Isto, apesar do facto de Luttwak permanecer temperamentalmente apaixonado, à sua maneira, pelo impulso revolucionário. “Prefiro ‘A Marselhesa’ à Missa”, diz ele, “Mayaski à cruz de São Jorge”. Ele acrescenta: “As revoluções são maravilhosas. Pessoas se divertindo. Estive em Paris em 1968. . . . Havia uma sensação maravilhosa de possibilidade.” Mas embora Luttwak possa ansiar por uma política transformadora, ela permanece fora do seu alcance, um objecto de nostalgia não só para ele, mas para a maioria dos intelectuais.
 \par 
Exceto, ao que parece, por William F. Buckley Jr., o bad boy original da direita americana. No final da nossa entrevista, peço a Buckley para imaginar uma versão mais jovem de si mesmo, um aspirante a enfant terrible político se formando na faculdade em 2000, trazendo ao mundo político de hoje o mesmo espírito insurgente que Buckley trouxe para o seu. Que tipo de política esse jovem Buckley abraçaria? "Eu seria um socialista", ele responde. "Um socialista Mike Harrington." Ele faz uma pausa. "Eu diria até mesmo um comunista."
 \par 
Será que ele consegue realmente imaginar um jovem comunista, Bill Buckley? Ele admite que é difícil. O Bill Buckley original tinha o benefício da União Soviética como inimiga; sem o seu equivalente, o seu doppelgänger enfrentaria uma tarefa mais complicada. “Esse novo Buckley teria que apontar para outras coisas”, diz ele. Buckley apresenta uma longa lista de causas de esquerda – pobreza global, morte por SIDA. Mas até ele parece repentinamente oprimido pelo projeto de (no típico Buckleyese) “juntar tudo isso em uma inspiração cativante”. Assustado pelo desafio de pensar fora do mercado livre, Buckley faz uma pausa e finalmente diz: “Vou deixar isso para você”.