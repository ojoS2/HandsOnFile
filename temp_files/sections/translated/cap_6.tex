 
 \chapter{The Excons}  

 \label{The Excons}  
 
 
\par
 
 
 \textit{	}  

 
\par
 
 
 
\par
 

 \textbf{\textit{	} }  

 
\par
 

 \footnote{This chapter originally appeared as an article in Lingua Franca (February 2001): 24–33.}  

 \textbf{\textit{In the spring of 2000, Alex Star, editor of the now-defunct Lingua Franca, commissioned me to write a profile of John Gray and Edward Cutaway, two conservative intellectuals who had moved to the left. Throughout the summer and fall, I interviewed Gray and Cutaway as well as other conservatives such as William F. Buckley, Irving Bristol, and Nor-man Podhoretz. It was a difficult time for the right. Bill Clinton was still president; 9/11 had not yet occurred. Prosperity was a given, war was a distant memory, and learned people still spoke of the end of history. The moment had a vastly diff erent feel from today, and it affected how conservatives thought about their ideas and politics. While some of the references and statements in this article are now dated, and some of its claims I no longer believe, I have decided not to revise the piece in order to pre-serve the mood of that moment. In chapter 8, I revisit some of the issues discussed here in light of 9/11, the war on terror, and the Iraq War.} } 

 
\par
 

 \textbf{\textit{There is another reason I have not revised this article. Though I had read Burke, Makeshift, and Notice in college and graduate school, researching and writing this article was my first sustained encounter with the worldview of the right. (It remains an unfortunate reality of American higher education that social scientists and historians can get through} } 

 
\par
 

 
\par
 

 \textbf{\textit{Their training with only the most passing acquaintance with conservatism.) This article became a kind of sentimental education for me, my introduction to the agony and the ecstasy of the conservative mind. While I would certainly revise much of it today—particularly the underlying premise that the conservatives I discuss here are different from the main-stream—the article nevertheless provides the reader with a glimpse of what first interested me about the right and how I came to write this book.} }  
 
 
\par
 
Conforme o mito popular, foi Winston Churchill quem disse: “Qualquer homem com menos de trinta anos que não seja liberal não tem coração, e qualquer homem com mais de trinta anos que não seja conservador não tem cérebro”. Ele não o disse, mas o seu imprimátur transformou uma piada inteligente de proveniência incerta num axioma da biografia política: o radicalismo é um privilégio da juventude, o conservadorismo uma responsabilidade da idade, e cada pessoa pensante acaba por renunciar ao primeiro pelo segundo. De Max Eastman a Eugene Genovese, de Whittaker Chambers a Ronald Radish, os intelectuais migram da esquerda para a direita como se obedecessem a uma lei da natureza.
 
\par
 
Ou não? Afinal, John Stuart Mill publicou The Subjection of. Women quando tinha sessenta e três anos. Nos últimos dez anos da sua vida, Diderot saudou a Revolução Americana e criticou a França como a reencarnação da Roma imperial. E quando George Bernard Shaw abordou a questão da política e do envelhecimento, sugeriu exatamente o oposto do que Churchill teria supostamente dito. “As pessoas mais ilustres”, escreveu Shaw em 1903, “tornam-se mais revolucionárias à medida que envelhecem”.
 {\color{blue} 1}  

 
\par
 
Desde o fim da Guerra Fria, vários conservadores proeminentes seguiram a prescrição de Shaw e viraram à esquerda. Michael Lind, que já foi editor-chefe do The National Interest, de Irving Bristol, denunciou seus anteriores aliados por promoverem uma “guerra de classes contra os americanos assalariados”. As suas teorias orientadas para o mercado, escreve ele, são “pouco convincentes” e as suas políticas econômicas são “terríveis”. Arianna Huffington, antiga confederada de Newt Gingrich, agora investe contra os Estados Unidos onde a grande maioria está “sufocada com a poeira dos touros galopantes de Wall Street”.
 {\color{blue} 2}  
Glenn Lowry, economista e antigo queridinho dos neoconservadores, ostenta o emblema característico da pertença à esquerda: tornou-se um dos ex-amigos de Norman Podhoretz. Mas os expatriados mais extravagantes de hoje são um inglês, John Gray, e um judeu emigrado da Transilvânia, Edward Cutaway.
 
\par
 
Na década de 1970, John Gray era uma estrela em ascensão da Nova Direita Britânica. Filósofo político formado em Oxford, ele escreveu poemas em prosa para o livre mercado, cruzou o Atlântico para se abastecer do libertarianismo de alta octanagem dos think tanks de direita americanos e, diz um amigo de longa data, cativou seus camaradas até altas horas da noite com visões da vindoura Utopia "anarco capitalista". Mas depois que o Muro de Berlim caiu, gray desertou. Primeiro, ele criticou o triunfalismo da Guerra Fria da tese do "fim da história" de Francis Fukuyama e aconselhou contra o fim do Serviço Nacional de Saúde da Grã-Bretanha. E então, em 1998, de sua posição recém-estabelecida como professor de pensamento europeu na London School of Economics (LSE), ele entregou False Dawn, uma denúncia feroz da globalização econômica. Ao atacar as “tropas de choque do mercado livre”, gray alertou que o capitalismo global poderia “rivalizar” com a antiga União Soviética “no sofrimento que inflige”.
 {\color{blue} 3}  
Agora ele é um colaborador regular do The Guardian e do New Statesman, os principais veículos de esquerda da Grã-Bretanha. Sua conversão é tão profunda que ninguém menos que Margaret Thatcher teria se perguntado: “O que aconteceu com John Gray? Ele costumava ser um de nós.
 {\color{blue} 4}  

 
\par
 
E quanto a Edward Cutaway? Antigamente, ele foi um dos intelectuais da corte de Ronald Reagan, um falcão militar brilhante que criticou impiedosamente as políticas de defesa liberais e forneceu a fundamentação filosófica para a escalada militar americana na década de 1980.
 
\par
 
Os críticos liberais o chamavam de “Eddie louco”, mas representando uma figura que era em parte o amor do Dr. Estranho e em parte o Dr.
 {\color{blue} 5}  
Hoje, ele está desiludido com a vitória. Ele considera os Estados Unidos um pesadelo capitalista, “um aviso sombrio” aos líderes que procuram libertar forças de mercado livre nos seus próprios países. Utilizando o mesmo humor ácido que outrora usou contra os tempos de paz liberais, ele zomba das “pretensões napoleônicas” dos líderes empresariais americanos, desafia a sabedoria convencional de que o capitalismo e a democracia são companheiros inevitáveis ​​(“mercados livres e sociedades menos livres andam de mãos dadas”), e condena as desigualdades selvagens produzidas pelo “turbo capitalismo”. Ele critica os europeus de centro-esquerda, como Tony Blair, por abandonarem as suas raízes socialistas e pela sua relutância em “arriscar qualquer ação inovadora” em nome dos “trabalhadores comuns”. Com o seu “desdém pelos pobres e outros perdedores” e “desprezo pelas amplas massas da classe trabalhadora”, escreve Cutaway, os Novos Democratas Clintonistas e as Terceiras Águas Europeias “só podem produzir políticas de direita”.
 {\color{blue} 6}  

 
\par
 
Nas suas encarnações originais, Gray e Cutaway entusiasmaram-se com duas das paixões galvanizadoras do conservadorismo – o anticomunismo e o mercado livre. Mas desde a queda da União Soviética, têm colocado questões sobre o mercado que nunca teriam ousado colocar.
 
\par
 
No entanto, apesar de toda a sua repulsa pelo capitalismo desenfreado, Gray e Cutaway têm dificuldade em abraçar qualquer uma das alternativas: o máximo que Gray irá é caracterizar-se como “centro-esquerda”. Nem a esquerda está muito ansiosa para reivindicar qualquer um deles. Um crítico de False Dawn escreveu no These Times que Gray era apenas um porta-estandarte do antigo regime, movido menos por “um ódio genuíno à desigualdade, à injustiça ou à pobreza” do que por “um profundo medo da instabilidade política”.
 {\color{blue} 7}  
Com o comunismo em ruínas e o mercado omnipotente, a paixão agonística que originalmente inspirou Cutaway e Gray encontra-se agora sem lar. São os exilados mais comoventes da atualidade, perdidos numa diáspora que eles próprios criaram.
 
\par
 
Os conservadores geralmente se autodenominam céticos castigados que defendem a linha contra o entusiasmo político. Enquanto os radicais se inclinam para o utópico, os conservadores contentam-se com um realismo cansado do mundo. Mas, na realidade, os conservadores têm sido temperamentalmente antagônicos, politicamente insurgentes e totalmente opostos às convenções morais estabelecidas. Desde Edmund Burke, pensadores, de Samuel Taylor Coleridge a Martin Heidegger, têm procurado um modo de experiência mais intenso e quase extático nas esferas da religião, da cultura e até mesmo da economia – todas as quais, eles acreditam, são repositórios do misterioso e o inefável. Entregando-se ao romantismo político, eles recorrem ao patrimônio do contra-iluminismo, celebrando a vitalidade inebriante da luta enquanto denunciam as normas exangues da razão e dos direitos. Como Isaiah Berlin observou sobre Joseph de Maistre:
 
\par
 

 \textbf{\textit{His violent preoccupation with blood and death belongs to a world different. . . From the slow, mature wisdom of the landed gentry, the deep peace of the country houses great and small. . . . The facade of Maître’s system may be classical, but behind it there is something terrifyingly modern, and violently opposed to sweetness and light. {{\color{blue} 8} } } }  
 
 
\par
 
A batalha no século XX contra o comunismo e a social-democracia forneceu o veículo perfeito para essas sensibilidades conservadoras. Para figuras como John Gray, a União Soviética e o estado de bem-estar social eram os símbolos máximos do racionalismo frio do Iluminismo, e o livre mercado era a personificação do contra iluminismo romântico. Mas os românticos revolucionários, em última análise, sofrem o destino de todos os românticos: desilusão. E assim, hoje, com o comunismo em ruínas e o livre mercado triunfante, o espírito dissidente que originalmente inspirou Gray agora dispara uma apostásia igualmente militante.
 
\par
 
Gray nasceu em 1948 e cresceu nos arredores de Newcastle, uma cidade portuária perto do Mar do Norte, numa região de mineração de carvão a apenas 80 quilômetros da Escócia. Num país onde o sotaque é o destino, ainda se ouvem tênues vestígios de suas origens operárias nordestinas, sobre as quais ele se posiciona ligeiramente na defensiva. Seu pai era carpinteiro; toda a sua família votou no Partido Trabalhista. Gray chegou a Oxford em 1968, o anus mirabilis para jovens esquerdistas em toda a Europa. Ostentando o traje da época – “meu cabelo era comprido, mas o cabelo de todo mundo era comprido” – ele viajou para Londres para se manifestar contra a Guerra do Vietnã. Após se formar em filosofia, política e economia, gray permaneceu em Oxford para fazer pós-graduação, escrevendo uma tese sobre John Stuart Mill e John Rawls, ambos simpáticos a um socialismo liberal que Gray inicialmente achou atraente.
 
\par
 
Mas à medida que se debruçava sobre Uma Teoria da Justiça de Rails, Gray cansou-se do esforço para extrair políticas socialistas de fórmulas liberais. Parte de seu mal-estar foi induzido pela prosa congestionada de Rails. “É um livro quase ilegível”, diz ele. O estilo árduo de Rails parecia espelhar o tédio político mais profundo da social-democracia. O seu trabalho, diz Gray, foi “uma dedução transcendental do Partido Trabalhista em 1963”. Tal como muitos novos esquerdistas nos Estados Unidos, Gray considerava o negócio do Estado-providência enfadonho e sem inspiração, o chá fraco de burocratas incolores. Como ele descreveria mais tarde, o Estado-Providência foi o produto de um “conluio triangular entre empregadores, sindicatos e governo”. Era um “aparelho colossal” que extraía recursos e energia de uma população enervada. O compromisso morno era a regra do dia; os líderes políticos tentaram ser tudo para todas as pessoas. Recusaram-se a “admitir a realidade dos conflitos”, que “uma igualdade, uma exigência de justiça, pode competir com outra”.
 {\color{blue} 9}  
Em suma, o Estado-providência estava muito longe do radicalismo vital da classe trabalhadora que o produziu.
 
\par
 
No fletcherismo, gray vislumbrou a eternidade revolucionária. “Havia um aspecto revolucionário, de fato bolchevique, no projeto Thatcherite no início, que eu achava emocionante e necessário”, ele diz. Thatcher assumiu a liderança do Partido Conservador quase na época da conversão de Gray ao capitalismo. Ela prometeu libertar a Grã-Bretanha da rotina sufocante da social-democracia e o livre mercado das correntes do planejamento estatal. Embora não fosse igualitária, Thatcher alimentou as ambições dos eleitores da classe média e trabalhadora que viam o livre mercado como um veículo de mobilidade ascendente.
 
\par
 
O seu momento mais impressionante ocorreu em 1980, após o seu primeiro ano no poder, quando as suas políticas pareciam estar a empurrar a economia para o desastre. Após ter denunciado o seu antecessor, Edward Heath, por executar a sua notória “reviravolta”, quando capitulou à pressão da esquerda após prometer um retrocesso da social-democracia, Thatcher enfrentou pressão dos moderados do seu próprio partido – os conservadores “Wets” – para inverter a situação. curso. Em vez de recuar, ela enfrentou desafiadoramente seus críticos contemporizadores, declarando de forma memorável: “Você vira se quiser. A senhora não gosta de virar.
 {\color{blue} 10}  
Os conservadores ficaram impressionados. Norman Barry, outro companheiro de Thatcher e até recentemente amigo próximo de Gray, lembra: “Eu pensava que ela era apenas uma vencedora das eleições que não era trabalhista. Mas quando ela levantou os controlos cambiais, pensei: ‘Esta miúda conhece economia de mercado.’ Então pensei: ‘Sim!’ E então ela começou a privatização e outras coisas. E então ela não faria meia-volta, pensei: ‘Isso é real’”. Muitos Hatcheries se consideravam revolucionários do livre mercado, e Gray trouxe para sua causa um brio romântico nem sempre associado à economia neoclássica. Em 1974, começou a ler a obra de Friedrich Hayek, economista nascido na Áustria e crítico ferrenho do planejamento estatal. Dez anos depois, Gray publicou Hayek on Liberty, que o próprio mestre descreveu como “o primeiro levantamento do meu trabalho que não apenas compreende completamente, mas é capaz de levar adiante minhas ideias além do ponto em que parei”. O Hayek retratado por Gray não era um defensor anti-séptico dos direitos de propriedade e de impostos baixos. Ele era um explorador exótico das correntes subterrâneas e quase racionais da vida humana, uma voz vienense que tinha mais em comum com Sigmund Freud e Ludwig Wittgenstein do que com Milton Friedman ou Robert Nozick. Se Hayek em Liberty era uma ode apaixonada ao mercado, gray era o seu anseio, Byron.
 
\par
 
Enquanto muitos conservadores viam em Hayek o cumprimento lógico de uma tradição calma e quintessencialmente britânica de economia política que remontava a Adam Smith, Gray detectou uma “modernidade intransigente” na visão de Hayek do mercado livre.
 {\color{blue} 11}  
Fermentação intelectual, extremismo político e decadência social caracterizaram Phi node- desde Viena, o meio em que Hayek nasceu. Desse turbilhão surgiram a psicanálise, o fascismo e a economia moderna. Cada um desafiou antigas ordens de conhecimento e política. Hayek seguiu os passos da escola austríaca do final do século XIX, afirmando que “o valor econômico – o valor de um ativo ou recurso – é-lhe conferido pelas preferências ou avaliações dos indivíduos e não por qualquer uma das suas propriedades objectivas”.
 {\color{blue} 12}  
Enquanto os economistas clássicos, de David Ricardo a Karl Marx, acreditavam que devia haver algo real – o mais importante, o trabalho físico – por trás do misterioso véu dos preços, Hayek argumentava que eram apenas as preferências excêntricas de determinados seres humanos que davam valor aos bens no mundo. Mundo. Uma subjetividade quase hiperativa – comparável ao id anárquico de Freud – assombrava Hayek de Gray, refletindo a “experiência de Viena de uma tendência aparentemente inexorável para a dissolução”.
 {\color{blue} 13}  

 
\par
 
Contra filósofos que elevaram a razão teórica à mais alta forma de conhecimento, Hayek, escreveu Gray, acreditava que o entendimento racional era apenas a ponta do iceberg. Abaixo dele, havia um estrato obscuro de pensamento “raramente expressável em termos teóricos ou técnicos”, e era o gênio particular do livre mercado aproveitar essas premonições para a atividade econômica cotidiana.
 {\color{blue} 14}  

 
\par
 
Os empreendedores foram os meios sublimes desse “conhecimento tácito”, canalizando as suas verdades profundas para outros intervenientes no mercado. Eram heróis românticos possuídos por lampejos de visão quase poética. “Visão ou percepção empreendedora”, explicou Gray, não era uma questão de aprendizado de livro, mas de “acaso e talento”. Foi “uma atividade criativa insuscetível de formulação em regras rígidas e rápidas”. Situada “além dos nossos poderes de controlo consciente”, a “percepção empreendedora” apareceu apenas raramente, atingindo-a subitamente e sem aviso.
 {\color{blue} 15}  
Quando apareceu, reordenou o universo.
 
\par
 
O mercado, em suma, proporcionou um refúgio para a autoexpressão e a criatividade, um santuário para o contra-iluminismo arrebatador. Os escritores sem imaginação contentaram-se em argumentar que os mercados “alocam os recursos escassos de forma mais eficiente” ou que o mercado “permite a motivação do interesse próprio”. Mas tais defesas ignoravam uma verdade mais elementar: os mercados permitiam a expressão de “toda a variedade de motivos humanos, em toda a sua complexidade e misturas”.
 {\color{blue} 16}  
O mercado fornecia um teatro para a autorrevelação dramática, um palco onde os indivíduos podiam projetar as suas visões mais irreprimíveis e os seus desejos mais vigorosos.
 
\par
 
Todos os casos amorosos chegam ao fim, mas o rompimento de Gray com o mercado foi particularmente venenoso. Ele agora denuncia isso como o flagelo da civilização. Nos Estados Unidos, escreve ele, o mercado livre “gerou um longo boom econômico do qual a maioria dos americanos quase não beneficiou”. Os americanos sofrem com “níveis de desigualdade” que “se assemelham aos dos países latino-americanos”. A classe média desfruta dos encantos duvidosos da “insegurança econômica sem bens que afligiu o proletariado do século XIX”. Os Estados Unidos estão perigosamente perto de uma perturbação social massiva, que só foi controlada “por uma política de encarceramento em massa” de afro-americanos e outras pessoas de cor. “O profeta da América de hoje”, afirma Gray, “não é Jefferson ou Madison. . . . É Jeremy Bentham” – o homem que sonhou com uma sociedade “reconstruída segundo o modelo de uma prisão ideal”.
 {\color{blue} 17}  

 
\par
 
Ainda mais terrível, escreve Gray, é que as elites globais têm procurado fazer do capitalismo americano o modelo para o mundo. Embora os regimes de mercado variem consoante a cultura e o país, os sumos sacerdotes da globalização impõem um modelo americano que sirva para todos – com o seu estado social mínimo, regulamentações empresariais e ambientais fracas e impostos baixos. “Conforme o ‘Consenso de Washington’”, escreve Gray, “as múltiplas culturas e sistemas econômicos que o mundo sempre conteve serão redundantes. Eles serão fundidos num único mercado livre universal” baseado no “último grande regime iluminista do mundo, os Estados Unidos”.
 {\color{blue} 18}  

 
\par
 
Quando Gray pronunciou pela primeira vez estas heresias, muitos dos seus amigos conservadores ficaram chocados. Assim como Gray, Norman Barry é um teórico político que escreveu sobre Hayek. Professor da Universidade de Buckingham, a única universidade totalmente privada da Grã-Bretanha, ele foi o padrinho do segundo casamento de Gray, mas agora raramente fala com ele. Barry não consegue afastar a suspeita de que a viragem política de Gray foi motivada por puro oportunismo. “Acredito numa proposta da economia neoclássica: todos são maximizadores de utilidade”, explica ele. “Pode ter sido um bom passo na carreira se distanciar do libertarianismo. Estou especulando, mas não descontroladamente. Os libertários não conseguem as melhores posições nas universidades.” Quando Gray era apenas um membro de uma pequena faculdade de Oxford, afirma Barry, “ele costumava dizer: ‘Bem, da maneira como o mundo funciona, eu não conseguiria uma cadeira’. . . Você não consegue um cargo de professor na LSE se for um fanático pelo livre mercado.” A única continuidade na posição de Gray que Barry reconhece é a sua propensão para a “promiscuidade filosófica”. Gray, diz Barry, “estava sempre passando de pessoa para pessoa, de filósofo para filósofo. . . . Ele não conseguia formar um relacionamento estável com nenhum pensador. Ele experimentou um pouco de Popper. Tentei Hayek. Claro, mais tarde ele largou Hayek. Ele tentaria descartar outros escritores. Gray afirma que mudou de ideia por dois motivos. Durante o final da década de 1980, diz ele, começou a suspeitar que o pensamento político da direita tinha um fim rígido numa ideologia obsoleta – não muito diferente do monótono banezianismo do qual ele fugiu há tanto tempo. Gray já havia pensado no Thatchprisma como taticamente flexível e politicamente experiente, um movimento sensível aos sentimentos populares, cujo líder era um virtuoso maquiavélico da mudança política. Mas ele agora acreditava que o movimento havia perdido o seu talento artístico; o pensamento flexível degenerou em encantamento mecânico. Gray diz: “O que foi surpreendente no bolchevismo foi que Lenin era tão extraordinariamente flexível. Então endureceu no trotskismo. E da mesma forma o fletcherismo começou a endurecer. . . . Era um hábito de pensamento que considerei profundamente repugnante.” O colapso da União Soviética também forçou Gray a questionar a sua fé no mercado livre. Até 1989, diz Gray, fazia sentido pensar no Estado como “o principal inimigo do bem-estar”, que era a atitude dentro da “atmosfera reconhecidamente de estufa da direita, pense obrigado”. Mas depois da queda do império soviético, da ex-Iugoslávia mergulhar numa guerra civil genocida e dos defensores do livre mercado ocidentais aplicarem terapia de choque aos países anteriormente comunistas com resultados desastrosos, gray passou a pensar que o Estado era um mal necessário, talvez até um bem positivo. Foi a única força que poderia impedir que as sociedades caíssem no caos total, na desigualdade extrema e na pobreza.
 
\par
 
Mas há uma razão mais profunda para a mudança de Gray: por si só, o mercado não conseguiu sustentar a sua afeição. Sem a União Soviética e o Estado-providência como símbolos divertidos do racionalismo iluminista, gray já não podia acreditar no mercado como antes. O mercado, ele agora tinha que admitir, patrocina um “culto à razão e à Effie – principalmente”. Ele “rompe os fios da memória e espalha o conhecimento local”. Ele costumava pensar que o mercado livre surgiu espontaneamente e que o controle estatal da economia não era natural. Mas observando Jeffrey Sachs e o Fundo Monetário Internacional na Rússia, ele não pôde deixar de ver o mercado livre como “um produto de artifício, desígnio e coerção política”. O mercado teve de ser criado, muitas vezes com a ajuda do poder estatal implacável. Hoje, ele argumenta que Thatcher construiu o mercado livre esmagando os sindicatos, esvaziando o Partido Conservador e incapacitando o Parlamento. Ela “colocou a sociedade britânica numa marcha forçada para a modernidade tardia”. Gray acredita que “o marxismo-leninismo e o racionalismo econômico de livre mercado têm muito em comum”. Ambos, escreve ele, “demonstram pouca simpatia pelas vítimas do progresso econômico”.
 {\color{blue} 19}  
Há apenas uma diferença: o comunismo está morto.
 
\par
 
Em um momento de descuido, Norman Barry confessa que não consegue imaginar a mudança de Gray. “Talvez eu o tenha entendido mal”, diz ele, “mas pensei que ele acreditava profundamente. De qualquer forma, ninguém poderia ter lido tanta coisa sem acreditar em alguma coisa. Eu me pergunto se ele alguma vez fez isso. Gray acreditava, mas sua crença era diferente da de Barry. Barry adora o mercado porque ele opera de acordo com “as leis férreas da economia”. Como ele diz, estas podem “levar um pouco mais de tempo do que as leis newtonianas. Se eu deixar cair este disco, ele cairá em um segundo. Se eu introduzir o controle dos aluguéis, levaria talvez seis meses para criar a situação de sem-teto.” Mas, acrescenta, “é igualmente decisivo”. Em contraste, gray já acreditou no capitalismo precisamente porque procurava escapar às leis de Newton. Tendo percebido que o mercado inibe a autoexpressão apaixonada, Gray foi forçado a reconhecer a verdade da máxima de Irving Bristol: “O capitalismo é a concepção menos romântica de uma ordem pública que a mente humana alguma vez concebeu”.
 {\color{blue} 20}  

 
\par
 
Quando Edward Cutaway tinha quarenta e poucos anos, já havia ultrapassado os nazistas, escapado dos comunistas e sido alvejado por guerrilheiros esquerdistas na América Central. Mas até hoje ele se lembra da mudança de infância de Palermo para Milão como o acontecimento mais “traumático” de sua vida. Nascido em 1942 numa rica família judia na Romênia, Cutaway cresceu no sul da Transilvânia, que foi brevemente ocupada pelos nazis em 1944. Quando tinha cinco anos, a sua família fugiu de uma iminente tomada comunista e estabeleceu-se em Palermo. Era inverno, lembra Cutaway, e “Paris e Londres tremiam. Houve falta de combustível. Milano estava tremendo. As coisas estavam muito sombrias. Mas em Palermo “a ópera estava em pleno andamento”. Era “a terra das laranjas e dos limões”, diz ele, onde as pessoas podiam nadar e esquiar quase o ano todo. Cinco anos depois, a família de Cutaway mudou-se novamente, desta vez para Milão, o centro industrial da Itália. “Entupido e cheio de neblina”, Milan deixou Cutaway infeliz. “Não havia onde brincar. Os parques eram uma vergonha. Perdi todos os meus amigos de Palermo. Eu me achei. . . No meio de um bando de garotos muito burgueses.” A boa vida no Mediterrâneo havia chegado ao fim, destruída pelos severos industriais do norte.
 
\par
 
Durante a maioria de sua vida adulta, Cutaway travou uma luta militante contra o comunismo. Inspirado por uma visão militar estratégica que ligava as Guerras Gálicas às guerras civis da América Central, trabalhou em estreita colaboração com o Departamento de Defesa dos EUA como consultor, aconselhando todos, desde oficiais subalternos até ao alto escalão. Mas Cutaway era mais do que um guerreiro frio. Ele era um guerreiro, ou pelo menos um fervoroso teórico da “arte da guerra”. Enquanto os generais pensavam que a vitória dependia da imitação dos estilos de gestão da IBM, Cutaway defendeu antigas táticas de campo de batalha e manobras esquecidas do Império Romano. Cutaway instou os militares a buscarem orientação em Adriano, e não em Henry Ford. Foi uma luta árdua, com os oficiais agindo mais frequentemente como homens de organização do que como soldados. Mais uma vez, Cutaway viu o seu modo de vida preferido ameaçado pela cultura do capitalismo.
 
\par
 
Cutaway ganhou notoriedade pela primeira vez na Grã-Bretanha, onde se estabeleceu após receber seu diploma de graduação em economia na London School of Economics. Em 1968, ele publicou Coup d’Eat: A Practical Handbook. O autor de 26 anos deslumbrou seus leitores com este audacioso guia prático, levando um encantado John le Carré a escrever: “Sr. Cutaway compôs um guia gastronômico profano sobre o veneno político. Aqueles que forem corajosos o suficiente para olhar para sua cozinha nunca mais comerão tão pacificamente.” Em 1970, Cutaway publicou um artigo igualmente malicioso na Esquire, “A Scenario for a Military Coup d’Eat in lhe United States”. Dois anos depois, mudou-se para os Estados Unidos para escrever uma dissertação em ciência política e história clássica na Johns Hopkins, conduzindo uma extensa pesquisa usando fontes originais em latim, alemão, francês, inglês e italiano. O resultado foi que elogiaram amplamente a Grande Estratégia do Império Romano. Enquanto estava na pós-graduação, Cutaway começou a trabalhar como consultor para vários ramos das forças armadas dos EUA, fazendo recomendações sobre tudo, desde como a OTAN deveria conduzir manobras táticas até que tipo de rifle os soldados do exército salvadorenho deveriam portar.
 
\par
 
Quando Ronald Reagan concorreu à presidência em 1980, Cutaway estava no topo do jogo. Membro do Centro de Estudos Estratégicos e Internacionais de Georgetown e colaborador frequente da Commentary, argumentou que os Estados Unidos deveriam acelerar a corrida armamentista de alta tecnologia, forçando a União Soviética a participar numa competição que não poderia vencer. Os conselheiros mais próximos de Reagan acolheram Cutaway com entusiasmo no seu círculo íntimo. Logo após a eleição de Reagan, Cutaway participou num jantar em Bethesda, juntamente com Jean Kirkpatrick, Fred Idle e outros luminares do establishment republicano da defesa. Richard Allen, que se tornaria o primeiro conselheiro de segurança nacional de Reagan, agitou a multidão, fingindo distribuir cargos na administração como se fossem favores partidários. Como noticiou o Washington Post, Cutaway recusou, explicando sobre a torta de chocolate Tia Maria: “Não acredito que escribas como eu devam se envolver em política. É como caviar. Ótimo, mas apenas em pequenas quantidades.” Quando pressionado por Allen, ele brincou: “Só quero ser vice-cônsul em Florença”. Allen respondeu: “Você não quer dizer procônsul?”
 {\color{blue} 21}  

 
\par
 
A bonomia dos gladiadores da escola preparatória evaporou antes do final do primeiro mandato de Reagan. Cutaway pode ter sido um trunfo inestimável ao pressionar por mais gastos com defesa, mas fez inimigos com as suas críticas ruidosas – e cada vez mais sarcásticas – à má gestão do Pentágono. Em 1984, publicou O Pentágono e a Arte da Guerra, onde, entre outras coisas, retratou o secretário da Defesa, Caspar Weinberger, mais como um astuto vendedor de carros usados ​​do que como um verdadeiro estadista. Os políticos militares contra-atacaram, retirando Cutaway de uma lista de consultores para o bem público do Pentágono (ele continuou a fazer trabalho contratado noutros locais do sistema de defesa). Em 1986, Weinberger explicou ao Los Angeles Times que Cutaway “acabou de perder cargos de consultoria por total incompetência, só isso”.
 {\color{blue} 22}  

 
\par
 
Mas foram mais do que as críticas de Cutaway a Weinberger que o colocaram em problemas com o Departamento de Defesa. O seu verdadeiro erro em O Pentágono e a Arte da Guerra foi perseguir a conduta dos militares durante a Guerra do Vietnã. Cutaway minimizou as explicações favoritas das forças armadas para a sua derrota no Vietnã – políticos de vontade fraca, a imprensa traiçoeira, um público derrotista. Em vez disso, argumentou que a elite guerreira da América tinha simplesmente perdido o gosto pelo sangue. Durante a Guerra do Vietnã, escreveu ele, “oficiais de escritório” estavam sempre “longe do combate”. A sua propensão para o “luxo absoluto” teve um efeito devastador sobre o moral das tropas. Embora Júlio César “mantivesse concubinas e calamidades em seu quartel-general da retaguarda, comesse em pratos de ouro e bebesse seu vinho Damião em taças de joias”, quando estava na linha de frente com seus soldados, ele “comia apenas o que eles comiam, e dormiam como eles - debaixo de uma tenda, se as tropas tivessem tendas, ou simplesmente enroladas em um cobertor, se não as tivessem. Em contraste, os oficiais americanos recusaram “partilhar as dificuldades e os riscos mortais da guerra”.
 {\color{blue} 23}  

 
\par
 
Burocratas de cabeça pontiaguda também minaram a força dos militares, de acordo com Cutaway. Sempre procurando cortar custos, os responsáveis ​​do Pentágono insistiram que as armas, a maquinaria e os programas de investigação e desenvolvimento fossem padronizados. Mas isto apenas tornou os militares vulneráveis ​​ao ataque inimigo. Os sistemas de armas padronizados foram facilmente superados; tendo subjugado um, um inimigo poderia subjugar todos eles. No que diz respeito aos militares, concluiu Cutaway, “precisamos de mais ‘fraude, desperdício e má gestão’”.
 {\color{blue} 24}  

 
\par
 
Os principais generais eram obcecados pela eficiência, em parte porque aprenderam os métodos de gestão empresarial em vez da arte da guerra. Para cada oficial formado em história militar, havia mais cem “cuja maior realização pessoal é uma pós-graduação em administração de empresas, gestão ou economia”. “Por que deveriam os pilotos de caça receber uma educação universitária completa”, perguntou Cutaway no Washington Quarterly, “em vez de serem ensinados a caçar e matar com as suas máquinas?”
 {\color{blue} 25}  

 
\par
 
A fonte final da disfunção militar foi sua adoção da cultura corporativa e dos valores empresariais americanos. Como Robert McNamara, que o presidente Kennedy transferiu da Ford Motor Company para o Pentágono, a maioria dos secretários de defesa estava cativada por "objetivos de estilo corporativo". Eles buscavam os meios menos arriscados e mais econômicos para um determinado fim. Eles preferiam ternos cinza, evitando "excentricidades pessoais em vestimenta, discurso, maneiras e estilo porque qualquer característica incomum pode irritar um cliente ou um banqueiro nos encontros casuais comuns nos negócios". Os oficiais eram apenas "gerentes uniformizados", disse Cutaway à Forbes. Mas, ele observou, "o que é bom para os negócios não é bom para conflitos mortais". Embora "vestimentas conservadoras e estilo inofensivamente convencional" pudessem funcionar em um escritório, eles podiam ser mortais no campo de batalha; eles sufocavam iniciativas ousadas e gênios idiossincráticos.
 {\color{blue} 26}  
Insinuando que o capitalismo tinha colonizado – na verdade, destruído – esferas da sociedade que não eram estritamente econômicas, Cutaway esteve perigosamente perto de se identificar com as principais vozes da tradição marxista – Jürgen Habermas, Georg Lukács e até o próprio Marx.
 
\par
 
Enquanto a União Soviética ainda existia, Cutaway conseguiu canalizar o seu desprezo pelos valores empresariais e empresariais em propostas de reforma militar. A luta contra o bolchevismo capturou totalmente a sua imaginação, falando de princípios de individualismo, independência e dignidade pessoal que ele aprendeu quando era filho de judeus ateus. Os pais de Cutaway lhe ensinaram, diz ele, que “você queria ter os ombros abertos andando pela rua. O mestre do seu destino. Não andar curvado, com medo de que Deus te castigue se você comer um sanduíche de presunto.” Ele continua: “Havia um certo desprezo pela piedade. A piedade não era vista como compatível com a dignidade.” Dignidade, prossegue ele, “é o que defendíamos na Guerra Fria. Foi ideológico. Foi muito apropriado para mim estar nos Estados Unidos, tornar-me americano, porque os americanos eram e são o povo ideológico. Foram perfeitamente escolhidos para serem alistados numa luta ideológica.” Mas agora que a batalha contra o comunismo foi vencida, Cutaway perdeu o interesse na maioria dos assuntos militares; ele não vê mais nenhuma razão ideológica convincente para se preocupar com estratégia e tática. “Os problemas de segurança e outros tornaram-se periféricos, para todos os países e para as pessoas, e para mim também. Não envolvo minha existência em algo periférico. . . . Havia um imperativo de estar envolvido. Agora não existe.” Cutaway ocasionalmente reúne energia para um projeto específico. Durante uma de nossas entrevistas, ele fala por telefone com um funcionário do Departamento de Estado sobre fazer um trabalho de consultoria para a guerra contra as guerrilhas colombianas. Mas quando lhe pergunto se vale a pena defender o governo colombiano, ele mostra-se estranhamente hesitante, confessando finalmente: “Não sei se vale a pena defender alguma coisa, mas penso que vale a pena lutar contra as guerrilhas”. Eu pergunto por quê, e ele responde que as guerrilhas estão alinhadas com os traficantes de drogas que “fazem de tudo, desde ocupar os lugares das pessoas nos restaurantes de Medellín num sábado à noite – as pessoas estão esperando para ocupar seus lugares e esses caras entram e pegam suas mesas”. — tudo, desde isso até ao assassinato.” A luta militar pode já não ter qualquer encanto ideológico para Cutaway, mas o seu descontentamento proporciona-lhe o tempo e o espaço intelectual para confrontar o inimigo que tem perseguido durante toda a sua vida: o próprio capitalismo. “O mercado”, diz ele, “invade todas as esferas da vida”, produzindo uma “sociedade infernal”. Da mesma forma que os valores de mercado antes ameaçavam a segurança nacional, ameaçam agora o bem-estar econômico e espiritual da sociedade. “Um sistema de produção ideal é um sistema de produção completamente desumano”, explica ele, “por quê. . . Você está constantemente mudando o número de pessoas que emprega, você as movimenta, você faz coisas diferentes, e isso não é compatível com alguém ser capaz de organizar uma existência para si.” Embora Cutaway escreva em seu livro de 1999, Turbocapitalism, “Eu acredito profundamente. . . Nas virtudes do capitalismo”, a sua oposição à difusão dos valores de mercado é tão aguda que o coloca no extremo oposto do espectro político atual – uma posição de que Cutaway desfruta congenialmente.
 {\color{blue} 27}  
“Edward é um sujeito muito perverso, intelectualmente e de muitas outras maneiras”, diz o ex-editor do Commentary Norman Podhoretz, um dos primeiros defensores do Cutaway durante a década de 1970. “Ele é um opositor. Ele gosta de confundir expectativas. Mas eu, francamente, nem sei o quão sério ele é nesta última encarnação.” Cutaway insiste que ele é bem sério. Ele clama por medicina socializada. Ele defende um forte estado de bem-estar social, alegando: “Se eu tivesse escolha, proibiria qualquer forma de caridade doméstica.” Caridade é uma “saída fácil”, ele diz: tira a dignidade dos pobres.
 
\par
 
A única coisa que desperta a ira de Cutaway mais do que o capitalismo desenfreado são os seus entusiastas de elite – os intelectuais, políticos, decisores políticos e empresários que afirmam que “só porque o mercado é sempre mais eficiente, o mercado deve sempre governar”. Alan Greenspan merece o desprezo especial de Cutaway: “Alan Greenspan é um spenceriano. Isso faz dele um fascista econômico.” Pessoas de Spencerian, como Greenspan, acreditam que “as pressões econômicas mais severas” irão “estimular algumas pessoas a fazê-lo. . . Atos economicamente heroicos. Eles se tornarão grandes empreendedores ou qualquer outra coisa, e quanto aos que fracassarem, deixem-nos fracassar.” A outra bête noire de Cutaway é “Chainsaw Al” Dunlap, o CEO peripatético que colhe retornos inimagináveis ​​para os acionistas corporativos ao contratar um número substancial de funcionários das empresas. “Chainssaw faz isso”, diz Cutaway, referindo-se às medidas de redução de pessoal de Dunlap, “porque ele é simplório, duro e cruel”. É apenas “sadismo econômico”. Contra Greenspan e Dunlap, Cutaway afirma: “Acredito que se deve ter apenas a eficiência de mercado necessária, porque tudo o que valorizamos na vida humana está dentro do reino da ineficiência – amor, família, apego, comunidade, cultura, velhos hábitos, sapatos velhos e confortáveis”. As deserções de Cutaway e Gray sugerem quão cruel foi o fim da Guerra Fria para o movimento conservador. É cada vez mais claro que a frágil coligação de libertários, tradicionalistas e entusiastas do mercado livre, outrora mantida unida pela cola do anticomunismo, já não irá aderir. O fim da União Soviética “privou-nos de um inimigo”, diz-me Irving Bristol, o padrinho intelectual do neoconservadorismo. “Na política, ser privado de um inimigo é um assunto muito sério. Você tende a ficar relaxado e desanimado. Volte-se para dentro.” Famoso pela sua autoconfiança, Bristol confessa agora uma triste perplexidade no mundo pós-comunista. “Essa é uma das razões pelas quais não escrevo muito atualmente”, diz ele. “Não sei as respostas.” Poderíamos pensar que o triunfo do mercado livre emocionaria os intelectuais de direita. Mas mesmo os mais venerados patriarcas conservadores preocupam-se com o fato de o mercado, por si só, não conseguir sustentar as energias de declínio do movimento. Afinal, Reagan e Thatcher convocaram os conservadores para uma cruzada política, mas a ideologia do mercado livre que desencadearam é suspeita de todas as crenças políticas. A lógica do mercado glorifica a iniciativa privada, a ação individual, o brilho do não planejado e do aleatório. Neste contexto, é difícil pensar em política – muito menos em transformação política. William F. Buckley Jr. me diz: “O problema com a ênfase no conservadorismo no mercado é que isso se torna um tanto enfadonho. Você ouve uma vez e domina a ideia. A ideia de dedicar sua vida a isso é horrível, apenas porque é muito repetitivo. É como sexo. Bristol acrescenta: “O conservadorismo americano carece de imaginação política. É tão influenciado pela cultura empresarial e pelos modos de pensar empresariais que lhe falta qualquer imaginação política, o que sempre foi, devo dizer, uma propriedade da esquerda.” Ele prossegue: “Se lermos Marx, aprenderemos o que uma imaginação política pode fazer”. Mas se os conservadores estão a lutar para encontrar uma visão, poderão os ex-conservadores fazer muito melhor? Ao contrário de Bristol, que fugiu da esquerda e lançou o movimento neoconservador, Cutaway e Gray não formularam alternativas coerentes, filosóficas ou políticas, aos seus antigos credos. Como diz Cutaway: “Em vez de propor toda uma contra-ideologia, o que proponho simplesmente é que a sociedade diga conscientemente que certas coisas devem ser protegidas do mercado e mantidas fora do mercado”. Isto, apesar do fato de Cutaway permanecer temperamentalmente apaixonado, à sua maneira, pelo impulso revolucionário. “Prefiro ‘A Marselhesa’ à Missa”, diz ele, “Mayaski à cruz de São Jorge”. Ele acrescenta: “As revoluções são maravilhosas. Pessoas se divertindo. Estive em Paris em 1968. . . . Havia uma sensação maravilhosa de possibilidade.” Mas embora Cutaway possa desejar uma política transformadora, ela permanece fora do seu alcance, um objeto de nostalgia não só para ele, mas para a maioria dos intelectuais.
 
\par
 
Exceto, ao que parece, William F. Buckley Jr., o vai boy original da direita americana. No final da nossa entrevista, peço a Buckley que imagine uma versão mais jovem de si, um aspirante a entanto terrível político que se formou na faculdade em 2000, trazendo para o mundo político de hoje o mesmo espírito insurgente que Buckley trouxe para o seu. Que tipo de política esse jovem Buckley adotaria? “Eu seria um socialista”, ele responde. “Um socialista Mike Harrington.” Ele faz uma pausa. “Eu diria até um comunista.” Será que ele consegue realmente imaginar um jovem comunista, Bill Buckley? Ele admite que é difícil. O Bill Buckley original tinha o benefício da União Soviética como inimiga; sem o seu equivalente, o seu doppelgänger enfrentaria uma tarefa mais complicada. “Esse novo Buckley teria que apontar para outras coisas”, diz ele. Buckley apresenta uma longa lista de causas de esquerda – pobreza global, morte por SIDA. Mas até ele parece repentinamente oprimido pelo projeto de (no típico Buckeyes) “juntar tudo isso em um AF-flavus impressionante”. Assustado pelo desafio de pensar fora do mercado livre, Buckley faz uma pausa e finalmente diz: “Vou deixar isso para você”.
 
\par
  
 
999999
