\chapter{5 Reprodução Econômica}\label{5 Reprodução Econômica}
 \par 
O capítulo anterior examinou um circuito único do capital industrial. Para o capital como um todo, existem muitos circuitos diferentes, cada um movendo-se ao seu próprio ritmo e cada um expandindo-se ao seu próprio ritmo, e estes circuitos devem ser integrados entre si. Marx analisa esses processos no Volume {\color{blue}2} de O Capital dividindo a economia em dois grandes setores, o departamento 1, que produz meios de produção (MP, adquiridos com capital constante, c) e o departamento 2, que produz meios de consumo (adquiridos por trabalhadores fora do equivalente salarial do capital variável, v, e pelos capitalistas da mais-valia, s). Este capítulo examina o processo de reprodução do capital como um todo. Começa com a reprodução simples, onde não há acumulação de capital. Posteriormente, examina a reprodução ampliada, onde se investe parte da mais-valia. Por fim, considera a reprodução social da economia capitalista.
 \par 
\section{Reprodução Simples}
 \par 
Na Figura {\color{blue}5}. {\color{blue} 1 } {\par} , o equilíbrio entre os departamentos {\color{blue}1} e {\color{blue}2} em condições de reprodução simples é ilustrado por um diagrama de fluxo que mostra valores e mercadorias de cada sector e dinheiro. Os dois circuitos são mostrados, M1 - C1… P1 … C'1 - M'1, e M2 - C2… P2 … C'2 - M'2 (com M'1 e M'2 sendo absorvidos no pool central de dinheiro, M, após o qual eles fluem novamente). A figura também mostra os fluxos de mercadorias. Estes vão na direção oposta ao dinheiro que é usado para comprá-los, com trabalhadores e
 \par 
\section{Reprodução Simples}
 \par 
\section{Reprodução Simples}
 \par 
TTTTTT capitalistas comprando bens de consumo do departamento {\color{blue}2} com seus salários, v1 e v2, e mais-valia, s1 e s2, e capitalistas comprando meios de produção, c1 e c2, do departamento {\color{blue}1} (os trabalhadores não compram meios de produção, e nós ignoramos poupança).
 \par 
Se não houver mudança técnica, e se os capitalistas gastarem todo o seu valor excedente em consumo e meramente repetirem o padrão anterior de produção, a economia pode se reproduzir no mesmo nível de atividade. Isso é o que Marx chama de reprodução simples, que implica um certo equilíbrio entre os valores produzidos pelos dois departamentos. O valor da produção do departamento {\color{blue}1} é  $$ c_{1} + v_{1} + s_{1},$$   e o valor de suas vendas de meios de produção é  $$ c_{1} + v_{1} $$   Então, na reprodução simples:
 \par 
O valor TTTTTT das vendas de meios de consumo dá:
 \par 
\section{Reprodução Simples}
 \par 
Esta é a famosa equação de Marx para o equilíbrio entre os dois departamentos na reprodução simples.
 \par 
\section{Reprodução Simples}
 \par 
Se os capitalistas não consumirem a totalidade da sua mais-valia, mas gastarem parte dela na compra de meios de produção adicionais, ocorre a acumulação de capital. Neste caso, as compras de meios de produção pelos capitalistas,  $$ c_{1} + v_{1} + s_{1},$$   para o próximo período excedem o uso corrente,  $$ c_{1} + v_{1} $$   Segue-se que, para reprodução expandida,  $$ c_{1} + v_{1} + s_{1} > c_{1} + c_{2}:$$   com a extensão da desigualdade dependendo da taxa de acumulação.
 \par 
O esquema de reprodução de Marx foi interpretado de várias maneiras. Uma das mais populares é que oferecem uma análise das condições de equilíbrio econômico, seja ela estática (no caso da reprodução simples) ou dinâmica (reprodução ampliada). Alternativamente, tomando como modelo a teoria dominante do crescimento, a reprodução expandida é vista simplesmente como uma versão ampliada da reprodução simples. A economia parece a mesma em todos os aspectos, exceto que é maior.
 \par 
Nenhuma das interpretações está dentro do espírito da análise de Marx. Em primeiro lugar, a sua metodologia opõe-se fortemente à utilização do equilíbrio como conceito organizador para a análise do capitalismo. Em segundo lugar, no esquema de reprodução, Marx está preocupado em mostrar como, apesar da coordenação aparentemente caótica dos produtores na troca, tanto a reprodução simples como a expandida existem dentro do sistema capitalista. Em outras palavras, a reprodução simples e ampliada não são alternativas, nem teórica nem empiricamente. Pelo contrário, a primeira existe dentro da segunda: a reprodução expandida depende simultaneamente e rompe com as condições associadas à reprodução simples que são o seu ponto de partida - tanto nas magnitudes de valor agregado, como nos valores das próprias mercadorias, uma vez que estas estão sujeitas ao aumento da produtividade. como resultado da acumulação. Além disso, Marx nunca tira a implicação, como na teoria do equilíbrio geral ou para os proponentes do laissez-faire, de que diferentes produtores e consumidores são harmoniosamente coordenados através do mercado em elevados níveis de emprego de recursos. Pelo contrário, o esquema de Marx aponta para três equilíbrios distintos exigidos pela reprodução e acumulação de capital.
 \par 
A primeira está em valores, como foi ilustrado acima. A segunda diz respeito à moeda e aos preços, uma vez que o circuito exige que estes se equilibrem à medida que os fluxos ocorrem. E, terceiro, a necessidade de equilíbrio aplica-se também aos valores de uso, pois as quantidades apropriadas de mercadorias têm de ser produzidas e trocadas entre si, tanto dentro como entre os dois departamentos.
 \par 
De acordo com o esquema acima, as quantidades de valor exibidas têm uma relação quantitativa não especificada com os valores de uso envolvidos. No entanto, eles não podem ser totalmente independentes um do outro. Dados os fluxos de valor, o crescimento assimétrico da produtividade, por exemplo, levaria à transferência de recursos entre os dois departamentos e a uma alteração dos fluxos de valor de uso entre eles. Portanto, o equilíbrio de Marx entre os dois departamentos deve envolver não apenas a coordenação em toda a economia entre os fluxos de valor já especificados, juntamente com os fluxos complementares de dinheiro, cujas magnitudes são determinadas pelo sistema de preços, mas também o equilíbrio dos fluxos de dinheiro. use valores determinados pela tecnologia, pela composição do resultado e assim por diante.
 \par 
O diagrama de reprodução econômica na Figura {\color{blue}5}.1 pode ser usado para reforçar as visões parciais da economia que foram apresentadas à luz do circuito único do capital no capítulo anterior. Pouco é adicionado qualitativamente, mas a figura sugere o que pode ser considerado como os fatores que determinam o nível de atividade econômica. Observe, primeiro, que a teoria econômica e a ideologia tradicionais tendem a se concentrar na "caixa" central da atividade de troca, em relação à qual as duas esferas de produção parecem ser estranhas. Geralmente, isso apoia a visão errônea de que a produção pode ser tomada como certa, ou que é simplesmente uma relação técnica que forma a base não problemática para as relações de troca, como na função de produção neoclássica.
 \par 
Isto é mais evidente na teoria neoclássica do equilíbrio geral, onde as trocas de “mercado livre” são consideradas suficientes para garantir a igualdade da oferta e da procura no pleno emprego dos recursos económicos. E, na análise de estabilidade, torna-se uma questão de saber se a desproporção entre as várias quantidades incorporadas nos circuitos é autocorrigível através de movimentos de preços em resposta ao excesso de oferta e procura.
 \par 
Para a teoria keynesiana, o papel da demanda agregada se torna determinante. Se nos concentrarmos no multiplicador de investimento, o nível de  $$ c_{1} + c_{2} $$  assume um papel central. Se também incluirmos o papel do consumo, então essa despesa fora da renda nacional (v1 +  $$ v_{2} + s_{1} $$  + s2) também se torna importante. Nessa forma, a função de consumo tem mais afinidade com os métodos pós-keynesianos de determinação da demanda agregada, nos quais a renda é dividida em salários e lucros. Mas o ponto importante continua sendo que, dessas perspectivas, um conjunto particular de fluxos de despesas impulsiona a atividade econômica agregada. No entanto, nessas abordagens há
 \par 
Uma economia pós-keynesiana mais sofisticada inclui o papel do dinheiro. Nesta abordagem, o nível de actividade económica é determinado pela dimensão dos fluxos de dinheiro que saem do pool central, M. Se estes forem restringidos, seja por causa da timidez empresarial ou através de políticas monetárias contraccionistas impostas pelo banco central, o a economia vacilará. Os papéis do sistema bancário e da taxa de juros são abordados no Capítulo {\color{blue}12}. Aqui é importante notar que, deste ponto de vista, a fonte do desemprego encontra-se na atividade cambial insuficiente, que determina o (em )capacidade da economia para gerar rentabilidade. Na própria teoria de Keynes, isto depende em grande parte de ondas de pessimismo, nas quais as fracas expectativas sobre a rentabilidade das empresas (e as expectativas de taxas de juro elevadas) tornam-se profecias auto-realizáveis. De um modo mais geral, os desenvolvimentos recentes no âmbito da teoria económica dominante conferiram às expectativas (chamadas “racionais”) um papel consideravelmente reforçado na determinação da trajetória da economia.
 \par 
Finalmente, uma teoria mais radical da economia vê o nível de actividade económica como sendo determinado pelas relações de distribuição entre capital e trabalho. Esta visão está associada ideologicamente tanto à direita como à esquerda, com a primeira a argumentar que o poder dos sindicatos precisa de ser restringido para restaurar a rentabilidade, e a última a argumentar que os conflitos envolvidos são irreconciliáveis ​​dentro dos limites do capitalismo. Analiticamente, esta perspectiva depende de uma compreensão da economia como um “bolo fixo”, em que o rendimento nacional  $$ v_{1} + v_{2} + s_{1} + s_{2} $$  é dividido entre as duas classes, com uma ganhando apenas à custa da outra. Por exemplo, se os salários, representados por  $$ v_{1} + v_{2} $$   subirem demasiado, então os lucros, representados por  $$ s_{1} + s_{2} $$   deverão cair, e isto mina tanto a motivação como a capacidade de acumular.
 \par 
Apesar da sua aparência “radical”, esta visão diverge acentuadamente da apresentação do próprio Marx da estrutura da economia capitalista.
 \par 
A atribuição de um papel central à distribuição na determinação da rentabilidade só é possível confinando a análise (a uma parte da) arena da troca. Uma vez incorporada também a esfera da produção, a aparente simetria entre capital e trabalho, nas relações distributivas e no recebimento de lucros e salários a partir da renda nacional, evapora; pois o pagamento de salários é uma pré-condição para o início do processo de produção (ou, mais exactamente, isto é verdade para a compra de força de trabalho, cujo pagamento efectivo pode muito bem ocorrer mais tarde). Em contraste, os lucros são o resíduo após o pagamento dos salários e outros custos de produção, em vez de serem uma “fatia do bolo” com uma dimensão que pode ser negociada antecipadamente. Para Marx, as relações de distribuição entre capital e trabalho não são do tipo bolo fixo, mesmo que, ceteris paribus, os lucros sejam mais elevados se os salários forem mais baixos (embora os pós-keynesianos possam argumentar o contrário, tendo em conta a procura inadequada). Os lucros dependem, em primeiro lugar, da capacidade dos capitalistas de extrair mais-valia na produção: seja qual for o nível de salários, os capitalistas precisam de coagir o trabalho a trabalhar para além do tempo de trabalho necessário para produzir esses salários, com uma produtividade que depende tanto de a maquinaria disponível e o nível de disciplina no local de trabalho.
 \par 
A incerteza sobre a produção de mais-valia é apenas um dos aspectos da incerteza que os capitalistas enfrentam. Quatro outros tipos de incerteza também são relevantes. Primeiro, tendo produzido mais-valia, os capitalistas não têm certeza sobre quanto pode ser realizado até que a produção seja vendida. Em segundo lugar, a extracção de mais-valia em condições competitivas conduz a mudanças técnicas contínuas que aumentam a produtividade. Contudo, foi demonstrado acima que a mudança técnica perturba os equilíbrios de valor e valor de uso na economia (e pode contribuir para relações antagónicas no chão de fábrica), aumentando ainda mais a incerteza. Terceiro, como é mostrado nos Capítulos {\color{blue}12} e 14, o crédito disponibiliza os recursos de todo o sistema financeiro para os capitalistas individuais, facilitando uma acumulação de capital que nem sempre pode ser sustentada e criando condições que podem levar à crise financeira e económica. Por exemplo, o crédito pode induzir em erro os capitalistas industriais, levando-os a antecipar retornos favoráveis ​​quando nenhum deles está disponível e, quando novo crédito é utilizado para pagar obrigações a vencer, a expansão excessiva da acumulação pode criar condições de crise económica. Finalmente, a incerteza torna-se ainda maior quando ocorre o comércio de dinheiro, criando uma classe de negociantes de dinheiro apenas vagamente ligados à produção e ao comércio. O comércio de dinheiro e de instrumentos relacionados com dinheiro poderá conduzir à especulação e à fraude desestabilizadoras, criando ainda mais incerteza mesmo para aqueles que não estão diretamente envolvidos em tais atividades.
 \par 
Por um lado, para Marx a produção de mais-valia absoluta e relativa é crucial para a compreensão das relações distributivas; mas este último não pode ser deduzido apenas das condições de produção. Por outro lado, a incerteza gerada pela produção capitalista (e não pelos humores mutáveis ​​dos capitalistas industriais e financeiros) desempenha um papel essencial na produção de mais-valia, bem como no desencadeamento de crises.
 \par 
\section{Reprodução Simples}
 \par 
As secções anteriores centraram-se na reprodução simples e expandida apenas dentro do sistema económico. Em princípio, com uma excepção crucial, os circuitos do capital parecem ser auto-sustentáveis. A notável excepção é a força de trabalho, cuja reprodução exige, em primeiro lugar, que o fornecimento de bens salariais seja adequado para esse fim. Em segundo lugar, em virtude da liberdade dos trabalhadores uma vez terminada a jornada de trabalho (bem como da sua resistência no chão de fábrica e fora dela), o capital tem pouco controle sobre os processos de reprodução da força de trabalho e, em certo sentido, é aqui que a reprodução social assume o controle. Este último envolve um conjunto complexo de relações, processos, estruturas, poderes e conflitos não económicos que, interpretados em termos restritos, incluem os processos necessários para a reprodução da força de trabalho, tanto biologicamente como como trabalhadores assalariados complacentes. De forma mais geral, a reprodução social preocupa-se com a forma como a sociedade como um todo é reproduzida e transformada ao longo do tempo.
 \par 
Em suma, e de forma bastante apropriada, a reprodução social tornou-se um termo genérico dentro do qual se reúnem todos os factores não económicos. Abrange todo o terreno entre a categoria abstrata do capital e a realidade empírica do capitalismo. Mas mesmo isto é apenas uma compreensão parcial do alcance e do significado da reprodução social. O capitalismo depende claramente de uma reprodução económica e social satisfatória, da qual a reprodução económica faz parte. A percepção errada da relação entre as duas é comum, como se a reprodução económica e social estivessem separadas uma da outra, como o trabalho e a casa. Em grande medida, a justaposição inadequada do económico e do social (este último como uma “superestrutura” política, cultural ou qualquer outro tipo) é mais marcada nas fronteiras disciplinares entre as ciências sociais.
 \par 
Um dos locais mais significativos de reprodução social e económica é o Estado. Através do Estado constituem-se e expressam-se relações, processos e conflitos políticos distintos, mas não independentes, daqueles de reprodução económica. Até que ponto o Estado depende da economia é altamente controverso. As opiniões variam, para colocá-lo em termos um tanto unidimensionais, desde aquelas em que o Estado é redutível a imperativos económicos, especialmente capitalistas, até aquelas em que o Estado é visto como autónomo da economia. A natureza do Estado capitalista será abordada no Capítulo 15, mas as questões aqui do que foi denominado reducionismo, por um lado, e da autonomia, por outro, são de significado metodológico, teórico e empírico mais geral. O ponto importante é reconhecer o significado causal da economia capitalista para o não-económico - que tipo de Estado, direito de propriedade, costumes, política, e assim por diante, prevalecem em cada tipo de sociedade.
 \par 
Considerações semelhantes aplicam-se às áreas de reprodução social que estão fora da órbita imediata do Estado, o que é frequentemente referido como “sociedade civil”. A reprodução social também depende do sistema doméstico ou familiar e das áreas mais gerais da actividade privada, e não menos importante do consumo e de outras actividades da classe trabalhadora que a induzem e permitem que ela se apresente diariamente para trabalhar.
 \par 
A ênfase até agora tem sido colocada na reprodução social do trabalho; mas a reprodução económica depende igualmente da formação e transformação das condições que permitem a reprodução dos circuitos do capital como um todo - nomeadamente o mercado e os sistemas monetários e de crédito que exigem leis, regulamentos, e assim por diante. Estas promovem inevitavelmente os interesses de alguns capitalistas em detrimento de outros, bem como evitam que a rivalidade entre capitalistas seja indevidamente destrutiva. Tais questões são igualmente objecto da política, do Estado e da sociedade civil. No nível abstrato desta exposição introdutória, apenas as condições necessárias e induzidas pela reprodução económica podem ser identificadas, juntamente com a forma como a reprodução económica e social são estruturadas uma em relação à outra: como é acomodada a acumulação de capital? socialmente e o conflito sobre ele contido? Para ir além disso, é necessário introduzir a especificidade histórica, uma tarefa que foge ao escopo deste texto.
 \par 
\section{Reprodução Simples}
 \par 
Tal como sugerido no capítulo anterior, a análise de Marx no Volume {\color{blue}2} de O Capital foi negligenciada e, portanto, relativamente livre de controvérsia. Para uma análise mais aprofundada do Volume 2, com alguma ênfase nos três circuitos de capital diferentes, mas integralmente relacionados (associados ao capital monetário, ao capital produtivo e ao capital mercadoria) e com implicações para a ideologia económica e as crises, ver Ben Fine (1975). O mesmo não pode ser dito da reprodução social. Isto tem sido fortemente debatido, seja dentro do marxismo ou contra ele. A controvérsia abrange a relação entre o económico e o não-económico (e como eles dependem ou não um do outro), e os diferentes aspectos do próprio não-económico, desde a natureza da autonomia do Estado e da política até ao papel da “sociedade civil”.
 \par 
Este capítulo concentra-se no material contido em Karl Marx (1978b, pt.{\color{blue}3}). A interpretação da reprodução social desenvolvida acima baseia-se em Ben Fine (1992b, 2013), Ben Fine e Ellen Leopold (1993) e Ben Fine, Michael Heasman e Judith Wright (1996); veja também John Weeks (1983). O valor da força de trabalho e a reprodução da classe trabalhadora são discutidos por Ben Fine (1998, 2002, 2003, 2012a), Ben Fine, Costas Lapavitsas e Alfredo Saad-Filho (2004) e Alfredo Saad-Filho (2002, cap. .{\color{blue}4}); ver também Kenneth Lapides (1998), Michael Lebowitz (2003a e 2009, cap.{\color{blue}1}) e David Spencer (2008), e o debate de Ben Fine com Michael Lebowitz em Fine (2008, 2009) e Lebowitz (2006, 2010).