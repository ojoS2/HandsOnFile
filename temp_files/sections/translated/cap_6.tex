\begin{figure}
	\centering
	\includegraphics[width=1.\textwidth]{temp\_files/images/UP\_logo.png }
	\caption{Alexandra Kollontai (1872-1952): Uma teórica socialista da emancipação das mulheres e uma defensora estridente das relações sexuais livres de todas as considerações econômicas. Após a Revolução de Outubro, Kollontai se tornou a Comissária do Povo para o Bem-Estar Social e ajudou a fundar o Zhenotdel (Seção Feminina). Ela supervisionou uma ampla variedade de reformas legais e políticas públicas para ajudar a libertar as mulheres trabalhadoras e criar a base de uma nova moralidade sexual comunista. Mas os russos não estavam prontos para sua visão de emancipação, e ela foi enviada para a Noruega para servir como a primeira embaixadora russa (e apenas a terceira embaixadora feminina no mundo). Cortesia da Biblioteca do Congresso dos EUA.}
	\label{ }
\end{figure}
 \par 
CAPITALISMO ENTRE OS > LENÇÓIS: SOBRE SEXO (PARTE {\color{blue}1})
 \par 
Meu melhor amigo da faculdade, a quem chamarei de Ken, = era formado em economia e morreu em {\color{blue}11} de setembro de 2001, na Torre Norte do World Trade Center. Ao longo de nossa amizade de treze anos, acumulamos contas telefônicas de longa distância e atravessamos continentes e oceanos para nos encontrar pessoalmente quando uma ligação não era suficiente. Nós nos reunimos em Hong Kong após seu divórcio; eu o ouvi soluçar enquanto tomava seu martíni de vodca em um bar chamado Rick's Café. Ele tirou as fotos no meu casamento em 1998 e, no fim de semana do Dia do Trabalho de 2001, ele voou para Berkeley para apalpar minha barriga quando eu estava grávida de sete meses. Ele se foi antes de eu dar à luz.
 \par 
Nascido e criado fora dos Estados Unidos, Ken viveu o sonho americano, começando em Wall Street em 1989 e negociando moedas até chegar à companhia de milionários. Antes de seu casamento, ele aproveitou a vida amorosa bem-sucedida de um rico solteiro de Nova York. Em algum bar de gim em Oakland, uma vez trocamos notas sobre o que as pessoas precisavam para um relacionamento romântico saudável. Ainda tenho o papel em que Ken escreveu: "Kristy diz: físico, emocional, intelectual, espiritual. Eu digo: belas pernas com belo tornozelo, olhos tristes,
 \par 
107
 \par 
108
 \par 
CAPITALISMO ENTRE OS LENÇÓIS: SOBRE SEXO (PARTE I) peitos 34C bem torneados + um pouco de cérebro.” Eu estava tentando argumentar que havia quatro tipos de conexões entre as pessoas e os melhores romances eram aqueles em que você se conectava de cada uma das quatro maneiras, mas Ken insistia que gostava de suas mulheres bonitas e inteligentes o suficiente para não serem estúpidas. “Eu amo mulheres gostosas!!!” ele escreveu. Mas quando sua esposa interesseira o abandonou imediatamente depois que eles ganharam seus green cards, e então o enganou para um pagamento de pensão alimentícia enorme, Ken começou a questionar seu gosto por mulheres.
 \par 
“Eu nunca quis namorar mulheres profissionais”, ele me disse ao telefone depois que a dor do divórcio começou a passar, e ele se sentiu pronto para começar a namorar novamente. “Elas têm muita coisa acontecendo em suas vidas, e elas não podem estar lá para você quando você quer que elas estejam. Eu saí com uma advogada uma vez, e tudo o que ela fez foi falar sobre seus casos.”
 \par 
“Você fala sobre seu trabalho”, eu disse. “Eu sei”, ele disse, “e eu quero que minha namorada me escute.”
 \par 
Ken respirou fundo do outro lado do telefone. “Mas, sabe, acho que eu deveria tentar namorar mulheres mais inteligentes. Estou cansado das interesseiras.”
 \par 
“Sério?”, eu disse. “Isso seria fora do personagem.” Ele continuou descrevendo uma epifania recente. Ken explicou que, como eu suspeitava, ele evitava mulheres inteligentes e independentes porque elas o faziam se sentir menos masculino, menos no controle do relacionamento. Mas um de seus colegas de trabalho tinha se casado recentemente com uma advogada corporativa “impossivelmente gostosa”. Na recepção do casamento, depois de cinco taças de vinho a mais, Ken observou o novo casal dançando e decidiu que seu colega de trabalho era, na verdade mais másculo porque ele não se sentia intimidado por estar com uma mulher bem-sucedida. “Quer dizer,
 \par 
KRISTEN R. GHODSEE pense nisso”, Ken me disse. “É fácil conseguir garotas gostosas se você as quiser. Mas é mais difícil conseguir uma garota gostosa e inteligente com seu próprio dinheiro. E se ela tem seu próprio dinheiro, você sabe que ela não está com você pelo seu.”
 \par 
Ele suspirou. "Acho que eles realmente se amam." Para Ken, atração e amor sempre estiveram ligados a dinheiro e poder. Ele usou sua riqueza para atrair mulheres e se deleitou no papel de macho alfa. Mas o que Ken descobriu (bem tarde em sua curta vida) foi a ideia de que relacionamentos mais igualitários criam menos oportunidades para subterfúgios emocionais e ressentimentos entre parceiros. Ken adorava sua ex-esposa e presumia que ela genuinamente retribuía suas afeições. Ela certamente o levou a acreditar nisso, cedendo-lhe todo o poder em seu casamento de curta duração. Depois que ela o trocou por outro homem, Ken questionou se sua esposa o amou ou simplesmente o usou para imigrar para os Estados Unidos. Mas o que mais o incomodava era que ele não conseguia perceber a diferença; ela havia desempenhado o papel de esposa atenciosa e amorosa até o momento em que pediu o divórcio. Ele nunca duvidou de sua autenticidade e temia repetir o mesmo erro em seu próximo relacionamento. Infelizmente, ele nunca teve a chance; ele tinha apenas trinta e poucos anos quando as torres do World Trade Center caíram.
 \par 
Como Ken era formado em economia e um capitalista o mais rápido possível, sei que ele teria adorado um artigo de pesquisa publicado apenas três anos após sua morte. Em 2004, um artigo controverso — “Economia Sexual: Sexo como Recurso Feminino para Troca Social em Interações Heterossexuais” — propôs que sexo seja algo que os homens compram das mulheres
 \par 
109
 \par 
110
 \par 
CAPITALISMO ENTRE OS LENÇÓIS: SOBRE SEXO (PARTE |) com recursos monetários ou não monetários, e que amor e romance são meros véus cognitivos que os humanos usam para ocultar a natureza transacional de nossos relacionamentos pessoais. Em seu artigo, Roy Baumeister e Kathleen Vohs deram um salto teórico ousado e aplicaram a disciplina da economia ao estudo da sexualidade humana. Sua visão precipitou um debate acalorado entre psicólogos sobre os comportamentos “naturais” de homens e mulheres no namoro.”
 \par 
A teoria da economia sexual, ou teoria da troca sexual, propõe que os estágios iniciais do flerte sexual e da sedução entre homens e mulheres podem ser caracterizados como um mercado onde as mulheres vendem sexo e os homens compram com recursos não sexuais. “A teoria da economia sexual se baseia em suposições básicas padrão sobre mercados econômicos, como a lei da oferta e da demanda. Quando a demanda excede a oferta, os preços são altos (favorecendo os vendedores, ou seja, as mulheres). Em contraste, quando a oferta excede a demanda, o preço é baixo, favorecendo os compradores (homens).” A ideia básica é que o sexo é uma mercadoria controlada pelas mulheres, porque, conforme os autores, os impulsos sexuais das mulheres são mais fracos do que os dos homens. Devido ao princípio do menor interesse, e porque as mulheres são menos governadas por seus impulsos sexuais, elas têm poder em relacionamentos sexuais com homens. Elas podem exigir compensação dos homens porque os homens querem a mercadoria (sexo) mais do que as mulheres. É também em uma tentativa de manter o preço do sexo alto que outras mulheres suprimem supostamente a sexualidade de seus colegas vendedores. Assim, Baumeister e Vohs argumentam que o patriarcado não é responsável pela sul-chaminé. Em vez disso, são outras mulheres que querem punir aquelas que vendem seu sexo muito barato e, assim, reduzir seu preço geral.’
 \par 
Os autores não estão falando sobre trabalho sexual, no qual
 \par 
KRISTEN R. GHODSEE sexo é trocado diretamente por dinheiro (embora eles usem a prevalência do trabalho sexual como um exemplo em apoio à sua teoria). Então, com o que os homens compram os serviços sexuais das mulheres? Os proponentes da teoria da economia sexual explicam:
 \par 
\textit\textbf{ {Uma ampla gama de bens valiosos pode ser trocada por sexo.} }
 \par 
Em troca de sexo, as mulheres podem obter amor, compromisso,
 \par 
\chapter{Respeito, atenção, proteção, favores materiais, oportunidades}\label{Respeito, atenção, proteção, favores materiais, oportunidades}
 \par 
\section{Nidades, notas de cursos ou promoções no local de trabalho, bem como}
 \par 
A troca padrão TTTTTT tem sido que um homem faz uma longa
 \par 
Compromisso de prazo para fornecer recursos à mulher
 \par 
(muitas vezes os frutos do seu trabalho) em troca de sexo - ou,
 \par 
TTTTTT sexualidade feminina. Se alguém aprova tal ex-
 \par 
Mudá-las ou condená-las não vem ao caso. Em vez disso, o fato principal é que essas oportunidades existem quase exclusivamente para mulheres. Os homens geralmente não podem trocar sexo por outros benefícios.’
 \par 
A teoria da economia sexual foi atacada por outros psicólogos por ser baseada na suposição falha de que os impulsos sexuais das mulheres são mais fracos do que os dos homens e as mulheres têm um desejo "natural" de extrair recursos dos homens em troca de sexo. As feministas também apontaram para as suposições profundamente patriarcais e misóginas embutidas na teoria da economia sexual, uma vez que o preço do sexo também varia com a desejabilidade percebida da mulher que o oferece (conforme determinado pelos compradores homens). Outros criticaram o pensamento economista que reduz o romance e a afeição mútua a uma competição adversária entre homens e
 \par 
111
 \par 
112
 \par 
CAPITALISMO ENTRE OS LENÇÓIS: SOBRE SEXO (PARTE I) mulheres em que cada lado está tentando obter o melhor acordo. Embora essas críticas sejam importantes, a teoria da economia sexual conquistou muitos seguidores porque parece intuitiva, especialmente na cultura individualista e materialista dos Estados Unidos.
 \par 
Na verdade, alguns direitistas americanos adotaram a teoria da economia sexual como uma forma de culpar as mulheres pelos males atuais da nossa sociedade. De fato, um vídeo animado viral do YouTube de 2014 do conservador Austin Institute for lhe Study of. Family ano Culture extrapolou o trabalho de Baumeister e Vohs e culpou a queda na taxa de casamento e o desajuste social de homens jovens nos Estados Unidos em mulheres soltas que tornaram o preço do sexo muito baixo. Em sua visão de mundo, a disponibilidade de controle de natalidade (e presume-se o aborto) reduziu os riscos associados ao sexo, uma vez que agora é menos provável que resulte em uma gravidez indesejada que deve ser levada a termo. Quando o sexo implicava o risco da paternidade, eles argumentam, as mulheres extraíam um preço muito mais alto pelo acesso a seus corpos, no mínimo um compromisso sério e, idealmente, casamento. Mas uma vez que o controle de natalidade reduziu o risco de gravidez, as mulheres puderam fazer com seus corpos o que quisessem, e o preço que exigiam pelo sexo caiu, principalmente porque tinham outras oportunidades de ganhar dinheiro.
 \par 

 \par 
Isso é algo terrível? Além da queda na taxa de casamento (o velho argumento “Por que comprar a vaca quando você pode obter o leite de graça?”), um preço baixo para o sexo prejudica os homens que, de acordo com essas teorias, aparentemente não têm incentivo para fazer nada com suas vidas além da busca por sexo. Isso não é uma piada, eu lhe asseguro. Conforme os ideólogos do Austin Institute, os jovens de hoje em dia
 \par 
KRISTEN R. GHODSEE estão acampando no porão dos pais, jogando videogame e sobrevivendo com pizza da Dominó's porque sexo barato está a apenas uma mensagem de texto de distância. Quando as mulheres não têm controle de natalidade, o preço do sexo é mais alto. Quando as mulheres não têm acesso ao aborto, o preço é ainda mais alto. Quando as mulheres têm menos oportunidades educacionais ou econômicas fora de seus relacionamentos com homens, o preço do sexo é geralmente o casamento. Quando o preço do sexo é muito alto, de acordo com essa visão de mundo, os homens famintos por sexo têm incentivos para sair e conseguir empregos, ganhar dinheiro e fazer algo de suas vidas para poderem comprar acesso à sexualidade de uma mulher para o resto da vida por meio do casamento. Em culturas com mais homens do que mulheres, por exemplo, os economistas mostraram que há uma taxa maior de empreendedorismo masculino. Quando o preço do sexo é muito baixo, no entanto, os homens não têm incentivo intrínseco para fazer nada produtivo.* Para ser justo, os autores originais da teoria da economia sexual não sugerem nenhuma mudança normativa para nossa sociedade; eles são apenas observadores, reunindo evidências para seu modelo teórico. Eles também reconhecem que os mercados sexuais estão inseridos em contextos culturais específicos que influenciam a oferta e a demanda por sexo. Para apoiar suas alegações, os proponentes da teoria da economia sexual postulam que o posição das mulheres na sociedade é um fator importante que afeta a operação subjacente do mercado de sexo. Eles observam, por exemplo, que a emancipação das mulheres reduz o preço do sexo porque as oportunidades educacionais e o emprego remunerado dão às mulheres outras vias para suprir suas necessidades básicas. Seu modelo prevê que o preço do sexo é mais alto em sociedades mais tradicionais, onde as mulheres são excluídas da vida política e econômica.
 \par 
\section{Como dinheiro. Ao longo da história da civilização, um}
 \par 
113
 \par 
114
 \par 
CAPITALISMO ENTRE OS LENÇÓIS: SOBRE SEXO (PARTE {\color{blue}1})
 \par 
Mendoza correlacionou os resultados de uma pesquisa global sobre sexo com uma medida independente de desigualdade de gênero para mostrar que a oportunidade econômica para as mulheres resulta em sexo mais livre. Eles descobriram que em países onde homens e mulheres são mais iguais, havia “mais sexo casual, mais parceiros sexuais per capita, idades mais jovens para o primeiro sexo e maior tolerância/aprovação para sexo antes do casamento”. Assim, os autores argumentam que a independência econômica das mulheres geralmente acompanha um afrouxamento dos costumes sociais em torno da sexualidade. “Conforme a teoria da economia sexual”, Baumeister e Mendoza explicam, “quando as mulheres não têm acesso direto ou fácil a recursos como influência política, assistência médica, dinheiro, educação e empregos, o sexo se torna um meio crucial pelo qual as mulheres podem obter acesso a uma vida boa e, portanto, é vital para o interesse próprio feminino manter o preço do sexo alto”. As mulheres fazem isso reduzindo a oferta (não mais sexo casual), o que aumenta o preço. É de acordo com uma lógica similar que, para um certo grupo de conservadores sociais extremos, a única maneira de “Tornar a América Grande Novamente” é abolir o controle de natalidade e o aborto, ao mesmo tempo, em que garante que as mulheres tenham poucas oportunidades econômicas para pagar por bens básicos além de vender seu sexo. Quando sua sexualidade é seu único meio de sobrevivência, eles supostamente aumentarão seu preço e, assim, salvarão uma geração inteira de homens de uma vida de preguiça.
 \par 

 \par 
A teoria da economia sexual pressupõe uma economia capitalista subjacente na qual as mulheres têm um bem (sexo) que podem escolher vender ou doar como trabalhadoras do sexo ou de maneiras menos abertas, mas não menos transacionais, como sugar bebês, namoradas ou esposas. Para atender às suas necessidades básicas (comida, abrigo, assistência médica, educação), elas devem vender seu sexo ou ganhar dinheiro para pagar por esses recursos de outra forma.
 \par 
KRISTEN R. GHODSEE
 \par 
Quanto mais oportunidades elas têm de ganhar dinheiro (por exemplo, em sociedades com altos níveis de igualdade de gênero), menos dependentes elas são da venda de seu sexo e provável é que façam sexo por prazer. Da mesma forma, também se poderia supor que mulheres vivendo em uma sociedade que fornece a seus cidadãos acesso subsidiado a necessidades básicas como comida, abrigo, assistência médica e educação teriam menos incentivos para acumular seu sexo a fim de manter seu preço alto. Em outras palavras, em sociedades com altos níveis de igualdade de gênero, com fortes proteções para a liberdade reprodutiva e com grandes redes de segurança social, as mulheres raramente precisam se preocupar com o preço que seu sexo vai render no mercado aberto. Sob essas circunstâncias, o modelo da teoria da economia sexual preveria que a sexualidade das mulheres deixaria de ser uma mercadoria vendável.
 \par 
Sendo alguém que frequentemente critica os modelos economistas reducionistas, sou fascinado pela teoria da economia sexual e penso que o modelo proporciona uma visão valiosa sobre como a sexualidade é experienciada nas sociedades capitalistas. Essencialmente, a teoria da economia sexual está certa, mas apenas dentro dos limites do sistema de mercado livre. Na verdade, surge uma bela confluência quando se lê as obras de Baumeister e dos seus colegas juntamente com as críticas socialistas à sexualidade capitalista. Embora possam não o perceber, os teóricos da economia sexual abraçam basicamente uma crítica socialista de longa data ao capitalismo: que ele mercantiliza todas as interações humanas e reduz as mulheres a bens móveis. Em 1848, Karl Marx e Friedrich Engels observaram que o capitalismo
 \par 
\textit\textbf{ {Não deixou nenhum outro nexo entre o homem e o homem além do interesse próprio e do cruel “pagamento em dinheiro”.} }
 \par 
115
 \par 
116
 \par 
CAPITALISMO ENTRE OS LENÇÓIS: SOBRE SEXO (PARTE {\color{blue}1})
 \par 
\textit\textbf{ {Afogou os êxtases mais celestiais da religião} }
 \par 
\section{Muitas vezes, mais precisamente, para acesso sexual exclusivo a esse}
 \par 
O TTTTTT transformou o valor pessoal em valor de troca e, em vez das inúmeras liberdades inalienáveis ​​e garantidas,
 \par 
Estabeleceu aquela liberdade única e inconcebível — Livre
 \par 
Comércio... A burguesia arrancou da família o seu véu sentimental e reduziu as relações familiares
 \par 
\section{Para provar este ponto, Roy Baumeister e Juan Pablo}
 \par 
Já na era do socialismo utópico na década de 1830, os teóricos argumentavam que as sociedades pós-capitalistas gerariam uma nova forma de moralidade sexual. Em seu livro de 1879, Woman ano Socialism, August Bebel escreveu que o desejo sexual era natural e saudável, e que as mulheres precisavam ser libertas das relações de propriedade então socialmente aceitas que distorciam e suprimiam sua sexualidade para torná-la escassa:
 \par 
SSSSSS chamada de independente, ela não está mais sujeita a nem mesmo um
 \par 
KRISTEN R. GHODSEE
 \par 
\textit\textbf{ {A mulher da sociedade futura é social e economicamente} }
 \par 
\section{Fervor, de entusiasmo cavalheiresco, de sentinela filisteia}
 \par 
TTTTTT ing. o objeto de seu amor, a mulher, como o homem, é livre e
 \par 
Sem impedimentos. Ela corteja ou é cortejada e entra em um
 \par 
\section{Mentalismo, nas águas geladas do cálculo egoísta. Isto}
 \par 
\section{Para uma mera relação monetária.}
 \par 
Os instintos TTTTTT não causam danos ou desvantagens aos outros,
 \par 
O indivíduo deve cuidar de suas próprias necessidades. A satisfação do instinto sexual é tanto uma preocupação privada
 \par 
Como a satisfação de qualquer outro instinto. Ninguém é responsável por isso perante os outros e nenhum juiz não solicitado tem o direito de interferir. O que comerei, como comerei
 \par 
Beber, dormir e vestir-me são da minha conta, assim como também é da minha conta.
 \par 
\section{Uma igual ao homem e dona do seu destino. .}
 \par 
\section{. Em mastigações-}
 \par 
Lendo essas palavras no século XXI, é difícil entender o quão radicais elas teriam soado no final do século XIX, quando seu livro foi publicado pela primeira vez. Bebel realmente acreditava que a sexualidade era uma preocupação privada (e foi celebrado por ativistas modernos dos direitos LGBT como o primeiro político a defender publicamente os direitos dos gays em 1898). Friedrich Engels também argumentou, em 1884, que a subjugação das mulheres resultou do desejo masculino de herdeiros legítimos para herdar sua riqueza. Para garantir que seus filhos fossem realmente seus, o homem precisava controlar a sexualidade das mulheres por meio da instituição do casamento monogâmico. A fidelidade e a capacidade reprodutiva das mulheres, portanto, tornaram-se mercadorias a serem trocadas entre os homens com o propósito de projetar sua riqueza e poder acumulados nas futuras gerações de seus descendentes. Mas a monogamia era
 \par 
117
 \par 
118
 \par 
CAPITALISMO ENTRE OS LENÇÓIS: SOBRE SEXO (PARTE {\color{blue}1}) principalmente monogamia para a mulher, já que os homens podiam ter relações sexuais fora do casamento com impunidade, e o contrato de casamento privava a maioria das mulheres não apenas do controle de seus corpos, mas também de seus direitos fundamentais como indivíduos. O casamento reduzia as mulheres ao posição de propriedade de seus maridos."°
 \par 
\begin{figure}
	\centering
	\includegraphics[width=1.\textwidth]{temp\_files/images/UP\_logo.png }
	\caption{Alexandra Kollontai se rebelou contra essa contínua mercantilização das mulheres. Nascida em uma família da nobreza russa em 1872, ela demonstrou uma profunda empatia pelas condições atrozes das classes trabalhadoras da Rússia desde cedo e foi lentamente atraída para o trabalho político, o que muitas vezes a colocou em problemas com as autoridades czaristas. Ao observar a situação das mulheres em sua própria classe, Kollontai passou a abominar a troca da sexualidade feminina por dinheiro, bens, serviços ou posição social. Quando criança, ela viu sua mãe pressionar sua irmã de vinte anos a se casar com um homem cinquenta e um anos mais velho porque ele era considerado um "bom par". Kollontai rejeitou casamentos de conveniência e queria se casar por amor, pelo que ela chamou de "grande paixão". Ela escreveu: "No que diz respeito às relações sexuais, a moralidade comunista exige, antes de tudo, o fim de todas as relações baseadas em considerações financeiras ou outras considerações econômicas. A compra e venda de carícias destrói o senso de igualdade entre os sexos e, assim, mina a base de solidariedade sem a qual a sociedade comunista não pode existir.”}
	\label{ }
\end{figure}
 \par 
Em 1894, ela leu Woman ano Socialism, de August Bebel, e ele forneceu a base para suas próprias visões sobre uma nova forma de moralidade progressiva. Como Bebel, ela acreditava que a sexualidade precisava ser liberta da estigmatização social: “O ato sexual deve ser visto não como algo vergonhoso e pecaminoso, mas como algo que é tão natural quanto as outras necessidades
 \par 
KRISTEN R. GHODSEE de [um] organismo saudável, como fome e sede. Tais fenômenos não podem ser julgados como morais ou imorais.” Kollontai argumentou que somente sob o socialismo as pessoas amariam e fariam sexo umas com as outras como indivíduos livres, com base na atração e afeição mútuas e sem consideração por dinheiro ou posição social. Mas é importante perceber que Kollontai nunca defendeu a promiscuidade desenfreada ou uma forma de “amor livre” na busca exclusiva do prazer hedonista. Em vez disso, ela acreditava que, ao destruir a ligação entre propriedade e sexualidade, homens e mulheres teriam relacionamentos mais autênticos e significativos. Embora tenha sido posteriormente caracterizada como uma libertina sexual, ela era relativamente conservadora (pelos padrões modernos) nas suas opiniões, defendendo a realização sexual apenas dentro de relações heterossexuais baseadas no amor.’
 \par 
Kollontai considerava o sexo por prazer como uma distração burguesa do trabalho necessário da revolução, contrastando o “Eros sem asas” do sexo físico puro com seu “Eros alado” idealizado de conexão emocional e até espiritual. Esse amor romantizado entre homens e mulheres deveria contribuir para o amor generalizado pela humanidade que sustentava a base da ideologia socialista (Kollontai pode realmente ser a hippie original). Em seu panfleto de 1921, Teses sobre a moralidade comunista na esfera das relações conjugais, Kollontai escreveu: “A atitude burguesa em relação às relações sexuais como uma questão de sexo deve ser criticada e substituída por uma compreensão de toda a gama de experiências amorosas alegres que enriquecem a vida e proporcionam maior felicidade. Quanto maior o desenvolvimento intelectual e emocional do indivíduo, menos espaço haverá em seu relacionamento para o mero
 \par 
119
 \par 
120
 \par 
CAPITALISMO ENTRE OS LENÇÓIS: SOBRE O SEXO (PARTE!) lado fisiológico do amor, e mais brilhante será a experiência amorosa.”
 \par 
Kollontai via o casamento como uma instituição que perpetuava a subjugação das mulheres, e foi essa instituição que ela tentou desmantelar nos primeiros anos após a Revolução de outubro de 1917 na Rússia. Ela e um pequeno grupo de juristas radicais tentaram desafiar a base tradicional do matrimônio substituindo os casamentos religiosos por cerimônias civis, liberalizando as leis de divórcio, legalizando o aborto, descriminalizando a homossexualidade, igualando os direitos para filhos legítimos e ilegítimos e mobilizando as mulheres para a força de trabalho, enquanto socializava o trabalho doméstico por meio do estabelecimento de lavanderias públicas, refeitórios e lares para crianças. Mas, como discutido anteriormente, Lenin e os outros bolcheviques homens tinham preocupações que consideravam mais urgentes do que a questão da mulher, e Kollontai acabou sendo enviada como diplomata para a Noruega (para tirá-la do país). Refletindo sobre sua vida em 1926, Kollontai escreveu: “Não importa quais outras tarefas eu esteja realizando, está perfeitamente claro para mim que a libertação completa da mulher trabalhadora e a criação da fundação de uma nova moralidade sexual sempre permanecerão como o objetivo mais elevado da minha atividade e da minha vida.”
 \par 
A visão de Kollontai de uma sexualidade livre de considerações econômicas foi partilhada por muitos jovens soviéticos na
 \par 
Década de 1920. Por exemplo, uma pesquisa de 1922 com {\color{blue}1}.{\color{blue}552} estudantes da Universidade Comunista de Sverdlov, em Moscou, descobriu que apenas 21% dos homens e 14% das mulheres consideravam o casamento como a maneira ideal de organizar a vida sexual. Em contraste, dois terços das mulheres e metade dos homens preferiam um relacionamento de longo prazo baseado no amor.
 \par 
KRISTEN R. GHODSEE
 \par 
Mas essas atitudes liberais não se estenderam ao resto da população. Os conservadores tradicionais da cultura camponesa russa, combinados com o conselho especializado de um estabelecimento médico pudico, conspiraram para subverter as tentativas de reforma social de Kollontai. Sem acesso a um controle de natalidade confiável, as mulheres não conseguiam controlar sua fertilidade, e os homens que declaravam seu amor eterno desapareciam quando uma criança estava a caminho. Os tribunais tentaram impor pagamentos de pensão alimentícia, mas os homens se esquivaram de suas responsabilidades. Os salários das mulheres não eram altos o suficiente para sustentar as crianças, e muitas se voltaram para o trabalho sexual para sobreviver, precisamente o tipo de troca econômica que Kollontai esperava erradicar. O estado soviético tentou criar uma rede de orfanatos para cuidar de crianças desabrigadas, mas todo o projeto era muito caro. Kollontai fez uma última tentativa de substituir a pensão alimentícia por um fundo de seguro geral que permitiria ao estado sustentar todas as crianças, mas suas ideias foram ridicularizadas e rejeitadas. Em meados da década de 1920, centenas de milhares de órfãos vermelhos vagavam pelas ruas da Rússia Soviética, mendigando, roubando e personificando os fracassos de uma tentativa prematura de revolução sexual.”
 \par 
Stalin, que ascendeu ao poder ditatorial no final da década de 1920, decidiu que era muito mais fácil retornar a um sistema em que as mulheres faziam toda a criação e parto de graça dentro dos limites de formas mais tradicionais de casamento, ao mesmo tempo, em que as forçava a trabalhar fora de casa para construir o poder industrial soviético. Muitos conservadores sociais nos Estados Unidos encontrariam muito o que amar nas políticas de Josef Stalin: ele proibiu o aborto novamente, promoveu a abstinência pré-marital, reprimiu discussões públicas sobre sexualidade, perseguiu gays e enfatizou a
 \par 
121
 \par 
122
 \par 
CAPITALISMO ENTRE OS LENÇÓIS: SOBRE SEXO (PARTE!) papéis de gênero no casamento heterossexual e monogâmico. Mesmo após a morte de Stalin, quando a lei do aborto foi novamente liberalizada, a maioria dos estudos confirma que o discurso público sobre sexualidade na URSS era inexistente. Antes disso, a maioria das mulheres soviéticas via o sexo como um dever conjugal com o único propósito de procriação, e a sociedade soviética era decididamente pudica. Kollontai morreu em 1952, muito antes de sua visão de uma sexualidade soviética baseada no amor e na afeição mútua ter a chance de se desenvolver.'®
 \par 
No entanto, a concepção de Kollontai de uma sociedade na qual a sexualidade é livre de restrições econômicas continuou a inspirar o pensamento feminista desde o início do século XX. Entre sua visão socialista de uma sexualidade baseada em afeição mútua e a visão proposta pela teoria da economia sexual, temos duas visões concorrentes de como organizar a sexualidade heterossexual. Uma visão celebra a independência econômica das mulheres como um pré-requisito para uma forma mais autêntica de amor, e a outra visão vê a independência econômica das mulheres como apenas um fator que afeta o preço relativo do sexo num mercado em que o sexo é uma mercadoria a ser comprada por homens. Embora certamente exista uma grande variedade de posições entre esses dois modelos, para fins de argumentação, vamos nos concentrar nessas duas visões como polos em um espectro de modelos possíveis para relacionamentos heterossexuais. Qual seria melhor?
 \par 
Claramente, não há uma resposta fácil. A sexualidade humana é complexa e bastante difícil de estudar, tornando qualquer tipo de julgamento normativo sobre sexo repleto de problemas. Mas deixando de lado as pessoas que escolheriam o trabalho sexual sem
 \par 
KRISTEN R. GHODSEE necessidade econômica, vou me arriscar aqui e sugerir que o sexo não é tão bom quando você é forçado a vendê-lo para pagar seu aluguel. Além disso, se um homem sente que está pagando uma mulher para acessar seu corpo, por que ele se importaria com o prazer dela? Ele acredita que ela está sendo compensada pela atividade de maneiras não sexuais. Se ele contratasse uma mulher para limpar sua casa, ele se importaria com o quanto ela gostava? Deveria ser esperado que ele fizesse isso? Por outro lado, duas pessoas — trocando livremente suas afeições sem pensar no que mais elas podem ganhar com isso — provavelmente estão muito mais atentas às necessidades uma da outra do que aquelas que estão consciente ou inconscientemente preocupadas com a natureza econômica da troca. Mas como podemos saber?
 \par 
Não precisamos nos limitar à especulação. É aqui que as experiências do socialismo de estado na Europa Oriental fornecem um experimento natural interessante para aumentar nossa compreensão dos efeitos da economia política no namoro heterossexual. Apesar de suas deficiências, como vimos, os países do outro lado da Cortina de Ferro implementaram uma ampla gama de políticas para promover a independência econômica das mulheres (embora com muita variação na região), o que teria causado a queda do preço do sexo, conforme a teoria da economia sexual. Há evidências de que mulheres e homens começaram a ver a sexualidade feminina como algo a ser compartilhado em vez de trocado por recursos? Nossas relações íntimas são vivenciadas de forma diferente em países capitalistas contra socialistas? E o que aconteceu depois da queda do Muro de Berlim? Os mercados sexuais descritos por Baumeister e Vohs retornaram com a privatização e a mercantilização da economia pós-socialista?
 \par 
Todos os estudos sobre o que é chamado de “bem-estar subjetivo” — ou
 \par 
123
 \par 
124
 \par 
CAPITALISMO ENTRE OS LENÇÓIS: SOBRE SEXO (PARTE I) os próprios sentimentos autorrelatados de felicidade ou satisfação sexual das pessoas — compartilham o problema de que os estados emocionais das pessoas são difíceis de pesquisar de forma objetiva. Quando você estuda algo como câncer, um médico pode examinar um corpo humano e determinar empiricamente a presença ou ausência de células cancerígenas. Mas quando os médicos estudam a dor, eles têm que confiar no relato do próprio paciente sobre o quanto algo dói. Mas as pessoas variam na forma como relatam a dor. Os médicos geralmente usam uma escala de um a dez para medir a dor. Esta não é uma escala absoluta, mas relativa ao próprio limiar de dor do paciente. Quando você está no hospital, por exemplo, os médicos e enfermeiros continuamente pedirão que você classifique sua dor para ter uma noção de sua escala individual e tentar extrapolar a partir disso quanto e que tipo de medicamento você precisa. A dor existe objetivamente, e alguém com um fêmur quebrado deve sentir mais dor do que alguém com uma unha encravada, mesmo que a pessoa com a unha encravada lamente mais alto do que a pessoa com a perna fraturada. Sabemos disso ao agregar os níveis de dor autorrelatados de todos os pacientes que sofrem dessas duas condições e comparar as médias.
 \par 
Sentimentos de felicidade e satisfação sexual são mais como dor do que como câncer nesse aspecto. Psicólogos, sexólogos e outros pesquisadores identificam amostras representativas de populações definidas e então fazem perguntas individuais sobre seus estados emocionais ou seus sentimentos sobre certas experiências. A escolha das perguntas, a maneira como são feitas e a forma e sequência em que as respostas são esperadas são todos aspectos importantes dos estudos de bem-estar subjetivo. Em estudos bem projetados, os pesquisadores fazem diferentes formulações das mesmas perguntas várias vezes
 \par 
KRISTEN R. GHODSEE tempos para controlar vários tipos de mal-entendidos ou preconceitos. Em teoria, se o número de pessoas amostradas for suficientemente grande, surgem certos padrões e podem ser feitas afirmações generalizáveis ​​(pelo menos num determinado meio cultural).
 \par 
Acontece que historiadores, antropólogos e sociólogos contemporâneos têm se interessado muito em saber se a sexualidade não capitalista tinha um caráter diferente dos tipos de relações íntimas que as pessoas tinham (e têm) nas economias de mercado do Ocidente. Ao procurar fontes, eles descobriram uma série de estudos conduzidos antes e depois de 1989 que sugerem que havia algumas diferenças fascinantes na maneira como as pessoas vivenciavam sua sexualidade por trás da Cortina de Ferro, resultados que discutirei no próximo capítulo. Como os cientistas socialistas de estado estavam preocupados com a queda nas taxas de natalidade, eles se concentraram principalmente nas relações heterossexuais entre homens e mulheres, mas muitos de seus visões sobre os danos que as trocas de mercado podem causar aos relacionamentos humanos são relevantes para pessoas de todas as sexualidades. Novamente, a chave aqui não é glorificar ou sugerir que retornemos ao passado socialista de estado. Em vez disso, podemos entender melhor como o capitalismo afeta nossas experiências mais íntimas observando sociedades nas quais as forças de mercado tiveram menos impacto. Se a teoria da economia sexual descreve a maneira como o sistema capitalista reduz nossas afeições e atenções ao posição de bens vendáveis, que alavancas políticas poderíamos ter para empurrar de volta as operações do mercado livre irrestrito? Talvez possamos encontrar maneiras de ter vidas privadas mais gratificantes em uma sociedade que também garanta liberdades individuais e uma esfera pública robusta, minando as operações da teoria da economia sexual sem abraçar o autoritarismo.
 \par 
125
 \par 
\begin{figure}
	\centering
	\includegraphics[width=1.\textwidth]{temp\_files/images/UP\_logo.png }
	\caption{Inessa Armand (1874-1920): Nascida em Paris, Armand foi uma bolchevique e feminista franco-russa que foi uma figura-chave no movimento comunista pré-revolucionário. Depois de 1917, ela serviu como chefe do Conselho Econômico de Moscou, foi membro executivo do Soviete de Moscou e chefiou o Zhenotdel, liderando esforços para garantir a igualdade sexual e socializar o trabalho doméstico. Ela ajudou a organizar lares para crianças, refeitórios de massa e lavanderias públicas até sua morte prematura de cólera aos quarenta e seis anos. Cortesia da Sputnik.}
	\label{ }
\end{figure}