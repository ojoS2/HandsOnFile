\chapter{14 Financeirização, Neoliberalismo e a Crise}\label{14 Financeirização, Neoliberalismo e a Crise}
 \par 
Este livro preocupou-se principalmente em fornecer uma visão geral relativamente simples da economia política de Marx, especialmente conforme apresentada nos três volumes de O Capital. Este capítulo procura aplicar essa economia política à crise global do capitalismo no momento em que este artigo foi escrito, que se apresenta como decorrente de uma grande disfunção no sistema financeiro, com repercussões devastadoras em todos e cada um dos aspectos da reprodução económica e social. Mas, à luz das questões levantadas no capítulo anterior, e de outras questões de poder e conflito em torno da guerra, género, raça, pobreza e desenvolvimento, por exemplo, é importante ter em mente que a crise actual não é nem uma ruptura aguda com o passado, nem se limita a questões económicas estritamente definidas. Com efeito, as crises tendem a acentuar e, nessa medida, a revelar a natureza e as contradições da sociedade em que vivemos; isto é especialmente bem ilustrado pela queda em desgraça da fraternidade financeira. No entanto, a luz impiedosa que a crise brilha não faz obviamente do capitalismo contemporâneo um livro aberto, que possa ser facilmente lido de capa a capa em letras grandes. Assim, embora o neoliberalismo tenha sofrido temporariamente uma crise de legitimidade, além da sua crise económica, as razões desta última, bem como as propostas de resolução, permanecem controversas em todos os espectros intelectuais e políticos, e dentro do próprio marxismo.
 \par 
\section{A crise da financeirização}
 \par 
Cada crise incorpora características específicas, seja em virtude de causas próximas, profundidade, amplitude ou incidência na economia, ideologia ou sistema político, seja através do seu impacto diferencial dentro e entre sectores económicos ou sobre segmentos da classe trabalhadora em cada país, ou por outras razões. Mas a crise actual - mesmo o colapso - é notável numa série de dimensões distintas, bem como na sua combinação. Em primeiro lugar, a crise não foi iniciada por um bolbo de tulipa, pela Ilha dos Mares do Sul ou pela bolha das pontocom, nem mesmo por um frenesim do mercado bolsista ou pela quebra das matérias-primas - embora os mercados bolsistas de diferentes países tenham testemunhado uma considerável turbulência especulativa no período que antecedeu a crise, como bem como em seu rastro. A crise propagou-se a partir do mercado sub-prime dos EUA, um mercado que fornecia financiamento hipotecário às famílias mais pobres do país. É claro que localizar a origem da crise ainda deixa em aberto a questão de saber por que razão deveria ter desencadeado uma explosão mundial tão grande.
 \par 
Em segundo lugar, ninguém culpa os pobres pelo boom especulativo ou pelo crash e pelas suas consequências. Longe disso; ao contrário de outros casos de mau funcionamento económico nos últimos tempos, os salários e benefícios “excessivos” em nenhum lugar foram considerados causais, como ocorreu no passado, de acordo com as opiniões neoclássicas, keynesianas ou mesmo marxistas de “compressão de lucros” - ajudando a legitimar, mais ou menos explicitamente, a transferência do fardo do ajustamento para os trabalhadores e os pobres. Desta vez, a culpa é obviamente das finanças e dos seus excessos, mas (espere!) as finanças devem ser resgatadas, a fim de evitar um impacto ainda pior sobre o resto de nós, cujos tempos difíceis nos próximos anos são assim legitimados. Não é sua culpa, nem de qualquer outra pessoa (deixando convenientemente de lado os incentivos neoliberais ao financiamento e à promoção generalizada dos interesses dos ricos); mas o leite foi derramado, a jarra quebrou, e por isso temos que trabalhar juntos para consertar isso, enquanto isso temos menos para gastar.
 \par 
Terceiro, apesar da sua gravidade, sem precedentes desde a década de 1930, a crise actual encerrou um período de {\color{blue}30} anos de relativo abrandamento da acumulação no Ocidente, após o boom “keynesiano” do pós-guerra, e anunciou uma “nova normalidade” de crescimento mais lento. taxas em todo o mundo que duram até um futuro indefinido. Quaisquer que sejam as suas causas imediatas no mercado imobiliário dos EUA e noutros lugares, o crash e a sua gravidade não são simplesmente o resultado de uma fase maníaca e sobrecarregada de acumulação financeira, cujas contradições, tensões e conflitos induziram uma reacção correspondente na direcção oposta e que pode espera-se que se resolva através da “purga” espontânea desses excessos. Pelo contrário, a crise está claramente inserida no modo de acumulação neoliberal que se consolidou após o desaparecimento do keynesianismo do pós-guerra.
 \par 
Em quarto lugar, a actual crise faz parte de uma sequência de crises financeiras ou da balança de pagamentos que têm afectado regularmente países pobres e de rendimento médio desde o final da década de 1970. Estas têm sido geralmente contidas, mesmo quando graves, em determinadas regiões, nomeadamente através da intervenção estatal multilateral arquitetada pelo Departamento do Tesouro dos EUA e implementada pelo Banco Mundial, pelo Fundo Monetário Internacional e pelas instituições da União Europeia. A situação de hoje é diferente. Porque os mecanismos de transmissão da crise actual superaram até mesmo o grau sem precedentes de intervenção estatal que procura controlar e moderar os seus piores efeitos e propagação geográfica. As limitações da política macroeconómica e da cooperação internacional, sobretudo assinaladas pelos efeitos dominó que emanam da própria crise do subprime, reflectem a complexidade das estruturas contemporâneas de activos financeiros. Isto levou a dificuldades significativas na selecção do que visar para o resgate, através de que critérios, com que fim, como, durante quanto tempo e a que custo, e que políticas complementares são necessárias a nível interno e interestadual.
 \par 
Estes factores são indicativos de uma crise mais ampla no neoliberalismo, exigindo uma explicação com alguma sofisticação. A um nível superficial, e apenas com pequenas excepções, parecia não haver mais neoliberais na sequência da crise. O fracasso dramático do sistema financeiro induziu uma procura desesperada de soluções através de um regresso ao keynesianismo brando e liderado pelas finanças e ao controlo estatal fragmentado e reativo, até mesmo à propriedade pública das finanças e da indústria, o que teria sido um anátema apenas alguns meses antes. As acrobacias ideológicas necessárias para justificar estas escolhas políticas, bem como as deficiências nos mecanismos institucionais de formulação e implementação de políticas, eram demasiado óbvias. Mesmo assim, as medidas extraordinariamente caras envolvidas no “resgate” da economia foram iniciadas pelo presidente ultraneoliberal dos EUA, G.W. Bush no crepúsculo da sua administração e foram prosseguidos sem problemas pelo seu sucessor, presumivelmente muito diferente, Barack Obama. A mesma continuidade fundamental entre actores políticos distintos também foi observada no Reino Unido, França, Itália e muitos outros países. Invariavelmente, as políticas de resposta à crise eram inequivocamente neoliberais e deveriam ser revertidas o mais rapidamente possível. Para colocar em perspectiva a profundidade da crise financeira e a extensão da intervenção estatal, dois factos são surpreendentes. Uma delas é que os recursos oferecidos para apoiar o sistema financeiro excedem em muito a receita total acumulada em todas as privatizações de sempre. A outra é que os pacotes de resgate teriam sido suficientes para eliminar a pobreza mundial durante os próximos {\color{blue}50} anos, se não indefinidamente.
 \par 
\section{A crise da financeirização}
 \par 
A um nível mais profundo, o neoliberalismo está ligado a uma mistura específica de ideologia, estudos e política na prática. Mas esta mistura passou por duas fases: a primeira, a fase de choque, baseou-se numa ampla intervenção estatal para promover o capital privado tanto quanto possível, com uma atenção limitada às consequências sociais, económicas e políticas - um Reagan/Thatcherismo que foi mais notoriamente imposto na Europa Oriental sob esta mesma terminologia de terapia de choque. Mas o espírito do “just do it” da primeira fase do neoliberalismo (que falava em deixar as coisas ao mercado, mas usava o Estado para promover o capital privado - sobretudo nas suas relações opressivas com os trabalhadores) não se originou nem foi confinado para economias em transição. A segunda fase, o Terceiro Wayismo ou o “mercado social”, que continua até hoje, testemunhou diferentes modalidades de intervenção estatal, tanto para atenuar os piores efeitos da primeira fase como, mais importante, para sustentar o que se tornou a característica definidora do próprio neoliberalismo: a financeirização. Nos últimos {\color{blue}40} anos, a financeirização prosperou através, e sob o pretexto, da promoção do mercado (ou seja, do capital privado) em geral. Na prática, isto significa a subordinação da reprodução social aos imperativos do mercado financeiro em tudo, desde a privatização e a desregulamentação até às metas de inflação, à comercialização de serviços públicos e à difusão do crédito pessoal e dos seguros privados, em oposição à dependência do bem-estar social.
 \par 
Inevitavelmente, a crise traz à tona a importância das finanças. É difícil exagerar a expansão do sistema financeiro nos últimos {\color{blue}40} anos. Tem havido uma proliferação e crescimento dos próprios mercados financeiros, em termos de derivados, futuros, câmbio, hipotecas, instrumentos governamentais, bem como acções e participações, e a penetração das finanças em áreas de reprodução económica e social que tinham sido removido do controle direto do capital privado na era anterior do bem-estar keynesiano e da “modernização”. Isto aplica-se à saúde, educação, energia, telecomunicações, transportes, financiamento habitacional, pensões, benefícios, assistência social e muito mais. Além disso, as empresas industriais foram completamente apanhadas pela financeirização, com um impulso para o “valor para os accionistas” através de negociações financeiras, reestruturações e mudanças na governação corporativa que dominam as fontes de rentabilidade, muitas vezes à custa do investimento para expandir e melhorar a capacidade e aumentar a produtividade.
 \par 
Estas considerações económicas estão inseridas num novo padrão de imperialismo (a chamada “globalização”), nomeadamente na sequência da Guerra Fria. Tanto os pontos fortes como os pontos fracos dos Estados Unidos como potência hegemónica intensificaram-se e foram expostos nos últimos anos. Em contraste, o colapso do socialismo de estilo soviético e a fraqueza dos movimentos progressistas, apesar de alguns rebentos verdes, na América Latina, por exemplo, são impressionantes. O mesmo acontece com a ascensão da China, a sua conversão ao capitalismo e o fornecimento de trabalho assalariado ao capitalismo mundial, totalizando dezenas, senão centenas de milhões de trabalhadores. Igualmente significativa é a relação peculiar da China com os Estados Unidos, no que diz respeito ao grande apoio que oferece à reciclagem dos défices fiscais, comerciais e da balança corrente dos EUA. A China está longe de estar sozinha nisto, mesmo no mundo “em desenvolvimento”, e a Alemanha e o Japão têm sido pelo menos igualmente importantes na sustentação do dólar e do défice comercial dos EUA durante ainda mais tempo. Isto revela uma combinação extraordinária de força e fraqueza dos EUA, com o dólar como moeda mundial a comandar o apoio externo: no momento em que este artigo foi escrito, quaisquer movimentos para substituir os seus papéis correspondentes como moeda de reserva e meio de pagamento eram, no máximo, marginais. O resultado é que o valor do dólar tem sido volátil; mas não caiu, apesar da sua fragilidade potencial e das fraquezas estruturais amplamente reconhecidas da economia dos EUA - fraquezas do tipo que levariam ao colapso do valor de qualquer outra moeda.
 \par 
\section{A crise da financeirização}
 \par 
Não é de surpreender que, à medida que a ortodoxia se debateu durante a crise, os estudos e comentários marxistas e heterodoxos tenham assumido um papel mais proeminente. A questão, contudo, é menos de observar do que de explicar, o que exige localizar estes desenvolvimentos num quadro analítico. Em particular, três questões precisam ser confrontadas. Em primeiro lugar, estão as razões para o abrandamento dos últimos {\color{blue}40} anos, especialmente dadas as condições que não poderiam ter sido mais propícias à acumulação de capital, incluindo incentivos legais e regulamentares ao capital, níveis estagnados, se não mesmo decrescentes, de dinheiro e salários sociais, fraqueza do trabalho e movimentos progressistas, expansão e “flexibilidade” da força de trabalho global e hegemonia neoliberal na política, na política e na ideologia. Sem uma explicação para o abrandamento, é impossível explicar por que razão surgiu uma tal crise financeira e por que foi tão grave, e especificar qual é a natureza da crise em si, para além dos seus parâmetros económicos imediatos.
 \par 
O segundo é desvendar a importância da financiarização e a sua relação com a acumulação de capital (produtivo). Paradoxalmente, embora as finanças e a financeirização tenham atraído grande atenção por parte dos académicos marxistas, tem havido relativamente poucas tentativas de incorporar as finanças na análise do próprio Marx. Isto estende-se até à tradição estabelecida por e através de Rudolf Hilferding - até porque, sem dúvida, porque a sua noção de capital financeiro parece insuficientemente sintonizada com a diversidade e extensão da financiarização actual, que vai muito além da relação entre bancos e indústria. Apesar da compreensível atração da economia política marxista à luz da crise, muito mais atenção tem sido dada a Hyman Minsky do que a Karl Marx quando se trata do papel das finanças na crise.
 \par 
A terceira é como localizar o papel da luta de classes nestas circunstâncias, nas quais ela parece ao mesmo tempo fraca e afastada da sua localização clássica para o marxismo, no ponto de produção. É claro que um dos mantras do neoliberalismo é a “flexibilidade” nos mercados de trabalho, que, na prática, é imposta em nome do capital através da intervenção estatal através da legislação e, quando necessário, do autoritarismo. Isto contribuiu para o declínio cumulativo da força, da organização e do activismo da classe trabalhadora, enquanto a influência do trabalho organizado na reprodução social também foi enfraquecida através da despolitização, da desorganização, da privatização, do declínio da segurança no emprego, e assim por diante. Estes colocam desafios analíticos e estratégicos, que, mesmo antes da crise, foram abordados em termos de argumentos que vão desde o “desaparecimento” da classe trabalhadora e do capitalismo tal como os conhecíamos até à emergência de novas (mais ou menos anticapitalistas) ) movimentos sociais.
 \par 
Além destas três questões analíticas - o abrandamento, a financiarização e o papel da classe - há uma quarta questão estratégica: como responder às terríveis circunstâncias da crise económica e dos movimentos progressistas enfraquecidos. A relação entre a reforma dentro do capitalismo e a revolução socialista para o transcender levanta o clássico enigma marxista de como fazer avançar uma sem comprometer a outra. Mas, actualmente, estas considerações parecem um luxo utópico, uma vez que, apesar da gravidade da crise económica e da correspondente crise de legitimidade do neoliberalismo, tanto a reforma radical como a revolução estão fora da agenda.
 \par 
A nossa própria abordagem a estas três questões analíticas consiste em implantar e desenvolver a teoria da acumulação de Marx, tanto lógica como historicamente, com base nas categorias de análise oferecidas nos três volumes de O Capital. Argumentámos que a teoria de Marx aborda a acumulação como a expansão quantitativa do capital produtivo através da sua reestruturação contínua e desigual, geralmente em unidades maiores e mais complexas, organizadas, no mundo de hoje, principalmente através de corporações transnacionais. Crucialmente, porém, o ritmo e o ritmo da reestruturação do capital dependem em grande parte de outras agências que não os próprios capitalistas industriais, especialmente das políticas estatais e da classe trabalhadora, e da reestruturação de outros capitais em mercados concorrentes e nas finanças, bem como através de transformações mais gerais da vida económica e social. Cada um destes elementos pode ser mais ou menos propício à acumulação por reestruturação, bem como ter efeitos desiguais. O seu impacto depende das mudanças nas configurações e dos conflitos de interesses económicos, políticos e ideológicos dentro dos limites estabelecidos pelo sistema de acumulação como um todo. O papel do Estado é fundamental em todos estes factores constituintes, incluindo a política económica implementada em conjunto com o exercício da força, e a política estatal.
 \par 
Esta explicação abstracta pode ser desenvolvida enfatizando, como já foi indicado, que o actual abrandamento não se deve à força ou à militância da classe trabalhadora e, consequentemente, que as explicações para a crise devem ser procuradas nas relações intracapitalistas. Em particular, crucial para a explicação é o processo de financeirização - algo que é agora reconhecido por todos. Mas isto parece um pouco diferente uma vez definido nas categorias de análise de Marx. Pois o que marca a financeirização na era neoliberal, como foi sugerido no Capítulo 12, é a expansão do capital portador de juros (IBC) em toda a economia, incluindo as operações financeiras de supostas corporações industriais independentes, bem como na saúde. , educação, bem-estar, crédito ao consumo, habitação e assim por diante. Assim, em formas híbridas, o IBC tem promovido activamente a acumulação de capital financeiro (fictício) à custa de activos produtivos. Embora rentável para os capitais individuais, e no curto prazo, isto tem sido disfuncional para a acumulação sustentada de capital em geral, tanto quantitativa como qualitativamente.
 \par 
Em suma, a financeirização é sustentada pela expansão quantitativa do IBC e pela sua extensão por toda a economia, por vezes impulsionando a reestruturação do capital industrial, e por vezes à custa dele, influenciando assim, directa e indirectamente, o impacto mais amplo do neoliberalismo sobre a sociedade. reprodução. A acumulação de activos financeiros tem tido prioridade, tanto sistemicamente como em termos políticos, sobre a acumulação de capital industrial, apesar (e, em certa medida, por causa) do rápido crescimento do proletariado em todo o mundo. Isto é surpreendentemente revelado na actual crise pela medida em que o Estado interveio em nome das finanças, quando, em circunstâncias muito mais favoráveis, foram negadas despesas de proporções muito mais modestas, não só para a saúde, a educação e o bem-estar, mas também para o
 \par 
\section{A crise da financeirização}
 \par 
Dada a nossa compreensão da desaceleração, da crise e dos fundamentos financeiros do neoliberalismo, como iremos localizar a luta de classes e a divisão reforma/revolução? Consideremos três posições extremas, possivelmente caricaturadas. Percebemos as finanças apenas como um fenómeno, o que implica que a estratégia deve centrar-se na classe trabalhadora, organizada no ponto de produção. O problema aqui é que esse activismo provou ser fraco e possivelmente enfraquecido, e estar desligado das lutas em torno de questões que irão, por necessidade, proliferar fora da produção - por exemplo, sobre salários, benefícios e provisão social, mas também sobre e em torno das catástrofes ambientais desencadeadas pelo capitalismo global. O segundo extremo é ignorar tanto a crise económica como as realidades da produção e concentrar-se, em vez disso, nos confrontos contínuos em torno do ambiente, das escolhas de estilo de vida e da multiplicidade de discriminações rotineiramente (re)produzidas pelo capitalismo contemporâneo. Por mais significativas que estas preocupações possam ser, a tentativa de enfrentá-las separadamente das suas raízes estruturais na produção provavelmente não terá mais sucesso no futuro do que tem sido no passado recente. A terceira é concentrar-se em algo semelhante ao ataque à “exploração em troca” pelas finanças, com base na antipatia popular para com os banqueiros desacreditados, ao mesmo tempo que contorna as questões sistémicas colocadas pela financeirização da produção e da reprodução social sob o neoliberalismo. Existem problemas analíticos e políticos significativos em colocar questões puramente em termos de finanças versus o resto de nós, quaisquer que sejam os méritos que isso possa ter como ponto de partida estratégico e oportuno. Por exemplo, e para reiterar o ponto anterior, o que dizer de outras formas de exploração e opressão, especialmente na própria produção, para as quais a reforma do sistema financeiro oferece pouco em termos de compra?
 \par 
Uma alternativa não é tanto rejeitar os três extremos que acabamos de apresentar, mas ir além deles, ligando a produção e as classes às lutas específicas engendradas pela reprodução económica e social sob o neoliberalismo. Como deveria ser evidente, as formas como a financeirização interveio na reprodução económica e social são ao mesmo tempo generalizadas e heterogéneas e o mesmo acontecerá, consequentemente, com as reacções mais ou menos espontâneas aos seus efeitos e com a procura de alternativas. De uma perspectiva marxista, e também de outras, é muito mais fácil ver a necessidade de destruir o sistema financeiro do que concretizá-lo ou ligá-lo a movimentos mais enraizados, eficazes e seguros para a transformação económica e social. Como disse Marx em O Dezoito Brumário de Luís Napoleão (1852):
 \par 
Os homens fazem a sua própria história, mas não a fazem como querem; eles não o fazem em circunstâncias escolhidas por eles próprios, mas em circunstâncias diretamente encontradas, dadas e transmitidas pelo passado. A tradição das gerações mortas pesa como um pesadelo no cérebro dos vivos.
 \par 
O que é verdade para o nosso cérebro é igualmente verdade para as nossas circunstâncias materiais. As crises no financiamento hipotecário (e a sua ligação ao fornecimento de habitação) são distintas das crises ambientais (e do impulso neoliberal para o comércio de futuros de carbono, por exemplo, que se transformou num vasto novo mercado que cria lucros reais enquanto finge para enfrentar os desastres ambientais do capitalismo) e de crises nas esferas produtivas públicas e privadas, seja na saúde, na educação ou no bem-estar. Necessariamente, estas arenas de luta serão tão diversas como o são as alianças que poderão ser formadas para desafiar facetas específicas do neoliberalismo, e que podem ajudar a fortalecer, alargar e transformar lutas individualizadas, muitas vezes financeiras, no sentido de uma visão renovada de modos alternativos de luta. prestação baseada nos valores do controlo democrático e da solidariedade, e não na extracção e distribuição de mais-valia. É pouco provável que esta transformação aconteça espontaneamente: uma plataforma positiva para a mobilização social, inspirada numa análise cuidadosa e na compreensão teórica, continua a ser essencial. Neste sentido, a contribuição oferecida pelas análises e experiências de luta marxistas permanece indispensável. Tais prognósticos estão lado a lado com o slogan que marca o epitáfio na lápide de Marx, uma citação da sua décima primeira tese sobre Feuerbach: “Até agora, os filósofos apenas interpretaram o mundo de várias maneiras; o objetivo é mudar isso.
 \par 
Tal como acontece com muitos dos escritos de Marx, este apelo aos socialistas do século XIX deve ser interpretado tanto como um meio de obter compreensão como também como um imperativo de acção. Permanece válido no século XXI, à medida que procuramos abolir a sociedade capitalista, recorrendo à reacção contra as contradições e desigualdades que ela provoca, ao seu estudo através das melhores ferramentas das ciências sociais e, mais importante, às experiências práticas dos luta de uma multiplicidade de grupos, associações, sindicatos, organizações políticas e das massas de milhões que lhes dão vida.
 \par 
\section{A crise da financeirização}
 \par 
Em geral, a literatura marxista sobre a financeirização e a crise actual divide-se entre aqueles que pensam que as finanças são cruciais e aqueles que pensam que não o são, e entre aqueles que argumentam que a crise actual é uma consequência retardada do fracasso na resolução das contradições da acumulação. no período pós-guerra («keynesiano») e aqueles que consideram que a actual crise se deve à reestruturação financeira e às suas consequências sociais e económicas. A financeirização foi examinada sob diferentes perspectivas na literatura marxista; para uma revisão, ver Fine (2012b, 2014). Muito também foi escrito sobre a crise em curso; ver, por exemplo, Gérard Duménil e Dominique Lévy (2011), David McNally (2011), Leo Panitch e Martijn Konings (2008), Martijn Konings e Leo Panitch (2008) e edições recentes do Cambridge Journal of Economics, New Left Review, Historical Materialism and the Socialist Register, e a riqueza de material disponível nos sites Dollars and Sense (www.dollarsandsense.org) e Socialist Project (www.socialistproject.ca), entre muitos outros. Todo e qualquer jornal e site de esquerda dedicado à economia política ou não incluirá uma grande quantidade de leituras úteis. No entanto, a financeirização tornou-se rapidamente tão difundida e amorfa na sua utilização nas ciências sociais (exceto na economia dominante, onde está notavelmente ausente), que Brett Christophers (2015a eb), em debate com os críticos, negou-lhe qualquer compra analítica. Esta é uma posição que consideramos insustentável uma vez que a financeirização é vista em termos da expansão extensiva (para novas áreas de actividade) e intensiva (dentro das áreas existentes) do capital portador de juros num contexto neoliberal. Veja especialmente Kate Bayliss et al. (2015).
 \par 
