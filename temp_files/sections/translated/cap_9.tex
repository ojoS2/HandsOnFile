\chapter{SUGESTÕES SELECIONADAS PARA LEITURA ADICIONAL}\label{SUGESTÕES SELECIONADAS PARA LEITURA ADICIONAL}
 \par 
Adamczak, Bini. Comunismo para crianças. Cambridge, MA: MIT Press,
 \par 
2017.
 \par 
Appel, Hilary e Mitchell Orenstein. Do triunfo à crise: reforma econômica neoliberal em países pós-comunistas. Cambridge, Reino Unido: Cambridge University Press, 2018.
 \par 
Aronoff, Kate, Peter Drier e Michael Kazin. We Own lhe Future: Democratic Socialism—American Style (Nós somos donos do futuro: socialismo democrático — estilo americano). Nova York: New Press, 2020.
 \par 
\section{Bastani, Aaron. Comunismo de Luxo Totalmente Automatizado. Nova York:}
 \par 
Versão, 2019.
 \par 
Bebel, agosto. Mulher e Socialismo. Traduzido por Meta L. Stern. Nova York: Socialist Literature, 1910. Disponível online em www. marxists.org.
 \par 
Bóxer, Marilyn J., e Jean H. Quataert. Socialist Women: European Socialist Feminism in lhe Nineteenth ano Early Twentieth Centuries (Mulheres Socialistas: Feminismo Socialista Europeu nos Séculos XIX e Início do XX). Nova York: Elsevier, 1978.
 \par 
\section{Bregman, Rutger. Utopia para Realistas: Como Podemos Construir o Ideal}
 \par 
\section{Mundo. Nova York: Little, Brown, 2017.}
 \par 
\[Brown, Wendy. Desfazendo o Demos: A Revolução Furtiva do Neoliberalismo.\]
 \par 
\section{Cambridge, MA: Zone Books, 2017.}
 \par 
Bucur, Maria. O Século das Mulheres: Como as Mulheres Transformaram o Mundo Desde 1900. Lanham, MD: Rowman & Littlefield, 2018.
 \par 
Bucur, Maria e Mihaela Miroiu. Nascimento da Cidadania Democrática: Mulher e Poder na Romênia Moderna. Bloomington: Indiana University Press, 2018.
 \par 
\section{Carleton, Gregory. Revolução Sexual na Rússia Bolchevique. Pittsburgh:}
 \par 
\section{Imprensa da Universidade de Pittsburgh, 2005.}
 \par 
179
 \par 
180
 \par 
SUGESTÕES SELECIONADAS PARA LEITURA ADICIONAL
 \par 
Clements, Barbara Evans. Feminista bolchevique: a vida de Aleksandra Kollontai. Bloomington: University of Indiana Press, 1979. Cobble, Dorothy Sue. O movimento das outras mulheres: justiça no local de trabalho e direitos sociais na América moderna. Princeton, NJ: Princeton University Press, 2005.
 \par 
Cobble, Dorothy Sue, Linda Gordon e Astrid Henry. Feminismo inacabado: uma breve e surpreendente história dos movimentos de mulheres americanas. Nova York: W. W. Norton, 2015.
 \par 
Cohen, G. A. Por que não o socialismo? Princeton, NJ: Universidade de Princeton-
 \par 
\section{Editora Sity, 2009.}
 \par 
Collins, Patricia Hill. Black Feminist Thought: Conhecimento, Consciência e a Política do Empoderamento. Nova York: Routledge, 2008.
 \par 
Davis, Angela. Mulheres, Raça e Classe. Nova York: Vintage, 1983. Dodge, Norton T. Mulheres na Economia Soviética. Baltimore: Johns
 \par 
\section{Imprensa da Universidade Hopkins, 1966.}
 \par 
\section{Drakulic, Slavenka. Como sobrevivemos ao comunismo e até mesmo}
 \par 
\section{Riu. Reimpressão. Nova York: Harper Perennial, 2016.}
 \par 
Du Plessix Gray, Francine. Mulheres soviéticas. Nova York: Bantam
 \par 
\section{Doubleday Dell, 1990.}
 \par 
Ehrenreich, Barbara e Arlie Russell Hochschild. Global Woman: Babás, empregadas domésticas e trabalhadoras do sexo na nova economia. Nova York: Henry Holt, 2004.
 \par 
\section{Elwood, R. C. Inessa Armand: Revolucionária e Feminista.}
 \par 
\section{Cambridge, Reino Unido: Cambridge University Press, 1992.}
 \par 
\[Engels, Friedrich. A Origem da Família, da Propriedade Privada e da\]
 \par 
\section{Estado. 1884. Disponível online em www.marxists.org.}
 \par 
\section{Evans, Kate. Rosa Vermelha: Uma Biografia Gráfica de Rosa Luxemburgo.}
 \par 
\section{Londres: Verso, 2015.}
 \par 
Featherstone, Liza. Vendendo Mulheres a descoberto: A Batalha Histórica pelos Direitos dos Trabalhadores no Walmart. Nova York: Basic Books, 2005. Federici, Silvia. Caliban e a Bruxa: Mulheres, o Corpo e a Primi-
 \par 
\section{Tive Accumulation. Nova Iorque: Autonomedia, 2004.}
 \par 
. Revolução no Ponto Zero: Trabalho Doméstico, Reprodução e Luta Feminista. Nova York: Noções Comuns, 2012.
 \par 
\section{Feffer, John. Aftershock: Uma viagem à Europa Oriental quebrada}
 \par 
\section{Sonhos. Londres: Zed, 2017.}
 \par 
\section{Fidelis, Malgorzata. Mulheres, Comunismo e Industrialização em}
 \par 
\section{Polônia pós-guerra. Nova York: Cambridge University Press, 2010.}
 \par 
SUGESTÕES SELECIONADAS PARA LEITURA ADICIONAL
 \par 
\section{Firestone, Shulamith. A dialética do sexo: o caso de uma feminista}
 \par 
\section{Revolução. Nova York: Farrar, Straus e Giroux, 2003.}
 \par 
Fisher, Mark. Realismo Capitalista: Não Há Alternativa? Ropley,
 \par 
\section{Hampshire, Reino Unido: Zero Books, 2009.}
 \par 
Fitzpatrick, Sheila. Stalinismo cotidiano: vida comum em tempos extraordinários: Rússia soviética na década de 1930. Nova York: Oxford University Press, 2001.
 \par 
TTTTTT ismo para Crise Neoliberal. Nova York: Verso, 2013.
 \par 
. Scales of. Justice: Reimagining Political Space in a Globalizing World (Escalas de Justiça: Reimaginando o Espaço Político em um Mundo Globalizante). Nova York: Columbia University Press, 2010.
 \par 
\section{Fraser, Nancy. Fortunas do Feminismo: Do ​​Capital Gerido pelo Estado-}
 \par 
\section{Frink, Helen H. Mulheres depois do comunismo: a experiência da Alemanha Oriental}
 \par 
Fullbrook, Mary. O Estado Popular: Sociedade da Alemanha Oriental de Hitler a Honecker. New Haven, CT: Yale University Press, 2008. Gal, Susan e Gail Kligman. A Política de Gênero Após o Socialismo: Um Ensaio Histórico-Comparativo. Princeton, NJ: Princeton University Press, 2000.
 \par 
Editores. Reproduzindo Gênero: Política, Públicos e Vida Cotidiana Após o Socialismo. Princeton, NJ: Princeton University Press, 2000.
 \par 
Ghodsee, Kristen. O Lado Esquerdo da História: Segunda Guerra Mundial e a Promessa Não Cumprida do Comunismo na Europa Oriental. Durham, NC: Duke University Press, 2015.
 \par 
— —. Vidas muçulmanas na Europa Oriental: gênero, etnia e a transformação do islamismo na Bulgária pós-socialista, Princeton, NJ: Princeton University Press, 2009.
 \par 
\section{Ence. Lanham, MD: University Press of. America, 2001.}
 \par 
\section{— —. Ressaca Vermelha: Legados do Comunismo do Século XX.}
 \par 
\section{Durham, Carolina do Norte: Duke University Press, 2017.}
 \par 
Mar Negro. Durham, NC: Duke University Press, 2005. . Segundo Mundo, Segundo Sexo: Ativismo Socialista de Mulheres e Solidariedade Global Durante a Guerra Fria. Durham, NC: Duke University Press, 2019.
 \par 
Goldman, Wendy Z. Mulheres, o Estado e a Revolução: Política Familiar Soviética e Vida Social, 1917-1936. Cambridge, Reino Unido: Cambridge University Press, 1993.
 \par 
Grant, Melissa Gira. Brincando de prostituta: o trabalho do trabalho sexual.
 \par 
\[— —. A Riviera Vermelha: Gênero, Turismo e Pós-socialismo na\]
 \par 
181
 \par 
182
 \par 
SUGESTÕES SELECIONADAS PARA LEITURA ADICIONAL
 \par 
Harsch, Donna. A vingança do doméstico: mulheres, a família e o comunismo na República Democrática Alemã. Princeton, NJ: Princeton University Press, 2008.
 \par 
Hensel, Jana. After lhe Wall: Confessions of. an. East German Childhood ano lhe Life That Came Next. Transitado por Jefferson Chase. Nova York: Public Affairs, 2004.
 \par 
Herzog, Dagmar. Sexo depois do fascismo: memória e moralidade na Alemanha do século XX. Princeton, NJ: Princeton University Press, 2005.
 \par 
\section{Nova York: Verso, 2014.}
 \par 
\section{— . Sexualidade na Europa: Uma História do Século XX. Cam-}
 \par 
Hochschild, Arlie Russell. The Managed Heart: Comercialização do Sentimento Humano. Berkeley: University of California Press, 2012. Holmstrom, Nancy. The Socialist Feminist Project: Um Leitor Contemporâneo em Teoria e Política. Nova York: Monthly Review Press, 2004.
 \par 
Holt, Alix. Alexandra Kollontai: Selected Writings. Nova Iorque: W. W.
 \par 
\section{Bridge, Reino Unido: Cambridge University Press, 2011.}
 \par 
\section{Norton, 1980.}
 \par 
\[Honneth, Axel. A Ideia do Socialismo: Rumo a uma Renovação. Londres:\]
 \par 
Humphrey, Caroline. O Desfazer da Vida Soviética: Economias Cotidianas Após o Socialismo. Ithaca, NY: Cornell University Press, 2002.
 \par 
\section{Política 2018.}
 \par 
\section{Illouz, Eva. Intimidades frias: a construção do capitalismo emocional.}
 \par 
\section{Londres: Polity, 2007.}
 \par 
\section{Editor. Emoções como mercadorias: capitalismo, consumo}
 \par 
. Porque o amor machuca: uma explicação sociológica? Londres: Polity,
 \par 
2013.
 \par 
Jaffe, Sarah. Problema necessário: americanos em revolta. Nova York: Na-
 \par 
\[E Autenticidade. Nova York: Routledge, 2019.\]
 \par 
Kligman, Gail. A política da duplicidade: controle. Reprodução na Romênia de Ceausescu. Berkeley: University of California Press, 1998.
 \par 
\section{Livros Tion, 2017.}
 \par 
\section{Kollontai, Alexandra. A autobiografia de uma mulher sexualmente emancipada}
 \par 
Krylova, Anna. Mulheres soviéticas em combate: uma história de violência na Frente Oriental. Nova York: Cambridge University Press, 2011.
 \par 
SUGESTÕES SELECIONADAS PARA LEITURA ADICIONAL
 \par 
Lapidus, Gail Warshofsky. Mulheres na Sociedade Soviética: Igualdade, Desenvolvimento e Mudança Social. Berkeley: University of California Press, 1978.
 \par 
\section{Mulher Comunista. Nova York: Prism Key, 2011.}
 \par 
\[Leo, Maxim. Red Love: A História de uma Família da Alemanha Oriental. Londres:\]
 \par 
Liskova, Katefina. Sexual Liberation, Socialist Style: Communist Czechoslovakia ano lhe Science of. Desire, 1945-89. Cambridge, Reino Unido: Cambridge University Press, 2018.
 \par 
\section{Pushkin, 2014.}
 \par 
\[Lorand, Zsofia. O Desafio Feminista ao Estado Socialista na Iugoslávia-\]
 \par 
Lorde, Audre. Sister Outsider: Ensaios e discursos. Nova York: Crossing, 2007.
 \par 
\section{Slavia. Londres: Palgrave MacMillan, 2018.}
 \par 
\section{Luxemburgo, Rosa. Reforma ou Revolução e Outros Escritos. Novo}
 \par 
McDuffie, Erik S. Sojourning for Freedom: Mulheres Negras, Comunismo Americano e a Criação do Feminismo de Esquerda Negra. Durham, NC: Duke University Press, 2011.
 \par 
McLellan, Josie. Amor na Época do Comunismo: Intimidade e Sexualidade na RDA. Nova York: Cambridge University Press, {\color{blue}20} in.
 \par 
\section{York: Dover Books, 2006.}
 \par 
\section{McNeal, Robert. Noiva da Revolução: Krupskaya e Lenin. Ann}
 \par 
\section{Arbor: Universidade de Michigan Press, 1972.}
 \par 
\[Meyer, Alfred G. O Feminismo e Socialismo de Lily Braun. Florescer-\]
 \par 
Millar, James R. Política, Trabalho e Vida Cotidiana na URSS: Uma Pesquisa de Antigos Cidadãos Soviéticos. Nova York: Cambridge University Press, 1987.
 \par 
\section{Inglês: Indiana University Press, 1985.}
 \par 
\section{Millet, Kate. Política Sexual. Reimpressão. Nova York: Columbia University-}
 \par 
Moraga, Cherrie e Glória E. Anzaldua. This Bridge Called My Back: Escritos de Mulheres Radicais de Cor. 4ª ed. Albany: SUNY Press, 2015.
 \par 
Moskowitz, P. E. O caso contra a liberdade de expressão: a primeira emenda, o fascismo e o futuro da dissidência. Nova York: Bold Type Books, 2019. . Como matar uma cidade: gentrificação, desigualdade e a luta pela vizinhança. Nova York: Bold Type Books, 2018.
 \par 
\section{Editora Sity, 2016.}
 \par 
\[Nicholas, John. A Palavra S: Uma Breve História de uma Tradição Americana\]
 \par 
183
 \par 
184
 \par 
SUGESTÕES SELECIONADAS PARA LEITURA ADICIONAL
 \par 
Penn, Shauna e Jill Massino. Política de gênero e vida cotidiana na Europa Central e oriental socialista estatal. Nova York: Palgrave McMillian, 2009.
 \par 
Piketty, Thomas. Capital e Ideologia. Cambridge, MA: Belknap,
 \par 
2020. . Capital no Século XXI. Cambridge, MA: Harvard University Press, 2014.
 \par 
\section{Tion... Socialismo. 2ª ed. Nova York: Verso, 2015.}
 \par 
\section{Porter, Cathy. Alexandra Kollontai: Uma Biografia. Chicago: Haymar-}
 \par 
Ruthchild, Rochelle Goldberg. Igualdade e Revolução: Direitos das Mulheres no Império Russo, 1905-1917. Pittsburgh: University of Pittsburgh Press, 2010.
 \par 
Sanders, Bernie. Nossa Revolução. Nova York: Thomas Dunne Books,
 \par 
2016.
 \par 
Stites, Richard. O Movimento de Libertação das Mulheres na Rússia: Feminismo, Niilismo e Bolchevismo, 1860-1930. Princeton, NJ: Princeton University Press, 1978.
 \par 
\section{Livros Ket, 2014.}
 \par 
\section{Stokowski, Margarete, Os Últimos Dias do Patriarcado. Berlim:}
 \par 
Sunkara, Bhaskar, editor. O ABC do Socialismo. Londres: Verso,
 \par 
2016. . O Manifesto Socialista: O Caso da Política Radical em uma Era de Desigualdade Extrema. Nova York: Basic Books, 2019.
 \par 
\section{Editoras de livros Rowohlt, 2018.}
 \par 
\section{Taylor, Keeanga-Yamahtta. De #BlackLivesMatter para Black Libera-}
 \par 
\section{Tação. Chicago: Haymarket Books, 2016.}
 \par 
\section{Traister, Rebecca, Good ano Mad: O poder revolucionário das mulheres}
 \par 
Vaizey, Hester. Nascido na RDA: Vivendo na Sombra do Muro.
 \par 
\section{A raiva de En. Nova York: Simon ano Schuster, 2018.}
 \par 
\section{São Paulo: Editora UFMG, 2017.}
 \par 
Breve História do Capitalismo. Londres: Bodley Head, 2017.
 \par 
\section{Varoufakis, Yanis. Conversando com minha filha sobre economia: uma}
 \par 
\section{Verdery, Katherine. O que era o socialismo e o que vem depois?}
 \par 
Wang, Zheng. Encontrando Mulheres no Estado: Uma Revolução Socialista Feminista na República Popular da China, 1949-1964. Berkeley: University of California Press, 2017.
 \par 
Weeks, Kathi. O Problema com o Trabalho: Feminismo, Marxismo, Política Antitrabalho e Imaginários Pós-Trabalho. Durham, NC: Duke University Press, 2011.
 \par 
SUGESTÕES SELECIONADAS PARA LEITURA ADICIONAL
 \par 
Weigand, Kate. Feminismo Vermelho: Comunismo Americano e a Criação do Movimento das Mulheres. Baltimore: Johns Hopkins University Press, 2001.
 \par 
\section{Princeton, NJ: Princeton University Press, 1996.}
 \par 
\[Weigel, Moira. Trabalho de Amor: A Invenção do Namoro. Nova Iorque:\]
 \par 
Yalom, Marilyn. A História da Esposa. Nova York: Harper Perennial, 2002.
 \par 
\section{Farrar, Straus e Giroux, 2016.}
 \par 
\[Zizek, Slavoj. A Coragem da Desesperança: Um Ano de Atuação Perigosa-\]
 \par 
185
 \par 
\begin{figure}
	\centering
	\includegraphics[width=1.\textwidth]{temp\_files/images/UP\_logo.png }
	\caption{Nadezhda Krupskaya (1869-1939): Uma pedagoga russa radical e uma figura proeminente no movimento comunista pré-revolucionário. Ela serviu como vice-ministra da educação da União Soviética por dez anos e é amplamente creditada por ajudar a construir um sistema de educação em massa e expandir uma rede de bibliotecas por toda a URSS. Krupskaya esteve envolvida na fundação das organizações de jovens soviéticas, os Jovens Pioneiros e o Komsomol. Cortesia de Domínio Público (Rússia).}
	\label{ }
\end{figure}