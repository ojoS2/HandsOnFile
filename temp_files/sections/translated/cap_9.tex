 
 \chapter{Remembrance of Empires Past}  

 \label{Remembrance of Empires Past}  
 
 
\par
 
 
 \textit{	In 2000, I spent the better part of a late summer interviewing William F. Buckley and Irving Bristol. I was writing an article for Lingua Franca (see chapter {{\color{blue} 5} }) on the defections to the left of right-wing intellectuals and wanted to hear what the movement’s founding fathers thought of their wayward sons. Over the course of our conversations, however, it became clear that Buckley and Bristol were less interested in these ex-conservatives than they were in the sorry state of the conservative movement and the uncertain fate of the United States as a global empire. The end of Communism and the triumph of the free market, they told me, were mixed blessings. While they were conservative victories, these developments had nevertheless rendered the United States ill-equipped for the postcold War era. Americans now possessed the most powerful empire in history. At the same time, they were possessed by one of the most anti-political ideologies in history: the free market.}  

 
\par
 
 
 
\par
 

 \textbf{\textit{	— Henry IV, Part 2} }  

 
\par
 

 \footnote{This chapter originally appeared as “Remembrance of Empires Past: 9/11 and the End of the Cold War,” in Cold War Triumphalism: The Misuse of History after the Fall of Communism, ed. Ellen Checker (New York: New Press, 2004), 274–297.}  
Conforme os seus idealistas, ou pelo menos um dos seus idealistas, o mercado livre é uma ordem harmoniosa, prometendo uma sociedade civil internacional de troca voluntária, exigindo pouco mais do Estado do que a aplicação ocasional de leis e contratos. Para Buckley e Bristol, esta era uma noção demasiado incruenta para fundar uma ordem nacional, muito menos um império global. Não proporcionou a paixão e o elã, a seriedade e a autoridade que o exercício do poder americano realmente exigia, no país e no estrangeiro. Encorajou a trivialidade e a política mesquinha, o interesse próprio acima do interesse nacional – o que não é a base mais promissora para lançar um império. Além disso, os direitistas responsáveis ​​pelo Partido Republicano não pareciam perceber isso. “O problema com a ênfase no conservadorismo no mercado”, disse-me Buckley, como vimos no capítulo 5, “é que se torna entediante. Você ouve uma vez e domina a ideia. A ideia de dedicar sua vida a isso é horrível, apenas porque é muito repetitivo. É como sexo. O conservadorismo, acrescentou Bristol, “é tão influenciado pela cultura empresarial e pelos modos de pensar empresariais que lhe falta qualquer imaginação política, o que sempre foi, devo dizer, uma propriedade da esquerda”. Bristol confessou um profundo anseio por um império americano: “Qual é o sentido de ser a maior e mais poderosa nação do mundo e não ter um papel imperial? É algo inédito na história da humanidade. A nação mais poderosa sempre teve um papel imperial.” Mas, continuou ele, os impérios anteriores não eram “democracias capitalistas com uma forte ênfase no crescimento econômico e na prosperidade econômica”. Devido ao seu compromisso com o mercado livre, os Estados Unidos não tinham coragem e visão para exercer o poder imperial. “É uma pena”, lamentou Bristol. “Acho que seria natural para os Estados Unidos. . . Desempenhar um papel muito mais dominante nos assuntos mundiais. Não o que estamos fazendo agora, mas comandar e dar ordens sobre o que está sendo feito. As pessoas precisam disso. Há muitas partes do mundo, em particular África, onde uma autoridade disposta a utilizar tropas pode fazer uma diferença ótima, uma diferença saudável.” Mas com a discussão pública moderada por contabilistas, Bristol considerou improvável que os Estados Unidos ocupassem o seu legítimo lugar como sucessores dos impérios do passado. “Lá está o Partido Republicano se envolvendo em nós. Sobre o quê? Prescrições para idosos? Quem se importa? Acho isso nojento. . . A política presidencial do país mais importante do mundo deveria girar em torno de receitas para idosos. Os futuros historiadores acharão isso muito difícil de acreditar. Não é Atenas. Não é Roma. Além disso, não é nada.”
 {\color{blue} 1}  

 
\par
 
Desde o 11 de setembro, tive muitas ocasiões para relembrar essas conversas. O 11 de setembro, disseram-nos na sequência, chocou os Estados Unidos da paz e prosperidade complacentes que se estabeleceram após a Guerra Fria. Forçou os americanos a olhar para além das suas fronteiras, a compreender finalmente os perigos que enfrentam uma potência mundial. Lembrou-nos os bens da vida cívica e o valor do Estado, pondo fim à fantasia de criar um mundo público a partir de altos privados de troca em interesse próprio. Além disso, restaurou na nossa confusa cultura cívica um sentido de profundidade e seriedade, de coisas “maiores do que nós”. O mais crítico de tudo é que deu aos Estados Unidos um propósito nacional coerente e um foco para o domínio imperial. Um país que durante algum tempo pareceu pouco disposto a enfrentar as suas responsabilidades internacionais estava agora preparado, mais uma vez, para suportar qualquer fardo, pagar qualquer preço, pela liberdade. Esta mudança de atitude, prosseguiu o argumento, foi boa para o mundo. Pressionou os Estados Unidos para criarem uma ordem internacional estável e justa. Também foi bom para os Estados Unidos. Além disso, forçou-nos a pensar em algo mais do que a paz e a prosperidade, lembrando-nos que a liberdade era uma fé combativa e não um poleiro confortável.
 
\par
 
Como qualquer momento histórico, o 11 de setembro – não os ataques terroristas ou o dia em si, mas a nova onda de imperialismo que gerou – tem múltiplas dimensões. Parte desta cultura política imperial rejuvenescida é o produto de um ataque surpresa a civis e dos esforços dos líderes dos EUA para fornecer alguma medida de segurança a uma população apreensiva. Uma parte dela provém da economia política subterrânea do petróleo, do desejo das elites dos EUA de garantir o acesso às reservas energéticas no Médio Oriente e na Ásia Central, e de utilizar o petróleo como um instrumento de geopolítica. Mas embora estes factores desempenhem um papel considerável na determinação da política dos EUA, não explicam inteiramente a política e a ideologia do próprio momento imperial. Para compreender essa dimensão, devemos olhar para o impacto sobre os conservadores americanos do fim da Guerra Fria, da queda do comunismo e da ascensão do mercado livre como princípio organizador da ordem interna e internacional. Pois foi a insatisfação conservadora com essa ordem que impulsionou, em parte, o seu esforço para criar uma nova.
 
\par
 
Para os neoconservadores que ficaram entusiasmados com a cruzada de Ronald Reagan contra o comunismo, tudo o que restou depois da Guerra Fria foi a outra paixão de Reagan – o seu empreendedorismo ensolarado e a sua alegria de viver de mercado – que encontrou um lar bem-vindo na América de Bill Clinton. Embora os neoconservadores certamente não se oponham ao capitalismo, eles não acreditam que o mercado livre seja a maior conquista da civilização. A visão deles é mais exaltada. Eles aspiram à grandeza épica de Roma, ao ethos do guerreiro pagão – ou cruzado moral – em vez do ethos do burguês confortável. Desde o fim da Guerra Fria, a visão imperial tem recebido pouca atenção, eclipsada pela adoção dos mercados livres e do comércio livre. Destruídos pelo seu próprio sucesso, os neoconservadores não estão satisfeitos com o mundo que criaram. E assim eles aceitaram o chamado do império, fornecendo o baixo prof. desfazer a um coro crescente. Embora tenham plena fé no poder americano, os neoconservadores sentem-se desconfortáveis ​​em usá-lo para a mera extensão do capitalismo. Procuram criar uma ordem internacional que será um monumento para sempre, um mundo que envolve algo mais do que dinheiro e mercados.
 
\par
 
Mas, como aprendemos, este imperial imaginado pode não proporcionar uma solução tão fácil para os desafios que os Estados Unidos enfrentam. Mesmo antes de a guerra no Iraque ir para o sul, o império americano enfrentava obstáculos assustadores no Médio Oriente e na Ásia Central, sugerindo quão evasiva era a ideia reinante dos imperialistas neoconservadores – de que os Estados Unidos podem governar os acontecimentos, que podem fazer história. – realmente é. (Na verdade, não faz muito tempo que a administração Bush dizia aos jornalistas: “Somos um império agora, e quando agimos, criamos a nossa própria realidade. E enquanto você estuda essa realidade – criteriosamente como quiser – agiremos novamente, criando outras novas realidades, que você poderá estudar.”)
 {\color{blue} 2}  
Internamente, a renovação cultural e política que muitos imaginaram que o 11 de setembro iria produzir revelou-se uma quimera, vítima de uma ideologia de mercado livre que não dá sinais de diminuir. Acontece que o 11 de setembro não cumpriu – e, na verdade, provavelmente não poderia – cumprir o papel que lhe foi atribuído pelos neoconservadores do império.
 
\par
 
Imediatamente após os ataques ao World Trade Center e ao Pentágono, intelectuais, políticos e especialistas – não da esquerda radical, mas dos principais conservadores e liberais – deram um suspiro audível de alívio, quase como se saudassem os ataques como uma libertação do miasma que Buckley e Bristol vinham criticando. O World Trade Center ainda estava em chamas e os corpos ali sepultados mal se recuperaram quando Frank Rich anunciou que “o pesadelo desta semana, agora está claro, despertou-nos de um sonho frívolo, se não decadente, de uma década”. Qual foi esse sonho? O sonho da prosperidade, de superar os obstáculos da vida com dinheiro. Durante a década de 1990, escreveu Maureen Down, esperávamos “superar a flacidez com dieta e exercícios, rugas com colágeno e Botox, flacidez da pele com cirurgia, impotência com Viagra, alterações de humor com antidepressivos, miopia com cirurgia a laser, cárie com humanos”. Hormônio do crescimento, doenças com pesquisa de células-tronco e bioengenharia.” “Renovamos as nossas cozinhas”, observou David Brooks, “remodelamos os nossos sistemas de entretenimento doméstico, investimos em mobiliário de jardim, jacuzzis e grelhadores a gás” – como se a riqueza pudesse libertar-nos da tragédia e das dificuldades.
 {\color{blue} 3}  
Este espírito teve consequências internas terríveis. Para Francis Fukuyama, encorajava o “comportamento autoindulgente” e a “preocupação com os próprios assuntos mesquinhos”. Também teve repercussão internacional. De acordo com Lewis “Scooter” Libby, o culto à paz e à prosperidade encontrou a sua expressão mais pura na política externa fraca e distraída de Bill Clinton, que tornou “mais fácil para alguém como Osama bin Laden levantar-se e dizer com credibilidade 'Os americanos não' não tenho estômago para se defender. Eles não sofrerão baixas para defender seus interesses. São moralmente fracos.’” De ​​acordo com Brooks, mesmo o observador mais casual da cena interna pré-11 de setembro, incluindo a Al-Qaeda, “poderia ter concluído que a América não era um país inteiramente sério”.
 {\color{blue} 4}  

 
\par
 
Mas depois daquele dia de setembro, como afirmaram alguns comentadores, o cenário nacional transformou-se. A América estava agora “mais mobilizada, mais consciente e, portanto, mais viva”, escreveu Andrew Sullivan. George Packer comentou sobre “o estado de alerta, a tristeza, a resolução e até o amor” despertados pelo 11 de setembro. “O que temo agora”, confessou Packer, “é um retorno à normalidade que todos deveríamos buscar”. Para Brooks, “o medo que prevalece no país” depois do 11 de setembro foi “um limpador, eliminando grande parte da autoindulgência da última década”. Reviver o medo eliminou a ansiedade da prosperidade, substituindo uma emoção incapacitante por uma paixão revigorante. “Trocamos as ansiedades da riqueza pelos medos reais da guerra.” 5
 
\par
 

 \textbf{\textit{Now upscales who once spent hours agonizing over which Men faucet head would go with their copper farmhouse-kitchen sink are suddenly worried about whether the water coming out of pipes has been poisoned. People who longed for} }  
 
 
\par
 

 
\par
 

 \textbf{\textit{Prada bags at Bloomingdale's are suddenly spooked by unattended bags at the airport. America, the sweet land of liberty, is getting a crash course in fear. {{\color{blue} 6} } } }  
 
 
\par
 
Hoje, concluiu Brooks, “a vida comercial parece menos importante que a vida pública. . . . Quando há lutas de vida ou morte, é difícil pensar em Bill Gates ou Jack Welch como particularmente heroicos.”
 {\color{blue} 7}  

 
\par
 
Escritores repetidamente acolheram a eletricidade moral galvanizadora que agora percorria o corpo político. Uma energia pulsante de determinação pública e comprometimento cívico, que restauraria a confiança no governo — talvez, de acordo com alguns liberais, até mesmo autorizaria um estado de bem-estar social renovado — e traria uma cultura de patriotismo e conexão, um novo consenso bipartidário, o fim da ironia e das guerras culturais, uma presidência mais madura e mais elevada.
 {\color{blue} 8}  
De acordo com um repórter do USA Today, o Presidente Bush estava especialmente entusiasmado com a promessa do 11 de setembro, de trazer ele próprio e a sua geração como Prova A no projeto de renovação interna. “Bush disse aos conselheiros que acredita que confrontar o inimigo é uma oportunidade para ele e os seus colegas bebê boomers reorientarem as suas vidas e provarem que têm o mesmo tipo de valor e compromisso que os seus pais demonstraram na Segunda Guerra Mundial.” E embora a fonte específica da euforia de Christopher Hitchens possa ter sido peculiarmente sua, a sua autodeclarada schadenfreude certamente não foi: “Talvez eu devesse confessar que em setembro
 {\color{blue} 11}  
Por último, após experimentar toda a gama habitual de emoções dos mamíferos, da raiva à náusea, descobri também que outra sensação estava competindo pelo domínio. Ao examiná-lo, e para minha própria surpresa e prazer, revelou-se uma alegria. Aqui estava o inimigo mais terrível – a barbárie teocrática – à vista. . . . Percebi que se a batalha continuasse até o último dia da minha vida, eu nunca ficaria entediado em levá-la ao máximo.”
 {\color{blue} 9}  
Com o seu espetáculo chocante de medo e morte, o 11 de setembro acabou e uma cultura morta ou moribunda tem a oportunidade de viver novamente.
 
\par
 
Internacionalmente, o 11 de setembro forçou os Estados Unidos a reaproximar-se do mundo, a assumir o fardo dos impérios sem constrangimento ou confusão. Onde os primeiros George Bush e Bill Clinton se atrapalharam no escuro, em busca de uma doutrina para orientar o exercício do poder dos EUA após o colapso da União Soviética, a missão dos Estados Unidos era agora clara: defender a civilização contra a barbárie, a liberdade contra o terror. Como disse Condoleezza Rice ao New Yorker: “Acho que a dificuldade em definir um papel já passou. Acho que o 11 de setembro foi um daqueles grandes terremotos que esclarecem e aguçam. Os eventos estão em um relevo muito mais nítido.” Uma América que se pensava estar perdida na areia movediça dos mercados livres, do individualismo e do isolamento foi agora chamada de volta à consciência de um mundo para além das suas fronteiras e inspirada a um compromisso de sustentar baixas em nome de uma ordem global liderada pelos EUA. Como concluiu o antigo subsecretário da Defesa de Clinton, “é pouco provável que os americanos voltem à complacência que marcou a primeira década após a Guerra Fria”. Eles agora entendiam, nas palavras de Brooks, que “o mal existe” e que “para preservar a ordem, as pessoas boas devem exercer poder sobre as pessoas destrutivas”.
 {\color{blue} 10}  

 
\par
 
Uma década depois, é difícil recapturar, e muito menos compreender, a mentalidade daquele momento. Não apenas porque desapareceu tão rapidamente, com o país a recair no seu estranho e azedo partidarismo - onde o volume do antagonismo retórico entre as partes só é igualado pela profundidade do seu acordo sobre os fundamentos econômicos (nesse aspecto, nós' ainda vivemos na América de Bill Clinton) – antes mesmo de o primeiro mandato de Bush ter terminado. Mais desconcertante é como tantos escritores e políticos puderam abrir os braços às consequências políticas das mortes em massa, aproveitando o 11 de setembro como uma oportunidade para expressar o seu desprezo, aparentemente de longa data, pela própria paz e prosperidade que o precedeu. No dia 12 de setembro, seria de esperar expressões de pesar pelo rebentamento das bolhas – econômica, cultural e política. Em vez disso, muitos viram o 11 de setembro como um julgamento estrondoso e um corretivo necessário para a frivolidade e o vazio da década de 1990. Teríamos de recuar quase um século – até aos primeiros dias da Primeira Guerra Mundial, quando explodiu o “gás do pântano do tédio e do vazio” que envolvia outro fim de século globalizante e de comércio livre – para encontrar um paralelo remotamente exato.
 {\color{blue} 11}  

 
\par
 
Para compreender este espírito de alívio exuberante, temos de revisitar os últimos dias da Guerra Fria, quando as elites americanas perceberam pela primeira vez que os Estados Unidos já seriam incapazes de definir a sua missão em termos da ameaça soviética. Embora o fim da Guerra Fria tenha desencadeado uma onda de triunfalismo, também provocou entre as elites uma incerteza ansiosa sobre a política externa dos EUA. Com a derrota do comunismo, muitos perguntaram: como deveriam os Estados Unidos definir o seu papel no mundo? Onde e quando deverá intervir em conflitos estrangeiros? Qual o tamanho das forças armadas que deveria colocar em campo? Subjacente a estes argumentos estava um profundo desconforto sobre o tamanho e o propósito do poder americano. Os Estados Unidos pareciam estar a sofrer de um excesso de poder, o que dificultava às elites formular quaisquer princípios coerentes para governar a sua utilização. Tal como Richard Cheney, então secretário da Defesa do primeiro presidente Bush, reconheceu em fevereiro de 1992: “Ganhamos tanta profundidade estratégica que as ameaças à nossa segurança, agora relativamente distantes, são mais difíceis de definir”. Quase uma década depois, os Estados Unidos ainda pareceriam, para os seus líderes, um gigante em dificuldades. Como observou Condoleezza Rice durante a campanha presidencial de 2000: “Os Estados Unidos têm tido enorme dificuldade em definir o seu ‘interesse nacional’ na ausência do poder soviético”. As elites políticas se tornaram tão incertas sobre o interesse nacional que um importante assessor de defesa de Clinton - e mais tarde reitor da Kennedy School de Harvard - acabou erguendo as mãos em derrota, declarando que o interesse nacional é tudo o que “os cidadãos, após a devida deliberação, dizem que é”. ”- uma abdicação simplesmente impensável durante o reinado dos Sábios na Guerra Fria.
 {\color{blue} 12}  

 
\par
 
Quando Clinton saiu do gelo, ele e os seus conselheiros fizeram um balanço desta situação sem paralelo – onde os Estados Unidos possuíam tanto poder que não enfrentavam, nas palavras do Conselheiro de Segurança Nacional de Clinton, Anthony Lake, nenhuma “ameaça credível a curto prazo para [os seus] existência” – e concluiu que as principais preocupações da política externa americana já não eram militares, mas sim econômicas. Após ensaiar sumariamente os vários perigos militares possíveis para os Estados Unidos, o Presidente Clinton declarou num discurso de 1993: “Ainda enfrentamos, abrangendo o resto, este desafio amorfo, mas profundo na forma como a humanidade conduz o seu comércio”. O grande imperativo da era pós-Guerra Fria era organizar uma economia global onde os cidadãos do mundo pudessem comercializar através das fronteiras. Para que isso acontecesse, os Estados Unidos tinham de pôr a sua própria casa econômica em ordem – “a renovação começa em casa”, disse Lake – reduzindo o défice (em parte através de reduções nas despesas militares), baixando as taxas de juro, apoiando a alta tecnologia. Indústria e promoção de acordos de livre comércio. Dado que outras nações também teriam de realizar uma dolorosa reforma econômica, Lake concluiu que o objectivo principal dos Estados Unidos era o “alargamento da comunidade mundial livre de democracias de mercado”.
 {\color{blue} 13}  

 
\par
 
A avaliação de Clinton dos desafios que os Estados Unidos enfrentam foi parcialmente inspirada pelo cálculo político. Ele tinha acabado de ganhar uma eleição contra um presidente em exercício que não só liderou os Estados Unidos na vitória na Guerra Fria, mas também planejou uma derrota impressionante sobre os militares iraquianos. Um governador do Sul sem experiência em política externa – e ainda por cima um evasivo do recrutamento – Clinton concluiu que a sua vitória sobre Bush significou que as questões da guerra e da paz já não ressoavam entre os eleitores americanos da mesma forma que teriam em tempos anteriores.
 {\color{blue} 14}  
Mas a visão de Clinton refletia também uma convicção, comum há década de 1990, de que a globalização do mercado livre tinha minado a eficácia do poder militar e a viabilidade dos impérios tradicionais. A força já não era o único ou mais eficaz instrumento da vontade nacional. O poder dependia agora do dinamismo e do sucesso da economia de uma nação e da atratividade da sua cultura. Como diria Joseph Nye, secretário adjunto da Defesa de Clinton, o “soft poder” – o capital cultural que tornou os Estados Unidos tão admirados em todo o mundo – era tão importante para a preeminência nacional como o poder militar. Talvez pela primeira vez para um funcionário dos EUA, Nye invocou a Gram si para argumentar que os Estados Unidos só manteriam a sua posição de hegemonia se persuadissem – em vez de forçar – outros a seguirem o seu exemplo. “Se eu conseguir que você queira fazer o que eu quero”, escreveu Nye, “então não preciso forçá-lo a fazer o que você não quer”.
 {\color{blue} 15}  
Para manter sua posição no mundo, os Estados Unidos teriam que superar outras economias nacionais, ao mesmo tempo, em que garantiriam a disseminação de seu modelo de livre mercado e cultura pluralista. O maior perigo que os Estados Unidos enfrentavam era que eles não reformariam sua economia ou que abusariam de sua superioridade militar e provocariam ódio internacional. O problema não era que os Estados Unidos não tinham poder suficiente, mas que tinham muito. Para tornar o mundo seguro para a globalização, os Estados Unidos teriam que ser perturbados ou, no mínimo, significativamente restringidos em suas aspirações imperiais.
 
\par
 
Para os conservadores que ansiavam e depois celebraram o fim do socialismo, a promoção de uma prosperidade tranquila por parte de Clinton foi um horror. A riqueza produziu uma sociedade sem dificuldades e adversidades. A satisfação material induziu uma perda de profundidade social e de significado político, uma diminuição da determinação e do entusiasmo heroico. “Naquela era de paz e prosperidade”, escreveria David Brooks, “a série de comédia mais popular era Seinfeld, um programa sobre nada”. Robert Kaplan emitiu farpas após farpas sobre os habitantes “saudáveis ​​e bem alimentados” da “sociedade burguesa”, demasiado consumidos com o seu próprio conforto e prazer para dar uma mão – ou empunhar uma arma – para tornar o mundo um lugar mais seguro. “Os bens materiais”, concluiu ele, “encorajam a docilidade”.
 {\color{blue} 16}  
Ao longo da década de 1990, o principal item de reclamação intelectual, em todo o espectro político, foi que os Estados Unidos não tinham uma mentalidade cívica ou marcial insuficiente, os seus líderes e cidadãos estavam demasiado distraídos pela prosperidade e pela riqueza para cuidarem das suas instituições herdadas, preocupações e defesa mundial. Supunha-se que o respeito pelo Estado diminuía, tal como a participação política e o voluntariado local.
 {\color{blue} 17}  
Na verdade, um dos sinais mais reveladores do declínio do imperativo da Guerra Fria foi o fato de a década de 1990 ter começado e terminado com dois incidentes – a controvérsia Clarence Thomas-Anita Hill e a decisão do Supremo Tribunal Bush v. na instituição política mais venerada do país.
 
\par
 
Para neocouma influentes, a política externa de Clinton era ainda mais anátema. Não porque os neocouma fossem unilateralistas argumentando contra o multilateralismo de Clinton, ou isolacionistas, ou realistas críticos de seu internacionalismo e humanitarismo.
 {\color{blue} 18}  
A política externa de Clinton, argumentaram, era demasiado impulsionada pelos imperativos da globalização do mercado livre. Foi a prova da decadência que tomou conta dos Estados Unidos após a derrota da União Soviética, um sinal de fibra moral enfraquecida e de espírito marcial perdido. Num influente manifesto publicado em 2000, Donald e Frederick Kagan mal conseguiram conter o seu desprezo pela “feliz situação internacional que emergiu em 1991”, que era “caracterizada pela difusão da democracia, do comércio livre e da paz” e que era “tão compatível com a América” com seu amor pelo “conforto doméstico”. Segundo Kaplan, “o problema das sociedades burguesas” como a nossa “é a falta de imaginação”. A mãe do futebol, por exemplo, tão insistentemente defendida tanto por republicanos como por democratas, não se preocupa com o mundo fora dos seus estreitos limites. “A paz”, queixou-se ele, “é prazerosa, e o prazer tem a ver com satisfação momentânea”. Pode ser obtido “apenas por uma forma de tirania, por mais sutil e branda que seja”. Apaga a memória do conflito estimulante, do desacordo robusto, do luxo de nos definirmos “em virtude de quem enfrentamos”.
 {\color{blue} 19}  

 
\par
 
Embora os conservadores tenham frequentemente a reputação de favorecer a riqueza e a prosperidade, a lei e a ordem, a estabilidade e a rotina – todos os confortos da vida burguesa – os críticos conservadores de Clinton odiavam-no pela sua busca destas mesmas virtudes. As obsessões de Clinton pelo mercado livre traíram uma relutância em abraçar o mundo obscuro do poder e do conflito violento, da tragédia e da ruptura. A sua política externa não era apenas irrealiza; estava insuficientemente escuro e taciturno. “O que mais impressiona no zeitgeist dos anos 1990”, queixou-se Brooks, “foi a presunção de harmonia. A época foi moldada pela ideia de que não existiam mais conflitos fundamentais.” Os conservadores prosperam num mundo cheio de mal misterioso e ódios insondáveis, onde o bem está sempre na defensiva e o tempo é um bem precioso na corrida cósmica contra a corrupção e o declínio. Lidar com um mundo assim exige coragem pagã e uma virtude quase bárbara, qualidades que os conservadores abraçam em detrimento dos bens mais prosaicos da paz e da prosperidade. Não é por acaso que Paul Wolfowitz, o mais sombrio destes príncipes sombrios do pessimismo, foi aluno de Allan Bloom (na verdade, Horowitz faz uma pequena aparição em Ravenstein, o romance de Saul Bellow sobre Bloom). Pois Bloom – como muitos outros neoconservadores influentes – foi um seguidor de Leo Strauss, cujas odes silenciosas à virtude clássica e à harmonia ordenada velaram no seu Nietzsche uma visão de conflito tortuoso e de luta violenta.
 {\color{blue} 20}  

 
\par
 
Mas havia outra razão para a insatisfação dos neoconservadores com a política externa de Clinton. Muitos deles consideraram-no insuficientemente visionário e consistente. Clinton, alegaram, foi reativo e improvisado, em vez de proativo e enérgico. Ele e os seus conselheiros não estavam dispostos a imaginar um mundo onde os Estados Unidos moldassem os acontecimentos, em vez de reagirem. Rompendo novamente com o estereótipo habitual dos conservadores como pragmáticos não ideológicos, figuras como Horowitz, Libby, Kaplan, Perl, Frank Gaff Na, Kenneth Adelman e as equipas de pai e filho de Kagan e Bristol apelaram a uma projeção ideologicamente mais coerente dos EUA. Poder, onde a “hegemonia benigna” do poderio americano espalharia “a zona da democracia” em vez de apenas alargar o mercado livre. Eles queriam uma política externa que fosse, nas palavras que Robert Kagan mais tarde usaria para elogiar o senador Joseph Lieberman, “idealista, mas não ingênua, pronta e disposta a usar a força e comprometida com um exército forte, mas também comprometida em usar o poder americano para espalhar democracia e fazer algo de bom no mundo.” Já no primeiro governo Bush, os neoconservadores insistiam que os Estados Unidos deveriam, nas palavras de Cheney, “moldar o futuro, determinar o resultado da história”, ou, como diriam mais tarde os Kagan, “intervir decisivamente em todas as regiões críticas” do mundo, “se lá existe uma ameaça visível”. Eles criticaram os republicanos, nas palavras de Robert Kagan, que “durante a década muda da década de 1990” sofreram de uma “hostilidade à 'construção da nação', da aversão ao 'trabalho social internacional' e da crença estreita de que 'as superpotências não façam janelas.'”
 {\color{blue} 21}  
O que estes conservadores desejavam era uma América que fosse genuinamente imperial – não apenas porque acreditavam que isso tornaria os Estados Unidos mais seguros ou mais ricos, e não apenas porque pensavam que isso melhoraria o mundo, mas porque literalmente queriam ver os Estados Unidos Os Estados fazem o mundo.
 
\par
 
Ao nível mais óbvio, o 11 de setembro confirmou o que os conservadores vinham dizendo há anos: o mundo é um lugar perigoso, cheio de forças hostis que não irão parar diante de nada para ver os Estados Unidos serem derrubados. Mais importante ainda, o 11 de setembro deu aos conservadores uma oportunidade de articular, sem constrangimento, a visão do poder imperial americano que vinham alimentando discretamente durante décadas. “As pessoas estão agora a sair do armário com a palavra império”, observou com precisão Charles Kraut-banner pouco depois do 11 de setembro. Ao contrário dos impérios do passado, este seria guiado por uma visão benigna e até benéfica de melhoria mundial. Devido ao sentido de justiça e propósito benevolente da América – ao contrário da Grã-Bretanha ou de Roma, os Estados Unidos não tinham intenção de ocupar ou apoderar-se de território próprio – este novo império não geraria a reação negativa que todos os impérios anteriores tinham gerado. Como disse um redator do Wall Street Journal: “somos um império atraente, aquele ao qual todos desejam aderir”. Nas palavras de Rice: “Teoricamente, os realistas preveriam que ao ter uma grande potência como os Estados Unidos, não demorará muito até que outras grandes potências se levantem para desafiá-la. E penso que o que estamos a ver é que desta vez há pelo menos uma predileção por avançar para relações produtivas e cooperativas com os Estados Unidos, em vez de tentar equilibrar os Estados Unidos.”
 {\color{blue} 22}  
Ao criar um império, os Estados Unidos já não teriam de responder a ameaças imediatas, de “esperar pelos acontecimentos, enquanto os perigos se acumulam”, como disse o Presidente Bush no seu discurso sobre o Estado da União de 2002. Iria agora “moldar o ambiente”, antecipar ameaças, pensando não em meses ou anos, mas em décadas, talvez séculos. Os objectivo eram os que Cheney, seguindo o conselho de Horowitz, delineou pela primeira vez no início da década de 1990: garantir que nenhuma outra potência se levantasse para desafiar os Estados Unidos e que nenhuma potência regional alguma vez alcançasse preeminência nos seus teatros locais. A ênfase estava no preventivo e preditivo, para pensar em termos de tornar-se, e não em termos de ser. Como disse Richard Perle, relativamente ao Iraque: “O que é essencial aqui não é olhar para a oposição a Saddam tal como ela é hoje, sem qualquer apoio externo, sem qualquer esperança realista de remover esse terrível regime, mas olhar para no que poderia ser criado.”
 {\color{blue} 23}  
Para os conservadores, os dois anos após o 11 de setembro foram uma época inebriante, um momento em que o seu compromisso e hostilidade simultâneos para com o mercado livre puderam finalmente ser satisfeitos. Não mais paralisados ​​pelas entorpecentes políticas de riqueza e prosperidade, eles acreditavam que podiam contar com o público para responder ao chamado do sacrifício e do destino, do confronto e do mal. Sendo “perigo” e “segurança” as palavras de ordem do dia, o Estado americano seria novamente santificado – sem abrir as comportas à redistribuição econômica. Eles esperavam que o 11 de setembro e o império americano resolveriam finalmente as contradições culturais do capitalismo que Daniel Bell tinha notado há muito tempo, mas que só vieram verdadeiramente à tona após a derrota do comunismo.
 
\par
 
Que diferença faz uma década – ou mesmo alguns anos. Muito antes de os Estados Unidos terem essencialmente de declarar vitória no Iraque e (mais ou menos) regressar a casa, muito antes de George W. Bush deixar o seu gabinete em desgraça, muito antes de a guerra no Afeganistão provar ser muito mais do que o povo americano poderia estômago, ficou claro que o império neoconservador assentava sobre uma base instável. No final de outubro e início de novembro de 2001, por exemplo, depois de apenas algumas semanas de bombardeamentos não terem conseguido desalojar os Taliban, os críticos começaram a murmurar os seus receios de que a guerra no Afeganistão fosse uma reprise do atoleiro do Vietnã.
 {\color{blue} 24}  
Assim que a guerra no Iraque pareceu não ser tão moleza quanto seus defensores proclamaram que seria, os democratas começaram a sondar, ainda que timidamente, as bordas da crítica aceitável. Já na campanha presidencial de 2004, expressar críticas à guerra se tornou uma espécie de teste decisivo entre os candidatos democratas.
 
\par
 
Nenhum destes críticos, é claro, desafiaria a premissa militar o mais rápido possível das políticas de Bush – e mesmo sob Obama, poucos questionariam as premissas básicas do alcance global da América – mas o aparecimento periódico de tais críticos, particularmente em tempos de dificuldades ou de derrota, sugere que a visão imperial só é politicamente viável enquanto for bem sucedida. É assim que deve ser: porque a peça central da promessa imperial é que os Estados Unidos podem governar os acontecimentos, que podem determinar o resultado da história, a promessa permanece ou cai em caso de sucesso ou fracasso. Com qualquer sugestão de que os acontecimentos estão fora do controlo do império, a visão imperial confunde-se. Na verdade, demorou apenas uma semana, em março de 2002, de terrível derramamento de sangue em Israel e nos Territórios Ocupados – e as acusações resultantes de que “Bush toca violino na Casa Branca ou no Texas, tocando Nero enquanto o Médio Oriente arde” – para que o império planejado fosse chamado em questão. Mal a violência no Médio Oriente começou a aumentar, até os defensores da administração começaram a desistir, sugerindo que qualquer invasão do Iraque teria de ser adiada indefinidamente. Como disse um dos assessores de segurança nacional de alto nível de Reagan: “A suprema ironia é que a maior potência que o mundo alguma vez conheceu provou ser incapaz de gerir uma crise regional”. O fato, acrescentou este assessor, de a administração ter estado tão maniaca mente “concentrada no Afeganistão ou no Iraque” – os dois principais postos avançados do confronto imperial – enquanto o Médio Oriente ardia, “reflete uma arrogância ou uma ignorância terríveis”.
 {\color{blue} 25}  

 
\par
 
Ironicamente, enquanto a administração Bush evitou esses conflitos, como aquele entre israelitas e palestinianos, onde poderia falhar - e, de facto, no momento em que este livro foi escrito, a administração Obama parece estar a seguir o mesmo caminho no que diz respeito a Israel e à Palestina – foi forçado a renunciar à própria lógica do imperialismo que procurava confessar. Tendo como premissa a capacidade dos Estados Unidos para controlar os acontecimentos, a visão imperial neoconservadora não podia acomodar-se ao fracasso. Mas, ao evitarem o fracasso, os imperialistas foram forçados a reconhecer que não podiam controlar os acontecimentos. Como observou o ex-secretário de Estado Lawrence Eagle Burger sobre o conflito israelo-palestiniano, Bush percebeu “que simplesmente inserir-se nesta confusão sem qualquer possibilidade de obter qualquer sucesso é, por si só, perigoso, porque demonstraria que de facto não, não temos nenhuma capacidade neste momento de controlar ou afetar eventos”
 {\color{blue} 26}  
– Precisamente a admissão que os neoconservadores não podiam permitir-se fazer. Este beco não foi um mero problema de lógica ou consistência: traiu a fragilidade essencial da própria posição imperial.
 
\par
 
Essa fragilidade refletiu também o vazio interno da visão imperial dos neoconservadores. Embora os neoconservadores vissem e continuem a ver o imperialismo como a contrapartida cultural e política do mercado livre, nunca chegaram a um acordo – nem mesmo dez anos depois – sobre como a oposição conservadora aos gastos do governo e o compromisso com as reduções de impostos tornam os Estados Unidos fará dificilmente os investimentos necessários na construção da nação que o imperialismo exige.
 
\par
 
Internamente, há poucas evidências que sugiram que a renovação política e cultural imaginada pela maioria dos comentadores – o renascimento do Estado, o regresso do sacrifício partilhado e da comunidade, o aprofundamento da consciência moral – tenha alguma vez ocorrido, mesmo nos dias mais inebriantes do rescaldo do 11 de setembro. De todos os incidentes que se poderiam citar daquela época, dois se destacam. Em março de 2002, sessenta e dois senadores, incluindo dezenove democratas, rejeitaram padrões mais elevados de eficiência de combustível na indústria automóvel, o que teria reduzido a dependência do petróleo do Golfo Pérsico. O republicano do Missouri, Christopher Bond, sentiu-se tão livre da necessidade de prestar homenagem às instituições estatais em tempos de guerra que afirmou no plenário do Senado: “Não quero dizer a uma mãe em meu estado natal que ela não deveria comprar um SUV. Porque o Congresso decidiu que seria uma má escolha.” Ainda mais revelador foi o quão vulneráveis ​​eram os proponentes de padrões mais elevados a estes argumentos antiestatistas. John McCain, por exemplo, foi imediatamente colocado na defensiva pela noção de que o governo estaria a interferir nas escolhas das pessoas no mercado privado. Resta-lhe argumentar que “nenhum americano será forçado a conduzir qualquer automóvel diferente”, como se isso fosse uma imposição terrível nesta nova era de sacrifício e solidariedade em tempo de guerra.
 {\color{blue} 27}  

 
\par
 
Alguns meses antes, Ken Feinberg, chefe do Departamento de setembro
 {\color{blue} 11}  
O Fundo de Compensação das Vítimas anunciou que as famílias das vítimas receberiam uma compensação pelas suas perdas com base, em parte, no salário que cada vítima recebia no momento da sua morte. Após os ataques ao World Trade Center e ao Pentágono, o Congresso tomou a medida sem precedentes de assumir a responsabilidade nacional pela restituição às famílias das vítimas. Embora a inspiração para esta decisão tenha sido a prevenção de dispendiosos processos judiciais contra a indústria aérea, muitos observadores consideraram-na como um sinal de um novo espírito no país: face à tragédia nacional, os líderes políticos estavam finalmente a romper com a sobrevivência na selva da Revolução Reaganclinton. Anos. Mas mesmo na morte, o mercado – e as desigualdades que gera – era a única língua que os líderes da América sabiam falar. Abandonando a noção de sacrifício partilhado, Feinberg optou pelas tabelas atuariais para calcular pacotes de remuneração adequados. A família de uma avó solteira de 65 anos que ganha 10 mil dólares por ano – talvez uma trabalhadora de cozinha que receba um salário mínimo – retiraria 300 mil dólares do fundo, enquanto a família de um comerciante de Wall Street de 30 anos receberia 3 870 064 dólares. Os homens e mulheres mortos em setembro
 {\color{blue} 11}  
Não eram cidadãos de uma democracia; eles eram assalariados e as recompensas seriam distribuídas de acordo. Praticamente ninguém – nem mesmo os comentadores e políticos que denunciaram o cálculo de Feinberg por outras razões – criticou este aspecto da sua decisão.
 {\color{blue} 28}  

 
\par
 
Mesmo dentro e em torno das forças armadas, o espírito do patriotismo e do destino partilhado permaneceu secundário em relação à ideologia do mercado. Num artigo pouco notado de outubro de 2001 no New York Times, os recrutadores militares confessaram que ainda procuravam atrair os alistados não com o apelo do patriotismo ou do dever, mas com a promessa de oportunidades econômicas. Como disse um recrutador: “Tudo continua como sempre. Não forçamos a rotina de ‘Ajudar o nosso país’.” Quando um patriota ocasional irrompeu em uma área de recrutamento e disse: “Quero lutar”, explicou um recrutador, “preciso acalmá-los. Não somos apenas lutadores e bombardeios. Estamos falando de empregos. Além disso, tratamos de educação.”
 {\color{blue} 29}  
Os recrutadores admitiram que continuaram a visar imigrantes e pessoas de cor, no pressuposto de que foi a falta de oportunidades destes círculos eleitorais que os levou ao serviço militar. O objectivo publicamente reconhecido do Pentágono, na verdade, era aumentar o número de latinos nas forças armadas de
 {\color{blue} 10}  
Por cento para
 {\color{blue} 22}  
Por cento. Os recrutadores chegaram mesmo a infiltrar-se no México, com promessas de cidadania instantânea aos não-cidadãos pobres dispostos a pegar em armas em nome dos Estados Unidos. De acordo com um recrutador de San Diego, “é uma prática mais ou menos comum que alguns recrutadores vão à Tijuana distribuir panfletos ou, em alguns casos, procuram alguém para distribuir informações no lado mexicano”.
 {\color{blue} 30}  
Em dezembro de 2002, enquanto os Estados Unidos se preparavam para invadir o Iraque, o congressista democrata de Nova York Charles Rangel decidiu enfrentar essa questão de frente, propondo a reintegração do recrutamento. Observando que imigrantes, pessoas de cor e os pobres estavam arcando com uma porcentagem maior do fardo militar do que seus números na população justificavam, Rangel argumentou que os Estados Unidos deveriam distribuir os custos domésticos do império de forma mais equitativa. Se crianças brancas de classe média fossem forçadas a portar armas, ele alegou, a administração e seus apoiadores poderiam pensar duas vezes antes de ir para a guerra. O projeto de lei não deu em nada.
 
\par
 
O fato de a guerra nunca ter imposto o tipo de sacrifício à população que normalmente acompanha as cruzadas nacionais provocaram preocupações significativas entre as elites políticas e culturais. “O perigo, a longo prazo”, escreveu R. W. Apple do time antes de morrer, “é a perda de interesse. Com grande parte da guerra sendo conduzida fora da vista de todos por comandos, diplomatas e agentes de inteligência, uma nação que passou décadas em fácil autoindulgência permanecerá focada?” Pouco após declarar que a era do brilho e do brilho acabou, Frank Rich se viu publicamente agonizando que “você nunca imaginaria que esta fosse uma nação em guerra”. Antes do 11 de setembro, “o governo disse que poderíamos ter tudo”. Desde o 11 de setembro, o governo vinha dizendo praticamente o mesmo. Um ex-assessor de Lyndon Johnson disse ao New York Times: “As pessoas vão se envolver nisso. Até agora é um esforço do governo, como deveria ser, mas as pessoas não estão engajadas”.
 {\color{blue} 31}  
Sem consagrar a causa com sangue, temiam os observadores, os americanos não veriam o seu compromisso testado e a sua determinação aprofundada. Como Doris Kearns Goodwin reclamou no The Newshour:
 
\par
 

 \textbf{\textit{Well, I think the problem is we understand that it’s going to be a long war, but it’s hard for us to participate in that war in a thousand and one ways the way we could in World War II. You could have hundreds of thousands joining the armed forces. They could go to the factories to make sure to get those ships, tanks, and weapons built. They could have victory gardens. Furthermore, they could feel not simply as we’re being told: Go back to your ordinary lives. It’s harder now. We don’t have a draft in the same way we did, although there’s some indication I’d like to believe that that younger generation will want to participate. My own youngest son who just graduated from Harvard this June has joined the military. He wants that three-year commitment. He wants to be part of what this is all about instead of just going to work for a year and going to law school, he wants to be a part of this. And I suspect there will be a lot of others like that as well. But somehow you just keep wishing that the government would challenge us. Maybe we need a Manhattan Project for this antibiotics' vaccine production. We were able to get cargo ships down from {{\color{blue} 365} } days in World War II to one day by the middle with that kind of collective enterprise. And I think we need to be mobilized, our spirit, our productivity, much more than we were. {{\color{blue} 32} } } }  
 
 
\par
 
No que pode ter sido o espetáculo mais estranho de toda a guerra, os líderes da nação acabaram se esforçando para encontrar coisas para as pessoas fazerem — não porque houvesse muito a ser feito, mas porque sem algo para fazer, o ardor dos americanos comuns esfriaria. Como essas tarefas eram desnecessárias, e torná-las obrigatórias violaria as normas da ideologia de mercado, o melhor que o presidente e seus colegas conseguiram foi anunciar sites e números gratuitos onde homens e mulheres empreendedores poderiam encontrar informações sobre como ajudar no esforço de guerra. Como Bush declarou na Carolina do Norte no dia seguinte ao seu discurso do Estado da União de 2002, "Se você ouviu o discurso ontem à noite, sabe, as pessoas estavam dizendo: 'Bem, nossa, isso é legal, ele me chamou para a ação, onde eu procuro?' Bem, aqui está onde: no usafreedomcorps. Governador. Ou você pode ligar para este número — parece que estou fazendo um discurso, e estou. Esta é a coisa certa a fazer pela América. 1-877-USA- CORPS.” O governo não podia nem contar com os cidadãos para pagar pelo telefonema. E quais eram as tarefas que esses voluntários deveriam desempenhar? Se fossem médicos ou profissionais de saúde, eles poderiam se alistar para ajudar durante emergências. E todos os outros? Eles poderiam servir em programas de vigilância de bairro para se proteger contra ataques terroristas — na Carolina do Norte.
 {\color{blue} 33}  

 
\par
 
Desde o fim da Guerra Fria, alguns podem até dizer do Vietnã, tem havido uma crescente desconexão entre a cultura e a ideologia das elites empresariais dos EUA e a de guerreiros políticos como Horowitz e outros neocouma. Onde a Guerra Fria viu a criação de uma classe semi coerente de Homens Sábios que uniram, embora irregularmente, os mundos dos negócios e da política — homens como Dean Acheson e os irmãos Dulles — os anos Reagan e além testemunharam algo completamente diferente. Por um lado, temos uma geração mais jovem de magnatas corporativos que, embora implacáveis ​​em seus esforços para garantir benefícios do estado, não têm o respeito ou a paixão pelo estado de seus colegas mais velhos. Certamente dispostos a tirar do público até, eles desprezam a política e o governo. Esses novos CEOs respondem a seus colegas em Tóquio, Londres e outras cidades globais; contanto que o estado forneça a eles o que precisam e não interfira indevidamente em suas operações, eles deixam isso para os apparatchiks.
 {\color{blue} 34}  
Questionado por Thomas Friedman sobre a frequência com que ele fala sobre o Iraque, a Rússia ou guerras estrangeiras, um executivo do Vale do Silício disse: “Não mais do que uma vez por ano. Nós nem nos importamos com Washington. O dinheiro é extraído pelo Vale do Silício e então desperdiçado por Washington. Eu quero falar sobre pessoas que criam riqueza e empregos. Eu não quero falar sobre pessoas doentes e improdutivas. Se eu não me importo com os destruidores de riqueza em meu próprio país, por que eu deveria me importar com os destruidores de riqueza em outro país?”
 {\color{blue} 35}  

 
\par
 
Por outro lado, temos uma nova classe de elites políticas que têm pouco contacto com a comunidade empresarial, cujas principais experiências fora do governo têm sido na academia, no jornalismo, em grupos de reflexão ou em qualquer outra parte da indústria cultural. Homens como Horowitz e Brooks, os Kagan e os Bristol, traficam ideias e veem o mundo como uma paisagem de projeção intelectual. Livres dos constrangimentos mesmo dos interesses mais interessados, consideram-se livres para fazer avançar a sua causa, no Médio Oriente e noutros lugares. Tal como os seus homólogos empresariais, os neoconservadores beém o mundo como o seu palco; ao contrário dos seus homólogos empresariais, estão a preparar-se para um drama totalmente mais teatral e sobrenatural. O seu objectivo final, se é que existe, é um confronto apocalíptico entre o bem e o mal, a civilização e a barbárie – categorias de conflito pagão diametralmente opostas à visão de mundo sem fronteiras da elite globalizante e de comércio livre da América.
 
\par
  
 
999999
