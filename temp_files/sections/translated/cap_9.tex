\chapter{Lembrança dos Impérios Passados}\label{Lembrança dos Impérios Passados}
 \par 
Em 2000, passei a maior parte do final do verão entrevistando William F. Buckley e Irving Kristol. Eu estava escrevendo um artigo para a Lingua Franca (ver capítulo {\color{blue}5}) sobre as deserções de intelectuais de direita para a esquerda e queria ouvir o que os pais fundadores do movimento pensavam dos seus filhos rebeldes. No entanto, ao longo das nossas conversas, tornou-se claro que Buckley e Kristol estavam menos interessados ​​nestes ex-conservadores do que no lamentável estado do movimento conservador e no destino incerto dos Estados Unidos como império global. O fim do comunismo e o triunfo do mercado livre, disseram-me, foram bênçãos mistas. Embora tenham sido vitórias conservadoras, estes desenvolvimentos deixaram os Estados Unidos mal equipados para a era pós-Guerra Fria. Os americanos agora possuíam o império mais poderoso da história. Ao mesmo tempo, estavam possuídos por uma das ideologias mais antipolíticas da história: o mercado livre.
 \par 
Este capítulo apareceu originalmente como “Remembrance of Empires Past: 9/11 and the End of the Cold War”, em Cold War Triumphalism: The Misuse of History after the Fall of Communism, ed. Ellen Schrecker (Nova York: New Press, 2004), 274–297.
 \par 
De acordo com os seus idealistas, ou pelo menos um dos seus idealistas, o mercado livre é uma ordem harmoniosa, prometendo uma sociedade civil internacional de troca voluntária, exigindo pouco mais do Estado do que a aplicação ocasional de leis e contratos. Para Buckley e Kristol, esta era uma noção demasiado incruenta para fundar uma ordem nacional, muito menos um império global. Não proporcionou a paixão e o élan, a seriedade e a autoridade que o exercício do poder americano realmente exigia, no país e no estrangeiro. Encorajou a trivialidade e a política mesquinha, o interesse próprio acima do interesse nacional – o que não é a base mais promissora para lançar um império. Além do mais, os direitistas responsáveis ​​pelo Partido Republicano não pareciam perceber isso.
 \par 
“O problema com a ênfase no conservadorismo no mercado”, disse-me Buckley, como vimos no capítulo 5, “é que se torna bastante aborrecido. Você ouve uma vez e domina a ideia. A ideia de dedicar sua vida a isso é horrível, apenas porque é muito repetitivo. É como sexo. O conservadorismo, acrescentou Kristol, “é tão influenciado pela cultura empresarial e pelos modos de pensar empresariais que lhe falta qualquer imaginação política, o que sempre foi, devo dizer, uma propriedade da esquerda”. Kristol confessou um profundo anseio por um império americano: “Qual é o sentido de ser a maior e mais poderosa nação do mundo e não ter um papel imperial? É algo inédito na história da humanidade. A nação mais poderosa sempre teve um papel imperial.” Mas, continuou ele, os impérios anteriores não eram “democracias capitalistas com uma forte ênfase no crescimento económico e na prosperidade económica”. Devido ao seu compromisso com o mercado livre, os Estados Unidos não tinham coragem e visão para exercer o poder imperial. “É uma pena”, lamentou Kristol. “Acho que seria natural para os Estados Unidos. . . Desempenhar um papel muito mais dominante nos assuntos mundiais. Não o que estamos fazendo agora, mas comandar e dar ordens sobre o que está sendo feito. As pessoas precisam disso. Há muitas partes do mundo, em particular África, onde uma
 \par 
Uma autoridade disposta a usar tropas pode fazer uma diferença muito boa, uma diferença saudável.” Mas com a discussão pública moderada por contabilistas, Kristol considerou improvável que os Estados Unidos ocupassem o seu legítimo lugar como sucessores de impérios passados. “Lá está o Partido Republicano se envolvendo em nós. Sobre o que? Prescrições para idosos? Quem se importa? Acho isso nojento. . . A política presidencial do país mais importante do mundo deveria girar em torno de receitas para idosos. Os futuros historiadores acharão isso muito difícil de acreditar. Não é Atenas. Não é Roma. Além disso, não é nada.”{\color{blue}1}
 \par 
Desde o {\color{blue}11} de setembro, tive muitas ocasiões para relembrar essas conversas. O {\color{blue}11} de Setembro, disseram-nos na sequência, chocou os Estados Unidos da paz e prosperidade complacentes que se estabeleceram após a Guerra Fria. Forçou os americanos a olhar para além das suas fronteiras, a compreender finalmente os perigos que enfrentam uma potência mundial. Lembrou-nos os bens da vida cívica e o valor do Estado, pondo fim à fantasia de criar um mundo público a partir de actos privados de troca em interesse próprio. Além disso, restaurou na nossa confusa cultura cívica um sentido de profundidade e seriedade, de coisas “maiores do que nós”. O mais crítico de tudo é que deu aos Estados Unidos um propósito nacional coerente e um foco para o domínio imperial. Um país que durante algum tempo pareceu pouco disposto a enfrentar as suas responsabilidades internacionais estava agora preparado, mais uma vez, para suportar qualquer fardo, pagar qualquer preço, pela liberdade. Esta mudança de atitude, prosseguiu o argumento, foi boa para o mundo. Pressionou os Estados Unidos para criarem uma ordem internacional estável e justa. Também foi bom para os Estados Unidos. Além disso, forçou-nos a pensar em algo mais do que a paz e a prosperidade, lembrando-nos que a liberdade era uma fé combativa e não um poleiro confortável.
 \par 
Como qualquer momento histórico, o {\color{blue}11} de Setembro – não os ataques terroristas ou o dia em si, mas a nova onda de imperialismo que gerou – tem múltiplas dimensões. Alguma parte deste rejuvenescido império
 \par 
A cultura política é o produto de um ataque surpresa a civis e dos esforços dos líderes dos EUA para fornecer alguma medida de segurança a uma população apreensiva. Uma parte dela provém da economia política subterrânea do petróleo, do desejo das elites dos EUA de garantir o acesso às reservas energéticas no Médio Oriente e na Ásia Central, e de utilizar o petróleo como um instrumento de geopolítica. Mas embora estes factores desempenhem um papel considerável na determinação da política dos EUA, não explicam inteiramente a política e a ideologia do próprio momento imperial. Para compreender essa dimensão, devemos olhar para o impacto sobre os conservadores americanos do fim da Guerra Fria, da queda do comunismo e da ascensão do mercado livre como princípio organizador da ordem interna e internacional. Pois foi a insatisfação conservadora com essa ordem que impulsionou, em parte, o seu esforço para criar uma nova.
 \par 
Para os neoconservadores que ficaram entusiasmados com a cruzada de Ronald Reagan contra o comunismo, tudo o que restou depois da Guerra Fria foi a outra paixão de Reagan – o seu empreendedorismo ensolarado e a sua alegria de viver de mercado – que encontrou um lar bem-vindo na América de Bill Clinton. Embora os neoconservadores certamente não se oponham ao capitalismo, eles não acreditam que o mercado livre seja a maior conquista da civilização. A visão deles é mais exaltada. Eles aspiram à grandeza épica de Roma, ao ethos do guerreiro pagão – ou cruzado moral – em vez do ethos do burguês confortável. Desde o fim da Guerra Fria, a visão imperial tem recebido pouca atenção, eclipsada pela adoção dos mercados livres e do comércio livre. Destruídos pelo seu próprio sucesso, os neoconservadores não estão satisfeitos com o mundo que criaram. E assim eles aceitaram o chamado do império, fornecendo o baixo prof desfazer a um coro crescente. Embora tenham plena fé no poder americano, os neoconservadores sentem-se desconfortáveis ​​em usá-lo para a mera extensão do capitalismo. Procuram criar uma ordem internacional que será um monumento para sempre, um mundo que envolve algo mais do que dinheiro e mercados.
 \par 
Mas, como aprendemos, este imperium imaginado pode não proporcionar uma solução tão fácil para os desafios que os Estados Unidos enfrentam. Mesmo antes de a guerra no Iraque ir para o sul, o império americano enfrentava obstáculos assustadores no Médio Oriente e na Ásia Central, sugerindo quão evasiva era a ideia reinante dos imperialistas neoconservadores – de que os Estados Unidos podem governar os acontecimentos, que podem fazer história. – realmente é. (Na verdade, não faz muito tempo que a administração Bush dizia aos jornalistas: “Somos um império agora, e quando agimos, criamos a nossa própria realidade. E enquanto você estuda essa realidade – criteriosamente como quiser – agiremos novamente, criando outras novas realidades, que você pode estudar.”) {\color{blue}2} Internamente, a renovação cultural e política que muitos imaginaram que o {\color{blue}11} de Setembro produziria provou ser uma quimera, vítima de uma ideologia de livre mercado que não mostra nenhuma sinal de diminuição. Acontece que o {\color{blue}11} de Setembro não cumpriu – e, na verdade, provavelmente não poderia – cumprir o papel que lhe foi atribuído pelos neoconservadores do império.
 \par 
Imediatamente após os ataques ao World Trade Center e ao Pentágono, intelectuais, políticos e especialistas — não da esquerda radical, mas conservadores e liberais tradicionais — deram um suspiro audível de alívio, quase como se acolhessem os ataques como uma libertação do miasma que Buckley e Kristol estavam criticando. O World Trade Center ainda estava em chamas e os corpos sepultados ali mal se recuperaram quando Frank Rich anunciou que "o pesadelo desta semana, agora está claro, nos despertou de um sonho frívolo, se não decadente, de uma década". Qual era esse sonho? O sonho de prosperidade, de superar os obstáculos da vida com dinheiro. Durante a década de 1990, Maureen Dowd escreveu, esperávamos "superar a flacidez com dieta e exercícios, rugas com colágeno e Botox, flacidez da pele com cirurgia, impotência com Viagra, alterações de humor com antidepressivos, miopia com cirurgia a laser, decadência com hormônio do crescimento humano, doenças com células-tronco
 \par 
Pesquisa e bioengenharia.” “Renovámos as nossas cozinhas”, observou David Brooks, “remodelámos os nossos sistemas de entretenimento doméstico, investimos em mobiliário de jardim, jacuzzis e grelhadores a gás” – como se a riqueza pudesse libertar-nos da tragédia e das dificuldades. {\color{blue}3} Este espírito teve consequências internas terríveis. Para Francis Fukuyama, encorajava o “comportamento auto-indulgente” e a “preocupação com os próprios assuntos mesquinhos”. Também teve repercussão internacional. De acordo com Lewis “Scooter” Libby, o culto à paz e à prosperidade encontrou a sua expressão mais pura na política externa fraca e distraída de Bill Clinton, que tornou “mais fácil para alguém como Osama bin Laden levantar-se e dizer com credibilidade 'Os americanos não' não tenho estômago para se defender. Eles não sofrerão baixas para defender seus interesses. São moralmente fracos.’” De ​​acordo com Brooks, mesmo o observador mais casual da cena doméstica anterior ao {\color{blue}11} de Setembro, incluindo a Al-Qaeda, “poderia ter concluído que a América não era um país inteiramente sério”.{\color{blue}4}
 \par 
Mas depois daquele dia de Setembro, como afirmaram alguns comentadores, o cenário nacional transformou-se. A América estava agora “mais mobilizada, mais consciente e, portanto, mais viva”, escreveu Andrew Sullivan. George Packer comentou sobre “o estado de alerta, a tristeza, a resolução e até o amor” despertados pelo {\color{blue}11} de Setembro. “O que temo agora”, confessou Packer, “é um retorno à normalidade que todos deveríamos buscar”. Para Brooks, “o medo que prevalece no país” depois do {\color{blue}11} de Setembro foi “um limpador, eliminando grande parte da auto-indulgência da última década”. Reviver o medo eliminou a ansiedade da prosperidade, substituindo uma emoção incapacitante por uma paixão revigorante. “Trocamos as ansiedades da riqueza pelos medos reais da guerra.”{\color{blue}5}
 \par 
Agora, os moradores de luxo que antes passavam horas agonizando sobre qual torneira Moen combinaria com a pia de cobre da cozinha de sua casa de fazenda, de repente estão preocupados com a possibilidade de a água que sai dos canos ter sido envenenada. Pessoas que ansiavam
 \par 
As malas Prada na Bloomingdales ficam subitamente assustadas com malas desacompanhadas no aeroporto. A América, a doce terra da liberdade, está a passar por um curso intensivo de medo.{\color{blue}6}
 \par 
Hoje, concluiu Brooks, “a vida comercial parece menos importante que a vida pública. . . . Quando há lutas de vida ou morte, é difícil pensar em Bill Gates ou Jack Welch como particularmente heróicos.”{\color{blue}7}
 \par 
Os escritores saudaram repetidamente a eletricidade moral galvanizante que agora percorre o corpo político. Uma energia pulsante de determinação pública e compromisso cívico, que restauraria a confiança no governo – talvez, de acordo com alguns liberais, até autorizaria um estado de bem-estar social renovado – e traria uma cultura de patriotismo e conexão, um novo consenso bipartidário, o fim da ironia e as guerras culturais, uma presidência mais madura e mais elevada. {\color{blue}8} De acordo com um repórter do USA Today, o Presidente Bush estava especialmente entusiasmado com a promessa do {\color{blue}11} de Setembro, de trazer ele próprio e a sua geração como Prova A no projecto de renovação interna. “Bush disse aos conselheiros que acredita que confrontar o inimigo é uma oportunidade para ele e os seus colegas baby boomers reorientarem as suas vidas e provarem que têm o mesmo tipo de valor e compromisso que os seus pais demonstraram na Segunda Guerra Mundial.” E embora a fonte específica da euforia de Christopher Hitchens possa ter sido peculiarmente a sua própria, a sua autodeclarada schadenfreude seguramente não o foi: “Talvez eu devesse confessar que no dia {\color{blue}11} de Setembro passado, depois de ter experimentado toda a gama habitual de emoções dos mamíferos, da raiva à náusea, descobri também que outra sensação disputava o domínio. Ao examiná-lo, e para minha própria surpresa e prazer, revelou-se uma alegria. Aqui estava o inimigo mais terrível – a barbárie teocrática – à vista. . . . Percebi que se a batalha continuasse até o último dia da minha vida, eu nunca ficaria entediado em levá-la ao máximo.” {\color{blue}9} Com o seu espectáculo chocante de medo e morte, o {\color{blue}11} de Setembro acabou e uma cultura morta ou moribunda tem a oportunidade de viver novamente.
 \par 
A nível internacional, o {\color{blue}11} de Setembro forçou os Estados Unidos a reaproximar-se do mundo, a assumir o fardo dos impérios sem constrangimento ou confusão. Onde os primeiros George Bush e Bill Clinton se atrapalharam no escuro, em busca de uma doutrina para orientar o exercício do poder dos EUA após o colapso da União Soviética, a missão dos Estados Unidos era agora clara: defender a civilização contra a barbárie, a liberdade contra o terror. Como disse Condoleezza Rice ao New Yorker: “Acho que a dificuldade em definir um papel já passou. Acho que o {\color{blue}11} de Setembro foi um daqueles grandes terremotos que esclarecem e aguçam. Os eventos estão em um relevo muito mais nítido.” Uma América que se pensava estar perdida na areia movediça dos mercados livres, do individualismo e do isolamento foi agora chamada de volta à consciência de um mundo para além das suas fronteiras e inspirada a um compromisso de sustentar baixas em nome de uma ordem global liderada pelos EUA. Como concluiu o antigo subsecretário da Defesa de Clinton, “é pouco provável que os americanos voltem à complacência que marcou a primeira década após a Guerra Fria”. Eles agora entendiam, nas palavras de Brooks, que “o mal existe” e que “para preservar a ordem, as pessoas boas devem exercer poder sobre as pessoas destrutivas”.{\color{blue}10}
 \par 
Uma década depois, é difícil recapturar, e muito menos compreender, a mentalidade daquele momento. Não apenas porque desapareceu tão rapidamente, com o país a recair no seu estranho e azedo partidarismo - onde o volume do antagonismo retórico entre as partes só é igualado pela profundidade do seu acordo sobre os fundamentos económicos (nesse aspecto, nós ' ainda vivemos na América de Bill Clinton) – antes mesmo de o primeiro mandato de Bush ter terminado. Mais desconcertante é como tantos escritores e políticos puderam abrir os braços às consequências políticas das mortes em massa, aproveitando o {\color{blue}11} de Setembro como uma oportunidade para expressar o seu desprezo, aparentemente de longa data, pela própria paz e prosperidade que o precedeu. No dia {\color{blue}12} de Setembro, seria de esperar expressões de pesar pelo rebentamento das bolhas – económica, cultural e política. Em vez de,
 \par 
Muitos viram o {\color{blue}11} de Setembro como um julgamento estrondoso e um corretivo necessário para a frivolidade e o vazio da década de 1990. Teríamos de recuar quase um século – até aos primeiros dias da Primeira Guerra Mundial, quando explodiu o “gás do pântano do tédio e do vazio” que envolvia outro fim de século globalizante e de comércio livre – para encontrar um paralelo remotamente exacto.{\color{blue}11}
 \par 
Para compreender este espírito de alívio exuberante, temos de revisitar os últimos dias da Guerra Fria, quando as elites americanas perceberam pela primeira vez que os Estados Unidos já não seriam capazes de definir a sua missão em termos da ameaça soviética. Embora o fim da Guerra Fria tenha desencadeado uma onda de triunfalismo, também provocou entre as elites uma incerteza ansiosa sobre a política externa dos EUA. Com a derrota do comunismo, muitos perguntaram: como deveriam os Estados Unidos definir o seu papel no mundo? Onde e quando deverá intervir em conflitos estrangeiros? Qual o tamanho do exército que deveria servir?
 \par 
Subjacente a estes argumentos estava um profundo mal-estar relativamente à dimensão e ao propósito do poder americano. Os Estados Unidos pareciam estar a sofrer de um excesso de poder, o que tornava difícil às elites formular quaisquer princípios coerentes para governar a sua utilização. Tal como Richard Cheney, então secretário da Defesa do primeiro presidente Bush, reconheceu em Fevereiro de 1992: “Ganhámos tanta profundidade estratégica que as ameaças à nossa segurança, agora relativamente distantes, são mais difíceis de definir”. Quase uma década depois, os Estados Unidos ainda pareceriam, para os seus líderes, um gigante em dificuldades. Como observou Condoleezza Rice durante a campanha presidencial de 2000: “Os Estados Unidos têm tido enorme dificuldade em definir o seu ‘interesse nacional’ na ausência do poder soviético”. As elites políticas tornaram-se tão incertas sobre o interesse nacional que um importante assessor de defesa de Clinton – e mais tarde reitor da Kennedy School de Harvard – acabou por levantar as mãos em derrota, declarando que o interesse nacional é tudo o que “os cidadãos, após deliberação adequada, o digam”.
 \par 
É” – uma abdicação simplesmente impensável durante o reinado dos Sábios na Guerra Fria.{\color{blue}12}
 \par 
Quando Clinton saiu do gelo, ele e os seus conselheiros fizeram um balanço desta situação sem paralelo – onde os Estados Unidos possuíam tanto poder que não enfrentavam, nas palavras do Conselheiro de Segurança Nacional de Clinton, Anthony Lake, nenhuma “ameaça credível a curto prazo para [os seus ] existência” – e concluiu que as principais preocupações da política externa americana já não eram militares, mas sim económicas. Depois de ensaiar sumariamente os vários perigos militares possíveis para os Estados Unidos, o Presidente Clinton declarou num discurso de 1993: “Ainda enfrentamos, abrangendo todo o resto, este desafio amorfo mas profundo na forma como a humanidade conduz o seu comércio”. O grande imperativo da era pós-Guerra Fria era organizar uma economia global onde os cidadãos do mundo pudessem comercializar através das fronteiras. Para que isso acontecesse, os Estados Unidos tinham de pôr a sua própria casa económica em ordem – “a renovação começa em casa”, disse Lake – reduzindo o défice (em parte através de reduções nas despesas militares), baixando as taxas de juro, apoiando a alta tecnologia. indústria e promoção de acordos de livre comércio. Dado que outras nações também teriam de realizar uma dolorosa reforma económica, Lake concluiu que o objectivo principal dos Estados Unidos era o “alargamento da comunidade mundial livre de democracias de mercado”.{\color{blue}13}
 \par 
A avaliação de Clinton dos desafios que os Estados Unidos enfrentam foi parcialmente inspirada pelo cálculo político. Ele tinha acabado de ganhar uma eleição contra um presidente em exercício que não só liderou os Estados Unidos na vitória na Guerra Fria, mas também planejou uma derrota impressionante sobre os militares iraquianos. Um governador do Sul sem experiência em política externa – e ainda por cima um evasivo do recrutamento – Clinton concluiu que a sua vitória sobre Bush significou que as questões da guerra e da paz já não ressoavam entre os eleitores americanos da mesma forma que teriam em tempos anteriores. {\color{blue}14} Mas a visão de Clinton reflectia também uma convicção, comum à década de 1990, de que a globalização dos países livres
 \par 
O mercado minou a eficácia do poder militar e a viabilidade dos impérios tradicionais. A força já não era o único ou mais eficaz instrumento da vontade nacional. O poder dependia agora do dinamismo e do sucesso da economia de uma nação e da atratividade da sua cultura. Como diria Joseph Nye, secretário adjunto da Defesa de Clinton, o “soft power” – o capital cultural que tornou os Estados Unidos tão admirados em todo o mundo – era tão importante para a preeminência nacional como o poder militar. Talvez pela primeira vez para um funcionário dos EUA, Nye invocou Gramsci para argumentar que os Estados Unidos só manteriam a sua posição de hegemonia se persuadissem – em vez de forçarem – outros a seguirem o seu exemplo. “Se eu conseguir que você queira fazer o que eu quero”, escreveu Nye, “então não preciso forçá-lo a fazer o que você não quer”. {\color{blue}15} Para manter a sua posição no mundo, os Estados Unidos teriam de superar a concorrência de outras economias nacionais, assegurando ao mesmo tempo a difusão do seu modelo de mercado livre e da sua cultura pluralista. O maior perigo que os Estados Unidos enfrentavam era o de não reformarem a sua economia ou de abusarem da sua superioridade militar e provocarem o ódio internacional. O problema não era que os Estados Unidos não tivessem poder suficiente, mas sim que tinham demasiado. Para tornar o mundo seguro para a globalização, os Estados Unidos teriam de ser perturbados ou, no mínimo, significativamente restringidos nas suas aspirações imperiais.
 \par 
Para os conservadores que ansiavam e depois celebraram o fim do socialismo, a promoção de uma prosperidade tranquila por parte de Clinton foi um horror. A riqueza produziu uma sociedade sem dificuldades e adversidades. A satisfação material induziu uma perda de profundidade social e de significado político, uma diminuição da determinação e do entusiasmo heróico. “Naquela era de paz e prosperidade”, escreveria David Brooks, “a sitcom mais popular era Seinfeld, um programa sobre nada”. Robert Kaplan emitiu farpas após farpas sobre os habitantes “saudáveis ​​e bem alimentados” da “sociedade burguesa”, demasiado consumidos com o seu próprio conforto e prazer para emprestar
 \par 
Uma mão – ou uma arma no ombro – para tornar o mundo um lugar mais seguro. “Os bens materiais”, concluiu ele, “encorajam a docilidade”. {\color{blue}16} Ao longo da década de 1990, o principal item de reclamação intelectual, em todo o espectro político, foi que os Estados Unidos não tinham uma mentalidade cívica ou marcial insuficiente, os seus líderes e cidadãos estavam demasiado distraídos pela prosperidade e riqueza para cuidarem das suas instituições herdadas, preocupações comuns e defesa mundial. Supunha-se que o respeito pelo Estado estava a diminuir, tal como a participação política e o voluntariado local. {\color{blue}17} Na verdade, um dos sinais mais reveladores do declínio do imperativo da Guerra Fria foi o facto de a década de 1990 ter começado e terminado com dois incidentes – a controvérsia entre Clarence Thomas e Anita Hill e a decisão do Supremo Tribunal Bush v. suspeita sobre a instituição política mais venerada do país.
 \par 
Para os neoconservadores influentes, a política externa de Clinton era ainda mais anátema. Não porque os neoconservadores fossem unilateralistas, argumentando contra o multilateralismo de Clinton, ou porque fossem isolacionistas ou realistas críticos do seu internacionalismo e humanitarismo. {\color{blue}18} A política externa de Clinton, argumentaram, era demasiado motivada pelos imperativos da globalização do mercado livre. Foi a prova da decadência que tomou conta dos Estados Unidos após a derrota da União Soviética, um sinal de fibra moral enfraquecida e de espírito marcial perdido. Num influente manifesto publicado em 2000, Donald e Frederick Kagan mal conseguiram conter o seu desprezo pela “feliz situação internacional que emergiu em 1991”, que era “caracterizada pela difusão da democracia, do comércio livre e da paz” e que era “tão compatível com a América” com seu amor pelo “conforto doméstico”. Segundo Kaplan, “o problema das sociedades burguesas” como a nossa “é a falta de imaginação”. A mãe do futebol, por exemplo, tão insistentemente defendida tanto por republicanos como por democratas, não se preocupa com o mundo fora dos seus estreitos limites. “A paz”, queixou-se ele, “é prazerosa, e o prazer tem a ver com satisfação momentânea”. Pode
 \par 
Ser obtido “apenas por meio de uma forma de tirania, por mais sutil e branda que seja”. Apaga a memória do conflito estimulante, do desacordo robusto, do luxo de nos definirmos “em virtude de quem enfrentámos”.{\color{blue}19}
 \par 
Embora os conservadores tenham frequentemente a reputação de favorecer a riqueza e a prosperidade, a lei e a ordem, a estabilidade e a rotina – todos os confortos da vida burguesa – os críticos conservadores de Clinton odiavam-no pela sua busca destas mesmas virtudes. As obsessões de Clinton pelo mercado livre traíram uma relutância em abraçar o mundo obscuro do poder e do conflito violento, da tragédia e da ruptura. A sua política externa não era apenas irrealista; estava insuficientemente escuro e taciturno. “O que mais impressiona no zeitgeist dos anos 1990”, queixou-se Brooks, “foi a presunção de harmonia. A época foi moldada pela ideia de que não existiam mais conflitos fundamentais.” Os conservadores prosperam num mundo cheio de mal misterioso e ódios insondáveis, onde o bem está sempre na defensiva e o tempo é um bem precioso na corrida cósmica contra a corrupção e o declínio. Lidar com um mundo assim exige coragem pagã e uma virtude quase bárbara, qualidades que os conservadores abraçam em detrimento dos bens mais prosaicos da paz e da prosperidade. Não é por acaso que Paul Wolfowitz, o mais sombrio destes príncipes sombrios do pessimismo, foi aluno de Allan Bloom (na verdade, Wolfowitz faz uma pequena aparição em Ravelstein, o romance de Saul Bellow sobre Bloom). Pois Bloom – como muitos outros neoconservadores influentes – era um seguidor de Leo Strauss, cujas odes silenciosas à virtude clássica e à harmonia ordenada velavam a sua visão nietzschiana de conflitos torturantes e lutas violentas.{\color{blue}20}
 \par 
Mas havia outra razão para a insatisfação dos neoconservadores com a política externa de Clinton. Muitos deles consideraram-no insuficientemente visionário e consistente. Clinton, alegaram, foi reativo e ad hoc, em vez de proativo e enérgico. Ele e os seus conselheiros não estavam dispostos a imaginar um mundo onde os Estados Unidos moldassem,
 \par 
Em vez de responder, eventos. Rompendo novamente com o estereótipo habitual dos conservadores como pragmáticos não ideológicos, figuras como Wolfowitz, Libby, Kaplan, Perle, Frank Gaffney, Kenneth Adelman e as equipas de pai e filho de Kagan e Kristol apelaram a uma projecção ideologicamente mais coerente dos EUA. poder, onde a “hegemonia benigna” do poderio americano espalharia “a zona da democracia” em vez de apenas alargar o mercado livre. Eles queriam uma política externa que fosse, nas palavras que Robert Kagan mais tarde usaria para elogiar o senador Joseph Lieberman, “idealista mas não ingênua, pronta e disposta a usar a força e comprometida com um exército forte, mas também comprometida em usar o poder americano para espalhar democracia e fazer algo de bom no mundo.” Já no primeiro governo Bush, os neoconservadores insistiam que os Estados Unidos deveriam, nas palavras de Cheney, “moldar o futuro, determinar o resultado da história”, ou, como os Kagans diriam mais tarde, “intervir decisivamente em todas as regiões críticas” do mundo, “se lá existe uma ameaça visível”. Eles criticaram os republicanos, nas palavras de Robert Kagan, que “durante a década muda da década de 1990” sofreram de uma “hostilidade à 'construção da nação', da aversão ao 'trabalho social internacional' e da crença estreita de que 'as superpotências não faça janelas.'” {\color{blue}21} O que esses conservadores desejavam era uma América que fosse genuinamente imperial – não apenas porque acreditavam que isso tornaria os Estados Unidos mais seguros ou mais ricos, e não apenas porque pensavam que isso tornaria o mundo melhor, mas porque eles literalmente queria ver os Estados Unidos fazerem o mundo.
 \par 
Ao nível mais óbvio, o {\color{blue}11} de Setembro confirmou o que os conservadores vinham dizendo há anos: o mundo é um lugar perigoso, cheio de forças hostis que não irão parar diante de nada para ver os Estados Unidos serem derrubados. Mais importante ainda, o {\color{blue}11} de Setembro deu aos conservadores uma oportunidade de articular, sem constrangimento, a visão do poder imperial americano que vinham alimentando discretamente durante décadas. "As pessoas são
 \par 
Agora saindo do armário com a palavra império”, observou com precisão Charles Kraut-hammer logo após o {\color{blue}11} de Setembro. Ao contrário dos impérios do passado, este seria guiado por uma visão benigna e até benéfica de melhoria mundial. Devido ao sentido de justiça e propósito benevolente da América – ao contrário da Grã-Bretanha ou de Roma, os Estados Unidos não tinham intenção de ocupar ou apoderar-se de território próprio – este novo império não geraria a reacção negativa que todos os impérios anteriores tinham gerado. Como disse um redator do Wall Street Journal: “somos um império atraente, aquele ao qual todos desejam aderir”. Nas palavras de Rice: “Teoricamente, os realistas preveriam que quando se tem uma grande potência como os Estados Unidos, não demorará muito até que outras grandes potências se levantem para desafiá-la. E penso que o que estamos a ver é que desta vez há pelo menos uma predileção por avançar para relações produtivas e cooperativas com os Estados Unidos, em vez de tentar equilibrar os Estados Unidos.” {\color{blue}22} Ao criar um império, os Estados Unidos já não teriam de responder a ameaças imediatas, de “esperar pelos acontecimentos enquanto os perigos se acumulam”, como disse o Presidente Bush no seu discurso sobre o Estado da União de 2002. Iria agora “moldar o ambiente”, antecipar ameaças, pensando não em meses ou anos, mas em décadas, talvez séculos. Os objectivos eram os que Cheney, seguindo o conselho de Wolfowitz, delineou pela primeira vez no início da década de 1990: garantir que nenhuma outra potência surgisse para desafiar os Estados Unidos e que nenhuma potência regional alguma vez alcançasse preeminência nos seus teatros locais. A ênfase estava no preventivo e preditivo, para pensar em termos de tornar-se, e não em termos de ser. Como disse Richard Perle, relativamente ao Iraque: “O que é essencial aqui não é olhar para a oposição a Saddam tal como ela é hoje, sem qualquer apoio externo, sem qualquer esperança realista de remover esse terrível regime, mas olhar para no que poderia ser criado.” {\color{blue}23} Para os conservadores, os dois anos após o {\color{blue}11} de Setembro foram uma época inebriante, um momento em que o seu compromisso e hostilidade simultâneos para com o mercado livre puderam finalmente ser satisfeitos. Não mais paralisado por
 \par 
Com a entorpecente política de riqueza e prosperidade, eles acreditavam que podiam contar com o público para responder ao chamado do sacrifício e do destino, do confronto e do mal. Sendo “perigo” e “segurança” as palavras de ordem do dia, o Estado americano seria novamente santificado – sem abrir as comportas à redistribuição económica. Eles esperavam que o {\color{blue}11} de Setembro e o império americano resolveriam finalmente as contradições culturais do capitalismo que Daniel Bell tinha notado há muito tempo, mas que só vieram verdadeiramente à tona após a derrota do comunismo.
 \par 
Que diferença faz uma década – ou mesmo alguns anos. Muito antes de os Estados Unidos terem essencialmente de declarar vitória no Iraque e (mais ou menos) regressar a casa, muito antes de George W. Bush deixar o seu gabinete em desgraça, muito antes de a guerra no Afeganistão provar ser muito mais do que o povo americano poderia estômago, ficou claro que o império neoconservador assentava sobre uma base instável. No final de Outubro e início de Novembro de 2001, por exemplo, depois de apenas algumas semanas de bombardeamentos não terem conseguido desalojar os Taliban, os críticos começaram a murmurar os seus receios de que a guerra no Afeganistão fosse uma reprise do atoleiro do Vietname. {\color{blue}24} Assim que a guerra no Iraque pareceu não ser exactamente a moleza que os seus defensores proclamaram que seria, os Democratas começaram a sondar, ainda que hesitantemente, os limites da crítica aceitável. Já na campanha presidencial de 2004, expressar críticas à guerra tornou-se uma espécie de teste decisivo entre os candidatos democratas.
 \par 
Nenhum destes críticos, é claro, desafiaria a premissa militar a todo vapor das políticas de Bush – e mesmo sob Obama, poucos questionariam as premissas básicas do alcance global da América – mas o aparecimento periódico de tais críticos, particularmente em tempos de dificuldades ou de derrota , sugere que a visão imperial só é politicamente viável enquanto for bem sucedida. É assim que deve ser: porque a peça central da promessa imperial é que os Estados Unidos podem governar os acontecimentos,
 \par 
Que pode determinar o resultado da história, a promessa permanece ou cai em caso de sucesso ou fracasso. Com qualquer sugestão de que os acontecimentos estão fora do controlo do império, a visão imperial confunde-se. Na verdade, demorou apenas uma semana, em Março de 2002, de terrível derramamento de sangue em Israel e nos Territórios Ocupados – e as acusações resultantes de que “Bush toca violino na Casa Branca ou no Texas, tocando Nero enquanto o Médio Oriente arde” – para que o império planeado fosse chamado em questão. Mal a violência no Médio Oriente começou a aumentar, até os defensores da administração começaram a abandonar o navio, sugerindo que qualquer invasão do Iraque teria de ser adiada indefinidamente. Como disse um dos assessores de segurança nacional de alto nível de Reagan: “A suprema ironia é que a maior potência que o mundo alguma vez conheceu provou ser incapaz de gerir uma crise regional”. O facto, acrescentou este assessor, de a administração ter estado tão maniacamente “concentrada no Afeganistão ou no Iraque” – os dois principais postos avançados do confronto imperial – enquanto o Médio Oriente ardia em chamas, “reflecte uma arrogância ou uma ignorância terríveis”.{\color{blue}25}
 \par 
Ironicamente, na medida em que a administração Bush evitou esses conflitos, como aquele entre Israelitas e Palestinianos, onde poderia falhar - e, de facto, no momento em que este livro foi escrito, a administração Obama parece estar a seguir o mesmo caminho no que diz respeito a Israel e à Palestina – foi forçado a renunciar à própria lógica do imperialismo que procurava confessar. Tendo como premissa a capacidade dos Estados Unidos para controlar os acontecimentos, a visão imperial neoconservadora não podia acomodar-se ao fracasso. Mas, ao evitarem o fracasso, os imperialistas foram forçados a reconhecer que não podiam controlar os acontecimentos. Como observou o ex-secretário de Estado Lawrence Eagleburger sobre o conflito israelo-palestiniano, Bush percebeu “que simplesmente inserir-se nesta confusão sem qualquer possibilidade de obter qualquer sucesso é, por si só, perigoso, porque demonstraria que de facto nós não temos nenhuma capacidade neste momento de controlar ou afetar os acontecimentos”26 – precisamente a admissão que os neoconservadores não poderiam
 \par 
Pagar para fazer. Este beco sem saída não foi um mero problema de lógica ou consistência: traiu a fragilidade essencial da própria posição imperial.
 \par 
Essa fragilidade reflectiu também o vazio interno da visão imperial dos neoconservadores. Embora os neoconservadores vissem e continuem a ver o imperialismo como a contrapartida cultural e política do mercado livre, nunca chegaram a um acordo – nem mesmo dez anos depois – sobre a forma como a oposição conservadora aos gastos do governo e o compromisso com as reduções de impostos tornam os Estados Unidos dificilmente fará os investimentos necessários na construção da nação que o imperialismo exige.
 \par 
Internamente, há pouca evidência para sugerir que a renovação política e cultural imaginada pela maioria dos comentaristas — o renascimento do estado, o retorno do sacrifício compartilhado e da comunidade, o aprofundamento da consciência moral — tenha ocorrido, mesmo nos dias mais inebriantes após o {\color{blue}11} de setembro. De todos os incidentes que se poderia citar daquela época, dois se destacam. Em março de 2002, sessenta e dois senadores, incluindo dezenove democratas, rejeitaram padrões mais altos de eficiência de combustível na indústria automobilística, o que teria reduzido a dependência do petróleo do Golfo Pérsico. O republicano do Missouri Christopher Bond sentiu-se tão desimpedido pela necessidade de prestar homenagem às instituições estatais em tempos de guerra que afirmou no plenário do Senado: "Não quero dizer a uma mãe no meu estado natal que ela não deve comprar um SUV porque o Congresso decidiu que seria uma má escolha". Ainda mais revelador foi o quão vulneráveis ​​os proponentes de padrões mais elevados eram a esses argumentos antiestatistas. John McCain, por exemplo, foi imediatamente colocado na defensiva pela noção de que o governo estaria interferindo nas escolhas de mercado privado das pessoas. Ele foi deixado para argumentar que “nenhum americano será forçado a dirigir um automóvel diferente”, como se isso fosse uma imposição terrível nesta nova era de sacrifício e solidariedade em tempos de guerra.{\color{blue}27}
 \par 
Alguns meses antes, Ken Feinberg, chefe do Fundo de Compensação das Vítimas do {\color{blue}11} de Setembro, anunciou que as famílias das vítimas receberiam uma compensação pelas suas perdas com base, em parte, no salário que cada vítima recebia no momento da sua morte. Após os ataques ao World Trade Center e ao Pentágono, o Congresso tomou a medida sem precedentes de assumir a responsabilidade nacional pela restituição às famílias das vítimas. Embora a inspiração para esta decisão tenha sido a prevenção de dispendiosos processos judiciais contra a indústria aérea, muitos observadores consideraram-na como um sinal de um novo espírito no país: face à tragédia nacional, os líderes políticos estavam finalmente a romper com o espírito de sobrevivência na selva do governo Reagan. -Clinton anos. Mas mesmo na morte, o mercado – e as desigualdades que gera – era a única língua que os líderes da América sabiam falar. Abandonando a noção de sacrifício partilhado, Feinberg optou pelas tabelas actuariais para calcular pacotes de remuneração adequados. A família de uma avó solteira de {\color{blue}65} anos que ganha {\color{blue}10} mil dólares por ano – talvez uma trabalhadora de cozinha que receba um salário mínimo – retiraria {\color{blue}300} mil dólares do fundo, enquanto a família de um comerciante de Wall Street de {\color{blue}30} anos receberia {\color{blue}3} {\color{blue}870} 064 dólares. Os homens e mulheres mortos no {\color{blue}11} de Setembro não eram cidadãos de uma democracia; eles eram assalariados e as recompensas seriam distribuídas de acordo. Praticamente ninguém – nem mesmo os comentadores e políticos que denunciaram o cálculo de Feinberg por outras razões – criticou este aspecto da sua decisão.{\color{blue}28}
 \par 
Mesmo dentro e em torno das forças armadas, o espírito do patriotismo e do destino partilhado permaneceu secundário em relação à ideologia do mercado. Num artigo pouco notado de Outubro de 2001 no New York Times, os recrutadores militares confessaram que ainda procuravam atrair os alistados não com o apelo do patriotismo ou do dever, mas com a promessa de oportunidades económicas. Como disse um recrutador: “Tudo continua como sempre. Não forçamos a rotina de ‘Ajudar o nosso país’.” Quando um patriota ocasional irrompeu em uma área de recrutamento e disse:
 \par 
“Eu quero lutar”, explicou um recrutador, “tenho que acalmá-los. Não somos apenas lutadores e bombardeios. Estamos falando de empregos. Além disso, tratamos de educação.” {\color{blue}29} Os recrutadores admitiram que continuaram a visar imigrantes e pessoas de cor, no pressuposto de que foi a falta de oportunidades destes círculos eleitorais que os levou ao serviço militar. Na verdade, o objectivo publicamente reconhecido do Pentágono era aumentar o número de latinos nas forças armadas de 10% para 22%. Os recrutadores chegaram mesmo a infiltrar-se no México, com promessas de cidadania instantânea aos não-cidadãos pobres dispostos a pegar em armas em nome dos Estados Unidos. De acordo com um recrutador de San Diego, “é uma prática mais ou menos comum que alguns recrutadores vão a Tijuana para distribuir panfletos ou, em alguns casos, procuram alguém para ajudar a distribuir informações no lado mexicano”. {\color{blue}30} Em Dezembro de 2002, enquanto os Estados Unidos se preparavam para invadir o Iraque, o congressista democrata de Nova Iorque, Charles Rangel, decidiu enfrentar esta questão de frente, propondo a reintegração do projecto. Observando que os imigrantes, as pessoas de cor e os pobres suportavam uma percentagem maior da carga militar do que o seu número na população justificava, Rangel argumentou que os Estados Unidos deveriam distribuir os custos internos do império de forma mais equitativa. Se as crianças brancas da classe média fossem forçadas a usar os braços, afirmou ele, a administração e os seus apoiantes poderiam pensar duas vezes antes de ir para a guerra. A conta não deu em nada.
 \par 
O facto de a guerra nunca ter imposto à população o tipo de sacrifícios que normalmente acompanham as cruzadas nacionais provocou uma preocupação significativa entre as elites políticas e culturais. “O perigo, a longo prazo”, escreveu R. W. Apple do Times antes de morrer, “é a perda de interesse. Com grande parte da guerra a ser conduzida fora da vista de todos, por comandos, diplomatas e agentes de inteligência, será que uma nação que passou décadas em fácil auto-indulgência permanecerá focada?” Não muito depois de ter declarado o fim da era do brilho e do brilho, Frank Rich se viu publicamente angustiado por “você ter
 \par 
Nunca imagine que esta é uma nação em guerra.” Antes do {\color{blue}11} de Setembro, “a administração disse que poderíamos ter tudo”. Desde o {\color{blue}11} de Setembro, a administração vinha dizendo praticamente a mesma coisa. Um ex-assessor de Lyndon Johnson disse ao New York Times: “As pessoas vão se envolver nisso. Até agora é um esforço do governo, como deveria ser, mas as pessoas não estão engajadas.” {\color{blue}31} Sem consagrar a causa com sangue, temiam os observadores, os americanos não veriam o seu compromisso testado e a sua determinação aprofundada. Como Doris Kearns Goodwin reclamou no The News-Hour:
 \par 
Bem, penso que o problema é que compreendemos que será uma guerra longa, mas é difícil para nós participar nessa guerra de mil e uma maneiras, como poderíamos na Segunda Guerra Mundial. Você poderia ter centenas de milhares ingressando nas forças armadas. Eles poderiam ir às fábricas para garantir a construção desses navios, tanques e armas. Eles poderiam ter jardins da vitória. Além disso, eles poderiam sentir não apenas o que nos dizem: voltem para suas vidas normais. É mais difícil agora. Não temos um rascunho da mesma forma que tínhamos, embora haja alguma indicação de que gostaria de acreditar que a geração mais jovem desejará participar. Meu filho mais novo, que acabou de se formar em Harvard em junho, ingressou no exército. Ele quer esse compromisso de três anos. Ele quer fazer parte de tudo isso, em vez de apenas trabalhar por um ano e cursar direito, ele quer fazer parte disso. E suspeito que haverá muitos outros assim também. Mas de alguma forma você continua desejando que o governo nos desafie. Talvez precisemos de um Projeto Manhattan para a produção desta vacina contra antibióticos. Conseguimos reduzir os navios de carga de {\color{blue}365} dias na Segunda Guerra Mundial para um dia no meio com esse tipo de empreendimento coletivo. E acho que precisamos estar mobilizados, nosso espírito, nossa produtividade, muito mais do que estávamos.{\color{blue}32}
 \par 
No que pode ter sido o espetáculo mais estranho de toda a guerra, os líderes da nação acabaram lutando para encontrar coisas para as pessoas fazerem – não porque houvesse muito a ser feito, mas porque, sem algo para fazer, o ardor dos americanos comuns iria esfriar. Dado que estas tarefas eram desnecessárias e que a sua obrigatoriedade violaria as normas da ideologia do mercado, o melhor que o presidente e os seus colegas conseguiram fazer foi anunciar websites e números gratuitos onde homens e mulheres empreendedores pudessem encontrar informações sobre como ajudar o esforço de guerra. Como Bush declarou na Carolina do Norte no dia seguinte ao seu discurso sobre o Estado da União em 2002: “Se você ouviu o discurso de ontem à noite, você sabe, as pessoas estavam dizendo: 'Bem, meu Deus, isso é legal, ele me chamou para a ação, onde posso fazer isso? Eu olho? Bem, aqui é onde: no usafreedomcorps. Governador ou você pode ligar para este número - parece que estou fazendo uma proposta, e estou. Esta é a coisa certa a fazer pela América. 1-877-EUA-CORPS.” O governo não podia nem contar com os cidadãos para pagar a ligação. E quais eram as funções que esses voluntários deveriam desempenhar? Se fossem médicos ou profissionais de saúde, poderiam se alistar para ajudar em emergências. E todos os outros? Eles poderiam servir em programas de vigilância de bairro para proteção contra ataques terroristas – na Carolina do Norte.{\color{blue}33}
 \par 
Desde o fim da Guerra Fria, alguns poderiam até dizer do Vietname, tem havido uma desconexão crescente entre a cultura e a ideologia das elites empresariais dos EUA e a de guerreiros políticos como Wolfowitz e os outros neoconservadores. Onde a Guerra Fria viu a criação de uma classe semicoerente de Homens Sábios que uniram, ainda que de forma irregular, os mundos dos negócios e da política – homens como Dean Acheson e os irmãos Dulles – os anos Reagan e mais além testemunharam algo completamente diferente. Por um lado, temos uma geração mais jovem de magnatas empresariais que, embora implacáveis ​​nos seus esforços para garantir benefícios do Estado,
 \par 
Não têm o respeito ou a paixão pelo estado dos seus homólogos mais velhos. Certamente dispostos a tirar proveito do público, eles desprezam a política e o governo. Estes novos CEOs respondem aos seus homólogos em Tóquio, Londres e outras cidades globais; desde que o Estado lhes forneça o que necessitam e não interfira indevidamente nas suas operações, eles deixam isso para os apparatchiks. {\color{blue}34} Questionado por Thomas Friedman sobre a frequência com que fala sobre o Iraque, a Rússia ou guerras estrangeiras, um executivo de Silicon Valley disse: “Não mais do que uma vez por ano. Nós nem nos importamos com Washington. O dinheiro é extraído pelo Vale do Silício e depois desperdiçado por Washington. Quero falar sobre pessoas que criam riqueza e empregos. Não quero falar sobre pessoas pouco saudáveis ​​e improdutivas. Se não me importo com os destruidores de riqueza no meu próprio país, por que deveria me preocupar com os destruidores de riqueza em outro país?”{\color{blue}35}
 \par 
Por outro lado, temos uma nova classe de elites políticas que têm pouco contacto com a comunidade empresarial, cujas principais experiências fora do governo têm sido na academia, no jornalismo, em grupos de reflexão ou em qualquer outra parte da indústria cultural. Homens como Wolfowitz e Brooks, os Kagans e os Kristols, trafegam em ideias e veem o mundo como uma paisagem de projeção intelectual. Livres dos constrangimentos mesmo dos interesses mais interessados, consideram-se livres para fazer avançar a sua causa, no Médio Oriente e noutros lugares. Tal como os seus homólogos empresariais, os neoconservadores vêem o mundo como o seu palco; ao contrário dos seus homólogos empresariais, estão a preparar-se para um drama totalmente mais teatral e sobrenatural. O seu objectivo final, se é que existe, é um confronto apocalíptico entre o bem e o mal, a civilização e a barbárie – categorias de conflito pagão diametralmente opostas à visão de mundo sem fronteiras da elite globalizante e de comércio livre da América.