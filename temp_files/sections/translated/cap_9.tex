\chapter{8 As Composições do Capital}\label{8 As Composições do Capital}
 \par 
Este capítulo explica os conceitos de Marx sobre as composições técnicas, orgânicas e de valor do capital, como um prelúdio ao estudo da lei da tendência de queda da taxa de lucro (LTRPF) e do problema da transformação, nos Capítulos {\color{blue}9} e 10, respectivamente. Este prelúdio é importante por duas razões. Em primeiro lugar, embora as composições do capital sejam essenciais para a compreensão da relação entre valores e preços, mudanças técnicas, crises económicas e outras estruturas e processos na economia capitalista, elas têm sido geralmente explicadas superficialmente e compreendidas apenas superficialmente (e muitas vezes incorrectamente) em a literatura. Em segundo lugar, o LTRPF é tradicionalmente visto como tendo apenas uma relação passageira com o problema da transformação. Isto está errado, pois eles estão intimamente relacionados entre si através das composições do capital.
 \par 
\[A Composição Técnica do Capital\]
 \par 
No Volume {\color{blue}1} de O Capital, Marx examina o método capitalista de produção, ou seja, a maneira sistemática pela qual o capitalismo transforma o processo de trabalho por meio do sistema fabril e se apropria das outras condições de produção, por exemplo, os recursos naturais (ver Capítulos {\color{blue}6} e {\color{blue}15}). Neste volume, Marx também estabelece a tendência da produtividade do trabalho aumentar sistematicamente sob o capitalismo, o que é capturado pelo conceito de composição técnica do capital (TCC).
 \par 
O TCC é a razão física entre os insumos materiais consumidos e o trabalho vivo socialmente necessário para transformar esses insumos em produtos. Embora Marx mostre que a TCC tende a aumentar ao longo do tempo (sendo esta a expressão do aumento da produtividade do trabalho sob o capitalismo), tenta medir a TCC e as suas mudanças, ou contrastar a composição técnica dos capitais em diferentes sectores (agricultura e electricidade geração, por exemplo) enfrentam um problema grave: a CCT não pode ser medida diretamente, porque é a razão entre um conjunto heterogêneo de valores de uso (os insumos materiais) e as quantidades médias de trabalho gastas em cada empresa ou setor. Por outras palavras, a CCT só pode ser medida por um único índice na medida em que uma massa heterogénea de matérias-primas e de trabalho vivo é reduzida a um denominador comum.
 \par 
Para a teoria dominante, a medição do TCC é o que chamamos de “problema de número de índice”. Em contraste, na teoria de Marx, os valores das mercadorias constituem a base sobre a qual o TCC pode ser medido. Não se trata simplesmente da escolha de um índice em vez de outro. Reflete a proposição de Marx de que o valor é uma categoria legítima de análise para uma sociedade capitalista. Nesta sociedade, como foi demonstrado no Capítulo 2, diferentes trabalhos são regular e necessariamente colocados em equivalência entre si na produção e na troca, estabelecendo o domínio das relações de valor dentro do capitalismo. As medições de valor do TCC são legítimas (em vez de simplesmente “convenientes”, com as desvantagens associadas a qualquer número de índice) porque expressam as realidades subjacentes da produção, bem como as mudanças sistemáticas nas condições de produção sob o capitalismo, em termos de as relações sociais e de valor em que estão incorporados.
 \par 
\[A Composição Técnica do Capital\]
 \par 
Além da composição técnica, Marx distingue entre as composições orgânica e de valor do capital (OCC e VCC). O OCC e o VCC raramente foram distinguidos na literatura subsequente e muitas vezes têm sido usados ​​de forma intercambiável. Para ambos, a definição algébrica tem sido geralmente denotada por c ⁄ v (capital constante dividido por capital variável). No entanto, isto levanta a questão: que valores estão a ser utilizados para reduzir o conjunto heterogéneo de matérias-primas, no caso de c, e de trabalho vivo, no caso de v, a dimensões de valor único? Este é um problema pertinente neste contexto, uma vez que a utilização que Marx faz da composição do capital está preocupada com a acumulação e, portanto, com a redução sistemática dos valores das mercadorias através da mudança técnica (ver Capítulo {\color{blue}3}).
 \par 
Antes de abordar este problema no contexto dinâmico da acumulação, é útil, para fins expositivos, distinguir o VCC e o OCC num contexto estático. Consideremos, por exemplo, a produção de joias. Suponhamos que sejam utilizados exatamente o mesmo processo de trabalho e as mesmas máquinas e tecnologia para produzir anéis de prata e de ouro. Neste caso, ambos os processos produtivos terão o mesmo TCC, pois este mede a quantidade de matéria-prima relativa ao trabalho vivo. Mas a produção de anéis de ouro envolverá um VCC mais elevado, uma vez que utiliza matérias-primas de maior valor (ouro em oposição à prata). Para refletir a falta de diferença nos processos de produção do ponto de vista técnico, Marx define o OCC como igual para os dois processos de produção. Segue-se que o OCC mede o TCC em termos de valor, mas deixando de lado as diferenças criadas pelo maior ou menor valor das matérias-primas utilizadas.
 \par 
Isto cria alguma dificuldade na medição do OCC, uma vez que os valores apropriados para definir a razão de c para v não são especificados. Deveríamos, por exemplo, usar o valor do ouro, o valor da prata ou algo intermediário? Este problema de medição é criado pela tentativa de fazer a distinção num contexto estático, no qual o TCC e o VCC por si só seriam suficientes. Só quando os processos de produção estão a mudar é que a distinção entre o OCC e o VCC pode ajudar a esclarecer a equivalência ou não entre (mudanças nos) processos de produção do ponto de vista orgânico.
 \par 
Consideremos agora um exemplo dinâmico, envolvendo a indústria siderúrgica. Suponhamos que, devido a melhorias técnicas na produção, o valor do aço caia, com todo o resto constante. Quando um factor de produção amplamente utilizado como o aço se torna mais barato, o VCC em todos os sectores da economia muda de acordo com o conteúdo relativo do aço no seu capital constante e no valor da força de trabalho. Num caso simples, com uma força de trabalho homogênea, os VCCs variarão de acordo com o uso relativo do aço. Apesar destas alterações nos CCC, os OCC nos sectores não siderúrgicos permanecerão inalterados, porque - em primeira instância - não houve alteração nos seus CCT. Em contrapartida, o CCO da indústria siderúrgica aumentou (juntamente com o seu TCC) devido ao aprimoramento tecnológico original. Este exemplo mostra que o OCC mede as mudanças na produção em termos de valor, e que o OCC pode medir algo distinto do VCC (e, portanto, torna-se relevante na prática) apenas quando o TCC muda.
 \par 
Os dois exemplos dados acima servem para explicar a diferença entre o VCC e o OCC. A questão é diferente quando começamos a considerar condições de produção em constante mudança em toda a economia. Marx argumenta que, na sua fase desenvolvida, o capitalismo envolve a acumulação através da produção de mais-valia relativa, com a maquinaria a substituir sistematicamente o trabalho vivo. Isto resulta numa tendência para um aumento do TCC em toda a economia. Neste caso, o TCC pode ser medido em termos de valor de duas maneiras diferentes.
 \par 
Por um lado, apenas do ponto de vista das alterações na produção, o TCC é medido pelo OCC. As matérias-primas e a força de trabalho entram no processo de produção com determinados valores, conduzindo a um rácio definido entre capital constante e capital variável, de acordo com a medida em que o trabalho é coagido a transformar insumos em produtos. Se o colocássemos cronologicamente, o OCC mede o TCC nos valores “antigos” prevalecentes antes das alterações técnicas e da renovação do processo de produção. Por outro lado, sempre que há progresso técnico em algum ponto da economia há uma mudança (redução) nos valores das mercadorias. O VCC é medido após esta etapa, levando em consideração o TCC do ponto de vista da variação tanto do OCC quanto dos valores das mercadorias à medida que são realizadas em troca. Em termos cronológicos, o VCC é medido nos valores “novos” e não nos “antigos”. Em suma, o VCC capta as implicações contraditórias do aumento do TCC, bem como a queda dos valores das mercadorias devido ao progresso tecnológico. Portanto, o VCC tende a subir mais lentamente que o TCC e o OCC.
 \par 
A descrição da diferença entre o VCC e o OCC em termos de valores novos e antigos é conceptual e não cronológica: a qualquer momento, alguns capitais entrarão no processo de produção enquanto outros o abandonarão, enquanto a mudança técnica é omnipresente. O que a distinção faz é basear-se e construir num contexto mais complexo a separação entre as esferas da produção e da troca (ver Capítulo {\color{blue}4}). Na produção, as duas classes de capitalistas e trabalhadores confrontam-se durante o processo de produção e, à medida que a acumulação prossegue, há uma tendência para o aumento do TCC. Em troca, os capitalistas confrontam-se como concorrentes no processo de compra e venda e, à medida que a acumulação prossegue, há uma tendência para a redução dos valores e para o declínio do VCC. É mostrado no próximo capítulo que a interação destes processos é a principal preocupação da LTRPF de Marx. No Capítulo 10, a relação entre valores e preços em Marx é explicada através do papel do OCC na sua análise.
 \par 
\[A Composição Técnica do Capital\]
 \par 
Como mencionado, a literatura tem sido descuidada no tratamento da composição do capital. Geralmente, no contexto do LTRPF, a maior atenção tem sido centrada, pelo menos na terminologia, no OCC, com pouca consideração pelo TCC e pelo VCC. Ironicamente, apesar da predominância terminológica do OCC, o VCC tem sido o que se entende na prática. Isto reflecte o desrespeito pelas próprias distinções de Marx e a má interpretação do seu trabalho e intenção, colapsando a forma como as composições orgânicas e de valor são formadas distintamente (na produção e na troca, respectivamente) num único processo.
 \par 
Não surpreende que a literatura específica sobre a composição do capital seja escassa. Marx explica seus conceitos em Marx (1969, cap. {\color{blue} 12 } {\par} , 1972, cap. {\color{blue} 23 } {\par} , 1981a, cap.{\color{blue}8}). A interpretação neste capítulo baseia-se em Ben Fine (1990a) e Ben Fine e Laurence Harris (1979, cap.{\color{blue}4}). Esta interpretação é revista e desenvolvida à luz da literatura existente por Alfredo Saad-Filho (1993, 2001, 2002, cap.{\color{blue}6}). Para uma crítica vigorosa da nossa posição, ver Moseley (2015, cap.{\color{blue}11}).