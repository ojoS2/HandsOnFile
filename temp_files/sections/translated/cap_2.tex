\chapter{1 História e Método}\label{1 História e Método}
 \par 
Ao longo de sua vida adulta, Marx buscou a transformação revolucionária da sociedade capitalista, mais notoriamente por meio de seus escritos, mas também por meio da agitação e organização da classe trabalhadora — por exemplo, entre 1864 e 1876, ele foi um dos líderes da Primeira Associação Internacional dos Trabalhadores. Em suas obras escritas, Marx tenta descobrir o processo geral de mudança histórica, aplicar esse entendimento a tipos particulares de sociedades e fazer estudos concretos de situações históricas específicas. Este capítulo analisa brevemente o desenvolvimento intelectual de Marx e as principais características de seu método. O restante do livro analisa em mais detalhes outros aspectos de sua obra, especialmente aqueles encontrados nos três volumes de O Capital, sua principal obra de economia política.
 \par 
\section{Filosofia de Marx}
 \par 
Karl Marx nasceu na Alemanha em 1818 e iniciou uma carreira universitária estudando Direito. Seu interesse rapidamente se voltou para a filosofia, que, naquela época, era dominada por Hegel e seus discípulos. Eram idealistas, acreditando que a realidade é o resultado de um sistema de conceitos em evolução, ou movimento em direcção à “Ideia Absoluta”, com uma estrutura de conceitos que liga o relativamente abstracto ao cada vez mais concreto. Os hegelianos acreditavam que o progresso intelectual explica o avanço do governo, da cultura e de outras formas de vida social. Portanto, o estudo da consciência é a chave para a compreensão da sociedade, e a história é um palco dramático no qual instituições e ideias lutam pela hegemonia. Neste conflito sempre presente, cada estágio de desenvolvimento é um avanço em relação aos que o precederam, mas também absorve e transforma elementos deles; isto é, contém as sementes de sua própria transformação em um estágio superior. Este processo de mudança, no qual as novas ideias não derrotam as antigas, mas resolvem conflitos ou contradições dentro delas, Hegel chamou de dialética.
 \par 
Hegel morreu em 1831. Quando Marx ainda era um jovem na universidade, dois grupos opostos de hegelianos, os Jovens (radicais) e os Velhos (reacionários), afirmavam ser os sucessores legítimos de Hegel. Os Antigos Hegelianos acreditavam que a monarquia absoluta prussiana, a religião e a sociedade representavam a conquista triunfante da Idéia em seu progresso dialético. Em contrapartida, os Jovens Hegelianos, perigosamente anti-religiosos, acreditavam que o desenvolvimento intelectual ainda tinha muito a avançar. Isto preparou o terreno para uma batalha entre as duas escolas, cada lado acreditando que uma vitória anunciava o progresso da sociedade alemã. Tendo observado o absurdo, a pobreza e a degradação de grande parte da vida alemã, Marx identificou-se inicialmente com os Jovens Hegelianos.
 \par 
No entanto, sua simpatia pelos Jovens Hegelianos foi extremamente curta, em grande parte pela influência de Feuerbach, que era um materialista. Isso não significa que Feuerbach estava grosseiramente interessado em seu próprio bem-estar - na verdade, suas visões divergentes lhe custaram sua carreira acadêmica. Ele acreditava que, longe da consciência humana dominar a vida e a existência, eram as necessidades humanas que determinavam a consciência. Em A Essência do Cristianismo, Feuerbach montou uma polêmica simples, mas brilhante, contra a religião. Os humanos precisavam de Deus porque a religião satisfazia uma necessidade emocional. Para satisfazer essa necessidade, os humanos projetaram suas melhores qualidades em uma figura de Deus, adorando o que eles haviam criado imaginativamente em pensamento a tal ponto que Deus havia assumido uma existência independente na consciência humana. Para recuperar sua humanidade, as pessoas precisam substituir o amor de Deus pelo amor umas pelas outras.
 \par 
Marx ficou imediatamente impressionado com esse insight. Inicialmente, ele criticou Feuerbach por ver as pessoas como indivíduos que lutam para cumprir uma determinada “natureza humana”, e não como seres sociais. No entanto, ele logo ultrapassou o materialismo de Feuerbach. Ele fez isso de duas maneiras. Primeiro, ele estendeu a filosofia materialista de Feuerbach a todas as ideias dominantes prevalecentes na sociedade, para além da religião, à ideologia e à concepção que as pessoas têm da sociedade como um todo. Em segundo lugar, estendeu as ideias de Feuerbach à história. A análise de Feuerbach foi inteiramente histórica e não dialética: os humanos satisfazem uma necessidade emocional através da religião, mas as origens e a natureza dessa necessidade permanecem inexplicáveis ​​e imutáveis, sejam satisfeitas por Deus ou de outra forma. Marx vê a solução para este problema nas condições materiais. A consciência humana é crucial no pensamento de Marx, mas só pode ser compreendida em relação às circunstâncias históricas, sociais e materiais. Desta forma, Marx estabelece uma estreita relação entre a dialética e a história, que se tornaria uma pedra angular do seu próprio método. A consciência é determinada principalmente pelas condições materiais, mas estas evoluem dialeticamente ao longo da história humana.
 \par 
Esta explicação revela uma propriedade comum no pensamento de Hegel, dos seus vários discípulos e críticos, e de Marx - que as coisas nem sempre aparecem imediatamente como são. Para Feuerbach, por exemplo, Deus não existe senão na mente, mas parece, ou é considerado, como um ser independente e, portanto, é capaz de satisfazer uma necessidade humana. Sob o capitalismo, um mercado de trabalho livre esconde a exploração; a existência de democracia política sugere igualdade e não a realidade das instituições políticas que apoiam a reprodução de privilégios e poder. Este divórcio entre a realidade (conteúdo ou essência) e a forma como ela aparece (de) é um aspecto central do pensamento dialético de Marx. Estabelece a ligação entre conceitos abstratos (como classe, valor e exploração) e a sua presença na vida quotidiana (através de salários, preços e lucros).
 \par 
A tarefa que Marx se propõe, principalmente para o capitalismo, é traçar a conexão e as contradições entre o abstrato e o concreto. Ele reconhece que isto é extremamente exigente, uma vez que, nas suas próprias palavras (no prefácio de 1872 à edição francesa do Capital), “[t]aqui não existe um caminho real para a ciência”. O projeto envolve a adoção de um método adequado, um ponto de partida criterioso na escolha dos conceitos abstratos (ponto de partida para a análise) e um desdobramento cuidadoso do conteúdo histórico e lógico de cada novo conceito, a fim de revelar a relação entre o modo como as coisas são e a maneira como parecem ser.
 \par 
Significativamente, como ficará claro a partir da discussão de Marx sobre o fetichismo da mercadoria (no Capítulo {\color{blue}2}), as aparências não são necessariamente simplesmente falsas ou ilusórias como, por exemplo, nas crenças religiosas na existência de Deus. Não podemos desejar os salários, os lucros e os preços, mesmo quando os reconhecemos como a forma pela qual o capitalismo organiza a exploração, tal como não podemos desejar os poderes do monarca ou do padre quando nos tornamos republicanos ou ateus, respectivamente. Pois, no caso dos salários, preços e lucros, as aparências são parte integrante da própria realidade, representando e ocultando aspectos mais fundamentais do capitalismo que uma dialética apropriada pretende revelar. Como essa complexidade pode ser desvendada?
 \par 
\section{Filosofia de Marx}
 \par 
Em contraste com os seus extensos escritos sobre economia política, história, antropologia, assuntos actuais e muito mais, Marx nunca escreveu um ensaio detalhado sobre o seu próprio método. Isto porque o seu trabalho é principalmente uma crítica ao capitalismo e aos seus apologistas, na qual a metodologia desempenha um papel essencial mas de apoio, e está invariavelmente submersa no próprio argumento. Isto sugere que o método de Marx não pode ser resumido num conjunto de regras universais: devem ser desenvolvidas aplicações específicas da sua dialética materialista para resolver cada problema. O exemplo mais conhecido da aplicação do método de Marx é o seu exame crítico do capitalismo em O Capital. Neste trabalho, a abordagem de Marx tem cinco características gerais importantes. Estas serão acrescentadas e refinadas, muitas vezes implicitamente, ao longo do texto abaixo (como, de facto, foram no corpus dos escritos do próprio Marx).
 \par 
Primeiro, os fenómenos e processos sociais existem, e podem ser compreendidos, apenas no seu contexto histórico. Generalizações trans-históricas, supostamente válidas em todos os lugares e para todos os tempos, são normalmente inválidas, ou vazias, ou ambas. As sociedades humanas são imensamente flexíveis. Podem ser organizadas de formas profundamente diferentes, e apenas uma análise detalhada pode oferecer insights válidos sobre a sua estrutura interna, funcionamento, contradições, mudanças e limites. Em particular, Marx considera que as sociedades se distinguem pelo modo de produção sob o qual estão organizadas - o feudalismo em oposição ao capitalismo, por exemplo - com variedades de formas de cada modo surgindo em diferentes momentos e em diferentes lugares. Cada modo de produção é estruturado de acordo com suas relações de classe, para as quais existem categorias de análise apropriadas. Tal como um trabalhador assalariado não é um servo, muito menos um escravo que recebe um salário, um capitalista não é um barão feudal que recebe lucro em vez de tributo. As sociedades distinguem-se pelos modos de produção e pelas modalidades de extracção de excedentes sob os quais estão organizadas (e não pelas estruturas de distribuição), e os conceitos utilizados para as compreender devem ser igualmente específicos.
 \par 
Em segundo lugar, a teoria perde a sua validade se for levada para além dos seus limites históricos e sociais. Isto é uma consequência da necessidade de extrair conceitos das sociedades às quais foram concebidos. Por exemplo, Marx afirma que no capitalismo os trabalhadores são explorados porque produzem mais valor do que apropriam através do seu salário (ver Capítulo {\color{blue}3}); isso dá origem à mais-valia. Esta conclusão, tal como a noção correspondente de mais-valia, é válida apenas para sociedades capitalistas. Pode lançar alguma luz indirecta sobre a exploração noutras sociedades, mas os modos de exploração e as raízes da mudança social e económica nestas sociedades devem ser procurados novamente - a análise do capitalismo, mesmo que correcta, não fornece automaticamente os princípios pelos quais se pode compreender as sociedades não capitalistas.
 \par 
Terceiro, a análise de Marx é internamente estruturada pela relação entre teoria e história. Em contraste com o idealismo hegeliano, o método de Marx não é centrado em derivações conceituais. Para ele, o raciocínio puramente conceitual é limitado, porque é impossível avaliar como e por que as relações que evoluem na cabeça do analista devem corresponder àquelas no mundo real. De forma mais geral, o idealismo erra porque busca explicar a realidade principalmente por meio do avanço conceitual, embora a realidade exista histórica e materialmente fora da cabeça pensante. Brincando, Marx sugeriu que os Jovens Hegelianos seriam capazes de abolir as leis da gravidade se pudessem escapar de acreditar nelas! Em contraste, Marx reconhece que a realidade é moldada por estruturas sociais e tendências e contratendências (que podem ser derivadas dialeticamente, dado o cenário analítico apropriado), bem como por contingências imprevisíveis (que são historicamente específicas e não podem ser derivadas assim). Os resultados das interações dessas tendências podem ser explicados à medida que se desenrolam, bem como retrospectivamente, mas não podem ser determinados com antecedência. Consequentemente, embora a dialética materialista possa ajudar a entender tanto o passado quanto o presente, o futuro é impossível de prever (a análise de Marx da lei da tendência da taxa de lucro cair (LTRPF), e suas contratendências, é um exemplo revelador dessa abordagem; veja o Capítulo {\color{blue}9}). O reconhecimento de Marx de que a análise histórica pertence ao método de estudo (ou que história e lógica são inseparáveis) não é uma concessão ao empirismo; ele apenas reconhece que uma realidade mutável não pode ser reduzida a, e muito menos determinada por, um sistema de conceitos.
 \par 
Quarto, a dialética materialista identifica os principais conceitos, estruturas, relacionamentos e níveis de análise necessários para a explicação dos resultados concretos, ou mais complexos e específicos. Em O Capital, Marx emprega a dialética materialista para identificar as características essenciais do capitalismo e suas contradições, para explicar a estrutura e a dinâmica desse modo de produção e para localizar as fontes potenciais de mudança histórica; por exemplo, por meio da luta de classes em particular e sua representação por meio de conflitos econômicos, políticos e ideológicos às vezes mais amplamente engajados. Seu estudo sistematicamente traz à tona conceitos mais complexos e concretos que são usados ​​para reconstruir as realidades do capitalismo no pensamento. Esses conceitos ajudam a explicar o desenvolvimento histórico do capitalismo e indicam suas contradições e vulnerabilidades. Ao fazer isso, conceitos em níveis distintos de abstração sempre coexistem na análise de Marx. O progresso teórico inclui a introdução de novos conceitos, o refinamento e a reprodução dos conceitos existentes em níveis maiores de concretude e complexidade, e a introdução de evidências históricas para fornecer um relato mais rico e determinado da realidade.
 \par 
Finalmente, o método de Marx centra-se na mudança histórica. No Manifesto Comunista, no Prefácio à Contribuição para a Crítica da Economia Política e na introdução aos Grundrisse, Marx resume de forma famosa a sua descrição da relação entre estruturas de produção, relações sociais (especialmente de classe) e mudança histórica. As opiniões de Marx têm sido por vezes interpretadas de forma mecânica, como se o desenvolvimento supostamente unilinear da tecnologia orientasse sem problemas a mudança histórica - caso em que a mudança social é estreitamente determinada pelo desenvolvimento da produção. Esta interpretação de Marx é inválida. Existem relações complexas entre tecnologia, sociedade e história (e outros factores), mas de formas que são invariavelmente influenciadas pelo modo de organização social e, especificamente, pelas relações de classe e pelas lutas de classes. Por exemplo, sob o capitalismo, o desenvolvimento tecnológico é impulsionado principalmente pelo imperativo do lucro em todas as actividades comerciais. Sob o feudalismo, a produção de bens e serviços (militares) de luxo e, até certo ponto, de instrumentos agrícolas é primordial, o que, na relativa ausência da motivação do lucro e dada a relativa inflexibilidade do modo de organização social, limita o alcance e ritmo do avanço técnico. Em contraste, Marx argumenta que nas sociedades socialistas (comunistas) o desenvolvimento tecnológico procuraria eliminar tarefas repetitivas, fisicamente exigentes, inseguras e pouco saudáveis, reduzir o tempo de trabalho global, satisfazer necessidades básicas e desenvolver o potencial humano (ver Capítulo {\color{blue}15}).
 \par 
\section{Filosofia de Marx}
 \par 
Em 1845-6, quando escrevia A Ideologia Alemã com Engels e as Teses sobre Feuerbach, Marx já tinha começado a ser influenciado pelos socialistas franceses. Suas ideias não podem ser discutidas aqui em detalhes. Basta dizer que foram fomentadas pela herança radical da Revolução Francesa e pelo fracasso da sociedade burguesa emergente em concretizar as exigências da “liberdade, elite, fraternidade”. Os socialistas franceses também estavam profundamente envolvidos na política de classe e muitos acreditavam na necessidade e na possibilidade da tomada revolucionária do poder pelos trabalhadores.
 \par 
A síntese de Marx entre a filosofia alemã e o socialismo francês teria permanecido incompleta sem a sua crítica da economia política britânica, que estudou mais tarde, especialmente durante o seu longo exílio em Londres, de 1849 até à sua morte em 1883. Dadas as suas concepções de filosofia e história, explicadas acima , era natural que Marx voltasse a sua atenção para a economia, a fim de compreender a sociedade capitalista contemporânea e identificar os seus pontos fortes e limitações, bem como o seu potencial de transformação no comunismo. Para isso, mergulhou na economia política britânica, desenvolvendo em particular a teoria do valor-trabalho a partir dos escritos de Adam Smith e, especialmente, de David Ricardo. Para Marx, é insuficiente basear a fonte do valor no tempo de trabalho da produção, como supõe Ricardo. Pois a visão de Ricardo dá como certa a existência de trocas, preços e mercadorias. O facto de as mercadorias serem mais valiosas porque incorporam mais trabalho levanta a questão de saber por que razão existem mercadorias, e muito menos se é relevante proceder como se, em geral, as mercadorias fossem trocadas em proporção ao tempo de trabalho necessário para a sua produção. Isto antecipa o próximo capítulo, mas ilustra uma característica fundamental do método de Marx e uma crítica comum feita por Marx a outros escritores. Marx considera outros economistas não apenas errados no conteúdo, mas também inadequados na intenção. O que os economistas tendem a assumir como características intemporais dos seres humanos e das sociedades, Marx estava determinado a erradicar e compreender no seu contexto histórico. Marx dá como certa a necessidade de a sociedade como um todo trabalhar para produzir e consumir. Contudo, a forma como a produção é organizada e a produção é distribuída tem de ser revelada. Muito brevemente, Marx argumenta que quando trabalham (ou não) - isto é, produzem as condições materiais para a sua reprodução contínua - as pessoas estabelecem relações sociais específicas umas com as outras: como escravos ou senhores, servos ou senhores, assalariados ou capitalistas, e breve. Os padrões de vida são determinados por essas condições sociais de produção e pelos lugares a serem preenchidos ao seu redor. Estas relações existem independentemente da escolha individual, embora tenham sido estabelecidas no decurso do desenvolvimento histórico da sociedade (por exemplo, ninguém pode “escolher” ocupar a posição social de proprietário de escravos nas sociedades capitalistas de hoje, e mesmo a “escolha” entre ser capitalista ou trabalhador assalariado não está disponível gratuitamente para todos e certamente não numa base de igualdade).
 \par 
Em todas as sociedades, excepto nas mais simples, as relações sociais de produção específicas de um determinado modo de produção (feudalismo, capitalismo, etc.) são mais bem estudadas como relações de classe. Estas relações são a base sobre a qual a sociedade se constrói e se reproduz ao longo do tempo. Tal como a liberdade de possuir, comprar e vender são características jurídicas fundamentais da sociedade capitalista, também a lealdade e as obrigações divinas ou tributárias são os fundamentos jurídicos do feudalismo. Além disso, são também estabelecidas formas políticas, jurídicas, intelectuais e de distribuição que se apoiam mutuamente e tendem a cegar e desencorajar todas as visões da sociedade, excepto as mais convencionais, seja por força do hábito, da moralidade, da educação, da lei ou de outra forma. O servo sente-se obrigado pela lealdade ao senhor e ao rei, muitas vezes através da igreja, e qualquer vacilação pode ser punida severamente. O assalariado tem liberdade e compulsão para vender força de trabalho. Pode haver luta por salários mais elevados, mas isso não põe em causa o sistema salarial ou o quadro jurídico e institucional que o apoia, desde a negociação colectiva até aos sistemas de segurança social e de crédito, e assim por diante. Em contraste, investigar a natureza do capitalismo é desaprovado pelas autoridades, pelos meios de comunicação social, pela lei e por outras vozes dominantes na sociedade. Embora a dissidência individual seja frequentemente tolerada, as grandes organizações anticapitalistas e os movimentos de massas são reprimidos ou pressionados à conformidade, sendo o protesto, por exemplo, canalizado para formas sistemicamente aceitáveis.
 \par 
Neste contexto, Marx castiga os economistas políticos clássicos e os utilitaristas por assumirem que certas características do comportamento humano, como o interesse próprio ou a ganância, são características permanentes da “natureza humana”, quando na realidade são características, motivações ou comportamentos emergentes. nos indivíduos através de sua vida em sociedades específicas. Tais teóricos também tomam como certas as características da sociedade capitalista que Marx considerou necessário explicar: o monopólio dos meios de produção (matérias-primas, maquinaria, edifícios fabris, etc.) por uma pequena minoria, o emprego assalariado da maioria , a distribuição dos produtos por troca monetária e a remuneração envolvendo as categorias econômicas de preços, lucros, juros, aluguéis, salários, taxas e transferências. A teoria do valor de Marx é uma contribuição penetrante para a ciência social na medida em que se preocupa com as relações que as pessoas estabelecem entre si, e não com as relações técnicas entre as coisas ou com a arte de economizar. Marx não está interessado em construir uma teoria dos preços, um conjunto de “critérios de eficiência” desencarnados, válidos em todo o lado e em todos os momentos, ou uma série de propostas de bem-estar; ele nunca pretendeu ser um “economista” ou mesmo um economista político clássico (britânico). Marx foi um cientista social crítico, cujo trabalho ultrapassa e rejeita as barreiras que separam as disciplinas acadêmicas. As questões cruciais para Marx dizem respeito à estrutura interna e às fontes de estabilidade e crises no capitalismo, e como a vontade de mudar o modo de produção pode evoluir para uma actividade transformadora (revolucionária) bem sucedida. Estas questões permanecem válidas no século XXI.
 \par 
\section{Filosofia de Marx}
 \par 
Várias biografias de Karl Marx estão disponíveis; ver, por exemplo, Mary Gabriel (2011), David McLellan (1974), Franz Mehring (2003), Francis Wheen (2000). A trajetória intelectual de Marx é revisada por Allen Oakley (1983, 1984, 1985) e Roman Rosdolsky (1977). A história da economia marxista é exaustivamente examinada por Michael Howard e John King (1989, 1991); ver também Ben Fine e Alfredo Saad-Filho (2012). Os conceitos-chave da literatura marxista são explicados com autoridade em Tom Bottomore (1991).
 \par 
Embora Marx raramente discuta o seu próprio método, há exceções significativas na introdução de Marx (1981a), nos prefácios e posfácios de Marx (1976) e no prefácio de Marx (1987). A literatura e a controvérsia subsequentes mais do que compensaram a aparente negligência do próprio Marx. Quase todos os aspectos de seu método foram sujeitos a escrutínio e interpretações divergentes tanto de defensores quanto de críticos. Nossa apresentação aqui é embaraçosamente simples e superficial em amplitude e profundidade. Baseia-se em Ben Fine (1980, cap. {\color{blue} 1 } {\par} , 1982, cap.{\color{blue}1}) e Alfredo Saad-Filho (2002, cap.{\color{blue}1}), que devem ser consultados para uma interpretação mais abrangente do método de Marx. Outros examinaram detalhadamente o papel da classe, dos modos de produção, da dialética, da história, da influência de outros pensadores, e assim por diante, na análise de Marx. Chris Arthur escreveu extensivamente sobre o método de Marx (por exemplo, Arthur 2002); ver também os ensaios de Andrew Brown, Steve Fleetwood e Michael Roberts (2002), Alex Callinicos (2014), Duncan Foley (1986, cap.{\color{blue}1}), Fred Moseley
 \par 
(1993) e Roman Rosdolsky (1977, pt.{\color{blue}1}). Interpretações mecanicistas de Marx, sugerindo uma determinação causal rígida entre, por exemplo, relações de classe e factores económicos e outros, são examinadas e criticadas minuciosamente por Ellen Meiksins Wood (1984, 1995), Michael Lebowitz (2009, pt.{\color{blue}2}) e Paul Blackledge ( 2006). As raízes históricas da economia política marxista são revistas por Dimitris Milonakis e Ben Fine (2009), com os desenvolvimentos subsequentes na economia dominante examinados em Ben Fine e Dimitris Milonakis (2009).