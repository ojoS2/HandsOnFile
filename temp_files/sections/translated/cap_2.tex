\chapter{~ VOCÊ PODE ESTAR SOFRENDO}\label{~ VOCÊ PODE ESTAR SOFRENDO}
 \par 
TTTTTT o argumento deste livro pode ser resumido sucintamente: O capitalismo desregulado é ruim para as mulheres, e se adotarmos algumas ideias do socialismo, as mulheres terão vidas melhores. Se feito corretamente, o socialismo leva à independência econômica, melhores condições de trabalho, melhor equilíbrio entre trabalho e família e, sim, até mesmo melhor sexo. Encontrar um caminho para um futuro melhor requer aprender com os erros do passado, incluindo uma avaliação cuidadosa da história do socialismo de estado do século XX na Europa Oriental.
 \par 

 \par 
É isso. Se você gosta da ideia de tais resultados, venha explorar como podemos mudar as coisas. Se tiver dúvidas porque não compreende porque é que o capitalismo, como sistema econômico, é exclusivamente mau para as mulheres, e se duvida que alguma vez possa haver algo de bom no socialismo, este breve tratado fornecerá alguma iluminação. Se você não dá a mínima para a vida das mulheres porque é um troll xenófobo da internet de direita, economize seu dinheiro e volte para o porão dos seus pais agora mesmo; este não é o livro para você.
 \par 
2
 \par 
VOCÊ PODE ESTAR SOFRENDO DE CAPITALISMO
 \par 
Claro, alguns podem argumentar que o capitalismo desregulado é uma droga para quase todo mundo, mas eu quero focar em como o capitalismo prejudica desproporcionalmente as mulheres. Os mercados de trabalho competitivos discriminam aquelas cuja biologia reprodutiva as torna as principais responsáveis ​​pela procriação. Hoje, isso significa humanos que ganham chapéus cor-de-rosa no hospital e a letra "F" ao lado do nome em sua certidão de nascimento (como se já tivéssemos falhado por não vir ao mundo como um menino). Os mercados de trabalho competitivos também desvalorizam aqueles que se espera que sejam os principais cuidadores das crianças. Embora as atitudes sociais tenham evoluído a esse respeito, nossa idealização da maternidade significa que a maioria de nós ainda acredita que o bebê precisa muito mais da mamãe do que do papai — pelo menos até que a criança tenha idade suficiente para praticar esportes.
 \par 
Outros argumentarão que o capitalismo desregulado não é ruim para todas as mulheres. Sim, para aquelas mulheres sortudas o suficiente para sentar no topo da distribuição de renda, o sistema funciona muito bem. Embora as mulheres no nível executivo ainda enfrentem disparidades salariais de gênero e permaneçam sub-representadas em posições de liderança, no geral as coisas não estão tão ruins para as Sheryl Sandberg do mundo. Claro, o assédio sexual ainda atrapalha o progresso até mesmo para aquelas no topo, e muitas mulheres acreditam que se você quiser correr com os grandes, você pode ter que engolir e ignorar as apalpadelas e avanços indesejados. E a raça também desempenha um papel importante; mulheres brancas se saem muito melhor no agregado do que mulheres de cor. Mas quando olhamos para a sociedade todo, em média, as mulheres estão comparativamente em pior situação em países onde os mercados são menos sobrecarregados por regulamentação, tributação e empresas públicas do que em nações onde as receitas estatais apoiam maiores níveis de redistribuição e maiores redes de segurança social.
 \par 
KRISTEN R. GHODSEE
 \par 
Escolha sua fonte de dados e você encontrará a mesma história. Desemprego e pobreza atormentam mulheres com filhos. Os empregadores discriminam mulheres sem filhos porque elas podem tê-los no futuro. Nos Estados Unidos, em 2013, mulheres com mais de {\color{blue}65} anos sofreram com a pobreza em taxas muito maiores do que os homens e dominaram aquelas na categoria de "pobreza extrema". Globalmente, as mulheres enfrentam taxas mais altas de privação econômica. As mulheres são frequentemente as últimas a serem contratadas e as primeiras a serem demitidas em crises cíclicas e, quando encontram emprego, os chefes pagam menos a elas do que aos homens. Quando os estados precisam cortar os gastos do governo em educação, assistência médica ou pensões de velhice, mães, filhas, irmãs e esposas devem compensar, desviando suas energias para cuidar dos jovens, dos doentes e dos idosos. O capitalismo prospera com o trabalho não remunerado das mulheres em casa porque o trabalho de cuidado das mulheres suporta impostos mais baixos. Impostos mais baixos significam lucros maiores para aqueles que já estão no topo da escala de renda — principalmente os homens.'
 \par 
Mas o capitalismo nem sempre foi tão selvagem. Durante grande parte do século XX, o socialismo de estado apresentou um desafio existencial aos piores excessos do livre mercado. A ameaça representada pelas ideologias marxistas forçou os governos ocidentais a expandir as redes de segurança social para proteger os trabalhadores dos imprevisíveis, mas inevitáveis, boons e crises da economia capitalista. Após a queda do Muro de Berlim, muitos celebraram o triunfo do Ocidente, consignando as ideias socialistas à lata de lixo da história. Mas, apesar de todas as suas falhas, o socialismo de estado forneceu um importante contraste para o capitalismo. Foi em resposta a um discurso global de direitos sociais e econômicos — um discurso que apelou não apenas às populações progressistas da África, Ásia e América Latina, mas também a muitos homens e mulheres na Europa Ocidental e América do Norte — que
 \par 
3
 \par 
4
 \par 
VOCÊ PODE ESTAR SOFRENDO COM O CAPITALISMO os políticos concordaram em melhorar as condições de trabalho para trabalhadores assalariados, bem como criar programas sociais para crianças, pobres, idosos, doentes e deficientes, mitigando a exploração e o crescimento da desigualdade de renda. Embora houvesse antecedentes importantes na década de 1980, uma vez que o socialismo de estado entrou em colapso, o capitalismo se livrou das restrições da regulamentação do mercado e da redistribuição de renda. Sem a ameaça iminente de uma superpotência rival, os últimos trinta anos de neoliberalismo global testemunharam um rápido encolhimento dos programas sociais que protegem os cidadãos da instabilidade cíclica e das crises financeiras e reduzem a vasta desigualdade de resultados econômicos entre aqueles no topo e na base da distribuição de renda.
 \par 
Durante grande parte do século XX, os países capitalistas ocidentais também se esforçaram para superar os países do Leste Europeu em termos de direitos das mulheres, alimentando mudanças sociais progressivas. Por exemplo, os socialistas de estado na URSS e na Europa Oriental foram tão bem-sucedidos em dar às mulheres oportunidades econômicas fora de casa que, inicialmente, por duas décadas após o fim da Primeira Guerra Mundial, o trabalho assalariado das mulheres foi confundido com os males do comunismo. O modo de vida americano significava homens chefes de família e mulheres donas de casa. Mas, lentamente, a defesa socialista da emancipação das mulheres começou a minar o ideal de Leave It to Beaver. O lançamento soviético do Sputnik em 1957 estimulou os líderes americanos a repensar os custos de manter os papéis tradicionais de gênero. Eles temiam que os socialistas de estado tivessem uma vantagem no desenvolvimento tecnológico porque tinham o dobro de capacidade intelectual; os russos educaram as mulheres e canalizaram os melhores e mais brilhantes para a pesquisa científica.”
 \par 
KRISTEN R. GHODSEE
 \par 
Temendo a superioridade do Bloco Oriental na corrida espacial, o governo americano aprovou o National Defense Education Act (NDEA) em 1958. Apesar do desejo cultural contínuo de que as mulheres ficassem em casa como esposas dependentes, o NDEA criou novas oportunidades para meninas talentosas estudarem ciências e matemática. Então, em 1961, o presidente John F. Kennedy assinou a Ordem Executiva 10980 para estabelecer a primeira Comissão Presidencial sobre o Posição das Mulheres, citando preocupações com a segurança nacional. Esta comissão, presidida por Eleanor Roosevelt, lançou as bases para o futuro movimento feminino dos EUA. Os americanos receberam um choque adicional em 1963, quando Valentina Tereshkova se tornou a primeira cosmonauta mulher, passando mais tempo orbitando a Terra do que todos os astronautas homens nos Estados Unidos juntos. Mais tarde, o domínio soviético e do Leste Europeu nas Olimpíadas estimulou a aprovação do Título IX, para que os Estados Unidos pudessem identificar e treinar mais atletas femininas para arrebatar medalhas de ouro do inimigo ideológico.
 \par 
Em resposta à proeza do socialismo estatal nas ciências, o governo americano patrocinou um importante estudo intitulado “Mulheres na Economia Soviética”. O chefe do estudo visitou a URSS em 1955, 1962 e 1965 para examinar as políticas soviéticas para integrar as mulheres na força de trabalho formal como um exemplo para os legisladores americanos. “A preocupação nos últimos anos com o desperdício do talento e do potencial laboral das mulheres levou à nomeação da Comissão Presidencial sobre o Estatuto da Mulher, que publicou uma série de relatórios sobre vários problemas que afetam as mulheres e a sua participação na vida econômica, política e social. Vida”, começava o relatório de 1966. “Para qualquer formulação de política dirigida ao melhor uso do poder das mulheres, é importante conhecer a experiência
 \par 
5
 \par 
6
 \par 
VOCÊ PODE ESTAR SOFRENDO COM O CAPITALISMO de outras nações na utilização das capacidades das mulheres. Por esta razão, assim como por outras, a experiência soviética é de particular interesse neste momento.” O precedente estabelecido pelos países socialistas estatais na Europa Oriental agiu como um exemplo influente para os políticos americanos no mesmo momento histórico em que Betty Friedan publicou The Feminine Mystique e revelou o quão insatisfeitas as mulheres brancas de classe média se sentiam com suas vidas domésticas circunscritas. Mas no clima político atual, pode ser difícil entender como uma rivalidade entre superpotências poderia ter despertado interesse no posição das mulheres.*
 \par 
Hoje, as ideias socialistas estão desfrutando de um renascimento, pois jovens em países como Estados Unidos e França encontram inspiração em políticos e movimentos como Bernie Sanders, Alexandria Ocasio-Cortez, Jean-Luc Mélenchon e o Mouvement des. gilês jalnes (coletes amarelos). Os cidadãos desejam um caminho político alternativo que leve a um futuro mais igualitário e sustentável. Para seguir em frente, devemos ser capazes de discutir o passado sem tentativas ideologicamente motivadas de encobrir ou retroceder nossa própria história, ou as realizações do socialismo de estado. Por um lado, qualquer relato matizado do socialismo de estado do século XX encontrará inevitavelmente a balbuciação e a fanfarronice daqueles que insistem que foi pura maldade, fim da história. Como o escritor tcheco Milan Kundera escreveu em seu famoso romance A Insustentável Leveza do Ser: "As pessoas que lutam contra o que chamamos de regimes totalitários não podem funcionar com perguntas e dúvidas. Eles também precisam de certezas e de verdades simples para fazer as multidões compreenderem, para provocar lágrimas coletivas.”° Sobre o
 \par 
KRISTEN R. GHODSEE, Por outro lado, alguns jovens de hoje brincam sobre “comunismo total agora”. Os millennials esquerdistas podem não saber (ou preferir ignorar) os verdadeiros horrores infligidos aos cidadãos em estados de partido único. Histórias horríveis sobre a polícia secreta, restrições de viagem, escassez de consumidores e campos de trabalho não são apenas propaganda anticomunista. Nosso futuro coletivo depende de um exame equilibrado do passado para podermos descartar o ruim e seguir em frente com o bom, especialmente no que diz respeito aos direitos das mulheres.
 \par 
Desde meados do século XIX, os teóricos sociais europeus argumentaram que o sexo feminino é excepcionalmente desfavorecido em um sistema econômico que preza os lucros e a propriedade privada sobre as pessoas. Ao longo da década de 1970, as feministas socialistas nos Estados Unidos também afirmaram que destruir o patriarcado não era suficiente. A exploração e a desigualdade persistiriam enquanto as elites financeiras construíssem suas fortunas nas costas de mulheres dóceis que reproduziam a força de trabalho de graça. Mas essas críticas iniciais eram baseadas em teorias abstratas com pouca evidência empírica para substanciá-las. Lentamente, ao longo da primeira metade do século XX, novos governos socialistas democráticos e socialistas estatais na Europa começaram a testar essas teorias, na prática. Na Alemanha Oriental, Escandinávia, União Soviética e Europa Oriental, os líderes políticos apoiaram a ideia da emancipação das mulheres por meio de sua incorporação total à força de trabalho. Essas ideias logo se espalharam para a China, Cuba e uma ampla variedade de países recém-independentes em todo o mundo. Experimentos com a independência econômica feminina alimentaram o movimento das mulheres do século XX e resultaram em uma revolução nos caminhos de vida abertos às mulheres, antes confinados à esfera doméstica. E em nenhum lugar
 \par 
7
 \par 
8
 \par 
VOCÊ PODERIA ESTAR SOFRENDO COM O CAPITALISMO se houvesse mais mulheres na força de trabalho do que no socialismo de estado.
 \par 
A emancipação das mulheres infundiu a ideologia de quase todos os regimes socialistas estatais, com a revolucionária franco-russa Inessa Armand declarando: "Se a libertação das mulheres é impensável sem o comunismo, então o comunismo é impensável sem a libertação das mulheres". Embora existissem diferenças importantes entre os países e nenhum alcançasse a igualdade total, na prática, essas nações gastaram vastos recursos para investir na educação e treinamento das mulheres e promovê-las em profissões anteriormente dominadas por homens. Compreendendo as demandas da biologia reprodutiva, elas também tentaram socializar o trabalho doméstico e o cuidado infantil construindo uma rede de creches públicas, jardins de infância, lavanderias e refeitórios. Licenças-maternidade estendidas e protegidas pelo emprego e benefícios para crianças permitiram que as mulheres encontrassem pelo menos um mínimo de equilíbrio entre trabalho e família. Além disso, o socialismo estatal do século XX melhorou as condições materiais da vida de milhões de mulheres; a mortalidade materna e infantil diminuiu, a expectativa de vida aumentou e o analfabetismo praticamente desapareceu. Para dar apenas um exemplo, a maioria das mulheres albanesas era analfabeta antes da imposição do socialismo em 1945. Apenas dez anos depois, toda a população com menos de quarenta anos sabia ler e escrever, e na década de 1980 metade dos estudantes universitários da Albânia eram mulheres.
 \par 
Enquanto diferentes países buscavam diferentes políticas, em geral, os governos socialistas estaduais reduziram a dependência econômica das mulheres em relação aos homens ao tornar homens e mulheres receptores iguais de serviços do estado socialista. Essas políticas ajudaram a desvincular o amor da intimidade
 \par 
KRISTEN R. GHODSEE de considerações econômicas. Quando as mulheres desfrutam de suas próprias fontes de renda, e o estado garante a seguridade social na velhice, doença e invalidez, as mulheres não têm razão econômica para permanecer em relacionamentos abusivos, insatisfatórios ou de outra forma prejudicial. Em países como Polônia, Hungria, Tchecoslováquia, Bulgária, Iugoslávia e Alemanha Oriental, a independência econômica das mulheres se traduziu em uma cultura na qual os relacionamentos pessoais poderiam ser liberados das influências do mercado. As mulheres não precisavam se casar por dinheiro.®
 \par 
É claro que, tal como podemos aprender com as experiências da Europa Oriental, não devemos ignorar as desvantagens. Os direitos das mulheres no Bloco de Leste não incluíram a preocupação com os casais do mesmo sexo e com a inconformidade de gênero. O aborto serviu como principal forma de controle de natalidade nos países onde estava disponível sob demanda. A maioria dos estados da Europa de Leste incentivou fortemente as mulheres a tornarem-se mães, com a Romênia, a Albânia e a URSS sob Estaline a forçarem as mulheres a terem filhos que não queriam. Os governos socialistas estaduais suprimiram discussões sobre assédio sexual, violência doméstica e estupro. E embora tentassem envolver os homens nas tarefas domésticas e no cuidado dos filhos, os homens resistiram na maioria aos desafios aos papéis tradicionais de gênero. Muitas mulheres sofreram com o duplo fardo do emprego formal obrigatório e do trabalho doméstico, como tão bem retratado na brilhante novela de Natalya Baranskaya, Uma semana como qualquer outra. Finalmente, em nenhum país os direitos das mulheres foram promovidos como um projeto para apoiar o individualismo ou a autorrealização das mulheres. Em vez disso, o Estado apoiou as mulheres como trabalhadoras e mães para que pudessem participar mais plenamente na vida colectiva da nação.’
 \par 
9
 \par 
10
 \par 
VOCÊ PODE ESTAR SOFRENDO DE CAPITALISMO
 \par 
Após a queda do Muro de Berlim em 1989, novos governos democráticos privatizaram rapidamente os ativos estatais e desmantelaram as redes de segurança social. Os homens sob essas economias capitalistas emergentes recuperaram seus papéis "naturais" como patriarcas da família, e esperava-se que as mulheres voltassem para casa como mães e esposas apoiadas por seus maridos. Em toda a Europa Oriental, os nacionalistas pós-1989 argumentaram que a competição capitalista aliviaria as mulheres do notório fardo duplo e restauraria a harmonia familiar e social, permitindo que os homens reafirmassem sua autoridade masculina como ganha-pão. No entanto, isso significava que os homens poderiam mais uma vez exercer poder financeiro sobre as mulheres. Por exemplo, a renomada historiadora da sexualidade Dagmar Herzog compartilhou uma conversa com vários homens da Alemanha Oriental em seus quarenta e tantos anos em 2006. Eles disseram a ela que "era realmente irritante que as mulheres da Alemanha Oriental tivessem tanta autoconfiança sexual e independência econômica. O dinheiro era inútil, eles reclamaram. Os poucos marcos orientais extras que um médico poderia ganhar em contraste com, digamos, alguém que trabalhasse no teatro, não faziam absolutamente nada de bom, eles explicaram, em atrair ou reter mulheres da maneira que o salário de um médico poderia e fazia no Ocidente. "Você tinha que ser interessante. Que pressão. E como um revelou: 'Tenho muito mais poder agora como homem na Alemanha unificada do que jamais tive nos dias comunistas.' Além disso, após a publicação do meu artigo de opinião no New York Times, "Por que as mulheres tinham sexo melhor sob o socialismo", fiz uma entrevista com Doug Henwood em seu programa de rádio, Behind lhe News. Uma ouvinte, uma mulher de quarenta e seis anos nascida na União Soviética, enviou um e-mail ao programa para dizer que eu tinha "acertado precisamente" na minha discussão sobre relações românticas no "antigo
 \par 

 \par 
KRISTEN R. GHODSEE país, como ela o chamou, "mas também a maneira como os homens dominam as mulheres com dinheiro aqui [nos Estados Unidos]".
 \par 
O colapso do socialismo de estado em 1989 criou um laboratório perfeito para investigar os efeitos do capitalismo na vida das mulheres. O mundo pôde assistir enquanto mercados livres eram conjurados dos escombros da economia planejada, e esses novos mercados afetavam de várias maneiras diferentes categorias de trabalhadores. Após décadas de escassez, os europeus orientais trocaram avidamente o autoritarismo pela promessa de democracia e prosperidade econômica, abrindo seus países ao capital ocidental e ao comércio internacional. Mas houve custos imprevistos.
 \par 
A rejeição do estado de partido único e a adoção de liberdades políticas vieram com o neoliberalismo econômico. Novos governos democráticos privatizaram empresas públicas para abrir espaço para novos mercados de trabalho competitivos, onde a produtividade determinaria os salários. Acabaram-se as longas filas para comprar papel higiênico e os mercados negros para jeans. Logo viria um glorioso paraíso do consumidor, livre de escassez, fome, polícia secreta e campos de trabalho. Mas depois de quase três décadas, muitos europeus orientais ainda esperam por um futuro capitalista brilhante. Outros abandonaram toda a esperança.”
 \par 
A evidência é incontestável: como tantas outras mulheres ao redor do globo, as mulheres na Europa Oriental são mais uma vez mercadorias a serem compradas e vendidas — seu preço determinado pelas flutuações inconstantes da oferta e da demanda. Escrevendo logo após o colapso do socialismo de estado, a jornalista croata Slavenka Drakulić explicou: “Vivemos cercadas por mercados recém-abertos
 \par 
11
 \par 
12
 \par 
VOCÊ PODE ESTAR SOFRENDO COM O CAPITALISMO lojas pornô, revistas pornô, megashows, stripteases, desemprego e pobreza galopante. Na imprensa, eles chamam Budapeste de "a cidade do amor, a Bangkok da Europa Oriental". Mulheres romenas estão se prostituindo por um único dólar na fronteira entre Romênia e Iugoslávia. No meio de tudo isso, nossos governos nacionalistas anti-escolha estão ameaçando nosso direito ao aborto e nos dizendo para multiplicar, para dar à luz mais poloneses, húngaros, tchecos, croatas, eslovacos". Hoje, noivas russas por correspondência, trabalhadoras sexuais ucranianas, babás moldavas e empregadas domésticas polonesas inundam a Europa Ocidental. Intermediários inescrupulosos colhem cabelos loiros de adolescentes bielorrussas pobres para fabricantes de perucas de Nova York. Em São Petersburgo, as mulheres frequentam academias para aspirantes a interesseiras. Praga é um epicentro da indústria pornográfica europeia. Traficantes de pessoas rondam as ruas de Sófia, Bucareste e Chișinău em busca de meninas infelizes que sonham com uma vida mais próspera no Ocidente."
 \par 
Cidadãos mais velhos da Europa Oriental lembram com carinho dos pequenos confortos e previsibilidade de sua vida antes de 1989: educação e assistência médica gratuitas, nenhum medo de desemprego e de não ter dinheiro para atender às necessidades básicas. Uma piada, contada em muitas línguas do Leste Europeu, ilustra esse sentimento:
 \par 
TTTTTT saiu da cama, os olhos cheios de terror. Seu marido assustado
 \par 
Observa-a correr para o banheiro e abrir o armário de remédios. Ela então corre para a cozinha e inspeciona
 \par 
O interior da geladeira. Finalmente, ela abre uma janela e olha para a rua abaixo do apartamento deles.
 \par 
Ment. Ela respira fundo e volta para a cama.
 \par 
\[DO CAPITALISMO\]
 \par 
“O que há de errado com você?”, diz o marido. “O que aconteceu?”
 \par 
SSSSSS que tínhamos o medicamento de que precisávamos, que nosso refinador-
 \par 
Ator estava cheio de comida e as ruas lá fora estavam
 \par 
\section{No meio da noite uma mulher grita e pula}
 \par 
SSSSSS pensou que os comunistas estavam de volta ao poder.”
 \par 
Pesquisas de opinião em toda a região continuam a mostrar que muitos cidadãos acreditam que suas vidas eram melhores antes de 1989, sob o autoritarismo. Embora essas pesquisas possam dizer mais sobre a decepção com o presente do que sobre a desejabilidade do passado, elas complicam a narrativa totalitária. Por exemplo, uma pesquisa aleatória de 2013 com {\color{blue}1}.{\color{blue}055} romenos adultos descobriu que apenas um terço relatou que suas vidas eram piores antes de 1989: 44% disseram que suas vidas eram melhores e 16% disseram que não houve mudança. Esses resultados também foram divididos por gênero de maneiras interessantes: enquanto 47% das mulheres achavam que o socialismo de estado era melhor para seu país, apenas 42% dos homens disseram o mesmo. Da mesma forma, enquanto 36% dos homens alegaram que a vida era pior antes de 1989, apenas 31% das mulheres disseram que a vida sob o ditador Nicolae Ceaușescu era pior do que o presente. E isto vem da Romênia, um dos regimes mais corruptos e opressivos do antigo Bloco Oriental, onde Ceausescu dourou a alavanca de descarga de seu banheiro particular. Resultados semelhantes surgiram de pesquisas na Polônia em 2011 e de uma pesquisa de opinião realizada em outros oito
 \par 
13
 \par 
14
 \par 
TTTTTT antigas nações socialistas em 2009. Para os cidadãos que tiveram a oportunidade de viver sob dois sistemas econômicos diferentes, muitos agora sentem que o capitalismo é pior do que o socialismo de estado que eles antes estavam tão ansiosos para deixar de lado.’*
 \par 
De volta aos Estados Unidos, o colapso do socialismo estatal do Leste Europeu inaugurou uma era de triunfalismo capitalista ocidental. As ideias da Grande Sociedade sobre como regular nossa economia e redistribuir riqueza para maximizar o bem-estar de todos os cidadãos, incluindo mulheres, caíram em desuso. A ascensão do que foi chamado de Consenso de Washington (nascido da Reaganomics) significou mercantilização, privatização e destruição de redes de segurança social em nome da eficiência. Ao longo das décadas de 1990 e 2000, os cidadãos testemunharam a crescente desregulamentação dos setor financeiro, de transporte e de serviços públicos e a crescente mercantilização da vida cotidiana. Nós confundimos liberdade com mercados livres. Após a crise financeira global em 2008, as elites econômicas miraram orçamentos estaduais já magros, cortando mais profundamente em programas sociais enquanto usavam o dinheiro dos contribuintes para socorrer os banqueiros que criaram grande parte da bagunça em primeiro lugar. O movimento Occupy Wall Street chamou a atenção para a desigualdade estrutural, mas políticos de ambos os lados responderam à crescente raiva pública com a mesma velha frase: não há alternativa ao capitalismo.
 \par 
Isto é mentira. Os guerreiros frios conservadores vão combater qualquer tentativa de complicar a história do socialismo de estado do século XX gritando sobre as fomes e expurgos de Stalin. Na imaginação deles, toda a experiência do socialismo de estado consistia em pessoas em filas de pão e delatando
 \par 
KRISTEN R. GHODSEE sobre seus vizinhos para a polícia secreta. Por setenta anos na União Soviética e quarenta e cinco anos na Europa Oriental, líderes totalitários aparentemente transportavam todos de um lado para o outro entre campos de trabalho e prisões, um pesadelo orwelliano sem Deus onde as pessoas usavam ternos cinza e unissex de Mao e ostentavam cabeças raspadas. Se bebês nasciam, não era porque as pessoas decidiam começar famílias, mas porque o Partido inseminava em massa a população para atender às cotas de produção humana predeterminadas. Os anticomunistas se recusam a reconhecer as diferenças importantes entre a grande variedade de sociedades que abraçaram o socialismo ou a creditá-las por suas várias conquistas em ciência, educação, resultados de saúde, cultura e esporte. Nos estereótipos promovidos pelos líderes ocidentais, o socialismo de estado era um sistema econômico ineficiente fadado ao colapso inevitável e uma ameaça vermelha aterrorizante que exigia bilhões de dólares do dinheiro do contribuinte para conter. É estranho considerar como poderia ter sido ambos.
 \par 
Na Europa Oriental hoje, vários institutos de pesquisa financiados pelo ocidente investigam os crimes do comunismo. Em países como Hungria, Bulgária e Romênia (todos aliados alemães na Segunda Guerra Mundial), os descendentes de colaboradores nazistas estão ansiosos para se pintarem como "vítimas do comunismo". Políticos locais e elites econômicas que se beneficiaram da transição para mercados livres (particularmente aqueles que tiveram a propriedade nacionalizada de seus avós instituída a eles depois de 1989) conspiram para criar uma narrativa totalitária oficial sobre o passado. Por exemplo, depois de uma palestra que dei em Viena em 2011, uma jovem búlgara na plateia enviou um e-mail me agradecendo por minha coragem em discutir alguns dos legados positivos
 \par 
15 colheres de chá
 \par 
VOCÊ PODE ESTAR SOFRENDO DE CAPITALISMO
 \par 
\[KRISTEN R. GHODSEE\]
 \par 
TTTTTT de Todor Zhivkov, líder da Bulgária de 1954 a 1989. “Ninguém [na Bulgária] pode falar sobre a nostalgia e as dores da transição sem ser enquadrado como comunista e como alguém que nega os crimes do regime de Zhivkov. Então, as questões importantes com as quais você lida não estão presentes no discurso ou na mídia.” Na vizinha Romênia, o estudioso literário Costi Rogozanu criticou a prática do Leste Europeu de usar histórias de terror sobre o passado socialista de estado para justificar a implementação contínua de políticas econômicas neoliberais no presente: “Você quer um aumento de salário? Você é comunista. Você quer serviços públicos? Você quer taxar os ricos e aliviar o fardo dos pequenos produtores e assalariados? Você é comunista e matou meus avós. Você quer transporte público em vez de rodovias? Você é mega comunista e uma era: hister.”
 \par 
Embora seja importante não romantizar o passado do socialismo de estado, as realidades horríveis não devem nos tornar completamente alheios aos ideais dos primeiros socialistas, às várias tentativas de reformar o sistema de dentro (como a primavera de Praga, a glasnost ou a perestroika) ou ao papel importante que os ideais socialistas desempenharam na inspiração de movimentos de independência nacional no Sul Global. Reconhecer o mal não nega o bem. Assim como há aqueles que gostariam de branquear a história americana minimizando, só para começar, Jim Crow, racismo institucional, violência armada ou a taxa de encarceramento sem precedentes, há aqueles que iriam retroceder a história do socialismo de estado, insistindo que tudo era mau.
 \par 
Hoje, temos mais de duzentos anos de experimentação com várias formas de socialismo, mas a palavra
 \par 
TTTTTT areal i us an. ed. o ele Um Açaí
 \par 
KRISTEN R. GHODSEE
 \par 
“Socialismo” ainda carrega conotações negativas. Uivos sobre o Gulag de Stalin e as fomes ucranianas encontram qualquer menção aos princípios socialistas. Os oponentes o condenam como um sistema econômico fadado ao fracasso e que inevitavelmente leva ao terror totalitário, enquanto ignoram as bem-sucedidas nações socialistas democráticas na Escandinávia. A Europa foi um campo de batalha na Guerra Fria, e os países do norte da Europa já tiveram grandes partidos comunistas e socialistas nacionais que participaram do processo parlamentar, promovendo políticas que garantiam a redistribuição e o bem-estar social. Na década de 1990, enquanto a Rússia, a Hungria e a Polônia liquidavam ativos estatais e desmantelavam suas redes de segurança social, a Dinamarca, a Suécia e a Finlândia mantiveram gastos públicos generosos financiados por indústrias estatais e impostos progressivos, apesar da moda global do neoliberalismo. As sociedades socialistas democráticas do norte da Europa mostram que é possível encontrar uma alternativa humana ao capitalismo neoliberal. E embora não sejam perfeitos ou fáceis de replicar — são etnicamente homogêneos e cada vez mais hostis aos imigrantes — eles encontraram maneiras de combinar as liberdades políticas do Ocidente com as seguranças sociais do Oriente.
 \par 
O norte da Europa não é apenas o lugar mais feliz para se viver no mundo, mas um oásis para mulheres que desfrutam de mais poder econômico e político do que em qualquer outro lugar do planeta. Em um artigo brilhante na Dissent, “Cockblocked bi Redistribution: A Pick-Up Artist in Denmark”, Katie J. M. Baker expôs como o mulherengo americano Daryush Valizadeh (também conhecido como Roosh) alertou seus fãs de que a Dinamarca era um verdadeiro deserto para homens em busca de mulheres fáceis. A generosa rede de segurança social do país e as políticas de igualdade de gênero tornam aparentemente a sedução do macho alfa de Valizadeh
 \par 
17
 \par 
18
 \par 
VOCÊ PODE ESTAR SOFRENDO COM CAPITALISMO técnicas inúteis porque as mulheres dinamarquesas não precisam de homens para segurança financeira. Em países menos igualitários, as mulheres entendem que relacionamentos sexuais fornecem uma avenida para mobilidade social — a fantasia da Cinderela. Mas quando as mulheres ganham seu próprio dinheiro e vivem em sociedades onde o estado apoia sua independência, o Príncipe Encantado perde seu apelo. O livro de Roosh, Don’t Bang Denmark, é um testamento à ideia de que políticas redistributivas podem fornecer às mulheres a estabilidade e a segurança que atenuam os efeitos da discriminação na vida diária.'” *
 \par 
Os jovens estão redescobrindo a ideia de que governos democráticos têm um papel em garantir uma economia justa. Hoje, corporações e elites ricas influenciam os políticos a fazerem suas vontades por meio de contribuições de campanha e lobistas contratados: cortar serviços para os pobres para cortar impostos para os ricos. A decisão da Suprema Corte de 2010 em Citizens United v. FEC afirmou a ideia de que dinheiro é igual a discurso e, portanto, merecia proteção sob a Primeira Emenda da Constituição. Mas enquanto os Estados Unidos permanecerem uma democracia representativa, as pessoas comuns podem votar em seus interesses econômicos e escolher líderes que buscarão políticas de redistribuição e apoiarão redes de segurança social para todos. Até 2020, os eleitores da geração Y constituirão o maior grupo demográfico do eleitorado americano. E as mulheres jovens constituem metade da população da geração Y. A matemática aqui é simples.
 \par 
Uma sondagem Gallup de junho de 2015 concluiu que os americanos com idades compreendidas entre os dezoito e os vinte e nove anos estavam mais dispostos a votar num candidato presidencial “socialista” do que em qualquer outra faixa etária,
 \par 
KRISTEN R. GHODSEE e isso foi bem antes da campanha primária de Bernie Sanders estar o mais rápido possível. Além disso, uma pesquisa YouGov de janeiro de 2016 perguntou aos americanos: "Você tem uma opinião favorável ou desfavorável sobre o socialismo?" Os resultados mostraram uma diferença gritante nas opiniões de diferentes grupos etários. Para aqueles com mais de {\color{blue}65} anos, 60% tinham uma opinião desfavorável sobre o socialismo, em comparação com os 23% que relataram uma opinião favorável. Para aqueles entre {\color{blue}30} e {\color{blue}64} anos, cerca de um quarto relatou uma ideia positiva sobre o socialismo, mas metade dos trinta a quarenta e quatro anos e 54% dos quarenta e cinco a sessenta e quatro anos mantiveram uma visão negativa. Entre os dezoito vinte e nove anos, apenas cerca de um quarto tinha uma visão desfavorável sobre o socialismo. Impressionantes 43% tinham uma opinião favorável, maior do que a porcentagem de dezoito vinte e nove anos que tinham uma opinião positiva sobre o capitalismo (32%)! Uma pesquisa de acompanhamento feita pela Victims of. Communism Memorial Foundation em outubro de 2017 descobriu que o apoio ao socialismo continuou a aumentar entre os jovens: “Para começar, a partir deste ano, mais Millennials prefeririam viver em um país socialista (44%) do que em um capitalista (42%). Ou mesmo em um país comunista (7%). A porcentagem de Millennials que preferiria o socialismo ao capitalismo é dez pontos inteiro maior do que a da população em geral. A significância desta descoberta não pode ser exagerada — a partir do ano passado, os Millennials ultrapassaram os Bebê Boomers como a maior coorte geracional na sociedade americana.”
 \par 
Este mesmo estudo revelou diferenças de gênero fascinantes nas opiniões sobre se os inquiridos viam o capitalismo ou o socialismo como “favoráveis” ou “desfavoráveis”.
 \par 
19
 \par 
20
 \par 
VOCÊ PODE ESTAR SOFRENDO DE CAPITALISMO
 \par 
2.{\color{blue}300} americanos pesquisados, as mulheres representavam 51% da amostra, e suas opiniões frequentemente divergiam significativamente das dos homens. Quando perguntados se tinham uma visão favorável do capitalismo como um sistema econômico, 56% dos homens pesquisados ​​concordaram, em comparação com apenas 44% das mulheres, uma diferença de {\color{blue}12} pontos percentuais. Alternativamente, 53% dos homens tinham uma visão desfavorável do socialismo, em comparação com apenas 47% das mulheres. Embora os homens tendam a ter opiniões políticas mais fortes em geral, essas diferenças de gênero sugerem que as eleitoras são mais inclinadas as políticas redistributivas. E essas mudanças nas opiniões políticas ocorrem apesar dos esforços dos políticos conservadores para confundir todos os ideais esquerdistas com os piores horrores do stalinismo. Talvez a geração Y não confie na autoridade dos guerreiros frios da geração bebê boomer, ou talvez as realidades econômicas dos dias atuais, com crescente desigualdade e ganhos estagnados para a metade inferior da distribuição de renda, sejam mais reais do que histórias de fantasmas sobre um "império do mal" que caiu antes de eles nascerem.
 \par 
George Orwell escreveu uma vez: “Quem controla o passado controla o futuro. Quem controla o presente controla o passado.”’° Os conservadores farão de tudo para suprimir evidências de que experimentos socialistas no século XX (apesar de seu colapso) fizeram algumas coisas boas para as mulheres, incluindo políticas que foram e podem ser implementadas em sociedades democráticas: licenças-maternidade pagas, creches financiadas publicamente, semanas de trabalho mais curtas e flexíveis, educação pós-secundária gratuita, assistência médica universal e outros programas que ajudariam homens e mulheres a levar vidas menos precárias e mais gratificantes. Muitas dessas políticas socialistas já existem em países ocidentais avançados,
 \par 
KRISTEN R. GHODSEE países onde a Fox News e o anticomunismo impulsivo não impedem os cidadãos de votar em seus interesses econômicos. O atual clima político hiperpolarizado atenua uma visão mais matizada do passado. Os críticos conservadores se importam pouco com a história do socialismo de estado do século XX e suas políticas em relação às mulheres. Eles querem manter o coisas como são. Por exemplo, a Victims of. Communism Memorial Foundation, sediada em Washington DC, afirma que "uma geração inteira de americanos está aberta a ideias coletivistas porque não conhece a verdade. Dizemos a verdade sobre o comunismo porque nossa visão é de um mundo livre da falsa esperança do comunismo". Observe o deslizamento entre "ideias coletivistas" e "comunismo", como se as primeiras sempre e inevitavelmente se tornassem as últimas. (Se eu quero ter meu soprador de neve em comum com meus vizinhos, deve ser porque secretamente espero que eles sejam enviados para o Gulag.) Esta fundação projeta currículos de ensino médio, paga por outdoors anticomunistas na Times Square e espera construir um museu de vítimas do comunismo perto do National Mall em Washington, DC (com financiamento de doadores explicitamente de direita). Eles querem controlar a história da mesma forma que a União Soviética fabricou o passado para atender aos seus próprios fins políticos. Se você desafiar seu foco obstinado nos piores aspectos do passado, você desafia sua afirmação de que o socialismo sempre falhará, não importa como ou onde seja tentado no futuro.”
 \par 
A geração Y e os membros da geração Z rejeitam a bagagem da Guerra Fria dos mais velhos, que uma vez proclamaram: “Melhor morto do que vermelho!” Os jovens questionam-se se as suas vidas seriam menos atormentadas, inseguras e stressantes se o governo assumisse um papel mais ativo na redistribuição.
 \par 
21
 \par 
22
 \par 
VOCÊ PODE ESTAR SOFRENDO DE CAPITALISMO
 \par 
Eles têm incentivos para votar em líderes que entendam que os mercados crescem e quebram e que as pessoas comuns precisam de proteção contra as flutuações repentinas e muitas vezes selvagens dos mercados livres. Líderes populistas de direita tentarão usar mulheres, pessoas de cor e imigrantes como bodes expiatórios para desviar a culpa das verdadeiras raízes da injustiça econômica: a alta concentração de riqueza nas mãos de cada vez menos pessoas. Enquanto homens e mulheres comuns lutam e se esforçam para cobrir suas necessidades básicas em uma economia que promete oportunidades iguais para mobilidade social, mas na qual 78% das crianças afro-americanas nascidas entre 1985 e 2000 cresceram em bairros altamente desfavorecidos (em comparação com apenas 5% das crianças brancas), os cidadãos devem se unir para efetuar uma mudança política real.”
 \par 
Sejamos claros: não defendo um retorno a nenhuma forma de socialismo de estado do século XX. Esses experimentos falharam sob o peso de suas próprias contradições: o vasto abismo entre seus ideais declarados e as práticas reais de líderes autoritários. Você não deveria ter que sacrificar papel higiênico por assistência médica. Liberdades políticas básicas não precisam ser trocadas por emprego garantido. Mas houve outros caminhos não tomados, como aqueles imaginados pelos primeiros teóricos socialistas como Karl Liebknecht e Rosa Luxemburgo. E nenhum experimento socialista foi autorizado a florescer sem enfrentar a oposição aberta ou encoberta dos Estados Unidos, sejam confrontos diretos como os da Coreia e do Vietnã ou operações secretas em lugares como Cuba, Chile ou Nicarágua. Alguém disse "caso Irã-Contra"? Além disso, as circunstâncias históricas do século XXI diferem das do século XX. À medida que nossa economia global evolui e muda em resposta a novas
 \par 
KRISTEN R. GHODSEE tecnologias, os cidadãos precisam ter acesso a um kit de ferramentas teóricas que contenha a mais ampla gama de possíveis soluções políticas para os problemas que enfrentaremos nos próximos anos.
 \par 
Assim como os camponeses europeus acreditavam que Deus ungiu reis e aristocratas para governá-los, hoje muitos acreditam que os super-ricos ganharam seu dinheiro em uma competição justa em mercados livres. Mas, à medida que as suspeitas da chamada economia fraudada aumentam, mais e mais jovens estão buscando alternativas. O filósofo do século XVII Spinoza supostamente disse: "Se você quer que o futuro seja diferente do presente, estude o passado". Mesmo que os experimentos passados ​​com o socialismo tenham falhado, houve alguns sucessos. Devemos estudar esses sucessos e salvar o que pudermos das ferramentas teóricas e práticas mais poderosas que temos para limitar os piores excessos do capitalismo global hoje. As mulheres jovens, em particular, têm pouco a perder e muito a ganhar com um esforço coletivo para construir sociedades mais justas, equitativas e sustentáveis". Este livro explica o porquê.
 \par 
23
 \par 
\begin{figure}
	\centering
	\includegraphics[width=1.\textwidth]{temp\_files/images/UP\_logo.png }
	\caption{Clara Zetkin (1857-1933): Editora do Die Gleichheit (Igualdade), um jornal do Partido Social-Democrata Alemão, Zetkin foi uma das principais arquitetas do ativismo das mulheres socialistas. Ela foi a fundadora do Dia Internacional da Mulher em 1910, celebrado todo ano em {\color{blue}8} de março. Após a eclosão da Primeira Guerra Mundial, ela se separou do Partido Social-Democrata Alemão e se tornou ativa no Partido Comunista Alemão, servindo como membro da Assembleia Constituinte durante a República de Weimar. Zetkin acreditava que homens e mulheres socialistas precisavam trabalhar juntos para derrubar a burguesia e desprezava feministas independentes. Cortesia de Archiv der sozialen Demokratie/ Friedrich-Ebert-Foundation.}
	\label{ }
\end{figure}