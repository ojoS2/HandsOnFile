 
 \chapter{Conservatism and Counterrevolution}  

 \label{Conservatism and Counterrevolution}  
 
 
\par
 
 
 \textit{	Whoever fi gets monsters should see to it that in the process he does not become a monster.}  

 
\par
 
 
 
\par
 

 \textbf{\textit{	—Friedrich Nietzsche, Beyond Good and Evil} }  

 
\par
 

 \footnote{This chapter originally appeared in Marital 30, no. {\color{blue} 1 } (Summer 2010): 1–17.}  
Quando John McCain anunciou Sarah Palin como sua companheira de chapa durante a campanha presidencial de 2008, vozes no movimento conservador expressaram surpresa, até mesmo choque. Não foi apenas o fato de McCain ter escolhido um novato político, um ingênuo e estranho às formas e meios de governação nos quarenta e oito estados da região inferior. Foi assim que ele a escolheu: com pouca ou nenhuma verificação e muita fé na superioridade da intuição e do impulso (dele e dela) sobre a razão e a reflexão. Além disso, parecia ser uma decisão muito derivada da União: impetuosa, mal considerada, imprudente.
 
\par
 
Esta não foi a primeira vez que um porta-estandarte do conservadorismo não conseguiu corresponder à autoimagem do conservador. Na primavera de 2003, vários conservadores manifestaram preocupação com a audácia da decisão de George W. Bush de travar o que era essencialmente uma guerra de escolha. Também salientaram o pedigree liberal de uma das justificações da Guerra do Iraque: a difusão da democracia e dos direitos humanos. Aqui estava um líder conservador, mais uma vez ao que parecia, agindo da forma mais nada conservadora: abandonando o realismo do seu pai e do seu partido por um internacionalismo há muito considerado propriedade exclusiva da esquerda, pressionando a marcha da história contra o coisas como são dá o Oriente Médio.
 
\par
 
Desde que Edmund Burke inventou o conservadorismo como ideia, o conservador tem-se autodenominado um homem de prudência e moderação, e a sua causa é um reconhecimento sóbrio – e sério – dos limites. “Ser conservador”, ouvimos Michael Makeshift declarar na introdução, “é preferir o familiar ao desconhecido. . . O tentado ao não experimentado, o fato ao mistério, o real ao possível, o limitado ao ilimitado, o próximo ao distante.”
 {\color{blue} 1}  
No entanto, os esforços políticos que levaram o conservador às suas mais profundas EC-reais – as reações contra as revoluções francesa e bolchevique; a defesa da escravidão e de Jim Crow; o ataque à social-democracia e ao Estado-providência; e as reações violentas em série contra o New Deal, a Grande Sociedade, os direitos civis, o feminismo e os direitos dos homossexuais – têm sido tudo menos isso. Seja na Europa ou nos Estados Unidos, neste século ou nos anteriores, o conservadorismo tem sido um movimento progressivo de mudança incansável e implacável, parcial à assunção de riscos e ao aventureirismo ideológico, militante na sua postura e populista nas suas orientações, amigável com os novatos e os insurgentes.  estrangeiros e recém-chegados. Embora o teórico conservador reivindique para a sua tradição o manto da prudência e da moderação, há uma tensão não tão subterrânea de imprudência e moderação que atravessa essa tradição – uma tensão que, por mais contra-intuitiva que pareça, liga Sarah Palin a Edmund Burke.
 
\par
 
Uma consideração desta vertente mais profunda do conservadorismo dá-nos uma noção mais clara do que é o conservadorismo. Embora o conservadorismo seja uma ideologia de reação – originalmente contra a Revolução Francesa, mais recentemente contra os movimentos de libertação dos anos sessenta e setenta – essa reação não foi bem compreendida. Longe de produzir uma defesa instintiva de um velho regime imutável ou de um tradicionalismo ponderado, o imperativo reacionário pressiona o conservadorismo em duas direções bastante diferentes: primeiro, para uma crítica e reconfiguração do antigo regime; e segundo, a uma absorção das ideias e tácticas da própria revolução ou reforma à qual se opõe. O que o conservadorismo procura realizar através dessa reconfiguração do antigo e da absorção do novo é tornar popular o privilégio, transformar um velho regime cambaleante num movimento de massas dinâmico e ideologicamente coerente. Um novo e velho regime, poder-se-ia dizer, que traz a energia e o dinamismo da rua às antigas desigualdades de uma propriedade dilapidada.
 
\par
 
À medida que o domínio de quarenta anos da direita começa a desaparecer, por mais adequado que seja, escritores como Sam Tannhauser, Andrew Sullivan, Jeff Red Hat, Sidney Blumenthal e John Dean afirmam que o conservadorismo entrou em declínio quando Pain, ou Bush, ou Reagan, ou Goldwater, ou Buckley, ou alguém tirou isso dos trilhos. Originalmente, prossegue o argumento, o conservadorismo era uma disciplina responsável das classes governantes, mas algures entre Joseph de Maistre e Joe, o Encanador, ele deixou-se levar por si. Tornou-se aventureiro, fanático, populista, ideológico. O que esta história de declínio ignora – quer emane da direita ou da esquerda – é que estes supostos vícios do conservadorismo contemporâneo estavam presentes no início, nos escritos de Burke e Maître, só que não estavam presentes. Vistos como vícios. Eles eram vistos como virtudes. O conservadorismo sempre foi um movimento mais selvagem e extravagante do que muitos imaginam – e é precisamente esta selvageria e extravagância que tem sido uma das fontes do seu apelo contínuo.
 
\par
 
Não é provocativo dizer que o conservadorismo surgiu em reação à Revolução Francesa. A maioria dos conservadores com mentalidade histórica concordaria.
 {\color{blue} 2}  
Mas se olharmos mais atentamente para duas vozes emblemáticas dessa reação – Burke e Maître – encontraremos vários elementos surpreendentes e raramente notados. A primeira é uma antipatia, beirando o desprezo, pelo antigo regime que reivindicam como causa. Os capítulos iniciais das Considerações sobre a França de Maître são um ataque implacável aos três pilares do antigo regime: a aristocracia, a igreja e a monarquia. Maître divide a nobreza em duas categorias: os traiçoeiros e os sem noção. O clero é corrupto, enfraquecido pela sua riqueza e pela sua moral negligente. A monarquia é branda e não tem vontade de punir. Maître descarta todos os três com uma frase de Racine: “Agora vejam os tristes frutos que suas falhas produziram, / Sintam os golpes que vocês mesmos induziram”.
 {\color{blue} 3}  

 
\par
 
No caso de Burke, a crítica é mais sutil, mas mais profunda. (Embora no final de sua vida ele falasse nos mesmos tons modulados de Maître.)
 {\color{blue} 4}  
Isso ocorre durante seu relato em Reflexões sobre a Revolução na França sobre a invasão do palácio de Versalhes e a captura da família real. Lá, Burke descreve Maria Antonieta como uma “visão encantadora. . . Brilhando como a estrela da manhã, cheia de vida, esplendor e alegria.” Burke considera a sua beleza um símbolo da beleza do antigo regime, onde os costumes e costumes feudais “tornaram o poder suave” e “por uma assimilação branda, incorporaram na política os sentimentos que embelezam e suavizam a sociedade privada”.
 {\color{blue} 5}  

 
\par
 
Desde que escreveu essas linhas, Burke tem sido ridicularizado por seu sentimentalismo. Mas os leitores do trabalho anterior de Burke sobre estética, Uma Investigação Filosófica sobre as Origens das Nossas Ideias do Sublime e do Belo, saberão que a beleza, para Burke, nunca é um sinal de vitalidade do poder; é sempre um sinal de decadência. A beleza desperta o prazer, que dá lugar à indiferença ou leva à dissolução total do eu. “A beleza atua”, escreve Burke, “relaxando os sólidos de todo o sistema”.
 {\color{blue} 6}  
É este relaxamento e dissolução dos corpos – corpos físicos, sociais e políticos – que torna a beleza um símbolo e agente tão potente de degeneração e morte. “Nossas instituições mais salutares e mais belas não produzem nada além de poeira e sujeira.”
 {\color{blue} 7}  

 
\par
 
O que estas duas declarações iniciais da persuasão conservadora sugerem é que o maior inimigo do antigo regime não é nem o revolucionário, nem o reformador; é o próprio antigo regime ou, para ser mais preciso, os defensores do antigo regime.
 {\color{blue} 8}  
Falta-lhes simplesmente os meios ideológicos para pressionar a causa do antigo regime com o vigor, a clareza e o propósito necessários. Tal como Burke declarou sobre George Grenville, no contexto muito diferente da relação britânica com as suas colônias americanas:
 
\par
 

 \textbf{\textit{But it may be truly said, that men too much conversant in off ice, are rarely minds of remarkable enlargement. . . . Persons who are nurtured in off ice do admirably well as long as things go on in their common order; but when the high roads are broken up, and the waters out, when a new and troubled scene is opened, and the file affords no precedent, then it is that a greater knowledge of mankind, and a far more extensive comprehension of things, is requisite, than ever off ice gave, or than off ice can ever give. {{\color{blue} 9} } } }  
 
 
\par
 
Mais tarde, os conservadores farão esta afirmação de várias maneiras. Por vezes acusarão os defensores do antigo regime de terem sido intimidados pelo desafio revolucionário ou reformista. De acordo com Thomas Dew, um dos primeiros e mais agressivos apologistas da escravatura americana, a rebelião de Nat Turner destruiu “todo o sentimento de segurança e confiança” entre a classe dominante. Eles ficaram tão assustados que “a razão foi quase banida da mente”. Não foi apenas a violência dos escravos que os assustou. Foi a acusação moral levantada pelos escravos e pelos abolicionistas, que de alguma forma se insinuou nas mentes dos proprietários de escravos e os deixou inseguros quanto à sua própria posição. “Nós próprios”, escreveu William Harvey, outro defensor da escravatura, “até certo ponto nos declaramos culpados do impeachment”.
 {\color{blue} 10}  

 
\par
 
Mais de um século depois, Barry Goldwater abordaria o mesmo tema. O primeiro parágrafo de A consciência de um conservador não dirige o seu fogo contra os liberais ou os democratas, ou mesmo contra o Estado de bem-estar social; visa a timidez moral do que mais tarde será chamado de “Estabelecimento Republicano”.
 
\par
 

 \textbf{\textit{I have been much concerned that so many people today with Conservative instincts feel compelled to apologize for them. Or if not to apologize directly, to qualify their commitment in a way that amounts to breast beating. “Republican candidates,” Vice President Nixon has said, “should be economic conservatives, but conservatives with a heart.” President Eisenhower announced during his first term, “I am conservative when it comes to economic problems but liberal when it comes to human problems.” . . . These formulations are tantamount to an admission that Conservatism is a narrow, mechanistic economic theory that may work very well as a bookkeeper’s guide, but cannot be relied upon as a comprehensive political philosophy. {{\color{blue} 11} } } }  
 
 
\par
 
Mais frequentemente, os conservadores argumentam que o defensor do antigo regime é simplesmente obtuso. Ele se tornou preguiçoso, gordo e complacente, desfrutando tão plenamente dos privilégios de sua posição que não consegue ver a catástrofe que se aproxima. Ou, se o consegue ver, não pode fazer nada para se defender, uma vez que os seus músculos políticos se atrofiaram há muito tempo. John C. Calhoun era um desses conservadores e, ao longo da década de 1830, quando os abolicionistas começaram a pressionar a sua causa, ele ficou furioso com a vida fácil e a ignorância obstinada dos seus camaradas na plantação. A sua fúria atingiu o auge em 1837, quando, num discurso no plenário do Senado, instou o Congresso a não receber uma petição abolicionista – um momento, como vimos na introdução, que ele recordaria até ao dia da sua morte. “Tudo o que queremos é um concerto”, implorou ele aos seus colegas sulistas, para “nos unirmos com zelo e energia para repelir os perigos que se aproximam”. Mas, continuou ele, “não ouso esperar que qualquer coisa que eu possa dizer desperte o Sul para a devida sensação de perigo. Temo que esteja além do poder da voz mortal despertá-lo a tempo da segurança fatal em que caiu.”
 {\color{blue} 12}  

 
\par
 
No seu influente ensaio, Makeshift argumentou que o conservadorismo “não é um credo ou uma doutrina, mas uma disposição”. Especificamente, pensou ele, é uma disposição para aproveitar o presente. Não porque o presente seja melhor do que as alternativas ou mesmo porque seja bom nos seus próprios termos. Isso implicaria um nível de reflexão consciente e de escolha ideológica que Makeshift acredita ser estranho ao conservador. Não, a razão pela qual o conservador gosta do presente é simples e meramente porque é familiar, porque está lá, porque está próximo.
 {\color{blue} 13}  

 
\par
 
A visão improvisada do conservador – e esta visão é amplamente partilhada tanto pela esquerda como pela direita – não é uma visão; é uma presunção. Ignora que o conservadorismo surge invariavelmente em resposta a uma ameaça ao antigo regime ou após o antigo regime ter sido destruído. (Makeshift admite abertamente que a perda ou ameaça de perda nos faz valorizar o presente, como argumentei na introdução, mas ele não permite que esse visão penetre ou desaloje sua compreensão geral do conservadorismo.) Makeshift é descrever o antigo regime em uma poltrona.  quando a sua mortalidade é uma noção distante e o tempo é um meio de aquecimento e não um solvente acre. Este é o antigo regime de Charles Roseau, que escreveu quase dois séculos antes da Revolução Francesa que a nobreza não tem “começo” e, portanto, não tem fim. Ele “existe no tempo fora da mente”, sem consciência ou consciência da passagem da história.
 {\color{blue} 14}  

 
\par
 
O conservadorismo surge em cena precisamente quando – e precisamente porque – tais declarações já não podem ser feitas. Walter Berns, um dos muitos futuros neoconservadores em Cornell que ficaram traumatizados em 1969 pela tomada de Willard Straight Hall pelos estudantes negros, afirmou no seu discurso de despedida quando se demitiu da universidade: “Tínhamos um mundo demasiado bom; não poderia durar.
 {\color{blue} 15}  
Nada perturba tanto o idílio da herança como a substituição repentina e muitas vezes brutal de um mundo por outro. Tendo testemunhado a morte daquilo que deveria viver para sempre, o conservador não pode mais considerar o tempo como o aliado natural ou habitat do poder. O tempo agora é o inimigo. A mudança, e não a permanência, é o governante universal, e a mudança não significa nem progresso, nem melhoria, mas morte, e ainda por cima uma morte precoce e não natural. “O decreto da morte violenta”, diz Maître, está “escrito nas próprias fronteiras da vida”.
 {\color{blue} 16}  
O problema do defensor do antigo regime, diz o conservador, é que ele não conhece esta verdade ou, se conhece, não tem vontade de fazer algo a respeito.
 
\par
 
O segundo elemento que encontramos nestas primeiras vozes de reação é uma admiração surpreendente pela própria revolução contra a qual escrevem. Os comentários mais arrebatadores de Maître são reservados aos jacobinos, cuja vontade brutal e propensão para a violência – a sua “magia negra” – ele claramente inveja. Os revolucionários têm fé, na sua causa e em si próprios, o que transforma um movimento de mediocridades na força mais implacável que a Europa alguma vez viu. Graças aos seus esforços, a França foi purificada e restaurada no seu legítimo lugar de destaque entre a família das nações. “O governo revolucionário”, conclui Maître, “endureceu a alma da França, temperando-a com sangue”.
 {\color{blue} 17}  

 
\par
 
Burke, novamente, é mais sutil, mas corta mais profundamente. O grande poder, sugere ele em O Sublime e o Belo, nunca deveria aspirar a ser – e nunca poderá realmente ser – bonito. O que um grande poder precisa é de sublimidade. O sublime é a sensação que experimentamos diante de extrema dor, perigo ou terror. É algo parecido com admiração, mas tingido de medo e pavor. Burke chama isso de “horror delicioso”. O grande poder deveria aspirar à sublimidade em vez da beleza, porque a sublimidade produz “a emoção mais forte que a mente é capaz de sentir”. É uma emoção cativante, mas revigorante, que tem o efeito simultâneo, mas contraditório, de nos diminuir e de nos engrandecer. Sentimo-nos aniquilados por um grande poder; ao mesmo tempo, nosso senso de identidade “incha” quando “estamos familiarizados com objetos terríveis”. O grande poder alcança a sublimidade quando é, entre outras coisas, obscuro e misterioso, e quando é extremo. “Em todas as coisas”, escreve Burke, o sublime “abomina a mediocridade”.
 {\color{blue} 18}  

 
\par
 
Nas Reflexões, Burke sugere que o problema na França é que o antigo regime é belo enquanto a revolução é sublime. O interesse fundiário, a pedra angular do antigo regime, é “lento, inerte e tímido”. Não pode defender-se “das invasões de capacidade”, sendo que a capacidade representa aqui os novos homens de poder que a revolução revisita. Noutra parte das Reflexões, Burke diz que o interesse monetário, que está aliado à revolução, é mais forte do que o interesse aristocrático porque está “mais pronto para qualquer aventura” e “mais disposto a novos empreendimentos de qualquer tipo”. O antigo regime, por outras palavras, é belo, estático e fraco; a revolução é feia, dinâmica e forte. E nos horrores que a revolução perpetra – a turba invadindo o quarto da rainha, arrastando-a seminua para a rua e levando-a e à sua família para Paris – a revolução atinge uma espécie de sublimidade: “Estamos alarmados a verdadeira Exxon”, escreve Burke sobre as ações dos revolucionários. "Nossas mentes. . . São purificados pelo terror e pela piedade; nosso orgulho fraco e irrefletido é humilhado, sob as dispensações de uma sabedoria misteriosa.”
 {\color{blue} 19}  

 
\par
 
Para além destas simples declarações de inveja ou admiração, o conservador, na verdade copia e aprende com a revolução a que se opõe. “Para destruir esse inimigo”, escreveu Burke sobre os jacobinos, “garantidamente, a força que se opõe a ele deveria ter alguma analogia e semelhança com a força e o espírito que esse sistema exerce”.
 {\color{blue} 20}  
Este é um dos aspectos mais interessantes e menos compreendidos da ideologia conservadora. Embora os conservadores sejam hostis aos objectivo da esquerda, particularmente ao empoderamento das castas e classes mais baixas da sociedade, são frequentemente os melhores alunos da esquerda. Por vezes, os seus estudos são autoconscientes e estratégicos, pois olham para a esquerda em busca de formas de adaptar os novos vernáculos, ou os novos meios de comunicação, aos seus objectivo subitamente legitimados. Temendo que os filosofes tivessem assumido o controlo da opinião popular em França, os teólogos reacionários de meados do século XVIII olharam para o exemplo dos seus inimigos. Eles pararam de escrever dissertações obscuras uns para os outros e começaram a produzir agitprop católico, que seria distribuído através das mesmas redes que levaram o esclarecimento ao povo francês. Eles gastaram vastas somas financiando concursos de redação, como aqueles em que Rousseau fez seu nome, para recompensar escritores que escreveram defesas acessíveis e populares da religião. Os tratados de fé anteriores, declarou Charleslouis Richard, eram “inúteis para a multidão que, sem armas e sem defesas, sucumbe rapidamente às filosofias”. Sua obra, ao contrário, foi escrita “com o propósito de colocar nas mãos de todos os que sabem ler uma arma vitoriosa contra os assaltos desta turbulenta Filosofia”.
 {\color{blue} 21}  

 
\par
 
Os pioneiros da Estratégia do Sul na administração Nixon, para citar um exemplo mais recente, compreenderam que depois das revoluções pelos direitos dos anos 60 já não podiam fazer simples apelos ao racismo branco. De agora em diante, eles teriam que falar em código, de preferência um código aceitável para a nova dispensação dos daltônicos. Como observou o chefe de gabinete da Casa Branca, H. R. Alderman, no seu diário, Nixon “enfatizou que é preciso encarar que todo o problema são realmente os negros. A chave é criar um sistema que reconheça isso, embora não pareça.”
 {\color{blue} 22}  
Relembrando esta estratégia em 1981, o estratega republicano Lee Atwater expôs os seus elementos de forma mais clara:
 
\par
 

 \textbf{\textit{You start out in 1954 by saying, “Nigger, nigger, nigger.” By 1968 you can’t say “nigger”—that hurts you. Backfires. So you say stuff like forced busing, states’ rights and all that stuff. You’re getting so abstract now you’re talking about cutting taxes, and} }  
 
 
\par
 

 
\par
 

 \textbf{\textit{All these things you’re talking about are totally economic things and a by-product of them is blacks get hurt worse than whites. And subconsciously maybe that is part of it. {{\color{blue} 23} } } }  
 
 
\par
 
Mais recentemente ainda, David Horowitz encorajou os estudantes conservadores “a usarem a linguagem que a esquerda utilizou de forma tão eficaz em nome das suas próprias agendas. Professores radicais criaram um “ambiente de aprendizagem hostil” para estudantes conservadores. Há uma falta de “diversidade intelectual” nas faculdades e nas salas de aula acadêmicas. O ponto de vista conservador está “sub-representado” no currículo e nas suas listas de leitura. A universidade deve ser uma comunidade ‘inclusiva’ e intelectualmente ‘diversa’”.
 {\color{blue} 24}  

 
\par
 
Outras vezes, a educação do conservador é inconsciente, acontecendo, por assim dizer, pelas suas costas. Ao resistir e, assim, envolver-se com o argumento progressista dia após dia, ele acaba por ser influenciado, muitas vezes contra sua vontade, pelo próprio movimento ao qual se opõe. Ao tentar dobrar um vernáculo à sua vontade, ele descobre que sua vontade é dobrada pelo vernáculo. Atwater afirma que foi precisamente isso que ocorreu dentro do Partido Republicano; após sugerir “inconscientemente, talvez isso faça parte”. Ele adiciona:
 
\par
 

 \textbf{\textit{I’m not saying that. But I’m saying that if it is getting that abstract, and that coded, that we are doing away with the racial problem one way or the other. You follow me—because obviously sitting around saying, “We want to cut this,” is much more abstract than even the busing thing, and a hell of a lot more abstract than “Nigger, nigger.” {{\color{blue} 25} } } }  
 
 
\par
 
Os republicanos aprenderam a disfarçar tão bem as suas intenções, argumenta Atwater, que o disfarce penetrou e transformou a intenção. Assumindo que tal transformação ocorreu de facto, poderíamos muito bem perguntar se o conservador deixou de ser o que pretendia ser. Mas essa é uma questão para outro dia.
 
\par
 
Mesmo sem se envolver diretamente com o argumento progressista, os conservadores podem absorver, por alguma osmose elusiva, as categorias e expressões idiomáticas mais profundas da esquerda, mesmo quando essas expressões idiomáticas vão diretamente contra sua posição oficial. Após anos se opondo ao movimento das mulheres, por exemplo, Phyllis School y parecia genuinamente incapaz de conjurar a visão pré-feminista das mulheres como esposas e mães deferentes. Em vez disso, ela celebrou o ativista “poder da mulher positiva”. E então, como se estivesse pegando emprestado uma página de The Feminine Mystique, ela protestou contra a falta de sentido e realização entre as mulheres americanas; só que ela culpou esses males no feminismo e não no sexismo.
 {\color{blue} 26}  
Quando ela se manifestou contra a Emenda de Direitos Iguais (ERA), ela não alegou que ela introduziu uma nova linguagem radical de direitos. Seu argumento foi o oposto. A ERA, ela disse ao Washington Star, “é uma retirada dos direitos das mulheres”. Ela “retirará o direito da esposa em um casamento em andamento, da esposa no lar”.
 {\color{blue} 27}  
A escola estava obviamente a utilizar a linguagem dos direitos de uma forma que se opunha aos objectivo do movimento feminista; ela estava usando o discurso sobre direitos para colocar as mulheres de volta em casa, para mantê-las como esposas e mães. Mas a questão é essa: o conservadorismo adapta e adota, muitas vezes inconscientemente, a linguagem da reforma democrática à causa da hierarquia.
 
\par
 
Também se pode detectar uma certa franqueza sexual – até mesmo preocupação feminista – nas primeiras conversas da direita cristã que teria sido impensável antes do movimento das mulheres. Em 1976, Beverly e Tim Lahore escreveram um livro, The Act of. Marriage, que Susan Faludi chamou corretamente de “o equivalente evangélico de The Joy of. sex”. Lá, os La Hayes afirmaram que “as mulheres são passivas demais ao fazer amor”. Deus, disseram os La Hayes às suas leitoras, “colocou [seu clitóris] lá para sua diversão”. Eles também reclamaram que “alguns maridos são remanescentes da Idade das Trevas, como aquele que disse à sua esposa frustrada: ‘Garotas bonitas não devem chegar ao clímax’.
 {\color{blue} 28}  

 
\par
 
O que o conservador aprende em última análise com os seus oponentes, intencional ou inconscientemente, é o poder da agência política e a potência das massas. Do trauma da revolução, os conservadores aprendem que homens e mulheres, seja por altos de força voluntários ou de algum outro exercício de ação humana, podem ordenar as relações sociais e o tempo político. Em cada movimento social ou momento revolucionário, os reformadores e radicais têm de inventar – ou redescobrir – a ideia de que a desigualdade e a hierarquia social não são fenômenos naturais, mas criações humanas. Se a hierarquia pode ser criada por homens e mulheres, pode ser descriada por homens e mulheres, e é isso que um movimento social ou uma revolução se propõe fazer. A partir destes esforços, os conservadores aprendem uma versão da mesma lição. Enquanto os seus antecessores no antigo regime pensavam na desigualdade como um fenômeno que ocorre naturalmente, uma herança transmitida de geração em geração, o encontro dos conservadores com a revolução ensina-lhes que, afinal de contas, os revolucionários tinham razão: a desigualdade é uma criação humana. E se pode ser descriado por homens e mulheres, pode ser recriado por homens e mulheres. “Cidadãos!” exclama Maître no final de Considerações sobre a França. “É assim que as contra-revoluções são feitas.”
 {\color{blue} 29}  
Sob o antigo regime, a monarquia – como o patriarcado ou Jim Crow – não é feita. Apenas isso. Seria difícil imaginar um Roseau ou Bosses declarando: “Homens” – muito menos cidadãos – “é assim que se faz uma monarquia”. Mas uma vez ameaçado ou derrubado o antigo regime, o conservador é forçado a perceber que é uma agência humana, a imposição voluntária do intelecto e da imaginação ao mundo, que gera e mantém a desigualdade ao longo do tempo. Saindo do seu confronto com a revolução, o conservador expressa o tipo de afirmação da agência política que se encontra neste editorial de 1957 da National Review de William F. Buckley: “A questão central que emerge” do movimento pelos direitos civis “é se os brancos comunidade do Sul tem o direito de tomar as medidas necessárias para prevalecer, política e culturalmente, em áreas nas quais não predomina numericamente? A resposta séria é sim – a comunidade branca tem esse direito porque, por enquanto, é a raça avançada.”
 {\color{blue} 30}  

 
\par
 
O revolucionário declara o Ano I, e em resposta o conservador declara o Ano Negativo I. A partir da revolução, o conservador desenvolve uma atitude particular em relação ao tempo político, uma crença no poder dos homens e mulheres de moldar a história, de impulsioná-la para frente ou para trás; e em virtude dessa crença, ele passa a adotar o futuro como seu tempo verbal preferido. Ronald Reagan fora e a destilação perfeita desse fenômeno quando ele invocou, repetidamente, o ditado de Thomas Paine de que "temos o poder de começar o mundo de novo".
 {\color{blue} 31}  
Mesmo quando o conservador afirma estar a preservar um presente que está ameaçado ou a recuperar um passado que está perdido, ele é impelido pelo seu próprio ativismo e agência a confessar que está a fazer um novo começo e a criar o futuro.
 
\par
 
Burke estava especialmente sintonizado com este problema e por isso muitas vezes se esforçava para lembrar aos seus camaradas na batalha contra a Revolução que tudo o que fosse reconstruído na França após a restauração inevitavelmente, como ele disse numa carta a um emigrado, “estaria em algum lugar”. Medir uma coisa nova.”
 {\color{blue} 32}  
Outros conservadores têm sido menos ambivalentes, afirmando alegremente as virtudes da criatividade política e da originalidade moral. Alexander Stephens, vice-presidente da Confederação dos EUA, declarou orgulhosamente que “nosso novo governo é o primeiro, na história do mundo” a ser fundado na “grande verdade física, filosófica e moral” de que “o negro não é igual para o homem branco; que a escravidão – subordinação à raça superior – é sua condição natural e normal.”
 {\color{blue} 33}  
Barry Goldwater disse simplesmente: “Nosso futuro, assim como nosso passado, será o que fizermos dele”.
 {\color{blue} 34}  

 
\par
 
A partir das revoluções, os conservadores também desenvolvem um gosto e um talento para as massas, mobilizando as ruas para demonstrações espetaculares de poder, ao mesmo tempo que garantem que o poder nunca é verdadeiramente partilhado ou redistribuído. Essa é a tarefa do populismo de direita: apelar às massas sem perturbar o poder das elites ou, mais precisamente, aproveitar a energia das massas para reforçar ou restaurar o poder das elites. Longe de ser uma inovação recente da Direita Cristã ou do movimento Tea Party, o populismo reacionário corre como um fio vermelho em todo o discurso conservador desde o início.
 
\par
 
Maître foi um pioneiro no teatro do poder de massa, imaginando cenas e encenando dramas em que os mais baixos dos mais baixos podiam ver-se refletidos nos mais altos dos mais altos. “A monarquia”, escreve ele, “é, sem contradição, a forma de governo que confere maior distinção ao maior número de pessoas”. As pessoas comuns “partilham” o seu “brilho” e brilho, embora Maître tenha o cuidado de acrescentar, nas suas decisões e deliberações: “o homem é honrado não como um agente, mas como uma porção da soberania”.
 {\color{blue} 35}  
Arquimonarquista que era, maître entendeu que o rei nunca poderia retornar ao poder se não tivesse um toque de plebeu. Assim, quando Maître imagina o triunfo da contra-revolução, tem o cuidado de enfatizar as credenciais populistas do monarca que regressa. O povo deveria identificar-se com este novo rei, diz Maître, porque, tal como eles, ele frequentou a “terrível escola do infortúnio” e sofreu na “dura escola da adversidade”. Ele é “humano”, com humanidade aqui conotando uma capacidade de erro quase pedestre e reconfortante. Ele será como eles. Ao contrário dos seus antecessores, ele saberá disso, o que “é muito”.
 {\color{blue} 36}  

 
\par
 
Mas para apreciar plenamente a inventividade do populismo de direita, temos de recorrer à classe magistral do Velho sul. O senhor de escravos criou uma forma quintessencial de feudalismo democrático, transformando a maioria branca numa classe senhorial, partilhando os privilégios e prerrogativas de governar a classe escrava. Embora os membros desta classe dominante soubessem que não eram iguais entre si, foram compensados ​​pela ilusão de superioridade – e pela realidade do domínio – sobre a população negra abaixo deles.
 
\par
 
Uma escola de pensamento – chamemos-lhe escola de oportunidades iguais – localizou a promessa democrática da escravatura no fato de colocar a possibilidade de domínio pessoal ao alcance de todos os homens brancos. A genialidade dos proprietários de escravos, escreveu Daniel Handler em seu livro Social Relations in Our Southern States, é que eles “não são uma aristocracia exclusiva. Todo homem branco livre em toda a União tem o mesmo direito de se tornar um oligarca.” Isto não era apenas propaganda: em 1860, havia 400 mil proprietários de escravos no Sul, tornando a classe dominante americana uma das mais democráticas do mundo. Os proprietários de escravos tentaram repetidamente aprovar leis que encorajassem os brancos a possuir pelo menos um escravo e até consideraram conceder incentivos fiscais para facilitar essa propriedade. O pensamento deles, nas palavras de um agricultor do Tennessee, era que “no minuto em que se tira do poder dos agricultores comuns comprar um homem ou uma mulher negra. . . Você faz dele um abolicionista imediatamente.”
 {\color{blue} 37}  

 
\par
 
Essa escola de pensamento lutou com uma segunda escola, possivelmente mais influente. A escravidão americana não era democrática, de acordo com esta linha de pensamento, porque representava uma oportunidade de domínio pessoal para os homens brancos: a escravidão americana era democrática porque tornava cada homem branco, proprietário de escravos ou não, um membro da classe dominante em virtude da cor de sua pele. Nas palavras de Calhoun: “Conosco, as duas grandes divisões da sociedade não são os ricos e os pobres, mas os brancos e os negros; e todos os primeiros, tanto os pobres como os ricos, pertencem à classe alta e são respeitados e tratados como iguais.”
 {\color{blue} 38}  
Ou, como disse seu colega júnior, James Henry Hammond: “Num país escravista, todo homem livre é um aristocrata”.
 {\color{blue} 39}  
Mesmo sem escravos ou sem os pré-requisitos materiais para a liberdade, um homem branco pobre poderia autodenominar-se membro da nobreza e, portanto, ser confiável para tomar as medidas necessárias em sua defesa.
 
\par
 
Quer se subscrevesse a primeira ou a segunda escola de pensamento, a classe dominante acreditava que o feudalismo democrático era um potente contra-ataque aos movimentos igualitários que então agitavam a Europa e a América jacksoniana. Os radicais europeus, declarou Dew, “desejam que toda a humanidade seja levada a um nível comum. Acreditamos que a escravidão, nos Estados Unidos, conseguiu isso.” Ao libertar os brancos de “regiões servis e inferiores”, a escravatura eliminou “a maior causa de distinção e separação das classes da sociedade”.
 {\color{blue} 40}  
Enquanto as classes dominantes do século XIX enfrentavam desafio após desafio ao seu poder, a classe dominante aumentava a dominação racial como forma de aproveitar a energia das massas brancas, em apoio, e não em oposição, aos privilégios e poderes. Das elites estabelecidas. Este programa encontraria o seu cumprimento final um século mais tarde e a um continente de distância.
 
\par
 
Estas correntes populistas podem ajudar-nos a compreender um elemento final do conservadorismo. Desde o início, o conservadorismo apelou e contou com pessoas de fora. Maître era da Sabóia, Burke da Irlanda. Alexander Hamilton nasceu fora do casamento em Nevis e, segundo rumores, era parte negro. Disraeli era judeu, tal como muitos dos neoconservadores que ajudaram a transformar o Partido Republicano de um coquetel em Darien no partido de Scalia, d’Souza, Gonzalez e Yew. (Foi Irving Bristol quem primeiro identificou “a tarefa histórica e o propósito político do neoconservadorismo” como a conversão “do Partido Republicano, e do conservadorismo americano em geral, contra as suas respectivas vontades, num novo tipo de política conservadora adequada para governar uma economia moderna”. Democracia.")
 {\color{blue} 41}  
Allan Bloom era judeu e homossexual. E como ela nunca se cansou de nos lembrar durante a campanha de 2008, Sarah Palin é uma mulher num mundo de homens, uma do Alasca que disse não a Washington (embora, na verdade não o tenha feito), uma dissidente que disparou contra outro dissidente.
 
\par
 
O conservadorismo não dependeu apenas de pessoas de fora; ele também se viu como a voz do de fora. Do grito de Burke de que “a galeria está no lugar da casa” à reclamação de Buckley de que o conservador moderno está “fora do lugar”, o conservador serviu como uma tribuna para os deslocados, seu movimento uma transmissão de suas queixas.
 {\color{blue} 42}  
Longe de ser uma invenção do politicamente correto, a vitimização tem sido um tema de discussão da direita desde que Burke condenou o tratamento dispensado pela multidão a Maria Antonieta. O conservador, sem dúvida, fala em nome de um tipo especial de vítima: aquela que perdeu algo de valor, em oposição aos miseráveis ​​da terra, cuja principal queixa é que nunca tiveram nada a perder. O seu eleitorado é constituído pelos contingentemente despossuídos – o “homem esquecido” de William Graham Sumner – e não pelos sobrenaturalmente oprimidos. Longe de diminuir o seu apelo, este tipo de vitimização confere à queixa conservadora um significado mais universal. Liga a sua deserdação a uma experiência que todos partilhamos – nomeadamente, a perda – e entrelaça os fios dessa experiência numa ideologia que promete que essa perda, ou pelo menos uma parte dela, pode ser reparada.
 
\par
 
As pessoas de esquerda muitas vezes não conseguem perceber isto, mas o conservadorismo realmente fala para e para as pessoas que perderam alguma coisa. Pode ser uma propriedade fundiária ou os privilégios da pele branca, a autoridade inquestionável de um marido ou os direitos irrestritos de um proprietário de fábrica. A perda pode ser tão material quanto o dinheiro ou tão etérea quanto a sensação de posição. Pode ser a perda de algo que nunca foi legitimamente possuído; pode, quando comparado com o que o conservador retém, ser pequeno. Mesmo assim, é uma perda, e nada é tão valorizado como aquilo que já não possuímos. Costumava ser uma das grandes virtudes da esquerda o fato de ser a única a compreender a natureza muitas vezes de soma zero da política, onde os ganhos de uma classe implicam necessariamente as perdas de outra. Mas à medida que esse sentimento de conflito diminui na esquerda, cabe à direita lembrar aos eleitores que existem realmente perdedores na política e que são eles – e só eles – que falam por eles. “Todo o conservadorismo começa com a perda”, observa Andrew Sullivan, com razão, o que faz do conservadorismo não o Partido da Ordem, como Mill e outros afirmaram, mas o partido do perdedor.
 {\color{blue} 43}  

 
\par
 
O principal objectivo do perdedor não é – e, na verdade não pode ser – a preservação ou a proteção. É recuperação e restauração. Acredito que esse seja um dos segredos do sucesso do conservadorismo. Apesar de todo o seu frisson demótico e grandiosidade ideológica, apesar de toda a sua insistência no triunfo e na vontade, no movimento e na mobilização, o conservadorismo pode ser, em última análise, um ar pedestre da AFF. Dado que as suas perdas são recentes – a direita agita contra a reforma em tempo real, e não milênios depois do fato – o conservador pode afirmar de forma credível perante o seu eleitorado, na verdade, perante o sistema político, em geral, que os seus objectivo são práticos e alcançáveis. Ele apenas procura recuperar o que é seu, e que já o teve — na verdade, provavelmente já o teve há algum tempo — sugere que ele é capaz de possuí-lo novamente. “Não é uma estrutura antiga”, declarou Burke sobre a França jacobina, mas “um erro recente”.
 {\color{blue} 44}  
Enquanto o programa de redistribuição da esquerda levanta a questão de saber se os seus beneficiários estão verdadeiramente preparados para exercer os poderes que procuram, o projeto conservador de restauração não sofre tal desafio. Além disso, ao contrário do reformador ou do revolucionário, que enfrenta a tarefa quase impossível de dar poder aos que não têm poder – isto é, de transformar as pessoas daquilo que são naquilo que não são – o conservador limita-se a pedir aos seus seguidores que façam mais daquilo que sempre fizeram. Feito (embora melhor e de forma diferente). Como resultado, a sua contrarrevolução não exigirá a mesma perturbação que a revolução causou no país. “Quatro ou cinco pessoas, talvez”, escreve Maître, “darão um rei à França”.
 {\color{blue} 45}  

 
\par
 
Para alguns, talvez muitos, no movimento conservador, este conhecimento é uma fonte de alívio: o seu sacrifício será pequeno, a sua recompensa grande. Para outros, é uma fonte de decepção. Para este subconjunto de ativistas e militantes, a batalha é toda. Saber que tudo acabará em breve e não exigirá tanto deles é suficiente para desencadear um complexo de desespero: repulsa pela mesquinhez do seu esforço, tristeza pelo desaparecimento do seu inimigo, ansiedade pela reforma antecipada em que se encontram. Foi forçado. Como se queixou Irving Bristol após o fim da Guerra Fria, a derrota da União Soviética e da esquerda em geral “privou” conservadores como ele “de um inimigo” e “na política, ser privado de um inimigo é um assunto muito sério”. . Você tende a ficar relaxado e desanimado. Volte-se para dentro.”
 {\color{blue} 46}  
A depressão assombra o conservadorismo tão certamente quanto a grande riqueza. Mas, mais uma vez, longe de diminuir o apelo do conservadorismo, esta dimensão mais sombria apenas o aumenta. No palco, o conservador enfeita Byronic, avaliando melancolicamente a soma de suas perdas diante de uma plateia de apaixonados e fascinados. Fora do palco e fora da vista, os seus gestores compilam silenciosamente a soma dos seus ganhos.
 
\par
  
 
999999
