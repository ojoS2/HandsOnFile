\chapter{6 Acumulação de Capital}\label{6 Acumulação de Capital}
 \par 
Os capítulos anteriores caracterizaram o capitalismo como um modo de produção. Isto proporciona um quadro no qual a acumulação de capital e o desenvolvimento histórico do capitalismo como modo de produção dominante no mundo podem ser compreendidos. Pois, tendo descoberto as relações de produção específicas do capitalismo, as forças sistémicas por trás da sua criação, operação, reprodução e desenvolvimento podem ser isoladas da massa de fenómenos que ocorrem mais ou menos simultaneamente.
 \par 
Marx dedica grandes secções do Volume {\color{blue}1} de O Capital à tarefa de interpretar a génese do capitalismo britânico e o papel fundamental desempenhado pela compulsão à acumulação. Isto deve constituir uma importante aplicação e confirmação da sua concepção de mudança histórica. Aqui apenas um esboço de seu trabalho pode ser oferecido. Para uma maior profundidade, os interessados ​​devem consultar o próprio Capital para a análise do próprio Marx, e os marxistas posteriores para estudos mais concretos das causas, natureza, timing e localização da primeira e subsequentes transições capitalistas e “revoluções industriais”.
 \par 
\section{Acumulação Primitiva}
 \par 
Uma característica essencial do capitalismo é a existência da força de trabalho como mercadoria. Uma condição necessária para isso é a separação entre o trabalho e a propriedade dos meios de produção. Os trabalhadores dependem de outra pessoa para os fornecer, pois se os trabalhadores tivessem acesso imediato aos meios de produção, o produto do trabalho, e não a capacidade de trabalhar, seria vendido (se a troca de produtos no mercado pudesse persistir em tais circunstâncias). Portanto, do outro lado da moeda deve estar o capitalista com dinheiro para avançar para comprar força de trabalho e os meios para manter a propriedade dos meios de produção. O estabelecimento histórico destas relações sociais de produção a partir das relações feudais na Grã-Bretanha detém a chave para o nascimento do capitalismo.
 \par 
Em qualquer sociedade para além da mais primitiva, haverá poupança da produção actual e produção de factores de produção duradouros para formar meios de produção para o futuro, seja na forma de sementes de milho, gado, armas de caça ou outros instrumentos. Uma das características distintivas do capitalismo é o aumento da taxa de poupança. Marx considerou comum, uma vez estabelecido o capitalismo, que os economistas atribuíssem o crescimento da poupança ao auto-sacrifício de empresários enérgicos, desviando lentamente os lucros crescentes de volta para os seus negócios. (Mais recentemente, a observação de que uma parte demasiado pequena do rendimento nacional é poupada nos países pobres é considerada por muitos economistas do desenvolvimento como uma grande barreira ao desenvolvimento.)
 \par 
Marx despreza a teoria do auto-sacrifício. O capitalismo baseia-se na separação forçada dos trabalhadores dos meios de produção existentes. Na Grã-Bretanha, a evidência histórica mostra que esta separação foi por vezes brutalmente imposta pelos grandes proprietários de terras, pela aristocracia e pelo Estado, em vez de ser o resultado cumulativo da parcimónia individual e da devoção altruísta ao trabalho em pequenas explorações agrícolas e empresas familiares que, muito gradualmente, conseguiram para enriquecer. Implicou a conversão da utilização tradicional (feudal) dos meios de produção e da força de trabalho existentes na sua utilização em unidades de produção capitalistas. Isto não requer, em primeira instância, qualquer acumulação adicional de meios de produção ou mesmo a sua utilização mais eficiente, apenas a sua redistribuição e funcionamento de acordo com novas relações. Uma vez que isto tenha ocorrido, o processo de acumulação competitiva ganha o seu próprio impulso (ver abaixo e os Capítulos {\color{blue}3} e {\color{blue}4}).
 \par 
Dado que a agricultura era, de longe, o sector de produção dominante na era pré-capitalista, tanto em termos de produção como de volume de emprego, este sector era a fonte de uma classe de trabalhadores assalariados “livres”. O segredo da acumulação primitiva ou original de capital residia, então, na expropriação da população agrícola da terra e na destruição do direito ou costume do cultivo individual independente (mesmo que as taxas feudais tivessem de ser pagas). Isto poderia ser realizado individualmente pelos proprietários de terras, respondendo ao crescente imperativo das trocas de mercado. Por exemplo, pode surgir de pressões por dinheiro devido à acumulação de dívidas por parte dos proprietários de terras, ao impacto da inflação secular, ao aumento dos preços da lã em relação aos cereais, à necessidade de menos pessoas trabalharem nos campos, e assim por diante.
 \par 
Quaisquer que fossem suas causas imediatas, essas transformações exigiam o poder do estado para fazer qualquer progresso em um processo violento e violentamente resistido. A intervenção do estado, representando os interesses da classe capitalista emergente, era dupla. Primeiro, os movimentos de cercamento desapropriaram o campesinato do uso comum e individual da terra. A resistência foi feroz, generalizada e brutalmente esmagada. A classe de trabalhadores sem terra foi criada. Segundo, a legislação salarial e os sistemas perversos de "previdência social", culminando na infame Lei dos Pobres de 1834, forçaram longas horas e disciplina industrial aos trabalhadores sem terra. O impacto combinado dessas transformações, ao longo de várias décadas, foi transformar a maioria dos camponeses em assalariados, criando a fonte potencial de mais-valia absoluta.
 \par 
Aqui, a ênfase de Marx está na utilização mutável dos meios de produção existentes, e não na sua acumulação. Não há dúvida de que o progresso técnico e a reorganização da produção contribuíram para o aumento da produção agrícola que alimentaria a indústria e também os trabalhadores industriais. Simultaneamente, mas secundariamente, o progresso técnico também contribuiu para o aumento da produção industrial procurada como factores de produção para a produção agrícola. No entanto, poucos trabalhadores sentiram os benefícios deste aumento da produção e, para aqueles que o fizeram, deve ter diminuído a insignificância face à deterioração das condições de trabalho e à destruição de um modo de vida. Ilustrativo disto é o papel essencial desempenhado pela força física e pelo Estado na criação do proletariado, envolvendo a polícia, o exército, os sistemas fiscais e judiciais, e assim por diante, em vez do bom funcionamento das forças de mercado que expressam as “preferências ' de proprietários de terras, camponeses e trabalhadores assalariados. A origem tumultuada do proletariado contrasta com a maioria das relações de trabalho actuais, onde a estúpida compulsão das necessidades económicas e o seu desenvolvimento através da tradição, da educação, do hábito e de leis firmemente estabelecidas induzem a classe trabalhadora a olhar para as condições do modo de produção capitalista. como evidente e moralmente justificado, bem como inevitável. A força raramente precisa de estar na vanguarda agora (embora esteja disponível se necessária), porque o trabalho está profundamente ligado ao capital, dando a impressão de que sempre foi e sempre deve ser assim.
 \par 
Este relato extremamente breve explica as origens das relações de produção capitalistas. No século XVII, o primeiro movimento de cercamento foi concluído (outro se seguiria no século XVIII), criando uma classe trabalhadora sem terra, bem como uma classe de capitalistas, que apareceram pela primeira vez como agricultores. No século XVIII, a utilização da dívida nacional, o sistema tributário, a política comercial protecionista e a exploração das colónias para acumular riqueza atingiram o seu clímax. A combinação de trabalho e riqueza nas relações capitalistas acompanhou estes processos, com o século XIX a anunciar o ritmo rápido da inovação tecnológica e o crescimento acelerado da sociedade industrial.
 \par 
É bom reconhecer, no entanto, que a criação do capitalismo na Grã-Bretanha foi bastante diferente de qualquer outro lugar. A expropriação forçada da terra do campesinato foi mais extensa do que no resto da Europa, e o seu carácter foi bastante diferente de desenvolvimentos semelhantes noutras partes do mundo. Na Grã-Bretanha, uma proporção maior da população foi transformada em trabalhadores assalariados. Isto foi feito através da criação de um sistema de propriedade fundiária em grande escala, de modo que um número relativamente pequeno de aristocratas passou a deter a grande maioria das terras de propriedade privada. Noutras partes da Europa, bem como no Nordeste dos Estados Unidos, o campesinato, ou sectores dele, revelou-se mais capaz de se defender tomando posse da terra em parcelas mais pequenas, tornando-se assim independente do trabalho assalariado numa proporção muito maior. extensão. A importância destas mudanças persiste até aos dias de hoje, com a agricultura britânica a continuar a ser caracterizada por explorações agrícolas maiores e a população activa britânica contendo muito menos empregados (e trabalhadores independentes) no sector agrícola do que no resto da Europa.
 \par 
Embora a análise de Marx da acumulação primitiva se concentre na Grã-Bretanha e, nessa medida, lide com uma excepção, a sua análise da formação da classe dos trabalhadores assalariados a partir da população agrícola continua a ser um ponto de partida essencial para o estudo das transições capitalistas na maior parte do mundo. o mundo.
 \par 
Embora, para Marx, o elemento crucial na transição para o capitalismo seja a formação de uma classe de trabalhadores assalariados a partir das relações de classe pré-capitalistas, isto deixa em aberto as causas e mecanismos imediatos pelos quais tais transições são alcançadas. Estas são diversas e complexas, abrangendo os diferentes factores na formação dos mercados, tanto antes como depois da transição, desde o papel do Estado na colonização, acesso ao crédito, mercados de exportação, mudanças no direito de propriedade, e assim por diante. Não é de surpreender, então, como já foi observado, que as transições para o capitalismo não só tenham sido variadas em conteúdo e trajetória, mas também tenham sido fortemente debatidas tanto dentro do marxismo como entre o marxismo e outras abordagens.
 \par 
\section{Acumulação Primitiva}
 \par 
Na Grã-Bretanha, o capitalismo ganhou destaque gradualmente, em grande parte através da coincidência de condições económicas favoráveis ​​- a descoberta e acumulação de metais preciosos, e rendas e salários baixos, bem como políticas económicas pró-activas, inspiradas em parte pelo mercantilismo. A génese subsequente do capitalismo industrial foi menos prolongada do que em qualquer outro lugar, desenvolvendo-se a partir de artesãos e corporações e da absorção dos trabalhadores expulsos pela agricultura capitalista. Simultaneamente, o fim do modo de vida amplamente autossuficiente do campesinato criou um mercado interno para o capital industrial. Anteriormente, os camponeses tinham sido geralmente capazes de satisfazer as suas próprias necessidades através do controlo dos meios de produção (especialmente a terra e as ferramentas agrícolas) de acordo com os costumes feudais. Com o advento do capitalismo, os restantes produtores independentes precisavam de dinheiro para comprar sementes, ferramentas e outros instrumentos agrícolas, e para pagar impostos; isso contribuiu para sua transformação em trabalhadores assalariados. Assim, o capital não destrói necessariamente a produção familiar em virtude da sua eficiência superior. Na verdade, a produção familiar persiste até hoje - por exemplo, em pequenas empresas e fábricas exploradoras. Pelo contrário, a produção independente é em grande parte destruída e geralmente subordinada à produção capitalista pelas mudanças sociais associadas à ascensão do capitalismo. O campesinato inglês, por exemplo, foi destruído pela expulsão forçada da terra e pela comercialização de factores de produção e produtos, e não pela concorrência das explorações agrícolas capitalistas.
 \par 
Nas fases iniciais da formação do capital industrial na Grã-Bretanha, os métodos técnicos de produção permaneceram praticamente inalterados. No entanto, os trabalhadores perderam o acesso directo aos meios de produção e aos factores de produção e, portanto, a possibilidade de controlar o seu próprio trabalho e produção. O processo de desapropriação do campesinato, acima descrito, tornou os trabalhadores assalariados “livres” em dois sentidos bastante distintos - livres dos senhores e dos deveres impostos pelo sistema feudal, e livres do acesso directo aos meios de produção. Estes trabalhadores “livres” tinham agora de vender força de trabalho regularmente para poderem adquirir os seus meios de subsistência. A expropriação é uma das principais fontes históricas da classe trabalhadora industrial britânica. A outra fonte principal é a contratação de artesãos independentes para produzir bens por encomenda e, mais tarde, para processar insumos entregues por e pertencentes a um intermediário capitalista (sistema de distribuição). A etapa histórica seguinte foi a reunião destes produtores independentes para trabalhar em “compostos” pertencentes aos capitalistas, as fábricas, inicialmente com tecnologias inalteradas (ver Capítulo {\color{blue}3}).
 \par 
O surgimento do sistema fabril não é simplesmente um desenvolvimento tecnológico. É também um processo de reorganização social que completa a transformação de artesãos independentes e camponeses despossuídos em trabalhadores assalariados. Marx chama isso de subordinação formal (subsunção formal) do trabalho ao capital. Essa escolha de terminologia destaca o fato de que, enquanto o trabalho foi efetivamente submetido ao capital, o próprio processo de trabalho permanece essencialmente inalterado. Nesse caso, a exploração depende principalmente da extração de mais-valia absoluta: a extensão da jornada de trabalho para 12, 14, {\color{blue}16} ou mais horas por dia; o emprego de crianças e a exploração brutal de cada membro da família por salários miseráveis; o desrespeito à segurança no local de trabalho; e a imposição de condições de vida degradantes à classe trabalhadora. Sujeira, doença, a ameaça de fome, pressões da igreja e do estado e a falta de alternativas obrigaram os trabalhadores "livres" "voluntariamente" a assinar o contrato de trabalho e aparecer "espontaneamente" para trabalhar, mesmo nas condições mais terríveis. Esta é a base do mercado de trabalho, uma instituição capitalista fundamental.
 \par 
Apesar das suas origens humildes, o sistema fabril tem implicações profundas para a organização da vida social e individual. Cria novas condições de trabalho e altera os processos de produção e reprodução social de forma irreconhecível. Dentro de cada fábrica, a maquinaria impõe gradualmente a sua própria disciplina, à medida que fragmenta o processo de trabalho em tarefas uniformes e repetitivas, que são mais facilmente monitorizadas pelos agentes do capital: os gestores de linha, os supervisores, os contabilistas, os cronometristas e a sua hierarquia de superiores, cujo próprio desempenho é avaliado pelo conselho de administração e, em última análise, no capitalismo desenvolvido, pelos bancos e acionistas da empresa.
 \par 
Através dos processos de mecanização, fragmentação do trabalho e controlo capitalista, o sistema fabril tende a transformar artesãos independentes e artesãos qualificados em apêndices das máquinas que são pagos para operar - os trabalhadores fabris são guardiões do capital fixo estrangeiro. Marx chama isso de subordinação real do trabalho ao capital. A cooperação detalhada do trabalho dentro da fábrica contrasta fortemente com a divisão mais refinada pelas tarefas dos trabalhadores que acompanha a especialização. A subordinação real do trabalho marca o início da produção capitalista propriamente dita, baseada na extração de mais-valia relativa. Estes são os aríetes económicos com os quais o capitalismo pode derrotar outras formas de produção com base na sua produtividade superior. Simultaneamente, fora da fábrica, as cidades tornam-se centros industriais em rápido crescimento, perturbando todas as relações entre a cidade e o campo, enquanto a própria vida é revolucionada pela difusão de métodos capitalistas de produção em toda a economia e em todo o mundo.
 \par 
\section{Acumulação Primitiva}
 \par 
A concorrência capitalista faz-se sentir através de vários canais. Na esfera da produção, as pressões competitivas levam à subordinação real do trabalho e à extração de mais-valia relativa através da mecanização. Institucionalmente, a mecanização está associada à difusão de sistemas interligados de propriedade e controlo, envolvendo hierarquias complexas de trabalhadores de “colarinho branco”, gestores, executivos, accionistas, o sistema financeiro e o Estado, procurando maximizar a eficiência corporativa independentemente do seu impacto sobre o bem-estar dos trabalhadores. Finalmente, ao nível da troca, as empresas estão imersas na concorrência em vários mercados simultaneamente, incluindo os dos meios de produção, da força de trabalho e dos produtos acabados. A todos os níveis, os capitalistas parecem encontrar-se à mercê de “forças de mercado” anónimas. Estas surgem do imperativo de acumulação do capital em geral, o que determina o comportamento de cada capital individual.
 \par 
Para distinguir entre estes canais de concorrência e explicar as suas consequências, Marx identifica dois tipos distintos de concorrência no capitalismo: a concorrência intersectorial (entre capitais do mesmo ramo da indústria, isto é, que produzem valores de uso idênticos), e a concorrência intersectorial. (entre capitais de ramos diferentes, produzindo valores de uso distintos).
 \par 
A concorrência intrassetorial é examinada no Volume {\color{blue}1} de O Capital. Este tipo de concorrência explica a tendência para a diferenciação das taxas de lucro dos capitais que produzem bens idênticos com tecnologias distintas, as fontes de mudança técnica e a possibilidade de crises de desproporção e superprodução (ver Capítulo {\color{blue}7}). Ao competirem contra outros capitais que produzem mercadorias idênticas, as empresas podem defender a sua quota de mercado e rentabilidade, e evitar a falência, apenas tentando tornar-se mais eficientes do que os seus rivais imediatos - isto é, através da redução dos custos unitários. Isto requer uma disciplina implacável e um controlo extensivo sobre o processo de trabalho, a mecanização e a introdução contínua de tecnologias, máquinas e processos de trabalho mais produtivos, bem como economias de escala (minimização de custos através da produção em grande escala, reduzindo os custos fixos médios).
 \par 
Impostas por imperativos sistêmicos, em vez de por maldade ou inquietação por parte de capitalistas individuais. Essas forças criam uma situação de acumulação competitiva para todos eles; participar é uma condição de sobrevivência. Os concorrentes, portanto, inovarão e adotarão todas as melhorias técnicas disponíveis, erodindo a vantagem das empresas inovadoras, ao mesmo tempo em que preservam os incentivos para mais progresso técnico em toda a economia.
 \par 
Estas convulsões contínuas são
 \par 
Travar esta batalha aumenta a eficiência económica e barateia as mercadorias produzidas em todas as empresas, explorações agrícolas, lojas ou escritórios, incluindo aquelas consumidas pelos trabalhadores (mais-valia relativa). Tende também a fortalecer os grandes capitais, que normalmente são mais capazes de investir somas maiores por períodos mais longos, de seleccionar entre uma gama mais ampla de técnicas de produção e de contratar os melhores trabalhadores. Desta forma, reforçam as suas vantagens iniciais e tendem a destruir os seus concorrentes mais fracos (contra-tendências importantes são a difusão de inovações técnicas entre empresas concorrentes, a capacidade dos capitais mais pequenos para minar as tecnologias existentes através da invenção e da experimentação, e a concorrência estrangeira).
 \par 
O segundo tipo de competição identificado por Marx é a competição intersetorial, entre capitais que produzem diferentes valores de uso. Este tipo de concorrência é examinado no Volume {\color{blue}3} do Capital. Em vez de conduzir à transformação das tecnologias de produção e das práticas de trabalho, explicada acima, a maximização dos lucros pode levar à migração de capital para outros sectores (presumivelmente mais lucrativos). Estes movimentos, em resposta a mudanças estruturais na procura, ao desenvolvimento de novos produtos ou oportunidades de lucro noutros locais, ou simplesmente devido ao reposicionamento de activos a curto prazo no mercado bolsista, alteram a distribuição do capital e do trabalho e o potencial produtivo da economia. Há uma tendência de aumento da oferta nos ramos mais rentáveis, reduzindo os seus lucros excedentes. Uma consequência imediata da concorrência intersectorial é a tendência para as taxas de lucro e os salários serem equalizados à medida que os agentes económicos procuram o máximo valor de troca para as suas mercadorias no mercado. Este tipo de concorrência também transforma a expressão dos valores em preços, à medida que estes se tornam preços de produção (ver Capítulo {\color{blue}10}).
 \par 
Marx argumenta que as forças conflitantes da concorrência dentro e entre setores operam em níveis diferentes, sendo as primeiras mais abstratas e relativamente mais importantes do que as últimas. Isto porque, primeiro, o lucro deve ser produzido antes de poder ser distribuído e equalizado tangencialmente. Em segundo lugar, embora a migração de capitais entre sectores possa aumentar a taxa de lucro dos capitais individuais, o progresso técnico pode aumentar a rentabilidade do capital como um todo. Devido a estes diferentes níveis de complexidade, as forças conflitantes desencadeadas por diferentes tipos de concorrência não podem simplesmente ser somadas na análise de Marx da dinâmica contraditória da acumulação de capital. Pela mesma razão, não pode haver nenhuma presunção de que a soma das implicações de diferentes formas de concorrência possa levar a resultados estáticos; como se, por exemplo, os movimentos contínuos de capital pudessem levar à equalização da taxa de lucro e ao equilíbrio estável de longo prazo, como na economia dominante, ou à concentração implacável de capital, como discutido nas análises do monopólio produzidas pela sociedade alemã do início do século XX. escritores democráticos.
 \par 
Mesmo que tal estado pudesse ser alcançado, seria imediatamente perturbado pela busca inevitável de vantagem competitiva. A concorrência nunca é um processo tranquilo e muitas vezes gera instabilidade e crises económicas. Para Marx, a análise da concorrência oferece a base sobre a qual podem ser compreendidas estruturas e processos mais complexos, influentes a diferentes níveis e em mercados distintos. A acumulação de capital é o resultado da interacção entre estes dois tipos de concorrência, ambos financiados pelo sistema financeiro.
 \par 
A capacidade de um capitalista competir é claramente limitada pelo potencial de acumulação. As fontes de acumulação são duplas. Por um lado, os lucros podem ser reinvestidos, acumulando capital ao longo do tempo. Marx chamou isso de processo de concentração. Por outro lado, um capitalista pode pedir emprestado e fundir-se, reunindo os recursos existentes. Isso Marx chamou de processo de centralização. A concentração é um processo lento, diluído pela herança; mas a centralização, através da alavanca de um mecanismo de crédito altamente desenvolvido e de mercados bolsistas, consegue num piscar de olhos o que a concentração poderia levar muitos anos a conseguir.
 \par 
À medida que o capitalista individual acumula, o que é verdadeiro para cada um é verdadeiro para o capital como um todo. Isso se reflete na acumulação social do capital, na reprodução do capital e suas relações de produção em escala expandida, no aumento do proletariado e no desenvolvimento das forças de produção. Mas a solução do capitalista individual para a competição não é reproduzida em escala social: a acumulação também é realizada por concorrentes, de modo que a própria competição é reproduzida, tanto dentro quanto entre setores. A competição causa acumulação, a acumulação cria competição. Aqueles que ficam para trás no processo de acumulação são destruídos. Primeiro, são os artesãos independentes e outros modos de produção que são varridos pelo avanço da produtividade, produção em massa e a regra de ferro da avaliação de mercado. Mais tarde, o capital se volta contra si mesmo, o grande capital destruindo o pequeno capital à medida que a centralização, o crédito e a concentração acumulam cada vez mais capital em menos mãos. Ao mesmo tempo, pequenos capitais surgem continuamente, muitas vezes introduzindo novas tecnologias que podem transformar o mercado e potencialmente superar capitais mais antigos e muito maiores. Em suma, o capital como valor autoexpansível existe em unidades rivais e separadas, e esse modo de existência desencadeia a competição, que é combatida pela acumulação. A necessidade de acumular é sentida por cada capitalista individual como uma força coercitiva externa. Acumule ou morra: há poucas exceções.
 \par 
\section{Acumulação Primitiva}
 \par 
O estudo de Marx sobre a acumulação primitiva na Grã-Bretanha pode ser encontrado em Marx (1976, pt.{\color{blue}8}). Estudos marxistas notáveis ​​​​sobre a origem histórica do capitalismo em diferentes regiões incluem Jairus Banaji (2010), Robert Brenner (1986, 2007), Terry Byres (1996), Neil Davidson (2010), Vladimir I. Lenin (1972), Michael Perelman ( 2003) e Ellen Meiksins Wood (1991, 2002), e as contribuições em Chris Wickham (2007). A explicação para as origens do capitalismo como transição do feudalismo tem sido altamente controversa, tanto dentro como contra o marxismo. O debate Dobb-Sweezy dizia respeito à importância relativa dos desenvolvimentos dentro da produção feudal e das suas relações de classe (conforme defendido por Dobb) em oposição ao papel externo e desintegrador do comércio (Sweezy), com ênfase correspondente no campo e na cidade, e nos produtores e comerciantes, respectivamente. Os textos-chave deste debate estão incluídos em Rodney Hilton (1976). Esta controvérsia foi levada adiante pelo chamado “debate de Brenner”; ver Trevor Ashton e Charles Philpin (1985). Ver também Stephen Marglin (1974) para a ideia de que a transição para o capitalismo é inicialmente mais sobre a forma como a produção é organizada e governada do que sobre métodos técnicos de produção como tais.
 \par 
Marx explica sua teoria da reprodução e acumulação capitalista em Marx (1976, pt.{\color{blue}7}). A análise da concorrência e da acumulação neste capítulo baseia-se em Ben Fine (1980, capítulos 2, {\color{blue}6}) e Alfredo Saad-Filho (2002, capítulo {\color{blue}5}); ver também Michael Burawoy (1979), Paresh Chattopadhyay (1994, cap.{\color{blue}2}), Diego Guerrero (2003), David Harvey (1999, cap. 4-7) e John Weeks (1985-6, 2010, cap.{\color{blue}6}).