\chapter{A CADA UM DE ACORDO COM SUAS NECESSIDADES: SOBRE SEXO (PARTE II)}\label{A CADA UM DE ACORDO COM SUAS NECESSIDADES: SOBRE SEXO (PARTE II)}
 \par 
» En e eu namoramos na faculdade por um curto período | “A em 1988. Embora ele fosse um veterano e eu uma aluna do primeiro ano, morávamos no mesmo dormitório e compartilhávamos um círculo de amigos. Mas eu tinha apenas dezoito anos e ele iria se formar; nós dois sabíamos que seria algo passageiro. Além disso, desde cedo, também entendemos que nossa amizade era mais valiosa do que nosso breve romance. Nossos interesses compartilhados eram principalmente intelectuais e passávamos muito tempo juntos discutindo livros, música e política. Compartilhamos um amor mútuo por Springsteen e Dylan. Ele me apresentou ao Dire Straits e eu compartilhei minha obsessão pelo U2. Eu revisei seus trabalhos e ele me ensinou o básico da teoria macroeconômica enquanto o ajudava a estudar para um de seus exames. Depois que ele se formou e se mudou para Nova York, abandonei a faculdade e voei para a Europa. Nos tornamos amigos por correspondência regulares (no papel) até que a invenção do e-mail mudou nossa comunicação do analógico para o digital. Quando voltei para a Califórnia para terminar meu BA, falávamos por telefone a cada poucas semanas. Ele ficava emocionado ao ouvir o que eu estava aprendendo na universidade. Acho que ele sempre suspeitou que eu me tornaria algum tipo de acadêmico,
 \par 
127
 \par 
128
 \par 
PARA CADA UMA DE ACORDO COM SUAS NECESSIDADES: SOBRE SEXO (PARTE II) e é provavelmente por isso que nunca pensamos em voltar a ficar juntos.
 \par 
Ken morreu antes de eu terminar meu doutorado, e até hoje sinto falta de suas perguntas persistentes e curiosidade infinita. Por muitos anos depois de 2001, eu me vi querendo ligar para ele para contar sobre um artigo que eu tinha acabado de ler ou para discutir a pesquisa que eu estava fazendo para um dos meus livros. Todo o modelo proposto pela teoria da economia sexual o teria fascinado, e ele teria fornecido inúmeros pontos de dados em apoio a essa visão do namoro heterossexual. Por muito tempo, incomodou (e depois fascinou) Ken que eu não me conformasse com sua ideia do que as mulheres querem. Eu era apenas uma exceção estatística no que lhe dizia respeito. Mas muitos anos depois, enquanto eu me aprofundava na bolsa de estudos sobre a relação entre a independência econômica das mulheres e a sexualidade, eu desejei poder dizer a Ken que sua visão das mulheres era exclusiva do capitalismo. O que ele considerava "natural" era, na verdade apenas um produto de uma maneira particular de organizar a sociedade.
 \par 
Para provar isso, eu teria começado por lhe enviar um estudo de caso da União Soviética que mostrasse que as ideias de Alexandra Kollontai sobre uma moralidade sexual socialista acabaram por se consolidar no mundo socialista no século XX.
 \par 
Décadas de 1970 e 1980. Duas sociólogas russas, Anna Temkina e Elena Zdravomyslova, conduziram entrevistas biográficas aprofundadas com dois grupos de mulheres russas de classe média em 1997 e 2005. Elas examinaram as mudanças geracionais na forma como as mulheres descreviam suas vidas amorosas durante e após a União Soviética. União. A pesquisa dos autores revelou cinco narrativas básicas que as mulheres usaram para discutir suas relações heterossexuais com os homens, o que elas chamam de “roteiros sexuais”: o roteiro pro natalista, o roteiro romântico, o roteiro
 \par 
KRISTEN R. GHODSEE roteiro de amizade, o roteiro hedonista e o roteiro instrumental. Em suas entrevistas de 1997, os pesquisadores russos descobriram que a “geração silenciosa” soviética (aqueles nascidos entre 1920 e 1945), principalmente relacionada ao roteiro pró-natalista, significando que sexo era algo que você suportava no casamento para ter filhos. Amor e prazer não tinham nada a ver com isso. E embora as mulheres soviéticas tivessem acesso ao aborto novamente depois de 1955, a falta de controle de natalidade e a dupla carga de trabalho e responsabilidades familiares conspiraram para deprimir a função sexual em muitas mulheres. Não há dúvida sobre isso: para esta geração, o sexo soviético era uma droga."
 \par 
Mas as coisas começaram a mudar após a morte de Stalin. Apesar da contínua falta de privacidade devido à escassez de moradia, à escassez de educação sexual oficial e à completa ausência de erotismo (toda pornografia era proibida), Temkina e Zdravomyslova descobriram que mulheres urbanas de classe média nascidas entre 1945 e 1965 descreveram um afastamento acentuado do roteiro pró-natalista. Embora a visão pró-natalista das relações sexuais continuasse, ela foi complementada por duas novas maneiras de falar sobre sexualidade: romance e amizade. O surgimento do roteiro romântico foi o resultado de uma mudança maior nas narrativas públicas soviéticas sobre sexualidade. No final da era soviética, médicos, psicólogos e outros especialistas começaram a enfatizar o papel do "amor verdadeiro", "interesses comuns" e "unidade espiritual" como base para um casamento bem-sucedido. "O roteiro romântico implica que a vida sexual é interpretada como parte integrante de fortes emoções e sentimentos", escrevem os pesquisadores russos. "O sexo é descrito como um atributo do amor, romance e paixão. O amor é a categoria central na narrativa da experiência sexual." Este roteiro romântico da sexualidade é exatamente o que os primeiros socialistas, como
 \par 
129
 \par 
130
 \par 
A CADA UM CONFORME AS SUAS NECESSIDADES: SOBRE O SEXO (PARTE I!) como August Bebel e Alexandra Kollontai teriam imaginado para uma sociedade em que as considerações econômicas tivessem menos influência na escolha de um parceiro amoroso.’
 \par 
A outra maneira de descrever o sexo que começou a surgir entre as mulheres de classe média no final do período soviético é o roteiro da amizade. Ao contrário do que chamaríamos de “amigas com benefícios” — sexo recreativo e sem compromisso com um parceiro do sexo oposto — o roteiro da amizade soviética descrevia o sexo que ocorria em um relacionamento significativo entre duas pessoas que trabalhavam juntas ou compartilhavam um círculo social, com os parceiros usando o sexo como uma forma de mostrar afeição e respeito reciprocamente. Esse roteiro de amizade provavelmente surgiu porque as mulheres tinham acesso aos seus próprios recursos e não dependiam dos homens para suprir suas necessidades materiais. Como algumas mulheres soviéticas urbanas se sentiam seguras em sua posição econômica, a sexualidade perdeu seu valor de troca e se tornou algo a ser compartilhado.
 \par 
Se a teoria da economia sexual estiver no caminho certo, seria de supor que a introdução de mercados livres e o rápido desmantelamento do Estado de bem-estar social após o colapso da URSS precipitariam o regresso de uma visão de mundo em que a sexualidade das mulheres é mais uma vez uma mercadoria. . E foi exatamente isto que Temkina e Zdravomyslova descobriram nas suas entrevistas de 1997 e 2005 com mulheres da geração pós-soviética. Além do “roteiro hedonista”, em que o sexo é puramente físico com o propósito de experimentar o prazer individual, muitas vezes auxiliado por brinquedos sexuais e outros produtos que podem ser adquiridos numa economia capitalista (um roteiro ausente, por razões óbvias, no era soviética), eles notam o surgimento de algo que chamaram de “roteiro instrumental”, que se tornou onipresente após o advento
 \par 
KRISTEN R. GHODSEE de mercados livres. “A comercialização de diferentes esferas da vida social, a polarização de gênero e a desigualdade, bem como a falta de recursos legitimam o roteiro instrumental da sexualidade”, escrevem Temkina e Zdravomyslova. “Esse roteiro pressupunha que a feminilidade sexualizada (bem como a tenra idade) poderia ser trocada lucrativamente por benefícios materiais e outros. Nesse roteiro, o casamento é representado como um cálculo.” A mercantilização da sexualidade feminina na Rússia pode ser observada no aumento dramático do trabalho sexual, pornografia, casamentos estratégicos por dinheiro e o que os autores chamam de “patrocínio”, pelo qual homens ricos patrocinam suas amantes. De acordo com Temkina e Zdravomyslova, esse roteiro instrumental era “muito raramente encontrado nas narrativas da vida sexual” das mulheres mais velhas que cresceram na União Soviética.*
 \par 
Evidências da prevalência pós-1991 desse roteiro instrumental também podem ser encontradas na exposição de Peter Pomerantsev de 2014 sobre o crescimento explosivo das academias russas de “caçadoras de ouro”. Ao observar uma aula nessa forma especial de instituição educacional em Moscou, ele descreveu “um grupo de garotas loiras sérias tomando notas cuidadosas” porque “encontrar um sugar caddy é um ofício, uma profissão”. As aspirantes a caçadoras de ouro pagam mil dólares por semana por esses cursos na esperança de encontrar um “patrocinador” para pagar suas contas. Para muitas mulheres jovens, treinar-se para encontrar um marido rico é um investimento melhor do que uma educação universitária ou seguir uma carreira. Depois que se formam nessas academias, Pomerantsev explica que essas mulheres espreitam “uma constelação de clubes e restaurantes projetados quase exclusivamente para o propósito de patrocinadores procurando garotas e garotas procurando patrocinadores. Os caras são conhecidos como ‘Forbeses’ (como
 \par 
131
 \par 
132
 \par 
PARA CADA UM DE ACORDO COM SUAS NECESSIDADES: SOBRE SEXO (PARTE I!) na lista dos ricos da Forbes); as meninas como ‘kit de ferramentas, gado. É um mercado de compradores: há dezenas, não, centenas, de ‘gado’ para cada ‘Forbes.’ Assim, a reintrodução de mercados livres na Rússia coincidiu com um retorno à mercantilização das mulheres, particularmente quando comparado com o passado soviético tardio.”
 \par 
O choque entre a visão socialista da sexualidade livre e a ideia capitalista da sexualidade mercantilizada também pode ser observado nas discussões e debates em torno da reunificação das duas Alemanhas — a República Democrática Alemã (RDA) e a República Federal da Alemanha (RFA). Até o fim da Segunda Guerra Mundial, a Alemanha era uma nação, mas após a derrota dos nazistas, os Aliados vitoriosos dividiram a Alemanha entre si. Como a Guerra Fria
 \par 
A guerra começou, a aliança entre Stalin e as potências ocidentais se quebrou. A Alemanha Oriental caiu no lado soviético da Cortina de Ferro sob o governo uni partidário do Partido da Unidade Socialista (SED).
 \par 
A divisão da Alemanha apresenta um interessante experimento natural em direitos e sexualidade das mulheres. As populações dos dois países eram quase idênticas em todos os aspectos, exceto pela divergência em seus sistemas políticos e econômicos. Por quatro décadas, as duas Alemanhas seguiram caminhos diferentes, particularmente no que diz respeito à construção de masculinidades e feminilidade ideais. Os alemães ocidentais abraçaram o capitalismo, os papéis tradicionais de gênero e o modelo de ganha-pão/dona de casa do casamento monogâmico burguês. No leste, o objetivo da emancipação das mulheres combinado com a escassez de mão de obra levou a uma
 \par 
KRISTEN R. GHODSEE mobilização massiva de mulheres para a força de trabalho. Como a historiadora Dagmar Herzog argumentou em seu livro de 2005, Sex After Fascism, o estado da Alemanha Oriental promoveu ativamente a igualdade de gênero e a independência econômica das mulheres como características únicas do socialismo, tentando demonstrar sua superioridade moral sobre o Ocidente capitalista democrático. Já na década de 1950, publicações estatais encorajaram os homens da Alemanha Oriental a participar do trabalho doméstico, dividindo o fardo do cuidado dos filhos de forma mais equitativa com suas esposas, que também eram empregadas em tempo integral.®
 \par 
Conforme a professora de estudos culturais alemães Ingrid Sharp, os alemães orientais criaram uma situação em que as mulheres não eram mais dependentes dos homens, dando a elas uma sensação de autonomia que encorajava um comportamento masculino mais generoso no quarto. Se as namoradas e esposas da Alemanha Ocidental estivessem descontentes com o desempenho sexual de seus parceiros homens, elas tinham poucas opções abertas para elas. Como as mulheres dependiam dos homens para apoiá-las financeiramente, na melhor das hipóteses elas poderiam gentilmente tentar cutucar seus parceiros para serem mais atenciosos às suas necessidades. No Leste, os homens que desejavam relações sexuais com mulheres não podiam depender de dinheiro para comprar acesso a eles, e tinham incentivos para melhorar seu comportamento. Sharp explica: "O divórcio na RDA era relativamente simples e tinha poucas consequências financeiras ou sociais para qualquer um dos parceiros. As taxas de casamento e divórcio eram muito mais altas do que no Ocidente. O SED argumentou que esses números refletiam um desejo benéfico por casamentos baseados no amor; relacionamentos obsoletos e insatisfatórios podiam ser facilmente dissolvidos e os produtivos facilmente iniciados. O fato de as mulheres instigarem a maioria dos processos de divórcio foi anunciado
 \par 
133
 \par 
134
 \par 
A CADA UM SEGUNDO SUAS NECESSIDADES: SOBRE SEXO (PARTE I!) como um sinal de sua emancipação. Ao contrário do Ocidente, as mulheres não eram forçadas pela dependência econômica a permanecer em casamentos dos quais não mais desfrutavam.””
 \par 
A independência econômica das mulheres e o declínio concomitante em relacionamentos baseados em troca econômica alimentaram as alegações da Alemanha Oriental de que os socialistas desfrutavam de vidas pessoais mais satisfatórias. Mas em vez de se concentrarem apenas no amor, como Kollontai teria feito, os pesquisadores da Alemanha Oriental se esforçaram para demonstrar que seus compatriotas tinham relações sexuais mais frequentes e satisfatórias. Eles argumentaram que o sistema socialista melhorou a vida sexual das pessoas precisamente porque o sexo não era mais uma mercadoria a ser comprada e vendida no mercado aberto. Herzog observa: “A principal preocupação no Leste era mostrar aos cidadãos que o socialismo fornecia as melhores condições para felicidade e amor duradouros. (Na verdade, autores orientais frequentemente apontavam que os relacionamentos sexuais eram realmente mais baseados no amor e, portanto, honrosos no Leste do que no Oeste, especificamente porque sob o socialismo as mulheres não precisavam se ‘vender’ para o casamento para se sustentar.)”®
 \par 
Como os investigadores da Alemanha Oriental se concentraram na satisfação sexual, e especialmente na satisfação sexual feminina, conduziram uma ampla variedade de estudos empíricos para tentar demonstrar a superioridade do socialismo no quarto. Tendo em mente os desafios metodológicos discutidos no capítulo anterior, estes estudos fornecem algumas evidências interessantes de que as pessoas tinham sexo melhor sob o socialismo. Por exemplo, em 1984, Kurt Starke e Walter Friedrich publicaram um livro com os resultados das suas pesquisas sobre o amor e a sexualidade entre os alemães orientais com menos de trinta anos. Os autores descobriram que os jovens da RDA, tanto homens como mulheres, eram
 \par 
KRISTEN R. GHODSEE estavam altamente satisfeitas com suas vidas sexuais, e dois terços das jovens mulheres relataram que atingiram o orgasmo "quase sempre", com mais 18% dizendo que o fizeram "frequentemente". Starke e Friedrich alegaram que esses níveis de satisfação pessoal no quarto resultaram da vida socialista: "a sensação de segurança social, responsabilidades educacionais e profissionais iguais, direitos e possibilidades iguais para participar e determinar a vida da sociedade".
 \par 
Estudos subsequentes corroborariam esses primeiros resultados. Em 1988, Kurt Starke e Ulrich Clement conduziram o primeiro estudo comparativo das experiências sexuais autorrelatadas de estudantes do sexo feminino da Alemanha Oriental e Ocidental. Eles descobriram que as mulheres da Alemanha Oriental disseram que gostavam mais de sexo e relataram uma taxa maior de orgasmo do que suas contrapartes ocidentais. Em 1990, outro estudo comparando as atitudes sexuais dos jovens nas duas Alemanhas descobriu que as preferências dos homens e mulheres da RDA estavam mais em sincronia entre si do que as dos homens e mulheres jovens no Ocidente. Por exemplo, uma pesquisa descobriu que 73% das mulheres da Alemanha Oriental e 74% dos homens da Alemanha Oriental queriam se casar. Em contraste, 71% das mulheres no Ocidente desejavam o casamento, mas apenas 57% dos homens ocidentais o faziam, uma diferença de quatorze pontos. Uma pesquisa diferente sobre experiências sexuais revelou níveis muito mais altos de prazer sensual autorrelatado entre as mulheres da Alemanha Oriental. Quando perguntados se seu último encontro os deixou satisfeitos, 75% das mulheres da RDA e 74% dos homens da RDA disseram que sim, em comparação com 84% dos homens da RFA e apenas 46% das mulheres da RFA. Finalmente, os entrevistados foram solicitados a relatar se se sentiram “felizes” após o sexo. Entre
 \par 
135
 \par 
136
 \par 
PARA CADA UMA DE ACORDO COM SUAS NECESSIDADES: SOBRE SEXO (PARTE I!) das mulheres da Alemanha Oriental, {\color{blue}82} por cento concordaram, enquanto entre as mulheres da Alemanha Ocidental, apenas {\color{blue}52} por cento relataram se sentir “felizes”. Para reverter essa estatística, apenas {\color{blue}18} por cento das mulheres da RDA não estavam “felizes” depois do sexo, em comparação com quase metade das mulheres pesquisadas na RFA.”
 \par 
\begin{figure}
	\centering
	\includegraphics[width=1.\textwidth]{temp\_files/images/UP\_logo.png }
	\caption{Quando a RFA e a RDA se unificaram sob a constituição da Alemanha Ocidental em 1990, as diferentes culturas sexuais das duas sociedades colidiram e se tornaram o assunto de muitos debates e mal-entendidos em andamento. Ingrid Sharp também estudou a “unificação sexual da Alemanha” e argumentou que os homens ocidentais inicialmente fetichizaram a ideia da apaixonada mulher da Alemanha Oriental. “Estatísticas concretas”, escreve Sharp (sem trocadilhos), “aparentemente confirmaram a maior responsividade sexual das mulheres orientais. Uma pesquisa sobre práticas sexuais femininas conduzida pelo Gewis-Institut, Hamburgo, para a Neue Revue relatou que 80% das mulheres orientais sempre tiveram orgasmo, em comparação com 63% das mulheres no Ocidente... O contexto [deste estudo] foi a batalha ideológica entre o Oriente e o Ocidente, a guerra fria sendo travada na arena da sexualidade, com o potencial orgástico substituindo a capacidade nuclear.” De fato, Sharp relata que as alegações contínuas de sexólogos orientais de que o maior prazer sensual das mulheres da RDA estava ligado à independência econômica e à autoconfiança das mulheres ameaçavam o senso de superioridade dos alemães ocidentais. A mídia da Alemanha Ocidental atacou a ideia de que qualquer coisa no Leste poderia ter sido melhor, lançando o que Sharp chamou de "A Grande Guerra do Orgasmo".!!}
	\label{ }
\end{figure}
 \par 

 \par 
Os debates em curso sobre a satisfação sexual comparativa dos alemães orientais e ocidentais inspiraram os historiadores Paul Betts e Josie McLellan a explorar mais o tópico,
 \par 
KRISTEN R. GHODSEE com o livro de 2011 deste último, Love in lhe Time of. Communism, fornecendo uma reflexão de {\color{blue}239} páginas sobre o assunto. Betts e McLellan confirmam a ideia de que a independência econômica feminina contribuiu para uma forma única, não modificada, talvez mais "natural" e "livre" de sexualidade que floresceu no Oriente, conferindo credibilidade à ideia de que a teoria da economia sexual fornece uma boa descrição dos mercados sexuais, mas apenas aqueles em sociedades capitalistas. No entanto, como Betts e McLellan observam, também houve outros fatores que contribuíram para as diferenças nas culturas sexuais. Em primeiro lugar, a igreja desempenhou um papel muito mais forte na regulamentação da moralidade e da sexualidade no Ocidente do que no Oriente secular e ateu (embora seja importante notar que o estudo de 1984 de Starke e Friedrich não encontrou nenhuma diferença entre ateus e aqueles que professavam afiliação religiosa em suas respostas). No entanto, a cultura da Alemanha Ocidental abraçou certamente os papéis tradicionais de gênero das igrejas protestante e católica em uma extensão muito maior do que a do Leste. Segundo, a natureza autoritária do regime da RDA excluiu a esfera pública dos alemães orientais, e eles responderam recuando para a esfera privada, onde construíram vidas privadas aconchegantes e não ideológicas como uma forma de encontrar refúgio do estado onipresente. Terceiro, havia menos o que fazer no Leste em comparação com as muitas distrações comerciais disponíveis no Oeste, então as pessoas provavelmente tinham mais tempo para sexo. E, finalmente, o regime da Alemanha Oriental encorajou as pessoas a aproveitarem suas vidas sexuais como uma forma de distraí-las da monotonia e da privação relativa da economia socialista e das restrições de viagem.” Além disso, assim como com Kollontai, a ideia de sexo da Alemanha Oriental permaneceu conservadora quando comparada à nossa
 \par 
137
 \par 
138
 \par 
PARA CADA UM DE ACORDO COM SUAS NECESSIDADES: SOBRE SEXO (PARTE II) padrões modernos. Gays e lésbicas, embora não abertamente perseguidos, viviam vidas circunscritas, confinadas à esfera privada. E por mais que o estado tentasse convencer os homens a ajudar em casa, as mulheres da Alemanha Oriental ainda realizavam a maioria do trabalho doméstico. Apesar da disponibilidade de controle de natalidade e aborto, a RDA, como todos os outros estados socialistas, ainda era fortemente pró-natalidade em sua perspectiva; ter filhos era considerado um dever das mulheres da Alemanha Oriental, e os socialistas tendiam a ver o sexo como algo que eventualmente levaria ao casamento e aos filhos. Finalmente, mesmo que quisessem que o sexo fosse prazeroso para homens e mulheres, o estado nunca foi a favor da promiscuidade desenfreada ou do sexo "hedonista". O sexo deveria ser uma expressão de amor e afeição entre camaradas iguais.
 \par 
Apesar dessas ressalvas importantes, muitos alemães orientais acreditavam que sua sexualidade pré-1989 era mais espontânea, natural e alegre em comparação à sexualidade comercializada e instrumentalizada que encontraram quando se juntaram à Alemanha Ocidental. Em vez de tentar preservar os melhores aspectos de ambos os sistemas enquanto descartava as partes ruins, a reunificação alemã levou ao apagamento do modo de vida da Alemanha Oriental, incluindo o apoio à independência econômica das mulheres. A introdução de mercados capitalistas também significou uma reavaliação radical do valor humano. “Sem dúvida, o mais devastador para o antigo Leste foi a perda de segurança econômica e a nova ideia de que o valor humano agora seria medido principalmente pelo dinheiro”, escreve Herzog. “Os cidadãos da Alemanha Oriental sentiam enormes ansiedades sobre a perda de empregos. E previdência social, aumento de aluguéis e futuros incertos. . Ao longo da década de 1990, e repetidamente, os orientais (gays e heterossexuais) articularam a convicção de que o sexo no
 \par 

 \par 
KRISTEN R. GHODSEE
 \par 
O sexo oriental era mais genuíno e amoroso, mais sensual e mais gratificante — e menos baseado no envolvimento próprio — do que o sexo da Alemanha Ocidental.”
 \par 
A Hungria próxima apresenta outro estudo de caso para nos ajudar a pensar sobre como o socialismo de estado moldou a moralidade sexual. A socióloga húngara Judit Takacs explorou as vidas íntimas de seus compatriotas antes de 1989 e sugere que suas vidas sexuais floresceram mesmo sob circunstâncias repressivas. Escrevendo retrospectivamente em 2014, Takacs propôs que, embora os húngaros sofressem com a falta de espaço privado como resultado da escassez de moradia e vivessem sob vigilância constante quando estavam em público, "eles pareciam ser capazes de negociar suas vidas entre as restrições do socialismo de estado e seu anseio por relacionamentos agradáveis ​​com parceiros de um gênero diferente e/ou do mesmo gênero". Em outras palavras, como na Alemanha Oriental e na União Soviética, havia uma disjunção considerável entre a vida privada e a esfera pública na Hungria, mas a independência econômica das mulheres contribuiu para uma cultura na qual o sexo era algo a ser compartilhado em vez de vendido. Além disso, embora os húngaros nunca tenham conseguido redefinir os papéis tradicionais de gênero, e o patriarcado doméstico tenha sido fortalecido por políticas familiares pró-natalistas, os húngaros mais jovens parecem compartilhar a mesma aversão à comercialização da sexualidade que os alemães orientais. Em um estudo sociológico conduzido no início da década de 1970, pesquisadores pesquisaram as atitudes sexuais de {\color{blue}250} jovens estudantes e trabalhadores entre dezoito e vinte e quatro anos. Jovens húngaros leram oito histórias sobre práticas sexuais
 \par 
139
 \par 
140
 \par 
PARA CADA UM DE ACORDO COM SUAS NECESSIDADES: SOBRE SEXO (PARTE II) que eram consideradas comuns em seu país e então as classificavam com base em se gostavam ou não dos protagonistas. Essas oito histórias incluíam (1) uma virgem que quer esperar até o casamento para fazer sexo, (2) uma mulher "semi-virgem" que se envolve com homens, mas não chega ao coito real, (3) uma mãe solteira abandonada por seu parceiro sexual depois que engravidou de seu filho, (4) uma prostituta que conhece homens aleatórios em bares e faz sexo por dinheiro, (5) um solteiro "mulherengo" que faz sexo com o máximo de mulheres possível, (6) um homem gay que tem relações discretas com homens, (7) um homem que satisfaz suas necessidades sexuais por meio de masturbação repetida e (8) um ​​jovem casal que se apaixona e faz sexo antes do casamento.
 \par 
Entre a grande maioria dos estudantes pesquisados, o casal solteiro, mas amoroso, foi classificado como o mais simpático (embora as trabalhadoras tenham classificado a mãe solteira um pouco mais alto do que o casal). A maioria dos estudantes pesquisados ​​também classificou a prostituta como o personagem menos simpático; ela era a mais odiada entre os estudantes e trabalhadoras e entre as trabalhadoras. Apenas os trabalhadores do sexo masculino acharam o homem gay menos simpático. Também no final da lista estavam o "mulherengo", o "semi-virgem" (provocador) e o crônico "autossatisfeito". A virgem estava em algum lugar no meio. Particularmente fascinantes, em relação à teoria da economia sexual, são as razões dadas para a retumbante desaprovação do personagem prostituta. Os entrevistados acreditavam que a prostituta não tinha razões legítimas para vender suas afeições, já que o estado socialista atendia às suas necessidades básicas. Eles também se preocupavam que o "sexo sem emoção" seria ruim para seu desenvolvimento pessoal. Curiosamente, o
 \par 
KRISTEN R. GHODSEE estudantes homens e mulheres eram mais simpáticos ao homem gay, e as estudantes mulheres, na verdade classificavam o "mulherengo" abaixo do homem gay, sugerindo que sua aversão à promiscuidade (tanto para homens quanto para mulheres) era maior do que sua homofobia do início dos anos 1970. A sexualidade socialista na Hungria (pelo menos entre esse grupo de homens e mulheres do século XVIII e vinte e Fourier) idealizava relacionamentos amorosos baseados em afeição mútua, assim como Kollontai sugeriu que fariam uma vez que os incentivos de mercado para "vender carícias" fossem superados.
 \par 
As atitudes desses estudantes em relação ao casamento, prostituição e maternidade solteira são confirmadas por dados mais amplos da opinião pública da primeira onda da Pesquisa de Valores Mundiais (1981-1984). Por exemplo, quando perguntados se o casamento era uma "instituição ultrapassada", 16% dos húngaros concordaram, em comparação com apenas 8% dos americanos. Na mesma pesquisa, pesquisadores perguntaram aos entrevistados na Hungria e nos Estados Unidos: "Se uma mulher quer ter um filho como mãe solteira, mas não quer ter um relacionamento estável com um homem, você aprova ou desaprova?" Apenas 8% dos húngaros disseram que "desaprovavam", em comparação com 56% dos americanos, demonstrando uma atitude muito mais liberal em relação às mães solteiras e à independência das mulheres no país socialista estatal. Além disso, enquanto 63% dos americanos relataram que a prostituição "nunca é justificável", 80% dos húngaros pesquisados ​​disseram o mesmo. Uma lacuna ainda maior aparece quando os dados para esta questão são desagregados por gênero: apenas {\color{blue}55} por cento dos homens americanos alegaram que a prostituição “nunca foi justificável”, em comparação com {\color{blue}76} por cento dos homens húngaros. Estes últimos eram talvez mais avessos à prostituição porque tinham
 \par 
141
 \par 
142
 \par 
PARA CADA UM DE ACORDO COM SUAS NECESSIDADES: SOBRE SEXO (PARTE II) fui criada em uma sociedade que se esforçava para separar sexo e romance da troca econômica.”
 \par 
Ao norte, a situação na Polônia católica nos permite considerar mais profundamente o papel da religião na formação dos comportamentos sexuais humanos. Devido à influência contínua da igreja, os poloneses fizeram pouco para desafiar os papéis tradicionais de gênero e, de fato, os sexólogos da era socialista tendiam a reforçar, em vez de minar, os ideais pré-socialistas de masculinidade e feminilidade (ao contrário da Alemanha Oriental). No entanto, as mulheres foram totalmente incorporadas à força de trabalho, e a organização feminina estatal polonesa garantiu que o aborto permanecesse legal e acessível após 1956 e que os jovens poloneses recebessem educação sexual nas escolas após 1969 (embora houvesse publicações relevantes circulando antes disso). Apesar de sua relativa independência, as responsabilidades domésticas em casa levaram a um fardo duplo que nem os parceiros masculinos, nem o Partido Comunista fizeram muito para aliviar. As mulheres também ganhavam significativamente menos do que os homens e, devido a seus deveres familiares, tinham menos oportunidades de progressão na carreira, tornando-as mais dependentes do que em outros países socialistas estatais. “No entanto”, escreve Agnieszka Koscianska, uma antropóloga polonesa, “o acesso ao trabalho assalariado, com o dinheiro ganho, bem como as redes sociais e a vida social construída através do local de trabalho, deu às mulheres independência e poder em relação aos homens, e muitas famílias lutaram com esse novo modelo de relações de gênero”. Por causa desses novos desafios ao ideal tradicional de relacionamentos heterossexuais poloneses, o estado socialista comprometeu recursos para o estudo científico da intimidade. Acadêmicos
 \par 
KRISTEN R. GHODSEE, que escreve no campo dos estudos da sexualidade, baseia-se fortemente no trabalho do teórico francês Michel Foucault e sua investigação sobre como o conhecimento médico especializado afeta nossas experiências subjetivas individuais de saúde e doença. Quando pensamos sobre sexo, por exemplo, a maneira como nos sentimos sobre isso será fortemente influenciada por valores religiosos e normas sociais, mas nossa compreensão sobre se nossa sexualidade é saudável ou "boa" também será moldada pelo que médicos e psicólogos consideram "normal" e "anormal". Assim, por exemplo, um jovem gay que cresce em uma cultura onde os médicos afirmam que a homossexualidade é uma doença a ser curada vai vivenciar sua sexualidade de forma diferente de um jovem que cresce em uma sociedade onde os médicos consideram a homossexualidade normal e saudável. Da mesma forma, as compreensões médicas e psicológicas sobre o que constitui um bom sexo para homens e mulheres vão influenciar a maneira como as pessoas julgam a qualidade de suas próprias vidas sexuais. Quando especialistas dizem que a falta de prazer feminino em relacionamentos heterossexuais não é “normal”, as mulheres podem se tornar melhores defensoras de suas próprias necessidades, apoiadas como estão pelas opiniões autoritárias do meio médico.
 \par 
Para explorar essas questões, Kosciariska pesquisou os conselhos de especialistas dados por sexólogos poloneses durante e após a era do socialismo estatal, e descobriu que as décadas de 1970 e 1980 foram uma espécie de “era de ouro” no que diz respeito à compreensão da sexualidade humana. As visões polonesas contrastavam com os modelos conceituais americanos tradicionais, que se concentravam na fisiologia e propunham que o “bom sexo” era o resultado de um ciclo universal de resposta sexual de quatro estágios. Com base nos experimentos de laboratório de William Masters e Virginia Johnson, essa visão biológica levou, em última análise, à radicalização e
 \par 
143
 \par 
144
 \par 
A CADA UM DE ACORDO COM SUAS NECESSIDADES: SOBRE SEXO (PARTE I!) farmaceuticarização de opções de tratamento para disfunção sexual. As empresas farmacêuticas buscaram (e continuam buscando) comercializar soluções para problemas sexuais, preferencialmente na forma de uma pílula patenteável, o que limita o escopo da pesquisa sexológica a encontrar curas que possam gerar lucros.'°
 \par 
Alternativamente, na Polônia socialista estatal, a sexologia desenvolveu-se numa “disciplina holística que abrange as realizações de vários ramos da medicina, das ciências sociais e das humanidades, com a psicologia, a sociologia, a antropologia, a filosofia, a história, os estudos religiosos e até mesmo a teologia fornecendo recursos para a educação sexual”. E terapia. A sexualidade era percebida como multidimensional e incorporada nos relacionamentos, na cultura, na economia e na sociedade em geral.” Ao contrário da maioria dos seus homólogos ocidentais, os terapeutas sexuais polacos da era socialista exploraram os desejos individuais de amor, intimidade e significado, e ouviram atentamente os sonhos e frustrações dos seus pacientes. O Estado socialista financiou os seus salários e orçamentos de investigação, num forte contraste com o domínio do financiamento empresarial no Ocidente. Isto teve impactos particularmente positivos na compreensão local da sexualidade das mulheres. Segundo Koécianiska, os sexólogos polacos “não limitaram o sexo às experiências corporais e sublinharam a importância dos contextos sociais e culturais para o prazer sexual. Mesmo o melhor estímulo - argumentaram - não ajudará a alcançar o prazer se uma mulher estiver estressada ou sobrecarregada, [ou] preocupada com seu futuro e estabilidade financeira.” Semelhante à linha seguida pelos alemães orientais, o sexo socialista era supostamente melhor porque as mulheres gozavam de maior segurança econômica e porque o sexo era menos mercantilizado do que no capitalismo.
 \par 
\[KRISTEN R. GHODSEE\]
 \par 
Oeste. E como os homens não estavam pagando por isso, eles talvez se importassem mais com o prazer de suas parceiras.
 \par 
Após o colapso do socialismo de estado, a Polônia experimentou um rápido ressurgimento de papéis de gênero conservadores, com liberdades reprodutivas antes garantidas rescindidas e uma reversão de muitas das conquistas do socialismo de estado com relação aos direitos das mulheres. A ascensão do nacionalismo na Polônia também anunciou um aumento na homofobia, xenofobia e antissemitismo. Mas, curiosamente, permanece um legado da visão mais holística da sexualidade desenvolvida durante as décadas de 1970 e 1980. Embora o campo da sexologia tenha sido forçado a lidar com as mesmas pressões de mercado prevalentes no Ocidente, a pesquisa sugere que as mulheres polonesas ainda relatam níveis mais altos de satisfação sexual do que as mulheres nos Estados Unidos. Koscianska cita um estudo de 2012 que descobriu que três quartos das mulheres polonesas estavam livres de "disfunção sexual" e contrasta isso com um estudo de 1999 que descobriu que apenas 55% das mulheres americanas poderiam dizer o mesmo.
 \par 
Mais uma vez, não podemos generalizar sobre as experiências de todos os países socialistas de estado na Europa Oriental antes de 1989. Cada um deles abordou a questão da mulher de forma única, mesmo que todos tenham começado de uma base teórica semelhante nas obras de Bebel, Engels ou Kollontai. Na minha opinião, o pior lugar para ser mulher era a Romênia, onde o socialismo de estado fez pouco para desafiar uma cultura despótica e patriarcal. Como a Romênia, a Albânia parece ter sido um clima bastante inóspito para relações íntimas. A Bulgária era um pouco mais pudica do que a Alemanha Oriental, mas a revista feminina estatal publicava regularmente uma coluna sobre sexologia.
 \par 
145
 \par 
146
 \par 
A CADA UM SEGUNDO SUAS NECESSIDADES: SOBRE SEXO (PARTE II)
 \par 
Em 1979, o governo também facilitou a publicação e ampla distribuição de um dos manuais sexuais mais populares da Alemanha Oriental, The Man ano Woman Intimately, de Siegfried Schnabl. Embora a linguagem fosse medica lizada e Schnabl fosse menos do que consideraríamos esclarecido sobre a homossexualidade e a masturbação, a edição búlgara que tenho abre com algumas estatísticas sobre a experiência feminina do orgasmo na RDA e inclui diagramas anatômicos de onde o clitóris está localizado. E como fica em vários estágios de excitação. Conforme o romancista búlgaro Georgi Gospodinov, o livro foi um grande best-seller e poucos lares búlgaros não tinham um exemplar escondido atrás dos volumes na estante mais alta. Em comparação com os seus vizinhos romenos do norte, as mulheres búlgaras gozavam de maior acesso ao controlo da natalidade e a sexualidade era menos tabu. Em resposta ao meu artigo sobre sexo e socialismo no New York Times, por exemplo, uma jovem búlgara publicou no Facebook: “Nasci no socialismo. Enquanto crescia, a sexualidade parecia a coisa mais normal: minha família falava sobre isso abertamente, havia livros de educação sexual semi-escondidos, íamos nus. A segunda coisa que minha mãe ainda me pergunta quando vai à praia. . Eu ligo para ela (depois de “Como vai você?”) É 'Você faz sexo com frequência suficiente?'... não estou dizendo que o socialismo foi ótimo, mas foi definitivamente interessante ler este artigo tendo uma experiência em primeira mão!” ”
 \par 
Um último caso de interesse é o da Checoslováquia socialista estatal, explorado em profundidade pela socióloga checa Katetina Liskova. Embora os checos e eslovacos tivessem uma longa
 \par 
KRISTEN R. GHODSEE histórico de interesse em sexologia que remonta à década de 1920, o advento do socialismo de estado produziu uma confluência única de ideologia socialista com discurso médico especializado. No início da década de 1950, sexólogos na Tchecoslováquia se concentraram no prazer feminino e argumentaram que "bom sexo" só era possível quando homens e mulheres eram socialmente iguais. Eles apoiaram o acesso das mulheres ao controle de natalidade e ao aborto, sua incorporação total à força de trabalho e medidas tomadas para aliviar seus encargos domésticos ou para compartilhá-los de forma mais equitativa com os homens. Como em outras sociedades socialistas de estado, todos os cidadãos tinham emprego e oportunidades de lazer garantidos, e eles desfrutavam de assistência médica universalmente acessível e da segurança de pensões para os idosos, o que reduzia a dependência econômica das mulheres em relação aos homens. Mais uma vez, a libertação do amor, sexo e romance da consideração econômica foi considerada uma característica única do socialismo de estado.
 \par 
Os sexólogos checoslovacos começaram a fazer pesquisas sobre o orgasmo feminino já em 1952 e, em 1961, organizaram uma conferência inteira para discutir as barreiras ao prazer sexual das mulheres. Com base em suas opiniões de especialistas, as mulheres não poderiam desfrutar plenamente do sexo se fossem economicamente dependentes dos homens. “A sociedade capitalista foi condenada principalmente do ponto de vista das mulheres”, escreve Liskov4 sobre os sexólogos checoslovacos. “Embora os autores fossem homens, eles viam e criticavam o capitalismo a partir da posição desfavorecida e marginalizada das mulheres. Esses sexólogos conectavam a discriminação pública e privada com a dependência econômica. Em sociedades economicamente desiguais, as pessoas, e especialmente as mulheres, não podiam buscar companheiros espirituais como seus parceiros de vida e sofriam em casamentos infelizes e com padrões sexuais duplos. ... A ordem capitalista era
 \par 
147
 \par 
148
 \par 
PARA CADA UM DE ACORDO COM SUAS NECESSIDADES: SOBRE SEXO (PARTE II) equiparado à subjugação das mulheres e ao patriarcado, e os arranjos socialistas foram aclamados como um antídoto à exploração capitalista das mulheres como propriedade.” Embora a ênfase inicial na igualdade de gênero fosse revertida depois que os tanques soviéticos esmagaram a primavera de Praga em 1968, e os tchecoslovacos se retirassem para a esfera privada para encontrar consolo durante o período de “Normalização”, os legados da era mais liberal do pós-guerra permaneceram.”
 \par 
As experiências de alguns dos países socialistas de estado na Europa de Leste sugerem que havia algo diferente nas relações sexuais sob o socialismo, e que pelo menos um fator significativo a este respeito são os apoios sociais criados para promover a independência econômica das mulheres. Embora estas políticas nunca tenham sido plenamente concretizadas, e tenham sido em parte implementadas para apoiar os objectivo de desenvolvimento da economia socialista, uma consequência destas políticas foi que as mulheres eram menos dependentes economicamente dos homens e, portanto, capazes de abandonar relações insatisfatórias mais facilmente do que as mulheres no mundo. Oeste. Além disso, em graus variados, os estados socialistas promoveram a ideia de que a sexualidade deveria ser desembaraçada do intercâmbio econômico e, no caso da Alemanha Oriental e da Checoslováquia, políticos e médicos afirmaram abertamente que isto tornava as relações mais “autênticas” e “honestas” do que no caso da Alemanha Oriental e da Checoslováquia. O Oeste. Em países como a Polônia e a Bulgária, os especialistas médicos apoiaram a ideia de que o prazer sexual das mulheres era importante para relacionamentos saudáveis ​​e divulgaram materiais educativos públicos (livros, panfletos, artigos, etc.) para educar os homens sobre os fundamentos da anatomia feminina. (Compare isto com os Estados Unidos, onde ainda hoje muitos jovens ainda não recebem educação adequada sobre como evitar
 \par 
KRISTEN R. GHODSEE gravidez, e muito menos informações sobre as complexidades do prazer feminino.)
 \par 
A ideia de que relacionamentos mais igualitários podem levar a um sexo melhor continua a intrigar pesquisadores em todo o mundo. Nos Estados Unidos, por exemplo, um estudo usando dados coletados entre o final da década de 1980 e o início da década de 1990 pareceu sugerir que homens e mulheres que compartilhavam tarefas domésticas faziam sexo com menos frequência do que aqueles que aderiam a uma divisão de gênero mais tradicional do trabalho doméstico, porque o desempenho de diferentes papéis de gênero aumentava aparentemente a atração sexual. Mas um estudo subsequente, "The Gendered Division of. Housework ano Couples' Sexual Relationships: A Reexamination", revisitou os dados originais e os comparou com novos dados coletados em 2006 de lares americanos de renda média com pelo menos um filho. Esses autores descobriram que a frequência sexual aumentava quando o cuidado com as crianças era compartilhado de forma mais uniforme. Os pesquisadores argumentaram que, à medida que os papéis de gênero americanos mudavam nos anos seguintes, mais homens e mulheres da classe trabalhadora e da classe média começaram a aceitar a ideia de que os homens deveriam ajudar em casa. A percepção de justiça na divisão de tarefas domésticas se tornou central para a intimidade dos casais, com os autores do estudo afirmando que “o sexo tem valor não apenas como uma desempenho de gênero, mas também como um meio de demonstrar amor e afeição. Como tal, os casais têm mais sexo e de maior qualidade quando estão satisfeitos com seus relacionamentos.”
 \par 
Outro estudo longitudinal de {\color{blue}1}.{\color{blue}338} casais heterossexuais alemães que estavam juntos há uma média de dez anos (69 por cento dos quais eram casados) corroborou que a percepção de justiça na divisão da casa
 \par 
149
 \par 
150
 \par 
PARA CADA UM DE ACORDO COM SUAS NECESSIDADES: SOBRE SEXO (PARTE I!) os deveres levaram a menos ressentimentos dentro do relacionamento. Este estudo foi elaborado para investigar a relação entre "contribuições do parceiro masculino para as tarefas domésticas e funcionamento sexual" ao longo de um período de cinco anos. Conforme os pesquisadores, "os resultados contam uma história clara: quando os homens contribuem de forma justa para as tarefas domésticas, o casal desfruta de sexo mais frequente e satisfatório no futuro". E como os homens da Alemanha Oriental aparentemente ainda contribuem mais em casa do que seus colegas da Alemanha Ocidental, parece que os legados do socialismo de estado continuam a influenciar a vida íntima no quarto". Não importa qual seja a divisão ideal do trabalho no lar, o problema com a sexualidade contemporânea é que a maioria dos relacionamentos humanos é formada num contexto social infundido com pensamento econômico e saturado de estresse. Não deveríamos ter que viver sob regimes autoritários para ter relacionamentos amorosos baseados mais em afeição mútua do que em troca material. O mercado atual para a sexualidade está cheio de muitos homens e mulheres jovens que são financeiramente inseguros e temerosos do futuro. Uma das minhas ex-alunas me disse que muitas de suas amigas e colegas na faixa dos vinte e poucos anos tomam antidepressivos para lidar com as pressões da vida diária. Essas drogas controlam a ansiedade, mas esmagam frequentemente a libido, transformando homens e mulheres jovens em autômatos viciados no trabalho obedientes que têm pouco tempo ou interesse em romance. O teórico cultural Mark Fisher argumentou que a deterioração da qualidade da saúde mental no Ocidente pode ser atribuída à precariedade do sistema econômico capitalista. Assim como as mudanças climáticas e a degradação ambiental, a incidência crescente de depressão e ansiedade são as externalidades negativas de um sistema que reduz o valor humano ao seu valor de troca.”
 \par 
KRISTEN R. GHODSEE
 \par 
Quer gostemos ou não, o capitalismo mercantiliza quase todos os aspectos de nossas vidas privadas, como prevê a teoria da economia sexual. Os relacionamentos pessoais tomam tempo e energia que poucos de nós temos para gastar enquanto lutamos para sobreviver na precária economia de bicos. Muitas vezes estamos exaustos e esgotados, sem vontade de investir os recursos emocionais necessários para manter relacionamentos amorosos sem compensação. Fico sempre surpreso com a prevalência de mulheres e homens jovens com ensino superior procurando por "sugar caddies" e "sugar momeies" em sites como Seekingarrangement.com, ou se inscrevendo em agências de acompanhantes para pagar as compras. Todos os relacionamentos exigem algum trabalho emocional, e os jovens estão aprendendo que também podem ser pagos por isso.**
 \par 
Muitos argumentarão que não há nada moralmente errado com o trabalho sexual, e que ele deve ser legalizado, protegido, sindicalizado e compensado de forma justa para aqueles que livremente escolhem procurar emprego neste setor da economia. O trabalho sexual existia muito antes do advento do capitalismo, continuou em graus variados em todos os países socialistas estatais e, sem dúvida, existirá de alguma forma no futuro. Mas muito trabalho sexual aberto, bem como as formas mais sutis de sexualidade mercantilizada para venda, é o resultado de um sistema econômico que fornece pouca segurança material para as mulheres e encoraja todas as pessoas a transformar tudo o que têm (seu trabalho, suas reputações, suas emoções, seus fluidos corporais e óvulos, e assim por diante) em um produto que pode ser vendido em um mercado onde os preços são determinados pelos caprichos da oferta e da demanda. Esta forma de troca amorosa não é um empoderamento positivo para o sexo para as mulheres, mas uma tentativa desesperada de sobreviver em um mundo com poucas redes de segurança social.
 \par 
151
 \par 
152
 \par 
Se tomarmos a teoria da economia sexual como um modelo extremo de como a sexualidade opera em uma economia capitalista, então considerar as experiências do outro extremo, o modelo socialista de estado, pode nos ajudar a pensar sobre possíveis maneiras de avançar em direção a algo que combine os aspectos bons de ambos os modelos, ao mesmo tempo, em que rejeita seus negativos óbvios. Ao implementar políticas socialistas para aumentar as oportunidades de emprego e liderança para mulheres (por meio de garantias de emprego ou alguma forma de cotas), bem como programas apoiados pelo estado para licença parental e assistência infantil subsidiada, as mulheres serão menos coagidas a vender sua sexualidade para atender às suas necessidades básicas. Até mesmo a assistência médica universal contribuiria muito para reduzir a dependência econômica das mulheres em relação aos homens. Construir um sistema de assistência médica universal está muito longe de implementar alguma forma de autoritarismo, não importa o que os especialistas de direita queiram que acreditemos. Os críticos do sistema de saúde americano frequentemente apontam que a assistência médica baseada no empregador prende os trabalhadores em empregos que eles odeiam porque os custos dos planos individuais são muito proibitivos. Mas é raramente mencionado que esposas dependentes também estão presas em seus casamentos porque nosso sistema de saúde lhes dá acesso a cuidados médicos por meio de seus maridos. No caso de divórcio, uma mulher perde o acesso ao plano baseado no empregador de seu ex-marido, deixando-a para se defender sozinha.”® Os americanos adoram os altares gêmeos da liberdade e da escolha, mas alguns aspectos fundamentais do nosso sistema econômico roubam das pessoas comuns a capacidade de tomar a decisão de deixar um emprego ou relacionamento insatisfatório porque podem perder o acesso a cuidados médicos básicos. “Essa paralisação de indivíduos eu considero o pior mal do capitalismo”, escreveu
 \par 
KRISTEN R. GHODSEE
 \par 
Albert Einstein em seu ensaio de 1949, “Por que o socialismo?” Vivendo em Princeton, Nova Jersey, nos últimos anos de sua vida, Einstein acreditava que “a anarquia econômica da sociedade capitalista” minou as liberdades humanas básicas, que poderiam ser restauradas se os americanos abraçassem certos aspectos do socialismo. E há muitas opções políticas abertas para nós, várias das quais operam nas democracias sociais europeias hoje, para aumentar nossas próprias liberdades pessoais.”
 \par 
O que me faz pensar em Ken e sua ex-esposa. Como ele era meu amigo próximo, sempre fiquei do lado dele no divórcio, compartilhando sua indignação com a grosseria do cálculo econômico dela. Mas, em muitos aspectos, ela também era uma vítima. Ken era um conhecido conquistador que usava sua riqueza para atrair mulheres. As regras do jogo eram claras: as mulheres lhe davam acesso à sexualidade delas, e ele pagava suas contas. Foi assim que ele conheceu a mulher com quem se casou, e ela entendeu as regras da troca. Mas em algum lugar ao longo do caminho, Ken se apaixonou por ela e esperava que ela retribuísse suas afeições. Ambos confundiram dinheiro com atratividade e a troca transacional de afeição com amor. O poder econômico de Ken deveria ser satisfatório no quarto. Mas Ken mudou de ideia e tentou reescrever as regras. Ele percebeu que a transação não era mais o suficiente para ele; ele queria uma conexão emocional real. Ken queria que ela o desejasse por quem ele era e não pelo que ele poderia comprar. Ele precisava saber que ela o amaria mesmo que ele perdesse sua riqueza. Para ela, a coisa honesta a fazer naquele momento seria contar a verdade e abandonar o relacionamento. Mas ela era pobre e sem educação, e sua proposta de casamento lhe ofereceu um bilhete dourado para um green card e uma nova vida na América. Então
 \par 
153
 \par 
154
 \par 
PARA CADA UM DE ACORDO COM SUAS NECESSIDADES: SOBRE SEXO (PARTE II) ela entrou no jogo. Das escolhas econômicas disponíveis para ela, fingir seu amor por um homem rico era uma opção ótima. Era culpa dela que ela realmente se apaixonou por outra pessoa, um homem que não tinha o dinheiro de Ken, mas por quem ela sentia uma atração genuína? Depois que sua autorização de residência chegou, ela não conseguiu mais fingir e fugiu para ficar com o homem que ela considerava seu "verdadeiro amor". O pobre Ken ficou de coração partido e amargurado por sua mentira, mas se ele tivesse parado por um minuto para pensar sobre o desequilíbrio de poder em seu relacionamento, ele teria visto que a dependência econômica dela em relação a ele alimentava seu subterfúgio contínuo. Naqueles últimos anos de sua curta vida, Ken percebeu que se ele quisesse um relacionamento com alguém que o amasse pelo que ele era (e não pelo que ele podia pagar), ele precisava imitar seu colega e encontrar uma mulher que pudesse atender às suas próprias necessidades básicas. Embora possa soar piegas para nossos ouvidos do século XXI, Bebel e Kollontai estavam basicamente certos. Relacionamentos íntimos que são relativamente livres do ethos transacional da teoria da economia sexual são geralmente mais honestos, autênticos e, bem, simplesmente melhores.
 \par 
