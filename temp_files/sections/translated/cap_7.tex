 
 \chapter{Affi native Action Baby}  

 \label{Affi native Action Baby}  
 
 
\par
 
 
 \textit{	}  

 
\par
 
 
 
\par
 

 \textbf{\textit{	} }  

 
\par
 
Ao lado de Clarence Thomas, Antonin Scalia é o juiz mais conservador da Suprema Corte. Ele também adora o programa de televisão
 {\color{blue} 24}  
. “Rapaz, naquelas primeiras temporadas”, ele diz ao seu biógrafo, “eu chegava às duas horas, porque você está no final de um [episódio] e dizia: 'Não, eu já tenho que ver o próximo.'” Scalia está especialmente impressionado com Jack Bauer, o herói fictício da série interpretado por Kiefer Sutherland. Bauer é um agente do governo em uma unidade de contraterrorismo de Los Angeles que frustra planos de assassinato em massa torturando suspeitos, sequestrando inocentes e executando colegas. Recusando-se a obedecer à lei, ele trava uma guerra em duas frentes, contra o terrorismo e a Constituição. E sempre que ele quebra uma regra ou quebra um osso, Scalia desmaia.
 \textbf{\textit{Jack Bauer saved Los Angeles. . . . He saved hundreds of thou-sands of lives. . . . Are you going to convict Jack Bauer? Say that criminal law is against him? You have the right to a jury trial? Is any jury going to convict Jack Bauer? I don’t think so. So the question is really whether we really believe in these absolutes. And ought we believe in these absolutes? 1} } 

 
\par
 
No entanto, Scalia passou a maioria da sua carreira como advogado, professor e jurista a dizer-nos que a Constituição é um absoluto, no qual devemos acreditar, mesmo quando – especialmente quando – ela nos diz algo que não queremos ouvir. A Constituição de Scalia não é uma declaração calorosa de propósito benevolente, facilmente adaptável às nossas necessidades em mudança. A sua Constituição está fria e morta, as suas proibições e injunções congeladas no tempo. Frases como “punição cruel e incomum” significam o que queriam dizer quando foram escritas na Constituição. Se isso produz resultados questionáveis ​​– digamos, a execução de crianças e de deficientes mentais – que pena. “Não creio”, escreve Scalia no caso Nixon v. Missouri Municipal League, que “evitar consequências infelizes seja uma base adequada para a interpretação de um texto”.
 {\color{blue} 2}  

 
\par
 
Scalia sente prazer especial com consequências infelizes. Ele aprecia a dificuldade e não gosta de quem a diminua ou negue. No caso Hamid v. Rumsfeld, uma pluralidade do Tribunal assumiu o que Scalia considerou uma posição frágil em relação ao poder executivo durante a guerra. O Tribunal decidiu que a Autorização para o Uso da Força Militar, aprovada pelo Congresso após o 11 de setembro, autorizava o presidente a deter cidadãos dos EUA indefinidamente como “combatentes inimigos ilegais” sem julgá-los num tribunal. Contudo, também decidiu que esses cidadãos tinham direito ao devido processo e podiam contestar a sua detenção perante algum tipo de tribunal.
 
\par
 
Scalia estava lívido. Escrevendo contra a pluralidade – bem como contra a administração Bush e outros conservadores no Tribunal – ele insistiu que um governo em guerra, mesmo um governo tão pouco convencional como a guerra ao terrorismo, tinha duas, e apenas duas, maneiras de manter um cidadão: tentar levá-lo a um tribunal ou fazer com que o Congresso suspenda o recurso de habeas corpus. Viva conforme as regras do devido processo, em outras palavras, ou suspenda-as. Tome uma posição, faça uma escolha.
 
\par
 
Mas o Tribunal evitou essa escolha, facilitando a vida ao governo e a si próprio. O Congresso e o presidente poderiam agir como se o habeas corpus estivesse suspenso, sem ter que suspendê-lo, e o Tribunal poderia agir como se o mandado não tivesse sido suspenso graças a um falso devido processo dos tribunais militares. Mais do que sair dos limites da Constituição, foi o “Sr. Fixit Mentality”, nas palavras de Scalia, sua “missão de fazer tudo dar certo”, que o enfureceu.
 {\color{blue} 3}  

 
\par
 
A missão de Scalia, por outro lado, é fazer com que tudo dê errado. A opinião de Scalia, tomando emprestada uma frase da escritora nova-iorquina Margaret Talbot, é “o equivalente jurisprudencial a quebrar uma guitarra no palco”.
 {\color{blue} 4}  
Scalia pode ter declarado uma vez que o Estado de direito é a lei das regras – levando alguns a confundi-lo com um conservador estereotipado – mas as regras e as leis têm um frisson particular para ele. Enquanto outros os procuram em busca de controlos estabilizadores ou de apoios tranquilizadores, Scalia procura impedimentos estimulantes e barreiras vertiginosas. Onde outros procuram segurança, Scalia procura sublimidade. Regras e leis tornam a vida mais difícil, e mais difícil é tudo. “Ser duro e tradicional é uma cruz pesada para carregar”, disse ele a um repórter. “Coação obriga.”
 {\color{blue} 5}  

 
\par
 
Isso, e não a fidelidade ao texto ou o conservadorismo como é convencionalmente entendido, é a ideia fixa da jurisprudência de Scalia – e a fonte da sua aparente paixão humana por Jack Bauer. Bauer nunca facilita as coisas para si; na verdade, ele se esforça para tornar as coisas o mais difíceis possível. Ele se oferece como voluntário para uma missão suicida quando outra pessoa faria isso (e provavelmente faria melhor); ele se transforma em um viciado como parte de um plano impossivelmente barroco para impedir um ato de bioterrorismo; ele coloca a esposa e a filha em risco, não uma, mas muitas vezes, e depois se culpa por fazer isso. Ele detesta o que faz, mas faz mesmo assim. Essa é a sua nobreza – alguns podem dizer masoquismo – e a razão pela qual ele aquece o coração de Scalia.
 
\par
 
Significa algo, claro, que Scalia identifique o caminho de maior resistência na fidelidade a um texto antigo, enquanto Bauer o encontre na traição a esse texto. Mas não tanto como se poderia pensar: como aprendemos com os casamentos dos nossos pregadores e políticos de direita, fidelidade é muitas vezes outra palavra para traição.
 
\par
 
Scalia nasceu em Trenton, Nova Jersey, em março de 1936, mas foi concebido no verão anterior em Florença, Itália. (Seu pai, um estudante de doutorado em línguas românicas em Columbia, ganhou uma bolsa para viajar para lá com sua esposa.) “Eu odiava Trenton”, diz Scalia; seu coração pertence à Florença. Devoto da ópera e da caça – “ele adora matar animais desarmados”, observa Clarence Thomas – Scalia gosta de traçar um perfil mediano de grande arte e grande crueldade. Ele apimenta suas decisões com alusões elegantes à literatura e à história. Era uma vez, ele gosta de dizer ao público, que ele era um originalismo muito “covarde” para defender a aceitação do século XVIII de cortes nas orelhas e açoites como formas de punição. Não mais. “Fiquei mais velho e mais mal-humorado”, diz ele, sempre a diva do desdém.
 {\color{blue} 6}  

 
\par
 
Quando Scalia tinha seis anos, seus pais se mudaram para Elmhurst, no Queens. Seu conservadorismo ao longo da vida é frequentemente atribuído à estrita educação católica italiana ali; aludindo a Burke, ele o chama de seu “pequeno pelotão”. Ele frequentou a Xavier High School, uma escola jesuíta em Manhattan, e Georgetown, uma universidade jesuíta em Washington, D.C. Em seu primeiro ano em Georgetown, a turma do último ano votou no senador Joseph McCarthy como o americano notável.
 {\color{blue} 7}  

 
\par
 
Mas Scalia aborda sua etnia e religião com uma atitude, emprestando à sua ideologia uma vantagem desfibriladora. (Esse desafio é muitas vezes considerado distintivo, em desacordo com os modos e costumes conservadores; mas, como vimos, não é.) Ele afirma que não entrou em Princeton, sua primeira escolha, porque “eu era um garoto italiano. Do Queens, não exatamente do tipo Princeton.” Mais tarde, após o Vaticano II ter liberalizado a liturgia e as práticas da Igreja, incluindo a igreja do seu bairro, no subúrbio de Washington, D.C., ele insistiu em levar a sua ninhada de sete crianças a quilômetros de distância para ouvir a missa dominical em latim. Mais tarde ainda, em Chicago, ele fez o mesmo, só que desta vez com nove filhos a tiracolo. Comentando como ele e sua esposa conseguiram criar filhos conservadores durante os anos 60 e 70 – sem jeans na casa dos Scalia – ele diz:
 
\par
 

 \textbf{\textit{They were being raised in a culture that wasn’t supportive of our values, that was certainly true. But we were helped by the fact that we were such a large family. We had our own culture. . . The first thing you’ve got to teach your kids is what my parents used to tell me all the time, “You’re not everybody else. . . . We have our own standards, and they aren’t the standards of the world in all respects, and the sooner you learn that the better.” {{\color{blue} 8} } } }  
 
 
\par
 
Acontece que o conservadorismo de Scalia é menos um pequeno pelotão do que uma contracultura Thoreauviana, um recuo e uma repreensão à corrente dominante, não muito diferente das comunas e grupúsculos hippies que ele uma vez tentou manter afastados. Não é um conservadorismo de tradição ou herança: os seus pais tinham apenas um filho, e a sua sogra queixava-se frequentemente de ter de conduzir quilômetros e horas em busca da única igreja verdadeira. “Por que vocês nunca parecem morar perto de igrejas?” ela perguntaria a Scalia e sua esposa.
 {\color{blue} 9}  
É um conservadorismo de invenção e escolha, informado pelo próprio espírito de rebelião que ele tão claramente detesta – ou pensa que detesta – na cultura em geral.
 
\par
 
Na década de 1970, enquanto lecionava na Universidade de Chicago, Scalia gostava de terminar o semestre com uma leitura de A Man for All Seasons, a peça de Robert Bolt sobre Thomas More. Embora o antiautoritarismo da peça pareça estar em desacordo com o conservadorismo de Scalia, o seu protagonista, pelo menos tal como é retratado por Bolt, não está. Literalmente mais católico que o papa, more é um verdadeiro crente na lei que se recusa a comprometer os seus princípios para acomodar os desejos de Henrique VIII. Ele paga por sua integridade com sua vida.
 
\par
 
O biógrafo de Scalia apresenta este detalhe biográfico com uma configuração reveladora: “No entanto, mesmo quando Scalia, na meia-idade, desenvolvia uma visão mais rígida da lei, ele ainda tinha explosões de idealismo”.
 {\color{blue} 10}  
Esse “ainda” está fora de lugar. A rigidez de Scalia não se opõe ao seu idealismo; é o seu idealismo. A sua leitura ultraconservadora da Constituição não reflete nem cinismo, nem convencionalismo; a ortodoxia e a piedade são, para ele, a essência da dissidência e da iconoclastia. Nenhuma acusação o entristece mais do que a afirmação, amplamente ensaiada nas suas Palestras Tanner em Princeton, em 1995, de que a sua filosofia é “de madeira”, “sem imaginação”, “pedestre”, “monótona”, “estreita” e “reflexiva”.
 {\color{blue} 11}  
Chame-o de bastardo ou idiota, cão do inferno ou radical em manto. Só não diga que ele é um terno.
 
\par
 
A filosofia de interpretação constitucional de Scalia – também chamada de originalismo, significado original ou significado público original – é frequentemente confundida com a intenção original. Embora o primeiro grupo de originalismos na década de 1970 tenha afirmado que o Tribunal deveria interpretar a Constituição conforme as intenções dos autores, originalismos posteriores como Scalia reformularam sabiamente esse argumento em resposta às críticas que recebeu. As intenções de um único autor são muitas vezes incognoscíveis e, no caso de muitos autores, praticamente indeterminadas. E cujas intenções devem contar: as do
 {\color{blue} 55}  
Homens que escreveram a Constituição, os 1.179 homens que a ratificaram, ou o número ainda maior de homens que votaram nos homens que a ratificaram? Do ponto de vista de Scalia, não são as intenções que nos governam. É a Constituição, o texto tal como foi escrito e reescrito por emendas. Esse é o objeto próprio da interpretação.
 
\par
 
Mas como recuperar o significado de um texto que pode passar de uma generalidade aterrorizante numa frase (“o poder executivo será investido num presidente”) para uma precisão monótona (os mandatos presidenciais são de quatro anos) na frase seguinte? Observe o significado público das palavras no momento em que foram adotadas, diz Scalia. Veja como eram utilizados: consulte dicionários, outros usos no texto, escritos influentes da época. Considere o contexto de sua declaração, como foram recebidos. A partir dessas fontes, construa um universo limitado de significados possíveis. As palavras não significam nada, admite Scalia, mas também não significam nada. Os juízes não devem ler a Constituição nem literal, nem vagamente, mas “razoavelmente” – isto é, de tal forma que cada palavra ou frase seja interpretada “para conter tudo o que significa de forma justa”. E então, garantidamente, aplicar esse significado aos nossos tempos muito diferentes.
 {\color{blue} 12}  

 
\par
 
Scalia justifica seu originalismo por dois motivos, ambos negativos. Numa democracia constitucional, é função dos representantes eleitos fazer a lei, e função dos juízes interpretá-la. Se os juízes não estiverem vinculados à forma como a lei, incluindo a Constituição, foi entendida no momento da sua promulgação – se consultarem a sua própria moral ou as suas próprias interpretações da moral do país – já não serão juízes, mas sim legisladores, e muitas vezes legisladores não eleitos. Em que. Ao vincular o juiz a um texto que não muda, o originalismo ajuda a conciliar a revisão judicial com a democracia e protege-nos do despotismo judicial.
 
\par
 
Se a primeira preocupação de Scalia é a tirania do banco, a segunda é a anarquia no banco. Uma vez que abandonamos a ideia de uma Constituição imutável, diz ele, abrimos as portas a todo e qualquer modo de interpretação. Como devemos entender uma Constituição que evolui? Olhando as pesquisas, a filosofia de John Rawls, os ensinamentos da Igreja Católica? Se a Constituição está sempre a mudar, que restrições podemos impor ao que é considerado uma interpretação aceitável? Nenhum, diz Scalia. Quando “cada dia” é “um novo dia” na lei, deixa de ser lei.
 {\color{blue} 13}  

 
\par
 
Esta mistura de tirania e anarquia não é uma fantasia vã, insistem Scalia e outros originalismos. Durante um período breve e terrível – desde o Warren Court da década de 1960 até ao Burger Court da década de 1970 – foi uma realidade. Em nome de uma “Constituição viva”, os juízes de esquerda refizeram (ou tentaram refazer) o país à sua própria imagem, forçando uma agenda de social-democracia, libertação sexual, igualdade de gênero, integração racial e relativismo moral no país. garganta. Palavras antigas adquiriram novas implicações e insinuações: subitamente o “devido processo legal” implicou um “direito à privacidade”, palavras-código para controlo de natalidade e aborto (e mais tarde sexo gay); a “igual proteção das leis” exigia um homem, um voto; a proibição de “buscas e apreensões injustificadas” significava que as provas obtidas ilegalmente pela polícia não podiam ser admitidas em tribunal; a proibição contra o “estabelecimento da religião” proibia a oração nas escolas. A cada lei que anulava e a cada direito que descobria, o Tribunal parecia inventar um novo terreno de ação. Foi um Carnaval constitucional, onde teorias exóticas de julgamento foram desfiladas com abandono libidinoso. Para os originalismos, o que havia de mais ultrajante nesta revolução vinda de cima – para além dos valores de esquerda que impôs à nação – foi o quão fora de sintonia estava com como o Tribunal justificava tradicionalmente as suas decisões de derrubar leis.
 
\par
 
Antes do Warren Court, diz Scalia, ou da década de 1920 (nunca fica claro quando exatamente a podridão começou), todo mundo era um originalismo.
 {\color{blue} 14}  
Isso não é bem verdade. Construções expansivas de significado constitucional são tão antigas e augustas quanto a própria fundação. E a autoconsciência teórica que Scalia e os seus seguidores trazem para a mesa é decididamente um fenômeno do século XX. Scalia, na verdade, muitas vezes parece ser um estudante graduado por volta de 1983. Ele diz que é um “comentário triste” que “os juízes americanos não tenham uma teoria inteligível sobre o que fazemos mais” e “ainda mais triste” que a profissão jurídica seja“ em geral. . . Despreocupados com que não temos uma teoria inteligível.”
 {\color{blue} 15}  

 
\par
 
Os conservadores costumavam zombar desse tipo de fetichismo teórico como sendo a marca de uma classe dominante inexperiente e ingênua; mesmo um originalismo declarado como Robert Bork admite que “a autoconfiguração das instituições jurídicas não exige tanto debate”. Mas Scalia e Bork forjaram as suas ideias na batalha contra uma jurisprudência liberal que era autoconsciente e teórica e, como tantos dos seus antecessores na direita, saíram dela parecendo mais inimigos do que amigos. Bork, de facto, admite abertamente que não é John Marshall ou Joseph Story – os grandes nomes tradicionais da revisão judicial – quem ele procura orientação; foi Alexander Nickel, indiscutivelmente o mais autoconsciente dos teóricos liberais do século XX, que “ensinou-me mais do que qualquer outra pessoa sobre este assunto”.
 {\color{blue} 16}  

 
\par
 
Como muitos originalismos, Scalia afirma que a sua jurisprudência nada tem a ver com o seu conservadorismo. “Tento fortemente impedir que as minhas opiniões religiosas, políticas ou filosóficas afetem a minha interpretação das leis.” No entanto, ele também disse que aprendeu com os seus professores em Georgetown a nunca “separar a sua vida religiosa da sua vida intelectual. Eles não estão separados.” Apenas alguns meses antes de Ronald Reagan o nomear para o Supremo Tribunal em 1986, ele admitiu que as suas opiniões jurídicas eram “inevitavelmente afetadas por percepções morais e teológicas”.
 {\color{blue} 17}  

 
\par
 
E, de facto, na gramática profunda das suas opiniões reside um conservadorismo que, se tem pouco a ver com a promoção dos interesses imediatos do Partido Republicano, tem ainda menos a ver com evitar as ameaças da tirania judicial e da anarquia judicial. É um conservadorismo que teria sido reconhecido pelos darwinistas sociais do final do século XIX, que mistura livremente o pré-moderno e o pós-moderno, o arcaico e o avançado. Não se encontra nos lugares óbvios – as opiniões de Scalia sobre o aborto, por exemplo, ou os direitos dos homossexuais – mas numa opinião divergente sobre o lugar mais ao estilo ONU-Scalia, o campo de golfe.
 
\par
 
Casey Martin era um jogador campeão de golfe (agora é ex-jogador de golfe) que, devido a uma doença degenerativa, não conseguia mais andar nos dezoito buracos de um campo de golfe. Depois que o PGA Tour recusou seu pedido de uso de carrinho de golfe na rodada final de um de seus torneios classificatórios, um tribunal federal emitiu uma liminar, com base na Lei dos Americanos Portadores de Deficiência (ADA), permitindo que Martin usasse um carrinho. O Título III da ADA afirma que “nenhum indivíduo será discriminado com base na deficiência no gozo pleno e igualitário dos bens, serviços, privilégios, vantagens ou acomodações de qualquer local de alojamento público por qualquer pessoa que possua, aluga (ou arrenda) ou opera um local de acomodação pública.” Quando o caso chegou ao Supremo Tribunal em 2001, as questões jurídicas resumiam-se a estas: Martin tem direito às proteções do Título III do ADA? Permitir que Martin usasse um carrinho “alteraria fundamentalmente a natureza” do jogo? Decidindo por 7–2 a favor de Martin – com Scalia e Thomas em desacordo – o Tribunal disse sim ao primeiro e não ao segundo.
 
\par
 
Ao responder à primeira questão, o Tribunal teve de contestar as alegações da PGA de que operava um “local de exibição ou entretenimento” em vez de um alojamento público, que apenas um cliente desse entretenimento se qualificava para as proteções do Título III, e que Martin estava não um cliente, mas um fornecedor de entretenimento. O Tribunal mostrou-se cético em relação às duas primeiras alegações. Mas mesmo que fossem verdade, disse o Tribunal, Martin ainda estaria protegido pelo Título III porque era de facto um cliente da PGA: ele e os outros concorrentes tiveram de pagar 3.000 dólares para tentarem participar no torneio. Alguns clientes pagaram para assistir ao torneio, outros para competir nele. A PGA também não poderia discriminar.
 
\par
 
Scalia ficou indignado. “Parece-me bastante incrível”, começou ele, que a maioria tratasse Martin como um “‘costume[r]’ de ‘concorrência’” e não como um concorrente. A PGA vendia entretenimento, o público pagava por isso, os jogadores de golfe forneciam; as rodadas de qualificação foram o pedido de contratação. Martin não era mais um cliente do que um ator que comparece a uma chamada de elenco aberta. Ele era um empregado, ou potencial empregado, cujo recurso adequado, se tivesse algum, não era o Título III da ADA, que abrangia acomodações públicas, mas o Título I, que abrangia o emprego. Mas Martin não teria esse recurso, admitiu Scalia, porque era essencialmente um contratante independente, uma categoria de funcionário não abrangida pela ADA. Martin acabaria assim numa terra de ninguém legal, sem qualquer proteção da lei.
 
\par
 
Na sugestão da maioria de que Martin era um cliente e não um concorrente, Scalia viu algo pior do que uma opinião erradamente decidida. Ele viu uma ameaça ao estatuto dos atletas em todo o mundo, cujo talento e excelência seriam sufocados pelo abraço forte do Tribunal, e também uma ameaça à ideia de competição em geral. Era como se os rivais homéricos da Grécia Antiga estivessem a ser arrancados dos seus jogos viris e forçados a percorrer os corredores de uma butique moderna.
 
\par
 
Os jogos têm uma valência especial para Scalia: são o espaço onde impera a desigualdade. “A própria natureza do esporte competitivo é a medição”, diz ele, “da excelência distribuída de forma desigual”. Essa desigualdade é o que “determina os vencedores e os perdedores”. Sob o sol da competição do meio-dia, não podemos esconder a nossa superioridade ou inferioridade, a nossa excelência ou inadequação. Os jogos tornam clara para o mundo a nossa natureza desigual; eles celebram “a distribuição desigual dos dons dados por Deus”. Na transposição de concorrente em cliente pelo Tribunal, Scalia viu a entrada forçada da democracia (uma “revolução”, na verdade) nesta reserva antiga. Com a “determinação da Fazenda Animal” – sim, Scalia vai lá – o Tribunal destruiu nossa única oportunidade de ver o quão desiguais realmente somos, quão injustamente Deus escolheu conceder suas bênçãos sobre nós. “O ano era 2001”, diz a última frase da dissidência de Scalia, “e 'todos finalmente eram iguais'”. Como o darwinista social e Nietzsche, Scalia é modernista demais, até mesmo pós-modernista, para ansiar pelo mundo perdido de cidades feudais fi. A modernidade viu demasiado fluxo para sustentar uma crença no estatuto hereditário. As marcas d'água do privilégio e da privação não são mais visíveis a olho nu; eles devem ser identificados, repetidas vezes, através da luta e da competição. Daí o apelo do jogo. No esporte, ao contrário do direito, cada dia é um novo dia. Cada competição é uma nova oportunidade para misturar tudo, para colocar as nossas hierarquias estabelecidas num relevo anárquico e permitir que surja uma nova face de supremacia ou abjecção. Desse modo, a ERS cria o casamento perfeito entre o feudal e o falível, o desigual e o instável. Para responder à segunda questão – andar num carrinho de golfe “altera fundamentalmente a natureza” do golfe – a maioria empreendeu uma história completa das regras do golfe. Em seguida, formulou um teste de duas partes para determinar se andar de carroça mudaria a natureza do golfe. O zelo e o cuidado, a seriedade com que a maioria encarava a sua tarefa, divertiam e irritavam Scalia.
 
\par
 

 \textbf{\textit{It has been rendered the solemn duty of the Supreme Court of the United States. . . To decide What Is Golf. I am sure that the Framers of the Constitution, aware of the 1457 edict of King James II of Scotland prohibiting golf because it interfered with the practice of archery, fully expected that sooner or later the paths of golf and government, the law and the links, would once again cross, and that the judges of this august Court would some day have to wrestle with that age-old jurisprudential question, for which their years of study in the law have so well-prepared them: Is someone riding around a golf course from shot to shot really a golfer?} }  
 
 
\par
 
Scalia está claramente se divertindo, mas sua alegria é um pouco confusa. A ADA define discriminação como
 
\par
 

 \textbf{\textit{A failure to make reasonable modifications in the policies, practices, or procedures, when such modifications are necessary} }  
 
 
\par
 
Para proporcionar tais bens, serviços, instalações, privilégios, vantagens ou acomodações a indivíduos com deficiência, a menos que a entidade possa demonstrar que fazer tais modificações alteraria fundamentalmente a natureza de tais bens, serviços, instalações, privilégios, vantagens ou acomodações que a entidade fornece.
 
\par
 
Qualquer determinação de discriminação requer uma determinação prévia sobre se a “modificação razoável” iria “alterar fundamentalmente a natureza” do bem em questão. A linguagem do estatuto, por outras palavras, obriga o Tribunal a investigar e decidir o que é golfe.
 
\par
 
Mas Scalia não aceitará nada disso. Recusando-se a ficar vinculado ao texto, prefere meditar sobre a futilidade e a fatuidade do inquérito do Tribunal. Ao procurar descobrir a essência do golfe, o Tribunal procura algo que não existe. “Dizer que algo é ‘essencial’”, escreve ele, “é normalmente dizer necessário para a realização de um determinado objetivo”. Mas os jogos “não têm outro objetivo exceto a diversão”. Na falta de um objeto, eles não têm essência. Portanto, é impossível dizer se uma regra é essencial. “Todas são arbitrárias”, escreve ele sobre as regras, “nenhuma é essencial”. O que constitui uma regra é a tradição ou, “em tempos mais modernos”, o decreto de um órgão de autoridade como a PGA. Num momento de descuido, Scalia considera a possibilidade de haver “algum ponto em que as regras de um jogo bem conhecido são alteradas a tal ponto que nenhuma pessoa razoável o chamaria de o mesmo jogo”. Mas ele rapidamente recua da sua incursão no essencialismo. Nada de Platão para ele; ele está com Nietzsche o tempo todo.
 {\color{blue} 18}  

 
\par
 
É difícil conciliar esta hostilidade quase românica à ideia da essência do golfe com as declarações anteriores de Scalia sobre “a própria natureza do esporte competitivo” ser a revelação de desigualdades divinamente ordenadas. (Também é difícil conciliar a indiferença de Scalia relativamente à linguagem do estatuto com o seu textualismo, mas isso é outra questão.) Se não for resolvida, contudo, a contradição revela os polos gêmeos da fé de Scalia: uma crença em regras como imposições arbitrárias de poder – refletindo nada (nem mesmo a vontade ou a posição de seus criadores), mas a superfície plana de seu significado rock-binário – ao qual devemos, no entanto, nos submeter; e uma crença em regras, zelosamente aplicadas, como a varinha mágica da nossa desigualdade inerradicável. Aqueles que conseguem passar por esses deuses vazios e estéreis são vencedores; todos os outros são perdedores.
 
\par
 
Nos Estados Unidos, observou Tocqueville, um juiz federal “deve saber como compreender o espírito da época”. Embora a personalidade de um juiz do Supremo Tribunal possa ser “puramente judicial”, as suas “prerrogativas” – o poder de derrubar leis em nome da Constituição – “são inteiramente políticas”.
 {\color{blue} 19}  
Se quiser exercer eficazmente essas prerrogativas, terá de ser tão culturalmente ágil e socialmente sintonizado como o político mais astuto.
 
\par
 
Como então explicar a influência de Scalia? Aqui está um homem que orgulhosa e desafiadoramente proclama seu desdém pelo “espírito da época” – isto é, quando ele não é embaraçosamente ignorante dele. Quando o Tribunal votou em 2003 para anular as leis estaduais que proíbem o sexo gay, Scalia viu o país descer uma ladeira escorregadia para a masturbação.
 {\color{blue} 20}  
Em 1996, ele disse a uma audiência de cristãos que “devemos orar por coragem para suportar o desprezo do mundo sofisticado”, um mundo que “não terá nada a ver com milagres”. Temos “que estar preparados para sermos considerados idiotas”.
 {\color{blue} 21}  
Numa divergência do mesmo ano, Scalia declarou: “Dia após dia, caso após caso, [o Tribunal] está ocupado a conceber uma Constituição para um país que não reconheço”.
 {\color{blue} 22}  
Como escreveu Maureen Down: “Ele é tão antiquado, é do Antigo Testamento”.
 {\color{blue} 23}  
E, de acordo com Elena Kagan, o mais novo membro do Tribunal, nomeado por Obama em 2010, Scalia “é o juiz que teve o impacto mais importante ao longo dos anos na forma como pensamos e falamos sobre a lei”. John Paul Stevens, o homem que Kagan substituiu e até à sua reforma o juiz mais liberal do Tribunal, diz que Scalia “fez uma enorme diferença, algumas delas construtivas, outras infelizes”. Além disso, a influência de Scalia provavelmente se estenderá ao futuro. “Ele está em sintonia com muitos membros da atual geração de estudantes de direito”, observa Ruth Bader Ginsburg, outra liberal da Corte.
 {\color{blue} 24}  
Dê-me uma estudante de direito em uma idade impressionável, poderia ter dito Jean Bodin, e ela será minha para o resto da vida. Não foram as posições específicas de Scalia que prevaleceram no Tribunal. Na verdade, algumas de suas opiniões mais famosas – contra o aborto, a ação afirmativa e os direitos dos homossexuais; a favor da pena de morte, da oração na escola e da discriminação sexual – são dissidências. (Com a adição de John Roberts ao Tribunal em 2005 e de Samuel Alito em 2006, no entanto, isso começou a mudar.) A posição de Scalia é mais evidente na forma como os seus colegas – e outros juristas, advogados e acadêmicos – apresentam os seus argumentos. .
 
\par
 
Durante muitos anos, o originalismo foi ridicularizado pela esquerda. Como William Brennan, o titã liberal do Tribunal da segunda metade do século XX, declarou em 1985: “Aqueles que restringiriam as reivindicações de direito aos valores de 1789 especificamente articulados na Constituição fecham os olhos ao progresso social e evitam a adaptação de princípios abrangentes para mudanças nas circunstâncias sociais”. Contra os originalismos, Brennan insistiu que “a genialidade da Constituição não reside em qualquer significado estático que possa ter tido num mundo que está morto e desaparecido, mas na adaptabilidade dos seus grandes princípios para lidar com os problemas e necessidades atuais”.
 {\color{blue} 25}  

 
\par
 
Contudo, apenas uma década mais tarde, a liberal Laurence Tribe, parafraseando o liberal Ronald Working, diria: “Agora somos todos listas de origem”.
 {\color{blue} 26}  
Isso é ainda mais verdadeiro hoje. Enquanto a geração de ontem de estudiosos constitucionais recorreu à filosofia – Rails, Hart, ocasionalmente Notice, Marx ou Nietzsche – para interpretar a Constituição, a de hoje olha para a história, para o momento em que uma palavra ou passagem se tornou parte do texto e adquiriu o seu significado. Não apenas à direita, mas também à esquerda: Bruce Ackerman, Akhil Amar e Jack Balkan são apenas três dos mais proeminentes originalismos liberais escritos hoje.
 
\par
 
Os liberais no Tribunal passaram por uma mudança semelhante. Na sua dissidência no Citizens United, Stevens escreveu um longo ecurus sobre os “entendimentos originais”, “expectativas originais” e “significado público original” da Primeira Emenda no que diz respeito ao discurso corporativo. Abrindo a sua discussão com um suspiro de obrigação – “Vamos começar do início” – Stevens sentiu-se compelido por Scalia, cuja voz e nome estiveram presentes durante todo o processo, a demonstrar que a sua posição era consistente com o significado original da liberdade de expressão.
 {\color{blue} 27}  

 
\par
 
Outros estudiosos e juristas ajudaram a provocar esta mudança, mas foi Scalia quem manteve a chama nos mais altos níveis da lei. Não por tato ou diplomacia. Scalia costuma ser um porco, zombando da inteligência de seus colegas e questionando sua integridade. Sandra Day O’Connor, que ocupou o banco de 1981 a 2006, foi objeto frequente de seu ridículo e desprezo. Scalia caracterizou um de seus argumentos como “desprovido de conteúdo”. Outro, escreveu ele, “não pode ser levado a sério”. Sempre que lhe perguntam sobre o seu papel no caso Bush v. Gore (2000), que colocou George W. Bush na Casa Branca por um modo de raciocínio questionável, ele zomba: “Supere isso!”
 {\color{blue} 28}  
Nem, ao contrário dos seus seguidores de campo, Scalia dominou a Corte pela força da sua inteligência. (“Quão inteligente ele é?” exala um admirador representativo.)
 {\color{blue} 29}  
Em uma corte onde todos são formados em Harvard, Yale ou Princeton, e os professores da Ivy League sentam-se em ambos os lados do banco, há muitos cérebros disponíveis.
 
\par
 
Vários outros factores explicam o domínio de Scalia no Tribunal. Para começar, Scalia tem a vantagem de uma filosofia simples e um método bacana. Enquanto ele e o seu exército marcham pelos arquivos, vasculhando documentos sobre o direito de portar armas, a cláusula de comércio e muito mais, a esquerda jurídica permanece “confusa e incerta”, nas palavras dos professores de direito de Yale, Robert Post e Reva Siegel.  “incapaz de avançar qualquer teoria robusta de interpretação constitucional” própria.
 {\color{blue} 30}  
Numa época em que falta à esquerda certeza e vontade, a autoconfiança de Scalia pode ser uma força potente e inebriante.
 
\par
 
Em segundo lugar, há uma afinidade eletiva, até mesmo um ajuste estreito, entre o originalismo da coação obriga e a ideia de jogo de Scalia. E essa é a visão de Scalia sobre o que uma vida boa implica: uma luta diária e árdua, onde a única garantia, se deixarmos as coisas como estão, é que os fortes vencerão e os fracos perderão. Acontece que Scalia não é nem de longe o iconoclasta que pensa que é. Longe de dizer “às pessoas o que elas não gostam de ouvir”, como afirma, ele diz à elite do poder exatamente o que elas querem ouvir: que são superiores e que têm um lugar à mesa porque são superiores.
 {\color{blue} 31}  
Afinal, Tocqueville, ao que parece, estava certo. Não é a clareza, mas a pertinência do juiz Scalia, a maneira como ele reflete, em vez de refratar, o espírito da época, que explica, pelo menos em parte, sua influência.
 
\par
 
Mas pode haver uma razão adicional, embora pequena e pessoal, para a presença descomunal de Scalia no nosso firmamento constitucional. E essa é a paciência e a tolerância, a decência geral e as boas maneiras que os seus colegas liberais lhe mostram. Enquanto ele discursa e delira, quebrando guitarras e bombardeando seus inimigos, eles tendem a responder com um encolher de ombros indulgente, um “isso é apenas Niño”, como O’Connor costumava dizer.
 {\color{blue} 32}  

 
\par
 
O fato pode ser pequeno e pessoal, mas a ironia é grande e política. Pois Scalia aproveita e lucra com a própria cultura do liberalismo que afirma abominar: a tolerância de pontos de vista opostos, as concessões generosas para as falhas de outras pessoas, a “compaixão benevolente” que ele ridiculariza na sua dissidência no campo de golfe. Caso os seus colegas alguma vez o obriguem a respeitar as mesmas regras de civilidade liberal, ou a tratá-lo como ele os trata, quem sabe o que poderá acontecer? Na verdade, como observaram dois observadores atentos do Tribunal – num artigo apropriadamente intitulado “Não cutuquem Scalia!”
 {\color{blue} 33}  
Propenso a acessos de raiva, mimado por um conjunto diferente de regras: agora isso é um afixo RMA - tempo de ação, bebê.
 
\par
 
Desde a década de 1960, tem sido um lugar-comum na nossa cultura política que as sutilezas liberais dependam de sutilezas conservadoras não tão delicadas. Um jantar no Upper West Side exige uma força policial que não conhece Miranda, a Primeira Emenda, um exército que não conhece Genebra. Esse, claro, é o conceito de
 {\color{blue} 24}  
(sem mencionar muitas outras produções de Hollywood como A Few Good Men). Mas essa formulação pode estar exatamente ao contrário: sem os seus colegas mais liberais o cederem e protegerem, Scalia – tal como Jack Bauer – teria muito mais dificuldades. O conservadorismo da coação obriga depende do liberalismo da noblesse oblige, e não o contrário. Esse é o verdadeiro significado do Juiz Scalia.
 
\par
  
 
999999
