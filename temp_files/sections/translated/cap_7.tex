\chapter{Afi nativa Action Baby}\label{Afi nativa Action Baby}
 \par 
Ao lado de Clarence Thomas, Antonin Scalia é o juiz mais conservador da Suprema Corte. Ele também adora o programa de televisão {\color{blue}24} horas. “Cara, naquelas primeiras temporadas”, diz ele ao biógrafo, “eu demoraria até as duas horas, porque você está no final de um [episódio] e você diga: 'Não, preciso ver o próximo'”. Scalia está especialmente impressionado com Jack Bauer, o herói fictício da série interpretado por Kiefer Sutherland. Bauer é um agente do governo em uma unidade de contraterrorismo de Los Angeles que frustra planos de assassinato em massa torturando suspeitos, sequestrando inocentes e executando colegas. Recusando-se a obedecer à lei, ele trava uma guerra em duas frentes, contra o terrorismo e a Constituição. E sempre que ele quebra uma regra ou quebra um osso, Scalia desmaia.
 \par 
Jack Bauer salvou Los Angeles. . . . Ele salvou centenas de milhares de vidas. . . . Você vai condenar Jack Bauer? Diga que o direito penal está contra ele? Você tem direito a um julgamento com júri? Algum júri vai condenar Jack Bauer? Eu não acho. Portanto, a questão é se realmente acreditamos nestes absolutos. E devemos acreditar nesses absolutos? {\color{blue}1}
 \par 
No entanto, Scalia passou a maior parte da sua carreira como advogado, professor e jurista a dizer-nos que a Constituição é um absoluto, no qual devemos acreditar, mesmo quando – especialmente quando – ela nos diz algo que não queremos ouvir. A Constituição de Scalia não é uma declaração calorosa de propósito benevolente, facilmente adaptável às nossas necessidades em mudança. A sua Constituição está fria e morta, as suas proibições e injunções congeladas no tempo. Frases como “punição cruel e incomum” significam o que queriam dizer quando foram escritas na Constituição. Se isso produz resultados questionáveis ​​– digamos, a execução de crianças e de deficientes mentais – que pena. “Não creio”, escreve Scalia no caso Nixon v. Missouri Municipal League, que “evitar consequências infelizes seja uma base adequada para a interpretação de um texto”.{\color{blue}2}
 \par 
Scalia sente prazer especial com consequências infelizes. Ele aprecia a dificuldade e não gosta de quem a diminua ou negue. No caso Hamdi v. Rumsfeld, uma pluralidade do Tribunal assumiu o que Scalia considerou uma posição frágil em relação ao poder executivo durante a guerra. O Tribunal decidiu que a Autorização para o Uso da Força Militar, aprovada pelo Congresso após o {\color{blue}11} de Setembro, autorizava o presidente a deter cidadãos dos EUA indefinidamente como “combatentes inimigos ilegais” sem julgá-los num tribunal. Contudo, também decidiu que esses cidadãos tinham direito ao devido processo e podiam contestar a sua detenção perante algum tipo de tribunal.
 \par 
Scalia estava lívido. Escrevendo contra a pluralidade – bem como contra a administração Bush e outros conservadores no Tribunal – ele insistiu que um governo em guerra, mesmo um governo tão pouco convencional como a guerra ao terrorismo, tinha duas, e apenas duas, maneiras de manter um cidadão: tentar levá-lo a um tribunal ou fazer com que o Congresso suspenda o recurso de habeas corpus. Viva de acordo com as regras do devido processo, em outras palavras, ou suspenda-as. Tome uma posição, faça uma escolha.
 \par 
Mas o Tribunal evitou essa escolha, facilitando a vida ao governo e a si próprio. O Congresso e o presidente poderiam agir
 \par 
Como se o habeas corpus fosse suspenso, sem necessidade de suspendê-lo, e a Corte pudesse agir como se o mandado não tivesse sido suspenso graças a um falso devido processo dos tribunais militares. Mais do que sair dos limites da Constituição, foi o “Sr. Fix-It Mentality”, nas palavras de Scalia, sua “missão de fazer tudo dar certo”, que o enfureceu.{\color{blue}3}
 \par 
A missão de Scalia, por outro lado, é fazer com que tudo dê errado. A opinião de Scalia, tomando emprestada uma frase da escritora nova-iorquina Margaret Talbot, é “o equivalente jurisprudencial a quebrar uma guitarra no palco”. {\color{blue}4} Scalia pode ter declarado uma vez que o Estado de direito é a lei das regras – levando alguns a confundi-lo com um conservador estereotipado – mas as regras e as leis têm um frisson particular para ele. Enquanto outros os procuram em busca de controlos estabilizadores ou de apoios tranquilizadores, Scalia procura impedimentos estimulantes e barreiras vertiginosas. Onde outros procuram segurança, Scalia procura sublimidade. Regras e leis tornam a vida mais difícil, e mais difícil é tudo. “Ser duro e tradicional é uma cruz pesada para carregar”, disse ele a um repórter. “Duresse obriga.”{\color{blue}5}
 \par 
Isso, e não a fidelidade ao texto ou o conservadorismo como é convencionalmente entendido, é a ideia fixa da jurisprudência de Scalia – e a fonte da sua aparente paixão humana por Jack Bauer. Bauer nunca facilita as coisas para si mesmo; na verdade, ele se esforça para tornar as coisas o mais difíceis possível. Ele se oferece como voluntário para uma missão suicida quando outra pessoa faria isso (e provavelmente faria melhor); ele se transforma em um viciado como parte de um plano impossivelmente barroco para impedir um ato de bioterrorismo; ele coloca a esposa e a filha em risco, não uma, mas muitas vezes, e depois se culpa por fazer isso. Ele detesta o que faz, mas faz mesmo assim. Essa é a sua nobreza – alguns podem dizer masoquismo – e a razão pela qual ele aquece o coração de Scalia.
 \par 
Significa algo, claro, que Scalia identifique o caminho de maior resistência na fidelidade a um texto antigo, enquanto Bauer o encontre na traição a esse texto. Mas não tanto quanto se poderia pensar: como temos
 \par 
Aprenda com os casamentos de nossos pregadores e políticos de direita: fidelidade é muitas vezes outra palavra para traição.
 \par 
Scalia nasceu em Trenton, Nova Jersey, em março de 1936, mas foi concebido no verão anterior em Florença, Itália. (Seu pai, um estudante de doutorado em línguas românicas em Columbia, ganhou uma bolsa para viajar para lá com sua esposa.) “Eu odiava Trenton”, diz Scalia; seu coração pertence a Florença. Devoto da ópera e da caça – “ele adora matar animais desarmados”, observa Clarence Thomas – Scalia gosta de traçar um perfil mediciano de grande arte e grande crueldade. Ele apimenta suas decisões com alusões elegantes à literatura e à história. Era uma vez, ele gosta de dizer ao público, que ele era um originalismo muito “covarde” para defender a aceitação do século XVIII de cortes nas orelhas e açoites como formas de punição. Não mais. “Fiquei mais velho e mais mal-humorado”, diz ele, sempre a diva do desdém.{\color{blue}6}
 \par 
Quando Scalia tinha seis anos, seus pais se mudaram para Elmhurst, no Queens. Seu conservadorismo ao longo da vida é frequentemente atribuído à estrita educação católica italiana ali; aludindo a Burke, ele o chama de seu “pequeno pelotão”. Ele frequentou a Xavier High School, uma escola jesuíta em Manhattan, e Georgetown, uma universidade jesuíta em Washington, D.C. Em seu primeiro ano em Georgetown, a turma do último ano votou no senador Joseph McCarthy como o americano notável.{\color{blue}7}
 \par 
Mas Scalia aborda sua etnia e religião com uma atitude, emprestando à sua ideologia uma vantagem desfibriladora. (Esse desafio é muitas vezes considerado distintivo, em desacordo com os modos e costumes conservadores; mas, como vimos, não é.) Ele afirma que não entrou em Princeton, sua primeira escolha, porque “eu era um garoto italiano. do Queens, não exatamente do tipo Princeton.” Mais tarde, depois de o Vaticano II ter liberalizado a liturgia e as práticas da Igreja, incluindo a igreja do seu bairro, no subúrbio de Washington, D.C., ele insistiu em conduzir a sua ninhada de sete crianças por quilómetros.
 \par 
Fora para ouvir a missa dominical em latim. Mais tarde ainda, em Chicago, ele fez a mesma coisa, só que desta vez com nove filhos a tiracolo. Comentando como ele e sua esposa conseguiram criar filhos conservadores durante os anos {\color{blue}60} e {\color{blue}70} – sem jeans na casa dos Scalia – ele diz:
 \par 
Eles estavam sendo criados em uma cultura que não apoiava nossos valores, isso certamente era verdade. Mas fomos ajudados pelo fato de sermos uma família tão grande. Tínhamos nossa própria cultura. . . A primeira coisa que você precisa ensinar aos seus filhos é o que meus pais costumavam me dizer o tempo todo: “Você não é todo mundo. . . . Temos nossos próprios padrões, e eles não são os padrões do mundo em todos os aspectos, e quanto mais cedo você aprender isso, melhor.”{\color{blue}8}
 \par 
Acontece que o conservadorismo de Scalia é menos um pequeno pelotão do que uma contracultura Thoreauviana, um recuo e uma repreensão à corrente dominante, não muito diferente das comunas e grupúsculos hippies que ele uma vez tentou manter afastados. Não é um conservadorismo de tradição ou herança: os seus pais tinham apenas um filho, e a sua sogra queixava-se frequentemente de ter de conduzir quilómetros e horas em busca da única igreja verdadeira. “Por que vocês nunca parecem morar perto de igrejas?” ela perguntaria a Scalia e sua esposa. {\color{blue}9} É um conservadorismo de invenção e escolha, informado pelo próprio espírito de rebelião que ele tão claramente detesta – ou pensa que detesta – na cultura em geral.
 \par 
Na década de 1970, enquanto lecionava na Universidade de Chicago, Scalia gostava de terminar o semestre com uma leitura de A Man for All Seasons, a peça de Robert Bolt sobre Thomas More. Embora o antiautoritarismo da peça pareça estar em desacordo com o conservadorismo de Scalia, o seu protagonista, pelo menos tal como é retratado por Bolt, não o é. Literalmente mais católico que o papa, More é um verdadeiro crente na lei que se recusa a comprometer os seus princípios para acomodar os desejos de Henrique VIII. Ele paga por sua integridade com sua vida.
 \par 
O biógrafo de Scalia apresenta este detalhe biográfico com uma configuração reveladora: “No entanto, mesmo quando Scalia, na meia-idade, desenvolvia uma visão mais rígida da lei, ele ainda tinha explosões de idealismo”. {\color{blue}10} Esse “ainda” está mal colocado. A rigidez de Scalia não se opõe ao seu idealismo; é o seu idealismo. A sua leitura ultraconservadora da Constituição não reflecte nem cinismo nem convencionalismo; a ortodoxia e a piedade são, para ele, a essência da dissidência e da iconoclastia. Nenhuma acusação o entristece mais do que a afirmação, amplamente ensaiada nas suas Palestras Tanner em Princeton, em 1995, de que a sua filosofia é “de madeira”, “sem imaginação”, “pedestre”, “monótona”, “estreita” e “reflexiva”. {\color{blue}11} Chame-o de bastardo ou idiota, cão do inferno ou radical em manto. Só não diga que ele é um terno.
 \par 
A filosofia de interpretação constitucional de Scalia – também chamada de originalismo, significado original ou significado público original – é frequentemente confundida com a intenção original. Embora o primeiro grupo de originalismos na década de 1970 tenha afirmado que o Tribunal deveria interpretar a Constituição de acordo com as intenções dos autores, originalismos posteriores como Scalia reformularam sabiamente esse argumento em resposta às críticas que recebeu. As intenções de um único autor são muitas vezes incognoscíveis e, no caso de muitos autores, praticamente indeterminadas. E quais intenções deveriam contar: as dos {\color{blue}55} homens que escreveram a Constituição, os {\color{blue}1}.{\color{blue}179} homens que a ratificaram, ou o número ainda maior de homens que votaram nos homens que a ratificaram? Do ponto de vista de Scalia, não são as intenções que nos governam. É a Constituição, o texto tal como foi escrito e reescrito através de emendas. Esse é o objeto próprio da interpretação.
 \par 
Mas como recuperar o significado de um texto que pode passar de uma generalidade aterrorizante numa frase (“o poder executivo será investido num presidente”) para uma precisão monótona (os mandatos presidenciais são de quatro anos) na frase seguinte? Observe o significado público das palavras no momento em que foram adotadas, diz Scalia. Veja como eles
 \par 
Foram utilizados: consulta a dicionários, outros usos no texto, escritos influentes da época. Considere o contexto de sua declaração, como foram recebidos. A partir dessas fontes, construa um universo limitado de significados possíveis. As palavras não significam nada, admite Scalia, mas também não significam nada. Os juízes não devem ler a Constituição nem literal nem vagamente, mas “razoavelmente” – isto é, de tal forma que cada palavra ou frase seja interpretada “para conter tudo o que significa de forma justa”. E então, de uma forma ou de outra, aplicar esse significado aos nossos tempos muito diferentes.{\color{blue}12}
 \par 
Scalia justifica seu originalismo por dois motivos, ambos negativos. Numa democracia constitucional, cabe aos representantes eleitos fazer a lei e aos juízes interpretá-la. Se os juízes não estiverem vinculados à forma como a lei, incluindo a Constituição, foi entendida no momento da sua promulgação – se consultarem a sua própria moral ou as suas próprias interpretações da moral do país – já não serão juízes, mas sim legisladores, e muitas vezes legisladores não eleitos. em que. Ao vincular o juiz a um texto que não muda, o originalismo ajuda a conciliar a revisão judicial com a democracia e protege-nos do despotismo judicial.
 \par 
Se a primeira preocupação de Scalia é a tirania do banco, a segunda é a anarquia no banco. Uma vez que abandonamos a ideia de uma Constituição imutável, diz ele, abrimos as portas a todo e qualquer modo de interpretação. Como devemos entender uma Constituição que evolui? Olhando as pesquisas, a filosofia de John Rawls, os ensinamentos da Igreja Católica? Se a Constituição está sempre a mudar, que restrições podemos impor ao que é considerado uma interpretação aceitável? Nenhum, diz Scalia. Quando “cada dia” é “um novo dia” na lei, deixa de ser lei.{\color{blue}13}
 \par 
Esta mistura de tirania e anarquia não é uma fantasia vã, insistem Scalia e outros originalismos. Durante um período breve e terrível – desde o Warren Court da década de 1960 até ao Burger Court da década de 1970 – foi uma realidade. Em nome de uma “Constituição viva”, os juízes de esquerda refizeram (ou
 \par 
Tentaram refazer o país à sua própria imagem, forçando uma agenda de social-democracia, libertação sexual, igualdade de género, integração racial e relativismo moral goela abaixo do país. Palavras antigas adquiriram novas implicações e insinuações: subitamente o “devido processo legal” implicou um “direito à privacidade”, palavras-código para controlo de natalidade e aborto (e mais tarde sexo gay); a “igual proteção das leis” exigia um homem, um voto; a proibição de “buscas e apreensões injustificadas” significava que as provas obtidas ilegalmente pela polícia não podiam ser admitidas em tribunal; a proibição contra o “estabelecimento da religião” proibia a oração nas escolas. A cada lei que anulava e a cada direito que descobria, o Tribunal parecia inventar um novo terreno de acção. Foi um Carnaval constitucional, onde teorias exóticas de julgamento foram desfiladas com abandono libidinoso. Para os originalismos, o que havia de mais ultrajante nesta revolução vinda de cima – para além dos valores de esquerda que impôs à nação – foi o quão fora de sintonia estava com a forma como o Tribunal tradicionalmente justificava as suas decisões de derrubar leis.
 \par 
Antes do Tribunal Warren, diz Scalia, ou da década de 1920 (nunca fica claro quando exatamente a podridão começou), todos eram um originalismo. {\color{blue}14} Isso não é bem verdade. Construções expansivas do significado constitucional são tão antigas e augustas quanto a própria fundação. E a autoconsciência teórica que Scalia e seus seguidores trazem à mesa é um fenômeno decididamente do século XX. Scalia, na verdade, muitas vezes soa como um estudante de literatura especializada por volta de 1983. Ele diz que é um "comentário triste" que "os juízes americanos não tenham uma teoria inteligível sobre o que mais fazemos" e "ainda mais triste" que a profissão jurídica seja "em geral... despreocupada com o fato de que não temos uma teoria inteligível".{\color{blue}15}
 \par 
Os conservadores costumavam zombar desse tipo de fetichismo teórico como sendo a marca de uma classe dominante inexperiente e ingénua; mesmo um originalismo declarado como Robert Bork admite que “a autoconfiguração das instituições jurídicas não exige tanto debate”. Mas Scalia
 \par 
E Bork forjou as suas ideias na batalha contra uma jurisprudência liberal que era autoconsciente e teórica e, como tantos dos seus antecessores na direita, saíram dela parecendo mais inimigos do que amigos. Bork, de facto, admite abertamente que não é John Marshall ou Joseph Story – os grandes nomes tradicionais da revisão judicial – quem ele procura orientação; foi Alexander Bickel, indiscutivelmente o mais autoconsciente dos teóricos liberais do século XX, que “ensinou-me mais do que qualquer outra pessoa sobre este assunto”.{\color{blue}16}
 \par 
Como muitos originalismos, Scalia afirma que sua jurisprudência não tem nada a ver com seu conservadorismo. “Tento fortemente evitar que minhas visões religiosas, políticas ou filosóficas afetem minha interpretação das leis.” No entanto, ele também disse que aprendeu com seus professores em Georgetown a nunca “separar sua vida religiosa de sua vida intelectual. Elas não são separadas.” Apenas alguns meses antes de Ronald Reagan nomeá-lo para a Suprema Corte em 1986, ele admitiu que suas visões legais eram “inevitavelmente afetadas por percepções morais e teológicas.”{\color{blue}17}
 \par 
E, de facto, na gramática profunda das suas opiniões reside um conservadorismo que, se tem pouco a ver com a promoção dos interesses imediatos do Partido Republicano, tem ainda menos a ver com evitar as ameaças da tirania judicial e da anarquia judicial. É um conservadorismo que teria sido reconhecido pelos darwinistas sociais do final do século XIX, que mistura livremente o pré-moderno e o pós-moderno, o arcaico e o avançado. Não se encontra nos lugares óbvios – as opiniões de Scalia sobre o aborto, por exemplo, ou os direitos dos homossexuais – mas numa opinião divergente sobre o lugar mais anti-Scaliaesco, o campo de golfe.
 \par 
Casey Martin era um jogador campeão de golfe (agora é ex-jogador de golfe) que, devido a uma doença degenerativa, não conseguia mais andar nos dezoito buracos de um campo de golfe. Depois que o PGA Tour recusou seu
 \par 
Solicitado o uso de carrinho de golfe na rodada final de um de seus torneios classificatórios, um tribunal federal emitiu liminar, com base na Lei dos Americanos Portadores de Deficiência (ADA), permitindo que Martin usasse carrinho. O Título III da ADA afirma que “nenhum indivíduo será discriminado com base na deficiência no gozo pleno e igualitário dos bens, serviços, privilégios, vantagens ou acomodações de qualquer local de alojamento público por qualquer pessoa que possua , aluga (ou arrenda) ou opera um local de acomodação pública.” Quando o caso chegou ao Supremo Tribunal em 2001, as questões jurídicas resumiam-se a estas: Martin tem direito às proteções do Título III do ADA? Permitir que Martin usasse um carrinho “alteraria fundamentalmente a natureza” do jogo? Decidindo por 7–2 a favor de Martin – com Scalia e Thomas em desacordo – o Tribunal disse sim ao primeiro e não ao segundo.
 \par 
Ao responder à primeira pergunta, o Tribunal teve que lidar com as alegações da PGA de que estava operando um “local de exibição ou entretenimento” em vez de uma acomodação pública, que apenas um cliente desse entretenimento se qualificava para as proteções do Título III e que Martin não era um cliente, mas um provedor de entretenimento. O Tribunal estava cético em relação às duas primeiras alegações. Mas mesmo se fossem verdadeiras, o Tribunal disse que Martin ainda estaria protegido pelo Título III porque ele era de fato um cliente da PGA: ele e os outros competidores tiveram que pagar US$ {\color{blue}3}.{\color{blue}000} para tentar entrar no torneio. Alguns clientes pagaram para assistir ao torneio, outros para competir nele. A PGA não poderia discriminar nenhum dos dois.
 \par 
Scalia ficou indignado. “Parece-me bastante incrível”, começou ele, que a maioria tratasse Martin como um “‘costume[r]’ de ‘concorrência’” e não como um concorrente. A PGA vendia entretenimento, o público pagava por isso, os jogadores de golfe forneciam; as rodadas de qualificação foram o pedido de contratação. Martin não era mais um cliente do que um ator que comparece a uma chamada de elenco aberta. Ele era um empregado, ou potencial empregado, cujo recurso adequado, se tivesse algum, era
 \par 
Não o Título III da ADA, que abrangia acomodações públicas, mas o Título I, que abrangia o emprego. Mas Martin não teria esse recurso, admitiu Scalia, porque era essencialmente um contratante independente, uma categoria de funcionário não abrangida pela ADA. Martin acabaria assim numa terra de ninguém legal, sem qualquer proteção da lei.
 \par 
Na sugestão da maioria de que Martin era um cliente e não um concorrente, Scalia viu algo pior do que uma opinião erradamente decidida. Ele viu uma ameaça ao estatuto dos atletas em todo o mundo, cujo talento e excelência seriam sufocados pelo abraço forte do Tribunal, e também uma ameaça à ideia de competição em geral. Era como se os rivais homéricos da Grécia Antiga estivessem a ser arrancados dos seus jogos viris e forçados a percorrer os corredores de uma boutique moderna.
 \par 
Os jogos têm uma valência especial para Scalia: são o espaço onde impera a desigualdade. “A própria natureza do desporto competitivo é a medição”, diz ele, “da excelência distribuída de forma desigual”. Essa desigualdade é o que “determina os vencedores e os perdedores”. Sob o sol da competição do meio-dia, não podemos esconder a nossa superioridade ou inferioridade, a nossa excelência ou inadequação. Os jogos tornam clara para o mundo a nossa natureza desigual; eles celebram “a distribuição desigual dos dons dados por Deus”.
 \par 
Na transposição de concorrente para cliente pelo Tribunal, Scalia viu a entrada forçada da democracia (uma “revolução”, na verdade) nesta reserva antiga. Com a “determinação da Fazenda Animal” – sim, Scalia vai lá – o Tribunal destruiu nossa única oportunidade de ver o quão desiguais realmente somos, quão injustamente Deus escolheu conceder suas bênçãos sobre nós. “O ano era 2001”, diz a última frase da dissidência de Scalia, “e 'todos finalmente eram iguais'”. Assim como os darwinistas sociais e Nietzsche, Scalia é modernista demais, até mesmo pós-modernista, para ansiar pelo mundo perdido de cidades feudais phi. A modernidade viu muito fluxo para sustentar uma crença na
 \par 
Estatuto hereditário. As marcas d'água do privilégio e da privação não são mais visíveis a olho nu; eles devem ser identificados, repetidas vezes, através da luta e da competição. Daí o apelo do jogo. No desporto, ao contrário do direito, cada dia é um novo dia. Cada competição é uma nova oportunidade para misturar tudo, para colocar as nossas hierarquias estabelecidas num relevo anárquico e permitir que surja uma nova face de supremacia ou abjecção. Oferece assim o casamento perfeito entre o feudal e o falível, o desigual e o instável. Para responder à segunda questão – andar num carrinho de golfe “altera fundamentalmente a natureza” do golfe – a maioria empreendeu uma história completa das regras do golfe. Em seguida, formulou um teste de duas partes para determinar se andar de carroça mudaria a natureza do golfe. O zelo e o cuidado, a seriedade com que a maioria encarava a sua tarefa, divertiam e irritavam Scalia.
 \par 
Foi considerado dever solene da Suprema Corte dos Estados Unidos. . . Para decidir o que é golfe. Estou certo de que os redatores da Constituição, cientes do edito de 1457 do rei Jaime II da Escócia, que proibia o golfe porque interferia na prática do tiro com arco, esperavam plenamente que, mais cedo ou mais tarde, os caminhos do golfe e do governo, a lei e as ligações , cruzariam mais uma vez, e que os juízes deste augusto Tribunal algum dia teriam que lutar com aquela antiga questão jurisprudencial, para a qual seus anos de estudo da lei os prepararam tão bem: alguém está andando em um campo de golfe? curso de tacada em tacada é realmente um jogador de golfe?
 \par 
Scalia está claramente se divertindo, mas sua alegria é um pouco confusa. A ADA define discriminação como
 \par 
Deixar de fazer modificações razoáveis ​​nas políticas, práticas ou procedimentos, quando tais modificações forem necessárias
 \par 
Para proporcionar tais bens, serviços, instalações, privilégios, vantagens ou acomodações a indivíduos com deficiência, a menos que a entidade possa demonstrar que fazer tais modificações alteraria fundamentalmente a natureza de tais bens, serviços, instalações, privilégios, vantagens ou acomodações que o entidade fornece.
 \par 
Qualquer determinação de discriminação requer uma determinação prévia sobre se a “modificação razoável” iria “alterar fundamentalmente a natureza” do bem em questão. A linguagem do estatuto, por outras palavras, obriga o Tribunal a investigar e decidir o que é golfe.
 \par 
Mas Scalia não aceitará nada disso. Recusando-se a ficar vinculado ao texto, prefere meditar sobre a futilidade e a fatuidade do inquérito do Tribunal. Ao procurar descobrir a essência do golfe, o Tribunal procura algo que não existe. “Dizer que algo é ‘essencial’”, escreve ele, “é normalmente dizer que é necessário para a realização de um determinado objetivo”. Mas os jogos “não têm outro objetivo exceto a diversão”. Na falta de um objeto, eles não têm essência. Portanto, é impossível dizer se uma regra é essencial. “Todas são arbitrárias”, escreve ele sobre as regras, “nenhuma é essencial”. O que constitui uma regra é a tradição ou, “em tempos mais modernos”, o decreto de um órgão de autoridade como a PGA. Num momento de descuido, Scalia considera a possibilidade de haver “algum ponto em que as regras de um jogo bem conhecido são alteradas a tal ponto que nenhuma pessoa razoável o chamaria de o mesmo jogo”. Mas ele rapidamente recua da sua incursão no essencialismo. Nada de Platão para ele; ele está com Nietzsche o tempo todo.{\color{blue}18}
 \par 
É difícil conciliar esta hostilidade quase Rortyiana à ideia da essência do golfe com as declarações anteriores de Scalia sobre “a própria natureza do desporto competitivo” ser a revelação de desigualdades divinamente ordenadas. (Também é difícil conciliar a opinião de Scalia
 \par 
Indiferença à linguagem do estatuto com seu textualismo, mas isso é outra questão.) Se não for resolvida, porém, a contradição revela os pólos gêmeos da fé de Scalia: uma crença em regras como imposições arbitrárias de poder – refletindo nada (nem mesmo a vontade). ou posição de seus criadores), mas a superfície plana de seu significado binário - ao qual devemos, no entanto, nos submeter; e uma crença em regras, zelosamente aplicadas, como a varinha mágica da nossa desigualdade inerradicável. Aqueles que conseguem passar por esses deuses vazios e estéreis são vencedores; todos os outros são perdedores.
 \par 
Nos Estados Unidos, observou Tocqueville, um juiz federal “deve saber como compreender o espírito da época”. Embora a personalidade de um juiz do Supremo Tribunal possa ser “puramente judicial”, as suas “prerrogativas” – o poder de derrubar leis em nome da Constituição – “são inteiramente políticas”. {\color{blue}19} Se quiser exercer essas prerrogativas de forma eficaz, deve ser tão culturalmente ágil e socialmente sintonizado como o político mais astuto.
 \par 
Como então explicar a influência de Scalia? Aqui está um homem que orgulhosamente e desafiadoramente proclama seu desdém pelo “espírito da época” – isto é, quando ele não é embaraçosamente ignorante dele. Quando o Tribunal votou em 2003 para anular as leis estaduais que proíbem o sexo gay, Scalia viu o país descer uma ladeira escorregadia para a masturbação. {\color{blue}20} Em 1996, ele disse a uma audiência de cristãos que “devemos rezar pela coragem de suportar o desprezo do mundo sofisticado”, um mundo que “não terá nada a ver com milagres”. Temos “que estar preparados para sermos considerados idiotas”. {\color{blue}21} Numa dissidência nesse mesmo ano, Scalia declarou: “Dia após dia, caso a caso, [o Tribunal] está ocupado a elaborar uma Constituição para um país que não reconheço”. {\color{blue}22} Como escreveu Maureen Dowd: “Ele é tão Old School, ele é Antigo Testamento”. {\color{blue}23} E, no entanto, de acordo com Elena Kagan, o mais novo membro do Tribunal, nomeado por Obama em 2010, Scalia “é o juiz que teve o impacto mais importante ao longo dos anos na forma como pensamos
 \par 
E fale sobre a lei. John Paul Stevens, o homem que Kagan substituiu e até à sua reforma o juiz mais liberal do Tribunal, diz que Scalia “fez uma enorme diferença, algumas delas construtivas, outras infelizes”. Além disso, a influência de Scalia provavelmente se estenderá ao futuro. “Ele está em sintonia com muitos membros da atual geração de estudantes de direito”, observa Ruth Bader Ginsburg, outra liberal da Corte. {\color{blue}24} Dê-me uma estudante de direito com uma idade impressionável, poderia ter dito Jean Brodie, e ela será minha para o resto da vida. Não foram as posições específicas de Scalia que prevaleceram no Tribunal. Na verdade, algumas de suas opiniões mais famosas – contra o aborto, a ação afirmativa e os direitos dos homossexuais; a favor da pena de morte, da oração na escola e da discriminação sexual – são dissidências. (Com a adição de John Roberts ao Tribunal em 2005 e de Samuel Alito em 2006, no entanto, isso começou a mudar.) A posição de Scalia é mais evidente na forma como os seus colegas – e outros juristas, advogados e académicos – apresentam os seus argumentos. .
 \par 
Durante muitos anos, o originalismo foi ridicularizado pela esquerda. Como William Brennan, o titã liberal do Tribunal da segunda metade do século XX, declarou em 1985: “Aqueles que restringiriam as reivindicações de direito aos valores de 1789 especificamente articulados na Constituição fecham os olhos ao progresso social e evitam a adaptação de princípios abrangentes para mudanças nas circunstâncias sociais”. Contra os originalismos, Brennan insistiu que “a genialidade da Constituição não reside em qualquer significado estático que possa ter tido num mundo que está morto e desaparecido, mas na adaptabilidade dos seus grandes princípios para lidar com os problemas e necessidades actuais”.{\color{blue}25}
 \par 
Apenas uma década depois, porém, a liberal Laurence Tribe, parafraseando o liberal Ronald Dworkin, diria: “Agora somos todos originistas”. {\color{blue}26} Isso é ainda mais verdadeiro hoje. Enquanto a geração de ontem de estudiosos constitucionais recorreu à filosofia – Rawls, Hart, ocasionalmente Nozick, Marx ou Nietzsche – para interpretar a Constituição, a de hoje olha para a história, para o momento
 \par 
Quando uma palavra ou passagem passou a fazer parte do texto e adquiriu seu significado. Não apenas à direita, mas também à esquerda: Bruce Ackerman, Akhil Amar e Jack Balkin são apenas três dos mais proeminentes originalismos liberais escritos hoje.
 \par 
Os liberais no Tribunal passaram por uma mudança semelhante. Na sua dissidência no Citizens United, Stevens escreveu um longo excursus sobre os “entendimentos originais”, “expectativas originais” e “significado público original” da Primeira Emenda no que diz respeito ao discurso corporativo. Abrindo a sua discussão com um suspiro de obrigação – “Vamos começar do início” – Stevens sentiu-se compelido por Scalia, cuja voz e nome estiveram presentes durante todo o processo, a demonstrar que a sua posição era consistente com o significado original da liberdade de expressão.{\color{blue}27}
 \par 
Outros acadêmicos e juristas ajudaram a provocar essa mudança, mas foi Scalia quem manteve a chama nos mais altos escalões da lei. Não por tato ou diplomacia. Scalia é frequentemente um porco, zombando da inteligência de seus colegas e questionando sua integridade. Sandra Day O'Connor, que ficou no tribunal de 1981 a 2006, foi um objeto frequente de seu ridículo e desprezo. Scalia caracterizou um de seus argumentos como "desprovido de conteúdo". Outro, ele escreveu, "não pode ser levado a sério". Sempre que lhe perguntam sobre seu papel em Bush v. Gore (2000), que colocou George W. Bush na Casa Branca por meio de um modo questionável de raciocínio, ele zomba: "Supere isso!" {\color{blue}28} Nem, ao contrário de seus seguidores, Scalia dominou a Corte pela força de sua inteligência. (“Quão inteligente ele é?”, exala um admirador representativo.) {\color{blue}29} Em uma Corte onde todos são formados em Harvard, Yale ou Princeton, e professores da Ivy League sentam-se em ambos os lados do tribunal, há muitos cérebros para todos.
 \par 
Vários outros factores explicam o domínio de Scalia no Tribunal. Para começar, Scalia tem a vantagem de uma filosofia simples e um método bacana. Enquanto ele e o seu exército marcham pelos arquivos, vasculhando documentos sobre o direito de portar armas, o
 \par 
Cláusula comercial, e muito mais, a esquerda jurídica permanece “confusa e incerta”, nas palavras dos professores de direito de Yale, Robert Post e Reva Siegel, “incapaz de avançar qualquer teoria robusta de interpretação constitucional” própria. {\color{blue}30} Numa época em que falta à esquerda certeza e vontade, a autoconfiança de Scalia pode ser uma força potente e inebriante.
 \par 
Em segundo lugar, há uma afinidade eletiva, até mesmo um ajuste estreito, entre o originalismo da coação obriga e a ideia de jogo de Scalia. E essa é a visão de Scalia sobre o que uma vida boa implica: uma luta diária e árdua, onde a única garantia, se deixarmos as coisas como estão, é que os fortes vencerão e os fracos perderão. Acontece que Scalia não é nem de longe o iconoclasta que pensa que é. Longe de dizer “às pessoas o que elas não gostam de ouvir”, como afirma, ele diz à elite do poder exactamente o que elas querem ouvir: que são superiores e que têm um lugar à mesa porque são superiores. {\color{blue}31} Tocqueville, ao que parece, estava certo, afinal. Não é a clareza, mas a pertinência do juiz Scalia, a maneira como ele reflete, em vez de refratar, o espírito da época, que explica, pelo menos em parte, sua influência.
 \par 
Mas pode haver uma razão adicional, embora pequena e pessoal, para a presença descomunal de Scalia no nosso firmamento constitucional. E essa é a paciência e a tolerância, a decência geral e as boas maneiras que os seus colegas liberais lhe mostram. Enquanto ele discursa e delira, quebrando guitarras e bombardeando seus inimigos, eles tendem a responder com um encolher de ombros indulgente, um “isso é apenas Nino”, como O’Connor costumava dizer.{\color{blue}32}
 \par 
O facto pode ser pequeno e pessoal, mas a ironia é grande e política. Pois Scalia aproveita e lucra com a própria cultura do liberalismo que afirma abominar: a tolerância de pontos de vista opostos, as concessões generosas para as falhas de outras pessoas, a “compaixão benevolente” que ele ridiculariza na sua dissidência no campo de golfe. Se os seus colegas alguma vez o forçarem a respeitar as mesmas regras de civilidade liberal, ou o tratarem como ele os trata, quem sabe o que poderá acontecer?
 \par 
Aconteceu? De fato, como dois observadores atentos do Tribunal notaram — em um artigo apropriadamente intitulado “Não provoque Scalia!” — sempre que os advogados perante o tribunal o submetem à mais gentil das provocações, ele fica rapidamente abalado e tirado do jogo. {\color{blue}33} Propenso a acessos de raiva, mimado por um conjunto diferente de regras: agora isso é um bebê de ação de afixo rmatime.
 \par 
Desde a década de 1960, tem sido um lugar-comum na nossa cultura política que as sutilezas liberais dependam de não-sonicidades conservadoras. Um jantar no Upper West Side exige uma força policial que não conhece Miranda, a Primeira Emenda, um exército que não conhece Genebra. Esse, claro, é o conceito de {\color{blue}24} Horas (sem mencionar muitas outras produções de Hollywood, como A Few Good Men). Mas essa formulação pode estar exactamente ao contrário: sem os seus colegas mais liberais o cederem e protegerem, Scalia – tal como Jack Bauer – teria muito mais dificuldades. O conservadorismo da coação obriga depende do liberalismo da noblesse oblige, e não o contrário. Esse é o verdadeiro significado do Juiz Scalia.