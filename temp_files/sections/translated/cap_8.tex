\chapter{7 Capitalismo e Crise}\label{7 Capitalismo e Crise}
 \par 
O capitalismo expande-se porque liberta forças económicas que obrigam todos os capitalistas e, até certo ponto, todos os trabalhadores, a comportarem-se de formas que sejam funcionais para a acumulação de capital como um todo. Apesar deste grau de coerência interna, o capitalismo também apresenta falhas profundas e irremediáveis, tanto porque o capitalismo prejudica sistematicamente o potencial humano, como porque a subordinação das necessidades humanas à motivação do lucro desencadeia crises e contradições que limitam o âmbito da reprodução do próprio capital. . Estas tensões e limites são discutidos abaixo e revisitados no Capítulo {\color{blue}15}. A crise mundial que começou em 2007 é examinada no Capítulo {\color{blue}14}.
 \par 
\section{Teoria da acumulação e da crise de Marx}
 \par 
A teoria de Marx da necessidade, em oposição à mera possibilidade de crises regulares nas economias capitalistas, baseia-se na interacção entre a concorrência, os conflitos de classe e as leis de acumulação. Estas se unem no seu tratamento da lei da tendência de queda da taxa de lucro (LTRPF). O LTRPF será discutido no Capítulo {\color{blue}9}. Por enquanto, é suficiente observar que as crises podem ocorrer independentemente de movimentos imediatos na taxa de lucro; na verdade, podem dever-se a factores provenientes de fora do circuito do capital; por exemplo, convulsões sociais, políticas ou técnicas. A possibilidade de erosão na taxa de lucro devido à incapacidade dos capitais de serem reestruturados para alcançar maior rentabilidade, e a fragilidade da bolsa de valores face às “más” notícias e as suas repercussões na reprodução económica, são demasiado familiares. Outras causas potenciais da crise incluem quedas de preços devido à superprodução nos principais sectores industriais, o colapso de importantes instituições financeiras e a instabilidade induzida pelo comércio externo ou pela turbulência política interna ou externa.
 \par 
Marx argumenta que as crises podem sempre surgir, devido à contradição entre a produção de valores de uso para o lucro e o seu consumo privado. É apenas sob o capitalismo, onde domina a produção para o lucro e não para o uso, que a superprodução de uma mercadoria pode revelar-se um embaraço. Noutras sociedades seria motivo de comemoração, porque significaria aumento do consumo. Mas para o capital o consumo não é suficiente; a acumulação sustentada requer a realização de lucro. Isto depende da venda e, se isso se tornar impossível, a produção poderá ser restringida e o capital como um todo forçado a operar numa escala reduzida, com sérias implicações para o emprego e o bem-estar social.
 \par 
Por exemplo, um conjunto de capitalistas produzindo uma mercadoria específica pode estar sujeito a alguma perturbação gerada na esfera econômica ou em outro lugar. No entanto, a reprodução expandida de seus próprios capitais está intimamente integrada a outros circuitos de capital. Seus insumos são os suprimentos de outros capitalistas e vice-versa. A economia pode ser vista como um sistema de circuitos em expansão conectados como rodas dentadas interligadas. Se um conjunto de rodas desacelera ou para, o mesmo ocorrerá com outras em todo o sistema. Por exemplo, para que a indústria de vestuário se expanda, deve haver um aumento coordenado na produção de têxteis, exigindo uma maior produção de linho e algodão, mais maquinário e assim por diante, e mais trabalhadores e financiamento devem estar disponíveis para todas essas indústrias. É o necessário, mas não planejado e competitivo entrelaçamento de capitais que leva Marx a falar da anarquia da produção capitalista. Nisso, Marx antecipa alguns dos melhores insights de Keynes, principalmente por meio de seus esquemas de reprodução. Mas a análise de Marx vai mais longe e mais fundo em muitos aspectos, estendendo a consideração do nível de demanda (efetiva) para suas fontes na produção e acumulação de mais-valia, e argumentando que as crises são mudanças forçadas no ritmo de acumulação, bem como em sua estrutura interna. Ele as vê como necessárias no sentido de que elas resolvem à força as contradições internas da acumulação que, de outra forma, persistiriam. As crises também são inevitáveis, como mostrado abaixo.
 \par 
\section{Teoria da acumulação e da crise de Marx}
 \par 
As teorias da crise geralmente partem do colapso dos circuitos individuais do capital, juntamente com as consequências sociais das decisões privadas sobre produção e compra. Um circuito de capital pode ser interrompido em qualquer uma das suas ligações (ver Capítulo 5, Figura {\color{blue}5}.1). A ruptura pode ser voluntária ou involuntária por parte do capitalista, que pode ser capaz, mas não querer, ou querer, mas não ser capaz, de permitir que o circuito continue. No primeiro caso, o capitalista estará a especular, quer antecipando que a rentabilidade poderá ser aumentada atrasando o circuito, quer esperando criar ou explorar uma posição de monopólio ao fazê-lo; alternativamente, o capitalista pode ser pessimista quanto à possibilidade de realização da mais-valia que pode ser produzida. No segundo caso, o capitalista está sujeito a forças fora do controlo imediato.
 \par 
É improvável que haja uma quebra do circuito na esfera da produção, a menos que os trabalhadores tomem medidas laborais ou haja grandes perturbações naturais ou técnicas (incluindo rápidas mudanças tecnológicas em circunstâncias financeiras desfavoráveis). Em vez disso, quase todas as crises parecem ter origem na esfera da circulação, como uma incapacidade ou falta de vontade de comprar, vender ou investir. Considere o momento MP M - C < LP no circuito industrial do capital (Capítulo {\color{blue}4}). Uma quebra voluntária aqui implica que C está disponível para venda, mas o proprietário de M pode antecipar um preço mais baixo para os factores de produção ou esperar criar um preço mais baixo. Em particular, para LP, isto pode ser feito através da redução (ou ameaça de redução) do nível de emprego como parte de uma estratégia para aumentar a taxa de mais-valia.
 \par 
A interrupção do circuito também pode ser involuntária. Os proprietários dos factores de produção podem tentar criar ou explorar uma posição de monopólio - em particular, os trabalhadores podem fazer greve. Alternativamente, os insumos podem não estar disponíveis porque, na ronda anterior de produção, os produtos - insumos parcialmente presentes - podem ter sido produzidos nas proporções erradas. Isto criará um excesso de procura de determinados produtos e, normalmente, um excesso de oferta noutros sectores. Se isto se generalizar a muitos produtores e sectores, a situação é denominada uma crise de desproporcionalidade. Estas observações precisam de ser modificadas se a mercadoria em falta for a força de trabalho, caso em que haverá um excesso de procura de trabalho, mas também um excesso de oferta de capital monetário (não utilizado).
 \par 
Uma ruptura na esfera de circulação também pode aparecer entre C' e M'. Um capitalista pode especular sobre o preço futuro do capital mercadoria, criando uma ruptura voluntária. Alternativamente, pode ser impossível vender produtos, o que significa que a mercadoria está em excesso. Isto pode dever-se à desproporcionalidade ou porque aqueles que normalmente compram a mercadoria podem não conseguir fazê-lo por não terem dinheiro em mãos, acesso ao crédito ou perspectivas lucrativas. Por exemplo, se outros circuitos foram interrompidos, por qualquer razão, os trabalhadores, os capitalistas e outros não receberão o seu fluxo regular de rendimento e, portanto, não realizarão as suas despesas habituais. Se isto se generalizar, é conhecido como crise de superprodução (ou, de outro ponto de vista, de subconsumo). Marx expôs toda a questão de forma clara quando sugeriu que as mercadorias estão apaixonadas pelo dinheiro, mas o curso do amor verdadeiro nunca correu bem.
 \par 
Os marxistas têm geralmente olhado para as crises de sobreprodução/subconsumo e desproporcionalidade dividindo a economia em dois setores, investimento e consumo, seguindo o esquema de Marx para a reprodução alargada (ver Capítulo {\color{blue}5}). Alguns argumentaram que existe uma tendência persistente para a oferta de bens de consumo ultrapassar a procura dos mesmos, outros que existe uma tendência para uma produção desproporcionalmente grande de bens de investimento. Ambos são logicamente possíveis, mas as desproporções (superprodução num sector, subprodução noutro) são tão prováveis ​​de ocorrer dentro dos sectores de bens de consumo e de investimento como entre os dois como agregados. Além disso, é fácil confundir uma crise de desproporcionalidade, em que os bens de consumo estão em excesso, com uma crise de sobreprodução. Este último será caracterizado por um excesso geral na oferta de mercadorias (um “excesso”) e pelo desenvolvimento anterior de capacidade produtiva excedentária. Uma crise de desproporcionalidade não pressupõe este excesso de oferta generalizado, mas apenas excessos localizados em vários sectores económicos influentes que podem, eventualmente, desencadear uma crise de sobreprodução.
 \par 
As rupturas nos circuitos individuais do capital ocorrerão frequentemente devido à anarquia da produção capitalista, às flutuações nos preços de mercado, às perturbações no comércio internacional, aos caprichos do sistema de crédito, à especulação financeira ou de outros tipos, à monopolização e à obsolescência económica do capital fixo como resultado do progresso tecnológico. Ocasionalmente, estes serão suficientemente importantes para gerar uma crise, dependendo a sua extensão dos padrões de perturbação e, subsequentemente, do ajustamento na reprodução económica. Contudo, esta descrição das possibilidades de crise é limitada, porque deixa implícito o motivo da produção capitalista: o lucro. A influência determinante na produção do ponto de vista do capitalista é a quantidade de lucro gerada pelo circuito do capital. Todos os obstáculos podem ser superados se s for suficientemente grande. Caso a rentabilidade esteja a melhorar, os capitalistas estarão relutantes em suspender as vendas a fim de especular sobre lucros maiores e posteriores, negar aumentos salariais ou de qualquer forma impedir o processo de obtenção de lucros. Isto acontece tanto que o sistema financeiro prolongará frequentemente um boom especulativo muito depois de a rentabilidade ter começado a mostrar sinais de fraqueza em qualquer outra coisa que não os termos do papel (ver Capítulo {\color{blue}14}). Os lucros podem compensar e abrir o caminho. Se a capacidade de obter lucros for restringida, então não só alguns capitalistas serão expulsos da produção por falência, como também reinará o pessimismo geral, a produção será reduzida e estará em perspectiva uma crise sistémica.
 \par 
Os movimentos na rentabilidade dependem não apenas das condições de venda, mas também dos movimentos nos valores. Como foi visto no Capítulo 3, o processo de acumulação competitiva traz reduções frequentes nos valores de todas as mercadorias. É uma característica contraditória do capitalismo que o lucro individual seja perseguido através da redução de valores através da relativa expulsão do trabalho vivo da produção, embora o trabalho seja a única fonte de mais-valia e, portanto, de lucro. Marx analisa esta contradição no contexto da LTRPF (ver Capítulo {\color{blue}9}).
 \par 
As teorias de crise centradas na superprodução, no subconsumo, na desproporcionalidade e na queda da taxa de lucro deram origem a uma extensa literatura; no entanto, isoladamente estas abordagens são limitadas. Em vez de serem apresentadas como teorias de crise marxistas concorrentes por direito próprio, podem ser mais utilmente analisadas como partes componentes da análise de Marx sobre a fragilidade sistémica e as crises económicas no capitalismo.
 \par 
A concorrência intra-sectorial (ver Capítulo {\color{blue}6}) cria uma tendência para o desenvolvimento desigual (desproporcional) entre sectores e uma tendência para a superprodução dentro de cada sector. Em determinadas circunstâncias, possivelmente associadas a uma descida da taxa de lucro, estes processos podem desencadear uma crise geral. Contudo, mais importante do que estas associações é a causa fundamental da crise. Para Marx, as crises capitalistas devem-se, em última análise, à contradição entre a tendência capitalista de desenvolver sem limites as forças produtivas (e expandir a mais-valia que deve ser realizada) e a limitada capacidade social de consumir o produto. A estabilidade económica nestas circunstâncias exige que uma parte crescente do produto seja comprada pelos capitalistas para fins de investimento ou consumo de luxo, o que nem sempre é possível. O capitalismo, portanto, tende sempre a ser instável e propenso à crise. A crise explode quando a produção se desenvolve além da possibilidade de realização lucrativa. Isto pode ocorrer por diversas razões, e o que importa para a explicação de crises específicas é como a sua causa subjacente - a subordinação da produção de valores de uso à produção de mais-valia - se manifesta através da desproporcionalidade, da superprodução, do subconsumo ou da queda da taxa de consumo. lucro.
 \par 
\section{Teoria da acumulação e da crise de Marx}
 \par 
No cenário mais simples possível, suponha que, à medida que o capital se acumula, a relação entre o capital constante e o capital variável adiantado (c ⁄ v) permanece inalterada; portanto, o emprego de mão de obra também deve aumentar. Seria irrealista esperar que a oferta de trabalho pudesse aumentar indefinidamente sem um aumento dos salários. Contudo, se a taxa salarial aumentar mais rapidamente do que a produtividade do trabalho no sector dos bens salariais, haverá uma pressão sobre a rentabilidade e, consequentemente, uma redução na taxa de acumulação (no limite, não haverá acumulação de capital quando os salários subirem tanto). tanto que a produção de qualquer mais-valia está ameaçada). No entanto, à medida que a acumulação diminui, o mesmo acontece com a procura de trabalho, e a pressão ascendente sobre a taxa salarial é reduzida à medida que o poder do trabalho diminui com o desemprego. A rentabilidade é restaurada, e com ela a acumulação, e o ciclo repete-se (este argumento tem de ser qualificado se o rácio c ⁄ v mudar; ver Capítulo {\color{blue}8}).
 \par 
Foi assim que Marx descreveu os ciclos económicos decenais do início do século XIX. Ele também associou-os à renovação sincronizada do capital fixo e à volatilidade do crédito comercial. Significativamente, e em contraste com os economistas políticos clássicos, Marx explicou as flutuações no emprego, nos salários e na rentabilidade por flutuações na taxa de acumulação, e não vice-versa. Ele considerou absurda a doutrina malthusiana de alternar a dizimação e a estimulação do tamanho do proletariado pela reprodução sexual em resposta a salários que caíam abaixo e depois subiam acima de algum nível de subsistência fisiológico. Isto dificilmente poderia explicar ciclos de dez anos. Marx também criticou fortemente o fascínio dos economistas clássicos pela ideia de rendimentos decrescentes na agricultura (ver Capítulo {\color{blue}13}). Em contraste, ele sublinhou a força motriz, sob o capitalismo, do aumento da produtividade da indústria transformadora.
 \par 
Descrita nestes termos agregados, a actividade económica parece flutuar suavemente, impulsionada por alterações na taxa de acumulação. Nada poderia estar mais longe da verdade. O quadro geral pode ocultar enormes variações entre sectores e regiões geográficas da economia. Além disso, já foi demonstrado que o capital tem uma tendência persistente para aumentar a produtividade e expulsar o trabalho vivo da produção.
 \par 
Marx argumenta que, sob o capitalismo, a mudança técnica não só salvaria absolutamente o trabalho vivo, mas também relativamente a outros meios de produção. Isto é conseguido principalmente pelas economias de escala devido às fábricas e ao uso de novas máquinas. Assim, a quantidade de maquinaria por trabalhador tenderá a crescer ao longo do tempo, aumentando a composição técnica do capital (ver Capítulo {\color{blue}8}) e acelerando a produção. Cada trabalhador entrega uma determinada massa de matéria-prima num tempo mais curto, reduzindo a quantidade de trabalho socialmente necessária para produzir cada mercadoria.
 \par 
A expulsão do trabalho vivo da produção pode ser acompanhada por uma expansão global do emprego, devido ao crescimento da produção total. Mas a acumulação competitiva prossegue de forma descoordenada. Em todos os sectores e regiões, os resultados e o emprego não crescerão de forma equilibrada. Com as mudanças tecnológicas haverá ora escassez, ora excesso de mão-de-obra e meios de produção disponíveis. Contudo, a expulsão do trabalho vivo de todos os processos de produção tenderá a produzir um aumento do desemprego (temperado, como explicado acima, pela expansão económica e pela abertura de novos sectores e vias de acumulação). Marx chamou isto de exército de reserva industrial, ou excedente de população - note-se que o excedente é criado e mantido ao longo do tempo pela acumulação de capital, e não através da reprodução biológica dos trabalhadores, como tinha sido sugerido por Malthus. Este excedente inclui uma camada de desempregados permanentes, condenados à pauperização pela combinação do ritmo e das características da acumulação e da sua própria inadequação percebida para o emprego capitalista, seja por causa da idade, género, origem, experiência (ou falta dela), deficiência, ou por qualquer motivo. Quanto maior for o exército de reserva em relação ao emprego, maior será a competição pelo emprego e menores serão os salários. Da mesma forma, quanto maior for o exército de reserva e a sua camada de desempregados permanentes, maior será a extensão da pobreza e da miséria. Marx destacou esta característica do capitalismo como a lei geral da acumulação capitalista.
 \par 
Até agora, analisámos as exigências que a acumulação de capital impõe ao proletariado - uma perturbação constante da vida individual e social. Mudanças específicas podem ser forçadas por coerção política, económica, ideológica e legal, ou induzidas através do mercado por alterações nos salários e nas exigências de competências. Tanto o método específico escolhido como o resultado dependerão da força da organização por trás das duas classes. Além disso, a força da classe capitalista aumenta à medida que a acumulação é acompanhada por uma maior centralização e, simultaneamente, por uma maior força, organização e poder coercitivo do Estado. Marx argumenta que, ao mesmo tempo que o capital é centralizado, também as massas de trabalhadores estão concentradas na produção. Essa organização económica tende a encorajar a organização e a consciência políticas e a luta pela mudança económica e social. À medida que a acumulação progride, a força, a organização e a disciplina do proletariado podem crescer com o desenvolvimento das suas condições materiais.
 \par 
O capitalismo cumpre o papel positivo de desenvolver o potencial produtivo da sociedade, transforma os princípios da eficiência económica em valores universalmente defendidos e cria as condições materiais para o comunismo. Ao mesmo tempo, o capitalismo é o modo de produção mais destrutivo da história. As economias capitalistas são cronicamente instáveis ​​devido às forças conflitantes de extração, realização e acumulação de mais-valia em condições competitivas. Esta instabilidade é estrutural e mesmo as melhores políticas económicas não conseguem evitá-la completamente. Foi demonstrado no Capítulo {\color{blue}6} que a concorrência força todos os capitais a encontrar formas de aumentar a produtividade do trabalho. Isto geralmente envolve mudanças técnicas que aumentam o grau de mecanização, a integração entre os processos de trabalho dentro e entre empresas e a escala potencial de produção. Mas esses processos são sempre desiguais e dispendiosos. Estão associadas a grandes investimentos de capital fixo, à especulação, às mudanças no mercado de trabalho, à desqualificação, ao desemprego estrutural, à falência, à crise e à incapacidade de satisfazer as necessidades básicas de todos, apesar da disponibilidade de meios para as satisfazer.
 \par 
A acumulação também contribui para o desenvolvimento do agente de destruição do capital, os trabalhadores organizados, e fornece a razão para essa destruição: a socialização da produção a ser realizada por um processo de planeamento coordenado e radicalmente democrático que aproveite o potencial produtivo da sociedade. O proletariado cumpre o seu papel histórico, a expropriação da classe dos capitalistas, quando supera as instituições que impõem a disciplina capitalista na produção e na sociedade em geral, e cria alternativas que permitem a abolição da exploração económica.
 \par 
Isto não ocorre necessariamente durante uma crise económica. Pois embora as crises estejam associadas à redução dos lucros, ao elevado desemprego e às pressões descendentes sobre os salários, uma recessão é também um momento em que a classe trabalhadora tende a ser enfraquecida. Além disso, as mudanças num modo de produção, e muito menos a transição de um para outro, não podem ser simplesmente interpretadas apenas a partir das condições económicas, porque dependem de condições políticas e ideológicas. Estes, juntamente com a posição económica do movimento laboral, tendem a ser mais fortes quando as condições são prósperas. Assim, a relação entre a análise económica e a revolução não é apenas complexa, mas depende também de outras influências (isto é explorado mais detalhadamente no Capítulo {\color{blue}15}).
 \par 
\section{Teoria da acumulação e da crise de Marx}
 \par 
A literatura sobre a teoria da crise é extensa, diversificada e fortemente contestada. Uma divisão ocorre entre aqueles que defendem uma teoria de rentabilidade decrescente (e há diferenças entre eles sobre como e porquê) e aqueles que não o fazem. Outras diferenças na literatura reflectem a ênfase relativa na produção, distribuição, troca, finanças e no equilíbrio de poder entre capital e trabalho e dentro da classe capitalista. Cada vez mais, o papel (económico) do Estado tem sido visto como uma fonte ou resposta à crise, embora em deferência à “globalização” isto tenha agora menos proeminência. Isto pode mudar mais uma vez na sequência da crise actual.
 \par 
O próprio Marx nunca discute sistematicamente a sua teoria da crise; ver, no entanto, Marx (1969, cap. {\color{blue} 17 } {\par} , 1972, cap.{\color{blue}20}). A interpretação neste capítulo é baseada em Ben Fine e Laurence Harris (1979, cap.{\color{blue}5}). Para uma visão geral da teoria da crise de Marx, consulte Simon Clarke (1994, 2012), Duncan Foley (1986, cap.{\color{blue}9}), David Harvey (1999, cap.{\color{blue}13}), Michael Heinrich (2013), Michael Howard e John King (1990). ), Michael Perelman (1987), Anwar Shaikh (1978), John Weeks (2010, capítulos 5, {\color{blue}8}) e Pesquisa em Economia Política (vol. {\color{blue} 18 } {\par} , 2000). As teorias subconsumistas são revisadas criticamente por Michael Bleaney (1976) e John Weeks (1982b). Um renascimento do debate sobre a crise foi desencadeado por Robert Brenner (1998, 2002). Para uma amostra da literatura que se seguiu, consulte Historical Materialism (vols 4-5, 1999) e Ben Fine, Costas Lapavitsas e Dimitris Milonakis (1999). Ver também referências no Capítulo {\color{blue}14} para a recente renovação do debate.