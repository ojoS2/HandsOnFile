 
 \chapter{A Colorcoded Genocide}  

 \label{A Colorcoded Genocide}  
 
 
\par
 
 
 \textit{	}  

 
\par
 
 
 
\par
 

 \textbf{\textit{	} }  

 
\par
 
Em 5 de dezembro de 1982, Ronald Reagan se encontrou com o presidente guatemalteco Efraín Ríos Montt em Honduras. Foi uma reunião útil para Reagan. “Bem, eu aprendi muito”, ele disse aos repórteres no Air Force One. “Vocês ficariam surpresos. Eles são todos países individuais.” Também foi uma reunião útil para o mês de Ríos. Reagan o declarou “um homem de grande integridade pessoal... Totalmente dedicado à democracia.” Ele também afirmou que o homem forte guatemalteco estava recebendo “uma má reputação” de organizações de direitos humanos pela campanha de seus militares contra guerrilheiros esquerdistas. No dia seguinte, Daniel Wilkinson nos conta em Silêncio na Montanha: Histórias de Terror, Traição e Esquecimento na Guatemala, um dos pelotões de elite da Guatemala entrou em uma aldeia na selva chamada Las Dos Erred e matou
 {\color{blue} 162}  
Dos seus habitantes,
 {\color{blue} 67}  
Deles crianças. Os soldados “agarraram” bebês e crianças pequenas pelas pernas, balançaram-nos no ar e “bateram” as suas cabeças “contra a parede”. As crianças mais velhas e os adultos foram forçados a “ajoelhar-se à beira de um poço”, onde um único “golpe de marreta” os fez cair abaixo. O pelotão violou então uma seleção de mulheres e raparigas que tinha “guardado para o final”, esmurrando-lhes os estômagos para forçar as grávidas entre elas a abortar. Eles jogaram as mulheres no poço e o encheram de terra, enterrando vivos alguns azarados. “Os únicos restos humanos que os visitantes [posteriores] encontrariam” eram “sangue nas paredes e placentas e cordões umbilicais no chão”.
 {\color{blue} 1}  

 
\par
 
No meio da hagiografia em torno da morte de Reagan em 2004, era provavelmente demasiado esperar que a comunicação social mencionasse o seu encontro com Ríos Month. Afinal, não era Reykjavík. Mas a sombra de Reiquiavique – ou a lançada por Reagan ao falar em frente ao Muro de Berlim – não explica inteiramente o silêncio sobre este encontro entre presidentes. Embora seja tentador atribuir a omissão à amnésia americana, uma causa provável é o profundo equívoco sobre a Guerra Fria sob a qual trabalha a maioria dos americanos. Para o observador casual, a Guerra Fria foi uma luta entre os Estados Unidos e a União Soviética, travada e vencida por justas elegantes em Berlim, de discussões anti-sépticas sobre os arsenais nucleares e da ousadia esperta dos líderes americanos. A América Latina raramente aparece nas discussões populares ou mesmo acadêmicas sobre a Guerra Fria; e enquanto isso acontece, são Cuba, o Chile e a Nicarágua, e não a Guatemala, que recebem a maioria da atenção.
 
\par
 
Mas a América Latina foi um campo de batalha da Guerra Fria tanto como a Europa, e a Guatemala foi a sua linha da frente. Em 1954, os Estados Unidos travaram a sua primeira grande disputa contra o comunismo no hemisfério ocidental, quando derrubaram o presidente democraticamente eleito da Guatemala, Jacobo Arena, que tinha trabalhado em estreita colaboração com o pequeno, mas influente Partido Comunista do país. Esse golpe fez com que um jovem médico argentino fugisse para o México, onde conheceu Fidel Castro. Cinco anos depois, Che Guevara declarou que 1954 lhe ensinara a impossibilidade de uma reforma eleitoral pacífica. Ele prometeu aos seus seguidores que “Cuba não será a Guatemala”. Em 1966, a Guatemala foi novamente o líder, desta vez pioneira nos desaparecimentos que definiriam as guerras sujas da Argentina, Uruguai, Chile e Brasil. Num ataque relâmpago, agentes de segurança treinados pelos EUA capturaram cerca de trinta esquerdistas, torturaram-nos e executaram-nos, e depois atiraram a maioria dos seus cadáveres para o Pacífico. Explicando a operação num memorando confidencial, a CIA escreveu: “A execução destas pessoas não será anunciada e o governo da Guatemala negará que tenham sido detidas”. Com a assinatura, em 1996, de um acordo de paz entre os militares guatemaltecos e as guerrilhas de esquerda, a Guerra Fria Latino-Americana finalmente chegou ao fim – no mesmo local em que começou – tornando a guerra civil na Guatemala a mais longa e letal do hemisfério. Cerca de 200 mil homens, mulheres e crianças morreram, praticamente todos nas mãos dos militares: mais do que o número de mortos na Argentina, no Uruguai, no Chile, no Brasil, na Nicarágua e em El Salvador juntos, e aproximadamente o mesmo número de mortos no Balcãs. Dado que as vítimas eram principalmente índios maias, a Guatemala tem hoje o único exército na América Latina considerado, por uma comissão da verdade patrocinada pelas Nações Unidas, como tendo cometido altos de genocídio.
 {\color{blue} 2}  

 
\par
 
Quando falamos da vitória da América na Guerra Fria, estamos a falar de países como a Guatemala, onde o comunismo foi combatido e derrotado através do massacre em massa de civis. Mas compreender a Guerra Fria exige mais do que contabilizar a contagem de corpos e discriminar as atrocidades. Exige que localizemos esta disputa mais global nos menores lugares, que encontremos sob a compostura duelista da rivalidade das superpotências um conflito sangrento sobre direitos e desigualdade, que vejamos por trás de uma simples história de moralidade do bem triunfando sobre o mal o acordo mais ambivalente que foi – e é – o fim da Guerra Fria. A tarefa, em suma, é mostrar como homens e mulheres fizeram a alta política e a alta política a fez, mostrar que a Guerra Fria foi travada não apenas nas arejadas salas de jogos dos estrategistas nucleares, mas, como escreve Greg Grandin em The Último Massacre Colonial, “nos recintos fechados da família, do sexo e da comunidade”.
 {\color{blue} 3}  

 
\par
 
Grand din. abre seu estudo com uma epígrafe de Sartre: “Uma vitória descrita em detalhes é indistinguível de uma derrota”.
 {\color{blue} 4}  
A vitória aqui referida é singular e já praticamente completa: a dos Estados Unidos sobre o comunismo. Mas as derrotas são diversas e as suas consequências continuam em curso. A primeira é a derrota da esquerda latino-americana, cujas aspirações iam do familiar (tomada armada do poder estatal) ao surpreendente (a criação do capitalismo). A seguir vem a derrota de uma social-democracia continental que teria permitido aos cidadãos exercer uma maior parcela do poder – e receber uma maior parcela dos seus benefícios – do que historicamente lhes era devido. Finalmente, e mais importante, é a derrota desse sonho ainda ilusório de homens e mulheres libertarem-se, graças à sua própria razão e esforço voluntário, dos laços da tradição e da opressão. Este tinha sido o sonho do Iluminismo transatlântico e, durante toda a Guerra Fria, os líderes americanos defenderam o seu nome (ou alguma versão dele) na luta contra o comunismo. Mas na América Latina, foi a esquerda quem assumiu a bandeira do Iluminismo, deixando aos Estados Unidos e aos seus aliados carregar o saco preto do contra-iluminismo. Mais do que impor aos Estados Unidos o fardo indesejado da hipocrisia liberal, a Guerra Fria inspirou-os a abraçar alguns dos ideais mais reacionários e personagens revanchistas do século XX.
 
\par
 
A esquerda latino-americana trouxe o liberalismo e o progresso para uma terra inundada de feudalismo. Já em pleno século XX, os plantadores de café da Guatemala presidiram a um regime de trabalho forçado que era tão medieval como a Rússia czarista. Usando leis contra a vadeagem e a atração do crédito fácil, os proprietários acumularam vastas propriedades e uma força de trabalho de camponeses que essencialmente lhes pertencia. Parecendo um trecho de Dead Souls, de Gogol, um anúncio de 1922 anunciava a venda de “5.000 acres e muitos bosós colonos [trabalhadores endividados] que viajariam para trabalhar em outras plantações”. Enquanto os trabalhadores sindicalizados noutros locais discriminavam o que os seus empregadores podiam ou não lhes pedir, os camponeses da Guatemala eram forçados a prestar uma variedade de serviços obrigatórios, incluindo sexo. Dois fazendeiros da região de Alta Verbal, primos de Boston, usaram seus cozinheiros indianos e moedores de milho para gerar mais de uma dúzia de filhos. “Eles foderam qualquer coisa que se mexesse”, observou um fazendeiro vizinho. Embora as plantações fossem ministrados – com prisões privadas, paliçadas e postes de chicote – os proprietários também dependiam do exército, dos juízes, dos presidentes de câmara e dos polícias locais para forçar os trabalhadores a submeterem-se à sua vontade. Os funcionários públicos detinham rotineiramente camponeses independentes ou fugitivos, enviando-os para plantações ou forçando-os a construir estradas. Um prefeito fez com que vagabundos locais pintassem sua casa. Acima de tudo, é esta visão do poder político como uma forma de propriedade privada que confirma a observação de Grandin de que em 1944 “apenas cinco países da América Latina – México, Uruguai, Chile, Costa Rica e Colômbia – poderiam nominalmente chamar-se democracias”. .”
 {\color{blue} 5}  

 
\par
 
E então, em dois anos, tudo mudou. Em 1946, “apenas cinco países – Paraguai, El Salvador, Honduras, Nicarágua e República Dominicana – não podiam” ser chamados de democracias. Virando a retórica antifascista da Segunda Guerra Mundial contra os antigos regimes do hemisfério, os esquerdistas derrubaram ditadores, legalizaram partidos políticos, construíram sindicatos e ampliaram o direito de voto. Galvanizados pelo New Deal e pela Frente Popular, reformadores como o presidente guatemalteco Juan José Arévalo declararam que “somos socialistas porque vivemos no século XX”. Todo o continente foi atingido por uma combinação de Karl Marx, da Declaração da Independência e de Walt Whitman, mas a Guatemala foi quem mais brilhou. Lá, uma luta de décadas para quebrar a espinha dorsal da aristocracia cafeeira culminou na eleição de Arena, em 1950, que, com a ajuda de um pequeno círculo de conselheiros comunistas, instituiu a Reforma Agrária de 1952. A legislação redistribuiu um milhão e meio de dólares. Hectares para cem mil famílias e também deu aos camponeses uma parcela significativa do poder político.
 
\par
 
Os comitês locais de reforma agrária, compostos principalmente por representantes dos camponeses, contornaram o governo municipal dominado pelos proprietários e forneceram aos camponeses e aos seus sindicatos uma plataforma a partir da qual podiam fazer e conquistar as suas reivindicações por equidade.
 {\color{blue} 6}  

 
\par
 
Indiscutivelmente a mais audaciosa experiência de democracia direita que o continente alguma vez viu, a Reforma Agrária implicou uma ironia central. Os autores da legislação – a maioria deles comunistas – não construíam o socialismo. Eles estavam criando o capitalismo. Eles eram escrupulosos em relação aos direitos de propriedade e ao Estado de direito. Os camponeses tiveram de apoiar as suas reivindicações com extensa documentação; apenas as terras não utilizadas foram expropriadas; e aos proprietários foram garantidos múltiplos direitos de recurso, até ao presidente. A Reforma Agrária impôs um regime de separação de poderes quase tão complicado quanto a Constituição de James Madison. (De acordo com um dos autores comunistas do projeto de lei, “era uma lei burguesa”. Quando ativistas populares reclamaram da lentidão da reforma, Arena respondeu: “Não me importo! Você tem que fazer as coisas direito!”) A Reforma Agrária transformou os camponeses sem terra em proprietários, dando-lhes o poder de negociação para exigir salários mais elevados aos seus empregadores. De acordo com Gran din., os reformadores esperavam que os camponeses se tornassem “consumidores de manufaturas nacionais”, enquanto “os proprietários, historicamente viciados em mão de obra e terra baratas e muitas vezes gratuitas”, seriam forçados a “investir em novas tecnologias” e, assim, “fazer uma economia”. Lucro."
 {\color{blue} 7}  

 
\par
 
Os socialistas da Guatemala fizeram mais do que criar democratas e capitalistas. Eles também transformaram camponeses em cidadãos. Enquanto liberais e conservadores há muito afirmam que as ideologias esquerdistas reduzem seus adeptos a autômatos, os ideais e movimentos esquerdistas despertaram os camponeses para seu próprio poder, dando-lhes amplas oportunidades de falar por si e agir em seu próprio nome. Efraín Reyes AAZ, por exemplo, foi um organizador camponês maia, nascido no mesmo ano da Revolução Bolchevique. "Se eu não tivesse estudado marx., eu seria chichi NI lemo nade [nem álcool, nem limonada]", diz Reyes. "Eu não seria nada. Mas a leitura me nutriu e aqui estou. Eu poderia morrer hoje e ninguém poderia tirar isso de mim." Onde outros camponeses raramente se aventuravam além de suas plantações, o Partido Comunista inspirou Reyes a viajar para o México e Cuba, e ele retornou à Guatemala com a convicção de que "todo revolucionário carrega um mundo inteiro em sua cabeça". O Partido Comunista não exigiu que Reyes desistisse de tudo o que sabia; deu-lhe ampla liberdade para sincronizar o indígena e o europeu, criando um “marxismo maia” que era tão flexível quanto o marxismo híbrido desenvolvido na Europa Central entre as guerras. Quando os anticomunistas puseram fim a esse despertar democrático em 1954, era tanto o apetite recém-descoberto do camponês por pensar e falar quanto a terra expropriada do fazendeiro que os preocupava. Como vimos na introdução, o arcebispo da Guatemala reclamou que os adventistas enviaram camponeses “dotados de facilidade com as palavras” para a capital do país, onde foram “ensinados... a falar em público”.
 {\color{blue} 8}  

 
\par
 
Na esperança de quebrar este exército de pensamento e conversa, os Guerreiros Frios da Guatemala fundiram uma aversão romântica ao mundo moderno com as tecnologias mais atualizadas de propaganda e violência, tornando o seu esforço mais semelhante ao fascismo do que a qualquer luta pela democracia liberal. Trabalhando através da Igreja Católica, o regime que substituiu Arena fez com que os prelados pregassem o evangelho contra o comunismo e o socialismo, e também contra a democracia, o liberalismo e o feminismo. Voltando à retórica de oposição à Revolução Francesa, os Padres da Igreja caracterizaram a Guerra Fria como uma luta entre a Cidade de Deus e “a cidade do diabo encarnado” e queixaram-se de que Arena, “longe de unir o nosso povo no seu avanço em direção ao progresso”, “os desorganiza em bandos opostos”. Os Arbencistas, alegavam, eram “corruptores profissionais da alma feminina”, elevando mulheres com “dons de proselitismo ou liderança” a “posições altas e bem remuneradas na burocracia extra-carvão”. Porque os anciãos da Igreja eram por vezes demasiado meticulosos para estimular as massas, emigrados da Espanha republicana, que eram partidários de Franco e Mussolini, tomavam frequentemente o seu lugar, apelando a uma fé mais extática para contrariar o apelo do comunismo: “Não queremos uma constipação. Catolicismo. Queremos santidade, santidade ardente, grande e alegre. . . Intransigente e fanático.”
 {\color{blue} 9}  

 
\par
 
Enquanto os ideais dos Guerreiros Frios pareciam retrógrados, as suas armas – fornecidas pelos Estados Unidos – e estratégias militares olhavam para o futuro. (Na verdade, uma das principais justificações dos americanos para as suas intervenções durante a Guerra Fria foi que o envolvimento dos EUA conteria não só o comunismo, mas também, nas palavras do Departamento de Estado, uma “contra-insurgência descontrolada” de direita. De um selvagem “terror branco”, as forças de segurança treinadas pelos EUA trabalhariam com a “esquerda democrática” anticomunista para combater uma Guerra Fria mais “racional”, “moderna” e “profissional”.) Durante o golpe de 1954, a CIA virou para a Madison Avenue, a socialização pop e a literatura da psicologia de massa para criar a ilusão de oposição em larga escala à Arena. Programas de rádio espalharam rumores de uma resistência clandestina, incitando oficiais do exército vacilantes a abandonar o juramento ao presidente democraticamente eleito. Nas décadas seguintes, a CIA dotou a Guatemala de uma agência de inteligência interna centralizada, equipada com telefones, rádios, câmaras, máquinas de escrever, papel carbono, armários longos, equipamento de vigilância – e armas, munições e explosivos. A CIA também reuniu os militares e a polícia em elegantes centros de comando urbanos, onde a informação poderia ser rapidamente analisada, distribuída, assolada e arquivada para utilização posterior. Depois de estes esforços terem alcançado os resultados mais espetaculares, com o desaparecimento, em 1966, da última geração de esquerdistas pacíficos da Guatemala, os guerrilheiros começaram a organizar seriamente a oposição armada nas zonas rurais.
 
\par
 
Em resposta, o regime lançou no campo um exército tão modernizado — e tão bem treinado e equipado pelos Estados Unidos — que em 1981 foi capaz de conduzir o primeiro genocídio codificado por cores da história: “Analistas militares marcaram comunidades e regiões conforme as cores. Branco poupou aqueles que se pensava não ter influência rebelde. Rosa identificou áreas nas quais os insurgentes tinham presença limitada; guerrilheiros suspeitos e seus apoiadores deveriam ser mortos, mas as comunidades deixadas de pé. Vermelho não deu trégua: todos deveriam ser executados e as aldeias arrasadas.”
 {\color{blue} 10}  

 
\par
 
Referindo-se a um massacre militar de índios em 1978 em Panos, uma cidade ribeirinha no Vale Polo chip, o título do livro de Gran din. evoca esta mistura de elementos modernos e actinodermos. Em maio
 {\color{blue} 29}  
Naquele ano, cerca de quinhentos camponeses maias reuniram-se no centro da cidade para pedir ao prefeito que ouvisse as suas queixas contra os proprietários locais, que seriam apresentadas por uma delegação sindical da capital. Atirando contra os manifestantes, um destacamento militar matou em algum lugar entre
 {\color{blue} 34}  
e
 {\color{blue} 100}  
Homens, mulheres e crianças. À primeira vista, o massacre parece apenas uma repetição do passado colonial da Guatemala: humildes peticionários indianos pedem aos funcionários públicos que intercedam em seu nome contra os governantes locais; as forças governamentais aliadas aos proprietários respondem com violência; os índios acabam flutuando rio abaixo. Olhando mais de perto, o massacre traz todas as marcas do século XX. Os índios eram liderados por ativistas de esquerda – um deles uma mulher indígena – treinados por organizadores comunistas clandestinos. Trabalharam com sindicatos, sediados na capital, refletindo a tentativa da esquerda de nacionalizar as queixas locais. Por sua vez, os soldados que atacavam os camponeses eram mais do que uma polícia local defendendo os interesses dos proprietários. Eram um contingente do exército recém-treinado da Guatemala, falavam fluentemente do anticomunismo e empunhavam espingardas de assalto Gall de fabrico israelita, sugerindo não apenas a nacionalização, mas também a internacionalização das lutas tradicionais da Guatemala pela terra e pelo trabalho.
 {\color{blue} 11}  

 
\par
 
Embora a Guerra Fria na América Latina tenha começado como uma negociação tensa entre o racionalismo americano e o revanchismo latino, ela terminou com os Estados Unidos se inclinando em direção ao último. Em uma reprise da lendária jornada ao coração das trevas, autoridades americanas retornaram de suas viagens ao sul ecoando as vozes mais sombrias do contra iluminismo. Um oficial da embaixada escreveu a seus superiores em casa: "Afinal, o homem não foi um selvagem desde o início dos tempos, então não fiquemos muito enjoados com o terror. Eu literalmente ouvi esses argumentos do nosso povo." Um funcionário da CIA aqui pediu a seus colegas que abandonassem todas as tentativas de persuasão em massa na Guatemala e, em vez disso, direcionassem seus esforços para o "coração, o estômago e o fígado (medo)". Buscando desestabilizar o Chile de Allende, outro homem da CIA proclamou: "Não podemos nos esforçar para incendiar o mundo se o próprio Chile for um lago plácido. O combustível para o fogo deve vir de dentro do Chile. Portanto, a estação deve empregar todos os estratagemas, todos os truques, por mais bizarros que sejam, para criar essa resistência interna.” Como Gran din. escreve, “Vontade de incendiar o mundo... Fé no lado noturno da alma, desprezo pela temperança democrática e pelo procedimento parlamentar: essas qualidades são geralmente atribuídas aos oponentes da civilidade liberal, da tolerância e do pluralismo — não aos seus defensores.”
 {\color{blue} 12}  
Com esta observação plangente, Grand din. conclui a sua notável história, sugerindo que se poderia dizer que a maior derrota da Guerra Fria foi a da própria América.
 
\par
  
 
999999
