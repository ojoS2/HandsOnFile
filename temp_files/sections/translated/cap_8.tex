\chapter{Um genocídio codificado por cores}\label{Um genocídio codificado por cores}
 \par 
Em {\color{blue}5} de dezembro de 1982, Ronald Reagan conheceu o presidente da Guatemala, Efraín Ríos Montt, em Honduras. Foi uma reunião útil para Reagan. “Bem, aprendi muito”, disse ele aos repórteres no Força Aérea Um. "Você ficaria surpreso. São todos países individuais.” Foi também um encontro útil para Ríos Montt. Reagan declarou-o “um homem de grande integridade pessoal. . . Totalmente dedicado à democracia.” Ele também alegou que o homem forte da Guatemala estava sendo “repreendido” por organizações de direitos humanos pela campanha de seus militares contra as guerrilhas de esquerda. No dia seguinte, Daniel Wilkinson conta-nos em Silêncio na Montanha: Histórias de Terror, Traição e Esquecimento na Guatemala, um dos pelotões de elite da Guatemala entrou numa aldeia na selva chamada Las Dos Erres e matou {\color{blue}162} dos seus habitantes, {\color{blue}67} dos quais crianças. Os soldados “agarraram” bebés e crianças pequenas pelas pernas, balançaram-nos no ar e “bateram” as suas cabeças “contra a parede”. As crianças mais velhas e os adultos foram forçados a “ajoelhar-se à beira de um poço”, onde um único “golpe de marreta” os fez cair abaixo. O pelotão estuprou então uma seleção de mulheres e meninas que havia “guardado para o final”, esmurrando-as no estômago.
 \par 
Ordem para forçar as grávidas entre elas a abortar. Eles jogaram as mulheres no poço e o encheram de terra, enterrando vivos alguns azarados. “Os únicos restos humanos que os visitantes [posteriores] encontrariam” eram “sangue nas paredes e placentas e cordões umbilicais no chão”.{\color{blue}1}
 \par 
No meio da hagiografia em torno da morte de Reagan em 2004, era provavelmente demasiado esperar que a comunicação social mencionasse o seu encontro com Ríos Montt. Afinal, não era Reykjavík. Mas a sombra de Reiquiavique – ou a lançada por Reagan ao falar em frente ao Muro de Berlim – não explica inteiramente o silêncio sobre este encontro entre presidentes. Embora seja tentador atribuir a omissão à amnésia americana, uma causa mais provável é o profundo equívoco sobre a Guerra Fria sob a qual trabalha a maioria dos americanos. Para o observador casual, a Guerra Fria foi uma luta entre os Estados Unidos e a União Soviética, travada e vencida através de justas elegantes em Berlim, de discussões anti-sépticas sobre os arsenais nucleares e da ousadia esperta dos líderes americanos. A América Latina raramente aparece nas discussões populares ou mesmo académicas sobre a Guerra Fria; e na medida em que isso acontece, são Cuba, o Chile e a Nicarágua, e não a Guatemala, que recebem a maior parte da atenção.
 \par 
Mas a América Latina foi um campo de batalha da Guerra Fria tanto como a Europa, e a Guatemala foi a sua linha da frente. Em 1954, os Estados Unidos travaram a sua primeira grande disputa contra o comunismo no hemisfério ocidental, quando derrubaram o presidente democraticamente eleito da Guatemala, Jacobo Arbenz, que tinha trabalhado em estreita colaboração com o pequeno mas influente Partido Comunista do país. Esse golpe fez com que um jovem médico argentino fugisse para o México, onde conheceu Fidel Castro. Cinco anos depois, Che Guevara declarou que 1954 lhe ensinara a impossibilidade de uma reforma eleitoral pacífica. Ele prometeu aos seus seguidores que “Cuba não será a Guatemala”. Em 1966, a Guatemala foi novamente a pioneira, desta vez pioneira nos desaparecimentos que viriam a definir as guerras sujas da Argentina,
 \par 
Uruguai, Chile e Brasil. Num ataque relâmpago, agentes de segurança treinados pelos EUA capturaram cerca de trinta esquerdistas, torturaram-nos e executaram-nos, e depois atiraram a maior parte dos seus cadáveres para o Pacífico. Explicando a operação num memorando confidencial, a CIA escreveu: “A execução destas pessoas não será anunciada e o governo da Guatemala negará que tenham sido detidas”. Com a assinatura, em 1996, de um acordo de paz entre os militares guatemaltecos e as guerrilhas de esquerda, a Guerra Fria Latino-Americana finalmente chegou ao fim – no mesmo local em que começou – tornando a guerra civil na Guatemala a mais longa e letal do hemisfério. Cerca de {\color{blue}200} mil homens, mulheres e crianças morreram, praticamente todos nas mãos dos militares: mais do que o número de mortos na Argentina, no Uruguai, no Chile, no Brasil, na Nicarágua e em El Salvador juntos, e aproximadamente o mesmo número de mortos no Balcãs. Dado que as vítimas eram principalmente índios maias, a Guatemala tem hoje o único exército na América Latina considerado, por uma comissão da verdade patrocinada pelas Nações Unidas, como tendo cometido actos de genocídio.{\color{blue}2}
 \par 
Quando falamos sobre a vitória da América na Guerra Fria, estamos falando de países como a Guatemala, onde o comunismo foi combatido e derrotado por meio do massacre em massa de civis. Mas entender a Guerra Fria requer mais do que contar contagens de corpos e listar atrocidades. Exige que localizemos essa mais global das disputas nos menores lugares, para encontrar sob a compostura duelante da rivalidade entre superpotências um conflito sangrento sobre direitos e desigualdade, para ver por trás de um simples conto moral do bem triunfando sobre o mal o acordo mais ambivalente que foi — e é — o fim da Guerra Fria. A tarefa, em suma, é mostrar como homens e mulheres fizeram alta política e a alta política os fez, para mostrar que a Guerra Fria foi travada não apenas nas salas de jogos arejadas dos estrategistas nucleares, mas, como Greg Grandin escreve em The Last Colonial Massacre, “nos aposentos fechados da família, do sexo e da comunidade”.{\color{blue}3}
 \par 
Grandin abre seu estudo com uma epígrafe de Sartre: “Uma vitória descrita em detalhes é indistinguível de uma derrota”. {\color{blue}4} A vitória aqui referida é singular e já praticamente completa: a dos Estados Unidos sobre o comunismo. Mas as derrotas são diversas e as suas consequências ainda estão em curso. A primeira é a derrota da esquerda latino-americana, cujas aspirações iam do familiar (tomada armada do poder estatal) ao surpreendente (a criação do capitalismo). A seguir vem a derrota de uma social-democracia continental que teria permitido aos cidadãos exercer uma maior parcela do poder – e receber uma maior parcela dos seus benefícios – do que historicamente lhes era devido. Finalmente, e mais importante, é a derrota desse sonho ainda ilusório de homens e mulheres libertarem-se, graças à sua própria razão e esforço voluntário, dos laços da tradição e da opressão. Este tinha sido o sonho do Iluminismo transatlântico e, durante toda a Guerra Fria, os líderes americanos defenderam o seu nome (ou alguma versão dele) na luta contra o comunismo. Mas na América Latina, foi a esquerda quem assumiu a bandeira do Iluminismo, deixando os Estados Unidos e os seus aliados a carregar o saco preto do contra-Iluminismo. Mais do que impor aos Estados Unidos o fardo indesejado da hipocrisia liberal, a Guerra Fria inspirou-os a abraçar alguns dos ideais mais reaccionários e personagens revanchistas do século XX.
 \par 
A esquerda latino-americana trouxe o liberalismo e o progresso para uma terra inundada de feudalismo. Já em pleno século XX, os plantadores de café da Guatemala presidiram a um regime de trabalho forçado que era tão medieval como a Rússia czarista. Usando leis contra a vadiagem e a atração do crédito fácil, os proprietários acumularam vastas propriedades e uma força de trabalho de camponeses que essencialmente lhes pertencia. Parecendo um trecho de Dead Souls, de Gogol, um anúncio de 1922 anunciava a venda de “5.{\color{blue}000} acres e muitos bozos colons [trabalhadores endividados] que viajariam para trabalhar em outras plantações”. Enquanto os trabalhadores sindicalizados em outros lugares detalhavam quais eram suas
 \par 
Os empregadores podiam e não podiam exigir-lhes, os camponeses da Guatemala foram forçados a fornecer uma variedade de serviços obrigatórios, incluindo sexo. Dois fazendeiros da região de Alta Verapaz, primos de Boston, usaram seus cozinheiros indianos e moedores de milho para gerar mais de uma dúzia de filhos. “Eles foderam qualquer coisa que se mexesse”, observou um fazendeiro vizinho. Embora as plantações fossem miniestados – com prisões privadas, paliçadas e postes de chicote – os proprietários também dependiam do exército, dos juízes, dos presidentes de câmara e dos polícias locais para forçar os trabalhadores a submeterem-se à sua vontade. Os funcionários públicos detinham rotineiramente camponeses independentes ou fugitivos, enviando-os para plantações ou forçando-os a construir estradas. Um prefeito fez com que vagabundos locais pintassem sua casa. Acima de tudo, é esta visão do poder político como uma forma de propriedade privada que confirma a observação de Grandin de que em 1944 “apenas cinco países da América Latina – México, Uruguai, Chile, Costa Rica e Colômbia – poderiam nominalmente chamar-se democracias”. .”{\color{blue}5}
 \par 
E então, em dois anos, tudo mudou. Em 1946, “apenas cinco países – Paraguai, El Salvador, Honduras, Nicarágua e República Dominicana – não podiam” ser chamados de democracias. Virando a retórica antifascista da Segunda Guerra Mundial contra os antigos regimes do hemisfério, os esquerdistas derrubaram ditadores, legalizaram partidos políticos, construíram sindicatos e ampliaram o direito de voto. Galvanizados pelo New Deal e pela Frente Popular, reformadores como o presidente guatemalteco Juan José Arévalo declararam que “somos socialistas porque vivemos no século XX”. Todo o continente foi atingido por uma combinação de Karl Marx, da Declaração da Independência e de Walt Whitman, mas a Guatemala foi quem mais brilhou. Lá, uma luta de décadas para quebrar a espinha dorsal da aristocracia cafeeira culminou na eleição de Arbenz, em 1950, que, com a ajuda de um pequeno círculo de conselheiros comunistas, instituiu a Reforma Agrária de 1952. A legislação redistribuiu um milhão e meio de dólares. hectares para cem mil famílias e também deu aos camponeses uma parcela significativa do poder político.
 \par 
Os comités locais de reforma agrária, compostos principalmente por representantes dos camponeses, contornaram o governo municipal dominado pelos proprietários e forneceram aos camponeses e aos seus sindicatos uma plataforma a partir da qual podiam fazer e conquistar as suas reivindicações por equidade.{\color{blue}6}
 \par 
Indiscutivelmente a mais audaciosa experiência de democracia directa que o continente alguma vez viu, a Reforma Agrária implicou uma ironia central. Os autores da legislação – a maioria deles comunistas – não estavam a construir o socialismo. Eles estavam criando o capitalismo. Eles eram escrupulosos em relação aos direitos de propriedade e ao Estado de direito. Os camponeses tiveram de apoiar as suas reivindicações com extensa documentação; apenas as terras não utilizadas foram expropriadas; e aos proprietários foram garantidos múltiplos direitos de recurso, até ao presidente. A Reforma Agrária impôs um regime de separação de poderes quase tão complicado quanto a Constituição de James Madison. (De acordo com um dos autores comunistas do projeto de lei, “era uma lei burguesa”. Quando ativistas populares reclamaram da lentidão da reforma, Arbenz respondeu: “Não me importo! Você tem que fazer as coisas direito!”) A Reforma Agrária transformou os camponeses sem terra em proprietários, dando-lhes o poder de negociação para exigir salários mais elevados aos seus empregadores. De acordo com Grandin, os reformadores esperavam que os camponeses se tornassem “consumidores de manufaturas nacionais”, enquanto “os proprietários, historicamente viciados em mão de obra e terra baratas e muitas vezes gratuitas”, seriam forçados a “investir em novas tecnologias” e, assim, “obter lucro”. .”{\color{blue}7}
 \par 
Os socialistas da Guatemala fizeram mais do que criar democratas e capitalistas. Eles também transformaram os camponeses em cidadãos. Embora os liberais e os conservadores há muito que afirmem que as ideologias de esquerda reduzem os seus adeptos a autómatos, os ideais e movimentos de esquerda despertaram os camponeses para o seu próprio poder, dando-lhes amplas oportunidades de falar por si próprios e de agir em seu próprio nome. Efraín Reyes Maaz, por exemplo, foi um organizador camponês maia, nascido no mesmo ano da Revolução Bolchevique. “Se eu não tivesse estudado
 \par 
Marx, eu seria chichi ni limonada [nem álcool nem limonada]”, diz Reyes. “Eu não seria nada. Mas a leitura me alimentou e aqui estou. Eu poderia morrer hoje e ninguém poderia tirar isso de mim.” Onde outros camponeses raramente se aventuravam para além das suas plantações, o Partido Comunista inspirou Reyes a viajar para o México e Cuba, e ele regressou à Guatemala com a convicção de que “cada revolucionário carrega um mundo inteiro na sua cabeça”. O Partido Comunista não exigiu que Reyes desistisse de tudo o que sabia; deu-lhe ampla liberdade para sincronizar o indígena e o europeu, criando um “marxismo maia” que era tão flexível quanto o marxismo híbrido desenvolvido na Europa Central entre as guerras. Quando os anticomunistas puseram fim a este despertar democrático em 1954, era tanto o recém-descoberto apetite dos camponeses por pensar e falar como as terras expropriadas dos proprietários que os preocupavam. Como vimos na introdução, o arcebispo da Guatemala queixou-se de que os arbencistas enviavam camponeses “dotados de facilidade com as palavras” para a capital do país, onde eram “ensinados. . . Para falar em público.”{\color{blue}8}
 \par 
Na esperança de quebrar este exército de pensamento e conversa, os Guerreiros Frios da Guatemala fundiram uma aversão romântica ao mundo moderno com as tecnologias mais actualizadas de propaganda e violência, tornando o seu esforço mais semelhante ao fascismo do que a qualquer luta pela democracia liberal. Trabalhando através da Igreja Católica, o regime que substituiu Arbenz fez com que prelados pregassem o evangelho contra o comunismo e o socialismo, e também contra a democracia, o liberalismo e o feminismo. Voltando à retórica de oposição à Revolução Francesa, os Padres da Igreja caracterizaram a Guerra Fria como uma luta entre a Cidade de Deus e “a cidade do diabo encarnado” e queixaram-se de que Arbenz, “longe de unir o nosso povo no seu avanço em direção ao progresso”, “os desorganiza em bandos opostos”. Os Arbencistas, afirmavam, eram “profissionais
 \par 
Corruptores da alma feminina”, elevando mulheres com “dons de proselitismo ou liderança” a “posições altas e bem pagas na burocracia fora do carvão”. Como os anciãos da Igreja eram às vezes muito exigentes para incitar as massas, emigrantes da Espanha republicana, que eram parciais a Franco e Mussolini, frequentemente tomavam seus lugares, clamando por uma fé mais extática para combater o apelo do comunismo: “Não queremos um catolicismo frio. Queremos santidade, santidade ardente, grande e alegre... Intransigente e fanático.”{\color{blue}9}
 \par 
Enquanto os ideais dos Guerreiros Frios pareciam retrógrados, as suas armas – fornecidas pelos Estados Unidos – e estratégias militares olhavam para o futuro. (Na verdade, uma das principais justificações dos americanos para as suas intervenções durante a Guerra Fria foi que o envolvimento dos EUA conteria não só o comunismo, mas também, nas palavras do Departamento de Estado, uma “contra-insurgência descontrolada” de direita. de um selvagem “terror branco”, as forças de segurança treinadas pelos EUA trabalhariam com a “esquerda democrática” anticomunista para combater uma Guerra Fria mais “racional”, “moderna” e “profissional”.) Durante o golpe de 1954, o A CIA recorreu à Madison Avenue, à sociologização pop e à literatura da psicologia de massas para criar a ilusão de oposição em larga escala a Arbenz. Programas de rádio espalharam rumores de uma resistência clandestina, incitando oficiais do exército vacilantes a abandonar o juramento ao presidente democraticamente eleito. Nas décadas seguintes, a CIA dotou a Guatemala de uma agência de inteligência interna centralizada, equipada com telefones, rádios, câmaras, máquinas de escrever, papel carbono, armários longos, equipamento de vigilância – e armas, munições e explosivos. A CIA também reuniu os militares e a polícia em elegantes centros de comando urbanos, onde a informação poderia ser rapidamente analisada, distribuída, actuada e arquivada para utilização posterior. Depois de estes esforços terem alcançado os resultados mais espectaculares, com o desaparecimento, em 1966, da última geração de esquerdistas pacíficos da Guatemala, os guerrilheiros começaram a organizar seriamente a oposição armada nas zonas rurais.
 \par 
Em resposta, o regime lançou no campo um exército tão modernizado – e tão bem treinado e equipado pelos Estados Unidos – que em 1981 foi capaz de conduzir o primeiro genocídio codificado por cores da história: “Analistas militares marcaram comunidades e regiões de acordo com as cores. White poupou aqueles que se pensava não terem influência rebelde. Pink identificou áreas nas quais os insurgentes tinham presença limitada; os supostos guerrilheiros e os seus apoiantes seriam mortos, mas as comunidades permaneceram de pé. Red não deu trégua: todos seriam executados e as aldeias arrasadas.”{\color{blue}10}
 \par 
Referindo-se a um massacre militar de índios em Panzós, uma cidade ribeirinha no Vale Polochic, em 1978, o título do livro de Grandin evoca esta mistura de elementos modernos e antimodernos. Em {\color{blue}29} de maio daquele ano, cerca de quinhentos camponeses maias reuniram-se no centro da cidade para pedir ao prefeito que ouvisse as suas queixas contra os proprietários locais, que seriam apresentadas por uma delegação sindical da capital. Atirando contra os manifestantes, um destacamento militar matou entre {\color{blue}34} e {\color{blue}100} homens, mulheres e crianças. À primeira vista, o massacre parece apenas uma repetição do passado colonial da Guatemala: humildes peticionários indianos pedem aos funcionários públicos que intercedam em seu nome contra os governantes locais; as forças governamentais aliadas aos proprietários respondem com violência; Os índios acabam flutuando rio abaixo. Olhando mais de perto, o massacre traz todas as marcas do século XX. Os índios eram liderados por ativistas de esquerda – um deles uma mulher indígena – treinados por organizadores comunistas clandestinos. Trabalharam com sindicatos, sediados na capital, reflectindo a tentativa da esquerda de nacionalizar as queixas locais. Por sua vez, os soldados que atacavam os camponeses eram mais do que uma polícia local defendendo os interesses dos proprietários. Eram um contingente do exército recém-treinado da Guatemala, falavam fluentemente do anticomunismo e empunhavam espingardas de assalto Galil de fabrico israelita, sugerindo não apenas a nacionalização, mas também a internacionalização das lutas tradicionais da Guatemala pela terra e pelo trabalho.{\color{blue}11}
 \par 
Embora a Guerra Fria na América Latina tenha começado como uma negociação tensa entre o racionalismo americano e o revanchismo latino, terminou com os Estados Unidos a inclinarem-se para este último. Numa reprise da lendária viagem ao coração das trevas, as autoridades norte-americanas regressaram das suas viagens para o sul, ecoando as vozes mais sombrias do contra-Iluminismo. Um oficial da embaixada escreveu aos seus superiores no seu país: “Afinal, Mann nunca foi um selvagem desde o início dos tempos, por isso não nos preocupemos muito com o terror. Eu literalmente ouvi esses argumentos do nosso povo.” Um funcionário da CIA instou os seus colegas a abandonarem todas as tentativas de persuasão em massa na Guatemala e, em vez disso, dirigirem os seus esforços para “o coração, o estômago e o fígado (medo)”. Procurando desestabilizar o Chile de Allende, outro homem da CIA proclamou: “Não podemos tentar inflamar o mundo se o próprio Chile for um lago plácido. O combustível para o fogo deve vir de dentro do Chile. Portanto, a emissora deveria empregar todos os estratagemas, todos os estratagemas, por mais bizarros que fossem, para criar essa resistência interna.” Como escreve Grandin: “Vontade de incendiar o mundo. . . Fé no lado noturno da alma, desprezo pela temperança democrática e pelo procedimento parlamentar: estas qualidades são geralmente atribuídas aos oponentes da civilidade liberal, da tolerância e do pluralismo – e não aos seus defensores.” {\color{blue}12} Com esta observação plangente, Grandin conclui a sua notável história, sugerindo que se poderia dizer que a maior derrota da Guerra Fria foi a da própria América.