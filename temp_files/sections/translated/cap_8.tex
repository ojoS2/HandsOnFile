\begin{figure}
	\centering
	\includegraphics[width=1.\textwidth]{temp\_files/images/UP\_logo.png }
	\caption{Rosa Luxemburgo (1871-1919): Uma das mais importantes teóricas sociais do marxismo europeu. Filósofa, economista e pacifista, Luxemburgo obteve seu doutorado na Universidade de Zurique em 1897. Uma oradora incrível e uma escritora apaixonada, ela foi considerada uma luminar entre os líderes socialistas alemães de sua época. Após a eclosão da Primeira Guerra Mundial, ela rompeu com seus colegas do Partido Social-Democrata Alemão e ajudou a fundar a Liga Spartacus, que eventualmente se tornou o Partido Comunista Alemão. Ela foi assassinada em 1919 junto com seu colega, Karl Liebknecht. Cortesia da Rosa Luxemburgo Stiftung.}
	\label{ }
\end{figure}
 \par 
\chapter{DAS BARRICADAS À VOTAÇÃO}\label{DAS BARRICADAS À VOTAÇÃO}
 \par 
CAIXAS: SOBRE CIDADANIA Em 2006, comprei um mapa que todos deveriam possuir. Pro- = cortado pelos cartógrafos de Oxford, a “Linha do tempo da história mundial: a ascensão e queda das nações” é um gráfico codificado por cores das diferentes civilizações e estados de 3000 aC a {\color{blue}2}.{\color{blue}000} cr. O eixo x do mapa é a linha do tempo que abrange cinco mil anos de história humana. O eixo y do mapa apresenta seis formações geográficas cujo tamanho relativo corresponde à quantidade de história escrita que temos sobre elas: as Américas, a África Subsaariana, a Europa, o Norte de África e o Médio Oriente, a Ásia e a Australásia. O que adoro neste mapa, e o que o torna tão útil como ferramenta de ensino, é que, ao mostrar que os impérios são temporários, ele fornece um infográfico para a possibilidade de mudança social. Tenho ensinado jovens há quase vinte anos. Fico maravilhado com como os millennials acreditam que o mundo é fixo e estático e com que facilidade esta visão do mundo os leva para um caminho de desespero político. Como cresci durante as últimas décadas da Guerra Fria e tinha dezenove anos quando o Muro de Berlim caiu e vinte e um quando a União Soviética implodiu, passei os meus vinte e trinta anos com a clara compreensão de que grandes mudanças políticas não são apenas possíveis
 \par 
157
 \par 
158
 \par 
TTTTTT, mas que eles podem vir quando você menos espera. Na verdade, no verão de 1989, decidi abandonar a faculdade para poder ver o mundo antes que tudo fosse destruído em uma guerra nuclear. A ameaça de destruição mútua assegurada era tão palpável no final dos anos 1980 que não fazia sentido tentar viver uma vida comum. Eu certamente não queria ficar preso em uma sala de aula fazendo uma prova de química quando as bombas começassem a cair. Comprei uma passagem só de ida para a Espanha e deixei os Estados Unidos no final de setembro de 1989. Menos de dois meses depois, a Guerra Fria acabou. Simples assim.
 \par 
No verão de 1990, enquanto eu amochilava pela Europa Oriental, lembro-me da euforia e da sensação de possibilidades infinitas. Os jovens estavam especialmente exultantes por terem um futuro livre e próspero, aproveitando oportunidades negadas a seus pais e avós. Naqueles meses inebriantes, muitas pessoas ainda acreditavam que as ruas de Nova York e Londres eram pavimentadas com ouro, e que a democracia e o capitalismo inaugurariam consumidor Xanadu de jeans Levi's ilimitados e perfumes Cacharel. Mais tarde, quando comecei a pesquisar na região, ouvi inúmeras histórias de suicídios e atos desesperados de automutilação cometidos nos poucos dias antes da queda do muro em {\color{blue}9} de novembro de 1989. Ao fazerem um balanço de suas vidas, esses homens e mulheres acreditavam que seu mundo nunca mudaria. Embora houvesse protestos crescentes em toda a Europa Oriental, poucos esperavam a escala da transformação que viria. Como eles poderiam saber que seu mundo seria tão diferente apenas alguns dias depois? Se tivessem aguentado por mais quarenta e oito horas, poderiam ter vivido o resto de suas vidas em circunstâncias radicalmente alteradas daquelas em que se sentiam tão presos. Se ao menos pudessem ter acreditado
 \par 
KRISTEN R. GHODSEE que esse presente em particular nunca se estende infinitamente para o futuro.
 \par 
Pessoas nascidas depois de 1989 chegaram a um mundo onde o capitalismo era triunfante. Foi o único sistema político e econômico que restou de pé após o turbulento século XX, com Francis Fukuyama declarando que a humanidade havia atingido o "fim da história", o zênite do nosso desenvolvimento civilizacional. Se eles se sentissem desencantados com o caos causado pelo neoliberalismo desenfreado, não havia alternativas. Sua consciência política foi forjada em um mundo onde a hegemonia americana parecia ossificada e incontestada. Para citar uma frase dos Borg: a resistência era inútil; você seria assimilado, gostasse ou não. O domínio ideológico do democrocapitalismo gerou apatia e inércia em muitos jovens, que repetiam o mesmo mantra ano após ano: "Nada vai mudar. É assim que as coisas são."!
 \par 
Sempre que ouço essa frase, ou alguma permutação dela, sempre pego minha Linha do Tempo da História Mundial da Oxford Cartographers e tento fazer os alunos pensarem sobre o que significa dizer que nada pode mudar. No meio deste mapa há uma forma enorme, laranja-avermelhada, que representa o Império Romano, com sua história de mil anos e seu domínio geográfico sobre a maioria da Europa, Norte da África e oriente Médio. O colapso de Roma mergulhou a Europa na Idade das Trevas, e a queda do Império Romano é simbolizada por uma linha abrupta que demarca o limite temporal. Imagine, digo aos alunos, que você nasceu em {\color{blue}456} EC, nos arredores da cidade de Roma. Você fez vigésimo aniversário em 1º de setembro de {\color{blue}476} e passou a vida inteira em um império que existiu por quase um milênio. Claro,
 \par 
159
 \par 
160 havia problemas com os bárbaros no Norte e todos os tipos de intrigas e conspirações minando a estabilidade política, mas esta era Roma. Ela havia sobrevivido as crises muito maiores do que alguns exércitos de visigodos raivosos.
 \par 
Você consegue imaginar como seria em {\color{blue}4} de setembro de 476, quando Flávio Odoacro depôs Rômulo Augusto no que é geralmente considerado o dia específico que marca o fim do Império Romano? Em vez de um imperador romano, um rei italiano agora governava, e todos os seus dias futuros seriam vividos em um estado de caos liminar e declínio irrevogável. Neste ponto, aponto para um pequeno retângulo roxo no canto inferior direito da Linha do Tempo da História Mundial. Isso representa a história dos Estados Unidos, que parece bastante pequena e insignificante em comparação com as longas histórias de outras culturas e civilizações. Ao examinar este mapa, é fácil ver o auto engano necessário para manter o mito de que as coisas nunca podem mudar. Toda a história do mundo é de constante agitação. Nações e impérios sobem e caem. Às vezes, eles são derrotados de fora. Às vezes, eles implodem de dentro. Normalmente, é uma combinação de ambos. Quase sempre é completamente inesperado. O antropólogo Alexei Yurchak capturou o zeitgeist de crescer na URSS da década de 1980 no título de seu ás) Tudo era para sempre até que não existisse mais.
 \par 
Mudanças positivas podem acontecer e acontecem, e embora sempre haja contingências históricas aleatórias em ação, são, em última análise, as pessoas trabalhando coletivamente que moldam a história. “Nunca duvide que um pequeno grupo de cidadãos atenciosos e comprometidos pode mudar o mundo”, começa uma citação atribuída à antropóloga Margaret Mead. “Na verdade, é a única coisa
 \par 
KRISTEN R. GHODSEE que já existiu.” É claro que as coisas nem sempre mudam para melhor, como descobriram Yurchak e muitos dos seus compatriotas na Europa Oriental. A regressão acontece com a mesma frequência que o progresso, e pode ser por isso que tantas pessoas se apegam ao coisas como são. Mas tentar permanecer na água e permanecer no mesmo lugar torna mais fácil para aqueles que tentam nos puxar para trás ter sucesso. Só um forte impulso para a frente pode contrariar o impulso das pessoas que esperam pelo regresso dos costumes sociais do passado.
 \par 
Nas semanas que antecederam a eleição presidencial dos EUA de 2016, a hashtag do Twitter #Repealthel9th começou a virar tendência em resposta a dois tuítes do prognosticador Nate Silver. Em seu popular site FiveThirtyEight.com, Silver decidiu prever o resultado da eleição se apenas homens ou se apenas mulheres votassem. O mapa eleitoral para os homens mostrou uma vitória conveniente para Donald Trump, enquanto o mapa das mulheres revelou uma vitória para Hillary Clinton. Alguns apoiadores de Trump sugeriram então que, para garantir uma vitória de Trump, os Estados Unidos deveriam revogar a Décima Nona Emenda à Constituição, que garantiu o sufrágio feminino. "Eu estaria disposta a prescindir do meu direito de votar para isso acontecer", escreveu uma apoiadora de Trump. A indignação no Twitter se seguiu, e a história foi coberta pela grande mídia, incluindo o Los Angeles Times, Salon e USA Today.
 \par 
161
 \par 
162
 \par 
DAS BARRICADAS ÀS URNAS: SOBRE A CIDADANIA atiçando ainda mais as chamas da histeria digital. Embora mais tarde tenha sido revelado que estavam mais pessoas usando a hashtag para tuitar contra a ideia, a hashtag refletiu uma crença popular entre alguns conservadores preocupados com as tendências demográficas e as perspectivas futuras do Partido Republicano.*
 \par 
Em 2007, a especialista de direita Ann Coulter disse a um entrevistador de rádio que o sistema político americano seria muito melhorado se o país revogasse a Décima Nona Emenda e deixasse apenas os homens irem às urnas. "Se tirássemos o direito das mulheres de votar", ela explicou, "nunca teríamos que nos preocupar com outro presidente democrata [assim usado]. É uma espécie de sonho, é uma fantasia pessoal minha". Coulter continuou dizendo que as mulheres votavam "estupidamente", especialmente as solteiras, e argumentou que o Partido Democrata deveria ter vergonha de que mais homens não votassem em seus candidatos. Coulter opinou que o Partido Democrata era o partido das mulheres, subornando "mães do futebol" com "assistência médica, mensalidade e dia <are".°
 \par 
A diatribe de Coulter contra o voto feminino, e especialmente o voto de mulheres solteiras, pode ter sido inspirada por um artigo influente que apareceu no Journal of. Political Economy em 1999. Os autores, John Lott e Lawrence Kenny, correlacionaram o crescimento dos gastos do governo dos EUA no início do século XX com a disseminação do sufrágio feminino em todos os estados (culminando com a emenda constitucional em 1920) para argumentar que as mulheres têm mais probabilidade de votar em candidatos mais progressistas socialmente do que os homens. Lott e Kenny sugerem que, como as mulheres têm salários mais baixos e mais barreiras à autossuficiência, elas podem preferir menos riscos e um papel maior para o governo. Usando evidências empíricas, os autores pretendem mostrar como as mulheres têm
 \par 
KRISTEN R. GHODSEE usou seus votos para aumentar constantemente o tamanho do governo: “Como as mulheres tendem a ter rendas mais baixas, elas se beneficiam mais de vários programas governamentais que redistribuem a renda para os pobres, como impostos progressivos.” E com o tempo, mulheres solteiras em particular entenderam que se beneficiam de serviços sociais mais robustos e votam de acordo. Lott e Kenny argumentam que, “depois que as mulheres têm que criar os filhos sozinhas, elas são mais propensas a se classificarem como liberais, votar em democratas e apoiar políticas como impostos progressivos de renda. . . Não é difícil ver que dar às mulheres o direito de votar provavelmente desempenhou algum papel na determinação do caminho dos gastos do governo ao longo do tempo.”
 \par 

 \par 
Para muitos conservadores, para quem a expansão dos gastos do governo é um anátema, Lott e Kenny colocam diretamente a culpa pelo crescimento histórico dos gastos federais nos Estados Unidos aos pés das mulheres que estão votando em seus próprios interesses econômicos. Se você passar algum tempo na internet lendo os blogs de ativistas dos “direitos dos homens”, verá que eles dependem muito do artigo de Lott e Kenny de 1999 para apoiar suas alegações de que as mulheres não devem mais votar (embora, na verdade, você provavelmente esteja melhor lendo rótulos de ração para cães do que lendo blogs sobre direitos dos homens). Embora a emancipação política das mulheres só apareça na Linha do Tempo da História Mundial no século passado, o impulso básico do movimento pelos direitos dos homens é que o sufrágio feminino destruiu a civilização ocidental. Em um livro, The Curse of. {\color{blue} 1920 } {\par} , o autor propõe que, “Os direitos das mulheres são como o câncer — se você fizer uma cirurgia e não tratar tudo, ele voltará. A única solução para uma abundância dos males que assolam a nossa nação é eliminar a causa: as mulheres na política e
 \par 
163
 \par 
164
 \par 
DAS BARRICADAS ÀS URNAS: SOBRE O GOVERNO DA CIDADANIA.” Em seu próprio pequeno discurso de março de 2017 sobre como salvar o Ocidente, Roosh V (da notoriedade do Don’t Bang Denmark) afirma inequivocamente que revogar a Décima Nona é a única maneira de salvar os Estados Unidos de uma certa ruína socialista. “Remova o direito de voto de uma mulher e, em apenas uma eleição nacional, todos os partidos de esquerda seriam esmagados. Em duas eleições, os políticos falariam diretamente aos homens e seu interesse inato pelo patriarcado, sucesso econômico, famílias estáveis ​​e uma distribuição equitativa de mulheres na sociedade.” Não tenho certeza de quem ficará encarregado dessa distribuição equitativa de mulheres, mas certamente não serão as próprias mulheres.*
 \par 
Embora sempre envoltos em um grosso cobertor de misoginia, o que todos os ativistas dos direitos dos homens concordam é que as mulheres votam em candidatos progressistas porque é do seu próprio interesse econômico fazê-lo. Embora o artigo de 1999 de Lott e Kenny continue a ser usado nessas diatribes de ódio às mulheres, uma leitura alternativa de sua pesquisa, na verdade confirma a ideia de que políticas redistributivas são uma melhor garantia da independência das mulheres do que o livre mercado desenfreado. Na verdade, os ativistas dos direitos dos homens sabem o que muitas mulheres americanas não conseguem perceber: as mulheres têm imenso poder político nas urnas.
 \par 
Assim como os proponentes da teoria da economia sexual admitem que o capitalismo mercantiliza a sexualidade das mulheres e a igualdade de gênero e as generosas redes de segurança social dão às mulheres outras maneiras de atender às suas necessidades básicas além de se venderem ao maior lance, Lott e Kenny fornecem evidências de que a participação política das mulheres resultou (pelo menos a longo prazo) em um governo que atende melhor às necessidades de muitos. De fato, repetidamente,
 \par 
KRISTEN R. GHODSEE, de extrema-direita, acusa as eleitoras de elegerem líderes "socialistas" empenhados em minar tanto o patriarcado quanto a propriedade privada. Claro, com poucas exceções, os Estados Unidos não tiveram nada próximo de um líder socialista, mas nessa reescrita paranoica do passado do nosso país, talvez os rapazes da alt.-right estejam mostrando às mulheres um caminho para um possível futuro.
 \par 
“Revogar o 19º” só significa algo no ano de 2018 porque a composição demográfica do eleitorado do futuro próximo é um mau presságio para os homens e “seu interesse inato pelo patriarcado, sucesso econômico, famílias estáveis ​​e uma distribuição equitativa de mulheres na sociedade”. Talvez seja por isso que os conservadores estão tão interessados ​​em manchar qualquer um que flerte com ideias socialistas com o pincel negro do stalinismo. Desesperados para desacreditar as demandas políticas dos “guerreiros da justiça social”, os oponentes gritarão sobre os expurgos, fomes e Gulag, argumentando que as tentativas apoiadas pelos eleitores de construir um sistema de saúde universal e de pagador único ou uma rede nacional de creches de qualidade inevitavelmente levarão este país por uma ladeira escorregadia em direção ao totalitarismo. Mas após anos de bullying e intimidação, as vozes da extrema-direita (embora ainda bem financiadas) estão começando a ser abafadas por uma onda crescente de millennials fartos da ideia de que o capitalismo é o único jogo na cidade.
 \par 
Os conservadores temem essa crescente insatisfação dos jovens com o capitalismo global. E eles se preocupam que as mulheres americanas, e especialmente as mulheres mais jovens da geração Y, votem em candidatos de esquerda ou socialistas, especialmente quando as mulheres entendem que elas se beneficiam desproporcionalmente do estado.
 \par 
165
 \par 
166
 \par 
DAS BARRICADAS ÀS URNAS: SOBRE CIDADANIA regulamentação de mercados, assistência médica de pagador único, educação pós-secundária gratuita, propriedade social de grandes empresas como serviços públicos ou bancos que são "grandes demais para falir" e outras políticas redistributivas. Hoje, os millennials e os membros da geração Z veem o socialismo democrático como uma resposta para suas muitas frustrações — um inibidor da libido a menos do que os inibidores seletivos da recaptação da serotonina. Em um artigo amplamente compartilhado para a Nation em janeiro de 2017, "Por que os millennials não têm medo do socialismo", Julia Mead relata sua descoberta pessoal dos ideais socialistas e como o discurso político americano encerrou sua discussão antes do surgimento de Bernie Sanders nas primárias democratas de 2016:
 \par 
O discurso TTTTTT durante a maioria da minha vida não foi uma história-
 \par 
\[DAS BARRICADAS ÀS URNAS: SOBRE CIDADANIA\]
 \par 
\section{O apagamento das ideias socialistas da vida política séria}
 \par 
\section{Cal fauce. A vitória do Ocidente na Guerra Fria — liberal}
 \par 
. Assim, o comunismo foi morto, e com ele foi qualquer discussão sobre socialismo e marxismo. Este foi o mundo da minha infância
 \par 
Os TTTTTT eram agressivamente centristas e tão dispostos quanto as conferências
 \par 
Vativos para privilegiar os interesses do capital sobre aqueles
 \par 
\section{Democracia para todos! - veio ao preço da iconografia}
 \par 
\section{Clamor, muito dele comemorativo. .}
 \par 
Durante a maioria da minha vida, eu teria tido muita dificuldade em
 \par 
Livros TTTTTT, nenhuma outra forma de organizar uma economia era
 \par 
Até mesmo reconhecido. Eu não sabia que poderia haver uma alternativa.’
 \par 
KRISTEN R. GHODSEE
 \par 
Mead argumentou que os millennials abraçam o socialismo porque estão "cansados ​​do mundo desigual que herdaram". Exatamente seis meses depois, a editora da Nation, Sarah Leonard, seguiu o artigo de Mead com seu próprio artigo de opinião para o New York Times, "Por que tantos eleitores jovens estão se apaixonando por velhos socialistas?" Refletindo sobre a popularidade de homens brancos seniores como Bernie Sanders nos Estados Unidos e Jeremy Corbyn na Grã-Bretanha, Leonard argumentou que o crescente apoio dos millennials ao socialismo tinha menos a ver com o radicalismo inerente da juventude e mais com as falhas dos partidos tradicionais em controlar os piores excessos do capitalismo: "Nossa política foi moldada por uma era de crise financeira e cumplicidade do governo. Especialmente desde 2008, vimos corporações tomarem as casas de nossas famílias, explorar nossa dívida médica e nos custar nossos empregos. Vimos governos imporem austeridade brutal para agradar aos banqueiros. Os capitalistas não fizeram isso por acidente, eles fizeram isso por lucro e investiram esse lucro em nossos partidos políticos. Para muitos de nós, o capitalismo é algo a temer, não a celebrar, e o nosso inimigo está em Wall Street e na City de Londres.”
 \par 
Para os políticos republicanos e seus apoiadores ricos, os sentimentos expressos por Mead e Leonard, ambas jovens mulheres de esquerda, representam uma ameaça real. Pela primeira vez na eleição de 2016, os eleitores da geração Y e da geração X superaram em número os bebê boomers. Na eleição de 2020, os eleitores da geração Y desfrutarão de enorme influência eleitoral se forem às urnas. Demograficamente, sua geração supera em número a geração X e se expandirá ainda mais com o número crescente de imigrantes naturalizados mais jovens. Para os republicanos do establishment que esperam mais desregulamentação e cortes de impostos para os ricos, o aumento das fileiras de eleitores mais jovens representa uma clara
 \par 
167
 \par 
168
 \par 
DAS BARRICADAS ÀS URNAS: SOBRE CIDADANIA e apresentam perigo para suas perspectivas políticas de longo prazo. De acordo com um relatório de julho de 2017 do Pew Research Center, os eleitores da geração Y são muito mais propensos do que seus pais ou avós a se identificarem como democratas ou independentes com inclinação democrata.”
 \par 
Em última análise, essa coisa que chamamos de “governo” não é inerentemente boa ou má. É uma embarcação dirigida por aqueles que por acaso a controlam em qualquer momento no tempo. É por isso que
 \par 
CAISTEM {\color{blue}0}. GAODSEE eles chamam isso de “o navio do estado”. Eu também me arrisco a dizer que essa coisa que chamamos de “mercado” não é nem boa, nem ruim, mas apenas uma ferramenta que pode ser usada por aqueles que acreditam que isso promoverá seus interesses. Hoje em dia, parece que o mercado é uma ferramenta usada pelos super-ricos para aumentar sua riqueza, que eles usam para comprar influência e poder sobre nosso governo. Embora tenhamos eleições presidenciais uma vez a cada quatro anos, o poder político real foi acumulado para os super-ricos, e nosso governo faz suas licitações enquanto finge representar o povo, assim como os governos socialistas estatais na Europa Oriental fizeram as licitações de ditadores e elites de alto escalão enquanto fingiam trabalhar para o bem do povo.
 \par 
A diferença entre governos e mercados é que governos, ou pelo menos governos democráticos, são ostensivamente feitos para servir os cidadãos. Essa é a ideia de uma pessoa, um voto. Os mercados, por outro lado, sempre serão manipulados em favor daqueles que começam o jogo com a maior pilha de dinheiro. E dada a confluência da maneira como os mercados funcionam com a decisão da Suprema Corte de 2010 no caso Citizens United, que permitiu doações ilimitadas de campanha, quanto mais dinheiro alguém tem, mais influência tem sobre o governo. Um ciclo vicioso surge, com mercados não regulamentados corroendo o poder do governo. Isso cria maiores lucros para aqueles que podem comprar mais influência sobre o governo para revogar as regulamentações que protegem nosso sistema educacional, nosso meio ambiente e nossos serviços sociais, o que torna os ricos ainda mais ricos.
 \par 
Do governo. É o controle real do cidadão? Devemos fazer o estado trabalhar no interesse das pessoas comuns. Democracia significa governo do povo; o grego
 \par 
TTTTTT em
 \par 
170 robot demos refere-se às pessoas comuns de um estado. Plutocracia, por outro lado, significa governo dos ricos, após a palavra grega fotos (riqueza). Os enormes resgates de Wall Street após a recessão global e o roubo de impostos de Donald Trump em 2017 demonstram claramente em qual desses dois sistemas políticos estamos vivendo. Pode parecer hiperbólico, mas não é impossível que os Estados Unidos se tornem um estado de partido único, governado pelo dinheiro obscuro de um Partido Plutocrático paralelo. Mas ainda não chegamos lá. Por enquanto, as elites econômicas continuam investidas em manter a fachada da democracia, e é aqui que as jovens americanas podem fazer uma grande diferença.
 \par 
Se as jovens não ficarem sábias e começarem a ir às urnas votar em seus próprios interesses econômicos e políticos de longo prazo, elas terão pouco poder para reverter as inevitáveis ​​convulsões sociais que o futuro reserva. À medida que os republicanos acumulam déficits irresponsáveis ​​no curto prazo, eles já estão de olho nos programas sociais que precisarão destruir para evitar que os Estados Unidos entrem em falência. Quando programas como a Previdência Social e o Medicare desaparecem porque o governo não pode mais pagar por eles, todo o trabalho de cuidado necessário para cuidar de nossos pais recairá sobre os ombros das mulheres que já estão em casa porque não podem pagar por creches para seus filhos. E sem alguma forma de assistência médica universal, futuros cortes no Medicaid significarão que mais e mais americanos precisarão de cuidados constantes em casa, atendidos, sem dúvida, por suas filhas, mães, irmãs e esposas. Com as mulheres responsáveis ​​por uma pilha crescente de trabalho de cuidado na esfera privada, sua autonomia diminuirá e elas se encontrarão economicamente dependentes e
 \par 
KRISTEN R. GHODSEE impotente para deixar relacionamentos insatisfatórios, violentos ou emocionalmente abusivos.
 \par 
Algumas pessoas argumentarão que já é tarde demais e que nosso sistema político está quebrado demais para ser consertado. Certamente, se os plutocratas estão enchendo urnas ou adulterando máquinas de votação, o jogo acabou, e os cidadãos americanos perderam. Então, realmente precisamos começar a pensar sobre o que fazer a seguir. Mas até lá, nosso processo democrático ainda oferece a possibilidade de uma mudança política radical ou o que Bernie Sanders chamou de "revolução de baixo". Se os eleitores mais jovens, especialmente as mulheres mais jovens, começarem a arrastar suas bundas para as urnas, eles têm o poder de fazer a diferença. É por isso que os conservadores-tiber querem tirar seu direito de votar. As mulheres da geração Y têm o poder demográfico para ter um impacto em nosso futuro coletivo, especialmente se puderem convencer seus pais bebê boomers de que terão que se defender se os republicanos conseguirem finalmente o que querem e desmantelarem a Previdência Social. Se os jovens conseguirem eleger líderes políticos que tornem o governo mais responsivo às necessidades de seus cidadãos, então os plutocratas, se quiserem manter o coisas como são, terão que abandonar completamente a fachada da democracia. Quando isso acontecer, não estaremos mais vivendo nos Estados Unidos da América, mas em algum outro país governado por um conjunto diferente de regras.
 \par 
Embora o primeiro passo seja definitivamente votar (e mobilizar outros para sair e votar), votar não é suficiente. Os jovens precisam se educar nos fundamentos da teoria política. Leia livros, assista a vídeos, ouça podcasts, examine infográficos — seja lá o que for que você precise fazer para expandir sua compreensão de como e por que nos organizamos em estados-nação e nos permitimos ser governados por outros,
 \par 
171
 \par 
172
 \par 
DAS BARRICADAS ÀS URNAS: SOBRE CIDADANIA e como e por que isso mudou ao longo do tempo. E não fique apenas na sua zona de conforto; abra sua mente para perspectivas opostas, não importa o quão doloroso isso possa ser. Se você é um leitor da Jacobin, clique nas páginas da Reason. Percorra o New York Times e o Wall Street Journal. Se você tiver estômago, saia e converse com as pessoas. Saia da sua bolha digital e envolva-se onde puder, compartilhando o que aprendeu: na escola, no trabalho, na igreja, na biblioteca local e assim por diante. Participe de um clube do livro ou de um grupo de leitura, ou inscreva-se no comitê organizador de um movimento antissocial ou partido político. Como introvertido, sei que é mais fácil falar do que fazer para alguns de nós, mas se você é do tipo gregário, encontre sua voz e use-a.
 \par 
Fora do processo eleitoral, outras estratégias políticas podem ser mobilizadas para forçar líderes empresariais e funcionários do governo a responder às necessidades das pessoas comuns. Por exemplo, você pode se juntar a outros para agitar por políticas para expandir o emprego público; fornecer creches subsidiadas de alta qualidade; garantir licenças parentais remuneradas e protegidas pelo emprego com incentivos embutidos para que os pais as tirem, assim como as mães; implementar cotas para aumentar todas as formas de diversidade de liderança; criar um sistema de saúde universal; e reduzir o custo da mensalidade da faculdade. Essas políticas contribuirão muito para mitigar a desigualdade e construir uma sociedade que trabalhe para as multidões na base, em vez de apenas para o 1% no topo. Uma rede de segurança social mais ampla, como as encontradas nos países contemporâneos do norte da Europa, aumentará em vez de diminuir a liberdade pessoal porque restaurará aos cidadãos a capacidade de tomar as decisões mais importantes sobre suas próprias vidas. Ninguém deveria ter que permanecer em um emprego que odeia pelo seguro de saúde, ou
 \par 
KRISTEN R. GHODSEE fica com um parceiro que bate nela porque não tem certeza de como vai alimentar as crianças, ou faz sexo com um velhote porque não tem dinheiro para comprar livros didáticos.
 \par 
Mais importante, recupere seu tempo, energia emocional e autoestima da lógica reducionista do capitalismo. Você não é uma mercadoria. Sua depressão e ansiedade não são apenas desequilíbrios químicos em seu cérebro, mas respostas razoáveis ​​a um sistema que prospera em sua desumanização. Como Mark Fisher argumentou em 2012, "saúde mental é uma questão política" e, enquanto nossas vidas privadas informam nossa saúde mental, os relacionamentos também são uma questão política. Devemos rejeitar a ideologia dominante que transforma nossos laços sociais em nós de troca econômica. Podemos compartilhar nossas atenções sem quantificar seu valor, dando e recebendo em vez de vender e comprar. As mulheres precisam estabelecer o que eu quero chamar de "soberania afetiva" para obter controle total de nosso trabalho emocional. No verão de 2017, o toldo de uma livraria em Munique dizia: "O amor mata o capitalismo". Se as pessoas estão felizes em suas vidas íntimas, se elas se sentem amadas e apoiadas por quem elas são, em vez do que elas possuem, o capitalismo perde uma das ferramentas mais valiosas que ele tem: ele não pode mais nos convencer de que precisamos comprar mais coisas para preencher o vazio deixado pela nossa falta de conexão pessoal. Nosso anime crescente é lucrativo. Ao impedir que nossas afeições se tornem mais uma coisa para ser comprada e vendida, estamos dando os primeiros passos de resistência.’ * Uma das coisas mais importantes que aprendi ao estudar o colapso do socialismo de estado do século XX na Europa Oriental é que as pessoas lá estavam completamente despreparadas para
 \par 
173
 \par 
174
 \par 
DAS BARRICADAS ÀS URNAS: SOBRE A CIDADANIA as mudanças repentinas introduzidas pela criação de mercados livres. Como seus governos controlavam o fluxo de informações sobre o Ocidente, os cidadãos comuns sabiam muito pouco sobre como as democracias capitalistas funcionavam na prática. Se ouvissem algo sobre falta de moradia, pobreza, desemprego ou os ciclos de expansão e retração do mercado, eles desconsideravam esses fatos como mera propaganda. Mais importante, os cidadãos do Leste Europeu não tinham acesso a alguns dos textos básicos que explicavam como e por que a democracia liberal diferia do que eles chamavam de "socialismo realmente existente" (para distinguir o que tinham do ideal pelo qual estavam se esforçando). Eles não tinham como explorar por si os contrastes nas filosofias políticas fundamentais que levaram o mundo à beira da aniquilação nuclear. Há um ditado popular em muitas nações do Leste Europeu hoje: "Tudo o que nos disseram sobre o comunismo era mentira. Mas tudo o que nos disseram sobre o capitalismo era verdade."
 \par 
No mundo ocidental, ninguém nos impede de ler o que quisermos, mas poucas pessoas dedicam tempo a pensar sobre que tipo de sociedade poderemos ter se a nossa democracia falhar e nos encontrarmos a viver numa nação (ou nações) que são pós-americanos. Como observei mudanças sociais radicais acontecerem nos países da Europa de Leste, sei que mesmo que se trate de uma dissolução pacífica ou de um divórcio aveludado (como foi referida a dissolução da Checoslováquia), o processo de reconstrução da confiança na sociedade será doloroso e desorientador. Se a súbita mudança social resultar em violência (como na Iugoslávia), muitas vidas serão perdidas desnecessariamente e serão necessárias décadas até que as feridas psíquicas cicatrizem para aqueles que sobreviverem. Sei que é antiquado falar de coisas como dever cívico, mas como as nossas democracias ocidentais
 \par 
KRISTEN R. GHODSEE tornam-se cada vez mais polarizados, aqueles que esperam por um mundo mais justo, sustentável e equitativo têm muito trabalho a fazer se quisermos ser capazes de empurrar as coisas numa direção progressista.
 \par 
Alguns argumentarão que as tentativas de reforma apenas prolongam a vida de um sistema econômico falido, e todos nós estaríamos melhor se simplesmente deixássemos o capitalismo se empalar em um espeto de suas próprias contradições internas. Mas um colapso repentino do capitalismo do século XXI teria repercussões globais massivas e causaria sofrimento humano generalizado a muitas das mesmas pessoas que acabariam se beneficiando de sua queda. Os autoproclamados revolucionários certamente discordarão, mas todas as formas de mudança de regime (mesmo as boas) criam danos humanos colaterais e, se pudermos, devemos tentar minimizar isso o máximo possível. Um dos maiores problemas com o socialismo de estado do século XX na Europa Oriental era que alguns líderes estavam muito ansiosos para sacrificar as vidas de seus próprios cidadãos em prol da construção de um futuro mais justo e igualitário. Reforma social rápida e revolução podem ser apenas caminhos diferentes para o mesmo objetivo final do socialismo. Sim, impérios sobem e caem, mas as pessoas comuns se saem melhor se entrarem em colapso com um gemido em vez de um estrondo. A estrela do capitalismo pode se apagar como uma supernova, mas a transição para o pós-capitalismo será mais fácil para a maioria de nós se ela morrer como uma anã branca.”
 \par 
O que me traz de volta ao meu mapa da Linha do Tempo da História Mundial e seus grandes blocos de cor representando a ascensão e queda de impérios. É uma coisa linda de se olhar quando você está se sentindo impotente sobre o futuro, frustrado com o ritmo glacial da mudança ou com medo dos visigodos atuais esperando nos mergulhar na Idade das Trevas {\color{blue}2}.0. Embora
 \par 
175
 \par 
176
 \par 
DAS BARRICADAS ÀS URNISAS: SOBRE A CIDADANIA, os cartógrafos de Oxford não as desenharam, o seu mapa é povoado por mais de cem mil milhões de pessoas - todos os homens e mulheres que alguma vez viveram na Terra. Cada uma dessas pessoas nasceu de mãe e, se sobreviveram à infância, cresceram até a idade adulta e viveram em algum tipo de clã ou comunidade. Eles comiam, bebiam, dormiam, sonhavam, faziam sexo, formavam famílias e, eventualmente, adoeciam e morriam de uma forma não muito diferente da que acontece hoje. Foram estes milhares de milhões de homens e mulheres que fizeram a nossa história, e não apenas aqueles mencionados nos nossos livros escolares. Foram as pessoas comuns que criaram os bebês, construíram as barragens, cultivaram as colheitas, travaram as guerras, ergueram os templos e iniciaram as revoluções.
 \par 
E a menos que um enorme meteoro caia no planeta e nos destrua amanhã, ainda são pessoas comuns que podem levar a história adiante. A ação coletiva coordenada pode ter um enorme impacto no mundo. Se dois bilhões de pessoas decidissem parar espontaneamente de usar o Facebook amanhã ou parassem de comprar na Amazon.com, duas das corporações mais ricas e poderosas do mundo poderiam deixar de existir. Se milhões de homens e mulheres fossem a qualquer banco e exigissem seus depósitos no mesmo dia, eles poderiam aleijar até mesmo os mais poderosos entre eles. Era uma vez, quando havia sindicatos fortes e os trabalhadores negociavam coletivamente, os cidadãos retinham uma parcela maior da riqueza que ajudavam a produzir. O inimigo mais perigoso da plutocracia é inúmeros cidadãos trabalhando juntos por uma causa comum. Não é coincidência que o capitalismo prospere em uma ideologia de interesse próprio e individualismo, e que seus defensores tentarão desacreditar os ideais coletivistas baseados no altruísmo e na cooperação.
 \par 
KRISTEN R. GHODSEE
 \par 
177
 \par 
Sinenabicae olho ter sene, Aa
 \par 
Sei que não é uma tarefa fácil encontrar uma causa comum e, ao mesmo tempo, respeitar nossas muitas diferenças, e devemos sempre estar atentos às hierarquias de poder que dão a alguns de nós mais privilégios do que a outros. Formar coalizões cidadãs poderosas que também reconheçam e apoiem nossa diversidade é uma tarefa urgente, e precisamos recorrer a um kit de ferramentas o mais profundo possível se quisermos encontrar uma saída coletiva para nosso atual atoleiro político e econômico. Os experimentos do século XX com o marxismo-leninismo falharam, mas seu fracasso deve fornecer lições para nos ajudar a evitar seus muitos erros, em vez de inspirar uma rejeição automática de todas as ideias comunalistas.
 \par 
Havia um bebê em toda aquela água do banho. Está na hora de salvá-lo.
 \par 
\begin{figure}
	\centering
	\includegraphics[width=1.\textwidth]{temp\_files/images/UP\_logo.png }
	\caption{August Bebel (1840-1913): Um cofundador do Partido Social-Democrata dos Trabalhadores da Alemanha que viria a liderar o Partido Social-Democrata da Alemanha. Autora do influente livro Woman and Socialism e uma proeminente defensora dos direitos das mulheres, Bebel argumentou que as mulheres só seriam libertadas de sua dependência econômica dos homens quando os trabalhadores possuíssem e controlassem coletivamente os meios de produção. Bebel é amplamente creditado como a primeira figura política a fazer um discurso público em favor dos direitos gays. Cortesia da Biblioteca do Congresso dos EUA.}
	\label{ }
\end{figure}