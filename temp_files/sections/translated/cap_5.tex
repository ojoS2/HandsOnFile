\chapter{4 O Circuito do Capital Industrial}\label{4 O Circuito do Capital Industrial}
 \par 
O Volume {\color{blue}1} de O Capital é em grande parte autocontido e fornece principalmente uma análise geral do capitalismo e do seu processo de desenvolvimento a partir da perspectiva da produção - quais são as relações sociais que permitem ao capital criar mais-valia e como é que estas dão origem a realidades económicas e desenvolvimentos sociais em torno da produção? Os outros dois volumes de O Capital dedicam-se tanto a elaborar como a ampliar esta análise geral. Por esta razão é apropriado que o início do Volume {\color{blue}2} analise o circuito do capital. Isto porque este circuito fornece a base para a compreensão de toda uma série de fenómenos - capital comercial, capital fixo e remunerado, distribuição do rendimento e da produção, rotação do capital, trabalho produtivo e improdutivo e crises - bem como proporciona um equilíbrio económico. estrutura na qual as relações sociais de produção analisadas no Volume {\color{blue}1} podem ser apresentadas de forma mais concreta. Por outras palavras, os Volumes {\color{blue}2} e {\color{blue}3} tratam de como as relações de valor de produção, estudadas no Volume 1, dão origem a resultados mais complexos através dos processos e estruturas de troca e distribuição.
 \par 
\section{O circuito monetário do capital}
 \par 
O volume {\color{blue}2} começa com uma descrição do circuito monetário do capital. Esta é uma expansão da caracterização do capital como valor autoexpansível (ver Capítulo {\color{blue}3}), tendo explicitamente em conta o processo de produção. A forma geral do circuito do capital industrial é:
 \par 
Nas circunstâncias mais gerais, e independentemente da mercadoria produzida, os capitalistas industriais adiantam capital monetário (M) para comprar insumos mercantis (C), compreendendo força de trabalho (LP) e meios de produção (MP). Deve-se compreender que o dinheiro é necessário para estas transações, mas por si só não as torna possíveis. É a separação da propriedade da força de trabalho dos meios de produção - uma relação de classe de produção - que permite a um grupo definido de pessoas (os capitalistas) contratar outros (os trabalhadores) em troca de um salário. Isto pode ser sublinhado separando explicitamente os meios de produção e a força de trabalho no circuito do capital:
 \par 
\section{O circuito monetário do capital}
 \par 
Na compra, os insumos (C) formam capital produtivo (P). A produção prossegue à medida que a força de trabalho é exercida sobre os meios de produção, e o resultado são diferentes produções de mercadorias, com um valor mais elevado (C'). C e C' estão ligados a P por pontos para indicar que a produção interveio entre a compra de insumos (C) e a venda de produtos (C'). As mercadorias produzidas são designadas por C' não porque o seu valor de uso seja diferente daquele dos meios de produção (embora este seja geralmente o caso), mas porque contêm mais-valia para além do valor do capital adiantado, M. Isto é mostrado pela venda da produção por mais capital monetário, M' > M.
 \par 
Foi demonstrado no Capítulo {\color{blue}3} que a mais-valia, s = M' - M, é criada na produção pela compra de força de trabalho pelo seu valor, que é inferior ao tempo de trabalho despendido (valor criado) na produção. A mais-valia aparece pela primeira vez na forma de mercadoria imediatamente após a produção. Uma vez que os factores de produção (especialmente força de trabalho, ferramentas, máquinas e edifícios) parecem simétricos na sua contribuição para a produção, é fácil creditar a criação de mais-valia à “produtividade” de todos os factores de produção sem distinção. Da mesma forma, é difícil creditar a mais-valia ao excesso do tempo de trabalho real em relação ao tempo de trabalho necessário, porque o aparecimento da mais-valia é adiado até depois da produção ter ocorrido, enquanto a livre troca da força de trabalho pelo seu valor ocorre antes da produção (mesmo se os salários forem pagos em atraso).
 \par 
O valor produzido (e a mais-valia) é agora convertido em dinheiro pela venda da produção no mercado. Tendo obtido o rendimento das vendas D', os capitalistas podem renovar o circuito do capital - quer na mesma escala (renovando o adiantamento original, M, com determinados preços e tecnologias, e gastando a mais-valia no consumo), ou embarcando numa circuito produtivo ampliado, através do investimento de parte da mais-valia (ver abaixo, e Capítulo {\color{blue}5}).
 \par 
\section{O circuito monetário do capital}
 \par 
Foi mostrado acima (e no Capítulo {\color{blue}3}) que o capital é a relação social que sustenta a autoexpansão do valor, ou a produção, apropriação e acumulação de mais-valia. O capital, como valor autoexpansível, é essencialmente o processo de reprodução de valor e produção de novo valor. O circuito do capital descreve esse movimento e destaca que o capital assume diferentes formas em seu processo de reprodução. A relação social que é o capital sucessivamente assume e renuncia às formas de dinheiro, capital produtivo e mercadorias.
 \par 
O circuito do capital industrial é melhor representado por um diagrama de fluxo circular (ver Figura {\color{blue}4}.1). Este circuito é importante para definir a estrutura básica da economia capitalista e para mostrar como as esferas de produção e troca são integradas entre si através do movimento de capital à medida que o valor (mais-valia) é produzido, distribuído e trocado. À medida que o circuito se repete, a(s) mais-valia(s) é(ão) descartada(s). Assim, o capital como autoexpansível
 \par 
\section{O circuito monetário do capital}
 \par 
\section{O circuito monetário do capital}
 \par 
\section{O circuito monetário do capital}
 \par 
O valor TTTTTT abrange não apenas relações sociais de produção definidas, mas é também um movimento circular que passa repetidamente pelas suas várias fases. Se s for acumulado para uso como capital, podemos pensar na reprodução expandida como sendo representada por um movimento espiral para fora.
 \par 
O capital industrial transforma-se sucessivamente nas suas três formas: capital monetário (M), capital produtivo (P) e capital mercadoria (C'). Cada forma pressupõe a existência das outras duas porque pressupõe o próprio circuito. Isto permite-nos distinguir a função específica de cada uma das formas de capital da sua função geral como capital. Nas sociedades onde existem, o dinheiro, os factores de produção e as mercadorias podem sempre funcionar, respectivamente, como meios de pagamento, meios de produção e depositários de valor de troca, mas só servem como capital (industrial) quando seguem estas funções sequencialmente no circuito. do capital. Então, o capital monetário actua como meio de compra de força de trabalho, o capital produtivo actua como meio de produção de mais-valia e o capital mercadoria actua como depositário da mais-valia a ser realizada como dinheiro após a venda.
 \par 
No movimento pelo circuito, podem ser identificadas duas esferas de atividade: produção e circulação (troca). A esfera da produção situa-se entre C e C'. Nesta esfera, transformam-se valores de uso e criam-se valor e mais-valia. Isto tem implicações profundas para a teoria da distribuição de Marx, porque explica o que há para ser distribuído, bem como as estruturas e processos de distribuição de bens e valores na economia. A esfera da circulação contém o processo de troca entre C' e C, e a realização da mais-valia, s. Foi demonstrado no Capítulo {\color{blue}3} que, mesmo que o capital e o trabalho sejam empregados em troca, não acrescentam valor à produção. Esta conclusão parece estranha aos economistas convencionais porque eles estão normalmente interessados ​​em obter uma teoria dos preços agregando a contribuição (supostamente independente) de todos os factores utilizados na produção e na troca. Mas Marx está interessado nas relações sociais de produção e distribuição e nas estruturas de distribuição dos valores produzidos durante o circuito. Por exemplo, ele argumenta que, embora o capital comercial não acrescente valor, isso não o impede de receber uma parte do valor (excedente) produzido (ver Capítulo {\color{blue}11}).
 \par 
Ao construir o circuito do capital em forma circular, como na Figura {\color{blue}4}. {\color{blue} 1 } {\par} , torna-se arbitrário abrir e fechar o circuito com capital monetário, uma vez que um círculo não tem começo nem fim. Note-se que o circuito monetário contém a interrupção da esfera de circulação pela esfera de produção. Ao caracterizar o capital como valor autoexpansível, ficou demonstrado que o motivo dos capitalistas é comprar para vender mais caro. Assim, para o capital visto da perspectiva do circuito monetário, a produção aparece como uma interrupção necessária mas infeliz (e até mesmo um desperdício) no processo de produção de dinheiro. O capital mercantil e o capital remunerado evitam esta interrupção, embora dependam da produção que ocorre noutro local. Contudo, o que é possível para um capitalista individual (comerciante ou detentor de juros) não se aplica a todos (ou mesmo à maioria) dos capitalistas. Se os capitalistas de uma nação fossem dominados pela tentativa de obter lucro sem o inevitável elo de produção, encontrar-se-iam num boom especulativo que acabaria por falhar. Neste ponto, a economia seria trazida de volta à realidade da necessidade de produção - a única fonte possível do valor necessário para pagar dividendos, liquidar dívidas, cumprir compromissos de juros e obrigações financeiras claras (ver Capítulos 7, {\color{blue}12} e {\color{blue}14}).
 \par 
Marx também analisa o circuito de duas outras perspectivas, aquelas do capital produtivo e do capital-mercadoria. O circuito do capital produtivo começa e termina com P, produção. O propósito do circuito parece ser a produção e, na medida em que o valor excedente é acumulado, a produção em uma escala estendida. Em contraste com o circuito do dinheiro, para o circuito produtivo a esfera da circulação aparece como uma intrusão necessária, mas indesejada, no processo de produção. Mas foi demonstrado acima que não é suficiente produzir valor (excedente); ele tem que ser realizado na venda. Economistas mais frequentemente do que capitalistas tendem a ignorar essa mediação necessária, mas incerta, pela troca, pois um capitalista que involuntariamente acumula um estoque crescente de mercadorias é logo trazido de volta à realidade com a perda do capital de giro. Finalmente, o circuito do capital-mercadoria começa e termina com C', e então seu propósito parece ser satisfazer as necessidades de consumo da sociedade. Como a esfera da circulação é seguida pela esfera da produção, nenhuma esfera é interrompida pela outra, então nenhuma parece desnecessária ou perdulária.
 \par 
Os três circuitos do capital derivam do circuito como um todo. Poderíamos perguntar-nos por que não existem quatro circuitos de capital, com cada “nó” do circuito (P, C', M e C) formando um ponto de partida e de chegada. A razão pela qual C não é a base para um circuito de capital é que não é capital. Os meios de produção adquiridos podem ser a produção de mercadorias de outro capitalista e, portanto, capital-mercadoria. Contudo, a força de trabalho nunca é capital até ser comprada, e então torna-se capital produtivo e não capital mercadoria, que deve conter mais-valia produzida. Assim, embora de um ponto de vista técnico o capitalismo possa ser autossuficiente em matéria de matérias-primas, depende sempre e necessariamente da reprodução social da força de trabalho fora do sistema puro de produção (ver Capítulo {\color{blue}5}). Isto implica o uso do poder político, ideológico e jurídico, bem como do poder económico, para fazer com que o trabalhador trabalhe. Os mesmos problemas não existem para fazer uma máquina funcionar.
 \par 
Foi demonstrado acima que diferentes visões do processo de reprodução do capital podem ser construídas, cada uma correspondendo a um dos circuitos do capital. Elas não precisam ser acríticas ao capitalismo, mas individualmente são sempre inadequadas, enfatizando um ou mais processos de produção, consumo, troca, geração de lucro e acumulação às custas dos outros. Por exemplo, apenas fugazmente, ao entrarem no circuito, a força de trabalho e os meios de produção produzidos parecem separados e então, não formando capital, não geram espontaneamente uma visão do circuito como um todo. Em parte por essa razão, a teoria econômica dominante parece eliminar completamente as relações de classe. No entanto, essas relações reentraram na teoria dominante como relações distributivas ou de troca, em vez de relações de produção.
 \par 
Em contraste, o circuito monetário sugere modelos de troca. Para a economia dominante, a correspondência entre a oferta e a procura torna-se o princípio e o fim de tudo, e o capital e o trabalho são vistos como meros serviços produtivos. As dificuldades estão associadas aos serviços de informação executados pelo mecanismo de preços (e taxas de juros). O circuito produtivo, por sua vez, tende a ignorar o mercado, e neste contexto podem ser citadas as teorias neoclássicas e a maioria das teorias do crescimento. Isto produz uma excelente análise input-output da reprodução económica, mas a economia não é de todo claramente capitalista. Finalmente, o circuito das mercadorias reflecte-se na teoria neoclássica do equilíbrio geral, onde a oferta e a procura interagem harmoniosamente através da produção e da troca para produzir o consumo final. Este circuito apoia o mito de que o objectivo da produção é o consumo e não o lucro ou a troca, e é bem ilustrado pelos diagramas de caixa de Edgeworth, familiares aos estudantes de economia. Um dos pontos fortes do circuito do capital de Marx é expor as limitações destas perspectivas. Ao mesmo tempo, revela as funções das formas em que o capital aparece e constrói uma base sobre a qual as principais categorias e fenómenos económicos podem ser compreendidos.
 \par 
\section{O circuito monetário do capital}
 \par 
A análise da troca de Marx, especialmente no Volume {\color{blue}2} de O Capital, tem sido relativamente negligenciada, apesar dos insights que oferece. Muitas vezes foi adoptada uma abordagem de complementar a sua teoria da produção com a teoria keynesiana da procura efectiva, como se os dois aspectos da teoria de Marx estivessem sujeitos a tratamentos separados. Tal como sugerido aqui e na própria explicação de Marx, a produção e a troca estão estruturalmente separadas, mas integralmente relacionadas através dos circuitos do capital. A própria análise de Marx do circuito do capital é desenvolvida em Marx (1976, pt. {\color{blue} 2 } {\par} , 1978b, pts 1-2). É explicado em Ben Fine (1980, cap.{\color{blue}2}) e Alfredo Saad-Filho (2002, caps 3-5). Sobre o Volume {\color{blue}2} de Capital, ver Chris Arthur e Geert Reuten (1998). Para interpretações semelhantes às oferecidas aqui, ver David Harvey (1999, cap.{\color{blue}3}) e Roman Rosdolsky (1977, pt.{\color{blue}4}). Os conceitos de dinheiro como dinheiro e dinheiro como capital são explicados por Costas Lapavitsas (2003a) e Roman Rosdolsky (1977, pt.{\color{blue}3}).