TTTTTT No ensino médio, eu era um nerd certificado do Modelo das Nações Unidas (MUN). O MUN é um tipo de clube de debate no qual os alunos pesquisam sobre as políticas externas dos países-membros da ONU e, em seguida, representam esses países em sessões simuladas com cenários políticos fabricados com base em eventos atuais reais. Para se destacar no MUN, você precisava aprender os meandros das relações internacionais, bem como entender os contextos sociais, políticos e econômicos que informavam as decisões de política externa de diferentes países ao redor do mundo. O maior prêmio de uma competição do MUN era o martelo, concedido ao aluno que representasse seu país de forma mais convincente em uma sessão simulada. Geralmente, os martelos de maior prestígio eram concedidos aos membros do Conselho de Segurança simulado, o mais poderoso dos comitês da ONU. Os alunos trabalhavam seu caminho desde os comitês menores, como a Assembleia Geral ou o Conselho Econômico e Social (ECOSOC), até que estivessem prontos para servir no Conselho de Segurança, onde os alunos mais brilhantes e informados discutiam e decidiam o destino do globo.
 \par 
Para aumentar suas chances de ganhar um martelo, você queria representar um dos cinco membros permanentes da
 \par 
77
 \par 
78
 \par 
CALÇAS NÃO SÃO SUFICIENTES: SOBRE A LIDERANÇA, o Conselho de Segurança - os Estados Unidos, a França, o Reino Unido, a URSS ou a China, os únicos países com poder de veto. Se tiver poder de veto, não poderá ser derrotado na votação e todos os outros delegados precisam de assegurar o seu apoio à sua resolução ou pelo menos garantir a sua abstenção. Numa grande competição, a sua escola teria muita sorte se lhe fosse atribuído um conjunto de países que incluísse um dos Cinco Grandes. Mas, quando menina, eu sabia que os meninos do meu clube não me deixariam representar os Estados Unidos, o Reino Unido e a França. Isto ainda durou mais de uma década antes de Madeleine Albright se tornar a primeira mulher secretária de Estado americana, e os rapazes argumentaram que era menos provável que um dos países ocidentais tivesse uma mulher como representante no Conselho de Segurança. Mesmo na era de Margaret Thatcher e Jeane Kirkpatrick, os homens ainda dominavam as relações exteriores.
 \par 
No entanto, era mais plausível para a China e a URSS. Zoya Mironova foi representante adjunta do conselho de segurança soviético de 1959 a 1962 e serviu como embaixadora na ONU em Genebra de 1966 a 1983. Eu não queria representar a China porque eles se abstiveram em tudo. Então me tornei especialista do Bloco Oriental, esperando que um dia, se tivéssemos a URSS, a cadeira no Conselho de Segurança seria minha. A lição que aprendi aos quinze anos foi que, embora fosse menos plausível que um país ocidental permitisse que as mulheres tomassem decisões cruciais de política externa no cenário mundial, isso era normal para a União Soviética. Mas como isso poderia ser? A democracia era boa, e o comunismo era ruim. Por que os bandidos permitiam que as meninas fizessem mais?
 \par 
Trinta anos depois, para novembro de 2016, eu estava sentado no sofá com minha filha de quinze anos. Estávamos assistindo à PBS e prontos para estourar o champanhe para
 \par 
KRISTEN R. GHODSEE comemora a eleição da primeira mulher presidente dos Estados Unidos. Independentemente dos meus sentimentos pessoais em relação à Hillary Clinton (já me conheces suficientemente bem para suspeitar que eu preferia Bernie Sanders), fiquei emocionado por este tecto de vidro ser finalmente quebrado. Enquanto eu tinha lutado para encontrar modelos de mulheres no poder, esperava que minha filha passasse os anos restantes do ensino médio com uma mulher no Salão Oval. A decepção daquela noite refletiu duas realidades desagradáveis ​​na América: a reação racista contra o primeiro presidente negro e um preconceito persistente contra as mulheres em posições de autoridade. Durante a Guerra Fria, a ascensão de um grande movimento interno de mulheres - combinado com receios políticos sobre o progresso percebido das mulheres nos países socialistas de estado - forçou os países ocidentais a proibir a discriminação com base no sexo e a promover políticas para apoiar a igualdade de gênero no ambiente de trabalho. Ao longo de duas curtas décadas, as mulheres desfrutaram de oportunidades de participação na força de trabalho em quase todos os sectores da economia, entrando em muitas profissões antes consideradas da competência exclusiva dos homens. Hoje, as mulheres constituem a maioria dos diplomados universitários em muitos países capitalistas avançados. Mas, apesar da sua experiência e educação, as mulheres ainda enfrentam barreiras para ocupar cargos de topo no governo e nas empresas. Mais de quarenta anos de ativismo das mulheres pouco fizeram para quebrar o domínio masculino sobre o poder político e econômico.
 \par 
Nos Estados Unidos, existe muita preocupação com a falta de mulheres em cargos de liderança. Embora estudos mostrem que a diversidade na liderança corporativa aumenta a lucratividade, os esforços para desafiar o coisas como são encontram poucos proponentes. Pesquisadores buscam explicações, culpando frequentemente as mulheres por não serem ambiciosas o suficiente ou por não
 \par 
79
 \par 
80
 \par 
TERNOS NÃO SÃO SUFICIENTES: SOBRE LIDERANÇA
 \par 
“Inclinando-se para dentro.” Alguns culpam os desafios de combinar trabalho com responsabilidades familiares e as frequentes interrupções de carreira para aqueles que realizam trabalho de cuidado em casa. Outros dizem que a competição por cargos de topo é desagradável e cheia de traição, e que as mulheres não estão dispostas a entrar na briga. Se o fizerem, os homens ambiciosos as apunhalarão pelas costas primeiro, acreditando que as mulheres são menos propensas a retaliar. Embora todas essas coisas possam contribuir para o problema, a questão subjacente é a persistência de estereótipos de gênero na sociedade, estereótipos internalizados por meninas desde a mais tenra idade. Assim como eu aprendi que não era plausível para mim representar meu país no Conselho de Segurança devido ao meu sexo, minha filha aprendeu que uma mulher bem qualificada com anos de experiência relevante poderia perder uma eleição para um empresário famoso sem experiência governamental.
 \par 
Duas pesquisas de 2014 do Pew Research Center revelaram que a maioria dos americanos reconhece a difusão dessa discriminação de gênero subjacente. Uma pesquisa perguntou aos americanos o que impedia as mulheres de assumirem "posições executivas de alto escalão" e "altos cargos políticos". Enquanto apenas 9% acreditavam que as mulheres não eram "duronas o suficiente" para o mundo dos negócios, 43% alegaram que "as mulheres são mantidas em padrões mais elevados" e que as empresas simplesmente não estavam prontas para contratar mulheres como líderes, apesar de suas qualificações iguais às dos homens. Em termos de altos cargos políticos, apenas 8% alegaram que as mulheres não eram "duronas o suficiente", mas 38% acreditavam que as candidatas eram mantidas em padrões mais elevados e 37% concordaram que os americanos simplesmente não estavam prontos para eleger uma mulher para uma posição de poder. Quando questionados sobre as perspectivas para as próximas décadas, a maioria dos americanos acreditava que "os homens continuarão a ocupar mais cargos de alto escalão do que as mulheres no futuro".
 \par 
KRISTEN R. GHODSEE
 \par 
Isso não é negar que a cultura americana mudou; é apenas notar que ela está mudando em um ritmo muito mais lento em comparação com muitos de nossos pares. Em 1990, apenas 7% dos membros do Congresso dos EUA eram mulheres. Em 2019, um recorde para mulheres membros do Congresso, esse número subiu para 24%. Comparado com alguns dos países escandinavos socialistas democráticos, o lar dos bravos parece um retardatário. A eleição de mulheres membros no parlamento sueco cresceu de 38% em 1990 para 47% em 2018. Na Noruega, 36% dos parlamentares eram mulheres em 1990 e 41% em 2017. Os números relevantes são 31% (1990) a 37% (2017) para a Dinamarca e 32%
 \par 
(1990) para {\color{blue}42} por cento (2015) na Finlândia. A Islândia ganha o prêmio pela quase completa paridade de gênero; a porcentagem de assentos femininos no parlamento cresceu de {\color{blue}21} por cento em 1990 para {\color{blue}48} por cento em 2015, embora tenha declinado para {\color{blue}38} por cento em 2017. Por que a diferença? Uma palavra: cotas.’
 \par 
Em termos de mulheres em cargos de liderança no mundo corporativo, os Estados Unidos ficam ainda mais para trás. Embora as mulheres representassem 45% dos funcionários nas principais empresas da Fortune {\color{blue}500} em 2016, elas ocupavam apenas 21% dos assentos do conselho e representavam apenas 11% dos maiores ganhadores. Compare isso com a Noruega, onde leis rígidas de cotas sobre representação no conselho significam que 42% dos assentos do conselho corporativo foram preenchidos por mulheres. Na Suécia, esse número é de 36%, e na Finlândia é de 31%. Mas mesmo países socialistas democráticos como a Suécia lutam para colocar mulheres na diretoria; a porcentagem de mulheres em cargos executivos ainda estava abaixo de 15% em 2012. E em 2014, o Wall Street Journal relatou que de {\color{blue}145} grandes empresas nórdicas, apenas 3% tinham mulheres
 \par 
81
 \par 
82
 \par 
TERNOS NÃO SÃO SUFICIENTES: SOBRE LIDERANÇA diretores executivos. Embora as mulheres tenham educação e experiência, os cargos de liderança de topo nos negócios em todos os lugares continuam a ser de gênero masculino. A única maneira de quebrar esse domínio contínuo é por meio de legislação que force ou incentive fortemente a paridade de gênero nas posições no topo.
 \par 
Então, o que dizer dos países socialistas estatais? Embora tenham havido esforços importantes para promover mulheres aos mais altos cargos, e eles certamente apoiaram a ideia de que as mulheres poderiam e deveriam estar em posições de poder, a história é complicada pela natureza específica dos regimes do Leste Europeu do século XX. Primeiro, embora houvesse cotas oficiais para mulheres nos parlamentos e nos Comitês Centrais dos Partidos Comunistas da maioria dos estados, a composição do Bureau Político de elite (Politburo), onde estava o poder real, permaneceu esmagadoramente masculina. Segundo, mesmo quando a participação política das mulheres aumentou no nível local e municipal, sua participação foi limitada pela natureza centralizada do estado de partido único. Em termos de posições gerenciais dentro da economia estatal, o quadro também foi misto. O poder de tomada de decisão estava nas mãos dos planejadores centrais, que eram na maioria (embora não exclusivamente) homens. Mas diferentes países tinham prioridades diferentes, e certos setores da economia eram mais receptivos à liderança feminina do que outros. As mulheres dominavam os campos da medicina, direito, academia e bancos e, pelo menos em um nível simbólico, os países socialistas de estado tinham um excelente histórico de promoção de mulheres para cargos de liderança em comparação com os países do Ocidente.
 \par 
KRISTEN R. GHODSEE
 \par 
Ao contrário do capitalismo, que distribui a riqueza da sociedade em um modelo competitivo baseado em ideais de meritocracia e sobrevivência do mais apto, o socialismo apoia uma ideologia igualitária. A desigualdade social é considerada um subproduto inevitável da propriedade privada dos meios de produção: fábricas, máquinas, tecnologias, propriedade intelectual e assim por diante. As economias capitalistas criam uma lacuna de riqueza cada vez maior entre aqueles que possuem os meios de produção e aqueles que devem vender seu trabalho por menos do que o valor que ele cria para atender às suas necessidades básicas. A exploração contínua daqueles que trabalham para viver aumenta a riqueza daqueles no topo; os ricos ficam mais ricos a uma taxa cada vez mais rápida, o que lhes permite controlar cada vez mais os meios de produção. As políticas socialistas interrompem essa tendência de crescente desigualdade por meio de uma série de mecanismos, incluindo a criação de empresas públicas ou de propriedade coletiva (cooperativas) e/ou redistribuição de riqueza por meio de tributação progressiva e a criação de redes de segurança social financiadas publicamente para evitar a miséria. Além de promover os interesses da maioria pobre sobre os da minoria rica, no entanto, nada inerente à ideologia socialista privilegia qualquer grupo social sobre outro. E a emancipação das mulheres foi fundamental para a visão socialista desde o seu início (mesmo que a identidade de classe das mulheres tenha sido sempre privilegiada sobre sua identidade de gênero).
 \par 
A ideia de que homens e mulheres compartilhariam o poder político tinha raízes nas primeiras encarnações dos ideais socialistas, que surgiram após a Revolução Francesa. Nas décadas de 1820 e 1830, os utópicos socialistas saint-simonianos se organizaram em pequenas comunidades religiosas em Paris, juntando suas rendas e vivendo coletivamente.
 \par 
83
 \par 
84
 \par 
TERNOS NÃO SÃO SUFICIENTES: SOBRE LIDERANÇA o líder inicial, Prosper Enfantin, serviu como o "papa" da comunidade; ele propôs compartilhar sua posição de autoridade com uma mulher que serviria como "papas". Ao contrário de Mary Wollstonecraft e John Stuart Mill, que basearam seus argumentos para igualdade sexual na racionalidade inata de homens e mulheres, os saint-simonianos acreditavam que homens e mulheres tinham naturezas diferentes, mas complementares, e que tanto a autoridade espiritual quanto a política exigiam representação de cada metade da humanidade. Após debates internos, as opiniões de Enfantin prevaleceram, e a comunidade saint-simoniana maior seria governada por um casal-papa que serviria como representantes vivos dos atributos masculinos e femininos de Deus. Todas as posições de poder deveriam ser compartilhadas por um representante de cada sexo: cada comunidade menor era chefiada por um casal homem-mulher, suas casas coletivas eram lideradas por um par de “irmãos” e “irmãs”, e cada um de seus sindicatos de trabalho era governado por um “diretor” e uma “direção”.
 \par 
Outro socialista utópico proeminente foi o francês Charles Fourier, que se acredita ter cunhado a palavra “feminismo” em 1837. Fourier era um defensor ferrenho dos direitos das mulheres e acreditava que todas as profissões deveriam ser abertas às mulheres com base em suas habilidades como indivíduos. Fourier entendeu que as mulheres europeias não eram melhores do que bens móveis para seus pais e maridos, e ele propôs que as sociedades iluminadas demonstrariam seu progresso moral libertando as mulheres dos papéis estreitos de gênero que as prendiam ao casamento convencional. Fourier promoveu a ideia de comunidades agrícolas de propriedade coletiva (chamadas de “falanges”) nas quais homens e mulheres trabalhariam lado a lado e compartilhariam os frutos de seus trabalhos em comum. Fourier escreveu: “O progresso social e as mudanças históricas ocorrem
 \par 
KRISTEN R. GHODSEE em virtude do progresso das mulheres em direção à liberdade, e a decadência da ordem social ocorre como resultado da diminuição da liberdade das mulheres.”
 \par 
Os saint-simonianos e Charles Fourier influenciaram o trabalho de outro importante socialista utópico francês, a fascinante Flora Tristan. Ela foi a primeira teórica a conectar a emancipação das mulheres com a libertação das classes trabalhadoras. Ela entendeu que a relação da esposa com o marido era análoga à do proletariado com a burguesia. Escrevendo e dando palestras no final da década de 1830 e início da década de 1840, Tristan via o feminismo e o socialismo como movimentos mutuamente dependentes que trariam uma transformação total da sociedade francesa; a emancipação das mulheres não poderia acontecer sem a libertação dos trabalhadores, e vice-versa. Em vez de um modelo no qual a igualdade sexual era enganada pelos ganhos legais e maiores oportunidades educacionais de mulheres ricas, Tristan acreditava que a criação de um sindicato grande e diverso de trabalhadores (composto por homens e mulheres) realizaria a igualdade sexual primeiro entre as classes trabalhadoras.
 \par 
Expandindo essas ideias, os socialistas alemães August Bebel e Friedrich Engels propuseram uma justificativa histórica para a emancipação das mulheres, argumentando que caçadores e coletores viveram em matriarcados comunitários primitivos. De acordo com suas teorias, os primeiros humanos sobreviveram — em clãs que consistiam em homens e mulheres que praticavam uma forma de casamento em grupo e criavam seus filhos coletivamente. Como a paternidade não podia ser estabelecida, a descendência era traçada pela mãe, e as mulheres tinham uma participação igual, se não maior, na tomada de decisões. Bebel e Engels argumentaram que foi somente após o advento da agricultura e
 \par 
85
 \par 
86
 \par 
TERNOS NÃO SÃO SUFICIENTES: NA LIDERANÇA propriedade privada que a riqueza poderia ser acumulada. Caçadores e coletores não acumulavam recursos; eles consumiam tudo o que caçavam e coletavam. Mas quando alguns humanos começaram a cercar grandes extensões de terra para produzir mais comida do que precisavam para sobreviver e começaram a vender o excedente, um novo conjunto de incentivos destruiu antigas estruturas sociais. Os proprietários de terras precisavam de trabalhadores para ajudá-los a criar maiores excedentes, e foi nesse momento da história que os corpos das mulheres se tornaram máquinas para fabricar mais trabalhadores. (Eles argumentam que essa era também coincidiu com a invenção da escravidão).*
 \par 
De acordo com Bebel e Engels, uma vez que os proprietários de terras começaram a acumular fortunas privadas, essa classe de homens desejou passar sua riqueza para herdeiros legítimos. Isso precipitou a invenção do casamento monogâmico e a fidelidade forçada da esposa. O antigo sistema matrilinear foi substituído por um sistema patrilinear pelo qual a descendência era rastreada através do pai. (Podemos ver a operação desse patrimônio hoje, quando as mulheres assumem os sobrenomes de seus maridos ao se casarem e os filhos recebem os sobrenomes de seus pais. Em um sistema matrilinear, seria o inverso.) Engels postulou que esse desejo de acumular riqueza roubou das mulheres sua autonomia anterior: “A derrubada do direito materno foi a derrota histórica mundial do sexo feminino. O homem assumiu o comando em casa também; a mulher foi degradada e reduzida à servidão, ela se tornou escrava de sua luxúria e um mero instrumento para a produção de filhos.” Para os primeiros socialistas, portanto, a abolição da propriedade privada levaria inevitavelmente à restauração do papel “natural” das mulheres como iguais aos homens.
 \par 
KRISTEN R. GHODSEE
 \par 
As ideias socialistas sobre a emancipação das mulheres ajudariam a alimentar impulsos revolucionários na Rússia em 1917. A revolução de fevereiro que derrubou o czar Nicolau II começou no Dia Internacional da Mulher, precipitada por mulheres grevistas. Enquanto um governo provisório tentava estabilizar a Rússia nos meses seguintes, essas mulheres exigiam sufrágio pleno. Em julho de 1917, elas ganharam o direito de votar e se candidatar a cargos públicos. Após a Revolução de outubro, Lenin e os bolcheviques permitiram que as mulheres votassem e concorressem nas eleições para a Assembleia Constituinte. A maioria das pessoas não percebe que a União Soviética não se tornou um estado autoritário de partido único muito rapidamente. Como Lenin esperava ganhar um mandato popular, ele permitiu "as eleições mais livres já realizadas na Rússia até depois do colapso da União Soviética em 1991", conforme a historiadora Rochelle Ruthchild. A votação começou em novembro de 1917 e durou cerca de um mês. A participação dos eleitores nas eleições da Assembleia Constituinte foi incrível, dado o caos da época, e a participação eleitoral das mulheres superou todas as expectativas. No entanto, Lenin dissolveu a Assembleia Constituinte democraticamente eleita quando ficou claro que seu partido bolchevique não teria maioria. O direito das mulheres soviéticas de votar tornou-se amplamente supérfluo na ditadura do proletariado.”
 \par 
Apesar da instituição do “comunismo de guerra” e da centralização da autoridade política, Lenin inicialmente capacitou um grupo de ativistas para lançar as bases para a emancipação total das mulheres. Alexandra Kollontai serviu como comissária do povo para o bem-estar social e ajudou a fundar a organização soviética de mulheres, a Zhenotdel. Como
 \par 
87
 \par 
88
 \par 
TERNOS NÃO SÃO SUFICIENTES: SOBRE LIDERANÇA discutido anteriormente, ela seria responsável por implementar uma ampla gama de políticas para apoiar a incorporação total das mulheres na força de trabalho soviética. A jornalista americana Louise Bryant ficou impressionada com o comprometimento e a falta de medo de Kollontai ao lidar com os homens bolcheviques. Bryant relatou em 1923:
 \par 
SSSSSS ruim. Ela tem coragem ilimitada e em várias ocasiões
 \par 
Sions se opôs abertamente a Lenin. Quanto a Lenin, ele a esmagou com sua habitual franqueza imperturbável. No entanto, apesar de seu entusiasmo ardente, ela entende “disposição partidária
 \par 
Ciline” e aceita a derrota como um bom soldado. Se ela tivesse abandonado a revolução quatro meses após seu início, ela poderia
 \par 
Descansou para sempre sobre os louros. Ela aproveitou aqueles primeiros momentos róseos de euforia, logo após as massas terem capturado
 \par 
Tured o estado, para incorporar à Constituição leis para as mulheres de longo alcance e sem precedentes.
 \par 
E os soviéticos estão muito orgulhosos dessas leis que já têm ao seu redor o halo de todas as coisas relacionadas com a Constituição.”
 \par 
Kollontai acabaria sendo enviada como embaixadora soviética na Noruega, a primeira mulher russa a ocupar um cargo diplomático tão alto (e a terceira embaixadora mulher no mundo), mas após a ascensão de Stalin ela cairia em relativa obscuridade, com muitos de seus sonhos originais de emancipação feminina desacreditados ou esquecidos.
 \par 
Entre as outras mulheres proeminentes que trabalharam com o Zhenotdel na década de 1920 estava Nadezhda Krupskaya,
 \par 
KRISTEN R. GHODSEE
 \par 
A esposa de Lenin, uma pedagoga radical que serviu como vice-ministra da educação de 1929 a 1939. Ela trabalhou para construir novas escolas e bibliotecas para uma população em que seis em cada dez pessoas não sabiam ler ou escrever em 1917, e seus ideais educacionais iriam inspirar reformadores educacionais de esquerda como Paulo Freire no Brasil. Outra bolchevique proeminente, Inessa Armand, trabalhou como líder no Conselho Econômico de Moscou, serviu como um membro importante do
 \par 
O Soviete de Moscou, e seria eventualmente o diretor do Zhenotdel. Inúmeras outras mulheres bolcheviques assumiriam posições de poder no governo soviético inicial enquanto o país lutava para sobreviver a uma guerra civil, uma fome horrível e a morte prematura de Lenin.”
 \par 
A era stalinista viu um retorno relativo aos papéis tradicionais de gênero, mesmo quando os soviéticos encorajaram as mulheres a se envolverem em treinamento militar. A historiadora Anna Krylova explorou a lenta integração das mulheres soviéticas nas forças armadas, apesar da resistência masculina inicial. Na Segunda Guerra Mundial, a URSS tinha esquadrões de pilotos de caça treinados. Isso incluía as infames Nachthexen (bruxas noturnas) do 588º Regimento de Bombardeiros Noturnos das Forças Aéreas Soviéticas, que voavam em modo furtivo à noite e lançavam bombas de precisão em alvos alemães. As pilotos estavam todas no final da adolescência e no início dos vinte anos, e voaram mais de vinte mil missões de 1941 a 1945. Embora outros países tivessem pilotos treinadas que voavam em funções de apoio, a União Soviética foi o primeiro país do mundo a permitir que as mulheres voassem em missões de combate. Os nazistas temiam essas pilotos, e qualquer piloto alemão que atirasse em uma "bruxa" do céu supostamente ganhava uma Cruz de Ferro automática. Em toda a Europa Oriental, a Segunda Guerra Mundial também inspirou
 \par 
89
 \par 
90
 \par 
CALÇAS NÃO SÃO SUFICIENTES: NA LIDERANÇA, milhares de mulheres pegariam em armas como guerrilhas antinazistas, e muitas seguiriam carreiras na política nacional e internacional. Por exemplo, vida Tom$i¢ era uma comunista eslovena que lutou como partidária contra os italianos e tornou-se ministra da política social do seu país após a guerra. Ela serviu em uma ampla variedade de cargos governamentais e tornou-se uma ativista dedicada às mulheres, tanto na Iugoslávia quanto internacionalmente, durante a Guerra Fria. Estudiosa do direito e jurista, Tomai¢ foi reverenciada como heroína nacional entre 1945 e 1991, e representou a Iugoslávia em vários cargos nas Nações Unidas.”
 \par 
A vizinha Bulgária também produziu mulheres antifascistas espirituosas que entrariam mais tarde na política. Elena Lagadinova foi a mais jovem partidária feminina lutando contra a monarquia aliada aos nazistas de seu país. Mais tarde, ela obteve um PhD em astrobiologia e trabalhou por treze anos como cientista pesquisadora antes de servir como presidente do Comitê do Movimento das Mulheres Búlgaras por vinte e dois anos. Lagadinova também foi membro do Parlamento, membro do Comitê Central e uma defensora apaixonada dos direitos das mulheres no cenário internacional, particularmente durante a Década das Nações Unidas para as Mulheres entre 1975 e 1985. Outra partidária búlgara foi Tsola Dragoycheva, que lutou contra o regime monarquista de direita da Bulgária a partir da década de 1920. Uma heroína do Partido Comunista Búlgaro, Dragoycheva serviu como a primeira mulher da Bulgária a ocupar um cargo de gabinete como ministra do Serviço Postal, Telégrafo e Telefone após a Segunda Guerra Mundial. De 1944 a 1948, ela também atuou como secretária-geral do Comitê Nacional da Frente Pátria, chefiou o Conselho de Ministros e exerceu grande influência sobre
 \par 
KRISTEN R. GHODSEE o desenvolvimento da economia recém-planejada da Bulgária. Mais tarde, ela se tornaria um membro pleno do Politburo Búlgaro, uma das poucas mulheres no Bloco Oriental a ascender a uma posição tão alta sem ser esposa ou filha de um líder comunista.'°
 \par 

 \par 
Outras mulheres socialistas na Europa Oriental entraram e saíram da prisão por suas atividades políticas na década de 1930 ou passaram um tempo como exiladas na União Soviética até poderem retornar para casa após o fim da Segunda Guerra Mundial. Na Romênia, a ascensão de "Tia Ana" Pauker mostrou ao mundo que o socialismo de estado permitiria que as mulheres assumissem os mais altos cargos no governo, chocando os observadores ocidentais. Escrevendo no New York Times em 1948, o jornalista W. H. Lawrence relatou: "Ana Pauker é arquiteta e construtora do novo estado comunista romeno [assim usado]. Ela não apenas planeja, mas traduz projetos políticos, econômicos e sociais em ação como secretária do Partido Comunista Romeno e Ministra das Relações Exteriores da recém-proclamada república — a primeira mulher no mundo a ter o título de. Do ponto de vista de Ministra das Relações Exterior internacional. .
 \par 
Comunismo, Ana Pauker é uma história de sucesso de Horatio Alger — de trapos políticos a riquezas políticas.” Em setembro de 1948, a Time apresentou seu retrato na capa e a rotulou como “a mulher mais poderosa viva”.
 \par 
Os países do Bloco Oriental também se destacaram em demonstrações estratégicas internacionais de seu comprometimento com os direitos das mulheres, particularmente no caso de Valentina Tereshkova. Em junho de 1963, apenas cinco anos após o lançamento do Sputnik, a primeira página do New York Herald Tribune dizia: “Loira soviética orbitando como primeira mulher no espaço”. No mesmo ano em que Betty Friedan publicou The Feminine Mystique,
 \par 
91
 \par 
92
 \par 
TERNOS NÃO SÃO SUFICIENTES: SOBRE LIDERANÇA, a manchete do Massachusetts Springfield Union declarou: "Soviéticos orbitam a primeira cassonete". Os soviéticos fizeram de Tereshkova um símbolo de sua política social progressista, e ela liderou suas delegações nas três conferências mundiais da ONU sobre mulheres em 1975, 1980 e 1985. Em 1982, a cosmonauta Svetlana Sevitskaya foi a primeira mulher a voar em uma estação espacial, um ano antes de Sally Ride se tornar a primeira astronauta americana. Dois anos depois, Sevitskaya completou a primeira caminhada espacial de uma mulher e se tornou a primeira mulher a completar duas missões espaciais separadas."®
 \par 
Embora as mulheres soviéticas raramente se aventurassem no reino da alta política, houve algumas exceções importantes. Em 1917, Elena Stasova foi a primeira mulher a se tornar membro candidata do Politburo soviético, o mais alto órgão político do país, embora seu mandato tenha sido muito breve. Décadas depois, em 1957, Ekaterina Furtseva foi eleita membro pleno do Politburo, servindo por quatro anos. Ela apoiou as políticas de dessalinização de Khrushchev e, eventualmente, deixou o Politburo para se tornar ministra da cultura de 1960 a 1974. Em setembro de 1988, Alexandra Biryukova se tornou membro candidata do Politburo, que tinha posição sem direito a voto. Finalmente, em 1990, Galina Semyonova foi a segunda mulher a se tornar membro pleno com direito a voto do Politburo. Nomeada pelo próprio Gorbachev como um primeiro passo em seu plano de colocar mais mulheres em posições de poder, Galina Semyonova obteve um doutorado em filosofia e passou trinta e um anos como jornalista ativa. Aos cinquenta e três anos, ela era mãe e avó. Sua eleição sinalizou que os soviéticos estavam prontos para levar as questões das mulheres domésticas mais a sério. Em uma entrevista de janeiro de 1991 com o Los Angeles Times, Semyonova criticou abertamente a
 \par 
KRISTEN R. GHODSEE
 \par 
Políticas anteriores do governo soviético em relação à liderança feminina. “Desde a fundação do nosso estado”, ela disse ao jornalista americano, “temos muitas leis muito humanas. Lenin assinou pessoalmente muitas decisões e leis sobre a família, sobre o casamento, os direitos políticos das mulheres, a liquidação do analfabetismo entre a população feminina. Mas essas leis, de fato, eram frequentemente neutralizadas pela prática socioeconômica. O resultado foi que as mulheres não estavam preparadas para assumir o papel de liderança na sociedade.” Usando as novas liberdades concedidas sob a perestroika, Semyonova esperava que colocar mais mulheres soviéticas em papéis de liderança tornaria a política “mais humana e evitaria que se tornasse muito agressiva”.
 \par 
Embora esses exemplos de alto nível demonstrem o comprometimento dos países socialistas estatais com o ideal dos direitos das mulheres, a prática real nem sempre correspondeu à retórica. Entre 2010 e 2017, passei mais de cento e cinquenta horas entrevistando a octogenária Elena Lagadinova, presidente da organização nacional de mulheres da Bulgária. Lagadinova admitiu que os estados socialistas não alcançaram tanto quanto ela esperava. Certa vez, perguntei a ela por que mais mulheres não ascenderam às posições mais altas de poder, dado o comprometimento geral com os direitos das mulheres. Lagadinova reconheceu que esse era um desafio contínuo para o comitê de mulheres búlgaras e afirmou que os países do Leste Europeu não tinham tempo suficiente para superar a ideia secular de que os líderes deveriam ser homens. Não era apenas que os homens não gostavam de mulheres no poder, argumentou Lagadinova; era que as mulheres também se sentiam desconfortáveis ​​com a liderança feminina. Como resultado, elas eram menos propensas a apoiar suas companheiras e mais reticentes
 \par 
93
 \par 
94
 \par 
TERNOS NÃO SÃO SUFICIENTES: SOBRE LIDERANÇA para buscar posições de autoridade. Elas preferiam trabalhar nos bastidores, ela disse. A alta política na Europa Oriental, assim como a alta política em outros lugares, era um lugar traiçoeiro, infundido com intrigas e traições. Lagadinova sugeriu que as mulheres eram menos inclinadas a se envolver nos subterfúgios necessários. Por outro lado, ela acreditava que a vida política poderia ter sido mais civilizada se houvesse mais mulheres no topo. Sua organização tentou promover candidatos qualificados quando podiam, mas a cultura patriarcal dos Bálcãs, combinada com a natureza autoritária do estado (governado pelo mesmo homem por trinta e cinco anos), desencorajou as mulheres de se envolverem.
 \par 
Para encorajar mais mulheres a se arriscarem na política, a Bulgária e outros países socialistas introduziram cotas para mulheres no parlamento, e eles tiveram porcentagens maiores de mulheres ocupando cargos políticos do que a maioria das democracias ocidentais durante a Guerra Fria. As posições das mulheres no aparato governamental de um estado de partido único eram amplamente simbólicas, mas o simbolismo era importante. Afinal, os membros masculinos do parlamento e do Comitê Central não gozavam de maior autoridade do que suas companheiras. As mulheres se saíram melhor em empregos de colarinho branco na economia planejada, muitas vezes dominando o setor bancário, a medicina, a academia e o judiciário. Parte dessa tendência refletiu políticas específicas para promover mulheres nas profissões, mas também era o caso de que os empregos industriais de colarinho azul pagavam salários mais altos sob o socialismo de estado, então os homens tendiam a concentrar seu trabalho nesses setores da economia. Mas, como discutido no Capítulo 1, as taxas de participação feminina na força de trabalho eram as mais altas do mundo. Como o número de mulheres na força de trabalho era maior, havia numericamente mais mulheres
 \par 
KRISTEN R. GHODSEE em cargos gerenciais. Além disso, os países do Bloco Oriental fizeram um excelente trabalho em canalizar mulheres para os setores de ciência, engenharia e tecnologia. Um artigo de {\color{blue}9} de março de 2018 no Financial Times revelou que oito dos dez principais países europeus com as maiores taxas de mulheres no setor de tecnologia estavam na Europa Oriental, um legado da era soviética, quando as mulheres eram encorajadas a seguir essas carreiras. De fato, entre 1979 e 1989, a porcentagem de mulheres na URSS trabalhando como "especialistas em engenharia e técnicas" aumentou de {\color{blue}48} para {\color{blue}50} por cento de todos os trabalhadores nesses campos — paridade exata. Também em 1989, {\color{blue}73} por cento de todos os "trabalhadores científicos, professores e educadores" soviéticos eram mulheres.
 \par 
Cotas obrigatórias pelo Estado para mulheres em cargos políticos, em conselhos corporativos e em empresas públicas foram implementadas em países democráticos ao redor do mundo, e estudos mostram que elas têm sido notavelmente eficazes, se aplicadas corretamente, no aumento do número de mulheres em posições de autoridade. Desde 1991, mais de noventa países implementaram algum tipo de sistema de cotas para mulheres em parlamentos nacionais, e a porcentagem de mulheres em posições de poder disparou, criando modelos para a próxima geração de meninas aspirantes a carreiras na política. Em 2017, dos quarenta e seis países que têm 30% ou mais de mulheres em seus parlamentos, quarenta deles têm alguma forma de sistema de cotas em vigor. Mas as cotas funcionam melhor em sistemas eleitorais baseados em representação proporcional, nos quais os cidadãos votam em partidos em vez de indivíduos. As cotas podem legislar que uma certa porcentagem dos nomes
 \par 
95
 \par 
96
 \par 
TERNOS NÃO SÃO SUFICIENTES: SOBRE LIDERANÇA em uma chapa eleitoral estão as de mulheres. Como os americanos votam em políticos individuais em círculos eleitorais uninominais, as cotas seriam difíceis de impor. Se os partidos políticos tivessem que apresentar um certo número de mulheres, eles poderiam concentrá-las em círculos eleitorais onde sabem que as mulheres perderão. Mas poderia haver cotas para cargos nomeados no gabinete, por exemplo, ou outras maneiras criativas de aumentar a participação das mulheres sem reformular o sistema eleitoral.”
 \par 
Cotas obrigatórias pelo Estado para mulheres nos conselhos executivos de corporações e empresas públicas promoveram com sucesso mulheres a cargos de liderança e são bastante viáveis ​​no contexto dos EUA. As cotas foram introduzidas pela primeira vez na Noruega em 2003; as empresas enfrentavam dissolução se não diversificassem seus conselhos. Para grandes empresas, 40% dos assentos no conselho precisavam ser ocupados por mulheres. Depois da Noruega, outros países europeus impuseram cotas às corporações, embora com penalidades mais brandas para o não cumprimento. Talvez não seja surpreendente que, quanto mais brando o mandato, menos empresas obedeciam. Embora a porcentagem de mulheres servindo nos conselhos de grandes empresas de capital aberto tenha aumentado de 11% em 2007 para 23% em 2016, esse número foi significativamente maior em países com cotas rígidas em vigor. 44% na Islândia, 39% na Noruega e 36% na França. Na Alemanha, onde as cotas são voluntárias, exceto para grandes empresas, a porcentagem é de apenas 26%. Como resultado, a Comissão Europeia tentou, em 2017, pressionar por uma lei em toda a UE que exigisse que as grandes empresas em todos os estados-membros impusessem uma cota de 40% para mulheres nos conselhos corporativos.”°
 \par 
É claro que nenhuma mulher quer se sentir uma cidadã de segunda classe ou ocupar uma posição apenas por ser mulher,
 \par 
KRISTEN R. GHODSEE, então é importante perceber que a discriminação contínua contra mulheres em posições de liderança não é porque os americanos acham que as mulheres são menos capazes ou não têm os atributos de liderança necessários. Uma pesquisa Pew de 2014 sobre mulheres e liderança descobriu que a maioria dos entrevistados não viu nenhuma diferença entre as habilidades inatas de homens e mulheres. Em algumas categorias, como honestidade, capacidade de orientar funcionários e disposição para se comprometer, os americanos que acreditavam que havia diferenças entre homens e mulheres achavam que as mulheres eram melhores do que os homens. A discriminação contra mulheres na liderança tem pouca base em conjuntos de habilidades diferenciais e mais a ver com atitudes sociais sobre mulheres no poder. Portanto, não se trata de colocar mulheres menos qualificadas em posições de liderança porque são mulheres; trata-se de tentar neutralizar os estereótipos de gênero profundos e inconscientes sobre homens como líderes e mulheres como seguidoras. Algumas pessoas simplesmente se sentem estranhas com uma chefe mulher.”
 \par 
Não associamos mulheres a posições de autoridade porque vimos muito poucas delas. E como há tão poucas mulheres em posições de autoridade, tanto homens quanto mulheres continuam a associar liderança a corpos masculinos, um ciclo vicioso do qual é difícil sair. (Um problema semelhante pode ser encontrado para mulheres nas ciências, engenharia e tecnologia.) Quando questionadas sobre fatores que afetam sua própria ambição e disposição para concorrer a eleições ou competir por cargos importantes, as mulheres culpam geralmente a falta de modelos femininos por sua reticência. Por exemplo, a empresa de consultoria KPMG conduziu um Estudo de Liderança Feminina em 2015, pesquisando {\color{blue}3}.{\color{blue}014} mulheres dos EUA entre dezoito e sessenta e quatro anos. Em termos de aprendizado sobre liderança, 67% alegaram que suas lições mais importantes vieram de outras mulheres.
 \par 
97
 \par 
98
 \par 
TERNOS NÃO SÃO SUFICIENTES: SOBRE LIDERANÇA, 88% relataram que foram encorajados ao ver mulheres em posições de liderança, e 86% concordaram com a declaração "Quando veem mais mulheres na liderança, elas são encorajadas a chegar lá por si mesmas". Finalmente, 69% das mulheres pesquisadas concordaram "que ter mais mulheres representadas na liderança sênior ajudará a mover mais mulheres para cargos de liderança no futuro". Devido à importância dos modelos, a KPMG recomendou a promoção de mulheres qualificadas para altos cargos gerenciais, para conselhos corporativos e para a diretoria. E um estudo de 2016 da Fundação Rockefeller descobriu que 65% dos americanos "dizem que é especialmente importante para as mulheres que estão começando suas carreiras ter mulheres em posições de liderança como modelos". Mas sabemos pelas experiências na Europa que isso não acontecerá sem algum tipo de intervenção externa".
 \par 
Embora a liderança feminina seja importante, também vale a pena notar que as cotas nos negócios e na política podem beneficiar apenas uma pequena porcentagem de mulheres brancas de classe média. Se nos concentrarmos em promover mulheres a posições de poder, excluindo outras questões urgentes que afetam mulheres pobres e da classe trabalhadora, particularmente mulheres de cor, caímos na perigosa armadilha do feminismo corporativo à la Ivanka Trump. Sim, o teto de vidro precisa ser quebrado, mas isso não significa que devemos ignorar os problemas urgentes desta posição que não está nem perto de ser tão alta na hierarquia. Tanto homens quanto mulheres executivos, assim como homens e mulheres na política, muitas vezes constroem suas carreiras nas costas de mulheres mais pobres: babás, Au país, cozinheiras, faxineiras, auxiliares de saúde domiciliar, enfermeiras e assistentes pessoais para quem terceirizam seu trabalho de assistência. Políticas para ajudar as mulheres a chegar ao topo sempre
 \par 
KRISTEN R. GHODSEE deve ser combinada com medidas práticas para ajudar as mulheres que estão em dificuldades, ou elas simplesmente agravarão as desigualdades existentes.
 \par 
Por exemplo, se os governos federal, estadual ou local algum dia abraçassem a ideia de programas de emprego para desempregados, seria sensato associar essa política a uma cota obrigatória estipulando que 50% de todos os empregos criados seriam reservados para mulheres. Não é de todo impensável que os políticos possam decidir criar um programa especial de empregos para homens, acreditando que as mulheres não precisam trabalhar, pois têm responsabilidades em casa. Nos primeiros anos da transição econômica em alguns países do Leste Europeu, as políticas de criação de empregos visavam os homens deslocados, partindo do princípio de que o modelo de ganha-pão masculino/dona de casa feminina era mais desejável do que o inverso. Como a criação de empregos no setor privado empalideceu em comparação com as perdas de empregos causadas pela rápida privatização ou liquidação de empresas estatais, simplesmente não havia empregos suficientes na economia para empregar todas as pessoas que ficaram redundantes pela transição econômica. Para controlar o desemprego, as mulheres foram forçadas a voltar para casa por políticas de reabilitação, e houve discriminação explícita no emprego contra as mulheres, pois a imposição de mercados livres veio com o retorno dos papéis tradicionais de gênero.
 \par 
Mas quando a maioria das pessoas fala sobre cotas, elas geralmente estão discutindo cotas para posições de elite de poder, e também é importante perceber que as cotas por si só não podem remover todas as barreiras. Pode haver outras maneiras de aumentar o número de mulheres em posições de liderança, mas a ideia central é criar modelos mais positivos, que podem começar a remodelar as atitudes sociais. Todas as mulheres e meninas são prejudicadas quando a sociedade
 \par 
99
 \par 
100
 \par 
PANTSUITS ARE NOT ENOUGH: ON LEADERSHIP retrata mulheres ambiciosas como perversas ou feias, imaginando poder e autoridade como um traço de caráter naturalmente masculino. A cultura patriarcal permeia a sociedade, e tanto homens quanto mulheres se sentem desconfortáveis ​​com mulheres no poder. Mulheres fortes e competentes são consideradas menos femininas, se não totalmente desagradáveis. Observe a linguagem usada na descrição da Time de 1948 da "mulher mais poderosa viva", Ana Pauker da Romênia: "Agora ela é gorda e feia; mas antes ela era magra e (seus amigos lembram) bonita. Antes ela era calorosa, tímida e cheia de pena pelos oprimidos, dos quais ela era um. Agora ela é fria como o Danúbio congelado, ousada como um Boyar em sua própria terra rica e implacável como uma foice no grão da Moldávia." A feiura de Pauker se desenvolve à medida que sua autoridade política se expande; sua natureza tímida e calorosa é corrompida por sua entrada nos corredores do poder dominados pelos homens. Não é de surpreender que a imagem de Pauker na capa da Time seja um perfil nada lisonjeiro de uma mulher de meia-idade, irritada, com cabelos curtos e grisalhos.”
 \par 
Essa imagem negativa de mulheres comunistas gordas e feias foi conscientemente produzida e reproduzida pela mídia americana durante a Guerra Fria. Crescendo na era Reagan, eu acreditava naqueles estereótipos horríveis que circulavam sobre mulheres soviéticas pouco atraentes. Lembro-me de um anúncio da rede de hambúrgueres Wendy's em meados da década de 1980 — um desfile de moda, estilo soviético. Brincando com os piores tropos americanos, o comercial mostra uma mulher gorda de meia-idade usando um avental cinza e um lenço de avó em volta do cabelo. Ela desfila para cima e para baixo em uma passarela abaixo de um retrato de Lenin. Outra mulher gorda e masculina em um uniforme militar verde-oliva grita: "Roupa para o dia", "Roupa para a noite" e "Roupa de banho" enquanto a primeira mulher sai usando o
 \par 
KRISTEN R. GHODSEE o mesmo avental, segurando apenas uma lanterna para "roupas de noite" e uma bola de praia para "roupas de banho". A narração do anúncio informa aos espectadores que eles têm uma escolha no Wendy's (ao contrário das pessoas na URSS), mas foi a imagem da feminilidade soviética (ou a falta dela) que tornou o comercial tão poderoso. Eu ainda era adolescente quando vi este anúncio pela primeira vez, e certamente me ocorreu que querer exercer poder de veto poderia de alguma forma me tirar minha feminilidade. Quando finalmente tive minha chance de uma cadeira no Conselho de Segurança, me perguntei se os meninos achavam que estavam me punindo ao me fazer representar o "império do mal".
 \par 
É claro que representar os países do Bloco de Leste também foi muito mais difícil do que representar os Estados Unidos, os
 \par 
Reino Unido ou França. Para ser um país ocidental, tudo o que você tinha que fazer era ler o jornal ou ler compulsivamente o U.S. News ano World Report. Descobrir as motivações ideológicas e práticas para as posições de política externa soviética e do Bloco Oriental exigia habilidades de pesquisa experientes. Naquela época, muito antes da internet, a pesquisa de política externa tinha que ser feita usando fontes impressas, geralmente disponíveis apenas em uma biblioteca. E se você quisesse ler registros reais das Nações Unidas, você tinha que encontrar uma maneira de chegar a uma biblioteca universitária. Mas para ganhar o prêmio principal, eu tinha que ler livros e relatórios produzidos pelos países do Bloco Oriental. Eu precisava entender suas visões de mundo, para que eu pudesse representá-los de forma mais convincente.
 \par 
Foi em 1987 que me deparei com um grande livro de capa dura para mesa de centro enquanto conduzia uma pesquisa de fundo para a conferência da MUN, onde eu representava a União Soviética no Conselho de Segurança. Publicado em 1975 para coincidir com o Ano Internacional das Mulheres das Nações Unidas,
 \par 
101
 \par 
102
 \par 
TERNOS NÃO SÃO SUFICIENTES: SOBRE LIDERANÇA
 \par 
Mulheres na Sociedade Socialista era uma elegante peça de propaganda da Alemanha Oriental, celebrando os ganhos das mulheres no Bloco Oriental. Embora eu desconfiasse do texto didático em inglês, fiquei encantado com as imagens. As fotos de Rosa Luxemburgo e Alexandra Kollontai, esta última uma jovem mulher incrivelmente bonita. A adorável Valentina Tereshkova de {\color{blue}26} anos em seu uniforme. Como se respondesse diretamente ao estereótipo ocidental das mulheres do Bloco Oriental como cansadas, gordas e feias, os alemães orientais incluíram um capítulo inteiro sobre "Mulheres, Socialismo, Beleza e Amor", completo com fotografias estilizadas em preto e branco de modelos deslumbrantes exibindo seus seios empinados pela causa. Espalhadas pelas páginas brilhantes, havia mulheres esbeltas e bonitas trabalhando em fábricas, laboratórios, salas de aula e sentadas em mesas de conferência com homens. Mulheres competindo nas Olimpíadas, mulheres sorrindo para seus filhos e mulheres rindo juntas como colegas de trabalho.
 \par 
Mais tarde, quando aprendi sobre a economia de comando, entendi que as imagens no livro representavam mais o ideal comunista do que a realidade vivida do socialismo de estado na Europa Oriental. No final dos anos 1990, quando morei pela primeira vez na Bulgária, vendedores ambulantes vendiam calcinhas femininas em cada esquina. Em bancas de jornais, você podia comprar uma tanga de renda com sua edição matinal, porque as pessoas estavam tentando compensar sua relativa privação antes de 1989. Sob o socialismo de estado, os planejadores centrais ignoravam os desejos das mulheres, e havia escassez persistente de apetrechos femininos que as mulheres tomam como garantidos no Ocidente, incluindo produtos básicos de higiene. Mulheres búlgaras de uma certa idade ainda se encolhem quando pensam no enchimento de algodão áspero que tinham que usar uma vez por mês (se pudessem encontrá-lo).
 \par 
KRISTEN R. GHODSEE
 \par 
Slavenka Drakulić capturou essa frustração quando viajou pela Europa Oriental para a Ms. em 1991, relatando a reclamação que ela “ouviu repetidamente de mulheres em Varsóvia, Budapeste, Praga, Sófia, Berlim Oriental: “Olhe para nós — nem parecemos mulheres. Não há desodorantes, perfumes, às vezes nem sabonete ou pasta de dente. Não há roupas íntimas finas, nem meia-calça, nem lingerie bonita. Pior de tudo, não há absorventes higiênicos. O que se pode dizer, exceto que é humilhante?” Embora as mulheres na Europa Oriental possam ter tido muito mais caminhos de carreira abertos para elas, elas certamente não tinham os produtos de consumo disponíveis para as mulheres no Ocidente.”{\color{blue}4}
 \par 
Mas como estudante do ensino médio, eu não sabia de nada disso ainda, e as imagens naquele livro brilhante da Alemanha Oriental me deram a confiança de que eu precisava para abraçar totalmente meu papel como diplomata soviética nas Nações Unidas. Como era a década de oitenta, comprei um terno vermelho brilhante de cetim amassado com ombreiras enormes, passei sombra nos olhos e rolei meu cabelo cacheado com spray de cabelo a alturas precárias. De alguma forma, ajudou saber que havia sociedades que imaginavam, mesmo que apenas em um mundo idealizado, que as mulheres poderiam ser ambiciosas e bonitas. Eu poderia ter meus seios e poder de veto também.
 \par 
No final, embora a cultura patriarcal mude em um ritmo glacial, especialistas, desde políticos na União Europeia até consultores da KPMG, acreditam que medidas afirmativas devem ser tomadas para promover a liderança das mulheres. Não há uma solução única para todos, mas as cotas podem ser uma parte importante do processo. Os estados têm um papel na formação de atitudes sociais para aumentar a diversidade e a inclusão, e é essencial que usemos as ferramentas de uma legislação bem pensada para
 \par 
103
 \par 
104
 \par 
TERNOS NÃO SÃO SUFICIENTES: SOBRE LIDERANÇA crie mais oportunidades para mulheres se candidatarem a cargos eletivos ou servirem em conselhos executivos. Sim, as atitudes populares precisam mudar, mas essa mudança requer que meninas cresçam vendo mais mulheres em posições de poder. A única maneira de meninas verem mulheres em posições de poder é encontrar uma maneira de desafiar as culturas políticas e econômicas que impedem sua participação em primeiro lugar.
 \par 
