\chapter{De dentro para fora}\label{De dentro para fora}
 \par 
“A década de 1960 é corretamente lembrada como anos de dissidência cultural e convulsão política, mas é erroneamente lembrada como anos agitados apenas pela esquerda”, escreve George Will no prefácio de uma edição reeditada de The Conscience of a Conservative, de Barry Goldwater. {\color{blue}1} Há várias décadas, tal afirmação teria suscitado olhares perplexos, se não assobios e vaias. Mas nos anos que se seguiram, a publicação de uma série de livros, cada um deles defendendo a noção de que a maior parte da inovação política do último meio século veio da direita, levou os historiadores a rever a sabedoria convencional sobre a América do pós-guerra, incluindo o década de 1960. O novo consenso reflecte-se na frase de abertura de The Rise of Conservatism in America, 1945–2000, de Ronald Story e Bruce Laurie: “A história central da política americana desde a Segunda Guerra Mundial é a emergência do movimento conservador”. {\color{blue}2} No entanto, por alguma razão, ainda sentirá que seus parentes são insuficientemente apreciados e reconhecidos.
 \par 
Este capítulo apareceu originalmente como uma resenha de The Conscience of a Conservative, de Barry Goldwater (Princeton, N.J.: Princeton University Press, 2007, 1960); Limite à direita: Tornando a América conservadora na década de 1970, ed. Bruce J. Schulman e Julian E. Zelizer (Cambridge, Mass.: Harvard University Press, 2008); e Eles sabiam que estavam certos: a ascensão dos neoconservadores, de Jacob Heilbrunn (Nova York: Doubleday, 2008) em The Nation (23 de junho de 2008): 25–33.
 \par 
Will dificilmente é o primeiro conservador a acreditar que é um exilado em seu próprio país. Um sentimento de exclusão tem assombrado o movimento desde o início, quando os emigrantes fugiram da Revolução Francesa e Edmund Burke e Joseph de Maistre assumiram sua causa. Nascido na sombra da perda — de propriedade, posição, memória, herança, um lugar ao sol — o conservadorismo continua sendo uma reunião de fugitivos. Mesmo quando seguro de sua posição, o conservador faz o papel de vadio. Seja instrumental ou sincero, essa fusão de pária e poder é uma das fontes de seu apelo. Como William F. Buckley escreveu na declaração de fundação da National Review, o distintivo de exclusão do conservador o tornou "praticamente a coisa mais quente da cidade".{\color{blue}3}
 \par 
Embora David Hume e Adam Smith sejam frequentemente citados pelos mais gentis defensores do conservadorismo como os líderes do movimento, os seus escritos não conseguem explicar, como vimos, o que é verdadeiramente bizarro no conservadorismo: uma classe dominante que baseia a sua pretensão de poder nos seus sentimento de vitimização, provavelmente pela primeira vez na história. Os guardiões de Platão eram sábios; O rei de Tomás de Aquino era bom; O soberano de Hobbes era, bem, soberano. Mas a melhor defesa da monarquia que Maistre conseguiu reunir foi que o seu aspirante a rei frequentou a “terrível escola do infortúnio” e sofreu na “dura escola da adversidade”. {\color{blue}4} Maistre tinha boas razões para rejeitar esta defesa: bancar a plebe, sabemos agora, é uma arma crítica no arsenal conservador. Ainda assim, é uma defesa confusa. Afinal, se a principal vantagem que um príncipe traz para a mesa é que ele é realmente um indigente, por que não sentar o indigente?
 \par 
Os conservadores pediram-nos que não os obedecêssemos, mas que sentíssemos pena deles – ou que os obedecêssemos porque sentimos pena deles. Rousseau foi o primeiro a articular uma teoria política da piedade e, por isso, foi chamado de “Homero dos perdedores”. {\color{blue}5} Mas Burke, com seu relato exagerado de Maria Antonieta que vimos no capítulo {\color{blue}1} – “esta mulher perseguida”, arrastada “quase nua” por
 \par 
“As fúrias do inferno” do seu quarto em Versalhes e marcharam até “uma Bastilha para reis” em Paris – também têm algum direito ao título? {\color{blue}6}
 \par 
Maria Antonieta era um tipo particular de perdedora, uma pessoa com tudo que se encontrava totalmente e ao mesmo tempo despossuída. Burke viu na queda dela um arquétipo da tragédia clássica, a grande pessoa derrubada pela fortuna. Mas na tragédia, o que quase qualquer herói pode esperar é compreender o seu destino: a roda do tempo não pode ser revertida; o sofrimento não pode ser desfeito. Os conservadores, porém, não se contentam com a iluminação. Querem a restauração, uma oportunidade apresentada pelas novas forças da revolução e da contra-revolução. Identificando-se como vítimas, tornam-se os mais modernos e hábeis concorrentes num mercado político onde os direitos e a sua alienação são mercadorias valorizadas.
 \par 
Os reformadores e radicais devem convencer os subordinados e desprovidos de direitos de que têm direitos e poder. Os conservadores são diferentes. Eles estão ofendidos e têm direito – ofendidos porque têm direito – e já convencidos da justiça da sua causa e da inevitabilidade do seu triunfo. Assim, podem bancar a vítima e o vencedor com uma convicção e uma destreza que os subalternos só podem imaginar. Isso os torna formidáveis ​​reivindicadores de nossa lealdade e afeto. Quer sejamos ricos ou pobres, ou algo entre os dois, o conservador é, como Hugo Young disse sobre Maggie Thatcher, um de nós.{\color{blue}7}
 \par 
Mas como eles nos convencem de que somos um deles? Tornando o privilégio democrático e a democracia aristocrática. O conservador não defende o Antigo Regime; ele fala em nome dos antigos regimes – na família, na fábrica, no campo. Lá, homens comuns, e às vezes mulheres, desempenham o papel de pequenos senhores e senhoras, supervisionando seus subordinados como se todos pertencessem a uma propriedade feudal. Muito antes de Huey Long gritar: “Cada homem é um rei”, uma espécie mais ambígua de democrata pronunciou praticamente as mesmas palavras, embora com efeitos diferentes: a promessa da democracia é governar
 \par 
Outro ser humano tão completamente quanto um monarca governa seus súditos. A tarefa desse tipo de conservadorismo — feudalismo democrático — fica clara: cercar esses velhos regimes com cercas e portões, protegê-los de intrusos intrometidos como o estado ou um movimento social, enquanto discorre sobre mobilidade e inovação, liberdade e o futuro.
 \par 
Tornar o privilégio palatável para as massas é um projeto permanente do conservadorismo; mas cada geração deve adaptar esse projecto ao contorno da sua época. O desafio de Goldwater estava definido no título do seu livro: mostrar que os conservadores tinham consciência. Não um coração – ele criticou Eisenhower e Nixon por tentarem provar que os republicanos eram compassivos {\color{blue}8} – ou um cérebro, do qual os liberais, de John Stuart Mill a Lionel Trilling, duvidaram, mas uma consciência. Os movimentos políticos têm frequentemente de convencer os seus seguidores de que podem ter sucesso, que a sua causa é justa e que os seus líderes são experientes, mas raramente têm de provar que a sua marcha é uma marcha de luzes interiores. Goldwater pensava de outra forma: para atrair novos eleitores e reunir os fiéis, o conservadorismo tinha de estabelecer o seu idealismo e integridade, a sua independência absoluta da influência da riqueza, do privilégio e do materialismo – da própria realidade. Se quisessem mudar a realidade, os conservadores teriam de divorciar-se, pelo menos na sua auto-compreensão, da realidade. {\color{blue}9} (A este respeito, ele não era totalmente diferente de Burke, que advertiu que embora as classes dominantes na Grã-Bretanha tivessem “um vasto interesse em preservar” contra a ameaça jacobina e “grandes meios de preservá-la”, elas eram como um “artífice. Sobrecarregado com suas ferramentas”. Possuir vastos “recursos”, concluiu Burke, “pode estar entre os impedimentos” na luta contra a revolução.) {\color{blue}10} Nos últimos anos, tornou-se moda descartar o republicano de hoje como um verdadeiro crente. que traiu o conservadorismo ao abandonar o seu cepticismo nativo e o seu espírito de ajustamento moderado. Goldwater era independente e
 \par 
Teimoso, prossegue o argumento, recuando diante de qualquer coisa tão estupidificante (e soviética) como uma ideologia; Bush (ou o neoconservador ou Tea Partier) é rígido e doutrinário, um aplicador de linhas brilhantes e verdades do evangelho. Mas o conservadorismo sempre foi um movimento de credos – pelo menos para se opor aos credos da esquerda. “O outro lado tem uma ideologia”, declarou Thatcher. “Devemos ter um também.” {\color{blue}11} Para contrariar a esquerda, a direita teve de imitar a esquerda. “Por mais pequenos que sejam”, escreveu John C. Calhoun com admiração sobre os abolicionistas, eles “adquiriram muita influência pelo caminho que seguiram”.{\color{blue}12}
 \par 
Goldwater entendeu isso. Durante a Era Dourada, os conservadores opuseram-se aos sindicatos e à regulamentação governamental, invocando a liberdade dos trabalhadores de contratarem os seus empregadores. Os liberais responderam que esta liberdade era ilusória: os trabalhadores não tinham meios para contratar como desejassem; a verdadeira liberdade exigia meios materiais. Goldwater concordou, mas apresentou o mesmo argumento contra o New Deal: os impostos elevados roubaram os salários dos trabalhadores, tornando-os menos livres e menos capazes de ser livres. Canalizando John Dewey, ele perguntou: “Como pode um homem ser verdadeiramente livre se lhe forem negados os meios para exercer a liberdade?” {\color{blue}13} Franklin Delano Roosevelt afirmou que os conservadores se preocupavam mais com o dinheiro do que com os homens. Goldwater disse o mesmo sobre os liberais. Centrando-se no bem-estar e nos salários, “olham apenas para o lado material da natureza do homem” e “subordinam todas as outras considerações ao bem-estar material do homem”. Os conservadores, pelo contrário, consideram “o homem inteiro”, fazendo da sua “natureza espiritual” a “preocupação principal” da política e colocando “as coisas materiais no seu devido lugar”. {\color{blue}14} Este uivo romântico contra o economista do New Deal – semelhante ao da Nova Esquerda – não foi um protesto contra a política ou o governo; Goldwater não era libertário. Foi uma tentativa de elevar a política e o governo, de direcionar a discussão pública para fins mais nobres e gloriosos do que a gestão do conforto da criatura e do bem-estar material. Ao contrário da Nova Esquerda, no entanto,
 \par 
Goldwater não rejeitou a sociedade rica. Em vez disso, ele transformou a aquisição de riqueza num acto de autodefinição através do qual o homem “incomum” poderia distinguir-se da “massa indiferenciada”. {\color{blue}15} Acumular riqueza não era apenas exercer a liberdade através de meios materiais, mas também uma forma de dominar os outros.
 \par 
No seu ensaio sobre o pensamento conservador, Karl Mannheim argumentou que os conservadores nunca foram loucos pela ideia de liberdade. Ameaça a submissão do subordinado ao superior. Contudo, como a liberdade é a língua franca da política moderna, os conservadores tiveram “um instinto suficientemente sólido para não atacá-la”. Em vez disso, fizeram da liberdade o cavalo de caça da desigualdade e da desigualdade o cavalo de caça da submissão. Os homens são naturalmente desiguais, argumentam. A liberdade exige que lhes seja permitido desenvolver os seus dons desiguais. Uma sociedade livre deve ser uma sociedade desigual, composta por particulares radicalmente distintos e hierarquicamente organizados.{\color{blue}16}
 \par 
Goldwater nunca rejeitou a liberdade; na verdade, ele a celebrou. Mas há pouca dúvida de que ele a via como um proxy para a desigualdade — ou guerra, que ele chamava de “o preço da liberdade”. Uma sociedade livre protegia a “diferença absoluta de cada homem de todos os outros seres humanos”, com a diferença representando superioridade ou inferioridade. Era a “iniciativa e ambição de homens incomuns” — os mais diferentes e excelentes dos homens — que tornavam uma nação grande. Uma sociedade livre identificaria esses homens nos primeiros estágios da vida e lhes daria os recursos de que precisavam para ascender à preeminência. Contra aqueles que subscreviam “a noção igualitária de que todas as crianças devem ter a mesma educação”, Goldwater defendia “um sistema educacional que taxaria os talentos e estimularia as ambições de nossos melhores alunos e... Assim, nos asseguraria o tipo de líderes de que precisaremos no futuro”.{\color{blue}17}
 \par 
Mannheim também argumentou que os conservadores muitas vezes defendem o grupo – raças ou nações – em vez do indivíduo. Corridas e
 \par 
As nações têm identidades únicas, que devem, em nome da liberdade, ser preservadas. São os equivalentes modernos das propriedades feudais. Eles têm características e funções distintas e desiguais; eles desfrutam de privilégios diferentes e desiguais. A liberdade é a protecção desses privilégios, que são a expressão exterior do génio interior único do grupo.{\color{blue}18}
 \par 
Goldwater rejeitou o racismo (embora não o nacionalismo); mas por mais que tentasse, ao discutir a liberdade, não conseguiu resistir ao puxão do feudalismo. Ele chamou os direitos dos estados de “a pedra angular” da liberdade, “nosso principal baluarte contra a invasão da liberdade individual” pelo governo federal. Em teoria, os estados protegiam os indivíduos e não os grupos. Mas quem eram esses indivíduos em 1960? Goldwater afirmou que eles eram qualquer um e todos, que os direitos dos estados não tinham nada a ver com Jim Crow. No entanto, até ele foi forçado a admitir que a segregação “é, hoje, a expressão mais visível do princípio” dos direitos dos Estados. {\color{blue}19} A retórica dos direitos dos Estados criou um cordão em torno dos privilégios brancos. Embora seja certamente a plataforma mais nociva da plataforma conservadora – acabou por ser abandonada – o argumento de Goldwater a favor dos direitos dos Estados enquadra-se perfeitamente numa tradição que vê a liberdade como um escudo para a desigualdade e um substituto para o feudalismo em massa.
 \par 
Goldwater perdeu muito nas eleições presidenciais de 1964. Os seus filhos e netos ganharam muito – alargando o círculo de descontentamento para além dos brancos do Sul, incluindo maridos e mulheres, evangélicos e de etnia branca, e continuando a absorver e a transmutar as expressões idiomáticas da esquerda. {\color{blue}20} A adaptação à esquerda não tornou o conservadorismo americano menos reaccionário – tal como o reconhecimento de Maistre ou Burke de que a Revolução Francesa tinha mudado permanentemente a Europa não moderou o conservadorismo naquele país. Em vez disso, tornou o conservadorismo mais flexível e mais bem-sucedido. Quanto mais se adaptava, mais reacionário se tornava o conservadorismo.
 \par 
Os cristãos evangélicos eram recrutas ideais para a causa, jogando habilmente a carta da vítima como forma de rejuvenescer o poder dos brancos. “É hora de o povo de Deus sair do armário”, declarou um televangelista do Texas em 1980. Mas não foi a religião que tornou os evangélicos homossexuais; era religião combinada com racismo. Um dos principais catalisadores da direita cristã foi a defesa das escolas privadas do Sul que foram criadas em resposta à dessegregação. Em 1970, {\color{blue}400}.000 crianças brancas frequentavam estas “academias de segregação”. Estados como o Mississippi concederam bolsas de estudo aos estudantes e, até a administração Nixon anular a prática, o IRS concedeu aos doadores destas escolas isenções fiscais. {\color{blue}21} De acordo com Richard Viguerie, pioneiro da Nova Direita e do correio directo, o ataque a estes subsídios públicos por activistas dos direitos civis e pelos tribunais “foi a faísca que acendeu o envolvimento da direita religiosa na política real”. Embora a ascensão das academias de segregação “fosse muitas vezes sincronizada exatamente com a dessegregação das escolas públicas anteriormente exclusivamente brancas”, escreve um historiador, seus defensores afirmavam estar defendendo as minorias religiosas em vez da supremacia branca (inicialmente não-sectária, a maioria das escolas tornou-se evangélica ao longo do tempo). tempo). A sua causa era a liberdade, não a desigualdade – não a liberdade de associação com os brancos, como a geração anterior de resistentes massivos tinha reivindicado, mas a liberdade de praticar a sua própria religião combativa. {\color{blue}22} Foi uma transposição astuta. De uma só vez, os herdeiros dos proprietários de escravos tornaram-se descendentes dos batistas perseguidos, e Jim Crow tornou-se uma heresia que a Primeira Emenda deveria proteger.
 \par 
A direita cristã foi igualmente galvanizada pela reação contra o movimento feminista. O antifeminismo chegou tardiamente à causa conservadora. No início da década de 1970, os defensores da Emenda de Direitos Iguais (ERA) ainda podiam contar com Richard Nixon, George Wallace e Strom Thurmond como apoiadores; até Phyllis Schlafl y descreveu o EEI como algo “entre inócuo e
 \par 
Ligeiramente útil.” Mas assim que o feminismo entrou “na arena sensível e intensamente pessoal das relações entre os sexos”, escreve a historiadora Margaret Spruill, as frases abstractas de igualdade jurídica adquiriram um significado mais íntimo e concreto. O EEI provocou uma contra-revolução, como vimos no capítulo 1, liderada por Schlafly e outras mulheres, que era tão popular e quase tão diversa como o movimento a que se opunha. {\color{blue}23} Esta contra-revolução foi tão bem sucedida – não apenas em descarrilar o EEI, mas em impulsionar o Partido Republicano ao poder – que pareceu provar o ponto de vista feminista. Se as mulheres pudessem ser tão eficazes como agentes políticas, porque não deveriam estar no Congresso ou na Casa Branca?
 \par 
Schlafl compreendeu a ironia. Ela compreendeu que o movimento das mulheres tinha explorado e desencadeado um desejo de poder e autonomia entre as mulheres que não podia simplesmente ser reprimido. Se as mulheres fossem enviadas de volta ao exílio das suas casas, teriam de encarar a sua retirada não como uma derrota, mas como mais uma vitória na longa batalha pela liberdade e pelo poder das mulheres. Como vimos no capítulo 1, ela descreveu-se como defensora, e não como opositora, dos direitos das mulheres. A ERA era “uma lição dos direitos das mulheres”, insistiu ela, o “direito da esposa a ser apoiada e a ter os seus filhos menores apoiados” pelo seu marido. Ao concentrar o seu argumento no “direito da esposa num casamento contínuo, da esposa no lar”, Schlafl y reforçou a noção de que as mulheres eram primeiro esposas e mães; a sua única necessidade era a protecção proporcionada pelos seus maridos. Ao mesmo tempo, ela descreveu essa relação na linguagem liberal dos direitos de titularidade. “A esposa tem direito ao apoio” do seu cônjuge, afirmou ela, tratando a mulher como uma reivindicante feminista e o seu marido como o Estado de bem-estar social.{\color{blue}24}
 \par 
Tal como os seus antecessores católicos na França do século XVIII, a direita cristã apropriou-se não apenas das ideias, mas também dos costumes e costumes dos seus oponentes. Billy Graham lançou um álbum
 \par 
Chamada Sessão de Rap: Billy Graham e Estudantes Rap sobre Questões da Juventude de Hoje. Os evangélicos criticaram a cultura do narcisismo – e depois colonizaram-na. James Dobson, do Focus on the Family, começou como psicólogo infantil na Universidade do Sul da Califórnia, competindo com o Dr. Spock como autor de um texto best-seller sobre educação infantil. As livrarias evangélicas, segundo o historiador Paul Boyer, “promoviam livros terapêuticos e de autoajuda que trazem conselhos sobre finanças, namoro, casamento, depressão e vício a partir de uma perspectiva evangélica”. O mais audacioso de tudo foi a versão cinematográfica do livro de Hal Lindsey, The Late Great Planet Earth. Enquanto o livro popularizava as profecias cristãs sobre o fim dos dias, o filme foi narrado por Orson Welles, o bad boy original da Frente Popular.{\color{blue}25}
 \par 
Os casos mais interessantes de apropriação da esquerda pela direita, porém, vieram das grandes empresas e da administração Nixon. A classe empresarial via o movimento estudantil como um eleitorado crítico. Utilizando uma linguagem moderna e informal, escreve a historiadora Bethany Moreton, os porta-vozes das empresas deixaram “os seus fatos xadrez no armário” para venderem o capitalismo como a concretização da libertação, participação e autenticidade ao estilo dos anos sessenta. Recuperando-se dos protestos contra a invasão do Camboja (e do massacre de quatro estudantes que se seguiu), os estudantes do estado de Kent formaram um capítulo do Students in Free Enterprise (SIFE), um dos {\color{blue}150} em todo o país. Eles patrocinaram uma “Batalha de Bandas”, para a qual um concorrente escreveu a seguinte letra:
 \par 
Você sabe que eu nunca poderia ser feliz apenas trabalhando das nove às cinco. Prefiro passar minha vida pobre. Então vivendo isso como uma mentira. Se eu pudesse economizar meu dinheiro ou talvez conseguir um empréstimo,
 \par 
Eu poderia começar meu próprio negócio e fazê-lo sozinho.
 \par 
Institutos de pequenas empresas foram criados em campi universitários, classificando “o empresário como vítima, não como agressor”. As empresas também trouxeram suas táticas Gramscianas para as escolas secundárias. No Arkansas, a SIFE apresentou esquetes em sala de aula da série Free to Choose de Milton Friedman na PBS. Em 1971, o Arizona aprovou uma lei que exigia que os diplomados do ensino secundário fizessem um curso de economia para que tivessem “alguma base sobre a qual se apoiar”, de acordo com o patrocinador do projecto de lei, quando se opuserem “a professores que são colectivistas ou socialistas”. Vinte estados seguiram o exemplo. Os estudantes do Arizona poderiam ser excluídos do curso se fossem aprovados num exame que lhes pedisse, entre outras coisas, para combinar a frase “a intervenção do governo num sistema de livre iniciativa” com “é prejudicial ao mercado livre”.{\color{blue}26}
 \par 
O mais ambidestro dos políticos, Nixon era mestre em falar para a esquerda enquanto caminhava para a direita. Nixon compreendeu que a melhor resposta ao movimento pelos direitos civis não era defender os brancos contra os negros, mas transformar os brancos em etnias brancas sobrecarregadas com as suas próprias histórias de opressão e exigindo os seus próprios movimentos de libertação. Enquanto os imigrantes do Sul e do Leste da Europa saltaram para o caldeirão e se tornaram brancos, Nixon e os revivalistas étnicos da década de 1970 “forneceram aos americanos de ascendência europeia um novo veículo para fazer valer os direitos de cidadania num momento em que se tornou cada vez mais ilegítimo fazer reivindicações sobre o estado com base na branquitude”, escrevem o historiador Tom Sugrue e o sociólogo John Skrentny. Sob a liderança de Nixon, o Partido Republicano foi transformado numa versão de direita da máquina urbana Democrática. Polacos e italianos foram nomeados para cargos de destaque na sua administração, e Nixon fez campanha vigorosa em bairros étnicos brancos. Ele até disse a uma multidão que “ele sentia como se tivesse sangue italiano”. Os esforços de Nixon ocasionalmente
 \par 
Foram além do simbólico – uma proposta de 1971 teria alargado a acção afirmativa a “membros de certos grupos étnicos, principalmente de ascendência da Europa Oriental, Média e Meridional, como italianos, gregos e grupos eslavos” – mas a maioria era retórica. Isso não os tornou menos potentes: o novo vocabulário da etnia branca ajudou a criar “um passado romantizado de trabalho árduo, disciplina, papéis de género bem definidos e famílias unidas”, proporcionando uma nova linguagem para uma nova era – e um regime muito antigo.{\color{blue}27}
 \par 
A mãe de Barry Goldwater era descendente de Roger Williams. Seu pai, que se converteu ao episcopalismo, era descendente de judeus poloneses. Quando Goldwater concorreu em 1964, Harry Golden brincou: “Sempre soube que o primeiro judeu a concorrer à presidência seria um episcopal”. {\color{blue}28} Se a história do conservadorismo servir de guia, talvez ele devesse ter concorrido como judeu.