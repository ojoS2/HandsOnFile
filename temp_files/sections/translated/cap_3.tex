TTTTTT landó eu tinha vinte e poucos anos, uma amiga querida, a quem chamarei de Lisa, trabalhava em recursos humanos para uma grande corporação em São Francisco. Lisa amava moda, e meu guarda-roupa ainda inclui conjuntos elegantes que ela montava para mim em nossas frequentes excursões de compras de pechinchas para o Filene's Basement e vários brechós na Fillmore Street. Ela tinha um talento especial para escolher tesouros de grife com desconto e montar roupas que misturavam Levi's com Dior vintage. Ao longo dos anos, mantivemos contato, lamentando o casamento e a nova maternidade. Mas enquanto eu comecei minha vida como uma mãe trabalhadora na carreira, lisa largou o emprego para se tornar uma mãe dona de casa assim que percebeu que estava grávida. Seu marido ganhava o suficiente para sustentá-la, e ele preferia que ela não estivesse empregada. Sua própria mãe tinha ficado em casa, e entre seus amigos próximos, vizinhos e colegas, esse era o arranjo normal. Lisa alegou que essa era sua escolha; ela queria uma pausa da corrida dos ratos da América corporativa. Ela teve um segundo filho logo depois do primeiro e abandonou a ideia de retornar ao mercado de trabalho. Lisa achou que seria mais fácil assim; ela
 \par 

 \par 
25
 \par 
26
 \par 
MULHERES — COMO HOMENS, MAS MAIS BARATAS: NO TRABALHO estaria fisicamente presente para suas filhas de uma forma que eu nunca poderia estar para as minhas.
 \par 
Naqueles primeiros anos, enquanto ela assava biscoitos e organizava amiguinhas, eu deixava minha filha em uma creche de período integral, cinco dias por semana, o que me custou uma pequena fortuna. Enquanto suas meninas dormiam, lisa lia romances, se exercitava e cozinhava refeições luxuosas. Meus primeiros quatro anos de maternidade coincidiram com meus primeiros três anos na carreira acadêmica. Minha vida era uma rotina esmagadora de dias agitados. A primeira vez que dei aula com minha blusa do avesso, estremeci de vergonha quando um aluno simpático apontou para minhas costuras. Mas depois da terceira vez, parei de me importar. Contanto que minha saia não estivesse ao contrário, estava tudo bem. Muitas vezes invejei a escolha de Lisa, mas ganhei meu doutorado e consegui um bom emprego. Eu não queria desistir. Quando minha filha fez cinco anos, as coisas ficaram um pouco mais fáceis. Meu primeiro livro foi lançado, ganhei estabilidade e minha filha começou a primeira série. Livre das contas exorbitantes da creche, comecei a colher as recompensas psicológicas e financeiras da minha perseverança.
 \par 
Alguns anos depois, passei um fim de semana com Lisa. O marido dela se ofereceu para ficar com nossas três meninas para que ela e eu pudéssemos ir ao shopping: jantar, um filme e talvez umas comprinhas. Nossos compromissos sociais geralmente incluíam nossos filhos, então isso foi um verdadeiro deleite. Eu ansiava por algumas horas de conversa adulta com uma velha amiga e nenhuma demanda urgente por suco ou sorvete, ou birras inesperadas. Uma verdadeira noite das meninas.
 \par 
Eu estava lá em cima na casa dela me arrumando quando percebi que tinha esquecido meu secador de cabelo. Queria perguntar a Lisa se poderia pegar o dela emprestado, mas, quando comecei a descer as escadas, ouvi Lisa brigando com o marido.
 \par 
KRISTEN R. GHODSEE, Sim, Re Um Sim
 \par 
\chapter{MULHERES — COMO OS HOMENS, MAS MAIS BARATAS: NO TRABALHO}\label{MULHERES — COMO OS HOMENS, MAS MAIS BARATAS: NO TRABALHO}
 \par 
“... Por favor, Bill. Vai ser constrangedor.” “Não. Você gastou dinheiro suficiente este mês. Vou dar
 \par 
\section{Com né}
 \par 
“Mas eu fiz compras para a casa e comprei roupas para as meninas. Não comprei nada para mim.”
 \par 
“Você está sempre comprando coisas para si e dizendo que é
 \par 
\section{Você pega o cartão novamente depois que a declaração rolar.”}
 \par 
“Mas é para as meninas. Elas continuam crescendo.” “Você tem roupas suficientes. Não precisa de mais nada. Eu te dei o suficiente para o jantar e o filme.”
 \par 
“Bill, por favor.” A voz de Lisa falhou. Vir-me-ei para subir as escadas na ponta dos pés, rezando para que não tivessem me ouvido. Escondi-me no banheiro até Lisa subir, com o maxilar cerrado e os olhos rosados.
 \par 
Nós dirigimos até o restaurante em silêncio. Pedimos dois pratos, e eu tentei prolongar o jantar até pouco antes do filme começar. Lisa pareceu grata por ficar.
 \par 
Depois da nossa segunda taça de Malbec, ela disse: “Bill e eu brigamos”.
 \par 
Olhei para o meu prato. “Ele diz que não fazemos sexo com frequência suficiente.” Olhei para cima. Não foi essa a briga que pensei ter ouvido. Ela girou seu copo vazio. “Você acha que temos tempo
 \par 
\section{Para as meninas.”}
 \par 
“Vá à frente”, eu disse. “Eu dirijo.” Ela bebeu uma terceira taça de vinho, e conversamos sobre as críticas do filme que planejávamos ver. Quando a conta chegou, ela abriu a carteira e empurrou algumas notas de vinte dólares para mim na mesa. Coloquei meu cartão de crédito.
 \par 
Ela olhou para o American Express com meu nome escrito e suspirou. “Bill só me dá dinheiro.”
 \par 
27
 \par 
28
 \par 
MULHERES — COMO OS HOMENS, MAS MAIS BARATAS: NO TRABALHO
 \par 
“Por que você não me deixa pegar isso?” Eu deslizei o dinheiro de volta para ela. “Fique com ele.”
 \par 
Ela olhou para a mesa por um longo momento. Finalmente, ela disse, "Obrigada", e colocou as notas de volta na carteira. "Eu vou transar com ele hoje à noite e te pagar de volta amanhã."
 \par 
Fiquei ali, atordoada. Lisa olhou para o relógio. “Se nos apressarmos, posso passar no balcão da Shiseido antes do filme começar.”
 \par 
Sentado no restaurante naquela noite, jurei a mim mesmo que não importava o quão difícil fosse equilibrar meu trabalho de tempo integral com o cuidado da minha filha, eu nunca me colocaria na posição de Lisa se tivesse alguma escolha no assunto. “O capitalismo age sobre as mulheres como um suborno contínuo para entrar em relações sexuais por dinheiro, seja dentro ou fora do casamento; e contra esse suborno não há nada além da respeitabilidade tradicional que o capitalismo destrói pela pobreza”, escreveu George Bernard Shaw em 1928. Direta ou indiretamente, sexo e dinheiro estão sempre ligados na vida das mulheres, um resquício de nossa longa história de opressão.
 \par 
Muitas mulheres se encontram na situação de Lisa, economicamente dependentes de homens para seus meios de subsistência básicos. Leis de divórcio e ordens judiciais para pensão alimentícia e pensão alimentícia oferecerão a Lisa alguma proteção (possivelmente inadequada) se Bill tentar se divorciar dela, mas ela permanece à mercê dele, enquanto eles estiverem casados. Todo o trabalho que ela realiza cuidando de seus filhos, organizando suas vidas e administrando sua casa é invisível no que diz respeito ao mercado. Lisa não recebe salários e não contribui com fundos para sua própria previdência social na velhice. Ela não acumula trabalho
 \par 
KRISTEN R. GHODSEE experiência e cria um buraco negro em seu currículo, um que exigirá explicações se ela espera voltar a trabalhar. Ela até acessa cuidados médicos por meio do empregador de seu marido. Tudo o que ela tem, ela deriva da renda de Bill, e ele pode negar a ela o acesso aos seus cartões de crédito conjuntos à vontade.
 \par 

 \par 
No romance distópico arrepiante de Margaret Atwood, The Handmaid's Tale, os fundadores da República de Gilead legislam uma proibição geral do emprego de mulheres e a apreensão de suas economias pessoais. De repente, qualquer mulher designada é demitida de seu emprego, e o dinheiro em sua conta bancária é transferido para as contas de seu marido ou parente mais próximo, o primeiro passo para devolver as mulheres ao seu "lugar de direito". A subjugação das mulheres começa tornando-as economicamente dependentes dos homens mais uma vez. Sem dinheiro e sem meios para ganhá-lo, as mulheres são incapazes de determinar o curso de suas próprias vidas. A independência pessoal requer recursos para fazer suas próprias escolhas.
 \par 
Os mercados livres discriminam as trabalhadoras. No início da revolução industrial, os grandes patrões consideravam as mulheres inferiores aos seus colegas homens (mais fracas, mais emocionais, menos confiáveis ​​e assim por diante). A única maneira de convencer um empregador a contratar uma mulher era por meio de incentivos financeiros: as mulheres custam menos e são mais dóceis que os homens. Se ela exigisse um salário igual ao de um homem, o empregador simplesmente contrataria um homem. Portanto, a vantagem comparativa das mulheres no local de trabalho desde os primeiros dias do capitalismo é que elas farão o mesmo trabalho que um homem por menos dinheiro. A ideia do salário familiar agrava o problema. Quando as mulheres finalmente entraram no
 \par 
29
 \par 
30
 \par 
MULHERES — COMO HOMENS, MAS MAIS BARATAS: NO TRABALHO força de trabalho industrial em massa e começou a dominar indústrias leves (como costura, tecelagem, lavanderia), os empregadores pagavam às mulheres salários para uma única pessoa, não para uma família, mesmo que fossem mães solteiras ou viúvas. A sociedade insistia que as mulheres eram dependentes dos homens, e as mulheres trabalhadoras eram convenientemente imaginadas como esposas e filhas ganhando dinheiro de bolso para comprar toalhas de renda para suas penteadeiras. Maridos e pais deveriam suprir suas principais necessidades de comida, abrigo e roupas.
 \par 
Culturas patriarcais reduzem as mulheres à dependência econômica, tratando-as como uma forma de bem móvel a ser negociado entre famílias. Durante séculos, a doutrina da abertura tornou as mulheres casadas propriedade de seus maridos, sem direitos legais próprios. Todos os bens pessoais de uma mulher eram transferidos para seu marido após o casamento. Se seu homem quisesse vender seus rubis por rum, você não tinha o direito de recusar. Mulheres casadas da Alemanha Ocidental não podiam trabalhar fora de casa sem a permissão de seus maridos até 1957. Leis proibindo mulheres casadas de entrar em contratos sem a permissão de seus maridos persistiram nos Estados Unidos até a década de 1960. As mulheres na Suíça não ganharam o direito de votar ao nível federal até 1971.
 \par 
Sob o capitalismo, o industrialismo reforçou uma divisão do trabalho que concentrou os homens na esfera pública do emprego formal e tornou as mulheres responsáveis ​​pelo trabalho não remunerado na esfera privada. Em teoria, os salários masculinos eram suficientemente elevados para permitir aos homens sustentar as suas esposas e filhos. O trabalho doméstico gratuito das mulheres subsidiou os lucros dos empregadores porque as famílias dos trabalhadores suportaram o custo de reprodução da futura força de trabalho. Sem controlo de natalidade, acesso à educação ou oportunidades de emprego significativo,
 \par 
KRISTEN R. GHODSEE a mulher estava presa dentro dos limites da família em perpetuidade. “Sob o sistema capitalista, as mulheres se encontravam em pior situação do que os homens”, Bernard Shaw escreveu em 1928, “porque, como o capitalismo fez do homem um escravo, e então, pagando as mulheres por meio dele, fez dela sua escrava, ela se tornou escrava de um escravo, o pior tipo de escravidão”.
 \par 
Já em meados do século XIX, feministas e socialistas divergiam sobre a melhor forma de libertar as mulheres. Mulheres mais ricas defendiam os Atos de Propriedade das Mulheres Casadas e o direito de votar sem questionar o sistema econômico geral que perpetuava a subjugação das mulheres. Socialistas, como os teóricos alemães Clara Zetkin e August Bebel, acreditavam que a libertação das mulheres exigia sua incorporação total à força de trabalho em sociedades nas quais as classes trabalhadoras possuíam coletivamente as fábricas e a infraestrutura produtiva. Essa era uma meta muito mais audaciosa e talvez utópica, mas todos os experimentos subsequentes com o socialismo incluiriam a participação das mulheres na força de trabalho como parte de seu programa para remodelar a economia em uma base mais justa e equitativa.
 \par 
A percepção de que o trabalho de uma mulher é menos valioso do que o de um homem persiste até hoje. Em um sistema capitalista, a força de trabalho (ou as unidades de tempo que vendemos aos nossos empregadores) é uma mercadoria negociada no mercado livre. As leis da oferta e da demanda determinam seu preço, assim como o valor percebido desse trabalho. Os homens são mais bem pagos porque empregadores, clientes e consumidores percebem que eles valem mais. Pense nisso: por que os restaurantes baratos sempre têm garçonetes, mas
 \par 
31
 \par 
32
 \par 
MULHERES — GOSTAM DE HOMENS, MAS SÃO MAIS BARATAS: NO TRABALHO restaurantes caros costumam ter garçons homens? No conforto de nossas casas, a maioria de nós cresce sendo servida por mulheres: avós, mães, esposas, irmãs e, às vezes, filhas. Mas ser servida por homens é raro, assim como ter homens cuidando de nossas necessidades básicas. Pagamos um prêmio para que um homem nos sirva o jantar porque percebemos esse serviço como mais valioso, mesmo que tudo o que ele faça seja colocar um prato na sua frente e moer pimenta fresca no seu filé mágnon. Da mesma forma, embora as mulheres tenham alimentado a humanidade por milênios, os homens dominam o mundo culinário. Aparentemente, os clientes preferem um lado de testosterona com seu purê de batatas.* No passado, as mulheres entendiam que o público valorizava menos seu trabalho e tomavam medidas para mitigar os efeitos da discriminação. Charlotte Brontë publicou seus primeiros romances sob o pseudônimo Currer Bell, e Mary Anne Evans escreveu como George Eliot. Mais recentemente, tanto J. K. Rowling quanto E. L. James publicaram livros usando suas iniciais para ocultar seu gênero. No caso de Rowling, sua editora pediu que ela fizesse isso para atrair leitores meninos que poderiam rejeitar um livro escrito por uma mulher. No mundo do ensino universitário, ter um nome que soe feminino resulta em avaliações de ensino piores, já que os alunos classificam consistentemente os professores homens mais alto do que suas colegas mulheres. Um estudo experimental de 2015 descobriu que instrutores assistentes que ministravam a mesma aula online sob duas identidades de gênero diferentes recebiam avaliações mais baixas para sua persona.° feminina
 \par 
O racismo exacerba a discriminação de gênero. Mulheres hispânicas e negras sofrem uma diferença salarial maior do que as mulheres brancas. Quando falamos sobre discriminação de gênero, temos que ter cuidado para não privilegiar o gênero como a principal categoria de análise, como algumas feministas fizeram no
 \par 
KRISTEN R. GHODSEE passado. A condição de ser mulher é complicada por outras categorias, como classe, raça, etnia, orientação sexual, deficiência, crença religiosa e assim por diante. Sim, sou mulher, mas também sou porto-riquenha-persa, de origem imigrante e da classe trabalhadora (minha avó cursou a terceira série e minha mãe só concluiu o ensino médio). O antigo conceito de irmandade ignora os aspectos estruturais do capitalismo que beneficiam as mulheres brancas da classe média, ao mesmo tempo que desfavorecem as mulheres negras da classe trabalhadora, algo que as mulheres ativistas socialistas compreenderam já no final do século XIX. Nos círculos de esquerda, os marxistas ortodoxos obcecados com a posição de classe são frequentemente chamados de “socialistas”, porque enfatizam a solidariedade dos trabalhadores sobre questões de raça e gênero. Algumas feministas e socialistas argumentarão que o foco excessivo nas políticas de identidade divide as pessoas e mina a potencial base de poder dos movimentos de massas para a mudança social, mas ao examinarmos as estruturas de opressão, devemos estar atentos às hierarquias de subjugação, mesmo quando construímos coligações estratégicas.
 \par 
Adotar uma abordagem interseccional, por exemplo, nos ajuda a ver como os empregos no setor público criaram oportunidades importantes para diferentes populações. Enquanto os homens brancos da classe trabalhadora já dominaram os empregos na indústria privada, o emprego no governo forneceu importantes caminhos para afro-americanos que eram (e continuam sendo) mais propensos do que os brancos a trabalhar no setor público. O setor público historicamente ofereceu empregos para minorias religiosas, pessoas de cor e mulheres que enfrentavam discriminação no setor privado, criando oportunidades de carreira para aqueles desfavorecidos por raça ou gênero em mercados de trabalho competitivos. Cortes no emprego no setor público após a Grande Recessão atingiram
 \par 
33
 \par 
34
 \par 
MULHERES — COMO OS HOMENS, MAS MAIS BARATAS: NO TRABALHO
 \par 
As mulheres afro-americanas são particularmente duras, forçando-as a procurar trabalho em empresas privadas, onde as percepções do valor do seu trabalho são mais influenciadas pela cor da sua pele e pelo seu gênero.
 \par 
Um estudo clássico mostrando a profunda persistência do preconceito de gênero envolveu audições para orquestras sinfônicas. Mulheres musicistas eram terrivelmente sub-representadas em orquestras profissionais antes da introdução de um processo de audição em que os músicos tocavam seus instrumentos atrás de telas que os separavam dos jurados. Para garantir o anonimato total de gênero, os músicos tiravam os sapatos para que os passos de homens e mulheres fossem indistinguíveis para aqueles que tomavam decisões. Quando aqueles que faziam as audições julgavam os músicos apenas por sua capacidade de tocar, "a porcentagem de mulheres musicistas nas cinco orquestras mais bem classificadas do país aumentou de 6% em 1970 para 21% em 1993". Esse sistema de audição de tela também eliminaria preconceitos raciais.
 \par 
Mas não podemos nos esconder atrás de telas em todas as nossas entrevistas de emprego e interações com potenciais empregadores. Nossos nomes nos denunciam e, mesmo que consigamos esconder nosso gênero por trás de iniciais ou pseudônimos masculinos, as referências usam pronomes e outras palavras que revelam nosso gênero. Provar discriminação é difícil e há poucas repercussões para aqueles que sistematicamente pagam às mulheres menos do que aos homens pelo mesmo trabalho. Além disso, como as mulheres ganham menos do que os homens, faz sentido econômico para as mães ficar em casa com crianças pequenas quando o cuidado infantil acessível é escasso. Quando as mulheres entram na força de trabalho como trabalhadoras de meio período ou flexíveis, elas geralmente o fazem sem benefícios e sem salários suficientes para cobrir suas necessidades básicas. E como
 \par 
KRISTEN R. GHODSEE as mulheres se retiram da força de trabalho para cuidar dos jovens, dos doentes ou dos parentes idosos, a discriminação contra as trabalhadoras se torna mais arraigada, uma vez que os empregadores as veem como menos confiáveis ​​(mais sobre isso no próximo capítulo), e o ciclo de dependência econômica das mulheres em relação aos homens continua.
 \par 
Para combater os efeitos da discriminação e da disparidade salarial, os países socialistas elaboraram políticas para encorajar ou exigir a participação formal das mulheres na força de trabalho. Em maior ou menor grau, todos os países socialistas estatais na Europa Oriental exigiam a incorporação total das mulheres ao emprego remunerado. Na União Soviética e particularmente na Europa Oriental após a Segunda Guerra Mundial, a escassez de mão de obra impulsionou essa política. As mulheres sempre foram usadas como um exército de reserva de trabalho quando os homens estavam em guerra (assim como o emprego das mulheres americanas durante a era da Segunda Guerra Mundial de Rosie, a Rebitadeira). Mas, diferentemente dos Estados Unidos e da Alemanha Ocidental, onde as mulheres eram "dispensadas" depois que os soldados voltavam para casa, os estados do Leste Europeu garantiam o pleno emprego das mulheres e investiam vastos recursos em sua educação e treinamento. Essas nações promoviam o trabalho feminino em profissões tradicionalmente masculinas, como mineração e serviço militar, e produziam em massa imagens de mulheres dirigindo máquinas pesadas, especialmente tratores.*
 \par 
Por exemplo, enquanto as mulheres americanas estavam abastecendo suas cozinhas com os últimos eletrodomésticos durante o boom econômico do pós-guerra, o governo búlgaro encorajou as meninas a seguir carreiras na nova economia. Em 1954, o estado produziu um curta-metragem documentário para celebrar as vidas de mulheres que ajudaram a transformar a Bulgária agrícola em uma potência industrial moderna. Este filme, J Am a Woman
 \par 
35
 \par 
36
 \par 
MULHERES — COMO OS HOMENS, MAS MAIS BARATAS: NO TRABALHO
 \par 
Estatísticas oficiais da Organização Internacional do Trabalho (OIT) demonstram a disparidade entre as taxas de participação da força de trabalho em economias socialistas estatais e aquelas em economias de mercado. Em 1950, a parcela feminina da força de trabalho soviética total era de 51,8%, e a parcela feminina da força de trabalho total na Europa Oriental era de 40,9%, em comparação com 28,3% na América do Norte e 29,6% na Europa Ocidental. Em 1975, o Ano Internacional das Mulheres das Nações Unidas, as mulheres representavam 49,7% da força de trabalho da União Soviética e 43,7% daquela no Bloco Oriental, em comparação com 37,4% na América do Norte e 32,7% na Europa Ocidental. Essas descobertas levaram a OIT a concluir que a “análise de dados sobre a participação das mulheres na atividade econômica na URSS e nos países socialistas
 \par 
KRISTEN R. GHODSEE da Europa mostra que homens e mulheres nesses países desfrutam de direitos iguais em todas as áreas da vida econômica, política e social. O exercício desses direitos é garantido ao conceder às mulheres oportunidades iguais às dos homens no acesso à educação e à formação profissional e no trabalho.”®
 \par 
Claro, os próprios relatos das mulheres complicam o quadro otimista pintado pela OIT em 1985. As disparidades salariais de gênero ainda existiam nos países do Leste Europeu. E apesar das tentativas de canalizar as mulheres para empregos tradicionalmente masculinos, permaneceu uma divisão de trabalho de gênero em que as mulheres trabalhavam em profissões de colarinho branco, no setor de serviços e na indústria leve, em comparação com os homens que trabalhavam nos setores mais bem pagos da indústria pesada, mineração e construção. Mas os salários importavam menos quando havia pouco para comprar com os salários e onde o próprio emprego formal garantia serviços sociais do estado. Em muitos países, as mulheres não tinham escolha; elas eram forçadas a trabalhar quando seus filhos tinham idade suficiente para ir ao jardim de infância. E as mulheres em países socialistas de estado sofriam o duplo fardo do trabalho doméstico e do emprego formal (um problema muito familiar para muitas mulheres trabalhadoras hoje). A escassez de consumidores assolava a economia, e tanto homens quanto mulheres esperavam em filas para adquirir bens básicos. Mas, como trabalhadoras, as mulheres contribuíam para suas próprias pensões e desenvolviam seus próprios conjuntos de habilidades. Elas se beneficiavam de assistência médica gratuita, educação pública e uma generosa rede de segurança social que subsidiava abrigo, serviços públicos, transporte público e alimentos básicos. Em alguns países, as mulheres podiam se aposentar do emprego formal até cinco anos antes dos homens.
 \par 
Apesar das deficiências da economia de comando, o sistema socialista também promoveu uma cultura na qual
 \par 
37
 \par 
38
 \par 
MULHERES — COMO HOMENS, MAS MAIS BARATAS: NO TRABALHO a participação formal das mulheres na força de trabalho era aceita e até celebrada. Antes da Segunda Guerra Mundial, os países do Bloco Oriental eram sociedades profundamente patriarcais e camponesas, com relações de gênero conservadoras emergindo tanto da religião quanto da cultura tradicional. Ideologias socialistas desafiaram séculos de subjugação das mulheres. Como o estado exigia educação das meninas e obrigava as mulheres a entrarem na força de trabalho, seus pais e maridos não podiam forçá-las a ficar em casa. As mulheres aproveitaram essas oportunidades de educação e emprego. Quando as taxas de natalidade começaram a cair no final da década de 1960, muitos líderes do Partido Comunista se preocuparam que seus investimentos em mulheres prejudicariam suas economias a longo prazo. Eles conduziram pesquisas sociológicas e descobriram que as mulheres lutavam realmente com suas responsabilidades duplas como trabalhadoras e mães. Alguns governos consideraram permitir que as mulheres voltassem para casa, mas quando perguntadas se seriam mais felizes se seus maridos ganhassem o suficiente para sustentar a família, a maioria das mulheres rejeitou o modelo tradicional de ganha-pão/dona de casa. Elas queriam trabalhar. Na novela de Natalya Baranskaya sobre uma mãe trabalhadora soviética atribulada, a protagonista nunca fantasia sobre deixar seu emprego, afirmando inequivocamente que ama seu trabalho."°
 \par 
Refletindo sobre as conquistas do socialismo de Estado em comparação com a situação das mulheres na maioria dos países da Europa Oriental antes da Segunda Guerra Mundial, a socióloga húngara Zsuzsa Ferge explicou: “Em suma. . A situação objectiva das mulheres provavelmente melhorou em todo o lado em comparação com a situação anterior à guerra. O seu trabalho remunerado fora de casa contribuiu para o bem-estar da família (pelo menos ajudou a fazer face às despesas); sua educação
 \par 

 \par 
O avanço de KRISTEN R. GHODSEE e o trabalho fora de casa enriqueceram (pelo menos geralmente) a sua experiência de vida; o seu estatuto de assalariados enfraqueceu a sua antiga opressão dentro e fora da família e tornou-os (um pouco) menos subservientes em algumas esferas da vida. Além disso, atenuou a pobreza feminina, especialmente no caso das mães que praticamente todas começaram a trabalhar, e das mulheres mais velhas que obtiveram uma pensão por direito próprio.” Os países socialistas de Estado podiam promover a autonomia econômica das mulheres porque o Estado era o principal empregador e garantia a cada homem e mulher o pleno emprego como um direito e um dever de cidadania. Nos países socialistas democráticos do Norte da Europa, o emprego das mulheres é voluntário, mas o Estado promove a sua participação na força de trabalho, fornecendo os serviços sociais necessários para ajudar os cidadãos a combinar os seus papéis como trabalhadores e pais.”
 \par 
Os estados socialistas também tentam combater a discriminação persistente contra as mulheres expandindo as oportunidades de emprego no setor público. Embora não sejam tão atraentes quanto as startups, os governos podem garantir que as mulheres recebam salários iguais (decentes) por trabalho igual e apoiar as mulheres em suas responsabilidades profissionais e familiares. De acordo com um relatório da Organização para Cooperação e Desenvolvimento Econômico (OCDE), os países escandinavos lideram o mundo não apenas em termos de igualdade de gênero, mas também em termos de emprego no setor público. Isso não é coincidência. Em 2015, 30% do emprego total na Noruega era emprego governamental, seguido por 29,1% na Dinamarca, 28,6% na Suécia e 24,9% na Finlândia. O Reino Unido,
 \par 
39
 \par 
40
 \par 
MULHERES — COMO HOMENS, MAS MAIS BARATAS: A ON WORK, por outro lado, empregava apenas 16,4% de sua população total empregada no setor público, e nos Estados Unidos esse número era de 15,3%. Ainda mais notável é que as mulheres representam cerca de 70% de todos os funcionários públicos na Noruega, Dinamarca, Suécia e Finlândia, e a média da OCDE é de 58%. Os autores do relatório explicam a super-representação das mulheres no setor público em parte porque professores e enfermeiros são dominados por mulheres
 \par 
Profissões, mas também devido a “condições de trabalho mais flexíveis no setor público do que no privado. Por exemplo, em dezesseis países da OCDE, o setor público oferece mais arranjos de cuidados infantis e familiares do que o setor privado.” Finalmente, estudos revelam menores diferenças salariais entre homens e mulheres no setor público.”
 \par 
As taxas de emprego no setor público costumavam ser mais altas nos Estados Unidos até que as agências federais começaram a terceirizar, subcontratar para o setor privado ou simplesmente cortar empregos. Um relatório de 2013 analisando as tendências de emprego nos EUA mostrou um declínio vertiginoso no emprego no setor público após a
 \par 
Grande Recessão, à medida que estados e localidades cortavam orçamentos após a crise. O Projeto Hamilton examinou as respostas do governo a recessões anteriores e descobriu que cortar empregos de professores, socorristas e controladores de tráfego aéreo durante um período de alto desemprego retardou a recuperação e infligiu maior sofrimento econômico aos cidadãos americanos, particularmente à geração mais jovem, que estava lotada em salas de aula maiores com menos educadores. “A recuperação em andamento, que começou quando a Grande Recessão terminou em junho de 2009, desvia-se drasticamente do padrão usual”, escrevem os pesquisadores do projeto. “Nos quarenta e seis meses seguintes ao fim das outras cinco recessões recentes,
 \par 
KRISTEN R. GHODSEE o emprego no governo aumentou em média 1,7 milhões. Durante a recuperação atual, no entanto, o emprego no governo diminuiu em mais de {\color{blue}500}. {\color{blue} 000 } {\par} , e um número desproporcional daqueles que perderam empregos eram mulheres. Juntas, as diferenças políticas levaram a 2,2 milhões de empregos a menos hoje. Uma contração tão grande do setor público durante uma recuperação não tem precedentes na história econômica americana recente.”
 \par 
As atitudes em relação ao emprego no setor público refletem divisões ideológicas sobre se o governo é mais ou menos eficiente do que o mercado. Nossos bancos são privados porque os americanos acreditam que os bancos estatais (como um banco postal) seriam mais burocráticos e menos amigáveis ​​ao consumidor do que aqueles forçados a competir em mercados livres pelo dinheiro dos depositantes (mesmo que o governo federal forneça seguro de depósito de até US$ {\color{blue}250}.000 e resgate bancos considerados "grandes demais para falir"). Da mesma forma, os Estados Unidos rejeitam um sistema nacional de saúde porque nosso seguro de saúde privado supostamente fornece melhor atendimento a preços mais baixos como resultado da competição de mercado. Embora inúmeros estudos mostrem que os americanos pagam mais dinheiro por assistência médica, os americanos se apegam à ideia de que os mercados produzem melhores resultados do que os programas estatais, mesmo quando são apresentadas evidências abundantes do contrário. Outro exemplo está no ensino superior, com a expansão de universidades com fins lucrativos. Um estudo de 2016 mostra que os empregadores não valorizam os diplomas universitários com fins lucrativos tanto quanto valorizam os diplomas de universidades públicas. No entanto, os fundos governamentais fornecem ajuda financeira substancial aos estudantes dessas universidades, subsidiando assim os lucros dos investidores quando esses fundos poderiam ser usados ​​para fortalecer a qualidade da educação pública.
 \par 
4 litros
 \par 
42
 \par 
Cidadãos de outras sociedades, até mesmo nossos aliados próximos no Canadá e no Reino Unido, entendem que o motivo do lucro às vezes prejudica o bem público.”
 \par 
Claro, alguns podem argumentar que, em vez de expandir o emprego público, o governo poderia legislar sobre igualdade salarial e impor disposições para garantir que as empresas do setor privado paguem às mulheres de forma justa, uma medida que o governo islandês tomou no início de 2018 e o estado de Massachusetts tomou após 1º de julho de 2018. Mas a legislação federal sobre igualdade salarial nos Estados Unidos tem sido relativamente fraca e sem dentes reais, uma vez que o ônus permanece nas mulheres para provar discriminação salarial no tribunal (e quem tem o dinheiro necessário para um processo?). Tentativas de fortalecer o Equal Pay Act de 1963 não conseguiram obter apoio republicano no Congresso, mais recentemente em abril de 2017 com o Paycheck Fairness Act, que não recebeu um único voto republicano.
 \par 
Os críticos também alegarão que o emprego expandido no setor público prejudica o crescimento e paralisa o setor privado, mas a expansão de empregos no setor privado não foi capaz de reverter a estagnação salarial, a ascensão da economia gig ou o incrível crescimento da desigualdade entre ricos e pobres, conforme revelado por Thomas Piketty. Economistas e legisladores terão que debater os detalhes, mas dado que em 2017, apenas oito homens possuíam a mesma quantidade de riqueza que os 3,6 bilhões de pessoas que compõem a metade mais pobre da humanidade, a redistribuição virá garantidamente. Os níveis atuais de desigualdade são insustentáveis ​​a longo prazo. Em uma economia global impulsionada pelos gastos do consumidor alimentados por crédito das massas, a bolha acabará estourando. Uma crise aguda de superprodução e subconsumo surge no horizonte.'°
 \par 
'
 \par 
KRISTEN R. GHODSEE
 \par 
A expansão dos serviços públicos apoiaria as mulheres de uma segunda maneira. Uma rede de segurança social mais ampla significa que os salários mais baixos das mulheres no setor privado não as prejudicam em termos de acesso à assistência médica, água limpa, creche, educação ou segurança na velhice. Em vez de tentar legislar sobre igualdade ou coagir empresas privadas a fornecer salários iguais para trabalho igual e dar às mulheres oportunidades iguais de promoção, as mulheres poderiam se unir para escolher líderes que diminuirão os custos sociais da discriminação de gênero por meio de políticas públicas. Outra ideia é alguma forma de emprego garantido, como o que tinham nos países socialistas estatais. Este é um antigo conceito econômico para evitar o sofrimento humano causado durante as crises econômicas. O Partido Trabalhista do Reino Unido propôs uma garantia de emprego na qual o estado atua como o empregador de último recurso para jovens de dezoito a vinte e cinco anos que estão dispostos a trabalhar, mas não conseguem encontrar emprego. Economistas debatem garantias de emprego há décadas e, em 2017, o Center for American Progress (CAP) apoiou uma proposta para um novo "Plano Marshall para a América", que criaria 4,4 milhões de novos empregos. A proposta do CAP pede um “programa permanente de emprego público e investimento em infraestrutura em larga escala — semelhante ao Works Progress Administration (WPA) durante a Grande Depressão, mas modernizado para o século XXI. Ele aumentará o emprego e os salários para aqueles sem diploma universitário, ao mesmo tempo, em que fornece serviços necessários que estão atualmente fora do alcance de famílias de baixa renda e governos estaduais e locais com dificuldades financeiras.”
 \par 
Em setembro de 2017, assisti à missa com minha avó de oitenta e nove anos na igreja de San Diego.
 \par 
43 foi quando criança. Naquele domingo, o padre introduziu a parábola dos trabalhadores na vinha (Mateus 20:1-16) explicando que, para os americanos, era uma das parábolas mais controversas. Na história de Jesus, um proprietário de terras vai à cidade contratar trabalhadores diaristas por um salário justo pela manhã. Ele então retorna para contratar mais homens ao meio-dia e mais tarde a tarde. Perto do pôr do sol, o proprietário de terras retorna e encontra mais homens ociosos. Ele pergunta por que eles não estão trabalhando, e eles explicam que ninguém os contratou naquele dia. O proprietário de terras os contrata e então passa a pagar a todos os trabalhadores o mesmo salário, não importa quanto tempo eles trabalharam. Quando os trabalhadores contratados no início da manhã reclamam da injustiça, o proprietário de terras os repreende: eu ofereci a vocês um salário justo, e vocês aceitaram. O proprietário de terras diz: Não estou sendo injusto com vocês. Ou vocês estão com inveja porque sou generoso? Embora as parábolas sejam tipicamente interpretadas alegoricamente, naquele dia o padre usou a história para falar sobre salários justos e imigração em sua homilia. “O proprietário de terras foi à cidade e contratou os homens que precisavam de trabalho”, ele nos contou. “Ele não pediu para ver os documentos deles.” Da mesma forma, talvez, a parábola também apoie a ideia de garantias de emprego. O proprietário de terras forneceu emprego a todos aqueles dispostos e capazes de trabalhar, e ele pagou a eles um salário justo, não importando quanto tempo eles realmente trabalharam no vinhedo. Da perspectiva do proprietário de terras, foi uma coisa generosa a se fazer pelas pessoas necessitadas. Para os americanos, tal generosidade soa suspeitamento socialista.'”
 \par 
Mas sejamos realistas: garantias de emprego não beneficiariam apenas as mulheres. A longo prazo, se robôs de propriedade privada e A.I. tomarem conta da nossa economia, os homens orgânicos podem se ver tão desvalorizados em mercados de trabalho competitivos quanto
 \par 
KRISTEN R. GHODSEE as mulheres orgânicas de hoje. Os donos da vida inorgânica podem ser os verdadeiros beneficiários dos nossos futuros mercados livres não regulamentados. Os medos da crescente automação levaram alguns a promover a ideia da Renda Básica Universal (RBU), às vezes chamada de Renda Cidadã Universal ou Dividendo Cidadão. Isso garantiria que todos os cidadãos qualificados recebessem um pagamento mensal fixo para atender às suas necessidades básicas. Um experimento generoso de RBU foi tentado na Finlândia, e muitas pessoas em todo o espectro político apoia a ideia de algum tipo de pagamento fixo para salvar as pessoas da devastação do desemprego. Essa receita poderia ser gerada pela tributação do setor privado ou pelos lucros de empresas públicas. A RBU poderia contribuir muito para promover a igualdade de gênero, já que o trabalho não remunerado das mulheres em casa seria compensado. Claro, alguns críticos temem que a RBU torne as pessoas preguiçosas, enquanto outros se preocupam que seja apenas uma maneira dos hiper-ricos destruírem o estado de bem-estar social e comprarem as massas com pequenos pagamentos em dinheiro enquanto elas se deleitam com suas riquezas. É uma ideia que requer muito mais debate, particularmente de uma perspectiva socialista."
 \par 
Aconteça o que acontecer, qualquer movimento em direção a garantias de emprego exigirá uma expansão substancial do setor público, que terá o benefício adicional de promover a igualdade de gênero ao eliminar a diferença salarial entre homens e mulheres. A ironia aqui é que enquanto os regimes socialistas de estado reduziram a dependência econômica das mulheres em relação aos homens ao tornar homens e mulheres igualmente dependentes do estado, em uma sociedade capitalista nosso futuro tecnológico pode reduzir a dependência econômica das mulheres em relação aos homens ao tornar homens e mulheres igualmente dependentes da generosidade daqueles que possuem nossos senhores robôs. Algum dia, em breve, Bill pode ser
 \par 
45
 \par 
46
 \par 
MULHERES — COMO HOMENS, MAS MAIS BARATAS: NO TRABALHO implorando para um computador dar a ele sua mesada de sexta à noite para que ele possa ir ao bar esportivo com seus amigos. Haverá alguma justiça cósmica quando Siri informar Bill que ele já assistiu esportes o suficiente este mês, e deveria ficar em casa e passar um tempo de qualidade com sua esposa e filhas.
 \par 
