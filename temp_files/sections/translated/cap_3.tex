\chapter{O Primeiro Contrarrevolucionário}\label{O Primeiro Contrarrevolucionário}
 \par 
A revolução enviou Thomas Hobbes para o exílio; a contrarrevolução o mandou de volta. Em 1640, os opositores parlamentares de Carlos I, como John Pym, denunciavam qualquer pessoa que “pregasse a favor da monarquia absoluta, para que o rei pudesse fazer o que desejasse”. Hobbes havia terminado recentemente de escrever The Elements of Law, que fez exatamente isso. Depois que o principal conselheiro do rei e um teólogo que defendia o poder real ilimitado foram presos, Hobbes decidiu que era hora de partir. Sem esperar que suas malas fossem feitas, ele fugiu da Inglaterra para a França.{\color{blue}1}
 \par 
Onze anos e uma guerra civil depois, Hobbes fugiu da França para a Inglaterra. Desta vez, ele estava fugindo dos monarquistas. Como antes, Hobbes acabara de terminar um livro. O Leviatã, explicaria mais tarde, “luta em nome de todos os reis e de todos aqueles que, sob qualquer nome, detêm os direitos dos reis”. {\color{blue}2} Era a segunda metade dessa afirmação, com a sua aparente indiferença relativamente à identidade do soberano, que agora o colocava em apuros. O Leviatã justificou, não, exigiu, que os homens se submetessem a qualquer pessoa ou pessoas capazes de protegê-los de ataques estrangeiros e agitação civil. Com a monarquia abolida e as forças de Oliver Cromwell no controle de
 \par 
Inglaterra e provendo a segurança do povo, Leviathan parecia sugerir que todos, incluindo os monarquistas derrotados, professassem sua lealdade à Comunidade. Versões desse argumento já haviam feito Anthony Ascham, embaixador da Comunidade, ser assassinado por exilados monarquistas na Espanha. Então, quando Hobbes soube que clérigos na França estavam tentando prendê-lo — Leviathan também era veementemente anticatólico, o que ofendeu a Rainha Mãe — ele saiu de Paris e voltou para Londres.{\color{blue}3}
 \par 
Não foi por acaso que Hobbes fugiu dos seus inimigos e depois dos seus amigos, pois estava a elaborar uma teoria política que destruiu alianças de longa data. Em vez de rejeitar o argumento revolucionário, ele absorveu-o e transformou-o. A partir de suas categorias e expressões mais profundas, ele derivou uma defesa intransigente da forma de governo mais rígida. Ele sentiu as forças centrífugas da Europa moderna – o sacerdócio de todos os crentes; os exércitos democráticos reunidos sob a bandeira dos antigos ideais republicanos; ciência e ceticismo – e procurou canalizá-los para um único centro: um soberano tão terrível e benigno que faria qualquer desafio a tal autoridade parecer imoral e irracional. Não muito diferente dos futuristas italianos, Hobbes colocou a dissolução ao serviço da resolução. Ele foi o primeiro e, junto com Nietzsche, o maior filósofo da contra-revolução, um blender savant la letter do modernismo cultural e da reação política que entendeu que para derrotar uma revolução, você deve se tornar a revolução.
 \par 
E como ele tem sido tratado pela direita? Nada bem. T. S. Eliot (ele próprio um hábil liquidificador) chamou Hobbes de “um daqueles pequenos arrivistas extraordinários que os movimentos caóticos da Renascença lançaram em uma eminência que dificilmente mereciam”. {\color{blue}4} Dos quatro teóricos políticos do século XX identificados por Perry Anderson como “A Direita Intransigente” {\color{blue}5} —Leo Strauss, Carl Schmitt, Michael Oakeshott e Friedrich Hayek — apenas Oakeshott viu em Hobbes um
 \par 
Espírito semelhante. {\color{blue}6} O resto o via como a fonte de um liberalismo maligno, jacobinismo ou mesmo bolchevismo.{\color{blue}7}
 \par 
Os guardiões ortodoxos do antigo regime muitas vezes confundem o contra-revolucionário com a oposição porque não conseguem acompanhar a alquimia do seu argumento. Tudo o que sentem é o que existe – uma nova forma de pensar que soa perigosamente como a do revolucionário – e o que não existe: a justificação tradicional da autoridade. Para os ortodoxos, o contra-revolucionário parece um revolucionário. Isso faz do contra-revolucionário um suspeito, aos seus olhos, e não um camarada. Nisso eles não estão totalmente errados. Nem de esquerda nem, convencionalmente, de direita – um dos textos mais famosos de Hayek chama-se “Porque não sou conservador” {\color{blue}8} – o contra-revolucionário é um pastiche de incongruências, altas e baixas, antigas e novas, ironia e fé. A contra-revolução tenta nada menos do que quadrar o círculo, tornando as prerrogativas populares e refazendo um regime que afirma nunca ter sido feito (o antigo regime foi, é e será; não é feito). Estas são tarefas que nenhum outro movimento político deve realizar. O contra-revolucionário não está disposto ao paradoxo; ele é simplesmente forçado a enfrentar contradições históricas, em nome do poder.
 \par 
Mas por que levar Hobbes perante o tribunal do conservadorismo, da direita e da contrarrevolução? Afinal, nenhum desses termos entrou em circulação até a Revolução Francesa ou depois, e a maioria dos historiadores não acredita mais que a Guerra Civil Inglesa foi uma revolução. As forças que derrubaram a monarquia podem ter procurado a República Romana ou a antiga constituição. Elas podem ter desejado uma reforma dos costumes religiosos ou limitações ao poder real. Mas uma revolução não estava em seus planos. Como Hobbes poderia ter sido um contrarrevolucionário se não houvesse uma revolução para ele se opor?
 \par 
Hobbes, por exemplo, pensava o contrário. Em Behemoth, seu tratamento mais ponderado da questão, ele declarou firmemente que a Civilização Inglesa
 \par 
Guerra, uma revolução. {\color{blue}9} E embora ele quisesse dizer com esse termo algo semelhante ao que os antigos queriam dizer – um processo cíclico de mudança de regime, mais parecido com a órbita dos planetas do que um grande salto em frente – Hobbes viu na derrubada da monarquia um processo zeloso (e, para dizer a verdade, sua mente, tóxica) anseio pela democracia, um desejo firme de redistribuir o poder a um maior número de homens. Essa era, para Hobbes, a essência do desafio revolucionário; e assim tem sido desde então – seja na Rússia em 1917, em Flint em 1937, ou em Selma em 1965. O facto de esta expansão democrática ter sido inspirada por visões do passado e não do futuro não precisa de nos deter mais do que fez com Hobbes – ou Benjamin Constant ou Karl Marx, aliás, ambos viram como era fácil para os franceses fazerem a sua revolução olhando (ou mesmo olhando) para trás.{\color{blue}10}
 \par 
Hobbes opôs-se claramente ao “democraticamente”, como chamou as forças parlamentares e os seus seguidores. {\color{blue}11} Uma soma considerável de sua energia filosófica foi gasta nesta oposição, e suas maiores inovações derivaram dela. {\color{blue}12} O seu alvo específico era a concepção de liberdade dos republicanos, a sua noção de que a liberdade individual implicava que os homens governassem colectivamente a si próprios. Hobbes desfez os laços republicanos entre a liberdade pessoal e a posse do poder político. Assim, ele foi capaz de argumentar que os homens poderiam ser livres numa monarquia absoluta – ou pelo menos não menos livres do que eram numa república ou numa democracia. Foi “um momento que marcou época na história do pensamento político anglófono”, diz Quentin Skinner. O resultado foi um novo relato da liberdade, ao qual continuamos em dívida até hoje.{\color{blue}13}
 \par 
Todo contrarrevolucionário enfrenta a mesma questão: como defender um antigo regime que foi ou está sendo destruído? O primeiro impulso – reiterar as antigas verdades do regime – é normalmente o pior, pois muitas vezes foram essas verdades que, em primeiro lugar, colocaram o regime em apuros. Ou o mundo mudou tanto que eles já não
 \par 
Comande o consentimento, ou eles se tornarão tão flexíveis que se transformarão em argumentos a favor da revolução. De qualquer forma, o contra-revolucionário deve procurar noutro lugar materiais a partir dos quais possa moldar a sua defesa do antigo regime. Esta necessidade pode colocá-lo em conflito, como Hobbes veio a perceber, não só com a revolução, mas também com o próprio regime que ele reivindica como a sua causa.
 \par 
Os defensores da monarquia na primeira metade do século XVII e dois tipos de argumentos, nenhum dos quais Hobbes poderia endossar. O primeiro foi o direito divino dos reis. Em si, uma inovação recente - Jaime I, pai de Carlos, foi o maior expoente na Grã-Bretanha - sustentava que o rei era o agente de Deus na terra (na verdade, era bastante parecido com Deus na terra), que ele era responsável apenas perante Deus, e que ele sozinho estava autorizado a governar e não deveria ser restringido pela lei, pelas instituições ou pelo povo. Como supostamente disse o conselheiro de Carlos, “o dedo mínimo do rei deveria ser mais grosso do que os lombos da lei”.{\color{blue}14}
 \par 
Embora tal absolutismo tenha agradado a Hobbes, a fundação da teoria era instável. A maioria dos teóricos do direito divino presumia que o que Hobbes e seus contemporâneos, particularmente no continente, acreditavam não existir mais: uma teleologia de fins humanos que refletia a hierarquia natural do universo e produzia definições inatacáveis ​​de bem e mal, justo e injusto. Após um século de derramamento de sangue sobre o significado desses termos e ceticismo sobre a existência de uma ordem natural ou nossa capacidade de conhecê-la, as defesas do direito divino não pareciam nem críveis nem confiáveis. Com suas premissas duvidosas, elas eram tão propensas a desencadear conflitos quanto a resolvê-los.
 \par 
Indiscutivelmente mais preocupante era que a teoria retratava um teatro político no qual havia apenas dois atores de alguma consequência: Deus e rei, cada um atuando para o outro. Embora Hobbes acreditasse que o soberano nunca deveria dividir o palco com ninguém, ele estava muito sintonizado com a indisposição democrática de seu
 \par 
É hora de não perceber que a teoria negligenciou um terceiro ator: o povo. Tudo estava muito bem quando o povo era quieto e respeitoso, mas durante a década de 1640 um drama secreto entre Deus e o rei não era mais viável. As pessoas estavam no palco, exigindo um papel de liderança; eles não podiam ser ignorados ou receber um pequeno papel.
 \par 
Mudanças na Inglaterra, em suma, tornaram o direito divino insustentável. O desafio que Hobbes enfrentou foi intrincado: como preservar o ímpeto da teoria (submissão inquestionável ao poder absoluto e indiviso) enquanto abandonava suas premissas anacrônicas. Com sua teoria do consentimento, na qual os indivíduos contratam uns com os outros para criar um soberano com poder absoluto sobre eles, e sua teoria da representação, na qual as pessoas são personificadas pelo soberano sem que ele seja obrigado a elas, Hobbes encontrou sua solução.
 \par 
A teoria do consentimento não fez suposições sobre a desfibrilação do bem e do mal, nem se baseou em uma hierarquia natural inerente ao universo, cujo significado deve ser aparente para todos. Ao contrário, a teoria do consentimento presumiu que os homens discordavam sobre tais coisas; na verdade, que eles discordavam tão violentamente que a única maneira de perseguir seus objetivos conflitantes e sobreviver era ceder todo o seu poder ao estado e se submeter a ele sem protesto ou desafio. Protegendo os homens uns dos outros, o estado garantiu a eles o espaço e a segurança para seguir com suas vidas. Quando combinada com o relato de representação de Hobbes, a teoria do consentimento tinha uma vantagem adicional: embora desse todo o poder ao soberano, o povo ainda podia se imaginar em seu corpo, em cada golpe de sua espada. O povo o criou; ele os representou; para todos os efeitos, eles eram ele. Só que não eram: o povo pode ter sido o autor do Leviatã — o nome infame de Hobbes para o soberano, derivado do Livro de Jó — mas, como qualquer autor, eles não tinham controle sobre sua criação. Foi um movimento inspirado, característico
 \par 
De todas as grandes teorias contrarrevolucionárias, nas quais o povo se torna ator sem papéis, um público que acredita estar no palco. O segundo argumento a favor e a favor da monarquia, a posição realista constitucional, tinha raízes mais profundas no pensamento inglês e era, portanto, mais difícil de combater. Afirmava que a Inglaterra era uma sociedade livre porque o poder real era limitado pela lei comum ou partilhado com o Parlamento. Essa combinação de Estado de direito e soberania partilhada, afirmou Sir Walter Raleigh, era o que distinguia os súbditos livres do rei dos escravos ignorantes dos déspotas do Oriente. {\color{blue}15} Foi este argumento e as suas ramificações radicais que aceleraram as reflexões mais profundas e ousadas de Hobbes sobre a liberdade.{\color{blue}16}
 \par 
Por trás da concepção constitucionalista de liberdade política existe uma distinção entre agir em prol da razão e agir sob o comando da paixão. O primeiro é um ato livre; o segundo não é. “Agir por paixão”, escreve Skinner no seu relato do argumento contra o qual Hobbes se posicionou, “não é agir como um homem livre, ou mesmo distintamente como um homem; tais ações não são uma expressão da verdadeira liberdade, mas de mera licenciosidade ou brutalidade animal”. A liberdade implica agir de acordo com o que desejamos; mas a vontade não deve ser confundida com apetite ou aversão. Como disse o bispo Bramhall, o grande antagonista de Hobbes: “Um ato livre é apenas aquele que procede da livre eleição da vontade racional”. E “onde não há consideração nem uso da razão, não há liberdade alguma”. {\color{blue}17} Ser livre implica agir de acordo com a razão ou, em termos políticos, viver sob leis em oposição ao poder arbitrário.
 \par 
Como o direito divino dos reis, o argumento constitucional havia se tornado anacrônico por desenvolvimentos recentes, mais notavelmente o fato de que nenhum monarca inglês na primeira metade do século XVII alegou acreditar nele. Com a intenção de transformar a Inglaterra em um estado moderno, James e Charles foram compelidos a avançar
 \par 
Afirmações muito mais absolutistas sobre a natureza de seu poder do que o argumento constitucional permitia.
 \par 
Mais preocupante para o regime, contudo, foi a facilidade com que o argumento constitucional poderia ser transformado num argumento republicano e usado contra o rei. Advogados comuns e suplicantes parlamentares argumentaram que, ao desrespeitar a lei comum e o Parlamento, Carlos ameaçava transformar a Inglaterra numa tirania; os radicais insistiam que qualquer coisa que não fosse uma república ou uma democracia, onde os homens viviam sob leis com as quais consentiram, constituía uma tirania. Toda monarquia, aos olhos dos radicais, era despotismo.
 \par 
Hobbes pensava que o último argumento derivava das “Histórias e Filosofia dos Antigos Gregos e Romanos”, que foram tão influentes entre os oponentes instruídos do rei. {\color{blue}18} Essa herança antiga ganhou nova vida através dos Discorsi de Maquiavel, traduzido para inglês em 1636, que pode ter sido o alvo final de Hobbes na sua admoestação contra o governo popular. Mas a premissa subjacente ao argumento republicano – de que o que distingue um homem livre de um escravo é que o primeiro está sujeito à sua própria vontade, enquanto o último está sujeito à vontade de outro – também poderia ser encontrada no direito consuetudinário inglês, como Skinner aponta, numa reprodução “palavra por palavra” do “Digest of Roman law”, já no século XIII. Da mesma forma, a distinção entre vontade e apetite, liberdade e licenciosidade estava “profundamente enraizada” tanto nas tradições escolásticas da Idade Média como na cultura humanista do Renascimento. Esta filosofia da vontade encontrou assim expressão não apenas nas posições realistas de Bramhall e da sua turma, mas também entre os radicais e regicidas que derrubaram o rei. Por baixo do abismo que separa monarquistas e republicanos, havia uma base profunda e volátil de suposições partilhadas sobre a natureza da liberdade. {\color{blue}19} A genialidade de Hobbes foi reconhecer essa suposição; sua ambição era esmagá-lo.
 \par 
Embora a noção de que a liberdade implica viver sob leis tenha dado apoio aos monarquistas constitucionais (que faziam grande parte da distinção entre monarcas legítimos e tiranos despóticos), não levou necessariamente à conclusão de que um regime livre tivesse de ser uma república ou uma democracia. Para avançar com esse argumento, os radicais tiveram de fazer duas reivindicações adicionais: primeiro, equiparar a arbitrariedade ou a ilegalidade a uma vontade que não é a própria, uma vontade que é externa ou alheia, como as paixões; e segundo, equiparar as decisões de um governo popular a uma vontade que é própria, como a razão. Estar sujeito a uma vontade que é minha – as leis de uma república ou democracia – é ser livre; estar sujeito a uma vontade que não é a minha – os decretos de um rei ou de um país estrangeiro – é ser um escravo.
 \par 
Ao fazer estas afirmações, argumenta Skinner, os radicais foram ajudados por uma compreensão peculiar, embora popular, da escravatura. O que tornava alguém um escravo, aos olhos de muitos, não era o fato de ele estar acorrentado ou o fato de seu dono impedir ou obrigar seus movimentos. Era que ele vivia e se movia sob uma rede, a vontade arbitrária e em constante mudança de seu mestre, que poderia cair sobre ele a qualquer momento. Mesmo que a rede nunca caísse – o senhor nunca lhe dissesse o que fazer ou nunca o punisse por não fazê-lo, ou ele nunca desejasse fazer algo diferente do que o senhor lhe disse – o escravo ainda era escravizado. O facto de “viver em total dependência” da vontade de outrem, de estar sob a jurisdição do senhor, “era suficiente por si só para garantir o servilismo” que o senhor “esperava e desprezava”.{\color{blue}20}
 \par 
A mera presença de relações de dominação e dependência... É considerada como uma redução do status de... “Homens livres” para o de escravos. Não é suficiente, em outras palavras, desfrutar de nossos direitos e liberdades cívicos como uma questão de fato; se quisermos ser considerados homens livres, é necessário desfrutá-los de uma maneira particular. Nunca devemos mantê-los meramente pela graça ou boa vontade de
 \par 
Alguém mais; devemos sempre mantê-los independentemente do poder arbitrário de alguém para tirá-los de nós.{\color{blue}21}
 \par 
No nível individual, liberdade significa ser dono de si mesmo; no nível político, requer uma república ou democracia. Só uma participação plena no poder público garantirá que desfrutemos da nossa liberdade da “maneira particular” que a liberdade exige; sem plena participação política, a liberdade será fatalmente restringida. É este duplo movimento entre o pessoal e o político que é sem dúvida o elemento mais radical da teoria do governo popular e, na opinião de Hobbes, o mais perigoso.
 \par 
Hobbes começa a destruir o argumento desde o início. Rompendo com os entendimentos tradicionais, ele defende uma explicação materialista da vontade. A vontade, diz ele, não é uma decisão resultante da nossa deliberação fundamentada sobre os nossos desejos e aversões; é simplesmente o último apetite ou aversão que sentimos antes de agir, que então incita o ato. A deliberação é como a haste oscilante de um metrônomo – nossas inclinações vão e voltam, alternando entre apetite e aversão – mas menos constante. Onde quer que a vara pare e produza uma ação ou, inversamente, nenhuma ação, acaba por ser a nossa vontade. Se esta concepção parece arbitrária e mecanicista, deveria: a vontade não está acima dos nossos apetites e aversões, julgando e escolhendo entre eles; a vontade são nossos apetites e aversões. Não existe vontade livre ou autônoma; existe apenas “o último apetite, ou aversão, aderindo imediatamente à ação ou à sua omissão”.{\color{blue}22}
 \par 
Imagine um homem com o maior apetite por vinho, correndo para dentro de um prédio em chamas para resgatar uma caixa; agora imagine um homem com a mais feroz aversão a cães, correndo para o mesmo prédio para escapar de um bando deles. Os oponentes de Hobbes veriam nestes exemplos a força da compulsão irracional; Hobbes vê a vontade
 \par 
Em ação. Estes podem não ser os atos mais sábios ou sãos, admite Hobbes, mas a sabedoria e a sanidade não precisam desempenhar nenhum papel na volição. Ambos os atos podem ser obrigatórios, mas o mesmo ocorre com as ações de um homem em um navio adernado que joga suas malas ao mar para aliviar a carga e se salvar. Escolhas difíceis, ações tomadas sob coação — essas são tantas expressões da minha vontade quanto as decisões que tomo na calma do meu estudo. Estendendo a analogia, Hobbes argumentaria que entregar a minha carteira a alguém que aponta uma arma para a minha cabeça também é um ato voluntário: escolhi a minha vida em vez da minha carteira.
 \par 
Contra os seus oponentes, Hobbes sugere que não pode existir ação voluntária contra a vontade de alguém; toda ação voluntária é uma expressão da vontade. Restrições externas, como estar trancado num quarto, podem impedir-me de agir de acordo com a minha vontade; estar em uma gangue pode me forçar a agir de maneiras que não desejei (quando meu vizinho dá um passo à frente ou levanta sua ferramenta, devo segui-lo, a menos que tenha força física suficiente para resistir a ele e ao sujeito atrás de mim). Mas não posso agir voluntariamente contra a minha vontade. No caso do assaltante, Hobbes diria que a sua arma mudou o meu testamento: passei de querer salvaguardar o dinheiro da minha carteira a querer proteger a minha vida.
 \par 
Se não posso agir voluntariamente contra a minha vontade, não posso agir voluntariamente de acordo com uma vontade que não é a minha. Se eu obedecer a um rei porque temo que ele me mate ou me aprisione, isso não significa ausência, confisco, traição ou sujeição de minha vontade; é minha vontade. Eu poderia ter desejado o contrário – centenas de milhares durante a vida de Hobbes o fizeram – mas a minha sobrevivência ou liberdade era mais importante para mim do que o que quer que possa ter exigido a minha desobediência.
 \par 
A definição de liberdade de Hobbes decorre de sua compreensão da vontade. Liberdade, diz ele, é “a ausência de. . . Impedimentos externos ao movimento”, e um homem livre “é aquele que, naquelas coisas que por sua força e inteligência ele é capaz de fazer, não tem capacidade de fazer o que deseja”. {\color{blue}23} Só posso tornar-me não-livre, insiste Hobbes, por
 \par 
Obstáculos ao meu movimento. Correntes e paredes são esses obstáculos; leis e obrigações são outro tipo, embora mais metafórico. Se o obstáculo estiver dentro de mim, não tenho capacidade para fazer alguma coisa; Tenho muito medo de fazer isso – me falta poder ou vontade, não liberdade. Hobbes, numa carta ao conde de Newcastle, atribui estas agências desfibriladoras à “natureza e qualidade intrínseca do agente”, e não às condições do ambiente político do agente.{\color{blue}24}
 \par 
E esse é o propósito do esforço de Hobbes: separar o estatuto da nossa liberdade pessoal do estado dos assuntos públicos. A liberdade depende da presença do governo, mas não da forma que o governo assume; o facto de vivermos sob um rei, uma república ou uma democracia não altera a quantidade ou a qualidade da liberdade de que desfrutamos. A separação entre liberdade pessoal e política teve o efeito dramático de fazer com que a liberdade parecesse menos presente e mais presente sob um rei do que os antagonistas republicanos e monarquistas de Hobbes tinham permitido.
 \par 
Por um lado, Hobbes insiste que não há como ser livre e sujeito ao mesmo tempo. A submissão ao governo implica uma perda absoluta de liberdade: onde quer que esteja obrigado pela lei, não sou livre para me mover. Quando os republicanos argumentam que os cidadãos são livres porque fazem as leis, afirma Hobbes, estão a confundir soberania com liberdade: o que o cidadão tem é poder político, não liberdade. Ele é tão obrigado (talvez mais obrigado, Rousseau sugerirá mais tarde) a submeter-se à lei e, portanto, tão não-livre, como estaria sob uma monarquia. E quando os monarquistas constitucionais argumentam que os súbditos do rei são livres porque o poder do rei é limitado pela lei, Hobbes afirma que estão simplesmente confusos.
 \par 
Por outro lado, Hobbes pensa que se a liberdade é um movimento desimpedido, é lógico que somos muito mais livres sob um monarca, mesmo um monarca absoluto, do que os monarquistas e os republicanos percebem (ou gostariam de admitir). {\color{blue}25} Em primeiro lugar e de forma mais simples, mesmo quando agimos por medo, estamos a agir livremente. “Medo e Liberdade
 \par 
São consistentes”, diz Hobbes, porque o medo expressa nossas inclinações negativas; essas inclinações podem ser negativas, mas isso não nega o fato de que são as nossas inclinações. Enquanto não formos impedidos de agir de acordo com eles, seremos livres. Mesmo quando estamos mais aterrorizados com as punições do Rei, somos livres: “todas as ações que os homens praticam nas Comunidades, sem a lei, são ações que os executores tiveram a liberdade de omitir”.{\color{blue}26}
 \par 
Mais importante ainda, onde quer que a lei seja omissa, nem comande nem proíba, somos livres. Basta contemplar todas as “formas pelas quais um homem pode se mover”, diz Hobbes em De Cive, para ver todas as maneiras pelas quais ele pode ser livre numa monarquia. Essas liberdades, explica Hobbes em Leviathan, incluem “a liberdade de comprar, vender e de outra forma contratar uns com os outros; escolher a sua própria alimentação, a sua própria dieta, o seu próprio ofício de vida e instituir os seus filhos como eles próprios acharem adequado; e similares.” {\color{blue}27} Na medida em que o soberano possa garantir a liberdade de movimento, a capacidade de cuidar dos nossos negócios sem o impedimento de outros homens, somos livres. A submissão ao seu poder, em outras palavras, aumenta a nossa liberdade. Quanto mais absoluta for a nossa submissão, mais poderoso ele será e mais livres seremos. Subjugação é emancipação.
 \par 
Apesar das isenções de responsabilidade da “Direita Intransigente”, o argumento hobbesiano continua a assombrar o conservadorismo moderno. A ideia de liberdade privada de Hobbes permeia o discurso libertário, enquanto o Leviatã lança uma longa sombra sobre o ideal conservador de um estado vigia noturno – onde o objetivo principal do governo é proteger os cidadãos de ataques estrangeiros e transgressões criminosas; onde as pessoas são livres para cuidar de seus negócios, desde que não interfiram nos movimentos de outras pessoas; e onde os contratos são aplicados e a segurança é garantida.
 \par 
Os libertários empalidecerão diante dessa associação: seja qual for a ressonância que as ideias hobbesianas possam encontrar em seus escritos, o pensamento hobbesiano
 \par 
O Estado é muito mais repressivo do que qualquer governo que um libertário jamais aprovaria. Exceto pelo fato de que não é. Milton Friedman reuniu-se com o ditador chileno Augusto Pinochet em 1975 para aconselhá-lo em questões económicas; Os Chicago Boys de Friedman trabalharam ainda mais estreitamente com a junta de Pinochet. Sergio de Castro, ministro das Finanças de Pinochet, fez a observação, reminiscente de Hobbes, de que “a verdadeira liberdade de uma pessoa só pode ser assegurada através de um regime autoritário que exerça o poder através da implementação de regras iguais para todos”. Hayek admirava tanto o Chile de Pinochet que decidiu realizar uma reunião da sua Sociedade Mont Pelerin em Viña del Mar, a estância balnear onde foi planeado o golpe contra Allende. Em 1978, escreveu ao London Times que “não tinha conseguido encontrar uma única pessoa, mesmo no muito difamado Chile, que não concordasse que a liberdade pessoal era muito maior sob Pinochet do que sob Allende”.{\color{blue}28}
 \par 
“Apesar do meu forte desacordo com o sistema político autoritário do Chile”, Friedman afirmaria mais tarde, “não considero mau um economista prestar aconselhamento técnico económico ao governo chileno”. {\color{blue}29} O casamento entre mercados livres e terror de Estado não pode ser anulado tão facilmente. Como Hobbes entendeu, é necessária uma enorme quantidade de repressão para criar o tipo de homens que possam exercer a sua “Liberdade de comprar, vender e, de outra forma, contratar uns com os outros” sem ficarem desleixados. {\color{blue}30} Eles devem ser livres para se movimentar – ou escolher – mas não tão livres a ponto de pensar em redesenhar a rodovia. Assumindo uma congruência demasiado fácil entre capitalismo e democracia, o libertário ignora quanta coerção é necessária para fazer com que os cidadãos utilizem a sua liberdade de forma responsável e aceitem a angústia sem recorrer ao Estado em busca de alívio.
 \par 
Foi preciso Margaret Thatcher, entre todas as pessoas, para explicar este facto à direita libertária. Quando pressionado por Hayek para impor a política de Pinochet
 \par 
Marca de terapia de choque na Grã-Bretanha, Thatcher respondeu: “Tenho a certeza de que concordarão que, na Grã-Bretanha, com as nossas instituições democráticas e a necessidade de um elevado grau de consentimento, algumas das medidas adoptadas no Chile são bastante inaceitáveis”. Estávamos em 1982, e sendo a democracia britânica o que era, Thatcher teve de ir devagar. Mas depois veio a Guerra das Malvinas e a greve dos mineiros. Assim que Thatcher percebeu que poderia fazer aos mineiros e aos sindicatos o que tinha feito ao presidente Galtieri e aos seus generais argentinos: “Tivemos de combater o inimigo externo nas Malvinas, e agora temos de combater o inimigo interno, que é muito mais difícil, mas igualmente perigoso para a liberdade” – o cenário estava montado para todo o mês hayekiano.{\color{blue}31}