\chapter{2 Produção de commodities}\label{2 Produção de commodities}
 \par 
Marx é conhecido pelo seu compromisso com o que é considerado a teoria do valor-trabalho. Muitos aspectos diferentes da sua análise do valor e do capital (ismo) têm sido objecto de feroz controvérsia, tanto entre aqueles que são a favor como entre aqueles que são contra Marx e, algo que está intimamente relacionado, mas distinto, sobre diferentes interpretações do que ele realmente quis dizer. : os comentaristas discordam sobre o que Marx está dizendo, bem como sobre se está correto ou não. Como resultado, existem várias interpretações da teoria do valor-trabalho, muitas das quais são impostas a Marx por ignorância, por um desejo de o despedir ou, perversamente, por tentar defendê-lo. Além disso, é muitas vezes possível relacionar as disputas sobre a economia política de Marx às diferenças sobre a sua teoria do valor. Com ou sem razão, duas questões têm sido fundamentais nestes debates contínuos: será que Marx privilegiou indevidamente o trabalho, de alguma forma, ao adoptar a teoria do valor-trabalho, e até que ponto a teoria do valor-trabalho serve como teoria dos preços?
 \par 
O objetivo deste capítulo é embarcar em uma jornada analítica que será levada adiante ao longo do restante do livro. Colocamos várias questões à teoria do valor-trabalho, questões que se aproximam do método e do conteúdo da obra de Marx. Para ele, a teoria do valor-trabalho não pode ser provada correta por meio de alguma magia conceitual ou por meio de acrobacias técnicas ou algébricas. Pelo contrário, a teoria do valor de Marx visa reproduzir no pensamento - em níveis crescentes de complexidade - as principais relações, processos e estruturas económicas prevalecentes na sociedade capitalista (ver Capítulo {\color{blue}1}). É contra este teste que a sua teoria do valor e as suas interpretações devem ser julgadas. Embora a teoria do valor de Marx tenha um início simples, que é o foco deste capítulo, torna-se mais rica e mais complicada à medida que se desenvolve gradualmente, a fim de confrontar, explicar e incorporar as complexidades do próprio capitalismo. Será demonstrado em capítulos posteriores que estas complexidades não negam a teoria do valor de Marx. Em vez disso, confirmam a sua consistência interna e poder explicativo, mas dentro de limites que precisam de ser reconhecidos para evitar o “reducionismo” - a noção de que tudo pode ser explicado pela teoria do valor. Trata-se de incorporar mais material historicamente específico para prosseguir.
 \par 
\[A Teoria do Valor do Trabalho\]
 \par 
Ao analisar um modo de produção, como o capitalismo, o ponto de partida de Marx é sempre a produção - como é que as sociedades capitalistas produzem as condições materiais da sua própria reprodução?
 \par 
Em qualquer sociedade, a produção cria valores de uso, ou seja, coisas úteis como alimentos, roupas e casas, bem como produtos imateriais como educação, saúde e outros serviços pessoais, todos eles (mais ou menos) necessários para a continuidade. existência da sociedade. Assim, a divisão do trabalho e a produção de valores de uso podem ser tidas como certas como características duradouras da organização humana desde a nossa origem como espécie. Mas quem produz o quê e como, e com que implicações para a economia e a sociedade, são questões cruciais em todas as ciências sociais. Diferentes disciplinas e ideologias deram respostas diferentes, desde a ordem natural e a tradição (religiosa), passando pela busca do interesse próprio, até à ideia da necessidade como mãe da invenção. A economia dominante (ortodoxa ou neoclássica), em particular, considerou a necessidade de consumo e a capacidade humana de fazer escolhas como justificativa de uma abordagem ou método universal em que a economia é a ciência preocupada com a alocação de recursos escassos para satisfazer necessidades insaciáveis. Deste ponto de vista, a economia pode ser organizada através do mercado, do Estado, do agregado familiar ou através da escravatura, por exemplo. Estes são apenas detalhes, em oposição à dualidade fundamental entre escassez e necessidade que é o foco da economia dominante e que fornece o critério para medir a eficiência relativa das formas alternativas de organizar a sociedade e as suas partes componentes, tais como empresas, a família ou o governo.
 \par 
Em contraste, para Marx, as relações sociais, especialmente de classe, são essenciais para distinguir uma economia de outra, bem como para revelar diferenças dentro de uma economia. Isto envolve não apenas as relações de propriedade e de distribuição que definem os modos de produção, ou quem possui o quê e porquê, mas também como a propriedade é organizada e dá origem a formas de controlo do trabalho e dos seus produtos, bem como outros aspectos da organização social. . Assim, por exemplo, uma característica crucial do capitalismo é o facto de ser um sistema altamente desenvolvido de produção de mercadorias - mas qual é o significado das mercadorias? Seguindo Adam Smith, Marx distingue o valor de uso do valor de troca dentro de cada mercadoria: a sua utilidade, que não pode ser quantificada em geral, da capacidade de troca com outras mercadorias, que pode ser quantificada. Toda mercadoria tem um valor de uso, ou capacidade de satisfazer as necessidades humanas, sem o qual não poderia ser vendida e, portanto, não seria produzida para venda. Mas nem todo valor de uso é uma mercadoria; por exemplo, os valores de uso criados naturalmente, disponíveis gratuitamente ou não trocados por dinheiro no mercado não têm valor de troca (por exemplo, luz solar, ar, espaços abertos, frutos silvestres, produção para uso pessoal, produção para ou em nome de, parentes ou amigos, ou “bens públicos”, incluindo o acesso a estradas abertas ou a sistemas públicos de saúde e educação disponíveis gratuitamente).
 \par 
O valor de troca incorpora uma relação de equivalência entre objetos. Esta relação tem de satisfazer certas propriedades que nos são familiares na vida quotidiana, especialmente no mercado e no cálculo comercial. Por exemplo, se x for trocado por y (que pode ser representado como x ~ y), então, em geral, 2x ~ 2y. Se, além disso, u ~ v, então (u e x) ~ (v e y) e assim por diante. Mas há um número ilimitado de relações que satisfazem estas propriedades, por exemplo, peso ou volume. A questão que Marx procura responder é: que relação social pode fornecer a base para trocas de mercado sistemáticas (em vez de fortuitas) e, mais genericamente, para a reprodução social sob as circunstâncias históricas específicas do capitalismo? O que permite que as mercadorias sejam equivalentes na troca? No caso do peso ou do volume, a equivalência deve-se a propriedades físicas ou naturais (nomeadamente massa e tamanho, respetivamente), propriedades que existem independentemente de serem e como são realmente medidas e que são independentes da troca ou do modo de produção. Além disso, embora cada mercadoria seja caracterizada pelas suas propriedades físicas específicas que, em parte, lhe conferem o seu valor de uso (sendo a outra parte derivada da cultura de consumo e uso), o seu valor de troca não está relacionado com essas propriedades. Como já foi mencionado, as coisas mais úteis, o ar, a luz solar e a água, muitas vezes têm pouco ou nenhum valor de troca.
 \par 
O que cria a relação de troca, então, não é uma relação física entre bens, mas uma relação social historicamente específica, sobretudo a forma como a produção de valores de uso é organizada - para o mercado e geralmente para o lucro, no caso de mercadorias produzidas nas sociedades capitalistas. A economia dominante começou recentemente a prestar mais atenção a isto, ao aceitar que as instituições, a confiança, a cultura, e assim por diante, influenciam as trocas, até porque os mercados são, invariavelmente, “imperfeitos”, em certo sentido. Mas isso significa interpretar o argumento de maneira errada. Antes de examinar as instituições como resposta às imperfeições do mercado, o próprio mercado tem de ser explicado como uma “instituição”, ou de outra forma. A um nível mais profundo, os mercados não são simplesmente estruturas neutras de troca, mas são específicos em cada caso, uma vez que reflectem fundamentalmente as relações sociais que os sustentam; como se pode observar, por exemplo, nas diferenças entre os mercados de divisas, petróleo, computadores e produtos alimentares.
 \par 
Isto leva Marx a sugerir que subjacente à equivalência entre mercadorias como valores de uso está uma relação social entre os produtores dessas mercadorias. Isto porque, para Marx, é axiomático que ao longo da história as pessoas viveram do seu trabalho: se todos parassem de trabalhar, nenhuma sociedade sobreviveria além de alguns dias. Além disso, em todas as sociedades, exceto nas mais simples, alguns sempre viveram sem trabalhar, isto é, viveram do trabalho de outros. Contudo, a apropriação do trabalho de uma pessoa (ou dos seus produtos) por outra assume diferentes formas e é justificada de diferentes maneiras. Sob o feudalismo, os produtos são frequentemente distribuídos por apropriação direta justificada pelo direito feudal ou mesmo divino. Sob o capitalismo, os produtos do trabalho geralmente assumem a forma de mercadorias e são geralmente distribuídos pelas trocas de mercado. A forma como esta liberdade de mercado provoca uma apropriação do trabalho de uma classe por outra será discutida no Capítulo {\color{blue}3}. Por enquanto, estamos apenas interessados ​​na natureza da relação de troca. Por outras palavras, numa sociedade produtora de mercadorias, o que há de especial na produção e no trabalho?
 \par 
Para responder a esta questão, Marx define mercadorias como valores de uso produzidos pelo trabalho para troca. Isto significa que nem tudo o que é trocado, mesmo através do mercado, é uma mercadoria. Talvez isto seja facilmente aceitável no caso de suborno, bens de segunda mão comercializados casualmente ou mesmo obras de arte, embora cada um destes possa impor um preço (ou seja, assumir a forma de mercadoria) à sua própria maneira. Mas, em parte para antecipar, para Marx estes são fenómenos incidentais, que não desempenham papéis fundamentais na reprodução económica e social, e devem ser abstraídos de forma causal e analítica quando se aborda a produção de mercadorias em geral e sob o capitalismo em particular.
 \par 
Segue-se que uma propriedade fundamental que todas as mercadorias têm em comum é que elas são produtos do trabalho. Essa propriedade se baseia na percepção fundamental de que as sociedades não podem viver (e os lucros não podem surgir) somente por meio da troca, mas, em vez disso, que a troca sistemática deve ser fundamentada em um modo específico de produção para sustentar a si mesma (e à sociedade). Da mesma forma, na sociedade de mercadorias, os trabalhos concretos (produzindo valores de uso específicos) não são realizados casualmente, mas como parte de uma intrincada divisão social do trabalho que os conecta uns aos outros por meio do mercado, ou seja, por meio da troca de seus produtos por dinheiro.
 \par 
Esta é uma relação social qualitativa e impessoal. Por exemplo, geralmente compramos mercadorias sem saber nada sobre quem as produziu e como. Porque a produção de mercadorias exige uma divisão do trabalho dentro e entre diferentes locais de trabalho, onde diferentes trabalhos são contribuídos, reunidos e medidos uns contra os outros, ainda que indirectamente, através do mercado. Este processo social é a base da teoria do valor-trabalho e incorpora relações que podem ser facilmente quantificadas teoricamente analisando a troca do ponto de vista do tempo de trabalho socialmente (e não individualmente) necessário para produzir mercadorias: por exemplo, a quantidade de o tempo de trabalho necessário para cozer um pão em comparação com o necessário para coser uma camisa (e, igualmente importante, como estes tempos de trabalho são determinados e modificados através de mudanças tecnológicas e outras). Segue-se que, para Marx, a teoria do valor-trabalho não é uma noção metafísica. Em vez disso, capta analiticamente os aspectos essenciais da vida material sob o capitalismo, relativos à forma como a produção é organizada e ligada ao mercado, e como os produtos do trabalho social estão relacionados entre si, apropriados e distribuídos dentro da sociedade.
 \par 
Marx percebe que em sociedades capitalistas, onde os produtos tipicamente assumem a forma de mercadorias, a produção é primariamente realizada para troca por lucro em vez de para uso direto ou imediato. O capitalismo é um sistema que visa produzir valores de uso social - valores de uso para outros desconhecidos por causa do anonimato do mercado. A produção de valores de uso social, trocas de mercado e obtenção de lucro estão intimamente ligadas umas às outras. Mas assim como os produtos incorporam valores de uso social, eles são criados pelo trabalho social no abstrato (por assalariados desconhecidos, contratados através do mercado de trabalho e disciplinados dentro de empresas concorrentes pelo imperativo do lucro, e fora delas pelo sistema de crédito e pelo mercado de ações). Dessa forma, os produtos do trabalho concreto contam como trabalho social abstrato em sociedades capitalistas, e a troca não diz respeito à qualidade ou tipo de trabalho concreto, mas apenas à quantidade de trabalho abstrato, expressa através dos preços das mercadorias. Em troca, o que importa em quanto você tem que pagar não é o valor de uso que você quer - se o tempo de trabalho foi gasto por um padeiro, alfaiate, motorista de ônibus ou programador de computador - mas quais quantidades de tempo de trabalho abstrato (socialmente necessário, em vez de individual e concreto) foram gastas.
 \par 
Para Marx, o valor de uma mercadoria é o tempo de trabalho socialmente necessário para a produzir, incluindo tanto os factores de trabalho directos (vivos) como os indirectos (mortos ou passados) - o tempo de trabalho necessário para produzir os meios de produção necessários.
 \par 
Isto não significa sugerir que as mercadorias devam ser negociadas pelos seus valores. Os preços de mercado serão afectados pelos rácios de mão-de-obra indirecta e directa, escassez, competências, monopólios, gostos e por variações mais ou menos acidentais na oferta e na procura, além da equalização dos lucros entre sectores concorrentes (ver Capítulo {\color{blue}10}). Estas influências contingentes têm sido os principais objectos de estudo dos economistas ortodoxos desde a revolução marginalista (neoclássica) da década de 1870, tendo sido feito pouco progresso nas ideias de Adam Smith da década de 1770, excepto através da crescente sofisticação matemática. Marx não ignorou estes factores complicadores, mas também não os colocou no centro das atenções, pois são irrelevantes para revelar as relações sociais de produção específicas do capitalismo. Se isto não pode ser feito partindo do pressuposto de que as mercadorias são negociadas pelos seus valores, certamente não pode ser feito nos casos mais complicados, quando isso não acontece. Ao longo deste livro, salvo indicação em contrário, assumir-se-á que as mercadorias são negociadas pelos seus valores. Isto não deve ser interpretado como uma teoria de preços completa, mas como uma tentativa de compreender a natureza do sistema de preços e os processos essenciais que sustentam a reprodução económica das sociedades capitalistas.
 \par 
Assim, o capitalismo, como produção generalizada de mercadorias com fins lucrativos, é caracterizado pela produção de valores de uso social e, portanto, pela troca dos produtos de trabalhos concretos que existem, e contribuem para o valor, como trabalho social abstrato. Metodologicamente, isto não é uma imposição analítica da noção de valor, mas simplesmente um reflexo do que o sistema de mercado faz - liga trabalhos concretos entre si e mede-os uns contra os outros. Marx não baseou o seu conceito de valor numa construção mental afastada do mundo real e que exigia pressupostos arbitrários (inventados) para se adequar à teoria. Pelo contrário, o seu argumento baseia-se no facto de que a redução de todos os tipos de trabalho a um padrão comum (de preços) é um produto necessário e espontâneo do próprio mundo real do capitalismo. Com base na análise metodológica do Capítulo 1, a teoria do valor-trabalho de Marx reproduz, antes de mais nada, no pensamento a forma como o capitalismo realmente organiza a produção dos bens e serviços necessários à reprodução social. Reconhece que a relação entre mercadorias como valores de uso (os seus preços relativos) é o resultado de uma relação social subjacente entre os produtores que expressa a equivalência entre os seus próprios trabalhos concretos qualitativamente diferentes como trabalho social abstracto. O importante é que a relação entre troca, preços e valores não é exclusivamente, ou mesmo principalmente, quantitativa; em vez disso, reflecte relações sociais definidas de produção, distribuição e troca. São estes que devem ser compreendidos.
 \par 
\[A Teoria do Valor do Trabalho\]
 \par 
A secção anterior mostrou que, na sociedade capitalista, a troca de diferentes tipos de produtos do trabalho ocorre através da troca de mercadorias. Isto poderia ocorrer sem o capitalismo, por exemplo, se os membros de uma hipotética sociedade de artesãos independentes trocassem directamente os seus produtos - o que é frequentemente denominado produção simples de mercadorias. No entanto, esta é uma possibilidade mais lógica do que um modo de produção que sempre foi historicamente dominante. A experiência mental serve para realçar que o que caracteriza o capitalismo não é a troca de produtos, mas a compra e venda da capacidade de trabalho dos trabalhadores e a sua utilização na produção de mercadorias com fins lucrativos.
 \par 
Para distinguir os próprios trabalhadores da sua capacidade ou capacidade de trabalhar, Marx chamou esta última de força de trabalho e o seu desempenho ou aplicação de trabalho. Esses conceitos são muito importantes para a teoria do valor-trabalho, mas muitas vezes são mal compreendidos.
 \par 
A característica distintiva mais importante do capitalismo é que a força de trabalho se torna uma mercadoria. O capitalista é o comprador, o trabalhador é o vendedor e o preço da força de trabalho é o salário. O trabalhador vende força de trabalho ao capitalista, que determina como essa força de trabalho deve ser exercida como trabalho para produzir mercadorias específicas. Como uma mercadoria, a força de trabalho tem um valor de uso, que é a criação de outros valores de uso. Essa capacidade humana é independente do tipo de sociedade em que a produção ocorre. No entanto, no capitalismo, os valores de uso são produzidos para venda e, como tal, incorporam tempo de trabalho abstrato ou valor. Nessas sociedades, a mercadoria força de trabalho também tem o valor de uso específico de que é a fonte de valor quando exercida como trabalho. Nisso, a força de trabalho é única: o consumo de todos os insumos auxilia na produção da saída, mas apenas o consumo da força de trabalho também adiciona valor como tempo de trabalho à saída.
 \par 
Na sociedade capitalista, o trabalhador não é um escravo no sentido convencional da palavra e vendido como outras mercadorias, mas possui e vende força de trabalho. Além disso, o período de tempo pelo qual a venda é feita ou formalmente contratada é frequentemente muito curto (uma semana, um mês ou, às vezes, apenas até que uma tarefa específica seja concluída; por exemplo, nos chamados "contratos de zero hora"). No entanto, em muitos outros aspectos, o trabalhador é como um escravo. O trabalhador tem pouco ou nenhum controle sobre o processo de trabalho ou seu produto. Existe a liberdade de se recusar a vender força de trabalho, mas esta é uma liberdade parcial, a alternativa, em última análise, sendo a fome ou a degradação social. Alguém poderia igualmente argumentar que um escravo poderia fugir ou se recusar a trabalhar, embora o nível e a imediatez da retribuição sejam de uma ordem completamente diferente. Por essas razões, os trabalhadores sob o capitalismo foram descritos como escravos assalariados, embora o termo seja um oxímoro. Você não pode ser escravo e trabalhador assalariado - por definição, o escravo não tem as liberdades que o trabalhador assalariado desfruta, independentemente de outras condições.
 \par 
Do outro lado da classe dos trabalhadores estão os capitalistas, que controlam os trabalhadores e os produtos do seu trabalho através do seu comando dos pagamentos salariais e da propriedade dos meios de produção. Esta é a chave para as relações de propriedade específicas do capitalismo. Pois o monopólio capitalista dos meios de produção vincula os trabalhadores à relação salarial, como explicado acima. Se os trabalhadores possuíssem ou tivessem o direito de utilizar os meios de produção independentemente do contrato salarial, não haveria necessidade de vender força de trabalho em vez de produtos acabados no mercado e, portanto, não haveria necessidade de se submeter ao controlo capitalista tanto durante a produção como durante a produção. fora dele.
 \par 
Estabelecer a distinção entre trabalho e força de trabalho mostra que a teoria do valor-trabalho não só capta as relações de distribuição estabelecidas através da troca de produtos de trabalho, mas também incorpora e expressa as relações de produção e exploração específicas do capitalismo. A troca social da força de trabalho por dinheiro, tal como a troca dos produtos do trabalho através do mercado, pressupõe, por um lado, o monopólio dos meios de produção pela classe dos capitalistas e, por outro, a existência de uma classe de trabalhadores assalariados sem acesso direto aos meios de produção (ver Capítulo {\color{blue}6}). Não é de surpreender que esta distinção extremamente importante entre trabalho e força de trabalho nunca seja traçada na economia dominante, com a sua terminologia “neutra” de factores de produção (incluindo trabalho) e produtos. A terminologia dominante sugere que os insumos trabalho e capital contribuem da mesma forma para o processo de produção, tanto que os trabalhadores são conceituados como “capital humano” e, portanto, reduzidos ao status de insumos físicos, assim como o próprio “capital”. em vez de serem vistos como decorrentes de relações de classe historicamente específicas.
 \par 
\[A Teoria do Valor do Trabalho\]
 \par 
Marx percebe que a troca de valores de uso produzidos reflete a organização social do trabalho que produziu essas mercadorias. Mas para muitos dos seus economistas contemporâneos e para quase todos os subsequentes, a relação entre os trabalhadores e os produtos do seu trabalho permanece apenas uma relação entre coisas, do tipo x pães = {\color{blue}1} camisa, ou uma semana de trabalho vale tanto. de um padrão de vida (o pacote salarial). Assim, embora o capitalismo organize a produção em relações sociais definidas entre capitalistas e trabalhadores, estas relações são expressas e aparecem, em parte, como relações entre coisas. Estas relações sociais ficam ainda mais mistificadas quando o dinheiro é levado em consideração e tudo é analisado em termos de preço. Marx chama esta perspectiva do mundo capitalista de fetichismo das mercadorias. É mais evidente na economia moderna onde, como foi mostrado acima, até a força de trabalho é tratada como um factor ou factor como qualquer outro. As recompensas dos factores são vistas, antes de mais, como sendo devidas às propriedades físicas dos factores de produção, como se o lucro, os juros ou a renda fossem directamente produzidos por maquinaria, dinheiro ou terra, e não por pessoas que coexistem em relações e sociedades específicas.
 \par 
Marx traça o paralelo brilhante entre o fetichismo da mercadoria e a devoção religiosa feudal; isto não é surpreendente, dada a influência anterior de Feuerbach. Deus é a própria criação da humanidade.
 \par 
Sob o feudalismo, as relações humanas com Deus ocultam e justificam as relações reais com os semelhantes, um vínculo absurdo de exploração senhorial, tal como parece à mente burguesa (capitalista). O capitalismo, no entanto, tem o seu próprio Deus e a sua própria Bíblia. A relação de troca entre as coisas também é criada pelas pessoas, ocultando a verdadeira relação de exploração e justificando-a pela doutrina da liberdade de troca.
 \par 
Mas há uma grande diferença entre o fetichismo religioso e o fetichismo mercantil. Pois, embora Deus seja uma criação das religiões, as mercadorias têm uma existência real e a sua troca representa e, em certa medida, oculta as relações sociais de produção (ver Capítulo {\color{blue}1}). Da mesma forma, o sistema de preços existe e está ligado ao sistema económico e social mais amplo, mas sem tornar transparente a natureza desse sistema. Em particular, a compra e venda de mercadorias não revela as circunstâncias pelas quais elas chegaram ao mercado, ou a exploração dos produtores directos (os trabalhadores assalariados) pela classe capitalista. Consequentemente, a ênfase de Marx está nos preços como um sistema de valores, determinado pelas relações de classe de produção e exploração. Mas não são apenas as relações de classe e de produção que são fetichizadas pela sua forma de mercadoria. Por exemplo, só remontando desde o mercado até à produção é que poderemos romper o véu da publicidade e descobrir se os produtos são, por exemplo, amigos do ambiente, ou “orgânicos”, ou livres da exploração do trabalho infantil, e breve.
 \par 
Sob esta luz, o fetichismo da mercadoria pode tornar-se a base de uma teoria da alienação ou reificação. Não só os trabalhadores estão divorciados do controlo do produto e do processo de produção, como também a sua visão desta situação é normalmente distorcida ou, no máximo, parcial. Além disso, os capitalistas estão sujeitos ao controlo social através da concorrência e da necessidade de rentabilidade. Tanto para os capitalistas como para os trabalhadores, parece que os poderes externos exercem este controlo, e não as relações sociais de produção e os seus efeitos sob o capitalismo. Mais uma vez, há um sentido em que isso é verdade. Por exemplo, a perda de emprego ou a falência podem ser atribuídas a uma coisa ou a uma força impessoal, como na infeliz introdução (ou, alternativamente, na avaria) de uma máquina, nas mudanças nas preferências dos consumidores, na concorrência internacional ou na crise económica de qualquer origem ou origem. causa. Mais recentemente, a “globalização” tem sido entendida em termos genéricos, quase religiosos, como sendo capaz de explicar todas as coisas boas ou más sobre o capitalismo contemporâneo (ver Capítulos {\color{blue}14} e {\color{blue}15}). Mas para dar vida analítica e explicativa à concorrência, à crise económica e à globalização, e ir além do misticismo, devemos começar com uma compreensão clara das relações sociais subjacentes à produção capitalista, em vez de fetichizar os seus efeitos.
 \par 
A distinção entre fetichismo religioso e fetichismo mercantil não é simplesmente acadêmica. Devido às suas origens imaginárias, o fetichismo religioso pode ser facilmente rejeitado, pelo menos em teoria, embora na realidade seja apoiado por forças e práticas materiais que lhe conferem uma influência considerável nas nossas vidas quotidianas e em toda a história humana. Em contraste, por mais bem compreendido que seja, não é possível eliminar o sistema de preços através de um acto de vontade, excepto em casos marginais e em tentativas frágeis de auto-sustentabilidade. Como resultado, e aqui novamente há um paralelo com o fetichismo religioso, é possível que as realidades capitalistas subjacentes sejam apreendidas de tempos em tempos através das consequências das práticas diárias e da reflexão sobre elas. Assim como se pode perceber que Deus não existe, também se pode ver que o capitalismo é um sistema de classes explorador que está longe de ser livre, qualquer que seja o grau de (des)igualdade perante o mercado. Isto abre terreno para lutas materiais e ideológicas. Pois a existência de lucros, juros e rendas indica que o capitalismo é explorador; como consequência, o desemprego, as crises económicas, as desigualdades, a degradação ambiental, e assim por diante, tornam-se tão transparentemente visíveis como a incapacidade dos mansos de herdarem a terra ou de comerem torta no céu quando morrem.
 \par 
Isto levanta duas questões estreitamente relacionadas e calorosamente debatidas dentro do marxismo e em todas as ciências sociais e no espectro político em geral. A primeira é a questão metodológica e analítica de como ordenar os diversos resultados empíricos associados ao capitalismo. Conseguiremos lidar com a desigualdade independentemente da classe, com a pobreza independentemente da repressão económica e outras formas de repressão, e com o crescimento independentemente da crise? Em segundo lugar, até que ponto tais condições são endémicas ou deformáveis ​​dentro do capitalismo? Pois não se trata simplesmente de uma questão de ligações lógicas entre as diferentes categorias da economia política, entre valor e preço, por exemplo. Um dos pontos fortes do Capital de Marx, reconhecido tanto por amigos como por inimigos, é ter apontado para o carácter sistémico do capitalismo e para as suas características essenciais. Da mesma forma, a antipatia do marxismo pelo reformismo, a não ser como parte de uma estratégia mais ampla para o socialismo, baseia-se nas inevitáveis ​​limitações do reformismo e na acomodação aos limites impostos pelo capitalismo. Em torno destas questões, permanece muito espaço para disputas sobre o método, a teoria e a política de reforma, tanto no debate dentro como contra o marxismo.
 \par 
Tais perspectivas lançam luz sobre o desenvolvimento intelectual do próprio Marx. Pois o seu conceito maduro de fetichismo da mercadoria estabelece uma ligação com trabalhos anteriores (nos Manuscritos Económicos e Filosóficos de 1844). Aqui, ao mesmo tempo que rompeu com o idealismo hegeliano e adoptou uma filosofia materialista, desenvolveu uma teoria da alienação. Isto concentrou-se na relação do indivíduo com a atividade física e mental e com os outros seres, e na consciência desses processos. Em O Capital, após extenso estudo económico, Marx consegue explicitar as forças coercitivas exercidas pela sociedade capitalista sobre o indivíduo. Estas podem ser a compulsão da rentabilidade e do trabalho assalariado, ou as distorções mais subtis pelas quais estas forças são justificadas ideologicamente: a abstinência, a ética do trabalho, o consumismo, a liberdade de troca e outros aspectos do fetichismo da mercadoria. Ao contrário de outras teorias da alienação, uma teoria marxista coloca o indivíduo numa posição de classe e examina as percepções dessa posição. Cada um não é visto, em primeira instância, como um indivíduo impotente numa situação inexplicável.
 \par 
‘Sistema’ de irracionalidade, personalidade, desigualdade, autoritarismo, burocracia, ou o que quer que seja. Estes fenómenos têm carácter e função próprios na sociedade capitalista num determinado momento. Eles só podem ser entendidos como um todo ou em relação aos indivíduos, na perspectiva do funcionamento do capitalismo, como será explicado nos capítulos seguintes.
 \par 
\[A Teoria do Valor do Trabalho\]
 \par 
A teoria do valor de Marx é extremamente controversa entre proponentes e oponentes. Um ponto de partida essencial na avaliação de debates é a distinção entre as abordagens de Ricardo e Marx, com muitos identificando erroneamente os dois como aderentes à (mesma) teoria do valor-trabalho. Mas Ricardo simplesmente conta o tempo de trabalho para explicar o preço, sem investigar por que os produtos assumem a forma de mercadorias. Este último é o ponto de partida de Marx, justificando o valor como uma categoria em sua abordagem, uma vez que a própria sociedade, por meio do processo de produção capitalista e do mercado, empreende a comparação qualitativa e quantitativa dos tempos de trabalho (concretos). Sobre isso, veja especialmente Geoffrey Pilling (1980) e as contribuições em Ben Fine (1986), Diane Elson (1979a) e Jesse Schwartz (1977, pt.{\color{blue}5}).
 \par 
A teoria do valor de Marx é discutida extensivamente ao longo de suas obras maduras, especialmente Marx (1976, pt. {\color{blue} 1 } {\par} , 1987). Para uma visão concisa da teoria e suas implicações, ver Marx (1981a, pt. {\color{blue} 7 } {\par} , 1998); ver também Friedrich Engels (1998, pt.{\color{blue}2}). A interpretação neste capítulo baseia-se em Ben Fine (1980, cap. {\color{blue} 6 } {\par} , 2001, 2002, cap.{\color{blue}3}), Ben Fine, Gong Gimm e Heesang Jeon (2010) e Alfredo Saad-Filho (2003b). Para opiniões semelhantes, consulte Diane Elson (1979b), Duncan Foley (1986, cap.{\color{blue}2}), David Harvey (1999, cap. {\color{blue} 1 } {\par} , 2009, 2010), Moishe Postone (1993) e John Weeks (1990, 2010, caps {\color{blue}1}). -2). Duncan Foley (2000) e Alfredo Saad-Filho (1997a, 2002, cap.{\color{blue}2}) e as contribuições de Simon Mohun (1995) explicam criticamente e revisam interpretações alternativas da teoria do valor de Marx.