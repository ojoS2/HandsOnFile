\chapter{Febre Potomac}\label{Febre Potomac}
 \par 
O ano de 1948, escreveu certa vez John Cheever, foi “o ano em que todos nos Estados Unidos estavam preocupados com a homossexualidade”. E ninguém estava mais preocupado do que o governo federal, que, segundo rumores, estava repleto de gays e lésbicas. Poderíamos pensar que as atenções de Washington estariam concentradas noutro lado – na União Soviética, digamos, ou nos espiões comunistas. Mas em 1950, os conselheiros do presidente Truman alertaram-no de que “o país está mais preocupado com as acusações de homossexuais no governo do que com os comunistas”. O poder executivo respondeu imediatamente. Naquele ano, o Departamento de Estado enviou “pervertidos” à razão de um por dia, mais do dobro do número de suspeitos de serem comunistas. As acusações de homossexualidade foram responsáveis ​​por um quarto a metade de todas as demissões no Departamento de Estado, no Departamento de Comércio
 \par 
Departamento e a CIA. Apenas 25% das cartas de fãs de Joseph McCarthy reclamaram de “infiltração vermelha”; o resto se preocupava com a “depravação sexual”.{\color{blue}1}
 \par 
O Lavender Scare, como é chamado, durou de 1947 até a década de 1970, e milhares de pessoas perderam seus empregos. Foi um exercício de humilhação – e hilaridade. Pois os homens e mulheres encarregados de enxaguar o rosa do Potomac eram surpreendentemente ignorantes sobre a sua presa. O senador Clyde Hoey, chefe do primeiro inquérito do Congresso sobre a ameaça, teve de perguntar a um assessor: “Pode dizer-me, por favor, o que é que duas mulheres podem fazer?” A senadora Margaret Chase Smith perguntou a uma testemunha do comitê Hoey se não havia um “teste rápido como um raio X que revela essas coisas”.{\color{blue}2}
 \par 
A justificativa oficial para o expurgo era que os homossexuais eram vulneráveis ​​à chantagem e poderiam ser transformados em espiões soviéticos. Mas os investigadores nunca encontraram um único caso deste tipo de chantagem durante a Guerra Fria. O melhor que conseguiram encontrar foi um caso duvidoso ocorrido antes da Primeira Guerra Mundial, quando os russos alegadamente usaram a homossexualidade do principal espião da Áustria para forçá-lo a trabalhar para eles.{\color{blue}3}
 \par 
A verdadeira justificação era ainda mais suspeita: os homossexuais eram desajustados sociais cuja patologia os tornava susceptíveis à doutrinação comunista. Muitos conservadores também acreditavam que o Partido Comunista era um movimento de e para libertinos, e a União Soviética um refúgio de amor livre e casamento aberto. Os gays, concluíram eles, não conseguiram resistir à tentação de se libertarem das restrições burguesas. Traçando paralelos com o declínio do Império Romano, McCarthy considerava a homossexualidade uma degeneração cultural que só poderia enfraquecer os Estados Unidos. Foi, como disse um tablóide, “a bomba atômica de Stalin”.{\color{blue}4}
 \par 
Como pôde uma nação que enfrenta tantas ameaças estrangeiras permitir-se ficar tão distraída? (Esta não é uma questão apenas para historiadores: ao longo da primeira década do século XXI, enquanto o
 \par 
Os Estados Unidos estavam supostamente enfrentando uma ameaça à sua própria existência, os militares dos EUA dedicaram uma energia considerável para expurgar os seus militares gays e lésbicas. Em 2009, os militares prenderam pelo menos sessenta falantes de árabe por serem gays. {\color{blue}5} Um caso foi descoberto depois de os investigadores perguntarem a um soldado se ele alguma vez tinha participado num teatro comunitário.) Com os soviéticos na posse da bomba e a Coreia em marcha, porque é que o Secretário de Estado Dean Ache-son foi enviado ao Congresso para defender a sua heterossexualidade? e a dos seus “diplomatas de pólvora”? {\color{blue}6} Ele não tinha coisas mais importantes para fazer do que organizar reuniões turbulentas de políticos e jornalistas que estavam acontecendo?
 \par 
Uma reminiscência de “despedidas de solteiro”, com grandes quantidades de uísque e bourbon, e mulheres sorridentes “cuja identidade permaneceu desconhecida”. Como observou um senador: “Isso me lembrou um pouco da temporada de corridas da fraternidade na faculdade”. Dean Acheson tentou aparecer como “um dos meninos”, dando tapinhas nas costas dos senadores. Um jornalista relatou que “seu cabelo estava despenteado e a gravata torta. A maneira e o discurso rígidos e precisos que antagonizaram muitos de nós desapareceram. Ele até parecia ter removido a cera do bigode.”{\color{blue}7}
 \par 
The Lavender Scare oferece uma parábola instrutiva sobre o proverbial equilíbrio entre liberdade e segurança, que tanto nos incomoda hoje. Sugere que não só raramente alcançamos o equilíbrio certo entre liberdade e segurança, mas que a própria metáfora do equilíbrio pode ser profundamente falha.
 \par 
O primeiro problema com a metáfora do equilíbrio entre liberdade e segurança é a sua suposição de que a segurança é um conceito transparente, imaculado pela ideologia e pelo interesse próprio. Porque a segurança beneficia a todos – “o mais vital de todos os interesses”, John Stuart Mill
 \par 
Chamou-o de algo que ninguém pode “possivelmente prescindir” – é imune à política. {\color{blue}8} No entanto, como Arnold Wolfers escreveu há anos, a segurança é um “símbolo ambíguo”, que “pode não ter qualquer significado preciso”. {\color{blue}9} Sob a bandeira de um valor aparentemente neutro e universal, as elites políticas são autorizadas, e na verdade encorajadas, a seguir linhas de acção partidárias e ideológicas que normalmente achariam difíceis de justificar. As ações do governo dos EUA durante a guerra contra o terrorismo confirmam esta afirmação. De acordo com duas comissões oficiais, uma das razões pelas quais as agências de inteligência dos EUA não anteciparam o {\color{blue}11} de Setembro foi que as guerras territoriais as impediram de partilhar informações. Os “obstáculos à partilha de informações eram mais burocráticos do que legais” e tinham pouco a ver “com os princípios constitucionais do devido processo, responsabilização ou controlos e equilíbrios”. {\color{blue}10} Mas embora o governo ignore os princípios constitucionais, pouco fez para remover estes obstáculos burocráticos. Até o Departamento de Segurança Interna, que deveria unir agências concorrentes, “está atolado na burocracia” e na “falta de planeamento estratégico”, segundo uma reportagem.{\color{blue}11}
 \par 
Na comunidade antiterrorista, para citar outro exemplo, é amplamente reconhecido que a prisão preventiva e a detenção preventiva de suspeitos de terrorismo frustram a recolha de informações. No entanto, desde o {\color{blue}11} de Setembro, o governo dos Estados Unidos tem confiado consistentemente em tais políticas. Nos dois anos que se seguiram ao {\color{blue}11} de Setembro, as autoridades federais prenderam preventivamente mais de {\color{blue}5}.{\color{blue}000} cidadãos estrangeiros. Em 2006, nenhum desses indivíduos foi “condenado por qualquer crime terrorista”.{\color{blue}12}
 \par 
O padrão é claro: não são tomadas medidas que melhorariam a segurança, enquanto as medidas que são tomadas não conseguem melhorar a segurança ou minam-na. Existem várias explicações para este paradoxo, incluindo os interesses míopes da burocracia dos serviços de informação. Mas um factor-chave é que os conservadores vêem a segurança nacional através das lentes da sua Kulturkampf em curso contra a década de 1960.
 \par 
Esta crença influencia as políticas republicanas, como vimos durante os anos Bush, mas também afecta os democratas, que estão perenemente na defensiva contra a acusação de serem insuficientemente agressivos. Consideremos a carreira de John Ashcroft, o primeiro procurador-geral de Bush, que ajudou a conceber muitas das medidas draconianas da guerra ao terrorismo. Como procurador-geral no Missouri, Ashcroft quase foi citado por desacato – o que geralmente não é uma boa jogada de carreira na política americana – por lutar contra a dessegregação de escolas ordenada pelo tribunal em St. Como senador, ele recebeu um diploma honorário da Universidade Bob Jones, que proibiu o namoro inter-racial, e deu uma entrevista amigável à Southern Partisan, uma revista simpática à Antiga Confederação. Como os reis bíblicos, seu pai ungiu sua cabeça com óleo quando se tornou governador e depois senador. Após a morte de seu pai, ele fez com que Clarence Thomas fizesse as honras quando Bush o nomeou procurador-geral. Convencido de que os gatos malhados eram sinais do diabo, ele teria feito sua equipe garantir que o Tribunal Internacional de Haia não tivesse nenhum em suas instalações.{\color{blue}13}
 \par 
As noções peculiares de Ashcroft refletem o descontentamento mais amplo de seu partido com a cultura política legada ou imposta aos Estados Unidos nas décadas de 1960 e 1970. Durante esses anos, liberais e esquerdistas não apenas derrubaram hierarquias raciais e de gênero legalizadas; eles também tentaram controlar o aparato de segurança. Eles limitaram o poder executivo, defenderam um judiciário ativista, aumentaram os direitos de dissidentes e criminosos e separaram a aplicação da lei da coleta de inteligência. Embora essas reformas tenham durado pouco — elas foram significativamente prejudicadas por Reagan e Clinton — o legado legal da década de 1960 passou a representar a cultura maior de liberdade que os conservadores detestavam e os liberais amavam há anos.
 \par 
Os conservadores gostam de evitar qualquer conversa sobre as “causas profundas” do terrorismo, mas quando se trata do liberalismo decadente que tem
 \par 
Supostamente prejudicados a capacidade do governo de combater os malfeitores no país e no estrangeiro, estão dispostos a abrir uma excepção. Os direitos constitucionais, insistiu Ashcroft depois do {\color{blue}11} de Setembro, são “armas para matar americanos”. Os terroristas “exploram a nossa abertura”. De acordo com o senador republicano Orrin Hatch, os terroristas “nada mais gostariam do que a oportunidade de usar todas as nossas proteções tradicionais do devido processo para prolongar o processo”. {\color{blue}14} Para os conservadores, o {\color{blue}11} de Setembro foi um julgamento estrondoso sobre trinta anos de traição – como se os ataques ao Pentágono e ao World Trade Center tivessem sido causados ​​não pela Al-Qaeda, mas pela leitura dos criminosos dos seus direitos Miranda – e uma oportunidade de ouro para avançar. na direcção oposta: expandir o poder da presidência à custa do Congresso e dos tribunais, e confundir os limites entre a recolha de informações, a vigilância política e a aplicação da lei.{\color{blue}15}
 \par 
Esta sinergia entre a segurança nacional e a ansiedade conservadora não é nova. O Lavender Scare refletiu uma reação geral contra o afrouxamento dos costumes sexuais e dos papéis de gênero que resultou do New Deal e da Segunda Guerra Mundial. O estado de bem-estar social de Roosevelt, argumentavam os conservadores, minou a energia e o vigor patriarcal da nação. Em vez de maridos robustos e pais firmes controlando as suas esposas e filhos, burocratas balbuciantes e assistentes sociais comandavam agora o espectáculo. A Segunda Guerra Mundial exacerbou o problema: com tantos homens na frente de batalha e mulheres a trabalhar nas fábricas, a autoridade masculina foi ainda mais desgastada. Citando estas “convulsões sociais e familiares”, J. Edgar Hoover argumentou que “o espírito de abandono e de ‘vale tudo’ dos tempos de guerra levou a um declínio da moral entre pessoas de todas as idades”.{\color{blue}16}
 \par 
Washington foi o centro desta Revolução Cultural. Uma cidade em expansão para jovens solteiros nas décadas de 1930 e 1940, tinha um mercado imobiliário apertado, forçando os homens a dormirem com outros homens e dando às mulheres muitas oportunidades de se sustentarem através de empregos públicos. O que acontece com os sites de cruzeiros anônimos de
 \par 
Lafayette Park (mesmo em frente à Casa Branca) e na companhia de colegas tolerantes da burocracia federal, os homossexuais conseguiram transformar Washington numa “cidade muito gay”. Hoover cresceu em Washington, quando este era um remanso racista do Velho Sul, e apesar da sua sexualidade ambígua, não ficou satisfeito com estas mudanças.{\color{blue}17}
 \par 
Após a guerra, os conservadores provocaram pânico sobre os papéis de género. “Uma grande ênfase”, segundo Cheever, “a título de defesa, foi dada à masculinidade, ao atletismo, à caça, à pesca e às roupas conservadoras, mas a esposa solitária se perguntou, olhando, sobre o marido em seu acampamento de caça, e o marido se perguntou com quem dividiu um tosco leito de pinheiros. Foi ele? Ele tinha? Ele queria? Ele alguma vez teve? Ao gerar esse pânico, os conservadores habilmente viraram o público contra um governo determinado a tornar todos gays. O New Deal, alegavam, era um Queer Deal; A América era governada por “fadas e negociantes justos”. {\color{blue}18} Devido a esta união ímpia de Democratas, Comunistas e bichas, os Estados Unidos estavam agora vulneráveis ​​à União Soviética.
 \par 
Os conservadores de hoje acreditam que décadas de reforma doméstica, impulsionadas desta vez por uma ternura excessiva sobre a Constituição, criaram uma sociedade desvitalizada que não tem a vontade e os meios para enfrentar ameaças estrangeiras. É por isso que Bush prometeu depois do {\color{blue}11} de setembro que não haveria "nenhuma rendição. Nenhuma equívoco. Nenhuma camada dessa coisa até a morte". É também por isso que Ashcroft se irritou com a noção de que o governo dos EUA deveria ler à Al-Qaeda "os direitos Miranda, contratar um advogado de defesa extravagante, trazê-los de volta aos Estados Unidos para criar uma nova rede a cabo da Osama TV". {\color{blue}19} Não está claro quem, se alguém, estava recomendando tal política, mas o fato de Ashcroft se sentir compelido a denunciá-la dá uma indicação do que ele encontra em questão quando fala sobre segurança. Os conservadores certamente acreditam que o Patriot Act e outras restrições às liberdades civis protegerão os americanos.
 \par 
Pessoas – se é do terrorismo que elas estão sendo protegidas é outra questão.
 \par 
Há um segundo problema com a noção de equilíbrio entre liberdade e segurança. Desde que a guerra se tornou uma questão de povos e não de reis, a bússola da segurança expandiu-se constantemente para além dos quartéis e do alto comando das forças armadas. Frederico II travou a guerra, escreveu Lukács, “de tal maneira que a população civil simplesmente não a notaria”. A guerra moderna insinua-se “na vida interior de uma nação”. {\color{blue}20} Requer a mobilização total dos recursos de um país e o apoio activo dos seus cidadãos. A limitação da liberdade nas partes mais remotas da sociedade pode assim ser justificada como um acto legítimo de defesa nacional. Pode-se encontrar um perigo claro e presente na economia política da nação, nas suas escolas e na cultura popular, até mesmo nas suas camas, e decidir suprimir a liberdade ali, a fim de evitar a ameaça. Quando liberais e conservadores afirmam a prioridade da segurança sobre a liberdade em tempos de guerra, não estão apenas a endossar as restrições governamentais ao que a imprensa noticia sobre os militares; estão também a autorizar a supressão de todo o tipo de dissidência, em toda a ordem social.
 \par 
Consideremos a vigilância do tráfego de telecomunicações nos Estados Unidos pela Agência de Segurança Nacional (NSA), que foi noticiada pela primeira vez no New York Times em 2005. Como escreve James Risen, que ajudou a divulgar a história, a NSA é “a maior organização no Comunidade de inteligência dos Estados Unidos, o dobro do tamanho da CIA e verdadeiramente o serviço de espionagem electrónica dominante no mundo.” Graças a uma ordem secreta emitida por Bush em 2002, “está agora a escutar até quinhentas pessoas nos Estados Unidos em qualquer momento e tem potencialmente acesso a chamadas telefónicas e e-mails de milhões de pessoas. Fá-lo sem mandados de busca aprovados pelo tribunal e com pouca supervisão independente.”{\color{blue}21}
 \par 
A justificação da administração Bush para este programa, que pode ser “a maior operação de espionagem doméstica desde a década de 1960”, é que, para monitorizar o tráfego internacional entre terroristas, deve explorar a rede doméstica. “Os switches transportam ligações de Cleveland para Chicago. . . Também pode transportar chamadas de Islamabad para Jacarta”, pelo que “agora é difícil dizer onde termina o sistema telefónico doméstico e começa a rede internacional”. A administração autorizou a NSA a trabalhar secretamente com empresas de telecomunicações para espionar este tráfego internacional e encorajou-as a encaminhar mais tráfego através dos Estados Unidos. Se ainda não o fizeram, a NSA e os seus ajudantes na indústria poderão em breve estar a espiar não só a América, mas também a Europa e a Ásia.{\color{blue}22}
 \par 
A expansão do domínio da segurança em todas as áreas da sociedade faz mais do que restringir a liberdade em abstrato: também fortalece as forças conservadoras da repressão política. Os conservadores influentes argumentam que a unidade nacional é uma arma de guerra essencial, que a oposição mina o esforço de guerra e que os dissidentes são perigosos, subversivos ou traidores. Após a vitória do candidato anti-guerra Ned Lamont sobre Joe Lieberman nas primárias democratas para o Senado em Connecticut em 2006, o vice-presidente Cheney declarou que a eleição de Lamont apenas encorajaria “os tipos da Al-Qaeda”, que estavam “apostando na proposição de que, em última análise, eles podem quebrar a vontade de o povo americano.”{\color{blue}23}
 \par 
A Lei Patriota, aprovada pelo Congresso seis semanas após o {\color{blue}11} de Setembro, leva a equação da dissidência com a subversão um passo mais longe, sugerindo que os opositores da guerra ao terrorismo não estão apenas a ajudar os terroristas, mas podem ser eles próprios terroristas. A Seção {\color{blue}802} da Lei define “terrorismo doméstico” como “atos perigosos para a vida humana que constituem uma violação das leis penais” e que “parecem ser intencionais. . . Influenciar a política de um governo por meio de intimidação ou coerção.” {\color{blue}24} Uma definição tão ampla e vaga como esta poderia facilmente ser usada
 \par 
Contra manifestantes que marcham sem autorização (um protesto pode impossibilitar a passagem de ambulâncias ou outros veículos de emergência). {\color{blue}25} Depois de manifestantes anti-guerra terem causado perturbações em Portland, Oregon, no Outono de 2002, os legisladores estaduais redigiram uma lei antiterrorista nesse sentido. Eles definiram o terrorismo como, entre outras coisas, qualquer ato destinado “por pelo menos um dos seus participantes” a perturbar “o comércio ou os sistemas de transporte do Estado de Oregon”.{\color{blue}26}
 \par 
Durante a Convenção Nacional Republicana, em Setembro de 2004, o Departamento de Polícia da Cidade de Nova Iorque prendeu {\color{blue}1}.{\color{blue}800} manifestantes anti-guerra sob várias acusações, a maioria das quais foram posteriormente rejeitadas pelos tribunais. Justificando estas detenções, o presidente da cidade, Michael Bloomberg, disse: “Algumas pessoas pensam que não deveríamos permitir que as pessoas se expressem. Foi exactamente isso que os terroristas fizeram, se pensarmos bem, no {\color{blue}11} de Setembro. Agora, este não é o mesmo tipo de terrorismo, mas não há dúvida de que estes anarquistas têm medo de deixar as pessoas falarem.”{\color{blue}27}
 \par 
Dado que a guerra mobiliza todas as esferas da sociedade, os defensores da ordem social afirmam que qualquer perturbação dessa ordem – proveniente, por exemplo, da greve dos sindicatos – é tão ameaçadora para o esforço de guerra como a oposição à própria guerra. Foi com base nestes fundamentos que, em 1950, o Supremo Tribunal confirmou a negação, por parte do governo federal, da protecção laboral aos sindicatos liderados pelos comunistas. Estes líderes sindicais, argumentou o tribunal, podem usar as suas posições de poder “num momento de crise externa ou interna” para convocar “greves políticas” e perturbar os canais de comércio. {\color{blue}28} Em Janeiro de 2003, o gabinete de Tom DeLay, então líder da maioria na Câmara, enviou uma carta de angariação de fundos aos apoiantes da Fundação Nacional pelo Direito ao Trabalho, um grupo empresarial que procurava livrar a América dos sindicatos. Alegando que o movimento trabalhista “apresenta um perigo claro e presente para a segurança dos Estados Unidos em casa e para a segurança de nossas Forças Armadas no exterior”, a carta denunciava “Grandes Chefes Trabalhistas. . . Disposto a prejudicar
 \par 
Trabalhadores amantes da liberdade, o esforço de guerra e a economia para adquirir mais poder!”{\color{blue}29}
 \par 
Os republicanos no Congresso também trabalharam em estreita colaboração com Bush para negar direitos sindicais e protecção a denunciantes a {\color{blue}170} mil funcionários do Departamento de Segurança Interna. Embora muitos deles sejam trabalhadores administrativos, e embora não sejam negados estes direitos aos funcionários do Departamento de Defesa, a administração alegou que a eliminação destes direitos e protecções tornaria o departamento tão “ágil e agressivo como os próprios terroristas”. Depois que o Congresso aprovou o projeto de lei antissindical em novembro de 2002, um funcionário da Casa Branca declarou que ele seria um modelo para todos os funcionários federais. {\color{blue}30} A natureza expansiva da segurança autoriza o governo não só a utilizar estas armas, mas também a partilhá-las com empregadores privados, que muitas vezes estão em melhor posição para usá-las e abusar delas. Como os empregadores não estão sujeitos às restrições da Primeira Emenda, são geralmente livres de usar os seus poderes de contratação e de recrutamento, de promoção e de despromoção, para silenciar a dissidência. Durante os anos McCarthy, por exemplo, o governo prendeu menos de duzentos homens e mulheres por razões políticas. Mas algo entre {\color{blue}20} e {\color{blue}40} por cento da força de trabalho foi monitorizada em busca de sinais de inconformidade ideológica, que incluíam o apoio aos direitos civis e aos sindicatos.{\color{blue}31}
 \par 
Os efeitos desta externalização da repressão são particularmente visíveis nos meios de comunicação social, pois os meios de comunicação social dos EUA praticam uma forma de censura que deve ser a inveja dos tiranos em todo o mundo. Sem que o governo mexa um dedo, a pressão informal e o carreirismo na redação são suficientes para fazer com que os repórteres sigam os limites. O ex-âncora de notícias da CBS, Dan Rather, afirma que os conservadores estão “em todos os seus telefones, em todos os seus e-mails”. Como resultado, “você diz para si mesmo: ‘Sabe, acho que estamos certos nessa história. Acho que entendemos isso no contexto certo, acho que entendemos isso na perspectiva certa, mas é melhor escolhermos outro dia.’” {\color{blue}32} Aqueles que estão na base entendem
 \par 
Mensagem rápida. O repórter de televisão Sam Donaldson, que cobriu a Casa Branca durante os anos Reagan, disse a Eric Boehlert:
 \par 
Hoje, nem todos os patrões apoiam os seus repórteres. Então, se você é um repórter na Casa Branca e está pensando em mais sucesso nos negócios, e está nervoso com a possibilidade de seu chefe receber uma ligação, talvez você tenha cuidado por causa do plano de carreira.{\color{blue}33}
 \par 
Os jornalistas que temem pelas suas carreiras provavelmente não questionarão o seu governo em tempos de guerra. E eles não fizeram isso. Ted Koppel, da ABC, um dos entrevistadores mais agressivos do ramo, admite que “éramos demasiado tímidos antes da guerra” no Iraque. O âncora da PBS Jim Lehrer diz: “Teria sido difícil ter debates [sobre a ocupação do Iraque]. . . Você teria que ter ido contra a corrente.” Os poucos jornalistas que contrariaram a tendência foram rapidamente punidos. Depois de criticar a mídia pela cobertura da guerra, Ashleigh Banfifield foi “levada para o depósito de lenha” por seus chefes, de acordo com uma reportagem do Newsday, e sua carreira na NBC foi encerrada. Uma repórter do Wall Street Journal enviou um e-mail pessoal descrevendo a terrível situação no Iraque: os seus editores tiraram-na do país e da história.{\color{blue}34}
 \par 
O último problema com a noção de equilíbrio entre liberdade e segurança é que ela assume erroneamente que os benefícios e encargos da liberdade e segurança serão distribuídos igualmente entre todos os membros da sociedade. Mas são sempre alguns membros da sociedade, muitas vezes os mais marginalizados e desprezados – gays e esquerdistas durante a Guerra Fria, árabes e muçulmanos (e ainda hoje gays e esquerdistas, embora em menor grau) – que são forçados a abdicar das suas liberdades. para que os demais possam desfrutar de sua segurança. Na verdade, é precisamente porque estes grupos são impotentes, e não porque sejam
 \par 
É perigoso que os poderosos possam exigir que eles suportem os custos. (Embora 2% dos homens americanos com idades compreendidas entre os {\color{blue}18} e os {\color{blue}21} anos sejam presos por conduzirem embriagados, o Supremo Tribunal decidiu que este facto não justifica negar aos homens dessa idade o direito de comprar álcool. Muito menos de 2% dos árabes e muçulmanos em os Estados Unidos estão envolvidos em actividades terroristas, mas o governo dos EUA negou a estes grupos muitos mais direitos fundamentais.) {\color{blue}35} O que a metáfora do equilíbrio entre liberdade e segurança esconde é o desequilíbrio fundamental de poder entre grupos na sociedade; custos desiguais são pagos em troca de ganhos desiguais.
 \par 
Em No Equal Justice (1999), David Cole transformou um lugar-comum — que americanos brancos e/ou ricos recebem melhor tratamento dos policiais e tribunais do que cidadãos negros e/ou pobres — em uma teorização surpreendente de um sistema de justiça dual na América. Conceder direitos máximos a todos os cidadãos teria um alto custo em termos de segurança, ele observou, enquanto negar esses direitos teria um alto custo em termos de liberdade. Então, o que a América faz? Ela faz as duas coisas: concede formalmente direitos a todos, mas os nega sistematicamente aos negros e aos pobres. A América branca e rica obtém liberdade máxima e segurança máxima, e "ignora a difícil questão de quanta proteção constitucional poderíamos pagar se estivéssemos dispostos a garantir que ela fosse desfrutada igualmente por todas as pessoas".{\color{blue}36}
 \par 
Em Enemy Aliens and Terrorism and the Constitution, Cole estende esse argumento a não-cidadãos em tempo de guerra. Desde a Lei de Estrangeiros de 1798, o primeiro impulso da América quando confrontados com uma ameaça estrangeira tem sido restringir os direitos dos imigrantes. A atracção de tais medidas é semelhante à atracção do sistema dual de justiça criminal. É uma “forma politicamente tentadora de mediar a tensão entre liberdade e segurança. Os cidadãos não precisam de renunciar aos seus direitos” para serem — ou sentirem-se — protegidos. Os não-cidadãos renunciam às suas, e porque “não têm voz directa no processo democrático através da qual possam registar as suas objecções”, poucas pessoas se queixam.{\color{blue}37}
 \par 
Depois do {\color{blue}11} de Setembro, medidas de segurança que teriam afectado todos os cidadãos – como a Operação TIPS, na qual funcionários de serviços públicos, entregadores e outros indivíduos espionavam os seus concidadãos, ou o programa Total Information Awareness do Pentágono, um projecto de vigilância massivo de registros de computadores públicos e privados – foram rapidamente bloqueados, até mesmo por líderes republicanos. Mas as medidas que afectam os não-cidadãos, especialmente os muçulmanos e os árabes, receberam um apoio público esmagador. Talvez seja por isso que, um ano depois do {\color{blue}11} de Setembro, apenas 7% dos americanos acreditavam ter sacrificado direitos e liberdades fundamentais.{\color{blue}38}
 \par 
Mas há uma diferença entre o tratamento dispensado aos estrangeiros em tempos de guerra e o tratamento dispensado aos negros e aos pobres em tempos de paz. As medidas de guerra infligidas a não-cidadãos eventualmente influenciam as medidas contra os cidadãos dos EUA, especialmente liberais ou progressistas. Em 1942, o governo federal colocou japoneses não-cidadãos e nipo-americanos em campos de internamento (supondo que, mesmo que fossem cidadãos, a sua herança racial os tornava estrangeiros). Vários anos mais tarde, o FBI compilou uma lista secreta de {\color{blue}12}.{\color{blue}000} cidadãos a serem detidos em caso de emergência nacional – uma iniciativa aprovada em 1950 pela aprovação da Lei de Segurança Interna, que permaneceu em vigor até 1971.{\color{blue}39} Se um uma mutação semelhante ocorrerá na guerra contra o terrorismo, ninguém sabe, mas as evidências até agora não são encorajadoras.
 \par 
O que uma análise mais completa da metáfora revela é que os itens equilibrados na escala não são liberdade e segurança, mas poder e impotência. Faz, portanto, todo o sentido que os conservadores utilizem a metáfora, pois ela esconde e protege o seu eleitorado natural. A verdadeira questão é: por que os liberais os obrigam?
 \par 
Talvez seja porque foram os liberais que inventaram o argumento. Foram os liberais que primeiro argumentaram que os indivíduos deveriam ser livres para dizer e fazer o que quisessem, desde que não prejudicassem ninguém.
 \par 
Outro. As democracias liberais deveriam usar a coerção apenas para punir atos ou tentativas de atos prejudiciais, incluindo ameaças à segurança da nação. Podem-se ver variantes deste argumento na explicação de Locke sobre a tolerância religiosa, que só poderia ser sacrificada pela “segurança e protecção da comunidade”; a teoria da liberdade de Mill, que só poderia ser limitada para evitar danos; e a defesa da liberdade de expressão por Oliver Wendell Holmes, que só poderia ser resumida para frustrar “um perigo claro e presente”.{\color{blue}40}
 \par 
O problema com estes argumentos é que é quase impossível definir o dano – ou o perigo, a ameaça, a ameaça – de uma forma neutra. Cada definição de dano e os seus cognatos de segurança nacional baseiam-se em pressupostos ideológicos sobre a natureza humana, a moralidade e a boa vida. E neste aspecto, os liberais são tão culpados quanto os conservadores. A única diferença é que muitas vezes têm menos poder para agir de acordo com as suas convicções – e para impedir que os seus oponentes ajam de acordo com as suas.
 \par 
Como nota de rodapé filosófica ao Lavender Scare, podemos recordar que, no preciso momento em que os Estados Unidos conduziam a sua purga de gays e lésbicas, dois ingleses – o jurista conservador Patrick Devlin e o filósofo liberal H. L. A. Hart – estavam envolvidos num debate de surpreendente relevância. para eventos do outro lado da água. Tudo começou em 1957, quando o Comité Wolfenden no Reino Unido recomendou, entre outras coisas, que o sexo gay entre adultos consentidos em privado fosse descriminalizado. Falando na Academia Britânica em março de 1959, Devlin refreou a afirmação do comitê de que existe “um domínio de moralidade e imoralidade privadas que, em termos breves e grosseiros, não é da conta da lei” e que apenas atos concretos de lesão ou dano deveriam ser processado e punido por lei. Não é assim, disse Devlin: “O que constitui qualquer tipo de sociedade é a comunidade de ideias, não apenas ideias políticas, mas também ideias sobre a forma como os seus membros devem comportar-se e governar as suas vidas”. Qualquer desafio a essas ideias – não importa quão
 \par 
Privados, incidentais ou simbólicos – minavam a coesão social e representavam uma ameaça tão grande à ordem cívica quanto a traição. Da mesma forma que a traição poderia levar à derrubada de um governo, a homossexualidade poderia produzir um “afrouxamento dos laços morais”, que “é muitas vezes o primeiro estágio de desintegração”. Assim, “a supressão do vício é tanto assunto da lei quanto a supressão de atividades subversivas”.{\color{blue}41}
 \par 
A resposta de Hart foi rápida – ele foi ao ar em julho, proferindo uma palestra na Rádio BBC que foi posteriormente publicada no The Listener – e furiosa. {\color{blue}42} “É grotesco”, declarou ele, “pensar no comportamento homossexual de dois adultos em privado como de alguma forma semelhante a traição ou sedição”. Não apenas grotesco, mas obtuso: Devlin assumiu erroneamente “que o desvio de um código moral geral está fadado a afetar esse código e a levar não apenas à sua modificação, mas à sua destruição”. Se os actos privados de um homem alterassem as crenças de uma sociedade – um grande se, insistiu Hart – tal mudança não constituiria um colapso, mas uma transformação da moralidade social. O análogo político adequado ao sexo gay, então, não era a traição, mas “uma mudança pacífica” numa forma de governo.{\color{blue}43}
 \par 
Os críticos tendem a pensar que Hart levou a melhor sobre Devlin. Mas eu me pergunto. Afinal, Hart nunca definiu o dano com precisão ou persuasão, e não está claro se ele poderia ter feito isso. Então, o que impediu Devlin de afirmar que a homossexualidade era tão prejudicial quanto a traição – ou, como afirmaram os seus homólogos americanos, que a homossexualidade era traição? Muito pouco, ao que parece, seja política ou filosoficamente. Pois quando o mal surge em tons de cinza, alguém, em algum lugar, inevitavelmente o verá em lilás e rosa – ou em qualquer outra cor desfavorecida do arco-íris.