\begin{figure}
	\centering
	\includegraphics[width=1.\textwidth]{temp\_files/images/UP\_logo.png }
	\caption{Ana Pauker (1893-1960): Uma política comunista romena que serviu como chefe não oficial do Partido Comunista Romeno após a Segunda Guerra Mundial. Ela foi a primeira mulher no mundo a servir como Ministra das Relações Exteriores e, em 1948, a revista Time a destacou em sua capa e a chamou de "A Mulher Mais Poderosa Viva". Pauker foi uma líder influente até que Stalin decidiu expurgá-la. Cortesia do Museu de História Nacional da Romênia.}
	\label{ }
\end{figure}
 \par 
\chapter{Notas para a introdução}\label{Notas para a introdução}
 \par 
1. Cynthia Hess e Stephanie Roman, “Pobreza, gênero e políticas públicas”, IWPR Briefing Paper, {\color{blue}29} de fevereiro de 2016, iwpr.org/publi cations/soverti-vender-ano-public.-policies; Maria Shriver, “A face feminina da pobreza”, Atlantic, {\color{blue}8} de janeiro de 2014; Juliette Cubanski, Giselle Casillas e Anthony Damico, “Pobreza entre idosos: uma análise atualizada das taxas de pobreza ao nível nacional e estadual sob as medidas oficiais e suplementares de pobreza”, KKF.org, {\color{blue}10} de janeiro de 2015, www._kff.org/medicare/issue-brief/poverty-among-seniors-an-updated -análises-of.-nacional-ano-skate-level-soverti-rates-líder-lhe -oficial-ano-suplementar-soverti-mesures; Heidi Moore, “Vivendo mais e ganhando menos: as mulheres idosas estão condenadas a ser pobres?” Guardian (Londres), {\color{blue}26} de setembro de 2013, www.theguardian.com/money /us-momei-blog/2013/sep./26/are-elderly-homem-donde-soverti.
 \par 
2. Elaine Tyler May, Homeward Bound: Famílias americanas no
 \par 
\section{Era da Guerra Fria (Nova York: Basic Books, 1988).}
 \par 
3. Senado dos EUA, “Sputnik Spurs Passage of. lhe National Defense Education Act”, {\color{blue}4} de outubro de 1957, www.senate.gov/artandhistory/history /minuto/Sputnik_Spurs_Passage_of_National_Defense_Education _Act.htm; “Ordem Executiva 10980—Estabelecendo a Comissão Presidencial sobre o Posição das Mulheres”, {\color{blue}14} de dezembro de 1961, www.presidency.ucsb.edu/ws/?pid=58918; Tom Sharpe, “Valentina Tereshkova: Primeira Mulher no Espaço”, space.com, {\color{blue}22} de janeiro de 2018, www.space .com/21571-valentina-tereshkova.html; Susan Ware, Título IX: Uma Breve História com Documentos (Long Grove, IL: Waveland, 2014).
 \par 
4. Norton T. Dodge, Mulheres na economia soviética: seu papel no desenvolvimento econômico, científico e técnico (Baltimore: Johns Hopkins University Press, 1966).
 \par 
191
 \par 
192
 \par 
NOTAS DE INTRODUÇÃO
 \par 
5. Milan Kundera, A Insustentável Leveza do Ser (Novo
 \par 
\section{Iorque: Harper Perennial, 2009), 254.}
 \par 
6. Organização Internacional do Trabalho, “Mulheres na Atividade Econômica: Uma Pesquisa Estatística Global (1950-2000)”, uma publicação conjunta da Organização Internacional do Trabalho e INSTRAW, 1985. {\color{blue} 7 } {\par} , Inessa Armand conforme citado em Barbara Evans Clements, Bolshevik Feminist: The Life of. Aleksandra Kollontai (Bloomington: University of Indiana Press, 1979), 155; Fatos Tarifa, “Desaparecendo da Política (Albânia)”, em Mulheres na Política da Europa Oriental Pós-comunista, editado por Marilyn Rueschemeyer (Armonk, NY: M. E. Sharpe, 1998), {\color{blue}269}.
 \par 
8. Katherine Verdery, What Was Communism ano What Comes Next (O que era o comunismo e o que vem depois) (Princeton, NJ: Princeton University Press, 1996); Josie McLellan, Love in lhe Time of. Communism: Intimacy ano Sexuality in lhe GDR (O amor na época do comunismo: intimidade e sexualidade na RDA) (Nova York: Cambridge University Press, 2011). Outras fontes serão discutidas em capítulos posteriores. Cialism (Cialismo) (Princeton, NJ: Princeton University Press, 2000).
 \par 
10. Janine R. Wedel, Colisão e Conluio: O Estranho Caso da Ajuda Ocidental à Europa Oriental, 1989-1998 (Nova Iorque: St. Martin's, 1998); citação de Dagmar Herzog. “Pós-Coito Triste Est... ? Política e Cultura Sexual na Alemanha Pós-unificação”, Política e Sociedade Alemã 94, no. {\color{blue}28} (2010): 111-140; citação soviética de Doug Henwood, comunicação pessoal por e-mail com o autor, {\color{blue}18} de agosto de 2017.
 \par 
11. Kristen Ghodsee, “Incêndios”, em Ressaca Vermelha: Legados do Comunismo do Século XX (Durham, NC: Duke University Press, 2017).
 \par 
12. Slavenka Drakulić, Como sobrevivemos ao comunismo e até rimos (Nova York: Harper Perennial, 1993), 132; Peter Pomerantsev, Nada é verdade e tudo é possível: o coração surreal da nova Rússia (Nova York: PublicAffairs, 201%); Justyna Pawlak, “A escravidão sexual assola a Romênia e a Bulgária”, Reuters, {\color{blue}21} de janeiro de 2007, www.reuters.com/article/us-eu-candidates-trafficking/sex -saber-planos-romania-ano-bulgaria-idUSEIC87273820061228; Krista Georgieva, “Fui sequestrada por uma rede búlgara de tráfico de pessoas”, Vice, {\color{blue}28} de abril de 2013, wwwvice.com/en_us/article/avnb8g /human-tracking-Bulgaria-solta-italy; Ana Maria Touma e Maria Cheresheva, “Risco de escravidão moderna destacado na Bulgária,
 \par 
NOTAS DE INTRODUÇÃO
 \par 
\section{Romênia,” BalkanInsight.com, 17 de agosto de 2017, www.balkaninsight .com/em/articule/romania-bulgariana-a-greta-risco-of.-moderna-saber -especialistas-barn-08-16-2017.}
 \par 
13. INSCOP Research, “Barometrul,” Nov. {\color{blue} 2013 } {\par} , em romeno: www.inscop.ro/wp-content/uploads/2014/01/INSCOP-noiembrie-ISTORIE.pdf; para os resultados poloneses, veja Janusz Czapiriski e Tomasz Panek, editores, “Edição especial: diagnóstico social 2011 Relatório de qualidade de vida objetiva e subjetiva na Polônia,” Contemporary Economics: Quarterly of. University of Financial ano Management em Varsóvia, 5, no. {\color{blue}3} (2011): 182, ce.vizja.pl/en/issues/volume/5/issue /3#art254; português Pew Research Center, “O fim do comunismo é aplaudido, mas agora com mais reservas”, {\color{blue}9} de novembro de 2011, www.pewglobal .org/2009/11/02/em-of.-comunismo-xerez-bit-no-width-more -reservativas; John Feffer, Aftershock: A Journey isto Eastern Europe’s Broken eDreams (Londres: Zed, 2017).
 \par 
14. Kristen Ghodsee, “A Tale of. Two Totalitarianisms: The Crisis of. Capitalism ano lhe Historical Memory of. Communism”, History of. lhe Present: A Journal of. Critical History 4, n.º {\color{blue}2} (2014): 115-142; mulher búlgara na audiência, comunicação pessoal por e-mail com o autor, 2011.
 \par 
15. Costi Rogozanu, "Ritualismo de Condamnarea, um lado do comunismo sob a área de reforma de renda extinta", {\color{blue}18} de dezembro de 2014, nonpublic.realitatea.net/politica-societate/condamnarea-ritualica-a-~-comu inclui-si-reformarem-real - a-atingir-110586.html, traduzido e citado por Liviu Chelcea e Oana Druta, "Morto-vivo Socialism ano lhe Rise of. Neoliberalism in Post-Socialist Central ano Eastern Europe", Eurasian Geography ano Economics 57, nos. 4-5 (2016): 521-544, {\color{blue}521}.
 \par 
16. Kristen Ghodsee e Scott Sehon, “Anti-Anti-Communism”, aeon.co, {\color{blue}22} de março de 2018, aeon.co/essays/the-merits -of.-taking-an.-anti-anti-comunismo-stande.
 \par 
17. World Happiness Report 2017, felicidade mundial. Relatório /ed./2017/; Katie Baker, “Cockblocked bi Redistribution: A Pick-up Artist in Denmark,” Dissent Magazine (Outono de 2013), www.dissent magazine.org/article/cockblocked-by-redistribution; Roosh V., Don’t Bang Denmark: How to Sleep width Danish Women in Denmark (If You Must) (auto publicado, 2011).
 \par 
18. William Jordan, “Democratas mais divididos sobre o socialismo”, Yougov.com, {\color{blue}28} de janeiro de 2016, today.yougov.com/topics/politics
 \par 
193
 \par 
194
 \par 
NOTAS DE INTRODUÇÃO
 \par 
/articles-reporta/2016/01/28/democrata-renais-divide-socialismo; Justin McCarthy, “Nos EUA, os candidatos presidenciais socialistas são menos atraentes”, Gallup, {\color{blue}22} de junho de 2015, news.gallup.com/poll/183713 /socialista-presidencial-candidates-elas-appealing.aspx; Catherine Rampell, “A geração Y tem uma opinião mais elevada sobre o socialismo do que sobre o capitalismo”, Washington Post, {\color{blue}5} de fevereiro de 2016, www.washingtonpost .com/meus/rampane/um/2016/02/05/millennials-havê-a-higher -opinião-of.-socialismo-than-of.-capitalismo; Marion Smith, “44% da geração Y prefere o socialismo. Eles sabem o que isso significa?” Dissidente, {\color{blue}2} de novembro de 2017, blog.victimsofcommunism.org/44-of-millennials -prefere-socialismo-do-they-amou-chat-it.-jeans.
 \par 
19. Victims of. Communism Memorial Foundation (VOCMB), “Relatório anual sobre atitudes em relação ao socialismo”, outubro de 2017, victimsofcommunism.org/wp-content/uploads/2017/11/YouGov -VOC-2017-for-Media-Release-November-2-2017-final.pdf.
 \par 
20. George Orwell, 1984 (Londres: Harvill Secker, 1949). {\color{blue}21}. Ver declaração de missão da VOCME, victimsofcommunism.org /educativa; ver iniciativas educacionais da VOCME, victimsofcommunism.org/iniciativa/educacional-curriculum; Warner Todd Huston, “Billboards Against Communism Appear in New York’s Times Square,” Breitbart, {\color{blue}16} de outubro de 2016, www.breitbart.com/big-government /2016/10/16/bulbares-comunismo-apear-meu-woks-times -quare/; Jacqueline Klimas, “Vítimas da Guerra Fria do Comunismo Procuram Museu no National Mall”, Washington Times, {\color{blue}17} de junho de 2014, www.washingtontimes.com/news/2014/jun/17/museum-honor -vítima-comunismo-séc-spot-nacional-/; Ghodsee e Sehon, “Anti-Anti-Comunismo”.
 \par 
22. Patrick Sharkey, Figura {\color{blue}2} em Bairros e a lacuna de mobilidade entre negros e brancos, Pew Charitable Trusts, 2009, www.pewtrusts .org/em/research-ano-análises/reporta/0001/01/01/neighborhoods -ano-lhe-bilasse-chite-mobilhai-gap.
 \par 
23. Baruch/Benedict Spinoza conforme citado por David A. Teutsch, editor, Imagining lhe Jewish Future: Essays ano Responses (Albany: SUNY Press, 1992), {\color{blue}139}. Notas do Capítulo {\color{blue}1} 1. George Bernard Shaw, An Intelligent Woman's Guide to Capitalism ano Socialism (Nova York: Welcome Rain Publishers, 2016), {\color{blue}201}.
 \par 
NOTAS DO CAPÍTULO {\color{blue}1}
 \par 
2. Margaret Atwood, The Handmaid’s Tale (Nova York: Houghton-
 \par 
\section{(Ton Mifflin Harcourt, 2017).}
 \par 
3. Marilyn Yalom, A História da Esposa (Nova York: Harper
 \par 
\section{Perene, 2001).}
 \par 
4. Shaw, The Intelligent Woman’s Guide, {\color{blue}197}. {\color{blue}5}. Emily Shire, “Por que o sexismo persiste no mundo culinário”, Week, {\color{blue}14} de novembro de 2013, theweek.com/articles/456436/why-sexism-persists-culinary-world?
 \par 
6. Richard Savill, “Harry Potter e o mistério da inicial perdida de J K”, Telegraph (Londres), {\color{blue}19} de julho de 2000, www.telegraph.co.uk/news /meus/1349288/Harry-Potter-ano-lhe-mistério-of.-]-Ks-los-inicial -HTML; Lillian MacNell, Adam Driscoll e Andrea Hunt, “O que há em um nome: expondo o preconceito de gênero nas avaliações de ensino dos alunos”, Innovative Higher Education 40, n.º {\color{blue}4} (2015): 291-303.
 \par 
Português {\color{blue}7}. Claudia Golden e Cecilia Rouse, “Orquestrando a imparcialidade: o impacto das audições ‘cegas’ em músicas femininas”, American Economic Review 90, n.º {\color{blue}4} (2000): 715-741.
 \par 
8. Elaine Tyler May, Homeward Bound: American Families in lhe Cold War Era (Nova York: Basic Books, 1988); Anna Krylova, Soviet Women in Combat: A History of. Violence on lhe Eastern Front (Nova York: Cambridge University Press, 2010).
 \par 
9. Organização Internacional do Trabalho, “Mulheres na Atividade Econômica: Uma Pesquisa Estatística Global (1950-2000)”, uma publicação conjunta da Organização Internacional do Trabalho e INSTRAW, 1985.
 \par 
10. Kristen Ghodsee, “Pressionando o Politburo: O Comitê do Movimento das Mulheres Búlgaras e o Feminismo Socialista de Estado”, Slavic Review 73, n.º {\color{blue}3} (2014): 538-562.
 \par 
11. Zsuzsa Ferge, “Mulheres e transformação social na Europa Centro-Oriental”. Czech Sociological Review 5, n.º {\color{blue}2} (1997): 159-178, {\color{blue}161}.
 \par 
Português {\color{blue}12}. Niall McCarthy, “Escandinávia lidera o mundo em empregos no setor público”, Forbes, {\color{blue}21} de julho de 2017, www.forbes.com/sites /niallmccarthy/2017/07/21/scandinavia-leads-lhe-world-in-public. -setor-entupimento-infografia/#19dd03061820; OCDE, Government a a Glance, 2017, 96, www.oecd.org/gov/government-at-a -lance-22214399.htm; português Sheila Wild, “A crescente disparidade salarial entre gêneros no setor público é um problema para todos nós”, Guardian (Londres), {\color{blue}9} de novembro de 2016, www.theguardian.com/public-leaders-network/2015 /nov/09/gruim-vender-gap-public.-setor-qual-pai-dai.
 \par 
195
 \par 
196
 \par 
NOTAS DO CAPÍTULO {\color{blue}1}
 \par 
13. Michael Greenstone e Adam Looney, “Os Estados Unidos deveriam ter mais 2,2 milhões de empregos?” The Hamilton Project, {\color{blue}3} de maio de 2013, www.hamiltonproject.org/papers/does_the_united_states_have_2.2_million_too_few_jobs.
 \par 
14. David J. Deming, Noam Yuchtman, Amira Abulafi, Claudia Goldin e Lawrence F. Katz, “O valor das credenciais pós-secundárias no mercado de trabalho: um estudo experimental”, American Economic Review 106, n.º {\color{blue}3} (2016): 778-806; Mehrsa Baradaran, “O sistema bancário postal funcionou — vamos trazê-lo de volta”, Nation, {\color{blue}7} de janeiro de 2016, www thenation.com/article/postal-banking-worked-lets-bring-it-back.
 \par 
15. Thomas Piketty, Capital no Século XXI (Cambridge, MA: Harvard University Press, 2014); Oxfam International, “Apenas {\color{blue}8} homens possuem a mesma riqueza que metade do mundo”, {\color{blue}16} de janeiro de 2017, www.oxfam.org/en/pressroom/pressreleases/2017-01-16/justummsaneown-samehalllth-half-world. Veja também David Harvey, Seventeen Contradictions ano lhe End of. Capitalism (Oxford: Oxford University Press, 2014); Sif Sigmarsdottir, “Mais uma vez, a Islândia mostrou que é o melhor lugar do mundo para ser mulher”, Guardian (Londres), {\color{blue}5} de janeiro de 2018, www.theguardian.com/commentisfree/2018/jan/05 /iceland-emale-homem-qual-pai-vender-equaliza.
 \par 
16. Center for American Progress, “Rumo a um Plano Marshall para a América: Reconstruindo Nossas Cidades, Vilas e a Classe Média”, {\color{blue}16} de maio de 2017, www.americanprogress.org/issues/economy/reports /2017/05/16/432499/toward-Marshall-Pan American; e Anne Lowrey, “O Governo Deve Garantir um Emprego a Todos?” Atlantic, {\color{blue}18} de maio de 2017, www.theatlantic.com/business/archive/2017/05 /soul-lhe-governante-garante-everyone-a-job/527208.
 \par 
17. Minha descrição da parábola é uma paráfrase baseada nas notas que tomei depois do sermão. Para uma discussão mais aprofundada das implicações de justiça social desta parábola, veja Matthew Skinner, “Matthew 20:1-16: Justice Comes in lhe Evening,” HuffingtonPost, {\color{blue}14} de setembro de 2011, www.huffingtonpost.com/matthew-I-skinner/parable-of-the-workers-in-the-vineyard-commentary_b_961120.html.
 \par 
18. Aditya Chakrabortty, “Uma Renda Básica para Todos? Sim, a Finlândia Mostra que Pode Realmente Funcionar”, Guardian (Londres), 1º de novembro de 2017, www.theguardian.com/commentisfree/2017/oct/31/finland-universal-basic-income; para uma excelente crítica da RBU, veja Alyssa Battistoni, “A Falsa Promessa da Renda Básica Universal”, Dissent (Primavera de 2017), www.dissentmagazine.org/article /false-promise-universal-básica-ícone-Andy-stern-ruger-bregma.
 \par 
NOTAS DO CAPÍTULO {\color{blue}2}
 \par 
Notas do Capítulo {\color{blue}2} 1. A. Michael Spence, Market Signaling: Informational Transfer in Hiring ano Related Screen Processes (Cambridge, MA: Harvard University Press, 1974).
 \par 
2. Alfred Meyer, O feminismo e o socialismo de Lily Braun
 \par 
\section{(Bloomington: Indiana University Press, 1986), 66.}
 \par 
3. Congresso Socialista Internacional, 1910; Segunda Conferência Internacional de Mulheres Socialistas, archive.org/details/International SocialistCongress1910SecondInternationalConferenceOf, {\color{blue}22}.
 \par 
4. Richard Stites, O movimento de libertação das mulheres na Rússia: feminismo, niilismo e bolchevismo, 1860-1930 (Princeton, NJ: Princeton University Press, 1978); Gail Lapidus, Mulheres na sociedade soviética: igualdade, desenvolvimento e mudança social (Berkeley: University of California Press, 1978); Beatrice Brodsky Farnsworth, “Bolchevismo, a questão da mulher e Aleksandra Kollontai”, American Historical Review 81, n.º {\color{blue}2} (1976): 292-316, 296; Elizabeth Wood, O baba e o camarada: gênero e política na Rússia revolucionária (Bloomington: Indiana University Press, 1997).
 \par 
5. Alexandre Avdeev, Alain Blum e Irina Troitskaya, “A história das estatísticas de aborto na Rússia e na URSS de 1990 a 1991”, população {\color{blue}7} (1995): {\color{blue}452}.
 \par 
6. CESifo: DICE [Base de dados para comparações internacionais na Europa], direitos à licença parental: perspectivas históricas (por volta de 1870-2014), www.cesifo-group.de/ifoHome/facts/DICE /Social-Policy/Family/Work-Family-Balance/parental-leve -entrementes-histórica-perspective/file Binary/parental-leve -entrementes-histórica-perspective.pdf.
 \par 
7. Artigo 43, parágrafo 1, Constituição da República Popular da Bulgária, 1971, parliament.bg/bg/19 (em búlgaro); Milanka Vidova, Nevyana Abadjieva e Roumyana Gancheva, {\color{blue}100} perguntas e respostas sobre as mulheres búlgaras (Sófia, Bulgária: Sofia Press, 1983).
 \par 
8. Reforçar o papel das mulheres na construção de uma sociedade socialista desenvolvida: Decisão do Politburo do Comitê Central do Partido Comunista Búlgaro de {\color{blue}6} de março de 1973 (Sofia: Sofia Press, 1974), {\color{blue}10}.
 \par 
9. Josie McLellan, Love in lhe Time of. Communism: Intimacy ano Sexuality in lhe GDR (Nova Iorque: Cambridge University Press, 2011); Kristen Ghodsee, “Pressuring lhe Politburo: The Committee
 \par 
197
 \par 
198
 \par 
NOTAS AO CAPÍTULO {\color{blue}2} * do Movimento das Mulheres Búlgaras e do Feminismo Socialista de Estado”, Slavic Review 73, n.º {\color{blue}3} (2014): 538-562.
 \par 
10. Um leitor, comunicação pessoal por e-mail com o autor,
 \par 
4 de outubro de 2017.
 \par 
11. Kristen R. Ghodsee, “Socialismo e Sexo”, Die Weltwoche, 2017, www.weltwoche.ch/spenden/2017-34/artikel/socialism -um-sex.-die-peluche-machicarem-342017.html; cooperativa de pornografia feminina de Cambridge, pornografia para mulheres (São Francisco: Chronicle Books, 2007).
 \par 
Português {\color{blue}12}. Myra Marx Ferre, Variedades do feminismo: a política de gênero alemã de uma perspectiva global (Stanford, CA: Stanford University Press, 2012), 161; Cynthia Gabriel, “‘Agora é completamente o contrário’: economias políticas de fertilidade na Alemanha reunificada”, em Estados áridos: a “implosão” populacional na Europa, editado por Carrie B. Douglass (Nova York: Berg, 2005), 73-92.
 \par 
13. Dagmar Herzog, Sexo após o fascismo: memória e moralidade na Alemanha do século XX (Princeton, NJ: Princeton University Press, 2005); Helen Frink, Mulheres após o comunismo: a experiência da Alemanha Oriental (Lanham, MD: University Press of. America, 2001).
 \par 
14. Gail Kligman, A política da duplicidade: controlar a reprodução na Romênia de Ceausescu (Berkeley: University of California Press, 1998).
 \par 
15. Jessica Deahl, “Países ao redor do mundo vencem os EUA em licença parental remunerada”, National Public Radio, {\color{blue}6} de outubro de 2016, www.npr.org/2016/10/06/495839588/countries-around-the-world-beat-the-u-s-on-paid-parental-leave.
 \par 
16. Steven Saxonberg e Tomas Sirovatka, “Política familiar fracassada na Europa Central pós-comunista”, Journal of. Comparative Policy Analysis 8, n.º {\color{blue}2} (2006): 185-202.
 \par 
17. Susan Gal e Gail Kligman, A política de gênero após a
 \par 
\section{Cialismo (Princeton, NJ: Princeton University Press, 2000).}
 \par 
18. Valentina Romei, “Eastern Europe Has lhe Largest Population Loss in Modern History,” Financial Times, {\color{blue}27} de maio de 2016, www ft.com/content/70813826-0c64-33d3-8a0c-72059aelb5e3; Ruth Alexander, “Why Is Bulgaria’s Population Falling Off a Cliff?” BBC News, {\color{blue}7} de setembro de 2017, www.bbc.co.uk/news/world-europe-41109572.
 \par 
19. CESifo: DICE, Direitos à Licença Parental; Nadja Popovich, “Os EUA ainda são o único país desenvolvido que não garante licença-maternidade remunerada”, Guardian (Londres),
 \par 
NOTAS DO CAPÍTULO {\color{blue}3}
 \par 
\section{3 de dezembro de 2014, www.theguardian.com/us-news/2014/dec/03/-sp -america-mui-developer-country-país-intermiti-leve.}
 \par 
20. Nancy L. Cohen, “Por que a América nunca teve assistência infantil universal”, New Republic, {\color{blue}24} de abril de 2013, newrepublic.com/article/113009 /child-caré-America-uás-veri-cuoze-universal-dai-caré; Richard Nixon, “387—Veto of. lhe Economic Opportunity Amendments of. 1971”, {\color{blue}9} de dezembro de 1971, www.presidency.ucsb.edu/ws/?pid=3251?
 \par 
21. “A relação do cuidado infantil com o desenvolvimento cognitivo e da linguagem: National Institute of. Child Health ano Human Development Early Child Care Research Network”, Child Development 71, n.º {\color{blue}4} (2000): 960-980, e Ellen Peisner-Feinberg e Margaret Burchinol, “Relações entre as experiências de cuidado infantil de crianças em idade pré-escolar e o desenvolvimento simultâneo: o estudo de custo, qualidade e resultados”, Merrill-Palmer Quarterly 43, n.º {\color{blue}3} (1997): 451-477; John R. Lott Jr., comunicação pessoal por e-mail com o autor, {\color{blue}14} e {\color{blue}15} de agosto de 2017; John R. Lott, “Escola pública, doutrinação e totalitarismo”, Journal of. Political Economy 107, n.º {\color{blue}6} (1999): $ 127-S157.
 \par 
22. John R. Lott, “The New York Times Wants Us to Believe Communists Have Better sex,” Fox News, {\color{blue}14} de agosto de 2017, www.foxnews .com/opinião/2017/08/14/meu-York-times-watts-us-to-bilhete-com mumizas-havê-better-sex.html. Notas do Capítulo {\color{blue}3} 1. Pew Research Center, Women ano Leadership: Public Says Women Are Equally Qualified, bit Barriers Persist, {\color{blue}14} de janeiro de 2015, www.pewsocialtrends.org/2015/01/14/women-and-leadership.
 \par 
2. Todos os números vêm do site da União Interparlamentar, www.ipu.org, particularmente seu banco de dados sobre mulheres na política: archive.ipu.org/wmn-e/classif.htm.
 \par 
3. Os dados sobre os Estados Unidos vêm do Catalyst, “Women lhe S&P {\color{blue}500} Companies,” www.catalyst.org/knowledge/women -sp-500-companhes, e os dados sobre mulheres na Escandinávia vêm de um relatório do Fórum da Faculdade de Direito de Harvard sobre Governança Corporativa e Regulamentação Financeira, “Paridade de Gênero em Conselhos ao Redor do Mundo,” {\color{blue}5} de janeiro de 2017, corp. gov.law.harvard.edu/2017/01/05/gender -parity-on-boaras-round-lhe-world; Nathan Hegedus, “Na Suécia, as mulheres compõem 45% do Parlamento, mas apenas 13% da liderança corporativa,” Quartz.com, {\color{blue}17} de dezembro de 2012, qz.com/37036/in
 \par 
199
 \par 
200
 \par 
NOTAS PARA O CAPÍTULO {\color{blue}3} mulheres-da-sueca-compõem-45-do-parlamento-mas-apenas-13-da-liderança-corporativa; Cristina Zander, “Até a Escandinávia tem uma lacuna de gênero entre CEOs”, Wall Street Journal, {\color{blue}21} de maio de 2014, www.wsj .com/artices/hou-sandai-scania-are-aderecem-lhe-ceo-vender -gap-1400712884.
 \par 
4. Ghodsee, “Pressionando o Politburo: O Comitê do Movimento das Mulheres Búlgaras e o Feminismo Socialista de Estado”, Slavic Review 73, n.º {\color{blue}3} (2014): 538-562.
 \par 
5. Clare Goldberg Moses, “Igualdade e diferença em perspectiva histórica: um exame comparativo dos feminismos dos revolucionários franceses e dos socialistas utópicos”, em Rebel Daughters: Women ano lhe French Revolution, editado por Sara Melzer e Leslie Rabine (Nova York: Oxford University Press, 1992), 231-253.
 \par 
6. L. Goldstein, “Temas feministas iniciais no socialismo utópico francês: os saint-simonianos e Fourier”, Journal of. lhe History of. Ideas 43, n.º {\color{blue}1} (1982); “Degradação das mulheres na civilização”, Théorie des. Quatre Mouvements et. des. Destinées Générales (A teoria dos quatro movimentos e dos destinos gerais), {\color{blue}3}.ª ed. (publicado originalmente em 1808, esta ed. 1841-1848, reimpresso em Women, lhe Family, ano Freedom: The Debate in Documents, Volume One, 1750-1880, editado por Susan Groag Bell e Karen M. Offen a CA: Stanford University Press, 1983), 40-41.
 \par 
7. §. Joan Moon, “Feminism ano Socialism: The Apus Synthesis of. Flora Tristan”, em Socialist Women: European Socialist Feminism in lhe Nineteenth ano Early Twentieth Centuries, editado por Marilyn J. Bóxer e Jean H. Quataert (Nova York: Elsevier, 1978).
 \par 
8. Friedrich Engels, A Origem da Família, da Propriedade Privada e do Estado, 1884, www.marxists.org/archive/marx/works/1884/origin -família/ch02c.htm.
 \par 
9. Engels, A Origem da Família. {\color{blue}10}. Veja também Rochelle Ruthchild, Igualdade e Revolução: Direitos das Mulheres no Império Russo 1905-1917 (Pittsburgh: University of Pittsburgh Press, 2010), 235, e Rochelle Ruthchild, “Ir às urnas é um dever moral para todas as mulheres”: a Grande Guerra e os direitos das mulheres na Rússia, em A Frente Interna da Rússia na Guerra e Revolução, 1914-22, livro 2: A experiência da guerra e da revolução, editado por Adele Lindenmeyr, Christopher Read e Peter Waldron (Bloomington, IN: Slavica, 2018), 1-38.
 \par 
11. Louise Bryant, “Espelhos sobre Moscou”, 1923, www.marxists
 \par 
.org/aclive/bryant/woks/1923-mom/colunai-fim.
 \par 
NOTAS DO CAPÍTULO {\color{blue}3}
 \par 
12. R. C. Elwood, Inessa Armand: Revolutionary ano Feminist (Nova York: Cambridge University Press, 2002); Robert McNeal, bride of. lhe Revolution: Krupskaya ano Lenin (Ann Arbor: University of Michigan Press, 1972); Cathy Porter, Alexandra Kollontai: A Biography (Chicago: Haymarket Books, 2014). Sobre campanhas de alfabetização russas, veja Ben Eklof, “Russian Literacy Campaigns 1861-1939,” em National Literacy Campaigns ano Movements: Historical ano Comparative Perspectives, editado por Robert F. Arnove e Harvey J. Graff (New Brunswick, NJ: Transaction Publishers, 2008), 128-129; Helen McCarthy e James Southern, “Mulheres, gênero e diplomacia: uma pesquisa histórica”, em Gênero e diplomacia, editado por Jennifer Cassidy (Nova York: Routledge, 2017), 15-31, {\color{blue}24}.
 \par 
13. Anna Krylova, Mulheres soviéticas em combate: uma história de violência na Frente Oriental (Nova York: Cambridge University Press, 2011).
 \par 
14. Mateja Jeraj, “Vida Tomsi¢,” em A Biographical Dictionary of. Women’s Movements ano Feminisms: Central, Eastern ano South-Eastern Europe, 19th ano 20th Centuries, editado por Francisca de Haan, Krasimira Daskalova e Anna Loutfi (Budapeste: Central European University Press, 2006), 575-579; Chiara Bonfigioli, “Revolutionary Networks: Women’s Political ano Social Activism in Cold War Italy ano Yugoslavia (1945-1957),” dissertações de doutorado, Universidade de Utrecht, 2012.
 \par 
15. Kristen Ghodsee, “The Left Side of. History: The Legacy of. Bulgaria’s Elena Lagadinova,” Foreign Affairs, {\color{blue}29} de abril de 2015, www foreignaffairs.com/articles/bulgaria/2015-04-29/left-side-history; Kristen Ghodsee, The Leof. Side of History: Worldanorlhe and the Unfulfilled of.omise of Communism in Eastern Europe (O lado esquerdo da história: a Segunda Guerra Mundial e a promessa não cumprida do comunismo na Europa Oriental) (Durham, NC: Duke University Press, 2015); Krassimira Daskalova, “A Woman Politilhen in the Cold War Balkans: From Biography to History,” Aspasia: The International Yof.rbook of Central, Eanoern, and Southeastern European anoen’s and Gender History {\color{blue}10} (2016); e John D. Bell, The Bulgarian Communisfonrty from Blagoev to Zhivkov (O partido comunista búlgaro de Blagoev a Zhivkov) (Stanford, CA: Hoover Institution Press, 1985).
 \par 
16. “A Girl Who Hated Cream Puffs”, time, {\color{blue}20} de setembro de 1948, 33; W. H. Lawrence, “Aunty Ana”, New York Times, {\color{blue}29} de fevereiro de 1948; Robert Levy, Ana Pauker: The Rise ano Fall of. a Jewish Communist (Berkeley: University of California Press, 2001); Valentina Tereshkova, Valentina Tereshkova, The First Lady of. Space: In Her Own Words, Spacebusiness.com, 2015.
 \par 
201
 \par 
202
 \par 
NOTAS DO CAPÍTULO {\color{blue}3}
 \par 
17. Michael Parks, “Perfil: Galina Semyonova: Nenhum mero símbolo no Politburo soviético”, Los Angeles Times, {\color{blue}15} de janeiro de 1991, artigos Jatimes.com/1991-01-15/news/wr-327_1_soviet-union; Leonid Lipilin, “Mulher de ação”, Soviet Life (março de 1991): 22-23; Bill Keller, “O ponto de vista de uma mulher soviética”, New York Times, {\color{blue}24} de janeiro de 1989, www.nytimes.com/1989/01/24/world/a-soviet-woman-s-point-of -view.html.
 \par 
18. Women in lhe Politics of. Postcommunist Eastern Europe, editado por Marilyn Rueschemeyer (Armonk, NY: M. E. Sharpe, 1998); Marilyn Rueschemeyer e Sharon Wolchik, editoras, Women in Power in pós-comunista Parliaments (Bloomington: Indiana University Press, 2009); The World’s Women 1970-1990: Trends ano Statistics, Nações Unidas, 1991; Kerin Hope, “Bulgária constrói legado de elite feminina de engenharia”, Financial Times, {\color{blue}9} de março de 2018, www.ft.com/content/e2fdfe6e-0513-1le8-9el12-af73e8db3c71; Peter Lentini, “Dados estatísticos sobre mulheres na URSS”, Lorton Paper n.º 10, 1994.
 \par 
19. Susan Wellford, “Uma cota que vale a pena fazer”, U.S. News ano World Report, {\color{blue}21} de fevereiro de 2017, www.usnews.com/opinion/civil-wars /artices/201lhe2-21soulsconsideracvenderrcotasr-quoincresseihomemse-wpolíticapolitics.
 \par 
20. As cotas não apenas promovem a diversidade de gênero, mas também aumentam o conjunto de candidatos qualificados, em geral. E algumas evidências sugerem que as corporações com mais diversidade em seus conselhos são, na verdade, mais lucrativas. Marcus Noland, Tyler Moran e Barbara Kotschwar, “Is Gender Diversity Profitable? Evidence fon a Global Survey,” Working Paper Series do Peterson Institute for International Economics, fev. 2016; Margarethe Weirsema e Marie Louise Mors, “What Board Directors Really Think of. Gender Cotas,” Harvard Business Review, {\color{blue}14} de novembro de 2016; Oliver Staley, “You Know Those Cotas for Female Board Members in Europe? They're Working,” Quartz.com, {\color{blue}3} de maio de 2016, qz.com/674276/you-know-those-quotas-for-female -voar-membés-in-Europe-cheire-forcing; português e Daniel Boffey, “UE pressionará por cota de 40% para mulheres em conselhos de administração”, Guardian (Londres), {\color{blue}20} de novembro de 2017, www.theguardian.com/world/2017/nov/20/eu-to-push-for-40-quota-for-women-on-company-boards.
 \par 
21. Pew Research Center, Mulheres e Liderança: O público diz que as mulheres são igualmente qualificadas, mas as barreiras persistem, {\color{blue}14} de janeiro de 2015, www.pewsocialtrends.org/2015/01/14/women-and-leadership. {\color{blue}22}. The Rockefeller Foundation, “Mulheres na liderança: por que
 \par 
NOTAS DO CAPÍTULO {\color{blue}4}
 \par 
It Matters”, {\color{blue}12} de maio de 2016, www.rockefellerfoundation.org/report /homem-in-leaders-chi-it.-matters, e KPMG, “KPMG Women’s Leadership Study”, {\color{blue}13} de setembro de 2016, home.kpmg.com/ph/en/home /visões/2016/09/kpmg-homem-s-leaders-study.html.
 \par 
23. “Uma garota que odiava bolinhos de creme”, {\color{blue}33}. {\color{blue}24}. Slavenka Drakulić, Como sobrevivemos ao comunismo e até mesmo
 \par 
\section{Riu (Nova York: Harper Perennial, 1993), 31.}
 \par 
\section{Notas do Capítulo 4}
 \par 
1. Roy Baumeister e Kathleen Vohs, “Economia Sexual: Sexo como Recurso Feminino para Troca Social em Interações Heterossexuais”, Personality ano Social Psychology Review 8, n.º {\color{blue}4} (2004): 339-363.
 \par 
2. Roy Baumeister, Tania Reynolds, Bo Winegard e Kathleen Vohs, “Competindo por amor: aplicando a teoria da economia sexual a concursos de acasalamento.” Journal of. Economic Psychology {\color{blue}63} (dez. 2017): 230-241.
 \par 
3. Baumeister et al., “Competindo pelo amor”. {\color{blue}4}. Veja, por exemplo, Laurie A. Rudman, “Mitos da teoria da economia sexual: implicações para a igualdade de gênero”, Psychology of. Women Quarterly 4, n.º {\color{blue}3} (2017): 299-313.
 \par 
5. O vídeo do Austin Institute The Economics of. sex pode ser encontrado aqui: www.youtube.com/watch?v=cOlifNaNABY. Para uma crítica interessante dos efeitos sobre os jovens que assistem, veja Laurie A. Rudman e Janelle Fetteroff, “Exposure to Sexual Economics Theory Promotes a Hostile View of. Heterosexual Relationships,” Psychology of. Women Quarterly 4, no. {\color{blue}1} (2017): 77-88. Sobre a economia do aborto, veja George Akerlof, Janet Yellen e Michael Katz, “An Analysis of. Out-of.-Wedlock Childbearing in lhe United States,” Quarterly Journal of. Economics 11, no. {\color{blue}2} (1996): 277-317. Os trabalhos de Shoshana Gossbard sobre a economia do amor e do casamento também são precursores importantes da teoria da economia sexual.
 \par 
6. Simon Chang, Rachel Connelly e Ping Ma, “O que você fará se eu disser ‘sim’?: O efeito da proporção sexual no uso do tempo entre casais casados ​​de Taiwan”, Population Research ano Policy Review 35, n.º {\color{blue}4} (20164): 471-500, e Simon Chang e Xiaobo Zhang, “Competição de acasalamento e empreendedorismo: evidências de um experimento natural em Taiwan”, IFPRI Discussion Paper 01203, {\color{blue}29} de agosto de 2013, papers.ssrn.com/sol3/papers.cfm?abstract_id=214301 {\color{blue}3}.
 \par 
7. Roy Baumeister e Juan Pablo Mendoza, “Cultural
 \par 
203
 \par 
204
 \par 
NOTAS AO CAPÍTULO {\color{blue}4}
 \par 
Variações no mercado sexual: igualdade de gênero se correlaciona com mais atividade sexual”, Journal of. Social Psychology 151, n.º {\color{blue}3} (2011): 350-360.
 \par 
8. Karl Marx e Friedrich Engels, O Manifesto Comunista,
 \par 
\section{1848, www.marxists.org.}
 \par 
Português 9, August Bebel, “Mulher do Futuro”, em Mulher e Socialismo, traduzido por Meta L. Stern (Nova York: Socialist Literature, 1910), www.marxists.org/archive/bebel/1879/woman-socialism /index.htm.
 \par 
10. Bebel, Mulher e Socialismo; John Lauritsen, “O Primeiro Político a Falar pelos Direitos Homossexuais”, 1978, paganpressbooks.com/jpl/BEBEL.HTM. Embora os países socialistas estatais do século XX vissem geralmente a homossexualidade como um resquício infeliz da decadência burguesa a ser perseguido, os sexólogos da Tchecoslováquia e da Alemanha Oriental acabariam assumindo uma posição menos crítica, com os alemães orientais se preparando para sancionar oficialmente as relações entre pessoas do mesmo sexo bem na véspera da queda do Muro de Berlim. Veja Josie McLellan, Love in lhe Time of. Communism: Intimacy ano Sexuality in lhe GDR (Nova York: Cambridge University Press, 2011); Friedrich Engels, The Origin of. lhe Family, Private Property ano lhe State, 1884, www.marxists. Dnglatchive/maniuara/162# /origin.-família/ch02c.htm.
 \par 
11. John Simkin, “Alexandra Kollontai,” Spartacus Educational, set. {\color{blue} 1997 } {\par} , atualizado out. {\color{blue} 2017 } {\par} , spartacus-educational.com /RUSkollontai.htm; Alexandra Kollontai, Autobiografia de uma mulher comunista sexualmente emancipada, 1926, www.marxists.org /aclive/coluna/1926/autobiography.htm; Alexandra Kollontai, “Teses sobre a moralidade comunista na esfera das relações conjugais,” {\color{blue} 1921 } {\par} , www.marxists.org/archive/kollonta/1921/theses-morality.htm. {\color{blue}12}. Cathy Porter, Alexandra Kollontai: A Biography (Chicago: Haymarket Books, 2014); Alix Holt, Alexandra Kollontai: Selected Writings (Nova York: W. W. Norton, 1980). {\color{blue}13}. Kollontai, “Teses sobre a moralidade comunista”. {\color{blue}14}. Wendy Z. Goldman, Mulheres, o Estado e a Revolução: Política familiar soviética e vida social, 1917-1936 (Cambridge: Cambridge University Press, 1993) e Kollontai, Autobiografia.
 \par 
15. Dados sobre estudantes de Moscou em 1922 citados em Sheila Fitzpatrick, “Sex ano Revolution: An Examination of. Literary ano Statistical Data on lhe Mores of. Soviet Students in lhe 1920s,” Journal of. Modern History 50, n.º {\color{blue}2} (1978): 252-278. Veja também Goldman, Women, lhe State,
 \par 
NOTAS AO CAPÍTULO {\color{blue}5} e Revolução; Richard Stites, O Movimento de Libertação das Mulheres na Rússia: Feminismo, Niilismo e Bolchevismo, 1860-1930 (Princeton, NJ: Princeton University Press, 1978); Gail Lapidus, editora, Mulheres, Trabalho e Família na União Soviética (Nova York: Routledge, 1981); Gail Lapidus, Mulheres na Sociedade Soviética: Igualdade, Desenvolvimento e Mudança Social (Berkeley: University of California Press, 1978).
 \par 
Notas do Capítulo {\color{blue}5} 1. Mikhail Stern e August Stern, sex in lhe Soviet Union, editado e traduzido por Mark Howson e Cary Ryan (Nova York: Times Books, 1980).
 \par 
2. Anna Temkina e Elena Zdravomyslova, “Os roteiros sexuais e a identidade das mulheres russas de classe média”, Sexuality & Culture {\color{blue}19} (2015): 297-320, {\color{blue}306}.
 \par 
Português {\color{blue}3}. Temkina e Zdravomyslova, “Os roteiros sexuais”, {\color{blue}307}. {\color{blue}4}. Temkina e Zdravomyslova, “Os roteiros sexuais”, {\color{blue}308}. {\color{blue}5}. Peter Pomerantsev, Nada é verdade e tudo é possível: o coração surreal da nova Rússia (Nova York: PublicA affairs, 2015). {\color{blue}6}. Dagmar Herzog, Sexo após o fascismo: memória e moralidade na Alemanha do século XX (Princeton, NJ: Princeton University Press, 2007); português Dagmar Herzog, “East Germany’s Sexual Evolution,” em Socialist Modern: East German Everyday Culture ano Politics, editado por Katherine Pence e Paul Betts (Ann Arbor: University of Michigan Press, 2008), {\color{blue}72}. Ver também Donna Harsch, Revenge of. lhe Domestic: Women, lhe Family, ano Communism in lhe German Democratic Republic (Princeton, NJ: Princeton University Press, 2008).
 \par 
7. Ingrid Sharp, “A Unificação Sexual da Alemanha”, Journal of. lhe History of. Sexuality 13, no {\color{blue}3} (julho de 2004): 348-365. {\color{blue}8}. Herzog, “East Germany's Sexual Evolution”, {\color{blue}73}. {\color{blue}9}. Kurt Starke e Walter Friedrich, Love ano Sexuality up to {\color{blue}30} (Berlim: Deutcher Verlag her. Wissenschaften, 1984), 187, 202-203, citado em Herzog, “East A evolução sexual da Alemanha”, {\color{blue}86}.
 \par 
10. Ulrich Clement e Kurt Starke, “Comportamento sexual e atitudes em relação à nossa sexualidade entre estudantes em sua DRD e em sua RDA”, Citschrift sair Sexualforschung {\color{blue}1} (1988) citado em Herzog, “East Germany’s Sexual Evolution”, 87; Werner Habermehl, “Sobre a sexualidade dos jovens na República Federal da Alemanha e na RDA”, em Sexualidade BDR/GDR in
 \par 
205
 \par 
206
 \par 
NOTAS DO CAPÍTULO {\color{blue}5}
 \par 
Verhleich, 20-40, 38; e Kurt Starke e Konrad Weller, “Diferenças na conduta sexual entre adolescentes da Alemanha Oriental e Ocidental antes da unificação”, artigo apresentado na Conferência Anual da Academia Internacional de Pesquisa Sexual, Praga, 1992, ambos citados em Sharp, “A unificação sexual da Alemanha”, 354-355. {\color{blue}11}. Sharp, “A unificação sexual da Alemanha”, {\color{blue}356}. {\color{blue}12}. Paul Betts, Within Walls: Private Life in lhe German Democratic Republic (Dentro de muros: vida privada na República Democrática Alemã) (Oxford: Oxford University Press, 2013), e Josie McLellan, Love in lhe Time of. Communism: Intimacy ano Sexuality in lhe GDR (O amor na época do comunismo: intimidade e sexualidade na RDA) (Nova York: Cambridge University Press, 2011).
 \par 
13. Herzog, “East Germany’s Sexual Evolution,” {\color{blue}90}. {\color{blue}14}. Todos os números vêm da primeira onda da World Values ​​Survey (1981-1984), usando a página da web “Online Data Analysis”: www.worldvaluessurvey.org/W VSOnline.jsp.
 \par 
15. Agnieszka Kosciariska, “Sexo em termos iguais? Sexologia polonesa sobre a emancipação das mulheres e o ‘bom sexo’ da década de 1970 até o presente”, Sexualities 19, nos. 1-2 (2016): 236-256. Sobre o comitê das mulheres polonesas, veja Jean Robinson, “Mulheres, o Estado e a necessidade da sociedade civil: a Liga Kobiet na Polônia”, em Comparative State Feminism, editado por Dorothy McBride Stetson e Amy G. Mazur (Thousand Oaks, CA: Sage, 1995), 203-220, e Malgorzata Fidelis, Mulheres, comunismo e industrialização na Polônia pós-guerra (Nova York: Cambridge University Press, 2014).
 \par 
16. Agnieszka Koécianska, “Além do Viagra: Terapia Sexual na Polônia”, Revisão Sociológica Tcheca 20, no. {\color{blue}6} (2014): 919-938, {\color{blue}919}.
 \par 
17. Agnieszka Koécianska, comunicação pessoal por e-mail com a autora, agosto de 2017; Agnieszka Koécianska, “Feminist ano Queer Sex Therapy: The Ethnography of. Especialista Knowledge in Poland”, em Rethinking Ethnography in centro-europeu, editado por Hana Cervinkova, Michal Buchowski e Zdenek Uherek (Nova York: Palgrave MacMillan, 2015).
 \par 
18. Agnieszka KoSciatiska, “Além do Viagra: Terapia Sexual na Polônia”, Czech Sociological Review 20, n.º {\color{blue}6} (2014): 919-938.
 \par 
19. Veja, por exemplo, Maria Bucur, “Sexo na Época do Comunismo”, 2012, www.publicseminar.org/2017/12/sex-in-the-time -of.-comunismo, e Gail Kligman, A Política da Duplicidade: Controlando a Reprodução na Romênia de Ceausescu (Berkeley: University of California Press, 1998); Georgi Gospodinov, “Cexc m0 Bpeme Ha COlMaIM3'bM.: XUTMeHa, MeAMIMHa MU PuskylTypa” em Jlo6oBTa mil
 \par 
NOTAS PARA O CAPÍTULO {\color{blue}6} cornalina, editado por Daniela Koleva (Sofia: RIBA, 2015). A citação vem de uma atualização de posição pública do Facebook, {\color{blue}12} de outubro de 2017.
 \par 
20. Katefina Liskova, “Sexo sob o socialismo: da emancipação das mulheres às famílias normalizadas na Checoslováquia”, Sexualities 19, n.º {\color{blue}2} (2016): 211-235.
 \par 
21. O estudo original usando apenas os dados de 1992-1994 é S. Kornrich, J. Brines e K. Leupp, “Egalitarianism, Housework, ano Sexual Frequency in Marriage”, American Sociological Review {\color{blue}78} (2013): 26-50. O estudo de acompanhamento usando os dados de 1992-1994 e os dados de 2006 é Daniel Carlson, Amanda Miller, Sharon Sassler e Sarah Hanson, “The Gendered Division of. Housework ano Couples’ Sexual Relationships: A Re-Examination”, Journal of. Marriage ano Family 78, n.º {\color{blue}4} (2016): 975-995.
 \par 
22. O estudo longitudinal de casais alemães é M. D. Johnson, N. L. Galambos e J. R. Anderson, “Skip lhe Dishes? Not So Fast! Sex ano Housework Revisited,” Journal of. Family Psychology 30, no. {\color{blue}2} (2016): 203-213.
 \par 
23. Mark Fisher, Realismo Capitalista: Não Há Alternativa?
 \par 
\section{(Ropley, Hampshire, Reino Unido: Zero Books, 2009).}
 \par 
24. Sobre trabalho emocional, veja Arlie Russell Hochschild, The Managed Heart: Commercialization of. Human Sensação (Berkeley: University of California Press, 2012).
 \par 
25. Elizabeth O'Brien, “People Are Stalling Their Divorce So They Don't Lose Health Care”, time, {\color{blue}24} de julho de 2017, time.com/money /4871186/pele-are-stalking-ter-divorcie-só- eles perderam os cuidados de saúde.
 \par 
26. Albert Einstein, “Por que o socialismo?”, Monthly Review {\color{blue}1} (1949),
 \par 
\section{mensalreview.org/2009/05/01/why-socialism.}
 \par 
Notas do Capítulo {\color{blue}6} 1. Para informações sobre os Borg, consulte www.startrek.com/database _ articule/borg.; Francis Fukuyama, The End of. History ano lhe Last Man (Nova York: Free Press, 1992).
 \par 
2. Alexei Yurchak, Tudo era para sempre até que não existisse mais
 \par 
\section{(Princeton, Nova Jersey: Princeton University Press, 2005).}
 \par 
3. Esta citação é atribuída a Margaret Mead, embora não haja nenhuma fonte escrita. Uma explicação completa da citação pode ser encontrada no site do Institute for Intercultural Studies: www.interculturalstudies.org/faq.html.
 \par 
207
 \par 
208
 \par 
NOTAS DO CAPÍTULO {\color{blue}6}
 \par 
4. Seema Mehta, “Apoiadores de Trump tuítam #Repealthel9th após pesquisas mostrarem que ele venceria se apenas os homens votassem”, Los Angeles Times, {\color{blue}12} de outubro de 2016, www.latimes.com/nation/politics/trailguide/la-Na-trail 9th-1476299001 guide-updates-trump-hackers-tweet-repelível -htmlstory.html.
 \par 
5. Matthew Biedlingmaier, “Coulter: ‘Se tirássemos o direito das mulheres de votar, nunca mais teríamos que nos preocupar com outro presidente democrata”, Media Matters, {\color{blue}4} de outubro de 2007, www.mediamatters.org/research/2007/10/04/coulter-if-we-took-away -homem-right-to-vote-uê/140037.
 \par 
6. John R. Lott e Lawrence Kenny, “O sufrágio feminino mudou o tamanho e o escopo do governo?” Journal of. Political Economy {\color{blue}107} (1999): 1163-1198, 1164.
 \par 
7. Veja o site The Curse of. 1920: thecurseof1920.com/index HTML.
 \par 
8. Gary D. Naler, A Maldição de 1920: A Degradação de Nossa Nação nos Últimos {\color{blue}100} Anos (Salem, MO: RTC Quest Publications, 2007), 48; Roosh V, “Como salvar a civilização ocidental”, rooshv.com, {\color{blue}6} de março de 2017, www.rooshv.com/how-to-save-western-civilization. O ensaio original de Ramzpaul “Como o sufrágio feminino destruiu a civilização ocidental” foi republicado aqui: www.theapricity.com/forum /showthread.php?164336-How-Female-Suffrage-Destroyed-Western -Civilization.
 \par 
9. Julia Mead, “Por que os millennials não têm medo do socialismo”, Nation, {\color{blue}10} de janeiro de 2017, www.thenation.com/article/why-millennials -art.-afra id-of.-lhe-s-word?
 \par 
10. Sarah Leonard, “Por que tantos eleitores jovens estão se apaixonando por velhos socialistas?” New York Times Sunday Review, {\color{blue}16} de junho de 2017, www.nytimes.com/2017/06/16/opinion/sunday/sanders-corbyn-socialsts .html.
 \par 
11. Richard Fry, “Millennials e Geração X superaram Boomers e Gerações Mais Velhas nas Eleições de 2016”, FactTank, {\color{blue}31} de julho de 2017, www.pewresearch.org/fact-tank/2017/07/31/millennials-and -gen.-zero-alvote-boomers-ano-puder-generativas-in-2016 -electrão.
 \par 
12. Mark Fisher, “Por que a saúde mental é uma questão política”, Guardian (Londres), {\color{blue}6} de julho de 2010, www.theguardian.com /commentisfree/2012/jul/16/mental-health-política-isso?
 \par 
13. David Harvey, Dezessete contradições e o fim do capitalismo (Oxford: Oxford University Press, 2015).
 \par 
