\chapter{10 O chamado problema da transformação}\label{10 O chamado problema da transformação}
 \par 
No Volume {\color{blue}1} de O Capital, Marx preocupa-se com a produção de valor e mais-valia, e no Volume {\color{blue}2} com a sua circulação e troca. A maior parte do Volume {\color{blue}3} trata das relações distributivas à medida que surgem da interação da produção com a troca. Na sua análise, Marx centra-se na distribuição pela economia da mais-valia produzida pelos capitais industriais concorrentes, incluindo a sua apropriação, em parte, pelo capital comercial e financeiro e pela classe proprietária de terras.
 \par 
O ponto de partida da análise da distribuição de Marx é o seu argumento de que capitais de igual dimensão produzem geralmente diferentes quantidades de mais-valia, porque cada capital emprega uma quantidade diferente de trabalho produtor de valor. Apesar disso, todos os capitais tendem a desfrutar de taxas de retorno iguais, caso contrário mudariam para áreas mais lucrativas da economia. Marx explica a distribuição do capital e do trabalho na economia e a distribuição da mais-valia produzida pelo capital industrial (na ausência de outras formas de capital), através da transformação de valores em preços de produção. Num nível de análise ainda mais concreto, os capitalistas comerciais e financeiros, e os proprietários de terras, capturam em troca parte da mais-valia produzida pelo capital industrial. Marx explica estes processos através da sua análise do lucro comercial, dos juros e da renda (abordados aqui, respectivamente, nos Capítulos 11, {\color{blue}12} e {\color{blue}13}).
 \par 
\section{Dos valores aos preços de produção}
 \par 
Para a distribuição da mais-valia entre os capitais industriais nos diferentes setores da economia, Marx concentra-se inicialmente na tendência de equalização da taxa de lucro. A taxa geral de lucro é r = S ⁄ (C + V), onde as quantidades de valor S, C e V são agregados de mais-valia e capital constante e variável para a economia como um todo. Marx argumenta que cada capitalista industrial participaria na mais-valia total produzida de acordo com a sua participação no capital adiantado, em vez de simplesmente se apropriar da mais-valia que eles próprios produziram: é como se cada capitalista recebesse um dividendo sobre uma participação no capital social. economia como um todo. Com isso, a participação nos lucros do capitalista, cujo adiantamento de capital constante e variável é ci + vi, seria representada por r (ci + vi). Por exemplo, se a taxa de lucro geral for de {\color{blue}50} por cento e o capitalista médio, ao produzir artigos, adiantar {\color{blue}100}.000 libras (compostas por capital variável e constante, incluindo a depreciação do capital fixo), os lucros anuais da empresa tenderiam a ser de {\color{blue}50}.{\color{blue}000} libras.
 \par 
Correspondente a isso seria um preço de produção para a mercadoria em questão, formado a partir do custo mais o lucro:
 \par 
Um exemplo simples ilustrará isso (ver Tabela {\color{blue}10}.{\color{blue}1}). Suponhamos que existam apenas dois capitais produzindo bens distintos, um dos quais utiliza 60c + 40v e o outro 40c + 60v, sendo a taxa de mais-valia de {\color{blue}100} por cento. (Aqui seguimos a notação de Marx ao adicionar c, v ou s após as quantidades de valor, {\color{blue}60} ou 40, para indicar a composição de valor da mercadoria.) Neste caso, o valor da produção do primeiro capital será 60c + 40v + 40s = 140, e o valor da produção da segunda capital será 40c + 60v + 60s = {\color{blue}160}.
 \par 
Este exemplo levanta um problema sério. Pois isso implica que os capitalistas avançam somas iguais de dinheiro, mas usam e t an r’ e c i r P distintos.
 \par 
\section{Dos valores aos preços de produção}
 \par 
\section{Dos valores aos preços de produção}
 \par 
TTTTTT e c i r P
 \par 
E t an r' e u l a V
 \par 
\section{Dos valores aos preços de produção}
 \par 
\section{Dos valores aos preços de produção}
 \par 
. E t an i r p o r p p a
 \par 
\section{Dos valores aos preços de produção}
 \par 
1. {\color{blue}0} 1 e lb a T
 \par 
TTTTTT e t an R
 \par 
S l a t i p a C
 \par 
\section{Dos valores aos preços de produção}
 \par 
\section{Dos valores aos preços de produção}
 \par 
s e g an rev an s a t o t
 \par 
S e t a c i d n i
 \par 
TTTTTT t é um l
 \par 
Proporções de c e v teriam diferentes taxas de lucro individuais. Em nosso exemplo, o primeiro capital colhe apenas r1 = {\color{blue}40} ⁄ (60 + {\color{blue}40}) = {\color{blue}40} por cento, enquanto o segundo capital desfruta de uma taxa de lucro maior, r2 = {\color{blue}60} ⁄ (40 + {\color{blue}60}) = {\color{blue}60} por cento. Isso se deve à diferença na composição dos capitais avançados, com uma proporção relativamente maior de capital variável levando a uma taxa de lucro maior. Isso não deveria ser surpreendente. Se apenas o trabalho cria valor (e, portanto, lucro), enquanto os meios de produção apenas transferem seu valor para a produção, o capital que emprega mais trabalho produz mais valor e mais-valia e, tudo o mais constante, tem uma taxa de lucro maior.
 \par 
Os capitais que obtêm taxas de lucro diferentes não coexistirão por muito tempo, dada a possibilidade de migração entre sectores. Por outras palavras, uma vez que cada capitalista contribui igualmente em capital adiantado (100), cada um deve partilhar igualmente no lucro distribuído (50 cada). Isto só pode acontecer se os preços de produção forem cada um de {\color{blue}150}. Isto apesar das diferenças nos valores produzidos nos dois sectores - a equalização das taxas de lucro entre capitais em diferentes sectores exige a transferência de valor (excedente) entre sectores do economia, que é afetada pelas diferenças entre os preços de produção e os valores das mercadorias.
 \par 
Uma vez que os capitais em diferentes sectores utilizarão geralmente proporções distintas de trabalho, matérias-primas e maquinaria para produzir mercadorias, Marx tira a conclusão de que os produtos não são trocados pelos seus valores, mas pelos preços de produção. Estes preços de produção diferem dos valores, pois a composição do capital, ci ⁄ vi, é maior ou menor que a média da economia como um todo. (Observe que para a primeira capital na Tabela {\color{blue}10}. {\color{blue} 1 } {\par} , c ⁄ v = 3∕2 e, para a segunda, c ⁄ v = 2∕3, em comparação com uma média de {\color{blue}1} para a economia como um todo.)
 \par 
\section{Dos valores aos preços de produção}
 \par 
A explicação de Marx sobre a relação entre valores e preços tem sido um dos aspectos mais controversos da sua teoria do valor.
 \par 
Levou alguns, mesmo que de outra forma simpatizantes do marxismo, a rejeitar a teoria do valor-trabalho como irrelevante ou mesmo errónea.
 \par 
A razão para esta reacção é que a solução de Marx para o problema da transformação é considerada incorrecta e que as consequências deste suposto “erro” são, supostamente, de grande alcance. O cerne da crítica é o seguinte: Marx mostrou que, quando os capitais competem entre sectores (e a migração de capital pode ocorrer de um sector para outro), as mercadorias já não são trocadas a preços iguais aos seus valores. Isto leva tanto à objecção empírica de que os valores são irrelevantes, uma vez que não impulsionam as trocas reais, como à crítica lógica de que, no Volume 3, Marx continuou a avaliar os factores de produção, c e v (e a taxa de “valor” de lucro, utilizado no cálculo dos preços de produção), como se fossem valores, e não preços. Em outras palavras, é como se, para os críticos, Marx presumisse que as mercadorias são compradas “a valores” (respectivamente, {\color{blue}140} e {\color{blue}160}), mas são vendidas “a preços” (150 e {\color{blue}150}) - o que é inconsistente, uma vez que vender e os preços de compra devem ser os mesmos.
 \par 
Para o problema de traduzir determinados valores em preços de produção numa economia em equilíbrio, isto seria uma deficiência, mas uma deficiência da qual Marx estava plenamente consciente e que pode ser facilmente corrigida. É apenas uma questão de transformar simultaneamente as entradas e as saídas através de um procedimento algébrico simples. A implicação desta “correcção” é simples: as mercadorias têm valores e também preços, e dois sistemas contabilísticos distintos (não necessariamente igualmente significativos, quer na teoria quer na prática) são possíveis. Um destes sistemas contabilísticos expressa o tempo de trabalho socialmente necessário para produzir cada mercadoria, e o outro a quantidade de dinheiro que, em geral, a mercadoria renderia à venda.
 \par 
Mais significativa do que a “solução” algébrica do “problema” de transformação é a observação de que a teoria do valor-trabalho de Marx não pode naufragar em tais enigmas quantitativos, como a procura de uma solução algébrica corrigida parece implicar. Crucialmente,
 \par 
Marx mostrou que os valores existem como consequência das relações sociais entre produtores e que a formação de preços traduz as condições de produção em relações de troca. Porque existem (em vez de serem apenas uma construção da imaginação), os valores não podem ser desafiados ou rejeitados de acordo com interpretações algébricas da teoria de Marx. Em vez disso, a relação real entre valores e preços tem de ser reconhecida teoricamente e explorada analiticamente - por exemplo, porque é que as relações dominantes de produção dão origem à forma de valor, como é que os valores aparecem como preços na prática e mudam ao longo do tempo, e como é que os valores aparecem como preços na prática e mudam ao longo do tempo? as tensões entre valores e preços contribuem para crises económicas?
 \par 
Sob esta luz, é significativo que a literatura sobre o problema da transformação se concentre tradicionalmente nas implicações das diferenças na composição de valor do capital (VCC) entre diferentes sectores da economia - como se c e v na Tabela {\color{blue}10}.{\color{blue}1} fossem quantidades de dinheiro, sendo {\color{blue}140} e {\color{blue}160} os preços “originais” da unidade de produção e {\color{blue}150} os preços unitários “modificados pela concorrência”.
 \par 
Este não é o caso de Marx. No Volume 3, Marx considera a transformação inteiramente em termos da composição orgânica do capital (OCC) que, como foi mostrado no Capítulo 8, está apenas preocupada com os efeitos das diferentes taxas a que as matérias-primas são transformadas em produtos (em contraste , os diferentes valores das entradas são capturados pelo VCC). Como tal, Marx está menos preocupado com a forma como os factores de produção (c e v) obtiveram originalmente os seus preços, e mais preocupado com a forma como os diferentes OCC, ou diferentes taxas de produtividade entre sectores, influenciam a formação de preços e lucros.
 \par 
O problema de Marx é o seguinte. Se uma determinada quantidade de trabalho vivo num sector (empregado através do adiantamento de capital variável v) produzir uma quantidade maior de matérias-primas, representada por c (independentemente do seu custo), do que noutro sector, as mercadorias produzidas terão uma maior quantidade de matérias-primas, representadas por c (independentemente do seu custo). preço em relação ao valor, conforme discutido anteriormente e ilustrado numericamente na Tabela {\color{blue}10}.{\color{blue}1}. Por outras palavras, a utilização de uma maior quantidade de trabalho na produção criará mais valor e mais mais-valia do que uma menor quantidade - independentemente do sector, do valor de uso produzido e do custo das matérias-primas. Esta é uma proposição completamente geral dentro da teoria do valor e sustenta a explicação de Marx sobre a existência de preços e lucro. A utilização que Marx faz do OCC em vez do VCC na sua transformação é significativa porque o OCC liga a taxa de lucro à esfera da produção, onde o trabalho vivo produz valor e mais-valia. Em contraste, o VCC liga a taxa de lucro à esfera da troca, onde as mercadorias são negociadas e onde os valores recentemente estabelecidos medem a taxa de acumulação de capital.
 \par 
A sua ênfase no OCC mostra que Marx está principalmente preocupado com o efeito sobre os preços das diferentes capacidades de criação de valor (excedente) dos capitais avançados, ou com o impacto sobre os preços das diferentes quantidades de trabalho necessárias para transformar os meios de produção. na produção - independentemente do valor dos meios de produção utilizados como matéria-prima. A utilização do OCC na análise da criação e distribuição de lucros é importante, porque atribui firmemente a fonte da mais-valia e do lucro ao trabalho não remunerado. Isto ajuda Marx a fundamentar as suas afirmações de que as máquinas não criam valor, que a mais-valia e o lucro não se devem a trocas desiguais e que o lucro industrial, os juros e a renda são partes da mais-valia produzida pelos trabalhadores assalariados produtivos.
 \par 
Na sua transformação de valores em preços de produção, Marx não está a lidar com a teoria do preço de equilíbrio (como na economia dominante e em muitas interpretações convencionais da teoria do valor-trabalho), mas com a relação entre diferenças ou mudanças na produção e na formação de preços. Isto funciona no Volume {\color{blue}3} como um prelúdio ao tratamento da lei da tendência de queda da taxa de lucro (LTRPF) (embora a ordem de apresentação seja invertida neste livro). Finalmente, o problema da transformação e o LTRPF têm sido geralmente considerados como dois problemas separados (embora a posição de um autor sobre cada um deles tenha sido frequentemente lida como um compromisso a favor ou contra a teoria do valor de Marx de forma mais geral). Contudo, neste capítulo e no anterior, através do uso consistente do OCC como distinto do VCC, descobriu-se que os dois problemas estão intimamente relacionados entre si. Ambos estão preocupados com as tensões criadas pela integração da produção com a troca e, especialmente, com as consequências das diferenças ou mudanças nas condições de produção para a formação de preços em particular e os movimentos na troca de forma mais geral.
 \par 
\section{Dos valores aos preços de produção}
 \par 
É notável como, mesmo entre aqueles que simpatizam com Marx, a transformação de valores em preços de produção flutuou livre de outros “problemas” na economia política de Marx para se tornar um debate sobre a formação de preços (de equilíbrio). Não é de surpreender que a literatura sobre o problema da transformação seja vasta. O tratamento original é apresentado em Marx (1981a, pontos 1-2). A interpretação da transformação neste capítulo foi iniciada por Ben Fine (1983a), e é explicada e desenvolvida por Alfredo Saad-Filho (1997b; 2002, cap.{\color{blue}7}). Várias abordagens alternativas estão disponíveis; para uma visão geral, ver Simon Mohun (1995) e Alfredo Saad-Filho (2002, cap.{\color{blue}2}). As análises Sraffianas, rejeitando a teoria do valor como irrelevante e/ou errônea, são apresentadas de forma concisa em Ian Steedman (1977) - para críticas, ver os artigos de Ben Fine (1986) e Bob Rowthorn (1980), bem como Anwar Shaikh (1981, 1982). Gérard Duménil (1980) e Duncan Foley (1982) propuseram uma “nova interpretação” do problema, centrando-se no valor do dinheiro como meio de resolver os supostos enigmas de Marx. Isto é revisado criticamente por Ben Fine, Costas Lapavitsas e
 \par 
Alfredo Saad-Filho (2004) e Alfredo Saad-Filho (1996). Debates mais recentes sobre a natureza e definição de valor, com implicações diretas para o problema da transformação, podem ser encontrados nas revistas Cambridge Journal of Economics, Capital & Class, Historical Materialism e Science & Society. Mais uma vez, Moseley (2015) oferece a sua própria interpretação original.