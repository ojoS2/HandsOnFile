 
 \chapter{Introduction}  

 \label{Introduction}  
 
 
\par
 
 
 \textit{	A political party may fi ND that it has had a history, before it is fully aware of or agreed upon its own permanent tenets; it may have arrived at its actual formation through a succession of metamorphoses and adaptations, during which some issues have been superannuated, and new issues have arisen. What its fundamental tenets are, will probably be found only by careful examination of its behavior throughout its history and by examination of what its more thoughtful and philosophical minds have said on its behalf; and only accurate historical knowledge and judicious analysis will be able to discriminate between the permanent and the transitory; between those doctrines and principles which it must ever, and in all circumstances, maintain, or manifest itself a fraud, and those called forth by special circumstances, which are only intelligible and justifi able in the light of those circumstances.}  

 
\par
 
 
 
\par
 

 \textbf{\textit{	’T. S. Eliot, “The Literature of Politics”} }  

 
\par
 
Desde o início da era moderna, homens e mulheres em posições subordinadas marcharam contra os seus superiores no estado, na igreja, no local de trabalho e outras instituições hierárquicas. Reuniram-se sob diferentes bandeiras – o movimento operário, o feminismo, a abolição, o socialismo – e gritaram diferentes slogans: liberdade, igualdade, direitos, democracia, revolução. Em praticamente todos os casos, os seus superiores resistiram-lhes, de forma violenta e não violenta, legal e ilegalmente, aberta e secretamente. Essa marcha e diligência da democracia é a história da política moderna ou pelo menos uma das suas histórias.
 
\par
 
Este livro é sobre a segunda metade dessa história, a diligência e as ideias políticas – também chamadas de conservadoras, reacionárias, revanchistas, contra-revolucionárias – que surgem dela e lhe dão origem. Estas ideias, que ocupam o lado direito do espectro político, são forjadas na batalha. Sempre foram, pelo menos desde que surgiram como ideologias formais durante a Revolução Francesa, batalhas entre grupos sociais e não entre nações; grosso modo, entre aqueles que têm mais poder e aqueles que têm menos. Para entender essas ideias, temos que entender essa história. Pois é isso que é o conservadorismo: uma meditação – e uma representação teórica – da experiência sentida de ter poder, vê-lo ameaçado e tentar reconquistá-lo.
 
\par
 
Apesar das diferenças muito reais entre eles, os trabalhadores numa fábrica são como secretárias num escritório, camponeses numa mansão, escravos numa plantação – até mesmo esposas num casamento – enquanto vivem e trabalham em condições de poder desigual. Submetem-se e obedecem, atendendo às exigências dos seus administradores e senhores, maridos e senhores. Eles são disciplinados e punidos. Além disso, fazem muito e recebem pouco. Às vezes, a sua sorte é escolhida livremente – os trabalhadores contratam os seus empregadores, as mulheres contratam os seus maridos – mas as suas implicações raramente o são. Afinal, que contrato poderia relacionar os meandros, as dores diárias e o sofrimento contínuo de um emprego ou de um casamento? Ao longo da história americana, de facto, o contrato serviu frequentemente como um canal para coerções e restrições imprevistas, particularmente em instituições como o local de trabalho e a família, onde homens e mulheres passam grande parte das suas vidas. Os contratos de trabalho e de casamento têm sido interpretados por juízes, eles próprios favoráveis ​​aos interesses dos empregadores e dos maridos, como contendo todo o tipo de disposições não escritas e indesejadas de servidão com as quais as esposas e os trabalhadores consentem tacitamente, mesmo quando não têm conhecimento de tais disposições ou desejam estipular o contrário.
 {\color{blue} 1}  

 
\par
 
Até 1980, por exemplo, era legal em todos os estados da união que um marido estuprasse sua esposa.
 {\color{blue} 2}  
A justificativa para isso remonta a um tratado de 1736 do jurista inglês Matthew Hale. Quando uma mulher se casa, argumentou Hale, ela concorda implicitamente em “entregar-se desta forma [sexualmente] ao marido”. O consentimento dela é tácito, embora inconsciente, “que ela não pode retratar” durante a união. Tendo dito sim uma vez, ela nunca poderá dizer não. Ainda em 1957 – durante a era do Tribunal Warren – um tratado jurídico padrão poderia afirmar: “Um homem não comete violação ao ter relações sexuais com a sua esposa legítima, mesmo que o faça pela força e contra a vontade dela”. Se uma mulher (ou homem) tentasse incluir no contrato de casamento a exigência de que o consentimento expresso tivesse que ser dado para que o sexo pudesse prosseguir, os juízes eram obrigados pela lei comum a ignorá-lo ou anulá-lo. O consentimento implícito era uma característica estrutural do contrato que nenhuma das partes poderia alterar. Como a opção de saída do divórcio não estava amplamente disponível até à segunda metade do século XX, o contrato de casamento condenou as mulheres a serem servas sexuais dos seus maridos.
 {\color{blue} 3}  
Uma dinâmica semelhante estava presente no contrato de trabalho: os trabalhadores consentiam em ser contratados pelos seus empregadores, mas até ao século XX esse consentimento era interpretado pelos juízes como contendo disposições implícitas e irrevogáveis ​​de servidão; entretanto, a opção de saída de desistir não estava tão disponível, legal ou praticamente, como muitos poderiam pensar.
 {\color{blue} 4}  

 
\par
 
Ocasionalmente, porém, os subordinados deste mundo contestam o seu destino. Eles protestam contra as suas condições, escrevem cartas e petições, aderem a movimentos e fazem exigências. Os seus objectivo podem ser mínimos e discretos – melhores guardas de segurança nas máquinas das fábricas, o fim da violação conjugal – mas, ao expressá-los, levantam o espectro de uma mudança mais fundamental no poder. Eles deixam de ser servos ou suplicantes e passam a ser agentes, falando e agindo em seu próprio nome. Mais do que as próprias reformas, é esta afirmação de agência por parte da classe subjugada – o aparecimento de uma voz de exigência insistente e independente – que irrita os seus superiores. A Reforma Agrária da Guatemala de 1952 redistribuiu um milhão e meio de acres de terra a 100 mil famílias camponesas. Isso não era nada, nas mentes das classes dominantes do país, comparado com a agitação do debate político que o projeto de lei parecia desencadear. Os reformadores progressistas, queixou-se o arcebispo da Guatemala, enviaram camponeses locais “dotados de facilidade com as palavras” para a capital, onde lhes foram dadas oportunidades “de falar em público”. Esse foi o grande mal da Reforma Agrária.
 {\color{blue} 5}  

 
\par
 
No seu último grande discurso ao Senado, John C. Calhoun, antigo vice-presidente e principal porta-voz da causa do Sul, identificou a decisão do Congresso, em meados da década de 1830, de receber petições abolicionistas como o momento em que a nação se estabeleceu. Em um curso irreversível de confronto sobre a escravidão. Numa carreira de quatro décadas que assistiu a derrotas para a posição dos proprietários de escravos, como a Tarifa das Abominações, a Crise de Nulificação e a Lei da Força, o mero aparecimento do discurso escravista na capital do país destacou-se para o moribundo Calhoun como o sinal de que a revolução havia começado.
 {\color{blue} 6}  
E quando, meio século mais tarde, os sucessores de Calhoun procuraram colocar o gênio abolicionista de volta na garrafa, foi esta mesma afirmação de agência negra que eles visaram. Explicando a proliferação em todo o Sul nas décadas de 1890 e 1900 de convenções constitucionais que restringiam o direito de voto, um delegado de uma dessas convenções declarou: “O grande princípio subjacente deste movimento da Convenção. . . Foi a eliminação do negro da política deste Estado.”
 {\color{blue} 7}  

 
\par
 
A história trabalhista americana está repleta de reclamações semelhantes das classes empregadoras e seus aliados no governo: não que os trabalhadores sindicalizados sejam violentos, perturbadores ou UNPROFOR tabu, mas que eles sejam independentes e auto-organizados. De fato, sua auto-organização é tão potente que ameaça — aos olhos de seus superiores — tornar supérfluos o empregador e o estado. Durante a Grande Revolta de 1877, os trabalhadores ferroviários em greve em St. Louis começaram a operar os trens eles mesmos. Temendo que o público pudesse concluir que os trabalhadores eram capazes de administrar a ferrovia, os proprietários tentaram impedi-los — na verdade, lançando uma greve própria para provar que eram os proprietários, e somente os proprietários, que poderiam fazer os trens funcionarem no horário. Durante a greve geral de Seattle em 1919, os trabalhadores se esforçaram muito para fornecer serviços básicos do governo, incluindo lei e ordem. Eles foram tão bem-sucedidos que o prefeito concluiu que era isso, a capacidade independente dos trabalhadores de limitar a violência e a anarquia, que representava a maior ameaça.
 
\par
 

 \textbf{\textit{The so-called sympathetic Seattle strike was an attempted revolution. That there was no violence does not alter the fact. . . . True, there were no fl ashing guns, no bombs, no killings. Revolution, I repeat, doesn’t need violence. The general strike, as practiced in Seattle, is of itself the weapon of revolution, all the more dangerous because quiet. . . . That is to say, it puts the government out of operation. And that is all there is to revolt— no matter how achieved. {{\color{blue} 8} } } }  
 
 
\par
 
No século XX, os juízes denunciaram regularmente os trabalhadores sindicalizados por formularem as suas próprias definições de direitos e por compilarem o seu próprio registo de regras de chão de fábrica. Trabalhadores como estes, afirmou um tribunal federal, viam-se como “expoentes de alguma lei superior a essa. . . Administrado por tribunais.” Exerciam “poderes pertencentes apenas ao Governo”, declarou o Supremo Tribunal, constituindo-se como um “tribunal autonomeado” da lei e da ordem.
 {\color{blue} 9}  
O conservadorismo é a voz teórica desta animosidade contra a agência das classes subordinadas. Fornece o argumento mais consistente e profundo sobre a razão pela qual as ordens inferiores não devem ser autorizadas a exercer a sua vontade independente, porque não devem ser autorizadas a governar a si mesmas ou ao sistema político? A submissão é o seu primeiro dever, a agência, a prerrogativa da elite.
 
\par
 
Embora seja frequentemente afirmado que a esquerda defende a igualdade enquanto a direita defende a liberdade, essa noção distorce o desacordo real entre direita e esquerda. Historicamente, o conservador favoreceu a liberdade para as ordens superiores e a restrição para as ordens inferiores. O que o conservador vê e não gosta na igualdade, em outras palavras, não é uma ameaça à liberdade, mas sua extensão. Pois nessa extensão, ele vê uma perda de sua própria liberdade. “Estamos todos de acordo quanto à nossa própria liberdade”, declarou Samuel Johnson. “Mas não estamos de acordo quanto à liberdade dos outros: pois na proporção em que tomamos, os outros devem perder. Acredito que dificilmente desejamos que a multidão tenha liberdade para nos governar.”
 {\color{blue} 10}  
Tal foi a ameaça que Edmund Burke viu na Revolução Francesa: não apenas uma expropriação de propriedade ou uma explosão de violência, mas uma inversão das obrigações de deferência e comando. “Os niveladores”, afirmou ele, “apenas mudam e pervertem a ordem natural das coisas”.
 
\par
 

 \textbf{\textit{The occupation of a hairdresser, or of a working tallow-chandler, cannot be a matter of honor to any person—to say nothing of a number of other more servile employments. Such descriptions of men ought not to suff her oppression from the state; but the state suffers oppression, if such as they, either individually or collectively, are permitted to rule. {{\color{blue} 11} } } }  
 
 
\par
 
Em virtude de serem membros de um sistema político, admitiu Burke, os homens tinham muitos direitos – aos frutos do seu trabalho, à sua herança, à educação e muito mais. Mas o único direito que ele se recusou a conceder a todos os homens foi a “partilha de poder, autoridade e direção” que eles poderiam pensar que deveriam ter “na gestão do Estado”.
 {\color{blue} 12}  

 
\par
 
Mesmo quando as exigências da esquerda se deslocam para a esfera econômica, a ameaça da extensão da liberdade é grande. Se as mulheres e os trabalhadores receberem os recursos econômicos para fazerem escolhas independentes, serão livres para não obedecer aos seus maridos e empregadores.
 
\par
 
É por isso que Lawrence Mead, um dos principais opositores intelectuais do Estado-providência nas décadas de 1980 e 1990, declarou que o beneficiário da assistência social “deve tornar-se menos livre em certos sentidos, em vez de mais”.
 {\color{blue} 13}  
Para o conservador, a igualdade pressagia mais do que uma redistribuição de recursos, oportunidades e resultados – embora ele certamente também não goste destes.
 {\color{blue} 14}  
O que igualdade significa, em última análise, é uma rotação na sede do poder.
 
\par
 
O conservador não está errado ao interpretar a ameaça da esquerda nestes termos. Antes de morrer, G. A. Cohen, uma das vozes mais perspicazes do marxismo contemporâneo, defendeu que grande parte do programa de redistribuição econômica da esquerda poderia ser entendido como implicando não um sacrifício da liberdade em prol da igualdade, mas uma extensão da liberdade do poucos para muitos.
 {\color{blue} 15}  
E, de facto, os grandes movimentos modernos de emancipação – da abolição ao feminismo e à luta pelos direitos dos trabalhadores e pelos direitos civis – sempre postularam um nexo entre liberdade e igualdade. Saindo da família, da fábrica e do campo, onde a liberdade e a desigualdade são o outro lado da mesma moeda, eles fizeram da liberdade e da igualdade as partes irredutíveis, mas que se reforçam mutuamente, de um único todo. A ligação entre liberdade e igualdade não tornou o argumento a favor da redistribuição mais palatável para a direita. Tal como um conservador se queixou da visão de John Dewey sobre a social-democracia: “As definições de liberdade e de igualdade têm sido tão manipuladas que ambas se referem aproximadamente à mesma condição”.
 {\color{blue} 16}  
Longe de ser um truque progressista, contudo, esta síntese de liberdade e igualdade é um postulado central da política de emancipação. Se a política está conforme o postulado é, obviamente, outra história. Mas para o conservador, a preocupação é menos a traição do postulado do que o seu cumprimento.
 
\par
 
Uma das razões pelas quais o exercício de agência do subordinado agita tanto a imaginação conservadora é que ele ocorre num ambiente íntimo. Cada grande explosão política – a tomada da Bastilha, a tomada do Palácio de inverno, a Marcha sobre Washington – é desencadeada por um estopim privado: a disputa pelos direitos e pela posição na família, na fábrica e no campo. Políticos e partidos falam de constituição e emendas, direitos naturais e privilégios herdados. Mas o verdadeiro tema das suas deliberações é a vida privada do poder. “Aqui está o segredo da oposição à igualdade das mulheres no Estado”, escreveu Elizabeth Cady Stanton. “Os homens não estão preparados para reconhecer isso em casa.”
 {\color{blue} 17}  
Por trás do motim nas ruas ou do debate no Parlamento está a empregada que responde à sua patroa, o trabalhador que desobedece ao seu patrão. É por isso que os nossos argumentos políticos – não apenas sobre a família, mas também sobre o Estado-providência, os direitos civis e muito mais – podem ser tão explosivos: tocam nas relações de poder mais pessoais. É também por isso que tantas vezes cabe aos nossos romancistas explicar-nos a nossa política. No auge do movimento pelos direitos civis, James Baldwin viajou para Tallahassee. Ali, num aperto de mão imaginário, encontrou a transcrição oculta de uma crise constitucional. 18
 
\par
 

 \textbf{\textit{I am the only Negro passenger at Tallahassee’s shambles of an airport. It is an oppressively sunny day. A black chauffeur, leading a small dog on a leash, is meeting his white employer. He is attentive to the dog, covertly very aware of me and respectful of her in a curiously watchful, waiting way. She is mid- damaged, beaming and powdery-faced, delighted to see both the beings who make her life agreeable. I am sure that it has never occurred to her that either of them has the ability to judge her or would judge her harshly. She might almost, as she goes toward her chauffeur, be greeting a friend. No friend could make her face brighter. If she were smiling at me that way I would expect to shake her hand. But if I should put out my hand, panic, bafflement, and horror would then overtake that} }  
 
 
\par
 

 
\par
 

 \textbf{\textit{Face, the atmosphere would darken, and danger, even the threat of death, would immediately fill the air.} }  
 
 
\par
 

 
\par
 

 \textbf{\textit{On such small signs and symbols does the southern cabala} }  
 
 
\par
 
O conflito sobre a escravidão americana — o precedente iminente para esse conjunto de peças da imaginação de Baldwin — deu à ERS um exemplo instrutivo. Uma das características distintivas da escravidão nos Estados Unidos é que, diferentemente dos escravos no Caribe ou dos servos na Rússia, muitos escravos no Sul viviam em pequenas propriedades com seus senhores residentes. Os senhores sabiam os nomes de seus escravos; rastreavam seus nascimentos, casamentos e mortes; e realizavam festas para homenagear essas datas. A interação pessoal entre senhor e escravo era incomparável, levando Frederick Law Olmsted, em visita, a comentar sobre a “estreita coabitação e associação de negros e brancos” na Virgínia, a “familiaridade e proximidade da intimidade que teriam sido notadas com espanto, se não com manifesto desagrado, em quase qualquer companhia casual no Norte”.
 {\color{blue} 20}  
Apenas as “relações de marido e mulher, pais e filhos, irmão e irmã”, escreveu o apologista da escravatura Thomas Dew, produziram “um laço mais estreito” do que o de senhor e escravo; a última relação, declarou William Harvey, outro defensor da escravatura, era “uma das relações mais íntimas da sociedade”.
 {\color{blue} 21}  
Por outro lado, depois da abolição da escravatura, muitos brancos lamentaram o frio nas relações entre as raças. “Gosto do negro”, disse um mississipi em 1918, “mas o vínculo entre nós não somos tão estreito como era entre meu pai e seus escravos”.
 {\color{blue} 22}  

 
\par
 
Grande parte desta conversa era propaganda e autoilusão, é claro, mas num aspecto não era: a proximidade do senhor com o escravo criava um modo de governo excepcionalmente pessoal. Os senhores concebiam e aplicavam regras “excepcionalmente detalhadas” para seus escravos, ditando quando eles deveriam se levantar, comer, trabalhar, dormir, cuidar do jardim, visitar e orar. Os senhores decidiam as companheiras e os casamentos dos seus escravos. Eles deram nomes aos filhos e, quando o mercado assim o determinou, separaram-nos dos pais. E embora os senhores – bem como os seus filhos e feitores – se aproveitassem dos corpos das suas escravas sempre que desejassem, eles acharam adequado patrulhar e punir toda e qualquer relação sexual entre os seus escravos.
 {\color{blue} 23}  
Vivendo com seus escravos, os senhores tinham meios diretos para controlar seu comportamento e um mapa detalhado de todo o comportamento que deveria ser controlado.
 
\par
 
As consequências desta proximidade foram sentidas não apenas pelo escravo, mas também pelo senhor. Vivendo todos os dias com sua mestria, ele se identificou inteiramente com ela. Esta identificação era tão completa que qualquer sinal de desobediência do escravo – muito menos de sua emancipação – era visto como um ataque intolerável à sua pessoa. Quando Calhoun declarou que a escravatura “cresceu com a nossa sociedade e instituições, e está tão interligada com elas, que a destruir seria destruir-nos como povo”, ele não se referia apenas à sociedade no seu conjunto ou abstrato.
 {\color{blue} 24}  
Ele estava pensando em homens individuais absortos na experiência cotidiana de governar outros homens e mulheres. Tire essa experiência e você destruirá não apenas o mestre, mas também o homem – e os muitos homens que procuraram se tornar, ou pensavam que já eram, o mestre.
 
\par
 
Como o mestre colocava tão pouca distância entre ele e seu domínio, ele faria de tudo para manter suas posses. Em todas as Américas, os proprietários de escravos defenderam os seus privilégios, mas em nenhum lugar com a intensidade ou a violência da classe senhorial no Sul. Fora do Sul, escreveu C. Van Wood-uari, o fim da escravatura foi “a liquidação de um investimento”. Por dentro, foi “a morte de uma sociedade”.
 {\color{blue} 25}  
E quando, depois da Guerra Civil, a classe dominante lutou com igual ferocidade para restaurar os seus privilégios e poder, foi a proximidade do comando, a proximidade do governo, que prevaleceu na sua mente. Como disse Henry Mcneal Turner, um republicano negro na Geórgia, em 1871: “Eles não se importam tanto com o fato de o Congresso admitir negros nos seus corredores. . . Mas eles não querem os negros sobre eles em casa.” Cem anos mais tarde, um meeiro negro no Mississipi ainda recorreria à expressão mais doméstica para descrever as relações entre negros e brancos: “Tínhamos de cuidar deles como os nossos filhos cuidam de nós”.
 {\color{blue} 26}  

 
\par
 
Quando o conservador olha para um movimento democrático a partir de baixo, isto (e o exercício da agência) é o que ele vê: uma terrível perturbação na vida privada do poder. Testemunhando a eleição de Thomas Jefferson em 1800, Theodore Sedgwick lamentou: “A aristocracia da virtude está destruída; a influência pessoal chegou ao fim.”
 {\color{blue} 27}  
Às vezes, o conservador está pessoalmente implicado nessa vida, às vezes não. Independentemente disso, é sua apreensão da queixa privada por trás da comoção pública que empresta à sua teoria sua engenhosidade tátil e ferocidade moral. “O verdadeiro objetivo” da Revolução Francesa, Burke disse ao Parlamento em 1790, é “quebrar todas essas conexões, naturais e civis, que regulam e mantêm unida a comunidade por uma cadeia de subordinação; levantar soldados contra seus oficiais; servos contra seus mestres; comerciantes contra seus clientes; artífices contra seus empregadores; inquilinos contra seus senhorios; curas contra seus bispos; e crianças contra seus pais.”
 {\color{blue} 28}  
A insubordinação pessoal rapidamente se tornou um tema regular e consistente dos pronunciamentos de Burke sobre o desenrolar dos acontecimentos em França. Um ano depois, ele escreveu numa carta que, devido à Revolução, “nenhuma casa está a salvo dos seus servos, nenhum oficial dos seus soldados, e nenhum Estado ou constituição da conspiração e da insurreição”.
 {\color{blue} 29}  
Num outro discurso perante o Parlamento em 1791, ele declarou que “uma constituição fundada no que foi chamado de direitos do homem” abriu a “caixa de Pandora” em todo o mundo, incluindo o Haiti: “Os negros levantaram-se contra os brancos, os brancos contra os negros, e cada um contra outro em hostilidade assassina; a subordinação foi destruída.”
 {\color{blue} 30}  
Nada para os jacobinos, declarou ele no final da vida, era digno “do nome de virtude pública, a menos que indique violência sobre o privado”.
 {\color{blue} 31}  

 
\par
 
Tão poderosa é essa visão de erupção privada que pode transformar um homem reformista num homem reacionário. Educado no Iluminismo, John Adams acreditava que o “consentimento do povo” era “o único fundamento moral do governo”.
 {\color{blue} 32}  
Mas quando a sua esposa sugeriu que uma versão silenciosa destes princípios fosse estendida à família, ele não gostou. “E, a propósito”, escreveu-lhe Abigail, “no novo código de leis que suponho que será necessário que você faça, desejo que você se lembre das damas e seja mais generoso e favorável a elas do que seus ancestrais. Não coloquem esse poder ilimitado nas mãos dos maridos. Lembre-se, todos os homens seriam tiranos se pudessem.”
 {\color{blue} 33}  
A resposta do marido:
 
\par
 

 \textbf{\textit{We have been told that our struggle has loosened the bands of government everywhere; that children and apprentices were disobedient; that schools and colleges were grown turbulent; that Indians slighted their guardians, and Negroes grew insolent to their masters. But your letter was the first intimation that another tribe, more numerous and powerful than all the rest, were grown discontented.} }  
 
 
\par
 
Embora ele tenha fermentado sua resposta com brincadeiras divertidas, ele rezou para que George Washington o protegesse do “despotismo da anágua”.
 {\color{blue} 34}  
— Adams ficou claramente abalado com esta aparência de democracia na esfera privada. Numa carta a James Sullivan, ele temia que a Revolução “confundisse e destruísse todas as distinções”, desencadeando em toda a sociedade um espírito de insubordinação tão intenso que toda a ordem seria dissolvida. “Não haverá fim para isso.”
 {\color{blue} 35}  
Por mais democrático que fosse o Estado, era imperativo que a sociedade continuasse a ser uma federação de domínios privados, onde os maridos governavam as esposas, os mestres governavam os aprendizes e cada um “deveria conhecer o seu lugar e ser obrigado a mantê-lo”.
 {\color{blue} 36}  

 
\par
 
Historicamente, o conservador tem procurado impedir a marcha da democracia tanto na esfera pública como na esfera privada, no pressuposto de que os avanços numa estimulam necessariamente os avanços na outra. “Para manter o Estado fora das mãos do povo”, escreveu o monarquista francês Louis de Bonald, “é necessário manter a família fora das mãos das mulheres e das crianças”.
 {\color{blue} 37}  
Mesmo nos Estados Unidos, este esforço tem produzido frutos periodicamente. Apesar da nossa narrativa esbranquiçada sobre a ascensão constante da democracia, o historiador Alexander Vassar demonstrou que a luta pelo voto nos Estados Unidos tem sido tanto uma história de refracção e contração como de progresso e expansão, “com tensões e apreensões de classe” por parte das elites políticas e econômicas constituindo “o obstáculo mais importante ao sufrágio universal. . . Do final do século XVIII até a década de 1960.”
 {\color{blue} 38}  

 
\par
 
Ainda assim, a posição mais profunda e profética da direita tem sido a de Adams: ceda o domínio do público, se for necessário, mantenha-se firme no privado. Permitir que homens e mulheres se tornem cidadãos democráticos do Estado; certifique-se de que eles permaneçam súditos feudais na família, na fábrica e no campo. A prioridade do argumento político conservador tem sido a manutenção de regimes privados de poder – mesmo à custa da força e da integridade do Estado. Vemos esta aritmética política em ação na decisão de um tribunal federalista em Massachusetts de que uma mulher legalista que fugiu da Revolução era ajudante de seu marido e, portanto, não deveria ser responsabilizada pela fuga e não deveria ter suas configurações de propriedade. Capturado pelo estado; na recusa dos proprietários de escravos do Sul em ceder os seus escravos à causa confederada; e a insistência mais recente do Supremo Tribunal de que as mulheres não poderiam ser legalmente obrigadas a fazer parte dos júris porque “ainda são considerados o centro do lar e da vida familiar”, com as suas “próprias responsabilidades especiais”.
 {\color{blue} 39}  
O conservadorismo, então, não é um compromisso com um governo e liberdade limitados – ou uma cautela em relação à mudança, uma crença na reforma evolutiva ou uma política de virtude. Estes podem ser os subprodutos do conservadorismo, um ou mais dos seus modos de expressão historicamente específicos e em constante mudança. Mas eles não são o seu propósito animador. O conservadorismo também não é uma fusão improvisada de capitalistas, cristãos e guerreiros, pois essa fusão é impulsionada por uma força mais elementar – a oposição à libertação de homens e mulheres dos grilhões dos seus superiores, particularmente na esfera privada. Tal visão pode parecer muito distante da defesa libertária do mercado livre, com a sua celebração do indivíduo atomista e autônomo. Mas não é. Quando o libertário olha para a sociedade, ele não vê indivíduos isolados; ele vê grupos privados, muitas vezes hierárquicos, onde um pai governa sua família e um proprietário seus empregados.
 {\color{blue} 40}  

 
\par
 
Não há uma simples defesa do próprio lugar e dos privilégios – o conservador, como já disse, pode ou não estar diretamente envolvido, ou beneficiar das práticas de governo que defende; muitos, como veremos, não o são – a posição conservadora deriva de uma convicção genuína de que um mundo assim emancipado será feio, brutal, vil e monótono. Faltará a excelência de um mundo onde o homem melhor comanda o pior. Quando Burke acrescenta, na carta citada acima, que o “grande objetivo” da Revolução é “erradicar aquela coisa chamada Aristocrata ou Nobre e Cavalheiro”, ele não está simplesmente se referindo ao poder da nobreza; ele também está se referindo à distinção que o poder traz ao mundo.
 {\color{blue} 41}  
Se o poder acabar, a distinção vai junto. Esta visão da ligação entre excelência e governo é o que une na América do pós-guerra aquela aliança improvável do libertário, com a sua visão do poder ilimitado do empregador no local de trabalho; o tradicionalista, com sua visão do governo do pai no lar; e o estatista, com a visão de um líder heroico pressionando a mão sobre a face da terra. Cada um à sua maneira subscreve esta afirmação típica, do século XIX, do credo conservador: “Obedecer a um verdadeiro superior. . . É uma das mais importantes de todas as virtudes – uma virtude essencial para a realização de qualquer coisa grande e duradoura.”
 {\color{blue} 42}  

 
\par
 
A noção de que as ideias conservadoras são um modo de prática contra-revolucionária é susceptível de levantar algumas sobrancelhas, até mesmo arrepios. Há muito que é um axioma da esquerda que a defesa do poder e dos privilégios é um empreendimento desprovido de ideias. “A história intelectual”, afirma um estudo recente sobre o conservadorismo americano, “nunca é indesejada”, mas “não é a abordagem mais direta para explicar o poder do conservadorismo na América”.
 {\color{blue} 43}  
Os escritores liberais sempre retrataram a política de direita como um pântano emocional e não como um movimento de opinião ponderada: Thomas Paine afirmou que a contra-revolução implicava “uma obliteração do conhecimento”; Lionel Trilling descreveu o conservadorismo americano como uma mistura de “gestos mentais irritáveis ​​que procuram assemelhar-se a ideias”; Robert Paxton chamou o fascismo de “ar AFF do intestino”, e não “do cérebro”.
 {\color{blue} 44}  
Os conservadores, por sua vez, tendem a concordar.
 {\color{blue} 45}  
Afinal, foi Palmerston, quando ainda era conservador, quem primeiro atribuiu o epíteto de “estúpido” ao Partido Conservador. Desempenhando o papel de proprietários rurais estúpidos, os conservadores abraçaram a posição de F. J. C. Earnshaw de que “é geralmente suficiente para fins práticos que os conservadores, sem dizer nada, apenas se sentem e pensem, ou mesmo que simplesmente se sentem”.
 {\color{blue} 46}  
Embora as conotações aristocráticas desse discurso já não ressoem, o conservador ainda mantém o rótulo de inculto e iletrado; faz parte do seu charme populista e apelo demótico. Como observa o conservador Washington Times, os republicanos “muitas vezes se autodenominam o ‘partido estúpido’”.
 {\color{blue} 47}  
Nada, como veremos, poderia estar mais longe da verdade. O conservadorismo é uma práxis movida por ideias, e nenhuma quantidade de presunção da direita ou polêmica da esquerda pode reduzir, ou apagar o catálogo de ideias que ali se encontra.
 
\par
 
Os próprios conservadores ficarão provavelmente desanimados com este argumento por uma razão diferente: ameaça a pureza e a profundidade das ideias conservadoras. Para muitos, a palavra “reação” conota uma busca impensada e humilde pelo poder.
 {\color{blue} 48}  
Mas a reação não é ex real. Começa a partir de uma posição de princípio – de que alguns estão aptos, e, portanto devem, governar outros – e depois recalibra esse princípio à luz de um desafio democrático vindo de baixo. Esta recalibração não é uma tarefa fácil, pois tais desafios tendem, pela sua própria natureza, a refutar o princípio. Afinal de contas, se uma classe dominante está verdadeiramente preparada para governar, por que e como permitiu que surgisse um desafio ao seu poder? O que o surgimento de um diz sobre a aptidão do outro?
 {\color{blue} 49}  
O conservador enfrenta um obstáculo adicional: como defender um princípio de governo num mundo onde nada é sólido, tudo está em fluxo? Desde o momento em que o conservadorismo entrou em cena, teve de enfrentar o declínio das ideias antigas e medievais de um universo ordenado, no qual hierarquias permanentes de poder refletiam a estrutura eterna do cosmos. A derrubada do antigo regime revela não só a fraqueza e a incompetência dos seus líderes, mas também uma verdade maior sobre a falta de design no mundo. (A ideia de que o conservadorismo reflete a revelação de que o mundo não tem hierarquias naturais pode parecer estranha na nossa era do Design Inteligente. Mas, como Kevin Matt son. e outros apontaram, o Design Inteligente não se baseia no mesmo tipo de suposição medieval de uma estrutura eterna e firme para o universo, e há mais do que um toque de relativismo e ceticismo nos seus argumentos. Na verdade, um dos principais proponentes do Design Inteligente afirmou que, embora “não seja pós-modernista”, ele “aprendeu muito” com o pós-modernismo.
 {\color{blue} 50}  
 Reconstruir o antigo regime diante de uma fé decrescente em hierarquias permanentes provou ser um feito difícil. Não surpreendentemente, também produziu algumas das obras mais notáveis ​​do pensamento moderno.
 
\par
 
Mas há outra razão pela qual devemos ser cautelosos relativamente ao esforço para rejeitar o impulso reacionário do conservadorismo, e essa razão é o testemunho da própria tradição. Desde Burke, tem sido motivo de orgulho entre os conservadores que o seu modo de pensamento seja contingente. Ao contrário dos seus adversários de esquerda, eles não elaboram um plano antes dos acontecimentos. Eles leem situações e circunstâncias, não textos e tomos; seu modo preferido é a adaptação e a insinuação, em vez da afirmação e da declamação. Há uma certa verdade nesta afirmação, como veremos: a mente conservadora é extraordinariamente flexível, alerta às mudanças no contexto e na sorte muito antes de os outros perceberem que estão a ocorrer. Com a sua profunda consciência da passagem do tempo, o conservador possui um virtuosismo tático que poucos conseguem igualar. Parece lógico que o conservadorismo esteja íntimo e com as suas antenas sempre sensíveis aos movimentos e contra-movimentos do poder esboçados acima. Estas são, como já disse, a história da política moderna, e pareceria estranho se uma mente tão sintonizada com as contingências circundantes não fossassem bem versada nessa história. Não apenas bem versado, mas despertado e despertado por ela como por nenhuma outra história.
 
\par
 
Na verdade, desde a afirmação de Burke de que ele e a sua turma tinham sido “alarmados com a verdadeira Exxon” pela Revolução Francesa até à admissão de Russell Kirk de que o conservadorismo é um “sistema de ideias” que “tem sustentado os homens. . . Na sua resistência contra as teorias radicais e a transformação social desde o início da Revolução Francesa”, o conservador tem afirmado consistentemente que o seu conhecimento é produzido em reação à esquerda.
 {\color{blue} 51}  
(Burke estabeleceria como seu “fundamento” a noção de que “nunca maior” “existiu” um mal do que a Revolução Francesa.)
 {\color{blue} 52}  
Às vezes, essa afirmação foi explícita. Três vezes primeiro-ministro, Salisbury escreveu em 1859 que “hostilidade ao radicalismo, hostilidade incessante e implacável, é a definição essencial do conservadorismo. O medo de que os Radicais possam triunfar é a única causa final que o Partido Conservador pode defender para a sua própria existência.”
 {\color{blue} 53}  
Mais de meio século depois, seu filho Hugh Cecil – entre outras coisas, padrinho de casamento de Winston Churchill e reitor de Eton – reafirmou a posição do pai: “Acho que o governo descobrirá no final que só há uma maneira de derrotar as táticas revolucionárias e isso é apresentar um corpo organizado de pensamento que seja não revolucionário. Esse corpo de pensamento eu chamo de conservadorismo.”
 {\color{blue} 54}  
Outros, como Peel, seguiram um caminho mais tortuoso para chegar ao mesmo lugar:
 
\par
 

 \textbf{\textit{My object for some years past, that which I have most earnestly labored to accomplish, has been to lay the foundation of a great party, which, existing in the House of Commons, and deriving its strength from the popular will, should diminish the risk and deaden the shock of a collision between the two deliberative branches of the legislature—which should enable us to check the too importunate eagerness of well-intending men, for hasty and precipitate changes in the constitution and laws of the country, and by which we should be enabled to say, with a voice of authority, to the restless spirit of revolutionary change, “Here are thy bounds, and here shall thy vibrations cease.” {{\color{blue} 55} } } }  
 
 
\par
 
Para não pensarmos que tais sentimentos – e circunlocuções – são peculiarmente ingleses, consideremos como o historiador da corte da direita americana abordou a questão em 1976. “O que é o conservadorismo?” George Nash perguntou em seu agora clássico The Conservative Intellectual Movement in America zinco 1945. Depois de uma página de hesitação - o conservadorismo resiste à definição, “varia enormemente com o tempo e o lugar” (que ideia política não varia?), não deveria ser“ confundido com a Direita Radical” – Nash decidiu-se por uma resposta que poderia ter sido dada (na verdade, foi dada) por Peel, Salisbury e seu filho, Kirk, e a maioria dos pensadores da Direita Radical. O conservadorismo, disse ele, é definido pela “resistência a certas forças percebidas como esquerdistas, revolucionárias e profundamente subversivas em relação ao que os conservadores da época consideravam digno de estimar, defender e talvez morrer”.
 {\color{blue} 56}  

 
\par
 
Estas são as profissões explícitas do credo contra-revolucionário. Mais interessantes são as declarações implícitas, onde a antipatia pelo radicalismo e pelas reformas está incorporada na própria sintaxe do argumento. Tomemos a famosa definição de Michael Makeshift em seu ensaio “On Being Conservative”: “Ser conservador, então, é preferir o familiar ao desconhecido, preferir o experimentado ao não experimentado, o fato ao mistério, o real ao possível, o limitado para o ilimitado, o próximo para o distante, o suficiente para o superabundante, o conveniente para o perfeito, o riso presente para a felicidade utópica. Parece que não se pode desfrutar dos fatos e do mistério, do próximo e do distante, do riso e da felicidade. É preciso escolher. Longe de afirmar uma simples hierarquia de preferências, o ou /, ou de Makeshort sinaliza que estamos num terreno existencial, onde a escolha não é entre algo e o seu oposto, mas entre algo e a sua negação. O conservador desfrutaria de coisas familiares na ausência de forças que procurassem a sua destruição, admite Makeshift, mas o seu prazer “será mais forte quando” for “combinado com um risco evidente de perda”. O conservador é um “homem que tem plena consciência de ter algo a perder e do qual aprendeu a cuidar”. E embora Makeshift sugira que tais perdas podem ser provocadas por uma variedade de forças, os engenheiros parecem invariavelmente trabalhar à esquerda. (Marx e Engels são “os autores do mais estupendo do nosso racionalismo político”, escreve ele noutro lugar. “Nada... se compara ao” seu utopismo abstrato.) Por essa razão, “não é de todo inconsistente ser conservador”. Em relação ao governo e radical em relação a quase todas as outras atividades.”
 {\color{blue} 57}  
Nada inconsistente — ou completamente necessário? O radicalismo é a razão de ser do conservadorismo; se ele vai, o conservadorismo também vai.
 {\color{blue} 58}  
Mesmo quando o conservador procura libertar-se deste diálogo com a esquerda, ele não consegue, pois os seus motivos mais líricos – mudança orgânica, conhecimento tácito, liberdade ordenada, prudência e precedente – são quase inaudíveis sem o apelo e a resposta da esquerda. Como Disraeli descobriu na sua Vindicação da Constituição Inglesa (1835), é apenas em contraste com um suposto racionalismo revolucionário que a invocação da sabedoria antiga e tácita pode ter alguma influência na mente moderna.
 
\par
 

 \textbf{\textit{The formation of a free government on an extensive scale, while it is assuredly one of the most interesting problems of humanity, is certainly the greatest achievement of human wit. Perhaps I should rather term it a superhuman achievement; for it requires such refined prudence, such comprehensive knowledge, and such perspicacious sagacity, united with such almost illimitable powers of combination, that it is nearly in vain to hope for qualities so rare to be congregated in a solitary mind. Assuredly this sum mum bonus is not to be found ensconced behind a revolutionary barricade, or floating in the bloody gutters of an incendiary metropolis. It cannot be scribbled down—this great invention—in a morning on the envelope of a letter by some charter-concocting monarch, or sketched with ludicrous facility in the conceited commonplace book of a Utilitarian sage. {{\color{blue} 59} } } }  
 
 
\par
 
Há mais nesta estrutura antagônica de argumento do que os simples anticorpos da política partidária, a tomada de posição de oposição que é um requisito para vencer eleições. Como argumentou Karl Mannheim, o que distingue o conservadorismo do tradicionalismo – a tendência “vegetativa” universal de permanecer apegado às coisas como elas são, que se manifesta em comportamentos não políticos, como a recusa em comprar um novo par de calças até que o par atual esteja em pedaços além do limite. Reparação – é que o conservadorismo é um esforço deliberado e consciente para preservar ou recordar “aquelas formas de experiência que já não podem ser obtidas de uma forma autêntica”. O conservadorismo “torna-se consciente e reflexivo quando aparecem em cena outros modos de vida e de pensamento, contra os quais é obrigado a pegar em armas na luta ideológica”.
 {\color{blue} 60}  
Onde o tradicionalista pode tomar os objetos de desejo como garantidos — ele pode apreciá-los como se estivessem à mão porque estão à mão — o conservador não pode. Ele busca apreciá-los precisamente como estão sendo — ou foram — tirados. Se ele espera apreciá-los novamente, ele deve contestar seu desinvestimento no domínio público. Ele deve falar deles em uma linguagem que seja politicamente útil e inteligível. Mas assim que esses objetos entram no meio do discurso político, eles deixam de ser itens de experiência vivida e se tornam incidentes de uma ideologia. Eles são envolvidos em uma narrativa de perda — na qual o revolucionário ou reformista desempenha um papel necessário — e apresentados em um programa de recuperação. O que era tácito se torna articulado, o que era fluido se torna formal, o que era prática se torna polêmico.
 {\color{blue} 61}  
Mesmo que a teoria seja um hino à prática – como muitas vezes é o conservadorismo – ela não pode escapar de se tornar uma polêmica. O conservador mais exigente que se dignasse a entrar na rua é obrigado pela esquerda a pegar numa pedra do pavimento e atirá-la contra as barricadas. Como disse Lord Hail chão em seu Caso pelo Conservadorismo de 1947:
 
\par
 

 \textbf{\textit{Conservatives do not believe that political struggle is the most important thing in life. In this they diff her from Communists, Socialists, Nazis, Fascists, Social Creditors and most members of the British Labour Party. The simplest among them prefer fox-hunting—the wisest religion. To the great majority of Conservatives, religion, art, study, family, country, friends, music, fun, duty, all the joy and riches of existence of which the poor no less than the rich are the indefeasible freeholders, all these are higher in the scale than their handmaiden, the political struggle. This makes them easy to defeat—at first. But, once, defeated, they will hold to this belief with the fanaticism of a Crusader and the doggedness of an Englishman. {{\color{blue} 62} } } }  
 
 
\par
 
Dado que há tanta confusão sobre a oposição do conservadorismo à esquerda, é importante que sejamos claros sobre o que o conservador é e o que não é a oposição na esquerda. Não é uma mudança abstrata. Nenhum conservador se opõe à mudança como tal ou defende a ordem como tal. O conservador defende ordens específicas – regimes de governo hierárquicos, muitas vezes privados – partindo do pressuposto, em parte, de que hierarquia é ordem. “A ordem não pode ser obtida”, declarou Johnson, “mas por subordinação”.
 {\color{blue} 63}  
Para Burke, era axiomático que “quando a multidão não está sob esta disciplina” dos “mais sábios, mais experientes e mais opulentos”, “dificilmente se pode dizer que estão na sociedade civil”.
 {\color{blue} 64}  
Além disso, ao defender tais ordens, o conservador invariavelmente lança-se num programa de reação e contra-revolução, exigindo muitas vezes uma revisão do próprio regime que defende. “Se quisermos que as coisas permaneçam como estão”, na formulação clássica da Medusa, “as coisas terão de mudar”.
 {\color{blue} 65}  
Para preservar o regime, como mostro na parte 1, o conservador deve reconstruir o regime. Este programa implica muito mais do que os clichés sobre “preservação através da renovação” poderiam sugerir: muitas vezes, pode exigir que o conservador tome as medidas mais radicais em nome do regime.
 
\par
 
Alguns dos mais enfadonhos partidários da ordem à direita têm ficado mais do que felizes, quando lhes convém, entregar-se a um pouco de caos e loucura. Kirk, o autodenominado Burkean, desejava “esposar o conservadorismo com a veemência de um radical. O conservador pensante, na verdade, deve assumir algumas das características externas do radical de hoje: ele deve fuçar nas raízes da sociedade, na esperança de restaurar o vigor de uma velha árvore estrangulada na vegetação rasteira das paixões modernas. .” Isso foi em 1954. Quinze anos depois, no auge do movimento estudantil, ele escreveu: “Tendo sido durante duas décadas um crítico mordaz do que é tolamente chamado de ensino superior na América, confesso que estou gostando um pouco. . . O cumprimento das minhas previsões e a situação atual do establishment educacional. Tenho até uma certa simpatia, de certa forma, pelos revolucionários do campus.” Em God ano Man at Yale, William F. Buckley declarou os conservadores “os novos radicais”. Ao ler os primeiros números da National Review, Dwight Macdonald estava inclinado a concordar: “Se [Buckley] tivesse nascido uma geração antes, ele estaria fazendo as cafeterias da 14th Street vibrarem com a dialética marxista”.
 {\color{blue} 66}  
Até o próprio Burke escreveu que “a loucura dos sábios” é “melhor do que a sobriedade dos tolos”.
 {\color{blue} 67}  

 
\par
 
Há uma razão bastante simples para a adoção do radicalismo na direita, e tem a ver com o imperativo reacionário que está no cerne da doutrina conservadora. O conservador não se opõe apenas à esquerda; ele também acredita que a esquerda tem estado no comando desde, dependendo de quem está contando, desde a Revolução Francesa ou a Reforma.
 {\color{blue} 68}  
Se quiser preservar o que valoriza, o conservador deve declarar guerra contra a cultura tal como ela é. Embora o espírito de oposição militante permeie todo o discurso conservador, Dinesh D’Souza expôs o caso de forma mais clara.
 
\par
 

 \textbf{\textit{Typically, the conservative attempts to conserve, to hold on to the values of the existing society. But. . . What if the existing society is inherently hostile to conservative beliefs? It is foolish for a conservative to attempt to conserve that culture. Rather, he must seek to undermine it, to thwart it, to destroy it at the root level. This means that the conservative must. . . Be philosophically conservative but temperamentally radical. {{\color{blue} 69} } } }  
 
 
\par
 
Por esta altura, também já deve estar claro que não é ao estilo ou ao ritmo da mudança que os conservadores se opõem. O teórico conservador gosta de traçar uma “distinção manifestamente marcada” entre a reforma evolutiva e a mudança radical.
 {\color{blue} 70}  
A primeira é lenta, incremental e adaptativa; a segunda é rápida, abrangente e intencional. Mas essa distinção, tão cara a Burke e aos seus seguidores, é muitas vezes menos clara, na prática do que o teórico permite.
 {\color{blue} 71}  
A teoria política é projetada para ser abstrata, mas que abstração impulsionou programas políticos diametralmente opostos como a preferência pela reforma sobre o radicalismo, evolução sobre a revolução? Em nome da mudança lenta, orgânica e adaptável, conservadores autodeclarados se opuseram ao New Deal (Robert Tibet, Kirk e Whittaker Chambers) e endossaram o New Deal (Peter Direct, Clinton Roster e Whittaker Chambers).
 {\color{blue} 72}  
A crença numa reforma evolucionista poderia levar-nos a adaptar uma defesa bayesiana do mercado livre ou do socialismo democrático de Edward Bernstein. “Mesmo os socialistas fabianos”, observa Nash sarcasticamente, “que acreditavam na ‘inevitabilidade da gradualidade’ podem ser rotulados de conservadores”.
 {\color{blue} 73}  
Por outro lado, como apontou Abraham Lincoln, é tão fácil para a esquerda reivindicar o manto da preservação quanto para a direita. “Vocês dizem que são conservadores”, declarou ele aos proprietários de escravos.
 
\par
 

 \textbf{\textit{Eminently conservative—while we are revolutionary, destructive, or something of the sort. What is conservatism? Is it not adherence to the old and tried, against the new and untried? We stick to, contend for, the identical old policy on the point in controversy which was adopted by “our fathers who framed the Government under which we live”; while you with one accord reject, and scout, and spit upon that old policy, and insist upon substituting something new. . . . Not one of all your various plans can show a precedent or an advocate in the century within which our Government originated. Consider, then, whether your claim of conservatism for yourself, and your charge of destructiveness against us, are based on the most clear and stable foundations. {{\color{blue} 74} } } }  
 
 
\par
 
Mais frequentemente, no entanto, a imprecisão da distinção permitiu que o conservador se opusesse à reforma sob o argumento de que ela levaria à revolução ou de que é uma revolução. (De fato, com exceção de Peel e Baldwin, nenhum líder conservador jamais buscou um programa consistente de preservação por meio da reforma, e nem mesmo Peel conseguiu persuadir seu partido a segui-lo.
 {\color{blue} 75}  
 O próprio Burke não ficou imune ao argumento de que a reforma leva à revolução. Embora tenha passado a maioria da década anterior à Revolução Americana contestando esse argumento, ele ainda se perguntava: “Quando você abre” uma constituição “à investigação de uma parte”, o que parece ser a definição de reforma lenta, “onde a investigação irá parar?”
 {\color{blue} 76}  
Outros conservadores argumentaram que qualquer exigência das classes inferiores ou em nome delas, por mais morna ou tardia que seja, é demasiado, demasiado cedo, demasiado rápido. Reforma é revolução, melhoria é insurreição. “Pode ser bom ou ruim”, escreveu um sombrio Lord Carnation sobre a Segunda Lei de Reforma de 1867 – um projeto de lei que estava sendo elaborado há vinte anos e que triplicou o tamanho do eleitorado britânico – “mas é uma revolução”. Sem a qualificação inicial, isto foi uma repetição do que Wellington havia dito sobre a primeira Lei de Reforma.
 {\color{blue} 77}  
Do outro lado do Atlântico, o contemporâneo de Wellington, Nicholas Biddle, denunciava o veto de Andrew Jackson ao Segundo Banco (o mais exercido constitucionalmente dos poderes constitucionais) em termos semelhantes: “Tem toda a fúria de uma pantera acorrentada mordendo as barras da sua jaula. É realmente um manifesto de anarquia – tal como Marat ou Robespierre poderiam ter emitido para a multidão.”
 {\color{blue} 78}  

 
\par
 
O conservador de hoje pode ter feito as pazes com algumas citações do passado; outros, como sindicatos e liberdade reprodutiva, ele ainda contesta. Mas isso não altera que quando essa emancipação surgiu pela primeira vez como questão, seja no contexto da revolução ou da reforma, o seu antecessor estava muito provavelmente contra ela. Michael Gerson, ex-redator de discursos de George W. Bush, é um dos poucos conservadores contemporâneos que reconhece a história da oposição conservadora à emancipação. Enquanto outros conservadores gostam de reivindicar o manto abolicionista ou dos direitos civis, Gerson admite que “a honestidade requer o reconhecimento de que muitos conservadores, em outros tempos, foram hostis às reformas motivadas pela religião” e que “o hábito mental conservador outrora se opôs à maioria dessas mudanças.”
 {\color{blue} 79}  
Na verdade, como sugeriu Samuel Huntington há meio século, dizer não a tais movimentos em tempo real pode ser o que torna alguém um conservador ao longo do tempo.
 {\color{blue} 80}  

 
\par
 
Forjado em resposta aos desafios vindos de baixo, o conservadorismo não tem a calma ou a compostura que acompanham uma herança duradoura de poder. Procurar-se-á em vão, em todo o cânone da direita, garantias constantes de uma Grande Cadeia do Ser. As declarações conservadoras de unidade orgânica, tais como são, ou têm um ar de desespero silencioso – e não tão silencioso – ou, como Kirk, carecem da textura, da sensação de conhecimento, de um testemunho de longa data do poder. Mesmo as profissões de providência divina de Maître não podem esconder ou conter a turbulenta democracia que as gerou. Feitas e mobilizadas para contrariar as reivindicações de emancipação, tais declarações não revelam uma densa ecologia de deferência; eles revelam, em vez disso, uma floresta cada vez mais rarefeita. O conservadorismo tem a ver com poder sitiado e poder protegido. É uma doutrina ativista para um tempo ativista. Aumenta em resposta aos movimentos vindos de baixo e diminui em resposta ao seu desaparecimento, como admitem Hayek e outros conservadores.
 {\color{blue} 81}  

 
\par
 
Longe de comprometer a visão de excelência acima exposta – na qual as prerrogativas do governo deveriam trazer um elemento de grandeza a um mundo que de outra forma seria monótono e inconstante – o imperativo ativista apenas a fortalece. “Luz e perfeição”, escreveu Matthew Arnold, “consistem não em descansar e ser, mas em crescer e se tornar, num avanço perpétuo em beleza e sabedoria”.
 {\color{blue} 82}  
Para o conservador, o poder em repouso é um poder em declínio. A “mera gestão dos recursos já existentes”, escreveu Joseph Schumpeter sobre as dinastias industriais, “por mais meticulosa que seja, é sempre característica de uma posição em declínio”.
 {\color{blue} 83}  
Se quisermos que o poder alcance a distinção que os conservadores lhe associam, ele deve ser exercido.
 {\color{blue} 84}  
E não há melhor maneira de exercer o poder do que defendê-lo contra um inimigo vindo de baixo. A contra-revolução, por outras palavras, é uma das formas pelas quais o conservador faz o feudalismo parecer novo e o medievalismo moderno.
 
\par
 
Mas não é o único caminho. O conservadorismo também oferece ao ERS uma defesa do governo, independente de seu imperativo contrarrevolucionário, que é agonístico e dinâmico e dispensa o tradicionalismo sóbrio e os registros harmônicos das hierarquias passadas. E aqui chegamos às mais profundas intimações do conservador sobre a boa vida, daquela utopia reacionária que ele espera um dia trazer à existência. Ao contrário do passado feudal, onde o poder era presumido e o privilégio herdado, o futuro conservador prevê um mundo onde o poder é demonstrado e o privilégio conquistado: não nos salões antissépticos e anódinos da meritocracia, onde a admissão é prontamente garantida — “o caminho para a eminência e o poder, a partir de uma condição obscura, não deve ser tornado muito fácil, nem uma coisa muito óbvia”
 {\color{blue} 85}  
— mas na árdua luta pela supremacia. Nessa luta, nada importa, nem a herança, as ligações sociais ou os recursos econômicos, mas a inteligência nativa e a força inata de alguém. A excelência genuína é revelada e recompensada, a verdadeira nobreza é garantida. “'Nitro in Adlersum' [Eu luto contra a adversidade] é o lema para um homem como eu”, declara Burke, após demitir um político nascido na mansão que foi “enfaixado, embalado e embalado em um legislador."
 {\color{blue} 86}  
Mesmo o racista mais biologicamente inclinado e determinista acredita que os membros da raça superior devem arrancar pessoalmente o seu direito de governar através da subjugação ou eliminação das raças inferiores.
 
\par
 

 \textbf{\textit{The recognition that race is the substratum of all civilization must not, however, lead anyone to feel that membership in a} }  
 
 
\par
 

 
\par
 

 \textbf{\textit{Superior race is a sort of comfortable couch on which he can go to sleep. . . . The biological heritage of the mind is no more imperishable than the biological heritage of the body. If we continue to squander that biological mental heritage as we have been squandering it during the last few decades, it will not be many generations before we cease to be the superiors of the Mongols. Our ethnological studies must lead us, not to arrogance, but to action. {{\color{blue} 87} } } }  
 
 
\par
 
O campo de batalha, como veremos na parte 2, é o campo de provas natural da superioridade; lá, é apenas o soldado, com sua inteligência e arma, que determina sua posição no mundo. Com o tempo, porém, o conservador encontrará outro campo de provas no mercado. Embora a maioria dos primeiros conservadores fosse ambivalente em relação ao capitalismo,
 {\color{blue} 88}  
Os seus sucessores passarão a acreditar que guerreiros de um tipo diferente podem provar o seu valor na produção e no comércio de mercadorias. Tais homens lutam pelos recursos da terra de e para a terra, tomando para si o que querem e estabelecendo assim a sua superioridade sobre os outros. Os grandes homens do dinheiro não nascem com privilégios ou direitos; eles o apoderam para si, sem permissão ou licença.
 {\color{blue} 89}  
“A liberdade é uma conquista”, escreveu William Graham Sumner.
 {\color{blue} 90}  
O ato primordial de transgressão – exigindo ousadia, visão e aptidão para violência e violação
 {\color{blue} 91}  
– É o que faz do capitalista um guerreiro, dando-lhe não só o direito a uma grande riqueza, mas também, em última análise, ao comando. Pois é isso que o capitalista é: não um Midas de riquezas, mas um governante de homens. Um título de propriedade é uma licença para dispor, e se um homem tem o título do trabalho de outro, ele tem uma licença para dispor dele – isto é, para dispor do corpo em movimento – como achar melhor.
 
\par
 

 \textbf{\textit{Such have been called “captains of industry.” The analogy with military leaders suggested by this name is not misleading. The} }  
 
 
\par
 

 
\par
 

 \textbf{\textit{Great leaders in the development of the industrial organization need those talents of executive and administrative skill, power to command, courage, and fortitude, which were formerly called for in military affairs and scarcely anywhere else. The industrial army is also as dependent on its captains as a military body is on its generals. . . . Under the circumstances there has been a great demand for men having the requisite ability for this function. . . . The possession of the requisite ability is a natural monopoly. {{\color{blue} 92} } } }  
 
 
\par
 
O guerreiro e o homem de negócios se tornarão ícones gêmeos de uma era na qual, como Burke previu, a filiação às classes dominantes deve ser conquistada, muitas vezes por meio das mais dolorosas e humilhantes lutas. “A cada passo do meu progresso na vida (pois a cada passo fui atravessado e enfrentado oposição), e em cada pedágio que encontrei, fui obrigado a mostrar meu passaporte e, repetidamente, provar meu único título à honra de ser útil ao meu país. . . . Caso contrário, nenhuma patente, nem mesmo tolerância, para mim.”
 {\color{blue} 93}  

 
\par
 
Embora a guerra e o mercado sejam as modernas Agnes do poder – sendo Nietzsche o teórico da primeira e Hayek da segunda – a aceitação do capitalismo pela direita nunca foi desqualificada. Até hoje, como mostro na parte 2, os conservadores continuam desconfiados da mesquinhez e superficialidade de ganhar dinheiro, do autismo político que o mercado parece induzir nas classes governantes, e da tolice e frivolidade da cultura de consumo. Para esta ala do movimento, a guerra continuará a ser sempre a única atividade onde o melhor homem pode verdadeiramente provar o seu direito de governar. É um negócio sangrento, com certeza, mas de que outra forma ser um aristocrata quando tudo o que é sólido se desmancha no ar? Nas últimas duas décadas, tem havido uma onda de interesse na direita americana, resultando num conjunto de estudos - muito dela por historiadores mais jovens, muitos deles de esquerda – isso transformou dramaticamente a nossa compreensão do conservadorismo nos Estados Unidos.
 {\color{blue} 94}  
Grande parte da minha própria leitura do pensamento conservador foi informada por esta literatura – a sua ênfase nas realidades vividas de raça, classe e gênero, tal como se manifestaram nas lutas partidárias do último meio século; o sincretismo entre a alta política e a cultura de massa; e a tensão criativa entre elites e ativistas, empresários e intelectuais, subúrbios e sulistas, movimento e meios de comunicação. Acreditando, como T. S. Eliot, que o conservadorismo é melhor compreendido pelo “exame cuidadoso do seu comportamento ao longo da sua história e pelo exame do que as suas mentes mais ponderadas e filosóficas disseram em seu nome”,
 {\color{blue} 95}  
Li a teoria à luz da prática (e a prática à luz da teoria). Com a ajuda desta bolsa de estudos, tenho ouvido o “pathos metafísico” do pensamento conservador – o zumbido das suas implicações, os pressupostos que invoca e as associações que evoca, a vida interior do movimento que descreve.
 {\color{blue} 96}  
A presença sentida dessa bolsa de estudos é o que distingue, espero, minha interpretação do pensamento conservador de outras interpretações, que tendem a ler a teoria isoladamente da prática ou em relação a um relato altamente estilizado dessa prática.
 {\color{blue} 97}  

 
\par
 
Por mais sofisticada que seja a literatura recente sobre o conservadorismo, ela sofre de três pontos fracos. A primeira é a falta de perspectiva comparativa. Os estudiosos da direita americana examinam raramente o movimento em relação ao seu homólogo europeu. Na verdade, entre muitos escritores, parece ser um artigo de fé que, como todas as coisas americanas, o conservadorismo nos Estados Unidos é excepcional. “Há um sentimento distintamente americano em Bush e nos seus defensores intelectuais”, escreve Matt Son. “Um conservadorismo que se baseia em Edmund Burke, um conservadorismo de sabedoria e tradição profundamente enraizado num contexto europeu” é “o tipo de conservadorismo que nunca se consolidou na América”.
 {\color{blue} 98}  
O compromisso com o capitalismo laissez faire deste lado do Atlântico diferencia supostamente o conservadorismo americano do tradicionalismo de Burke ou Disraeli; um pragmatismo nativo torna o conservadorismo americano inóspito ao pessimismo e ao fanatismo de um Donald; a democracia e o populismo tornam insustentáveis ​​os preconceitos aristocráticos de um Tocqueville. Mas esta suposição baseia-se, mostrarei, em equívocos sobre a direita europeia: nem mesmo Burke era tão tradicional como os escritores o fizeram parecer, enquanto Maître tinha opiniões sobre a economia que eram - como tantas outras coisas nos seus escritos revanchistas. – surpreendentemente moderno.
 {\color{blue} 99}  
Na verdade, existem pontos de contacto profundos – especialmente sobre questões de raça e violência – entre a direita radical na Europa e figuras como Calhoun, Teddy Roosevelt, Barry Goldwater e os neoconservadores. Na era do pós-guerra, muitos dos líderes do conservadorismo voltaram-se conscientemente para a Europa em busca de orientação e instrução, um serviço que os emigrados europeus – mais notavelmente, Hayek, Ludwig von Mises e Leo Strauss – ficaram muito felizes em prestar.
 {\color{blue} 100}  
Na verdade, apesar de todo o foco na Escola de Frankfurt e em Hannah Arendt, parece que os únicos movimentos políticos na América do pós-guerra que realmente sentiram a impressão da mente europeia foram os da direita.
 
\par
 
A segunda fraqueza é a falta de perspectiva histórica. Não importa até que ponto os escritores e acadêmicos empurrem as origens do conservadorismo contemporâneo (o último movimento defende um longo movimento conservador que liga o Tea Party há década de 1920),
 {\color{blue} 101}  
Há uma noção na literatura recente de que o conservadorismo contemporâneo é fundamentalmente diferente das iterações anteriores. A certa altura, prossegue o argumento, o conservadorismo americano rompeu com os seus antecessores – tornou-se populista, ideológico, e assim por diante – e foi esta ruptura, dependendo da perspectiva de cada um, que o salvou ou condenou.
 {\color{blue} 102}  
Mas este argumento ignora as continuidades entre figuras como Adams e Calhoun e vozes mais recentes da direita americana. Longe de ser uma inovação das últimas décadas, o populismo do Tea Party e o futurismo de Reagan ou Gingrich podem ser encontrados nas primeiras vozes do conservadorismo, em ambos os lados do Atlântico. Da mesma forma, o aventureirismo, o racismo e a propensão ao pensamento ideológico.
 
\par
 
A terceira fraqueza deriva da segunda. Quanto mais para trás os analistas traçam as origens do conservadorismo contemporâneo, menos inclinados eles estão a acreditar que é uma política de reação ou retrocesso. Se os compromissos do conservador contemporâneo podem ser situados nos escritos de Albert Jay Rock ou John Adams, esses acadêmicos argumentam, o conservadorismo deve refletir ideias e compromissos mais transcendentes do que a mera oposição à Grande Sociedade sugeriria.
 {\color{blue} 103}  
Mas o reconhecimento da longa história da direita não precisa de minar a afirmação de que o conservadorismo contemporâneo é uma política de contra-ataque. Em vez disso, a visão a longo prazo deverá ajudar-nos a compreender melhor a natureza e a dinâmica, bem como as idiossincrasias e contingências, dessa reação negativa. Na verdade, só colocando a direita contemporânea no contexto dos seus antecessores poderemos compreender a sua cidade e particularidade específicas.
 
\par
 
Contra estes três pressupostos, que se centram na diferença e na distinção, trato o direito como uma unidade, como um corpo coerente de teoria e prática que transcende as divisões tantas vezes enfatizadas por acadêmicos e especialistas.
 {\color{blue} 104}  
Utilizo as palavras conservador, reacionário e contrarrevolucionário de forma intercambiável: nem todos os contrarrevolucionários são conservadores – Walt Rostov vem-me imediatamente à mente – mas todos os conservadores são, garantidamente, contrarrevolucionários. Sento filósofos, estadistas, proprietários de escravos, escribas, católicos, fascistas, evangélicos, empresários, racistas e hackers na mesma mesa: Hobbes ao lado de Hayek, Burke em frente a Pain, Nietzsche entre Ayn Rand e Antonin Scalia, com Adams, Calhoun, Improvisado, Ronald Reagan, Tocqueville, Theodore Roosevelt, Margaret Thatcher, Ernst Jünger, Carl Schmitt, Winston Churchill, Phyllis School y, Richard Nixon, Irving Bristol, Francis Fukuyama e George W. Bush intercalados por toda parte.
 
\par
 
Isto não quer dizer que não haja mudança no conservadorismo ao longo do tempo ou do espaço. Se o conservadorismo é uma reação específica a um movimento específico de emancipação, é lógico que carregará cada reação os traços do movimento ao qual se opõe. Como defendo no capítulo 1, não só a direita reagiu contra a esquerda, mas, ao conduzir a sua reação, também recorreu consistentemente à esquerda. À medida que os movimentos da esquerda mudam – da Revolução Francesa, da abolição, à direita do voto, da direita à organização, da Revolução Bolchevique às lutas pela liberdade dos negros e pela libertação das mulheres – o mesmo acontece com as reações da direita.
 
\par
 
Para além destas mudanças contingentes, podemos também traçar uma mudança estrutural mais longa na imaginação da direita: nomeadamente, a aceitação gradual da entrada das massas na cena política. De Hobbes aos proprietários de escravos e aos neoconservadores, a direita tornou-se cada vez mais consciente de que qualquer defesa bem sucedida do antigo regime deve incorporar as ordens inferiores em alguma capacidade que não seja a de subordinados ou de fãs fascinados. As massas devem ser capazes de se localizar simbolicamente na classe dominante ou ter oportunidades reais de se tornarem eles próprios falsos aristocratas na família, na fábrica e no campo. O primeiro caminho conduz a um populismo invertido, em que os mais baixos dos mais baixos se beém projetados no mais alto dos mais altos; o último contribui para um feudalismo democrático, no qual o marido ou supervisor desempenha o papel de senhor. O primeiro caminho foi iniciado por Hobbes, Maître e vários profetas do racismo e do nacionalismo, o último por proprietários de escravos do Sul, imperialistas europeus e apologistas da Era Dourada. (E os apologistas da Neo-Era Dourada: “Não existe uma elite única na América”, escreve David Brooks. “Todos podem ser um aristocrata dentro do seu próprio Olimpo.”
 {\color{blue} 105}  
 Ocasionalmente, como na escrita de Werner Combat, os dois caminhos convergem: as pessoas comuns conseguem ver-se na classe dominante em virtude de pertencerem a uma grande nação entre as nações, e também conseguem governar seres inferiores através do exercício do domínio imperial.
 
\par
 

 \textbf{\textit{We Germans, too, should go through the world of our time in the same way, proud heads held high, in the secure feeling of being God’s people. Just as the German bird, the eagle, soars high over all animals on this earth, so the German must feel himself above all other peoples that surround him and that he sees in boundless depth below him.} }  
 
 
\par
 

 
\par
 

 \textbf{\textit{But aristocracy has its obligations, and this is true here, too. The idea that we are chosen people places formidable duties— and only duties—on us. We must above all maintain ourselves as a strong nation in the world. {{\color{blue} 106} } } }  
 
 
\par
 
Embora essas diferenças históricas na direita sejam reais, há uma afinidade subjacente que as une. Não se pode perceber essa afinidade focando em desacordos de política ou declarações contingentes de prática (direitos dos estados, federalismo e assim por diante); é preciso olhar para os argumentos subjacentes, os idiomas e metáforas, as visões profundas e o pathos metafísico evocados em cada desacordo e declaração. Alguns conservadores criticam o livre mercado, outros o defendem; alguns se opõem ao estado, outros o abraçam; alguns acreditam em Deus, outros são ateus. Alguns são vocalistas, outros nacionalistas e outros ainda internacionalistas. Alguns, como Burke, são todos os três ao mesmo tempo. Mas essas são improvisações históricas — táticas e substantivas — sobre um tema. Somente justapondo essas vozes — através do tempo e do espaço — podemos distinguir o tema em meio à improvisação.
 
\par
 
Para muitos, a noção de unidade à direita será a afirmação mais controversa deste livro. Embora continuemos a utilizar o termo “conservador” no nosso discurso quotidiano (na verdade, a discussão política seria inconcebível sem ele); embora o conservadorismo, tanto na Europa como nos Estados Unidos, tenha conseguido, durante mais de um século, atrair e manter unida uma coligação de tradicionalistas, guerreiros e capitalistas; embora a oposição entre esquerda e direita tenha provado ser uma “distinção política” duradoura da era moderna (apesar das tentativas, a cada geração ou mais, de negar ou superar esta oposição por uma “terceira via”)
 {\color{blue} 107}  
— muitos continuam a acreditar que as diferenças à direita são tão grandes que seria impossível dizer algo sobre a direita.
 {\color{blue} 108}  
Mas se é impossível dizer alguma coisa sobre o direito – definir, descrever, explicar, analisar e interpretar o direito como uma formação distintiva – como podemos dizer que ele existe? Ininteligível muito do que se passa na nossa política, alguns estudiosos recuaram para uma posição nominalista: conservadores são pessoas que se autodenominam conservadoras ou, mais elaboradamente, conservadores são pessoas que se autodenominam conservadoras chamadas conservadoras.
 {\color{blue} 109}  
Isto apenas levanta a questão: o que estas pessoas que se autodenominam conservadoras – ou que outras pessoas que se autodenominam conservadoras chamam de conservadoras – querem dizer com “conservador”? Por que optam por essa auto descrição em oposição a liberal, socialista ou porco-da-terra? A menos que estas pessoas pensem que estão a referir-se a identidades idiossincráticas – caso em que voltamos à posição cética – precisamos de compreender o que o termo significa, independentemente da sua utilização. De que outra forma podemos compreender por que razão indivíduo de diferentes épocas e lugares, adaptando diferentes posições sobre diferentes questões, se autodenominariam conservadores e aos seus espíritos afins? Embora nem todos os leitores precisem de aceitar a minha afirmação sobre o que une o direito, parece ser uma condição necessária para uma discussão inteligente que concordemos que existe algo chamado direito e que tem algum conjunto de características comuns que o tornam certo.
 
\par
 
Os onze capítulos deste livro foram extraídos de uma década de escritos sobre a direita. Alguns capítulos apareceram originalmente como longos ensaios de revisão para periódicos como The Nation e London Review of Books; outros são artigos de pesquisa acadêmica, artigos relatados ou ensaios independentes. Fiz algumas alterações nessas peças para dar conta de novos desenvolvimentos ou mudanças em meus pontos de vista. Ocasionalmente, eliminei seções inteiras porque elas não pareciam mais relevantes. Mas, no geral, tentei deixar as peças intactas, na esperança de que as suas abordagens variadas capturassem esta noção de direito como um conjunto de improvisações históricas sobre um tema contínuo. O livro está dividido em duas partes. Papel
 {\color{blue} 1}  
Abre com uma declaração geral sobre o impulso contra-revolucionário da política conservadora, desde a Revolução Francesa até hoje. Este capítulo centra-se menos nos objectivo e intenções da contrarrevolução e mais nos seus movimentos e manobras: como rompe com o próprio regime que defende e olha para a esquerda nos seus esforços para reconstruir a direita. Em seguida, passo cronologicamente, de um exame de Thomas Hobbes e da Guerra Civil Inglesa para uma análise conclusiva do Juiz Scalia e sua jurisprudência originalista. Ao longo do caminho, discuto Rand, Goldwater, a Nova Direita e os conservadores após a Guerra Fria. Papel
 {\color{blue} 2}  
Analisa o tema tenso da violência no conservadorismo. Embora eu comece com uma breve retrospectiva da Guerra Fria na América Latina e conclua com uma visão real mais geral sobre como a direita abordou a violência desde Burke, a maioria da discussão nestes capítulos é extraída da última década: 9/ 11, a guerra ao terror, a guerra no Iraque. Estes acontecimentos, e a vertigem que inspiraram entre os conservadores, mais do que qualquer coisa, levaram-me a pensar e a escrever sobre a direita. Como percebi, e como capítulo
 {\color{blue} 11}  
Argumenta que a paixão pela violência na direita de hoje não é uma aberração; é constitutivo da própria tradição.
 
\par
  
 
999999
