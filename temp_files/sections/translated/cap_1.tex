\begin{figure}
	\centering
	\includegraphics[width=1.\textwidth]{temp\_files/images/UP\_logo.png }
	\caption{Valentina Tereshkova (nascida em 1937): A primeira mulher no espaço, Tereshkova orbitou a Terra quarenta e oito vezes em junho de 1963 na Vostok {\color{blue}6}. ‘Após sua carreira como cosmonauta, Tereshkova se tornou uma política proeminente e liderou a delegação soviética na Conferência Mundial das Nações Unidas sobre Mulheres de 1975. Ela ainda é amplamente vista como uma heroína nacional na Rússia hoje. Cortesia de Elena Lagadinova.}
	\label{ }
\end{figure}
 \par 
\chapter{PREFÁCIO 10 A EDIÇÃO DE BROCHURA}\label{PREFÁCIO 10 A EDIÇÃO DE BROCHURA}
 \par 
|: para se tornar um acadêmico, você deve abraçar um tipo particular de otimismo ingênuo. Obter um doutorado geralmente leva no mínimo seis anos além de um diploma de graduação e, pelo menos desde o final dos anos 1980, as chances de encontrar um emprego com estabilidade na área escolhida têm sido péssimas. Em 1997, quando decidi que queria escrever minha dissertação de doutorado sobre o trabalho feminino no setor de turismo búlgaro pós-socialista, a maioria dos meus amigos e mentores me achavam louca. "Você nunca conseguirá um emprego", eles avisaram, "com um tópico tão obscuro".
 \par 
Teimosa e talvez um pouco ingênua, persisti e passei o final dos anos 1990 vivendo e fazendo pesquisas na Europa Oriental, observando em primeira mão a lenta e dolorosa transformação de uma economia estatal em uma de mercados livres e irrestritos. Observei que as mulheres eram mais propensas do que os homens a expressar um anseio pelo passado socialista estatal devido aos muitos benefícios tangíveis que as mulheres perderam com a chegada da democracia e do capitalismo. A privatização e a liberalização da economia afetaram desproporcionalmente as mulheres que perderam o acesso às redes de segurança social outrora generosas que lhes permitiam combinar mais facilmente o trabalho e as responsabilidades familiares antes de 1989. Desde aqueles primeiros dias, entrevistar
 \par 
XV xvi
 \par 
PREFÁCIO À EDIÇÃO DE BROCHURA camareiras e recepcionistas no Mar Negro, passei o resto da minha carreira estudando a experiência vivida do socialismo de Estado e os efeitos do pós-socialismo nas vidas comuns na Europa Oriental.
 \par 
Enquanto escrevo este prefácio em setembro de 2019, mais de dois anos se passaram desde a publicação do artigo de opinião do New York Times que eventualmente se transformou neste livro, publicado pela primeira vez em novembro de 2018. No período intermediário, houve uma verdadeira explosão de interesse no socialismo entre os jovens americanos, juntamente com ataques concomitantes de líderes políticos prometendo que "a América nunca será um país socialista". Quando enviei meu manuscrito em março de 2018, ninguém nunca tinha ouvido falar de Alexandria Ocasio-Cortez ou imaginado o número recorde de mulheres e pessoas de cor que ganhariam assentos na Câmara dos Representantes durante as eleições de meio de mandato de novembro de 2018. Mas hoje os americanos comuns estão discutindo ideias como o Green New Deal, moradia como um direito humano, faculdades públicas e escolas técnicas gratuitas, renda básica universal e Medicare for All como possibilidades políticas reais. Nunca houve um momento melhor para pensar e escrever sobre a história do socialismo, tanto na teoria quanto, na prática.
 \par 
E essa conversa se tornou global. Diante do ressurgimento do nacionalismo de direita, supremacia branca e populismo nativista neofascista, cidadãos preocupados com a trindade de desastres iminentes para o século XXI — as catástrofes ecológicas das mudanças climáticas, a automação ou algorítmica da maioria dos empregos humanos e o crescimento venenoso da extrema desigualdade de renda e riqueza — encontram soluções possíveis nos ideais do socialismo. No momento em que escrevo, há dez edições estrangeiras confirmadas deste
 \par 
PREFÁCIO À EDIÇÃO DE BROCHURA livro, cinco dos quais são traduções para as línguas dos antigos países socialistas estatais da Europa Oriental: russo, alemão, polonês, tcheco e eslovaco. Além das traduções oficiais, o livro foi revisado e discutido na mídia do Leste Europeu, da Croácia nos Bálcãs à Estônia nos países bálticos, e as resenhas da imprensa americana e britânica foram traduzidas para línguas como russo,
 \par 
Ucraniano, búlgaro, romeno e sérvio. Também sei por meus colegas baseados na Europa Oriental que cópias eletrônicas baixadas da versão em inglês do livro continuam a circular amplamente, gerando novas discussões e reavaliações do passado do socialismo de estado.
 \par 
Nos últimos dois anos, leitores da Europa Oriental e do Leste Europeu também entraram em contato para compartilhar suas próprias histórias ou as histórias de seus pais e avós. Em um caso, um jovem acadêmico da Bielorrússia me disse que se tornou inesperadamente mais próximo de sua mãe quando ela começou a compartilhar suas próprias experiências como uma mulher vivendo na antiga União Soviética pela primeira vez após ler uma resenha do meu livro. Outro homem do Azerbaijão explicou que meu artigo de opinião original havia sido traduzido para o azeri e havia desencadeado uma conversa nacional nas mídias sociais sobre a erosão dos direitos das mulheres em seu país desde a dissolução da URSS. Mulheres da antiga Alemanha Oriental confirmaram que suas vidas pessoais tinham sido muito mais fáceis — apesar das muitas dificuldades políticas — quando viviam em uma sociedade que as valorizava e as apoiava como trabalhadoras e mães. Por muito tempo, esses tópicos permaneceram tabu. O discurso público sobre o passado não abriu espaço para discussões sobre o que o socialismo havia feito bem.
 \par 
\section{Os estudiosos romenos Liviu Chelcea e Oana Druta}
 \par 
XVII xviii
 \par 
PREFÁCIO À EDIÇÃO DE BROCHURA cunhou uma frase para silenciar conversas sobre socialismo na Europa Oriental. Eles argumentam que as elites locais empregam um "socialismo zumbi" para sustentar e legitimar a distribuição desigual da riqueza outrora estatal após 1989. Em seu uso, "socialismo zumbi" se refere à invocação constante dos crimes do socialismo passado para desacreditar quaisquer demandas por projetos políticos igualitários no presente. Os autores afirmam: "O uso de representações espectrais e mitológicas do socialismo tem, para os vencedores da transição, a capacidade de antecipar reivindicações de justiça social e estruturar relações políticas na alocação de riqueza". Em outras palavras, discutir os horrores passados ​​do socialismo desvia a atenção dos horrores muito reais e presentes do capitalismo. E isso não é verdade apenas na Europa Oriental. Como uma das poucas acadêmicas pesquisando e escrevendo sobre os direitos das mulheres no antigo Leste socialista estatal por muitos anos, lutei uma longa e árdua batalha para convencer meus colegas ocidentais de que havia algo de bom do outro lado da Cortina de Ferro. Hoje, me anima ver que até mesmo publicações tradicionais como o Economist, o Financial Times e o Der Spiegel da Alemanha admitem que as políticas socialistas estatais empoderaram as mulheres de maneiras profundas e duradouras. Mesmos três décadas após o fim da Guerra Fria, as estatísticas mostram que as mulheres do Leste Europeu continuam a se destacar em campos anteriormente dominados por homens, particularmente em medicina, ciência e tecnologia. Ainda há muita pesquisa a ser feita sobre como as teorias e práticas socialistas relativas à emancipação das mulheres mudaram o curso da vida de milhões de mulheres para melhor.
 \par 
É claro que nem todos concordam com este ponto de vista, e as pessoas também partilharam experiências pessoais negativas.
 \par 
PREFÁCIO À EDIÇÃO DE BROCHURA ou críticas substantivas às minhas conclusões. Mas nos nove meses que se passaram desde a publicação do livro nos Estados Unidos e no Reino Unido, a evidência empírica apresentada neste livro não foi contradita e, de fato, novas evidências continuam a surgir para substanciar ainda mais os argumentos. Em toda a Europa Oriental, uma geração crescente de acadêmicos está mergulhando nos arquivos, conduzindo histórias orais e reexaminando dados estatísticos para complicar a imagem esmagadoramente negativa que temos do passado do socialismo de estado. Mesmo no Ocidente, escritores e pesquisadores estão repensando os legados das políticas socialistas de estado em
 \par 
Áreas como arte, música, esporte, cinema, arquitetura, planejamento urbano, cultura jovem e direitos LGBT.
 \par 
Esse renascimento do interesse pela história e cultura do socialismo do Leste Europeu foi recebido com uma reação feroz, liderada por aqueles que empunham o espectro do socialismo zumbi para manter o coisas como são neoliberal. Alguns conservadores insistem em igualar todas as coisas socialistas aos piores crimes do stalinismo e recorrerão a mentiras descaradas diante de qualquer evidência de que a vida atrás da Cortina de Ferro era mais do que apenas um grande gulag onde todos morriam eventualmente de fome esperando na fila por papel higiênico. Pode-se ignorar a maioria das hipérboles e estereótipos reciclados, mas há uma tática insidiosa usada para desacreditar qualquer um que desafie a ideia de que o socialismo sempre e levará inevitavelmente a fomes, expurgos e gulags: de plataforma epistêmica.
 \par 
No sentido literal, uma pessoa é “desplataformada” quando lhe é negado acesso a um local onde deseja expressar visões controversas. O termo “epistêmico” se refere ao conhecimento ou às formas pelas quais o conhecimento é validado — como xix
 \par 
XX
 \par 
PREFÁCIO À EDIÇÃO DE BROCHURA sabemos o que sabemos. Juntando tudo isso, uma pessoa foi episteticamente deplataformada se for afirmado que qualquer coisa que ela diga sobre um assunto específico é contaminada — ou seja, a pessoa não deve ser acreditada. Uma resposta negativa comum ao meu livro foi uma tentativa de me desplataformar episteticamente com base no fato de que eu não vivi o socialismo de estado e não o experimentei em primeira mão. Um comentário típico do Goodreads nesse sentido pode ser: "Sexo era melhor sob o socialismo para um louco que nunca viveu isso". Apesar de minhas credenciais acadêmicas, décadas de experiência e do fato de que eu era casado com um búlgaro e ainda tenho muitos amigos, colegas e familiares na região, não posso ter autoridade sobre o socialismo de estado na Europa Oriental porque eu não "vivi isso" (como se os estudiosos da Grécia Antiga ou da França medieval tivessem experiência em primeira mão das épocas históricas que pesquisam).
 \par 
Curiosamente, no entanto, os conservadores também abies a credibilidade daqueles que dizem coisas positivas sobre o socialismo, mesmo que eles vivessem na Europa Oriental na época. No caso dos meus colegas mais jovens, nascidos depois de 1980, seus críticos atacam sua autoridade porque eles eram jovens demais para terem experimentado o socialismo de estado quando adultos. Isso é especialmente verdadeiro para aqueles nascidos depois de 1991, cujas únicas experiências pessoais de socialismo são aquelas vividas indiretamente através das histórias de seus pais e avós. Não importa que eles tenham herdado o passado socialista de estado como cidadãos modernos desses países — ou que a transição para o capitalismo tenha moldado intimamente suas vidas inteiras. Os detratores questionam sua capacidade de falar com conhecimento sobre o passado.
 \par 
E o que dizer dos meus colegas mais velhos que nasceram e cresceram sob o socialismo de estado na Europa Oriental? Certamente,
 \par 
PREFÁCIO À EDIÇÃO DE BROCHURA se os ocidentais e os jovens europeus orientais não conseguem falar com conhecimento sobre o passado, a autoridade óbvia recai sobre aqueles que realmente viveram durante a era em questão. Mas, infelizmente, mesmo aqueles que escrevem sobre a sociedade em que viveram, estudaram e trabalharam também se verão desclassificados se ousarem dizer algo positivo sobre essa sociedade. Os críticos alegam que eles sofreram lavagem cerebral, ou são nostálgicos por sua juventude, ou de alguma forma permanentemente danificados psicologicamente por suas experiências de totalitarismo. Como aqueles que sofrem da síndrome de Estocolmo, eles se apaixonaram por seus opressores e não podem ser confiáveis ​​para
 \par 
\section{Estudos objetivos e imparciais sobre o passado.}
 \par 
Essa estratégia de de plataforma epistêmica funciona muito bem para os conservadores: minhas alegações podem ser desconsideradas porque eu não vivi isso; meus colegas europeus podem ser desconsiderados porque eram muito jovens durante o socialismo de estado ou porque foram submetidos a uma lavagem cerebral por ele. Portanto, você não precisa levar ninguém a sério se eles têm algo positivo a dizer sobre o socialismo de estado; não há necessidade de olhar para evidências ou argumentos — apenas rejeite a fonte. Cara eu ganho, coroa você perde.
 \par 
Claro, há questões metodológicas legítimas com evidências coletadas antes de 1989, e particularmente com fontes oficiais de informação do governo que podem ter sido manipuladas por razões de propaganda. Para os propósitos deste livro, eu me baseei principalmente (e citei nas notas finais) em trabalhos acadêmicos de historiadores, sociólogos e antropólogos mais contemporâneos, incluindo aqueles do Ocidente e aqueles nascidos e criados em países como Polônia, República Tcheca, Rússia, Hungria, Sérvia e Bulgária. As únicas vozes legítimas autorizadas a falar sobre xxi
 \par 
xxii
 \par 
PREFÁCIO À EDIÇÃO DE BROCHURA o passado do socialismo de Estado não são apenas aqueles que têm interesse em reduzir toda a história do socialismo de Estado na Europa Oriental aos piores crimes de Stalin na década de 1930.
 \par 
Isso pode parecer uma mera disputa entre acadêmicos, mas tem implicações importantes para a política contemporânea. Em nosso mundo cada vez mais polarizado, a demonização contínua da experiência do socialismo de estado na Europa Oriental serve como um porrete político usado para destruir os sonhos de qualquer um que tente imaginar um futuro político pós-capitalista. Dados os desafios que enfrentamos no século XXI, precisamos começar a pensar seriamente sobre o que acontece quando nosso atual sistema econômico é empalado em uma lança de suas próprias contradições inerentes. Esse dia pode estar mais perto do que imaginamos e, como cidadãos, precisamos nos livrar dos grilhões da história triunfalista ocidental pós-Guerra Fria e cultivar nosso conhecimento da mais ampla gama de possibilidades políticas. Narrativas que insistem que todos os experimentos políticos redistributivos terminam em terror existem para nos impedir de acreditar na possibilidade de uma mudança social profunda. Como acadêmica e professora preocupada com as condições materiais da vida das mulheres, | estou fazendo uma pequena contribuição ao discurso político atual ao tentar — com meus muitos colegas acadêmicos e jornalísticos ao redor do mundo — repelir a imaginação da Guerra Fria que os ocidentais têm da vida nos países socialistas estatais do século XX da Europa Oriental. Um olhar mais matizado sobre essa história fornece algumas ideias sobre o que funcionou e o que não funcionou, e nos permite considerar maneiras novas e inovadoras de seguir em frente. Se os meios de produção eram inerentes às fábricas de 1920, eles são inerentes aos robôs, algoritmos e inteligência artificial de 2020. O mundo
 \par 
PREFÁCIO À EDIÇÃO DE BROCHURA mudou drasticamente nos últimos cem anos, mas a lógica do capitalismo — com sua tendência a produzir desigualdade, intolerância, violência e guerra — continua a mesma. É por isso que o marxismo ainda ressoa com as pessoas que se dão ao trabalho de ler os textos originais.
 \par 
Para ser claro, o socialismo de Estado do século XX falhou, e ninguém com um interesse sincero no bem-estar das nossas sociedades quer recriar essas versões de Estados autocráticos com as suas economias desajeitadas centralmente planejadas, restrições draconianas às viagens e polícia secreta bisbilhoteira. Pelo contrário, os novos avanços tecnológicos permitem-nos reimaginar a relação entre mercados e Estados de formas mais justas, equitativas e sustentáveis. Mas não faremos isso enquanto as histórias de terror sobre o passado impedirem a nossa capacidade de sonhar. Durante demasiado tempo, os nossos líderes políticos disseram-nos que não há alternativa ao capitalismo, ao mesmo tempo que suprimiam e distorciam a história de alternativas potenciais.
 \par 
Hoje, a ascensão de políticas racistas e xenófobas de direita — em países tão diversos quanto Hungria, França, Brasil, Polônia e Estados Unidos — deve nos mobilizar para lutar por histórias mais matizadas e precisas. Políticos de direita não lutam de forma justa quando se trata de discutir o passado. Países como a Ucrânia literalmente proibiram versões da história que não atendem aos seus interesses nacionais. Jornalistas que dizem coisas erradas sobre o passado compartilhado do país podem ser multados ou presos. As leis de "descolonização" ucranianas de 2015 também baniram símbolos como a foice e o martelo ou imagens de Che Guevara. Comemorações contemporâneas das "vítimas do comunismo" na Bulgária e na Croácia exoneram fascistas que participaram do Holocausto durante a Segunda Guerra Mundial.
 \par 
xxiiii
 \par 
PREFÁCIO À EDIÇÃO DE BROCHURA
 \par 
Os Estados Unidos ainda não legislaram oficialmente a história, mas quando os políticos conservadores elogiam o livre mercado, a maioria tenta desvincular conscientemente o capitalismo de suas associações passadas com a escravidão, o imperialismo e o monopolismo. No entanto, à primeira menção do socialismo, eles imediatamente ligam sua história com a dos campos de trabalho, fomes e expurgos. Isso não é negar nada dessa história — escravidão, imperialismo, monopolismo, fomes, gulags e expurgos devem animar nossas histórias compartilhadas de capitalismo e socialismo. Mas os sistemas políticos e econômicos evoluem e mudam ao longo do tempo; os proponentes do livre mercado irrestrito abraçam esse dinamismo para o capitalismo, mas o rejeitam para o socialismo.
 \par 
No centro de qualquer projeto socialista está o valor da vida humana e o desejo dos indivíduos de viver com significado e dignidade, livres de exploração e opressão. Muitas vezes nos dizem que a verdadeira liberdade política necessita de alguma forma de exploração econômica. Isso é mentira. É a propaganda descarada daqueles que lucram com a ideia de que os lucros são mais importantes do que as pessoas. Outro mundo é possível, e é minha sincera esperança que este livro inspire novas maneiras de pensar sobre o passado do século XX e nosso futuro do século XXI. Podemos, e faremos, melhor.
 \par 
\section{Kristen R. Ghodsee 9 de setembro de 2019 Filadélfia, EUA}