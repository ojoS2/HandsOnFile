\chapter{Introdução}\label{Introdução}
 \par 
Um partido político pode descobrir que tem uma história, antes de estar plenamente consciente ou de concordar com os seus próprios princípios permanentes; pode ter chegado à sua formação real através de uma sucessão de metamorfoses e adaptações, durante as quais algumas questões foram obsoletas e novas questões surgiram. Quais são os seus princípios fundamentais, provavelmente só serão descobertos através de um exame cuidadoso do seu comportamento ao longo da sua história e através do exame do que as suas mentes mais ponderadas e filosóficas disseram em seu favor; e apenas o conhecimento histórico preciso e a análise criteriosa serão capazes de discriminar entre o permanente e o transitório; entre aquelas doutrinas e princípios que deve sempre, e em todas as circunstâncias, manter, ou manifestar-se como uma fraude, e aqueles suscitados por circunstâncias especiais, que só são inteligíveis e justificáveis ​​à luz dessas circunstâncias.
 \par 
Desde o início da era moderna, homens e mulheres em posições subordinadas marcharam contra os seus superiores no estado, na igreja, no local de trabalho e outras instituições hierárquicas. Reuniram-se sob diferentes bandeiras – o movimento operário, o feminismo, a abolição, o socialismo – e gritaram diferentes slogans: liberdade, igualdade, direitos, democracia, revolução. Em praticamente todos os casos, os seus superiores resistiram-lhes, de forma violenta e não violenta, legal e ilegalmente, aberta e secretamente. Essa marcha e démarche da democracia é a história da política moderna ou pelo menos uma das suas histórias.
 \par 
Este livro é sobre a segunda metade dessa história, a démarche e as ideias políticas – também chamadas de conservadoras, reacionárias, revanchistas, contra-revolucionárias – que surgem dela e lhe dão origem. Estas ideias, que ocupam o lado direito do espectro político, são forjadas na batalha. Sempre foram, pelo menos desde que surgiram como ideologias formais durante a Revolução Francesa, batalhas entre grupos sociais e não entre nações; grosso modo, entre aqueles que têm mais poder e aqueles que têm menos. Para entender essas ideias, temos que entender essa história. Pois é isso que é o conservadorismo: uma meditação – e uma representação teórica – da experiência sentida de ter poder, vê-lo ameaçado e tentar reconquistá-lo.
 \par 
Apesar das diferenças muito reais entre eles, os trabalhadores de uma fábrica são como secretárias em um escritório, camponeses em uma mansão, escravos em uma plantação — até mesmo esposas em um casamento — no sentido de que vivem e trabalham em condições de poder desiguais. Eles se submetem e obedecem, atendendo às demandas de seus gerentes e mestres, maridos e senhores. Eles são disciplinados e punidos. Além disso, eles fazem muito e recebem pouco. Às vezes, seu destino é escolhido livremente — os trabalhadores contratam com seus empregadores, as esposas com seus maridos — mas suas implicações raramente o são. Que contrato, afinal, poderia detalhar os prós e contras, as dores diárias e o sofrimento contínuo de um trabalho ou casamento? Ao longo da história americana, de fato, o contrato frequentemente serviu como um canal para coerção e restrição imprevistas, particularmente em instituições como o local de trabalho e a família, onde homens e mulheres passam grande parte de suas vidas. Os contratos de trabalho e de casamento foram interpretados pelos juízes, eles próprios favoráveis ​​aos interesses dos empregadores e dos maridos, como contendo todo o tipo de disposições não escritas e indesejadas de servidão, às quais as esposas e os trabalhadores consentem tacitamente, mesmo quando não têm conhecimento de tais disposições ou desejam estipular o contrário.{\color{blue}1}
 \par 
Até 1980, por exemplo, era legal em todos os estados da união um marido estuprar a esposa. {\color{blue}2} A justificação para estas remonta a um tratado de 1736 do jurista inglês Matthew Hale. Quando uma mulher se casa, argumentou Hale, ela concorda implicitamente em “entregar-se desta forma [sexualmente] ao marido”. O consentimento dela é tácito, embora inconsciente, “que ela não pode retratar” durante a união. Tendo dito sim uma vez, ela nunca poderá dizer não. Ainda em 1957 – durante a era do Tribunal Warren – um tratado jurídico padrão poderia afirmar: “Um homem não comete violação ao ter relações sexuais com a sua esposa legítima, mesmo que o faça pela força e contra a vontade dela”. Se uma mulher (ou homem) tentasse incluir no contrato de casamento a exigência de que o consentimento expresso tivesse que ser dado para que o sexo pudesse prosseguir, os juízes eram obrigados pela lei comum a ignorá-lo ou anulá-lo. O consentimento implícito era uma característica estrutural do contrato que nenhuma das partes poderia alterar. Como a opção de saída do divórcio não estava amplamente disponível até à segunda metade do século XX, o contrato de casamento condenou as mulheres a serem servas sexuais dos seus maridos. {\color{blue}3} Uma dinâmica semelhante estava presente no contrato de trabalho: os trabalhadores consentiam em ser contratados pelos seus empregadores, mas até ao século XX esse consentimento era interpretado pelos juízes como contendo disposições implícitas e irrevogáveis ​​de servidão; entretanto, a opção de saída de desistir não estava tão disponível, legal ou praticamente, como muitos poderiam pensar.{\color{blue}4}
 \par 
De vez em quando, porém, os subordinados deste mundo contestam o seu destino. Eles protestam contra as suas condições, escrevem cartas e petições, aderem a movimentos e fazem exigências. Os seus objectivos podem ser mínimos e discretos – melhores guardas de segurança nas máquinas das fábricas, o fim da violação conjugal – mas, ao expressá-los, levantam o espectro de uma mudança mais fundamental no poder. Eles deixam de ser servos ou suplicantes e passam a ser agentes, falando e agindo em seu próprio nome. Mais do que as próprias reformas, é esta afirmação de agência por parte da classe sujeita – o aparecimento de uma força de vontade insistente e
 \par 
Voz independente de demanda – isso irrita seus superiores. A Reforma Agrária da Guatemala de 1952 redistribuiu um milhão e meio de acres de terra a {\color{blue}100} mil famílias camponesas. Isso não era nada, nas mentes das classes dominantes do país, comparado com a agitação do debate político que o projecto de lei parecia desencadear. Os reformadores progressistas, queixou-se o arcebispo da Guatemala, enviaram camponeses locais “dotados de facilidade com as palavras” para a capital, onde lhes foram dadas oportunidades “de falar em público”. Esse foi o grande mal da Reforma Agrária.{\color{blue}5}
 \par 
Em seu último grande discurso ao Senado, John C. Calhoun, ex-vice-presidente e principal porta-voz da causa do Sul, identificou a decisão do Congresso, em meados da década de 1830, de receber petições abolicionistas como o momento em que a nação se colocou em uma situação difícil. curso irreversível de confronto sobre a escravidão. Numa carreira de quatro décadas que assistiu a derrotas para a posição dos senhores de escravos, como a Tarifa das Abominações, a Crise da Nulificação e a Lei da Força, a mera aparição do discurso escravista na capital do país destacou-se para o moribundo Calhoun como o sinal que a revolução havia começado. {\color{blue}6} E quando, meio século mais tarde, os sucessores de Calhoun procuraram colocar o génio abolicionista de volta na garrafa, foi esta mesma afirmação da agência negra que eles visaram. Explicando a proliferação em todo o Sul nas décadas de 1890 e 1900 de convenções constitucionais que restringiam o direito de voto, um delegado de uma dessas convenções declarou: “O grande princípio subjacente deste movimento da Convenção. . . Foi a eliminação do negro da política deste Estado.”{\color{blue}7}
 \par 
A história laboral americana está repleta de queixas semelhantes por parte das classes empregadoras e dos seus aliados no governo: não que os trabalhadores sindicalizados sejam violentos, perturbadores ou não lucrativos, mas que sejam independentes e auto-organizados. Na verdade, a sua auto-organização é tão potente que ameaça – aos olhos dos seus superiores – tornar supérfluos o empregador e o Estado. Durante a Grande Revolta de 1877, os trabalhadores ferroviários em greve em St.
 \par 
Operando os próprios trens. Temendo que o público pudesse concluir que os trabalhadores eram capazes de administrar a ferrovia, os proprietários tentaram impedi-los - na verdade, lançando uma greve própria para provar que eram os proprietários, e somente os proprietários, que poderiam fazer os trens circularem. na hora. Durante a greve geral de Seattle em 1919, os trabalhadores não mediram esforços para fornecer serviços governamentais básicos, incluindo a lei e a ordem. Eles tiveram tanto sucesso que o prefeito concluiu que era isso, a capacidade independente dos trabalhadores de limitar a violência e a anarquia, que representava a maior ameaça.
 \par 
A chamada greve simpática de Seattle foi uma tentativa de revolução. O fato de não ter havido violência não altera o fato. . . . É verdade que não houve armas de fogo, nem bombas, nem assassinatos. A revolução, repito, não precisa de violência. A greve geral, tal como praticada em Seattle, é em si a arma da revolução, ainda mais perigosa porque é silenciosa. . . . Ou seja, coloca o governo fora de operação. E isso é tudo o que há para se revoltar – não importa quão alcançada seja.{\color{blue}8}
 \par 
No século XX, os juízes denunciaram regularmente os trabalhadores sindicalizados por formularem as suas próprias definições de direitos e por compilarem o seu próprio registo de regras de chão de fábrica. Trabalhadores como estes, afirmou um tribunal federal, viam-se como “expoentes de alguma lei superior a essa. . . Administrado por tribunais.” Exerciam “poderes pertencentes apenas ao Governo”, declarou o Supremo Tribunal, constituindo-se como um “tribunal autonomeado” da lei e da ordem. {\color{blue}9} O conservadorismo é a voz teórica desta animosidade contra a agência das classes subordinadas. Fornece o argumento mais consistente e profundo sobre a razão pela qual as ordens inferiores não devem ser autorizadas a exercer a sua vontade independente, por que não devem ser autorizadas a governar a si mesmas ou ao sistema político. A submissão é o seu primeiro dever, a agência, a prerrogativa da elite.
 \par 
Embora se afirme frequentemente que a esquerda defende a igualdade enquanto a direita defende a liberdade, esta noção distorce o real desacordo entre direita e esquerda. Historicamente, o conservador tem favorecido a liberdade para as ordens superiores e a restrição para as ordens inferiores. Por outras palavras, o que o conservador vê e não gosta na igualdade não é uma ameaça à liberdade, mas a sua extensão. Pois nessa extensão ele vê uma perda da sua própria liberdade. “Todos estamos de acordo quanto à nossa própria liberdade”, declarou Samuel Johnson. “Mas não estamos de acordo quanto à liberdade dos outros: pois na proporção que tomamos, outros devem perder. Acredito que dificilmente desejamos que a multidão tenha liberdade para nos governar.” {\color{blue}10} Tal foi a ameaça que Edmund Burke viu na Revolução Francesa: não apenas uma expropriação de propriedade ou uma explosão de violência, mas uma inversão das obrigações de deferência e comando. “Os niveladores”, afirmou ele, “apenas mudam e pervertem a ordem natural das coisas”.
 \par 
A profissão de cabeleireiro ou de vendedor de sebo não pode ser uma questão de honra para ninguém – para não falar de uma série de outros empregos mais servis. Tais descrições de homens não deveriam suportar a opressão do Estado; mas o Estado sofre opressão se tais pessoas, individual ou coletivamente, tiverem permissão para governar.{\color{blue}11}
 \par 
Em virtude de serem membros de um sistema político, admitiu Burke, os homens tinham muitos direitos – aos frutos do seu trabalho, à sua herança, à educação e muito mais. Mas o único direito que ele se recusou a conceder a todos os homens foi a “partilha de poder, autoridade e direção” que eles poderiam pensar que deveriam ter “na gestão do Estado”.{\color{blue}12}
 \par 
Mesmo quando as exigências da esquerda se deslocam para a esfera económica, a ameaça da extensão da liberdade é grande. Se as mulheres e os trabalhadores receberem os recursos económicos para fazerem escolhas independentes, serão livres para não obedecer aos seus maridos e empregadores.
 \par 
É por isso que Lawrence Mead, um dos principais opositores intelectuais do Estado-providência nas décadas de 1980 e 1990, declarou que o beneficiário da assistência social “deve tornar-se menos livre em certos sentidos, em vez de mais”. {\color{blue}13} Para o conservador, a igualdade pressagia mais do que uma redistribuição de recursos, oportunidades e resultados – embora ele certamente também não goste destes. {\color{blue}14} O que a igualdade significa, em última análise, é uma rotação na sede do poder.
 \par 
O conservador não está errado ao interpretar a ameaça da esquerda nestes termos. Antes de morrer, G. A. Cohen, uma das vozes mais perspicazes do marxismo contemporâneo, defendeu que grande parte do programa de redistribuição económica da esquerda poderia ser entendido como implicando não um sacrifício da liberdade em prol da igualdade, mas uma extensão da liberdade do poucos para muitos. {\color{blue}15} E, de facto, os grandes movimentos modernos de emancipação – da abolição ao feminismo e à luta pelos direitos dos trabalhadores e pelos direitos civis – sempre postularam um nexo entre liberdade e igualdade. Saindo da família, da fábrica e do campo, onde a liberdade e a desigualdade são o outro lado da mesma moeda, eles fizeram da liberdade e da igualdade as partes irredutíveis, mas que se reforçam mutuamente, de um único todo. A ligação entre liberdade e igualdade não tornou o argumento a favor da redistribuição mais palatável para a direita. Tal como um conservador conservador se queixou da visão de John Dewey sobre a social-democracia: “As definições de liberdade e de igualdade têm sido tão manipuladas que ambas se referem aproximadamente à mesma condição”. {\color{blue}16} Longe de ser um truque progressista, contudo, esta síntese de liberdade e igualdade é um postulado central da política de emancipação. Se a política está de acordo com o postulado é, obviamente, outra história. Mas para o conservador, a preocupação é menos a traição do postulado do que o seu cumprimento.
 \par 
Uma das razões pelas quais o exercício de agência do subordinado agita tanto a imaginação conservadora é que ele ocorre de forma íntima.
 \par 
Contexto. Cada grande explosão política – a tomada da Bastilha, a tomada do Palácio de Inverno, a Marcha sobre Washington – é desencadeada por um estopim privado: a disputa pelos direitos e pela posição na família, na fábrica e no campo. Políticos e partidos falam de constituição e emendas, direitos naturais e privilégios herdados. Mas o verdadeiro tema das suas deliberações é a vida privada do poder. “Aqui está o segredo da oposição à igualdade das mulheres no Estado”, escreveu Elizabeth Cady Stanton. “Os homens não estão preparados para reconhecer isso em casa.” {\color{blue}17} Por trás do motim nas ruas ou do debate no Parlamento está a empregada que responde à sua patroa, o trabalhador que desobedece ao seu patrão. É por isso que os nossos argumentos políticos – não apenas sobre a família, mas também sobre o Estado-providência, os direitos civis e muito mais – podem ser tão explosivos: tocam nas relações de poder mais pessoais. É também por isso que tantas vezes cabe aos nossos romancistas explicar-nos a nossa política. No auge do movimento pelos direitos civis, James Baldwin viajou para Tallahassee. Ali, num aperto de mão imaginário, encontrou a transcrição oculta de uma crise constitucional.{\color{blue}18}
 \par 
Sou o único passageiro negro no aeroporto em ruínas de Tallahassee. É um dia opressivamente ensolarado. Um motorista negro, conduzindo um cachorrinho na coleira, está se encontrando com seu empregador branco. Ele é atencioso com a cadela, secretamente muito consciente de mim e respeitoso com ela de uma forma curiosamente vigilante e esperante. Ela está meio machucada, radiante e com o rosto empoado, encantada em ver os dois seres que tornam sua vida agradável. Tenho certeza de que nunca lhe ocorreu que algum deles tivesse a capacidade de julgá-la ou a julgaria com severidade. Ela quase poderia, ao se dirigir ao motorista, estar cumprimentando um amigo. Nenhum amigo poderia deixar seu rosto mais brilhante. Se ela estivesse sorrindo para mim daquele jeito, eu esperaria apertar sua mão. Mas se eu estendesse a mão, o pânico, a perplexidade e o horror tomariam conta disso.
 \par 
Face, a atmosfera escureceria e o perigo, até mesmo a ameaça de morte, preencheria imediatamente o ar.
 \par 
Com base em tais pequenos sinais e símbolos a cabala do sul
 \par 
O conflito sobre a escravidão americana — o precedente iminente para esse conjunto de peças da imaginação de Baldwin — oferece um exemplo instrutivo. Uma das características distintivas da escravidão nos Estados Unidos é que, diferentemente dos escravos no Caribe ou dos servos na Rússia, muitos escravos no Sul viviam em pequenas propriedades com seus senhores residentes. Os senhores sabiam os nomes de seus escravos; rastreavam seus nascimentos, casamentos e mortes; e realizavam festas para homenagear essas datas. A interação pessoal entre senhor e escravo era incomparável, levando Frederick Law Olmsted, que estava de visita, a comentar sobre a “estreita coabitação e associação de negros e brancos” na Virgínia, a “familiaridade e proximidade da intimidade que teriam sido notadas com espanto, se não com manifesto desagrado, em quase qualquer companhia casual no Norte”. {\color{blue}20} Somente as “relações de marido e mulher, pais e filhos, irmão e irmã”, escreveu o apologista da escravidão Thomas Dew, produziam “um vínculo mais próximo” do que o de senhor e escravo; a última relação, declarou William Harper, outro defensor da escravidão, era “uma das relações mais íntimas da sociedade”. {\color{blue}21} Por outro lado, depois que a escravidão foi abolida, muitos brancos lamentaram o esfriamento nas relações entre as raças. “Eu gosto do negro”, disse um mississippiano em 1918, “mas o vínculo entre nós não é tão próximo quanto era entre meu pai e seus escravos”.{\color{blue}22}
 \par 
Grande parte desta conversa era propaganda e auto-ilusão, é claro, mas num aspecto não era: a proximidade do senhor com o escravo criava um modo de governo excepcionalmente pessoal. Os senhores concebiam e aplicavam regras “excepcionalmente detalhadas” para seus escravos, ditando quando eles deveriam se levantar, comer, trabalhar, dormir, cuidar do jardim, visitar e orar. Os senhores decidiam as companheiras e os casamentos dos seus escravos. Eles
 \par 
Deram nomes aos filhos e, quando o mercado assim o determinou, separaram-nos dos pais. E embora os senhores – bem como os seus filhos e feitores – se aproveitassem dos corpos das suas escravas sempre que desejassem, eles acharam adequado patrulhar e punir toda e qualquer relação sexual entre os seus escravos. {\color{blue}23} Vivendo com os seus escravos, os senhores tinham meios diretos para controlar o seu comportamento e um mapa detalhado de todo o comportamento que havia para controlar.
 \par 
As consequências desta proximidade foram sentidas não apenas pelo escravo, mas também pelo senhor. Vivendo todos os dias com sua mestria, ele se identificou inteiramente com ela. Esta identificação era tão completa que qualquer sinal de desobediência do escravo – muito menos de sua emancipação – era visto como um ataque intolerável à sua pessoa. Quando Calhoun declarou que a escravatura “cresceu com a nossa sociedade e instituições, e está tão interligada com elas, que destruí-la seria destruir-nos como povo”, ele não se referia apenas à sociedade no seu conjunto ou abstracto. {\color{blue}24} Ele estava pensando em homens individuais absortos na experiência cotidiana de governar outros homens e mulheres. Tire essa experiência e você destruirá não apenas o mestre, mas também o homem – e os muitos homens que procuraram se tornar, ou pensavam que já eram, o mestre.
 \par 
Como o mestre colocava tão pouca distância entre si e sua maestria, ele faria esforços sem precedentes para manter suas posses. Por toda a América, os senhores de escravos defendiam seus privilégios, mas em nenhum lugar com a intensidade ou violência da classe dominante no Sul. Fora do Sul, escreveu C. Vann Woodward, o fim da escravidão foi “a liquidação de um investimento”. Dentro, foi “a morte de uma sociedade”. {\color{blue}25} E quando, após a Guerra Civil, a classe dominante lutou com igual ferocidade para restaurar seus privilégios e poder, foi a proximidade do comando, a proximidade do governo, que estava em primeiro lugar em sua mente. Como Henry McNeal Turner, um republicano negro na Geórgia, disse em 1871: “Eles não se importam tanto com o Congresso admitindo negros em seus salões... Mas eles
 \par 
Não quero que os negros cuidem deles em casa. Cem anos mais tarde, um meeiro negro no Mississipi ainda recorreria à expressão mais doméstica para descrever as relações entre negros e brancos: “Tínhamos de cuidar deles como os nossos filhos cuidam de nós”.{\color{blue}26}
 \par 
Quando o conservador olha para um movimento democrático a partir de baixo, isto (e o exercício da agência) é o que ele vê: uma terrível perturbação na vida privada do poder. Ao testemunhar a eleição de Thomas Jefferson em 1800, Theodore Sedgwick lamentou: “A aristocracia da virtude está destruída; a influência pessoal chegou ao fim.” {\color{blue}27} Às vezes o conservador está pessoalmente implicado nessa vida, às vezes não. Independentemente disso, é a sua apreensão da queixa privada por detrás da comoção pública que confere à sua teoria o seu engenho táctil e a sua ferocidade moral. “O verdadeiro objectivo” da Revolução Francesa, disse Burke ao Parlamento em 1790, é “romper todas as ligações, naturais e civis, que regulam e mantêm unida a comunidade através de uma cadeia de subordinação; levantar soldados contra seus oficiais; servos contra seus senhores; comerciantes contra seus clientes; artífices contra seus empregadores; inquilinos contra seus proprietários; curadores contra seus bispos; e filhos contra seus pais.” {\color{blue}28} A insubordinação pessoal tornou-se rapidamente um tema regular e consistente dos pronunciamentos de Burke sobre o desenrolar dos acontecimentos em França. Um ano depois, ele escreveu numa carta que, por causa da Revolução, “nenhuma casa está a salvo dos seus empregados, nenhum oficial dos seus soldados, e nenhum Estado ou constituição da conspiração e da insurreição”. {\color{blue}29} Num outro discurso perante o Parlamento em 1791, ele declarou que “uma constituição fundada no que foi chamado de direitos do homem” abriu a “caixa de Pandora” em todo o mundo, incluindo o Haiti: “Os negros levantaram-se contra os brancos, os brancos contra os negros, e cada um contra uns aos outros em hostilidade assassina; a subordinação foi destruída.” {\color{blue}30} Nada para os jacobinos, declarou ele no final da sua vida, era digno “do nome de virtude pública, a menos que indique violência sobre o privado”.{\color{blue}31}
 \par 
Tão poderosa é essa visão de erupção privada que pode transformar um homem reformista num homem reacionário. Educado no Iluminismo, John Adams acreditava que o “consentimento do povo” era “o único fundamento moral do governo”. {\color{blue}32} Mas quando a sua esposa sugeriu que uma versão silenciosa destes princípios fosse estendida à família, ele não gostou. “E, a propósito”, escreveu-lhe Abigail, “no novo código de leis que suponho que será necessário que você faça, desejo que você se lembre das damas e seja mais generoso e favorável a elas do que seus ancestrais. Não coloquem esse poder ilimitado nas mãos dos maridos. Lembre-se, todos os homens seriam tiranos se pudessem.” {\color{blue}33} A resposta do seu marido:
 \par 
Disseram-nos que a nossa luta afrouxou as amarras do governo em toda parte; que crianças e aprendizes eram desobedientes; que as escolas e faculdades se tornaram turbulentas; que os índios desprezavam seus tutores e os negros se tornavam insolentes com seus senhores. Mas a sua carta foi a primeira indicação de que outra tribo, mais numerosa e poderosa do que todas as outras, estava cada vez mais descontente.
 \par 
Embora ele tenha temperado sua resposta com brincadeiras lúdicas — ele rezou para que George Washington o protegesse do “despotismo da anágua” {\color{blue}34} — Adams estava claramente abalado por essa aparência de democracia na esfera privada. Em uma carta a James Sullivan, ele se preocupou que a Revolução iria “confundir e destruir todas as distinções”, desencadeando por toda a sociedade um espírito de insubordinação tão intenso que toda a ordem seria dissolvida. “Não haverá fim para isso.” {\color{blue}35} Não importa quão democrático seja o estado, era imperativo que a sociedade permanecesse uma federação de domínios privados, onde os maridos governassem as esposas, os mestres governassem os aprendizes e cada um “devesse saber seu lugar e ser obrigado a mantê-lo.”{\color{blue}36}
 \par 
Historicamente, o conservador tem procurado impedir a marcha da democracia tanto na esfera pública como na esfera privada, no pressuposto de que os avanços numa necessariamente estimulam os avanços na outra. “Para manter o Estado fora das mãos do povo”, escreveu o monarquista francês Louis de Bonald, “é necessário manter a família fora das mãos das mulheres e das crianças”. {\color{blue}37} Mesmo nos Estados Unidos, este esforço tem produzido frutos periodicamente. Apesar da nossa narrativa Whiggista sobre a ascensão constante da democracia, o historiador Alexander Keyssar demonstrou que a luta pelo voto nos Estados Unidos tem sido tanto uma história de retracção e contracção como de progresso e expansão, “com tensões e apreensões de classe” por parte das elites políticas e económicas constituindo “o obstáculo mais importante ao sufrágio universal. . . Do final do século XVIII até a década de 1960.”{\color{blue}38}
 \par 
Ainda assim, a posição mais profunda e profética da direita tem sido a de Adams: ceda o domínio do público, se for necessário, mantenha-se firme no privado. Permitir que homens e mulheres se tornem cidadãos democráticos do Estado; certifique-se de que eles permaneçam súditos feudais na família, na fábrica e no campo. A prioridade do argumento político conservador tem sido a manutenção de regimes privados de poder – mesmo à custa da força e da integridade do Estado. Vemos esta aritmética política em ação na decisão de um tribunal federalista em Massachusetts de que uma mulher legalista que fugiu da Revolução era ajudante de seu marido e, portanto, não deveria ser responsabilizada pela fuga e não deveria ter sua configuração de propriedade espalhada por o Estado; na recusa dos proprietários de escravos do Sul em ceder os seus escravos à causa confederada; e a insistência mais recente do Supremo Tribunal de que as mulheres não poderiam ser legalmente obrigadas a fazer parte dos júris porque “ainda são consideradas o centro do lar e da vida familiar”, com as suas “próprias responsabilidades especiais”. {\color{blue}39} O conservadorismo, então, não é um compromisso com um governo e liberdade limitados – ou uma cautela em relação à mudança, uma crença na evolução evolutiva.
 \par 
Reforma, ou uma política de virtude. Estes podem ser os subprodutos do conservadorismo, um ou mais dos seus modos de expressão historicamente específicos e em constante mudança. Mas eles não são o seu propósito animador. O conservadorismo também não é uma fusão improvisada de capitalistas, cristãos e guerreiros, pois essa fusão é impulsionada por uma força mais elementar – a oposição à libertação de homens e mulheres dos grilhões dos seus superiores, particularmente na esfera privada. Tal visão pode parecer muito distante da defesa libertária do mercado livre, com a sua celebração do indivíduo atomista e autónomo. Mas não é. Quando o libertário olha para a sociedade, ele não vê indivíduos isolados; ele vê grupos privados, muitas vezes hierárquicos, onde um pai governa sua família e um proprietário seus empregados.{\color{blue}40}
 \par 
Não há uma simples defesa do próprio lugar e dos privilégios – o conservador, como já disse, pode ou não estar diretamente envolvido ou beneficiar das práticas de governo que defende; muitos, como veremos, não o são – a posição conservadora deriva de uma convicção genuína de que um mundo assim emancipado será feio, brutal, vil e monótono. Faltará a excelência de um mundo onde o homem melhor comanda o pior. Quando Burke acrescenta, na carta citada acima, que o “grande objetivo” da Revolução é “erradicar aquela coisa chamada Aristocrata ou Nobre e Cavalheiro”, ele não está simplesmente se referindo ao poder da nobreza; ele também está se referindo à distinção que o poder traz ao mundo. {\color{blue}41} Se o poder desaparecer, a distinção desaparecerá com ele. Esta visão da ligação entre excelência e governo é o que une na América do pós-guerra aquela aliança improvável do libertário, com a sua visão do poder ilimitado do empregador no local de trabalho; o tradicionalista, com sua visão do governo do pai no lar; e o estatista, com a visão de um líder heróico pressionando a mão sobre a face da terra. Cada um, à sua maneira, subscreve esta declaração típica, do século XIX, do credo conservador:
 \par 
“Obedecer a um verdadeiro superior. . . É uma das mais importantes de todas as virtudes – uma virtude absolutamente essencial para a realização de qualquer coisa grande e duradoura.”{\color{blue}42}
 \par 
A noção de que as ideias conservadoras são um modo de prática contra-revolucionária é susceptível de levantar algumas sobrancelhas, até mesmo arrepios. Há muito que é um axioma da esquerda que a defesa do poder e dos privilégios é um empreendimento desprovido de ideias. “A história intelectual”, afirma um estudo recente sobre o conservadorismo americano, “nunca é indesejada”, mas “não é a abordagem mais direta para explicar o poder do conservadorismo na América”. {\color{blue}43} Os escritores liberais sempre retrataram a política de direita como um pântano emocional e não como um movimento de opinião ponderada: Thomas Paine afirmou que a contra-revolução implicava “uma obliteração do conhecimento”; Lionel Trilling descreveu o conservadorismo americano como uma mistura de “gestos mentais irritáveis ​​que procuram assemelhar-se a ideias”; Robert Paxton chamou o fascismo de “caso do intestino”, e não “do cérebro”. {\color{blue}44} Os conservadores, por seu lado, tenderam a concordar. {\color{blue}45} Afinal, foi Palmerston, quando ainda era conservador, quem primeiro atribuiu o epíteto de “estúpido” ao Partido Conservador. Desempenhando o papel de proprietários rurais estúpidos, os conservadores abraçaram a posição de F. J. C. Hearnshaw de que “é normalmente suficiente para fins práticos que os conservadores, sem dizer nada, apenas se sentem e pensem, ou mesmo que simplesmente se sentem”. {\color{blue}46} Embora as conotações aristocráticas desse discurso já não ressoem, o conservador ainda mantém o rótulo de inculto e iletrado; faz parte do seu charme populista e apelo demótico. Como observa o conservador Washington Times, os republicanos “muitas vezes autodenominam-se o ‘partido estúpido’”. {\color{blue}47} Nada, como veremos, poderia estar mais longe da verdade. O conservadorismo é uma práxis movida por ideias, e nenhuma quantidade de presunção da direita ou polêmica da esquerda pode reduzir ou apagar o catálogo de ideias que ali se encontra.
 \par 
Os próprios conservadores provavelmente ficarão desanimados com este argumento por uma razão diferente: ameaça a pureza e a profundidade das ideias conservadoras. Para muitos, a palavra “reação” conota uma busca impensada e humilde pelo poder. {\color{blue}48} Mas a reação não é real ex. Começa a partir de uma posição de princípio – de que alguns estão aptos, e portanto devem, governar outros – e depois recalibra esse princípio à luz de um desafio democrático vindo de baixo. Esta recalibração não é uma tarefa fácil, pois tais desafios tendem, pela sua própria natureza, a refutar o princípio. Afinal de contas, se uma classe dominante está verdadeiramente preparada para governar, porque e como permitiu que surgisse um desafio ao seu poder? O que o surgimento de um diz sobre a aptidão do outro? {\color{blue}49} O conservador enfrenta um obstáculo adicional: como defender um princípio de governo num mundo onde nada é sólido, tudo está em fluxo? Desde o momento em que o conservadorismo entrou em cena, teve de enfrentar o declínio das ideias antigas e medievais de um universo ordenado, no qual hierarquias permanentes de poder refletiam a estrutura eterna do cosmos. A derrubada do antigo regime revela não só a fraqueza e a incompetência dos seus líderes, mas também uma verdade maior sobre a falta de design no mundo. (A ideia de que o conservadorismo reflecte a revelação de que o mundo não tem hierarquias naturais pode parecer estranha na nossa era do Design Inteligente. Mas, como Kevin Mattson e outros salientaram, o Design Inteligente não se baseia no mesmo tipo de pressuposto medieval de uma empresa estrutura eterna para o universo, e há mais do que um toque de relativismo e ceticismo nos seus argumentos. Na verdade, um dos principais proponentes do Design Inteligente afirmou que, embora “não seja pós-modernista”, ele “aprendeu muito” com o pós-modernismo. ) Reconstruir o antigo regime face ao declínio da fé em hierarquias permanentes revelou-se uma tarefa difícil. Não é de surpreender que também tenha produzido algumas das obras mais notáveis ​​do pensamento moderno.
 \par 
Mas há outra razão pela qual devemos ser cautelosos relativamente ao esforço para rejeitar o impulso reaccionário do conservadorismo, e essa razão é o testemunho da própria tradição. Desde Burke, tem sido motivo de orgulho entre os conservadores que o seu modo de pensamento seja contingente. Ao contrário dos seus adversários de esquerda, eles não elaboram um plano antes dos acontecimentos. Eles lêem situações e circunstâncias, não textos e tomos; seu modo preferido é a adaptação e a insinuação, em vez da afirmação e da declamação. Há uma certa verdade nesta afirmação, como veremos: a mente conservadora é extraordinariamente flexível, alerta às mudanças no contexto e na sorte muito antes de os outros perceberem que estão a ocorrer. Com a sua profunda consciência da passagem do tempo, o conservador possui um virtuosismo tático que poucos conseguem igualar. Parece lógico que o conservadorismo esteja intimamente ligado e com as suas antenas sempre sensíveis aos movimentos e contra-movimentos do poder esboçados acima. Estas são, como já disse, a história da política moderna, e pareceria estranho se uma mente tão sintonizada com as contingências circundantes não fosse bem versada nessa história. Não apenas bem versado, mas despertado e despertado por ela como por nenhuma outra história.
 \par 
Na verdade, desde a afirmação de Burke de que ele e a sua turma tinham sido “alarmados e levados a uma verdadeira expulsão” pela Revolução Francesa até à admissão de Russell Kirk de que o conservadorismo é um “sistema de ideias” que “tem sustentado os homens. . . Na sua resistência contra as teorias radicais e a transformação social desde o início da Revolução Francesa”, o conservador tem afirmado consistentemente que o seu conhecimento é produzido em reação à esquerda. {\color{blue}51} (Burke continuaria estabelecendo como seu “fundamento” a noção de que “nunca” “existiu” um mal maior do que a Revolução Francesa.) {\color{blue}52} Por vezes, essa afirmação tem sido explícita. Três vezes primeiro-ministro, Salisbury escreveu em 1859 que “hostilidade ao radicalismo, hostilidade incessante e implacável, é a definição essencial do conservadorismo. O medo de que os Radicais possam triunfar é a única causa final que o Partido Conservador pode alegar
 \par 
Para sua própria existência.” {\color{blue}53} Mais de meio século depois, o seu filho Hugh Cecil – entre outras coisas, padrinho de casamento de Winston Churchill e reitor de Eton – reafirmou a posição do pai: “Penso que o governo acabará por descobrir que só existe uma forma de de derrotar as tácticas revolucionárias e isso é através da apresentação de um corpo organizado de pensamento que seja não revolucionário. Esse corpo de pensamento eu chamo de conservadorismo.” {\color{blue}54} Outros, como Peel, seguiram um caminho mais tortuoso para chegar ao mesmo lugar:
 \par 
Meu objetivo há alguns anos, aquele que tenho trabalhado arduamente para realizar, tem sido lançar as bases de um grande partido que, existindo na Câmara dos Comuns e derivando sua força da vontade popular, deveria diminuir o risco e amortecer o choque de uma colisão entre os dois ramos deliberativos da legislatura - o que deveria nos permitir conter a ânsia muito importuna de homens bem-intencionados por mudanças precipitadas e precipitadas na constituição e nas leis do país, e pela qual nós deveria ser capaz de dizer, com uma voz de autoridade, ao espírito inquieto da mudança revolucionária: “Aqui estão os teus limites e aqui cessarão as tuas vibrações”.{\color{blue}55}
 \par 
Para que não pensemos que tais sentimentos – e circunlocuções – são peculiarmente ingleses, consideremos como o historiador da corte da direita americana abordou a questão em 1976. “O que é o conservadorismo?” George Nash perguntou em seu agora clássico The Conservative Intellectual Movement in America since 1945. Depois de uma página de hesitação - o conservadorismo resiste à definição, “varia enormemente com o tempo e o lugar” (que ideia política não varia?), não deveria ser “ confundido com a Direita Radical” – Nash decidiu-se por uma resposta que poderia ter sido dada (na verdade, foi dada) por Peel, Salisbury e seu filho, Kirk, e a maioria dos pensadores da Direita Radical. O conservadorismo, disse ele, é definido pela “resistência a certas forças percebidas como
 \par 
Esquerdista, revolucionário e profundamente subversivo em relação ao que os conservadores da época consideravam que valia a pena valorizar, defender e talvez morrer.”{\color{blue}56}
 \par 
Estas são as profissões explícitas do credo contra-revolucionário. Mais interessantes são as declarações implícitas, onde a antipatia pelo radicalismo e pelas reformas está incorporada na própria sintaxe do argumento. Tomemos a famosa definição de Michael Oakeshott em seu ensaio “On Being Conservative”: “Ser conservador, então, é preferir o familiar ao desconhecido, preferir o experimentado ao não experimentado, o fato ao mistério, o real ao possível, o limitado para o ilimitado, o próximo para o distante, o suficiente para o superabundante, o conveniente para o perfeito, o riso presente para a felicidade utópica. Parece que não se pode desfrutar dos fatos e do mistério, do próximo e do distante, do riso e da felicidade. É preciso escolher. Longe de afirmar uma simples hierarquia de preferências, o ou/ou de Oakeshort sinaliza que estamos num terreno existencial, onde a escolha não é entre algo e o seu oposto, mas entre algo e a sua negação. O conservador desfrutaria de coisas familiares na ausência de forças que procurassem a sua destruição, admite Oakeshott, mas o seu prazer “será mais forte quando” for “combinado com um risco evidente de perda”. O conservador é um “homem que tem plena consciência de ter algo a perder e do qual aprendeu a cuidar”. E embora Oakeshott sugira que tais perdas podem ser provocadas por uma variedade de forças, os engenheiros parecem invariavelmente trabalhar à esquerda. (Marx e Engels são “os autores do mais estupendo do nosso racionalismo político”, escreve ele noutro lugar. “Nada... se compara ao” seu utopismo abstrato.) Por essa razão, “não é de todo inconsistente ser conservador”. em relação ao governo e radical em relação a quase todas as outras atividades.” {\color{blue}57} Nada inconsistente – ou totalmente necessário? O radicalismo é a razão de ser do conservadorismo; se for, o conservadorismo também vai. {\color{blue}58} Mesmo quando o conservador procura libertar-se deste diálogo com a esquerda, ele não consegue, pelos seus motivos mais líricos – mudança orgânica, conhecimento tácito, ordenação
 \par 
Liberdade, prudência e precedente — são quase inaudíveis sem o chamado e a resposta da esquerda. Como Disraeli descobriu em sua Vindication of the English Constitution (1835), é somente em contraste com um suposto racionalismo revolucionário que a invocação da sabedoria antiga e tácita pode ter alguma compra na mente moderna.
 \par 
A formação de um governo livre em larga escala, embora seja seguramente um dos problemas mais interessantes da humanidade, é certamente a maior conquista da inteligência humana. Talvez eu devesse considerá-lo uma conquista sobre-humana; pois requer uma prudência tão refinada, um conhecimento tão abrangente e uma sagacidade tão perspicaz, unidos a poderes de combinação tão quase ilimitados, que é quase em vão esperar que qualidades tão raras sejam reunidas em uma mente solitária. Certamente esta soma de bónus não se encontra escondida atrás de uma barricada revolucionária, ou flutuando nas sarjetas sangrentas de uma metrópole incendiária. Ela não pode ser rabiscada – esta grande invenção – numa manhã, no envelope de uma carta de algum monarca redigidor de estatutos, ou esboçada com ridícula facilidade no vaidoso livro de lugar-comum de um sábio utilitarista.{\color{blue}59}
 \par 
Há mais nesta estrutura antagónica de argumento do que os simples anticorpos da política partidária, a tomada de posição de oposição que é um requisito para vencer eleições. Como argumentou Karl Mannheim, o que distingue o conservadorismo do tradicionalismo – a tendência “vegetativa” universal de permanecer apegado às coisas como elas são, que se manifesta em comportamentos não políticos, como a recusa em comprar um novo par de calças até que o par atual esteja em pedaços além do limite. reparação – é que o conservadorismo é um esforço deliberado e consciente para preservar ou recordar “aquelas formas de experiência que já não podem ser obtidas de uma forma autêntica”. O conservadorismo “torna-se consciente e
 \par 
Reflexivo quando outros modos de vida e pensamento aparecem em cena, contra os quais é compelido a pegar em armas na luta ideológica.” {\color{blue}60} Onde o tradicionalista pode tomar os objetos de desejo como garantidos — ele pode apreciá-los como se estivessem à mão porque estão à mão — o conservador não pode. Ele busca apreciá-los precisamente como estão sendo — ou foram — tirados. Se ele espera apreciá-los novamente, ele deve contestar seu desinvestimento no domínio público. Ele deve falar deles em uma linguagem que seja politicamente útil e inteligível. Mas assim que esses objetos entram no meio do discurso político, eles deixam de ser itens de experiência vivida e se tornam incidentes de uma ideologia. Eles são envolvidos em uma narrativa de perda — na qual o revolucionário ou reformista desempenha um papel necessário — e apresentados em um programa de recuperação. O que era tácito se torna articulado, o que era fluido se torna formal, o que era prática se torna polêmico. {\color{blue}61} Mesmo que a teoria seja um hino à prática — como o conservadorismo costuma ser — ela não pode escapar de se tornar uma polêmica. O conservador mais exigente que se dignaria a entrar na rua é compelido pela esquerda a pegar uma pedra de calçada e jogá-la nas barricadas. Como Lord Hailsham colocou em seu Case for Conservatism de 1947:
 \par 
Os conservadores não acreditam que a luta política seja a coisa mais importante na vida. Nisso eles a diferenciam dos comunistas, socialistas, nazistas, fascistas, credores sociais e da maioria dos membros do Partido Trabalhista Britânico. Os mais simples dentre eles preferem a caça à raposa – a religião mais sábia. Para a grande maioria dos conservadores, a religião, a arte, o estudo, a família, o país, os amigos, a música, a diversão, o dever, todas as alegrias e riquezas da existência das quais os pobres, não menos que os ricos, são os proprietários indefectíveis, tudo isto é mais elevado. na escala do que sua serva, a luta política. Isso os torna fáceis de derrotar – no início. Mas, uma vez derrotados, manterão esta crença com o fanatismo de um cruzado e a obstinação de um inglês.{\color{blue}62}
 \par 
Dado que há tanta confusão sobre a oposição do conservadorismo à esquerda, é importante que sejamos claros sobre o que o conservador é e o que não é a oposição na esquerda. Não é uma mudança abstrata. Nenhum conservador se opõe à mudança como tal ou defende a ordem como tal. O conservador defende ordens específicas – regimes de governo hierárquicos, muitas vezes privados – partindo do pressuposto, em parte, de que hierarquia é ordem. “A ordem não pode ser obtida”, declarou Johnson, “mas por subordinação”. {\color{blue}63} Para Burke, era axiomático que “quando a multidão não está sob esta disciplina” dos “mais sábios, mais experientes e mais opulentos”, “dificilmente se pode dizer que estão na sociedade civil”. {\color{blue}64} Além disso, ao defender tais ordens, o conservador invariavelmente lança-se num programa de reacção e contra-revolução, exigindo muitas vezes uma revisão do próprio regime que defende. “Se quisermos que as coisas permaneçam como estão”, na formulação clássica de Lampedusa, “as coisas terão de mudar”. {\color{blue}65} Para preservar o regime, como mostro na parte 1, o conservador deve reconstruir o regime. Este programa implica muito mais do que os clichés sobre “preservação através da renovação” poderiam sugerir: muitas vezes, pode exigir que o conservador tome as medidas mais radicais em nome do regime.
 \par 
Alguns dos mais enfadonhos partidários da ordem à direita têm ficado mais do que felizes, quando lhes convém, entregar-se a um pouco de caos e loucura. Kirk, o autodenominado burkeano, desejava “esposar o conservadorismo com a veemência de um radical. O conservador pensante, na verdade, deve assumir algumas das características externas do radical de hoje: ele deve fuçar nas raízes da sociedade, na esperança de restaurar o vigor de uma velha árvore estrangulada na vegetação rasteira das paixões modernas. .” Isso foi em 1954. Quinze anos depois, no auge do movimento estudantil, ele escreveu: “Tendo sido durante duas décadas um crítico mordaz do que é tolamente chamado de ensino superior na América, confesso que estou gostando um pouco. . . O cumprimento das minhas previsões e a situação atual do establishment educacional. Eu até confesso que estou furtivo
 \par 
Simpatia, de certa forma, com os revolucionários do campus.” Em God and Man at Yale, William F. Buckley declarou os conservadores “os novos radicais”. Ao ler os primeiros números da National Review, Dwight Macdonald estava inclinado a concordar: “Se [Buckley] tivesse nascido uma geração antes, ele estaria fazendo as cafeterias da 14th Street vibrarem com a dialética marxista”. {\color{blue}66} Até o próprio Burke escreveu que “a loucura dos sábios” é “melhor que a sobriedade dos tolos”.{\color{blue}67}
 \par 
Há uma razão bastante simples para a adoção do radicalismo na direita, e tem a ver com o imperativo reacionário que está no cerne da doutrina conservadora. O conservador não se opõe apenas à esquerda; ele também acredita que a esquerda tem estado no comando desde, dependendo de quem está contando, desde a Revolução Francesa ou a Reforma. {\color{blue}68} Se quiser preservar o que valoriza, o conservador deve declarar guerra contra a cultura tal como ela é. Embora o espírito de oposição militante permeie todo o discurso conservador, Dinesh D’Souza expôs o caso de forma mais clara.
 \par 
Normalmente, o conservador tenta conservar, manter os valores da sociedade existente. Mas. . . E se a sociedade existente for inerentemente hostil às crenças conservadoras? É tolice um conservador tentar conservar essa cultura. Em vez disso, ele deve procurar miná-lo, frustrá-lo, destruí-lo na raiz. Isso significa que o conservador deve. . . Seja filosoficamente conservador, mas temperamentalmente radical.{\color{blue}69}
 \par 
Por esta altura, também já deve estar claro que não é ao estilo ou ao ritmo da mudança que os conservadores se opõem. O teórico conservador gosta de traçar uma “distinção manifestamente marcada” entre a reforma evolutiva e a mudança radical. {\color{blue}70} O primeiro é lento, incremental e adaptativo; a segunda é rápida, abrangente e intencional. Mas essa distinção, tão cara a Burke e aos seus seguidores, é muitas vezes menos clara na prática
 \par 
Do que o teórico permite. {\color{blue}71} A teoria política foi concebida para ser abstracta, mas que abstracção impulsionou programas políticos tão diametralmente opostos como a preferência pela reforma em vez do radicalismo, a evolução em vez da revolução? Em nome de uma mudança lenta, orgânica e adaptativa, os autodeclarados conservadores opuseram-se ao New Deal (Robert Nisbet, Kirk e Whittaker Chambers) e endossaram o New Deal (Peter Viereck, Clinton Rossiter e Whittaker Chambers). {\color{blue}72} A crença numa reforma evolucionista poderia levar alguém a adoptar uma defesa hayekiana do mercado livre ou do socialismo democrático de Edward Bernstein. “Mesmo os socialistas fabianos”, observa Nash sarcasticamente, “que acreditavam na ‘inevitabilidade da gradualidade’ podem ser rotulados de conservadores”. {\color{blue}73} Por outro lado, como apontou Abraham Lincoln, é tão fácil para a esquerda reivindicar o manto da preservação como é para a direita. “Vocês dizem que são conservadores”, declarou ele aos proprietários de escravos.
 \par 
Eminentemente conservadores – embora sejamos revolucionários, destrutivos ou algo do género. O que é conservadorismo? Não é adesão ao velho e experimentado, contra o novo e não experimentado? Mantemo-nos, defendemos, a mesma velha política sobre o ponto controverso que foi adoptada pelos “nossos pais que moldaram o governo sob o qual vivemos”; enquanto vocês, de comum acordo, rejeitam, exploram e cuspem nessa velha política e insistem em substituí-la por algo novo. . . . Nenhum de todos os seus vários planos pode apresentar um precedente ou um defensor no século em que o nosso Governo se originou. Considere, então, se a sua alegação de conservadorismo para si mesmo e a sua acusação de destrutividade contra nós se baseiam nos fundamentos mais claros e estáveis.{\color{blue}74}
 \par 
Mais frequentemente, porém, a indefinição da distinção permitiu que o conservador se opusesse à reforma alegando que esta conduziria à revolução ou que seria uma revolução. (Na verdade, com
 \par 
Com excepção de Peel e Baldwin, nenhum líder conservador alguma vez prosseguiu um programa consistente de preservação através de reformas, e mesmo Peel não conseguiu persuadir o seu partido a segui-lo. {\color{blue}75}) O próprio Burke não ficou imune ao argumento de que a reforma leva à revolução. Embora tenha passado a maior parte da década anterior à Revolução Americana contestando esse argumento, ele ainda se perguntava: “Quando você abre” uma constituição “à investigação de uma parte”, o que parece ser a definição de reforma lenta, “onde a investigação irá parar?” {\color{blue}76} Outros conservadores argumentaram que qualquer exigência proveniente ou em nome das classes inferiores, por mais tépida ou tardia que seja, é demasiado, demasiado cedo, demasiado rápido. Reforma é revolução, melhoria é insurreição. “Pode ser bom ou ruim”, escreveu um sombrio Lord Carnarvon sobre a Segunda Lei de Reforma de 1867 – um projeto de lei que estava sendo elaborado há vinte anos e que triplicou o tamanho do eleitorado britânico – “mas é uma revolução”. Sem a qualificação inicial, isto foi uma repetição do que Wellington havia dito sobre a primeira Lei de Reforma. {\color{blue}77} Do outro lado do Atlântico, o contemporâneo de Wellington, Nicholas Biddle, denunciava o veto de Andrew Jackson ao Segundo Banco (o mais exercido constitucionalmente dos poderes constitucionais) em termos semelhantes: “Tem toda a fúria de uma pantera acorrentada mordendo as barras da sua jaula. É realmente um manifesto de anarquia – tal como Marat ou Robespierre poderiam ter emitido para a multidão.”{\color{blue}78}
 \par 
O conservador de hoje pode ter feito as pazes com algumas mancitações do passado; outros, como sindicatos e liberdade reprodutiva, ele ainda contesta. Mas isso não altera o facto de que quando essa emancipação surgiu pela primeira vez como questão, seja no contexto da revolução ou da reforma, o seu antecessor estava muito provavelmente contra ela. Michael Gerson, ex-redator de discursos de George W. Bush, é um dos poucos conservadores contemporâneos que reconhece a história da oposição conservadora à emancipação. Enquanto outros conservadores gostam de reivindicar o manto abolicionista ou dos direitos civis, Gerson admite que “a honestidade requer o reconhecimento de que
 \par 
Muitos conservadores, noutras épocas, foram hostis às reformas de motivação religiosa” e que “o hábito mental conservador outrora se opôs à maioria destas mudanças”. {\color{blue}79} Na verdade, como sugeriu Samuel Huntington há meio século, dizer não a tais movimentos em tempo real pode ser o que torna alguém um conservador ao longo do tempo.{\color{blue}80}
 \par 
Forjado em resposta aos desafios vindos de baixo, o conservadorismo não tem a calma ou a compostura que acompanham uma herança duradoura de poder. Procurar-se-á em vão, em todo o cânone da direita, garantias constantes de uma Grande Cadeia do Ser. As declarações conservadoras de unidade orgânica, tais como são, ou têm um ar de desespero silencioso – e não tão silencioso – ou, como no caso de Kirk, carecem da textura, da sensação de conhecimento, de um testemunho de longa data do poder. Mesmo as declarações de Maistre sobre a providência divina não conseguem esconder ou conter a turbulenta democracia que as gerou. Feitas e mobilizadas para contrariar as reivindicações de emancipação, tais declarações não revelam uma densa ecologia de deferência; eles revelam, em vez disso, uma floresta cada vez mais rarefeita. O conservadorismo tem a ver com poder sitiado e poder protegido. É uma doutrina ativista para um tempo ativista. Aumenta em resposta aos movimentos vindos de baixo e diminui em resposta ao seu desaparecimento, como admitem Hayek e outros conservadores.{\color{blue}81}
 \par 
Longe de comprometer a visão de excelência acima exposta – na qual as prerrogativas do governo deveriam trazer um elemento de grandeza a um mundo que de outra forma seria monótono e inconstante – o imperativo activista apenas a fortalece. “Luz e perfeição”, escreveu Matthew Arnold, “consistem não em descansar e ser, mas em crescer e se tornar, num avanço perpétuo em beleza e sabedoria”. {\color{blue}82} Para o conservador, o poder em repouso é um poder em declínio. A “mera gestão dos recursos já existentes”, escreveu Joseph Schumpeter sobre as dinastias industriais, “por mais meticulosa que seja, é sempre
 \par 
Característica de uma posição em declínio.” {\color{blue}83} Se o poder deve atingir a distinção que o conservador lhe associa, ele deve ser exercido. {\color{blue}84} E não há melhor maneira de exercer poder do que defendê-lo contra um inimigo de baixo. A contrarrevolução, em outras palavras, é uma das maneiras pelas quais o conservador faz o feudalismo parecer fresco e o medievalismo moderno.
 \par 
Mas não é o único caminho. O conservadorismo também oferece uma defesa do governo, independente do seu imperativo contrarrevolucionário, que é agonística e dinâmica e dispensa o tradicionalismo sóbrio e os registos harmónicos das hierarquias do passado. E aqui chegamos às sugestões mais profundas do conservador sobre a boa vida, sobre aquela utopia reaccionária que ele espera um dia concretizar. Ao contrário do passado feudal, onde o poder era presumido e os privilégios herdados, o futuro conservador prevê um mundo onde o poder é demonstrado e os privilégios conquistados: não nos corredores anti-sépticos e anódinos da meritocracia, onde a admissão é facilmente garantida - “o caminho para a eminência e o poder, a partir de uma condição obscura, não deve ser facilitado demais, nem nada demais, é claro” {\color{blue}85} - mas na árdua luta pela supremacia. Nessa luta, nada importa, nem a herança, as ligações sociais ou os recursos económicos, mas a inteligência nativa e a força inata de alguém. A excelência genuína é revelada e recompensada, a verdadeira nobreza é garantida. “'Nitor in adversum' [Eu luto contra a adversidade] é o lema para um homem como eu”, declara Burke, depois de demitir um político nascido na mansão que foi “enfaixado, embalado e embalado como legislador. ” {\color{blue}86} Mesmo o racista mais biologicamente inclinado e determinista acredita que os membros da raça superior devem pessoalmente arrancar o seu direito de governar através da subjugação ou eliminação das raças inferiores.
 \par 
O reconhecimento de que a raça é o substrato de toda a civilização não deve, contudo, levar ninguém a sentir que pertencer a uma
 \par 
A raça superior é uma espécie de sofá confortável onde ele pode dormir. . . . A herança biológica da mente não é mais imperecível do que a herança biológica do corpo. Se continuarmos a desperdiçar essa herança mental biológica, como temos feito durante as últimas décadas, não demorarão muitas gerações antes de deixarmos de ser os superiores dos mongóis. Os nossos estudos etnológicos devem levar-nos, não à arrogância, mas à acção.{\color{blue}87}
 \par 
O campo de batalha, como veremos na parte 2, é o campo de provas natural da superioridade; lá, é apenas o soldado, com sua inteligência e arma, que determina sua posição no mundo. Com o tempo, porém, o conservador encontrará outro campo de provas no mercado. Embora a maioria dos primeiros conservadores fosse ambivalente em relação ao capitalismo,88 os seus sucessores passarão a acreditar que guerreiros de um tipo diferente podem provar a sua coragem no fabrico e no comércio de mercadorias. Tais homens lutam pelos recursos da terra de e para a terra, tomando para si o que querem e estabelecendo assim a sua superioridade sobre os outros. Os grandes homens do dinheiro não nascem com privilégios ou direitos; eles o apoderam para si mesmos, sem permissão ou licença. {\color{blue}89} “A liberdade é uma conquista”, escreveu William Graham Sumner. {\color{blue}90} O acto primordial de transgressão – que exige ousadia, visão e aptidão para a violência e a violação {\color{blue}91} – é o que faz do capitalista um guerreiro, conferindo-lhe não só o direito a uma grande riqueza, mas também, em última análise, ao comando. Pois é isso que o capitalista é: não um Midas de riquezas, mas um governante de homens. Um título de propriedade é uma licença para dispor, e se um homem tem o título do trabalho de outro, ele tem uma licença para dispor dele – isto é, para dispor do corpo em movimento – como achar melhor.
 \par 
Esses têm sido chamados de “capitães da indústria”. A analogia com líderes militares sugerida por este nome não é enganosa. O
 \par 
Os grandes líderes no desenvolvimento da organização industrial necessitam daqueles talentos de habilidade executiva e administrativa, poder de comando, coragem e firmeza, que antigamente eram necessários em assuntos militares e em quase nenhum outro lugar. O exército industrial também depende tanto dos seus capitães como um corpo militar depende dos seus generais. . . . Dadas as circunstâncias, tem havido uma grande procura de homens com a capacidade necessária para esta função. . . . A posse da habilidade necessária é um monopólio natural.{\color{blue}92}
 \par 
O guerreiro e o homem de negócios tornar-se-ão ícones gémeos de uma época em que, como previu Burke, a adesão às classes dominantes deve ser conquistada, muitas vezes através das lutas mais dolorosas e humilhantes. “A cada passo do meu progresso na vida (pois em cada passo fui atravessado e enfrentado oposição), e em cada pedágio que encontrei, fui obrigado a mostrar meu passaporte e, repetidas vezes, a provar meu único direito à honra de ser útil ao meu país. . . . Caso contrário, nenhuma posição, nem mesmo tolerância, para mim.”{\color{blue}93}
 \par 
Embora a guerra e o mercado sejam as agonias modernas do poder – sendo Nietzsche o teórico da primeira e Hayek da segunda – a aceitação do capitalismo pela direita nunca foi desqualificada. Até hoje, como mostro na parte 2, os conservadores continuam desconfiados da mesquinhez e superficialidade de ganhar dinheiro, do autismo político que o mercado parece induzir nas classes governantes, e da tolice e frivolidade da cultura de consumo. Para esta ala do movimento, a guerra continuará a ser sempre a única actividade onde o melhor homem pode verdadeiramente provar o seu direito de governar. É um negócio sangrento, com certeza, mas de que outra forma ser um aristocrata quando tudo o que é sólido se desmancha no ar?
 \par 
Nas últimas duas décadas, tem havido uma onda de interesse pela direita americana, resultando num conjunto de estudos - muitos dos quais escritos por historiadores mais jovens, muitos deles de esquerda - que transformaram dramaticamente a nossa compreensão do conservadorismo nos Estados Unidos.
 \par 
Estados. {\color{blue}94} Grande parte da minha leitura do pensamento conservador foi informada por esta literatura – a sua ênfase nas realidades vividas de raça, classe e género, tal como se manifestaram nas lutas partidárias do último meio século; o sincretismo entre a alta política e a cultura de massa; e a tensão criativa entre elites e activistas, empresários e intelectuais, subúrbios e sulistas, movimento e meios de comunicação. Acreditando, como T. S. Eliot, que o conservadorismo é melhor compreendido pelo “exame cuidadoso de seu comportamento ao longo de sua história e pelo exame do que suas mentes mais ponderadas e filosóficas disseram em seu nome”, {\color{blue}95} li a teoria à luz da prática (e a prática à luz da teoria). Com a ajuda desta bolsa de estudos, tenho ouvido o “pathos metafísico” do pensamento conservador – o zumbido das suas implicações, os pressupostos que invoca e as associações que evoca, a vida interior do movimento que descreve. {\color{blue}96} A presença sentida deste estudo é o que distingue, espero, a minha interpretação do pensamento conservador de outras interpretações, que tendem a ler a teoria isolada da prática ou em relação a um relato altamente estilizado dessa prática.{\color{blue}97}
 \par 
Por mais sofisticada que seja a literatura recente sobre o conservadorismo, ela sofre de três pontos fracos. A primeira é a falta de perspectiva comparativa. Os estudiosos da direita americana raramente examinam o movimento em relação ao seu homólogo europeu. Na verdade, entre muitos escritores, parece ser um artigo de fé que, como todas as coisas americanas, o conservadorismo nos Estados Unidos é excepcional. “Há um sentimento distintamente americano em Bush e nos seus defensores intelectuais”, escreve Mattson. “Um conservadorismo que se baseia em Edmund Burke, um conservadorismo de sabedoria e tradição profundamente enraizado num contexto europeu” é “o tipo de conservadorismo que nunca se consolidou na América”. {\color{blue}98} Supõe-se que o compromisso com o capitalismo laissez-faire deste lado do Atlântico diferencie o conservadorismo americano do tradicionalismo de Burke ou Disraeli; um nativo
 \par 
O pragmatismo torna o conservadorismo americano inóspito ao pessimismo e ao fanatismo de um Bonald; a democracia e o populismo tornam insustentáveis ​​os preconceitos aristocráticos de um Tocqueville. Mas esta suposição baseia-se, mostrarei, em equívocos sobre a direita europeia: nem mesmo Burke era tão tradicional como os escritores o fizeram parecer, enquanto Maistre tinha opiniões sobre a economia que eram - como tantas outras coisas nos seus escritos revanchistas. – surpreendentemente moderno. {\color{blue}99} Na verdade, existem pontos de contacto profundos – especialmente sobre questões de raça e violência – entre a direita radical na Europa e figuras como Calhoun, Teddy Roosevelt, Barry Goldwater e os neoconservadores. Na era do pós-guerra, muitos dos líderes do conservadorismo voltaram-se conscientemente para a Europa em busca de orientação e instrução, um serviço que os emigrados europeus – mais notavelmente, Hayek, Ludwig von Mises e Leo Strauss – ficaram muito felizes em prestar. {\color{blue}100} Na verdade, apesar de todo o foco na Escola de Frankfurt e em Hannah Arendt, parece que os únicos movimentos políticos na América do pós-guerra que realmente sentiram a impressão da mente europeia foram os de direita.
 \par 
A segunda fraqueza é a falta de perspectiva histórica. Não importa até que ponto os escritores e académicos recuem as origens do conservadorismo contemporâneo (o movimento mais recente defende um longo movimento conservador que liga o Tea Party à década de 1920),101 há uma noção na literatura recente de que o conservadorismo contemporâneo é fundamentalmente diferente das iterações anteriores. A certa altura, prossegue o argumento, o conservadorismo americano rompeu com os seus antecessores – tornou-se populista, ideológico, e assim por diante – e foi esta ruptura, dependendo da perspectiva de cada um, que o salvou ou condenou. {\color{blue}102} Mas este argumento ignora as continuidades entre figuras como Adams e Calhoun e vozes mais recentes da direita americana. Longe de ser uma inovação das últimas décadas, o populismo do Tea Party e o futurismo de um Reagan ou de um Gingrich podem ser encontrados nas primeiras vozes do conservadorismo, em ambos os lados
 \par 
Do Atlântico. Da mesma forma, o aventureirismo, o racismo e a propensão ao pensamento ideológico.
 \par 
A terceira fraqueza deriva da segunda. Quanto mais os analistas remontam às origens do conservadorismo contemporâneo, menos inclinados ficam a acreditar que se trata de uma política de reacção ou de contra-ataque. Se os compromissos do conservador contemporâneo podem ser situados nos escritos de Albert Jay Nock ou John Adams, argumentam estes estudiosos, o conservadorismo deve reflectir ideias e compromissos mais transcendentes do que a mera oposição à Grande Sociedade poderia sugerir. {\color{blue}103} Mas o reconhecimento da longa história da direita não precisa de minar a afirmação de que o conservadorismo contemporâneo é uma política de reação. Em vez disso, a visão a longo prazo deverá ajudar-nos a compreender melhor a natureza e a dinâmica, bem como as idiossincrasias e contingências, dessa reação negativa. Na verdade, só colocando a direita contemporânea no contexto dos seus antecessores poderemos compreender a sua cidade e particularidade específicas.
 \par 
Contra essas três suposições, que se concentram na diferença e na distinção, trato a direita como uma unidade, como um corpo coerente de teoria e prática que transcende as divisões tão frequentemente enfatizadas por acadêmicos e especialistas. {\color{blue}104} Uso as palavras conservador, reacionário e contrarrevolucionário de forma intercambiável: nem todos os contrarrevolucionários são conservadores — Walt Rostow imediatamente me vem à mente — mas todos os conservadores são, de uma forma ou de outra, contrarrevolucionários. Eu sento filósofos, estadistas, senhores de escravos, rabiscadores, católicos, fascistas, evangélicos, empresários, racistas e picaretas na mesma mesa: Hobbes ao lado de Hayek, Burke em frente a Palin, Nietzsche entre Ayn Rand e Antonin Scalia, com Adams, Calhoun, Oakeshott, Ronald Reagan, Tocqueville, Theodore Roosevelt, Margaret Thatcher, Ernst Jünger, Carl Schmitt, Winston Churchill, Phyllis Schlafly, Richard Nixon, Irving Kristol, Francis Fukuyama e George W. Bush intercalados por toda parte.
 \par 
Isto não quer dizer que não haja mudança no conservadorismo ao longo do tempo ou do espaço. Se o conservadorismo é uma reação específica a um movimento específico de emancipação, é lógico que cada reação carregará os traços do movimento ao qual se opõe. Como defendo no capítulo 1, não só a direita reagiu contra a esquerda, mas, ao conduzir a sua reacção, também recorreu consistentemente à esquerda. À medida que os movimentos da esquerda mudam – da Revolução Francesa, da abolição, à direita do voto, da direita à organização, da Revolução Bolchevique às lutas pela liberdade dos negros e pela libertação das mulheres – o mesmo acontece com as reacções da direita.
 \par 
Para além destas mudanças contingentes, podemos também traçar uma mudança estrutural mais longa na imaginação da direita: nomeadamente, a aceitação gradual da entrada das massas na cena política. De Hobbes aos proprietários de escravos e aos neoconservadores, a direita tornou-se cada vez mais consciente de que qualquer defesa bem sucedida do antigo regime deve incorporar as ordens inferiores em alguma capacidade que não seja a de subordinados ou de fãs fascinados. As massas devem ser capazes de se localizar simbolicamente na classe dominante ou ter oportunidades reais de se tornarem eles próprios falsos aristocratas na família, na fábrica e no campo. O primeiro caminho conduz a um populismo invertido, em que os mais baixos dos mais baixos se vêem projectados no mais alto dos mais altos; o último contribui para um feudalismo democrático, no qual o marido ou supervisor desempenha o papel de senhor. O primeiro caminho foi iniciado por Hobbes, Maistre e vários profetas do racismo e do nacionalismo, o último por proprietários de escravos do Sul, imperialistas europeus e apologistas da Era Dourada. (E os apologistas da Neo-Era Dourada: “Não existe uma elite única na América”, escreve David Brooks. “Todos podem ser um aristocrata dentro do seu próprio Olimpo.” {\color{blue}105}) Ocasionalmente, como nos escritos de Werner Sombart, os dois caminhos enganam. -verge: as pessoas comuns conseguem se ver na classe dominante em virtude de pertencerem a uma grande nação entre as nações, e também conseguem governar seres inferiores através do exercício do domínio imperial.
 \par 
Também nós, alemães, deveríamos percorrer o mundo do nosso tempo da mesma forma, com a cabeça erguida e orgulhosa, no sentimento seguro de sermos povo de Deus. Assim como o pássaro alemão, a águia, voa alto sobre todos os animais desta terra, o alemão deve sentir-se acima de todos os outros povos que o rodeiam e que ele vê em profundidades ilimitadas abaixo dele.
 \par 
Mas a aristocracia tem as suas obrigações, e isto também é verdade aqui. A ideia de que somos pessoas escolhidas impõe-nos deveres formidáveis ​​– e apenas deveres. Devemos acima de tudo manter-nos como uma nação forte no mundo.{\color{blue}106}
 \par 
Embora estas diferenças históricas à direita sejam reais, existe uma afinidade subjacente que une estas diferenças. Não se pode perceber esta afinidade concentrando-nos em divergências de políticas ou declarações contingentes de práticas (direitos dos estados, federalismo, e assim por diante); é preciso olhar para os argumentos subjacentes, as expressões idiomáticas e as metáforas, as visões profundas e o pathos metafísico evocados em cada desacordo e afirmação. Alguns conservadores criticam o mercado livre, outros defendem-no; alguns se opõem ao Estado, outros o abraçam; alguns acreditam em Deus, outros são ateus. Alguns são vocalistas, outros nacionalistas e ainda outros internacionalistas. Alguns, como Burke, são os três ao mesmo tempo. Mas estas são improvisações históricas – táticas e substantivas – sobre um tema. Somente justapondo essas vozes – através do tempo e do espaço – podemos decifrar o tema em meio à improvisação.
 \par 
Para muitos, a noção de unidade à direita será a afirmação mais controversa deste livro. Embora continuemos a utilizar o termo “conservador” no nosso discurso quotidiano (na verdade, a discussão política seria inconcebível sem ele); embora o conservadorismo, tanto na Europa como nos Estados Unidos, tenha conseguido, durante mais de um século, atrair e manter unida uma coligação de tradicionalistas, guerreiros e capitalistas; mesmo que a oposição
 \par 
Entre esquerda e direita provou ser uma “distinção política” duradoura da era moderna (apesar das tentativas, a cada geração ou mais, de negar ou superar esta oposição através de uma “terceira via”) {\color{blue}107} – muitos continuam a acreditar nas diferenças existentes. a direita é tão grande que seria impossível dizer alguma coisa sobre a direita. {\color{blue}108} Mas se é impossível dizer alguma coisa sobre o direito – definir, descrever, explicar, analisar e interpretar o direito como uma formação distintiva – como podemos dizer que ele existe?
 \par 
Na esperança de evitar esse ceticismo radical, que tornaria ininteligível muito do que se passa na nossa política, alguns estudiosos recuaram para uma posição nominalista: conservadores são pessoas que se autodenominam conservadoras ou, mais elaboradamente, conservadores são pessoas que se autodenominam conservadoras, chamam-se conservadoras. . {\color{blue}109} Isto apenas levanta a questão: o que é que estas pessoas que se autodenominam conservadoras – ou que outras pessoas que se autodenominam conservadoras chamam de conservadoras – querem dizer com “conservador”? Por que optam por essa autodescrição em oposição a liberal, socialista ou porco-da-terra? A menos que estas pessoas pensem que estão a referir-se a identidades idiossincráticas – caso em que voltamos à posição cética – precisamos de compreender o que o termo significa, independentemente da sua utilização. De que outra forma podemos compreender por que razão indivíduos de diferentes épocas e lugares, adoptando diferentes posições sobre diferentes questões, se autodenominariam conservadores e aos seus espíritos afins? Embora nem todos os leitores precisem de aceitar a minha afirmação sobre o que une o direito, parece ser uma condição necessária para uma discussão inteligente que concordemos que existe algo chamado direito e que tem algum conjunto de características comuns que o tornam certo.
 \par 
Os onze capítulos deste livro foram extraídos de uma década de escritos sobre a direita. Alguns capítulos apareceram originalmente como longos ensaios de revisão para periódicos como The Nation e London
 \par 
Resenha de Livros; outros são artigos de pesquisa acadêmica, artigos relatados ou ensaios independentes. Fiz algumas alterações nessas peças para dar conta de novos desenvolvimentos ou mudanças em meus pontos de vista. Ocasionalmente, eliminei seções inteiras porque elas não pareciam mais relevantes. Mas, no geral, tentei deixar as peças intactas, na esperança de que as suas abordagens variadas capturassem esta noção de direito como um conjunto de improvisações históricas sobre um tema contínuo. O livro está dividido em duas partes. A Parte {\color{blue}1} abre com uma declaração geral sobre o impulso contrarrevolucionário da política conservadora, desde a Revolução Francesa até hoje. Este capítulo centra-se menos nos objectivos e intenções da contrarrevolução e mais nos seus movimentos e manobras: como rompe com o próprio regime que defende e olha para a esquerda nos seus esforços para reconstruir a direita. Em seguida, passo cronologicamente, de um exame de Thomas Hobbes e da Guerra Civil Inglesa para uma análise conclusiva do Juiz Scalia e sua jurisprudência originalista. Ao longo do caminho, discuto Rand, Goldwater, a Nova Direita e os conservadores após a Guerra Fria. A Parte {\color{blue}2} analisa o tema tenso da violência no conservadorismo. Embora eu comece com uma breve retrospectiva da Guerra Fria na América Latina e conclua com uma reflexão mais geral sobre como a direita abordou a violência desde Burke, a maior parte da discussão nestes capítulos é extraída da última década: {\color{blue}11} de setembro, a guerra ao terror, a guerra no Iraque. Estes acontecimentos, e a vertigem que inspiraram entre os conservadores, mais do que qualquer coisa, levaram-me a pensar e a escrever sobre a direita. Como percebi, e como argumenta o capítulo 11, a paixão pela violência na direita de hoje não é uma aberração; é constitutivo da própria tradição.