\chapter{Prefácio à Sexta Edição}\label{Prefácio à Sexta Edição}
 \par 
O Capital de Marx foi originalmente escrito no início da década de 1970 e foi em grande parte um produto de sua época. Depois, na Grã-Bretanha e noutros lugares, o interesse pela economia política de Marx foi despertado após vários anos de intensa repressão sob o pretexto de culpar os trabalhadores e os movimentos de esquerda pelo fim do boom do pós-guerra. Este interesse cresceu e foi alimentado pelo declínio evidente da economia capitalista mundial e pela rejeição das explicações convencionais para o crescente mal-estar económico associado à estagflação. Muita coisa mudou desde então, e as sucessivas edições deste livro refletiram, à sua maneira, as mudanças no destino tanto da economia global como da economia política.
 \par 
A quarta edição de O Capital de Marx relançou este pequeno livro com a Pluto Press para novos tempos e um novo público em 2004, tendo a terceira edição surgido em 1989. A ascensão do neoliberalismo nas décadas de 1980 e 1990 remodelou o mundo capitalista, ampliou o domínio do capital global para a maioria dos cantos do planeta e remodelou o sistema político para apoiá-lo. As expectativas de mudança económica, política e social foram diminuindo ao longo do tempo, no que foi denominado o esvaziamento do Estado face ao declínio da força e da organização dos movimentos progressistas. À medida que as grandes mobilizações das décadas de 1960 e 1970 se distanciavam, uma nova geração cresceu com esperanças, exigências e expectativas muito reduzidas. Pela primeira vez desde meados do século XIX, parecia não haver alternativas ao capitalismo à vista, e as restantes excepções - invariavelmente marginais - mantinham-se precária e pouco atractivas nas fendas do admirável novo mundo “globalizado”. A quarta edição ofereceu uma pequena contribuição para as respostas emergentes a estes enormes desafios e foi bem recebida por um vasto público em vários países.
 \par 
A publicação da quinta e agora desta sexta edição antecipa, e esperamos que à sua maneira contribua para, um renascimento da economia política em geral e da economia política marxista em particular. Esse optimismo baseia-se numa série de factores.
 \par 
Primeiro, enquanto a economia tradicional reforçou seu controle intolerante sobre a disciplina, descartando a heterodoxia por não passar nos testes de rigor matemático e estatístico, há sinais crescentes de insatisfação com a ortodoxia, e há uma busca crescente por alternativas entre aqueles que estudam economia e outras ciências sociais, principalmente com as demandas por heterodoxia, pluralismo e alternativas no ensino de economia.
 \par 
Segundo, seguindo a predominância do pós-modernismo e, especialmente, do neoliberalismo na definição de agendas intelectuais nas ciências sociais nas últimas duas décadas, há agora uma reação contra os extremos de seus piores excessos na teoria e na prática. O pensamento crítico se voltou para a compreensão da natureza do capitalismo contemporâneo, como refletido mais notavelmente na ascensão de conceitos como neoliberalismo, financeirização, globalização e capital social. Inevitavelmente, o resultado é levantar a questão da economia fora da disciplina da economia em si, e buscar orientação da economia política.
 \par 
Terceiro, os desenvolvimentos materiais também promoveram a defesa da economia política. Estas incluem a crescente constatação de que a degradação ambiental, sobretudo através do aquecimento global, está intimamente relacionada com o capitalismo; as consequências do colapso da União Soviética e o reconhecimento de que o capitalismo não forneceu uma alternativa progressista, mesmo nos seus próprios termos restritos; e a erupção de guerras e ocupações imperiais, mesmo que travadas sob o nome de antiterrorismo ou de direitos humanos.
 \par 
Em quarto lugar, o longo período de relativa estagnação que se seguiu ao colapso do boom do pós-guerra e à ascensão do pós-modernismo e do neoliberalismo tiveram o efeito paradoxal de permitir que a economia capitalista fosse percebida como estando envolvida nos negócios habituais com um mínimo de sucesso. , mesmo que de forma lenta. A erupção das crises financeiras ao longo da última década, e de forma mais dramática a crise global que começou em meados de 2007, destruiu esta perspectiva. Trouxe à tona o papel fundamental desempenhado pelas finanças no capitalismo contemporâneo. As relações sistémicas entre as finanças, a indústria e o resto da economia em geral deveriam ocupar um lugar de destaque no tema da economia política. Tendo o capitalismo falhado de forma tão comprovada nos seus próprios termos, mesmo sob condições que lhe são indiscutivelmente as mais favoráveis, a defesa do socialismo precisa de ser defendida como nunca antes. E baseia-se numa análise marxista tanto pela sua crítica ao capitalismo como pela luz que lança sobre o potencial de alternativas.
 \par 
Cada uma dessas questões é reavaliada em maior ou menor grau nesta nova edição. Mas o principal objectivo do livro continua a ser fornecer uma exposição tão simples e concisa da economia política de Marx quanto a complexidade das suas ideias o permita. Como o livro é limitado a ser curto, os argumentos são condensados, mas permanecem simples e não complicados; no entanto, parte do material exigirá uma leitura cuidadosa, especialmente os capítulos posteriores. Não é de surpreender que, ao longo das suas várias edições, o texto tenha aumentado de tamanho, mais do que duplicando o seu comprimento original de {\color{blue}25}.{\color{blue}000} palavras, à medida que novos tópicos foram adicionados, extraídos tanto da própria economia política de Marx como da sua relevância contemporânea. Além disso, ao longo do tempo, adições específicas incluíram destaque capítulo por capítulo de controvérsias, questões para debate e sugestões para leitura adicional, que oferecerão orientação aos interessados ​​em textos mais acadêmicos. Lamentamos que isto tenha feito com que as edições sucessivas perdessem alguma da simplicidade das anteriores (embora, para facilitar a leitura, as notas de rodapé continuem a ser omitidas). Estas dificuldades (esperançosamente menores) são talvez agravadas pelas referências ocasionais à forma como a economia política de Marx difere da economia ortodoxa, colocando alguma pressão sobre os não-economistas. Mas tais complexidades podem ser ignoradas quando necessário e, caso contrário, oferecem insights compensadores.
 \par 
Esta sexta edição cuidadosamente revisada chega em um momento particularmente desafiador. O capitalismo neoliberal está no meio de uma crise sem precedentes, que revelou não só as limitações das finanças “liberalizadas”, mas, mais significativamente, lançou o projecto neoliberal global para a defensiva pela primeira vez, embora pareça notavelmente resiliente. No entanto, é agora possível que a corrente dominante questione abertamente a coerência e a sustentabilidade do neoliberalismo, e até mesmo a conveniência do próprio capitalismo. Estes debates emergentes, e o crescimento simultâneo, embora dolorosamente lento, de movimentos e organizações sociais radicais, têm sido apoiados pela percepção crescente de que o capitalismo desestabilizou fundamentalmente o ambiente do planeta e que representa uma ameaça imediata à sobrevivência de inúmeras espécies, incluindo a nossa própria. .
 \par 
O Capital de Marx não é um livro sobre o ambiente nem sobre o neoliberalismo, embora inclua uma breve secção sobre o primeiro e um capítulo atualizado sobre a crise atual. Os seus objectivos são mais restritos e, ao mesmo tempo, mais abstractos e ambiciosos: analisa e explica os elementos-chave da crítica mais sustentada, consistente e intransigente do capitalismo como sistema, que foi originalmente desenvolvida por Karl Marx. À medida que o capitalismo luta para conter a sua crise mais recente, os escritos de Marx aumentaram em imediatismo e relevância, e dispararam em popularidade. Eles agora ocupam posições de destaque em diversas listas de best-sellers, e edições rivais podem ser encontradas até mesmo nas principais livrarias, embora as obras de Marx também estejam amplamente disponíveis na web e possam ser baixadas gratuitamente.
 \par 
Esperamos que você faça uso deles. O Capital de Marx nunca procurou substituir o verdadeiro; em vez disso, pretende facilitar a sua leitura dos escritos económicos de Marx, fornecendo uma visão geral estruturada dos seus principais temas e conclusões. Esperamos que este livro apoie a sua própria tentativa de chegar a um acordo com o capitalismo, os seus pontos fortes e fracos, e informe as suas lutas contra ele.
 \par 
Gostaríamos de agradecer e encorajar aqueles que continuaram a estudar e ensinar economia marxista seriamente, durante um período em que foi extraordinariamente difícil fazê-lo.
 \par 
\section{Uma nota sobre leitura adicional}
 \par 
Cada capítulo deste livro inclui uma lista de “Questões e Leituras Adicionais”, que descreve algumas implicações do material examinado naquele capítulo e sugere um conjunto pequeno e cuidadosamente selecionado de leituras para ajudá-lo a se aprofundar. É claro que há muito mais por aí, e gostaríamos de receber sugestões de leituras a serem incluídas em futuras edições deste livro. Por favor, envie-nos um e-mail para nos informar se você encontrar algo especialmente útil, ou para discutir tópicos e problemas na teoria do valor, ou para sugerir alterações ou conteúdo adicional que possamos incluir em futuras edições deste livro. Nós gostaríamos de ouvir de você.
 \par 
Para começar, algumas sugestões gerais. As Obras Completas de Karl Marx e Friedrich Engels ainda são publicadas em alemão e estão sendo gradualmente traduzidas para o inglês e outras línguas. As obras mais significativas, incluindo O Capital, estão disponíveis gratuitamente no Marxists Internet Archive (www.marxists. Org) e em vários outros websites.
 \par 
Muitos comentários excelentes sobre o trabalho de Marx, e um bom número de visões gerais dos seus escritos económicos, estão disponíveis em fontes anglo-saxónicas, nas quais nos concentramos abaixo. Por exemplo, Chris Arthur preparou uma edição abreviada do Volume {\color{blue}1} de O Capital (Arthur 1992), sem notas de rodapé e com uma introdução explicativa, e Duncan Foley e David Harvey escreveram excelentes introduções à obra de Marx (Foley 1986; Harvey 1999, 2009, 2010). Harvey também conduz uma discussão online sobre Capital (http://davidharvey.org/readingcapital/). Alex Callinicos (2014) e Joseph Choonara (2009) publicaram visões gerais muito boas da teoria do valor de Marx, que complementam (e complementam) este livro. Um relato clássico das fontes do marxismo é fornecido por Vladimir Lenin (1913). Para uma visão mais avançada da teoria do valor de Marx, ver Dimitris Milonakis e Ben Fine (2009, especialmente o capítulo {\color{blue}3}) e Alfredo Saad-Filho (2002). Um exercício de avaliação igualmente avançado em todo o espectro da análise económica marxista é encontrado em Fine e Saad-Filho (2012). A pesquisa em economia política marxista é promovida pelo IIPPE (www.iippe.org) e apoiada por revistas como Capital & Class, Historical Materialism, Monthly Review, Review of Radical Political Economics e Science & Society. Finalmente, para economia, notícias e análises heterodoxas (incluindo marxistas), consulte www.heterodoxnews.com.
 \par 
\section{Uma nota sobre leitura adicional}