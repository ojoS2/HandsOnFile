\chapter{Seção sobre Sistemas Sociais e Saúde}\label{Seção sobre Sistemas Sociais e Saúde}
 \par 
O SOCIALISMO FALOU? UMA ANÁLISE DOS INDICADORES DE SAÚDE NO SOCIALISMO
 \par 
\section{Vicente Navarro}
 \par 
Este artigo analisa a suposição amplamente difundida na academia e na grande imprensa de que o capitalismo provou ser superior ao socialismo na resposta às necessidades humanas. O autor analisa as condições de saúde das populações mundiais, continente por continente, e mostra que, contrariamente à ideologia dominante, o socialismo e as forças socialistas têm sido, na sua maior parte, mais capazes de melhorar as condições de saúde do que o capitalismo e as forças capitalistas. No mundo subdesenvolvido, as forças e os regimes socialistas têm, mais frequentemente, melhorado os indicadores sociais e de saúde melhor do que as forças e os regimes capitalistas, e no mundo desenvolvido, os países com fortes forças socialistas têm sido mais capazes de melhorar as condições de saúde do que esses países. carentes ou com forças socialistas fracas. A experiência socialista também incluiu, evidentemente, desenvolvimentos negativos que negaram componentes importantes do projecto socialista. Ainda assim, as evidências apresentadas neste artigo mostram que a experiência histórica do socialismo não foi de fracasso. Pelo contrário: tem sido, na maior parte, mais bem sucedido do que o capitalismo na melhoria das condições de saúde das populações mundiais.
 \par 
Uma importante posição intelectual reproduzida na imprensa académica e na grande imprensa de hoje é que o conflito histórico entre duas abordagens ao desenvolvimento social humano foi resolvido em favor do capitalismo: o capitalismo provou ser superior ao socialismo na resposta às necessidades humanas. Esta posição, articulada pela primeira vez pelo funcionário do Departamento de Estado dos EUA, Francis Fukuyama (l), ganhou ampla aceitação nos centros intelectuais do mundo ocidental. A posição não é apenas descritiva, mas também não-nativa; o socialismo deve ser evitado e o capitalismo deve ser promovido para resolver as realidades dramáticas do nosso mundo, onde uma criança morre de fome a cada dois segundos e {\color{blue}15} milhões de crianças morrem de desnutrição todos os anos (2).
 \par 
Como afirma o Papa João Paulo I1 na sua encíclica Centesimus Annus (citada em {\color{blue}3}):
 \par 
Deveria o capitalismo ser o objectivo dos países que actualmente estão a envidar esforços para reconstruir a sua economia e a sua sociedade? Será este o modelo que deveria ser proposto aos países do Terceiro Mundo que procuram o caminho do verdadeiro progresso económico e civil? . . . Se por capitalismo se entende um sistema económico que reconhece o papel fundamental e positivo das empresas, do mercado, da propriedade privada e do
 \par 
Esta é uma versão modificada de um artigo publicado na Science and Society.
 \par 
International Journal of Health Services, Volume 22, Número 4, Páginas 583401, 1992 {\color{blue}0} 1992, Baywood Publishing Co., Inc.
 \par 
583 doi: {\color{blue}10}.2190/B2TP-3R5M-Q7UP-DUA2 http://baywood.com
 \par 
TTTTTT resultante da responsabilidade pelos meios de produção, bem como pela criatividade humana no sector económico, então a resposta é certamente afirmativa.
 \par 
Nos círculos socialistas do mundo capitalista desenvolvido ocidental, duas posições defensivas tornaram-se dominantes. Uma é a negação do carácter socialista das sociedades que se afirmam socialistas (4). O socialismo não falhou, nunca sequer existiu. O suposto fracasso das sociedades “socialistas” em satisfazer as necessidades humanas não tem qualquer influência na realização do projecto socialista, uma vez que esse projecto ainda não foi testado. É importante notar que a esmagadora maioria das contribuições teóricas que sustentam esta posição foram feitas em países capitalistas desenvolvidos.
 \par 
A outra posição defensiva é questionar a viabilidade de comparar sistemas como um todo. Como indicado por Adam Przeworski, “é impossível dizer se o modelo socialista ou o modelo capitalista teve mais sucesso na prática” (5). Não é claro, contudo, por que razão a questão extremamente importante de saber se o capitalismo é superior ao socialismo na resposta às necessidades humanas não pode ser objecto de investigação científica. Dado que a principal preocupação de Przeworski, expressa no título do seu mais recente artigo “Poderíamos alimentar todos?”, é se o capitalismo ou o socialismo são o melhor caminho para resolver os problemas da fome e da desnutrição, um investigador poderia comparar a evolução dos níveis nutricionais de populações que vivem actualmente sob dois regimes diferentes, mas que viveram sob condições capitalistas semelhantes no início do período histórico em estudo. A dificuldade de padronizar variáveis ​​pode enfraquecer a validade da comparação, mas raramente a ponto de tornar a comparação inútil. É também provável que a comparação tenha um viés inevitável a favor do capitalismo, uma vez que a experiência socialista evoca sempre enorme hostilidade, bloqueio económico e até intervenção militar. Tal comparação não seria entre capitalismo versus socialismo em circunstâncias normais, mas sim entre capitalismo em circunstâncias normais, articulado com um sistema mundial em que as relações capitalistas são dominantes, versus socialismo sob circunstâncias mais anormais. Ainda assim, apesar deste preconceito intrínseco, acredito que tais comparações têm validade e podem ser apresentadas para mostrar a superioridade de um sistema sobre outro na resposta às principais necessidades humanas, que incluem a prevenção da fome, da desnutrição, das doenças e da morte prematura. . Na verdade, nos países capitalistas desenvolvidos, onde ocorre a maior parte da produção teórica ocidental, é geralmente esquecido que a maioria dos seres humanos no nosso período histórico não tem direitos socioeconómicos básicos, como alimentação, água potável, esgoto tratado e a capacidade de ler. . A ausência destes direitos básicos limita todos os outros direitos humanos, tais como os direitos políticos civis, incluindo os direitos de organização e a liberdade de imprensa. O Presidente Franklin Roosevelt colocou-o muito bem na sua mensagem ao Congresso em {\color{blue}11} de Janeiro de 1944: “homens [e mulheres] necessitados não são homens [e mulheres] livres” (6).
 \par 
Ao contrário de Przeworski, acredito que a superioridade de um sistema sobre outro pode de facto ser demonstrada. E uma forma de o fazer é mostrar a evolução dos indicadores de saúde (tais como mortalidade infantil, esperança de vida, níveis de nutrição e baixo peso à nascença, sempre que tais dados estejam disponíveis) em países comparáveis ​​que seguiram diferentes caminhos de desenvolvimento, capitalistas versus socialista. Antes de nos concentrarmos nas informações empíricas, no entanto, vários pontos precisam ser levantados.
 \par 
Primeiro, contrariamente à crença predominante, o nível de saúde de uma população não é principalmente o resultado de intervenções médicas. Se o país A tiver melhores indicadores de saúde do que o país
 \par 
\section{584 1 Navarro}
 \par 
B, não é porque o país A tenha despesas médicas maiores. Não há correlação entre o nível de despesas médicas e o nível de saúde. O nível de saúde também não se correlaciona com o nível de consumo médico. (Existe uma extensa bibliografia sobre o impacto limitado dos cuidados médicos no nível de mortalidade e morbilidade das populações; ver {\color{blue}7}.) Baltimore, por exemplo, tem uma taxa de utilização de serviços de cuidados pré-natais acima da média para todos os sectores da população, incluindo os pobres, mas devido à sua pobreza generalizada tem uma das taxas de mortalidade infantil mais elevadas dos Estados Unidos (8). Esta observação não pretende descartar a importância dos cuidados médicos na melhoria da saúde da população. Pelo contrário, pretende indicar que a saúde da população é o resultado de todo um conjunto de intervenções sociais, económicas e políticas, entre as quais os cuidados médicos desempenham um papel menor. O país A com melhores indicadores de saúde do que o país B tem, em geral, melhores condições sociais e económicas para a maioria dos seus cidadãos do que o país B. Assim, os indicadores de saúde são bons indicadores de desenvolvimento social e económico.
 \par 
Em segundo lugar, devemos esclarecer o que se entende por capitalismo e socialismo. O capitalismo é a produção de bens e serviços para o lucro daqueles que possuem os meios pelos quais são produzidos. Nas sociedades capitalistas, os principais meios de produção são privados. O socialismo é um sistema de produção e distribuição em que os meios de produção são propriedade pública, com o Estado desempenhando o papel fundamental na produção. Este sistema é o resultado de um processo revolucionário autónomo em que grandes sectores da classe trabalhadora e/ou do campesinato foram as principais forças por trás do estabelecimento do Estado. Excluem-se desta definição os países onde o socialismo foi imposto de fora, como a Europa Oriental ou o Afeganistão, ou por um golpe militar, como a Etiópia. Estes casos têm sido os mais frequentemente utilizados para desacreditar todo o projecto socialista. Na minha opinião eles não são socialistas.
 \par 
Terceiro, a superioridade de um sistema sobre outro pode ser demonstrada não apenas olhando para países comparáveis ​​com regimes diferentes, mas também analisando países capitalistas comparáveis ​​com diferentes correlações de forças entre elementos pró-capitalistas e pró-socialistas. Por outras palavras, a superioridade, digamos, do socialismo sobre o capitalismo pode ser demonstrada comparando dois países capitalistas semelhantes, um com fortes forças socialistas (forças que reivindicam um compromisso com o socialismo) e o outro sem tais forças. Se o segundo país tiver maiores necessidades humanas não resolvidas do que o primeiro, acredito que a alegação de superioridade socialista é justificada. Por outras palavras, mesmo na ausência de formações socialistas no mundo capitalista desenvolvido, a superioridade do socialismo pode ser demonstrada.
 \par 
\section{O socialismo falhou? / 585}
 \par 
\[UMA ANÁLISE CONTINENTE POR CONTINENTE DA EXPERIÊNCIA SOCIALISTA\]
 \par 
Começando pelo nosso hemisfério, o desempenho socialista de Cuba pode ser medido em comparação com o desempenho de países latino-americanos comparáveis ​​com regimes capitalistas. A maioria destes países tinha uma distribuição demográfica semelhante e níveis de desenvolvimento económico e social semelhantes ou até melhores do que Cuba em 1958, quando ocorreu a revolução cubana. Desde então, os indicadores de saúde melhoraram mais rapidamente em Cuba do que no resto da América Latina. Em 1955 a esperança de vida em Cuba era de 59,5 anos,
 \par 
TTTTTT é mais curto que o Paraguai (62 anos), Argentina (62 anos) e Uruguai (66 anos), os países com maior expectativa de vida na América Latina. Em 1985, a esperança de vida em Cuba era de {\color{blue}75} anos, superior à de todos estes países e, de facto, a mais elevada da América Latina (9, Tabela 16, p. {\color{blue}26}). É também notável que em Cuba ocorreram aumentos na esperança de vida. Em níveis iniciais relativamente elevados, onde os aumentos são muitas vezes mais difíceis de alcançar, e as melhorias na esperança de vida foram maiores do que em países como a Argentina e o Uruguai, que tinham uma maior esperança de vida para começar e maiores rendimentos per capita do que Cuba. A melhora nas taxas de mortalidade ocorreu em todas as faixas etárias. Estas taxas são agora mais baixas em Cuba do que em qualquer outro país latino-americano (9, Tabela 17, p. {\color{blue}29}). Os Estados Unidos também oferecem uma comparação interessante. Entre 1950-55 e 1985-90, a esperança de vida nos Estados Unidos aumentou de {\color{blue}69} para 75,4 anos. Durante o mesmo período, a esperança de vida em Cuba aumentou de 59,3 para 75,2 anos (9, Tabela 16, p. {\color{blue}26}).
 \par 
Da mesma forma, em 1955, Cuba tinha uma taxa de mortalidade infantil de {\color{blue}81} mortes por {\color{blue}1}.{\color{blue}000} nados vivos, superior à de vários outros países latino-americanos, incluindo o Paraguai, o Uruguai e a Argentina (os países com a mortalidade infantil mais baixa). Em 1985, Cuba tinha a menor taxa de mortalidade infantil (13/1.{\color{blue}000} nascidos vivos) da América Latina (9, Tabela 1, p. {\color{blue}53}). A taxa de mortalidade de menores de cinco anos (por {\color{blue}1}.{\color{blue}000} nascidos vivos) caiu de {\color{blue}95} para {\color{blue}19} em Cuba entre 1960 e 1985. Mesmo os países com taxas de mortalidade de menores de cinco anos mais baixas em 1960 e rendimentos per capita mais elevados (Trindade e Tobago, Argentina, Uruguai ) foram superados por Cuba (10, {\color{blue}11}). Cuba também tem o nível mais baixo de desnutrição da América Latina
 \par 
\section{América latina}
 \par 
\[586 I Navarro\]
 \par 
\section{tabela 1}
 \par 
\section{Indicadores sociais e de saúde na América Latina, meados da década de 1980"}
 \par 
\section{Expectativa de vida ao nascer, anos}
 \par 
\section{Mortalidade infantil por 1,0OOb}
 \par 
Argentina Brasil Chile Costa Rica República Dominicana El Salvador Haiti Jamaica México Nicarágua Peru Uruguai Venezuela
 \par 
70 {\color{blue}63} 71 {\color{blue}73} 63 {\color{blue}64} 54 {\color{blue}70} 59 {\color{blue}58} 59 {\color{blue}70} {\color{blue}69}
 \par 
35 {\color{blue}71} 24 {\color{blue}20} 63 {\color{blue}70} {\color{blue}107} {\color{blue}28} 82 {\color{blue}84} 82 {\color{blue}38} {\color{blue}39}
 \par 
6,1 25,5 8,9 6,4 27,0 38,0 77,0 12,0 17,4 12,9 17,4 6,1' 15,3
 \par 
6,1 5,3 17,2 6,6 20,3 30,0 40,0 25,9 11,8 16,3 11,8 13,1 14,3
 \par 
74
 \par 
13
 \par 
3.{\color{blue}9}
 \par 
3.4e
 \par 
«Fonte: Multinacional Moniror, Abril de 1989. Estes são os últimos números disponíveis. bA mortalidade infantil é definida como a morte antes de {\color{blue}1} ano de idade por {\color{blue}1}.{\color{blue}000} nascidos, arredondada para o número inteiro mais próximo. 'Taxa de analfabetismo para maiores de {\color{blue}15} anos. Figura de 1975. 'Desemprego urbano e rural (Censos 1981).
 \par 
\section{Taxa de analfabetismo, %'}
 \par 
América para todas as faixas etárias (9, Tabela 88, p. 91; Tabela 91, p. {\color{blue}193}), embora os níveis de desnutrição, especialmente nas áreas rurais, fossem elevados (12) e comparáveis ​​aos da maioria dos países latino-americanos durante a década de 1950. Cuba é o país latino-americano com a menor percentagem de bebês com baixo peso ao nascer e de crianças menores de {\color{blue}5} anos que sofrem de desnutrição leve a moderada a grave, e com a melhor oferta calórica diária per capita como percentagem de necessidades (13, Tabela {\color{blue}2} , pág. {\color{blue}97}). Além disso, em 1956 Cuba era um dos países latino-americanos com as piores condições ambientais. Apenas {\color{blue}35} por cento da população vivia em casas ligadas a sistemas de abastecimento de água (Censo de Cuba de 1953, citado em {\color{blue}14}) (comparado, por exemplo, com {\color{blue}63} por cento na República Dominicana, {\color{blue}80} por cento em Honduras e {\color{blue}44} por cento na Argentina) ( {\color{blue}15}). Além disso, apenas {\color{blue}42} por cento da população vivia em casas ligadas a sistemas de esgotos (citado em {\color{blue}14}). Em 1980, Cuba tinha um dos melhores resultados em serviços ambientais. Setenta e quatro por cento da população vivia em habitações ligadas a sistemas de abastecimento de água (apenas Trinidad tinha uma percentagem maior na América Latina: {\color{blue}91} por cento), e {\color{blue}91} por cento da população tinha acesso a autoclismos, uma das percentagens mais elevadas da América Latina. (16). A taxa de mortalidade ajustada à idade por enterite e outras doenças diarreicas era de 2,8 por {\color{blue}100}.000 habitantes em 1988, uma das duas mais baixas da América Latina (9, Tabela 111, p. {\color{blue}370}). Cuba também tem a maior taxa de alfabetização da América Latina (96% da população adulta). Na década de 1950, a taxa de alfabetização era comparável à do resto dos países das Caraíbas, variando entre {\color{blue}30} e {\color{blue}40} por cento da população adulta (13, p. {\color{blue}101}). A Tabela {\color{blue}1} resume alguns indicadores sociais e de saúde atuais para Cuba e outros países latino-americanos.
 \par 
Tendo em conta esta informação, poder-se-ia concluir que a declaração do Papa João Paulo II no CenteshusAnnus definindo o capitalismo como o melhor sistema para responder às necessidades humanas no Terceiro Mundo, pelo menos para a América Latina, pode não ser totalmente justificada. A grande maioria dos camponeses e trabalhadores tem uma qualidade de vida mais elevada, com direitos humanos socioeconómicos mais substanciais sob o socialismo do que sob o capitalismo. Se o resto da América Latina tivesse a mesma taxa de mortalidade infantil que Cuba, mais de {\color{blue}2} milhões de vidas de crianças seriam salvas todos os anos. Cuba enfrenta actualmente grandes problemas económicos, devidos principalmente à descontinuidade da rede internacional de apoio resultante das mudanças na União Soviética e na Europa Oriental. Mas estas dificuldades não são maiores do que na maioria dos países latino-americanos, que enfrentam uma das maiores depressões deste século. A desnutrição e a fome estão reaparecendo em países como a Argentina e o Uruguai, onde estes fenómenos de massa não existiram nos últimos {\color{blue}40} anos (17). O aparecimento da cólera a nível continental é mais um sintoma desta deterioração socioeconómica (18).
 \par 
\section{Emprego urbano - atender, %}
 \par 
Na Ásia, a China Popular e a Índia podem ser comparadas com base no seu enorme tamanho populacional, composição multinacional e nível de desenvolvimento na época da revolução chinesa. As Tabelas {\color{blue}2} a {\color{blue}5} mostram como as condições de vida eram piores na China pré-revolucionária do que na Índia. Desde a revolução, contudo, os indicadores de bem-estar melhoraram muito mais rapidamente na China do que na Índia. A esperança de vida na China era inferior à da Índia na década de 1950; hoje, a expectativa de vida na China é melhor do que na Índia
 \par 
\section{O socialismo falhou? / 587}
 \par 
(Mesa {\color{blue}2}). Da mesma forma, as taxas de mortalidade infantil (Tabela {\color{blue}3}), de mortalidade de menores de cinco anos e de mortalidade infantil (1 a {\color{blue}4} anos) na China eram piores do que as da Índia antes da revolução, mas são agora muito melhores do que as da Índia. E as taxas de mortalidade de menores de cinco anos e de mortalidade infantil na China melhoraram mais rapidamente do que as da Índia (Tabelas {\color{blue}4} e {\color{blue}5}). Na década de 1980, a China também tinha melhores níveis nutricionais e melhores taxas de alfabetização do que a Índia (Tabela {\color{blue}6}).
 \par 
É importante notar que se a taxa de mortalidade infantil da Índia, por exemplo, fosse igual à da China, {\color{blue}4} milhões de vidas infantis seriam salvas em apenas um ano. As melhorias na China foram, em parte, resultado de uma melhor nutrição. As Tabelas {\color{blue}7} e {\color{blue}8} mostram que, embora as condições nutricionais fossem piores na China do que na Índia antes da revolução chinesa, melhoraram mais rapidamente na China do que na Índia. As taxas de aumento da altura por década para crianças dos {\color{blue}5} aos {\color{blue}7} anos têm sido tão elevadas ou superiores na China ao longo dos últimos {\color{blue}20} anos do que os aumentos por década na experiência europeia do século XX, onde os rendimentos têm sido muito elevados e crescem rapidamente (19) .
 \par 
Além disso, embora os indicadores de saúde tenham melhorado significativamente na China em comparação com a Índia, o fizeram a níveis semelhantes de PIB per capita. A China tem indicadores de saúde muito melhores, com níveis de PIB per capita semelhantes (Tabela {\color{blue}9}).
 \par 
É também importante notar que a dramática taxa de melhoria da mortalidade infantil no período 1949-1980 abrandou desde a introdução de elementos do capitalismo na China no início da década de 1980. A mortalidade infantil diminuiu significativamente até 1981, altura em que a taxa de declínio nas zonas rurais abrandou consideravelmente, enquanto nas zonas urbanas
 \par 
\section{Da população da América Latina -}
 \par 
\section{588 / Navarro}
 \par 
194045 1945-50 1950-55 1955-60 1m {\color{blue}5} 1%5-70 1970-75 1975-80 1982 1984 1986 1987
 \par 
27,7 30,5
 \par 
37,7 49,0 57,3 64,2 67,8 68,5 69,1 69,5
 \par 
45,5 (F) t
 \par 
47,9 (M)
 \par 
48,4 (1972) 51,7 (1977) 55,0 56,1 57,3 57,9
 \par 
(1961-70;
 \par 
“Fontes: Para a China, 194CL1980 Estimativas do Banco Mundial baseadas em dados oficialmente disponíveis; e Hill, K. China: An Evaluafion of Demographic Trends-1950-82, PIIN Technical Note DEM {\color{blue}4}. Para China, 1982-1988, e Índia: Banco Mundial. Tabelas Mundiais, Edição 1989-90, salvo indicação em contrário.
 \par 
H i mat e s para homens e mulheres. Fonte: O Impacto do Desenvolvimento Social e Económico na Mortalidade. Em Good Health af Low Cosf, editado por S. Halstead, J. Walsh e K. Warren. Fundação Rockefeller, Nova York, 1985.
 \par 
\section{mesa 2}
 \par 
\section{Expectativa de vida ao nascer na China e na Índia, 1940-1987”}
 \par 
Taxa de mortalidade infantil (por {\color{blue}1}.{\color{blue}000} nascidos vivos) na China e na Índia, {\color{blue}194} e 1987“
 \par 
1940-45 1945-50 1950-55 1955-60 1%5-70 1970-75 1975-80 1985' 1987
 \par 
290 {\color{blue}265} 236 {\color{blue}229} 137% {\color{blue}65} 36 {\color{blue}32}
 \par 
192 (1941-50)
 \par 
140(1951-60)
 \par 
135 (1972) {\color{blue}126} (1977) {\color{blue}105} {\color{blue}99}
 \par 
“Fontes: Para a China, 1940-1980, ver Tabela 1, nota de rodapé a. Para a Índia, 1940-1970, ver Tabela 1, nota de rodapé b. Anos restantes, para Índia e China: Banco Mundial. Tabelas Mundiais, Edição 1989-90, salvo indicação em contrário.
 \par 
\section{O socialismo falhou? / 589}
 \par 
\section{Tabela 3}
 \par 
Taxas de mortalidade de menores de cinco anos (por {\color{blue}1}.{\color{blue}000} nascidos vivos) na China e na Índia, 1960 e 1983'
 \par 
1960 1983
 \par 
340 {\color{blue}55}
 \par 
300 {\color{blue}165}
 \par 
\section{B ~u r c e : referência 11.}
 \par 
\section{Tabela 4}
 \par 
Taxas de mortalidade infantil (de {\color{blue}1} a {\color{blue}4} anos) (por {\color{blue}1}.{\color{blue}000}) na China e na Índia, vários anos, 1960-1981“
 \par 
1960 1%5 1970 1975 1977 1979 1981
 \par 
26,1 (1964-65) 17,7 10,7 10,3 9,0 7,4 7,2
 \par 
26,2 23,2 20,7 19,0 18,6 17,8 17,0
 \par 
“Fonte: Banco Mundial. Tabelas do Banco Mundial, Ed. {\color{blue}3}. Nova York,
 \par 
1983.
 \par 
\section{“%urce: referência 10.}
 \par 
\section{Tabela 5}
 \par 
Taxa de alfabetização, matrícula escolar e níveis nutricionais na China e na Índia, vários anos“
 \par 
Percentagem de adultos alfabetizados, homens/mulheres, 1985
 \par 
82/56
 \par 
57/29
 \par 
Percentagem de matriculados na escola primária, homens/mulheres, 1982-84
 \par 
100/93
 \par 
Fornecimento de calorias de capital Dailv Der como porcentagem das necessidades, 1983
 \par 
111
 \par 

 \par 
“Fonte: referências {\color{blue}10} e 11, e UNICEF, The State off he World’s Children, Oxford University Press,
 \par 
\section{590 / Navarro}
 \par 
\section{Tabela 6}
 \par 
Oferta de calorias per capita (como percentagem das necessidades) na China e na Índia, vários anos, 196CL1983'
 \par 
1960 1%5 1970 1975 1977 1979 1981 1983'
 \par 
78,8 (1964-65) n/d 88,7 94,3 96,6 104,9 107,0 111,0
 \par 
95,6 (1961-65) n/d 90,4 81,8 88,7 94,2 87,5 %.O
 \par 
“Fonte: Tabelas do Banco Mundial, Ed. {\color{blue} 3 } {\par} , 1983, salvo indicação em contrário.
 \par 
\section{Nova Iorque. 1985.}
 \par 
\section{Tabela 7}
 \par 
Fornecimento de proteínas per capita (gramas por dia) para a China e a Índia, vários anos, 1960-1980“
 \par 
1960 1%5 1970 1975 1977 1979 1980
 \par 
49,6 (1964-65) n/d {\color{blue}53} 58,1 59,7 65,5 66,8
 \par 
53,6 (1961-65) n/d 49,7 {\color{blue}45} 48,3 50,6 46,6
 \par 
\section{bFonte: referência 11.}
 \par 
\section{Tabela 8}
 \par 
\section{‘Fonte: Tabelas do Banco Mundial, Ed. 3, 1983.}
 \par 
PIB atual per capita (dólares americanos) para China e Índia, vários anos, 1968-1988"
 \par 
1968 1970 1972 1974 1976 1978 1980 1982 1984 1986 1988
 \par 
90 {\color{blue}120} 130 {\color{blue}160} 170 {\color{blue}220} 300 {\color{blue}320} 330 {\color{blue}310} {\color{blue}340}
 \par 
100 {\color{blue}110} 110 {\color{blue}140} 160 {\color{blue}190} 240 {\color{blue}280} 280 {\color{blue}290} {\color{blue}340}
 \par 
'Souroe: Tabelas do Banco Mundial, 1989-90. Estimativas do PIB per capita em valores atuais de compra (preços de mercado) em dólares americanos correntes, calculadas de acordo com a metodologia atual do Atlas do Banco Mundial.
 \par 
Áreas onde a taxa de mortalidade infantil (para o período 1983-1989) inverteu o seu declínio e começou a aumentar (Tabela {\color{blue}10}). A taxa de aumento do consumo de cereais nas zonas rurais diminuiu desde 1983, enquanto o consumo de cereais, carne e peixe atingiu um patamar depois de ter aumentado rapidamente desde a década de 1970 (Tabela {\color{blue}11}). Nas áreas urbanas, o consumo destes bens também atingiu um patamar e diminuiu.
 \par 
Na Índia, um país capitalista, o estado em que o bem-estar social da população melhorou mais substancialmente é também um dos estados onde as forças pró-socialistas foram mais fortes. Desde 1957, as forças socialistas (de tradição leninista) estiveram no governo em Kerala durante longos períodos de tempo. As taxas de mortalidade infantil eram bastante semelhantes em Kerala e no resto da Índia, pelo menos até ao final da década de 1950 (Tabela {\color{blue}12}). Os números da década de 1970 em diante - o período com a maior participação socialista no governo de Kerala - mostram uma redução dramática da mortalidade infantil em Kerala. Se compararmos as taxas de mortalidade infantil em Kerala e em toda a Índia na década de 1950 (antes da eleição das forças socialistas) com as taxas da década de 1980 (depois de quase três décadas de políticas predominantemente socialistas em Kerala), vemos que a taxa de mortalidade infantil em Kerala caiu {\color{blue}73} por cento durante este período, em comparação com uma queda de {\color{blue}26} por cento em toda a Índia (Tabela {\color{blue}13}).
 \par 
Mudanças semelhantes podem ser observadas nos dados sobre a esperança de vida. Tal como no caso da mortalidade infantil, não há diferença importante entre as taxas de aumento em Kerala e em toda a Índia até 1961-70 (Tabelas {\color{blue}14} e {\color{blue}15}). As melhorias nas taxas de alfabetização, especialmente para as mulheres, entre as décadas de 1950 e 1980 também são marcantes (Tabelas {\color{blue}16} e {\color{blue}17}). Todas as melhorias em Kerala ocorreram com rendimentos per capita semelhantes aos de toda a Índia (Tabela {\color{blue}18}).
 \par 
Outro grande país da Ásia com uma grande diversidade de nacionalidades é a (antiga) União Soviética. Uma comparação das repúblicas asiáticas da URSS com países comparáveis ​​nas suas fronteiras mostra que os indicadores de saúde são muito melhores no que costumavam ser as repúblicas socialistas da URSS do que nos países capitalistas vizinhos, embora estes indicadores fossem igualmente fracos antes do socialismo ser estabelecido no
 \par 
\section{O socialismo falhou? / 591}
 \par 
\section{Tabela 9}
 \par 
\section{592 / Navarro}
 \par 
UN.
 \par 
\section{Tabela 10}
 \par 
\section{Mortalidade infantil (por 1.000) na China, pré-1949 a 1989, várias fontes}
 \par 
Antes de 1949 1950 1954 1955 1958
 \par 
\section{Ministério da Saúde Pública da China Popular}
 \par 
195
 \par 
179
 \par 
200
 \par 
138,5'
 \par 
80Bb 50,8 89,1
 \par 
1960 1%5 1970 1973-1975 1975 1980 1981
 \par 
121 {\color{blue}81} {\color{blue}61}
 \par 
41 {\color{blue}38}
 \par 
47,0'
 \par 
34,76
 \par 
1983
 \par 
Cidades [=I' condados [58 em {\color{blue}12} províncias]
 \par 
13,6 26,5
 \par 
1985
 \par 
36 (1985)
 \par 
Cidades [36] Condados [72 em {\color{blue}15} províncias]
 \par 
14,0 25,1
 \par 
1989
 \par 
Cidades [82] Condados [72 em {\color{blue}15} províncias]
 \par 
13,8 21,7
 \par 
'De uma pesquisa realizada com {\color{blue}50}.{\color{blue}000} habitantes em {\color{blue}14} províncias. 'De uma pesquisa realizada na maioria das cidades e condados de {\color{blue}19} províncias. 'Do estudo retrospectivo nacional sobre mortalidade por câncer na China. dDo terceiro Censo (1982). 'O número de cidades ou condados é dado entre parênteses.
 \par 
URSS. A Tabela {\color{blue}19} mostra a evolução das taxas de mortalidade infantil nas repúblicas soviéticas, incluindo as repúblicas asiáticas. A taxa de mortalidade infantil estimada na Ásia Central de {\color{blue}46} por {\color{blue}1}.{\color{blue}000} nados vivos em 1975 era consideravelmente melhor do que a da Turquia (153/1.{\color{blue}000}), do Afeganistão (269/1.{\color{blue}000}) e do Irão (120/1.{\color{blue}000}).
 \par 
Em resumo, os dados empíricos apresentados nesta discussão da experiência asiática não parecem confirmar a posição de que o capitalismo teve um desempenho melhor do que o socialismo na melhoria da saúde das populações.
 \par 
\section{UNICEF}
 \par 
Em África, a experiência socialista é demasiado nova para ser capaz de detectar mudanças significativas. Na Europa, a comparação não é tão favorável ao socialismo. As repúblicas europeias da União Soviética não têm melhores indicadores de saúde do que a maioria dos países capitalistas
 \par 
\section{Cidades Condados}
 \par 
\section{África e Europa}
 \par 
Consumo de alimentos (em quilogramas) em populações rurais e urbanas na China Popular, vários anos, 1978-1988"
 \par 
\section{O socialismo falhou? / 593}
 \par 
\section{Tabela 11}
 \par 
\section{Todos os cereais}
 \par 
1978 1982 1985 1988
 \par 
248,00 260,00 257,45 259,51
 \par 
123,01 192,14 209,31 210,46
 \par 
5,76 9,05 10,97 10,71
 \par 
0,25 0,78 1,03 1,25
 \par 
0,84 1,32 1,a 1,91
 \par 
\section{Aves}
 \par 
1981 1982 1985 1988
 \par 
145,44 144,56 131,16 137,17
 \par 
18,60 18,67 20,16 19,75
 \par 
7,26 7,67 7,80 7,07
 \par 
\section{População rural}
 \par 
\section{População urbana}
 \par 
\section{'Fonte: Anuário Estatístico da China Popular 1991. bCarne de porco, carne bovina e carneiro.}
 \par 
\section{Tabela 12}
 \par 
\section{Mortalidade infantil (por 1.000) em Kerala e em toda a Índia, vários anos, 1911-1988"}
 \par 
1911-20 193140 1951-60 1971-75 1976-80 1981-85 1986-88
 \par 
242 {\color{blue}173} 120 {\color{blue}57} 46 {\color{blue}32} {\color{blue}27}
 \par 
278 {\color{blue}207} 140 {\color{blue}134} 124 {\color{blue}104} {\color{blue}95}
 \par 
'Fontes: Para 1911-1%0: compilado de várias publicações do Censo da Índia em The Impact of Social and Economic Development on Mortality; ver Tabela 1, nota de rodapé b. Para 1971-1988: Mari Bhat, P. N., e Irudaya Rajan, S. Transição demográfica em Kerala revisitada. Semanal Econômico e Político {\color{blue}25}.1957-1980.1990.
 \par 
\section{Querala}
 \par 
Diminuição da mortalidade infantil em Kerala e em toda a Índia entre as décadas de 1950 e 1980''
 \par 
Taxa para 1951-60, por {\color{blue}1}.{\color{blue}000} Taxa para 1981-85, por {\color{blue}1}.{\color{blue}000} Redução, como porcentagem da taxa de 1951-60
 \par 
120 {\color{blue}32} 73%
 \par 
140 {\color{blue}104} 26%
 \par 
\section{Toda a Índia}
 \par 
\section{Tabela 13}
 \par 
\section{'Fonte: Baseado na Tabela 12.}
 \par 
\[594 I Navarro\]
 \par 
\section{Tabela 14}
 \par 
1911-20 {\color{blue}192} 1-30 1951-60 1%1-70 1971-75 1976-80 1981-85 1986
 \par 
25,5 29,5 {\color{blue}49} 59,3 60,5 63,5 65,2 67,5
 \par 
27,4 32,7 48,3 59,3 {\color{blue}63} 67,4 71,5 {\color{blue}73}
 \par 
22,6 26,9 41,4 47,9 49,7 51,7 54,5 {\color{blue}56}
 \par 
23,3 26,6 {\color{blue}40} 45,5 48,3 51,8 54,9 56,5
 \par 
'Fontes: ver Tabela 12, nota de rodapé a
 \par 
\section{Expectativa de vida, por sexo, em Kerala e em toda a Índia, vários anos, 1911-1986'}
 \par 
Aumentos na esperança de vida, por sexo, em Kerala e em toda a Índia, entre as décadas de 1950 e 1980"
 \par 
\section{Fêmea}
 \par 
49 65,2 16,2
 \par 
48,3 71,5 23,2
 \par 
41,4 54,5 13,1
 \par 
40 54,9 14,1
 \par 
\section{Tabela 15}
 \par 
\section{Expectativa de vida, 1951-60 Expectativa de vida, 1981-85 Aumento, em anos}
 \par 
Taxas de alfabetização (como porcentagem da população) em Kerala e em toda a Índia, vários anos, 1951-1981
 \par 
1951 1%1 1971 1981
 \par 
40 {\color{blue}47} {\color{blue}60}
 \par 
50 {\color{blue}55} 67 {\color{blue}75}
 \par 
32 {\color{blue}39} 54 {\color{blue}66}
 \par 
17 {\color{blue}24} {\color{blue}30}
 \par 
25 {\color{blue}34} 40 {\color{blue}47}
 \par 
8 {\color{blue}13} 19 {\color{blue}25}
 \par 
'Fonte: O Impacto do Desenvolvimento Social e Económico na Mortalidade; ver Tabela 1, nota de rodapé b.
 \par 
\section{'Fonte: Baseado na Tabela 14}
 \par 
\section{Tabela 16}
 \par 
Aumento da alfabetização, por sexo, em Kerala e em toda a Índia, entre as décadas de 1950 e 1980”
 \par 
\section{O socialismo falhou? / 595}
 \par 
50 {\color{blue}75} {\color{blue}25}
 \par 
32 {\color{blue}66} {\color{blue}34}
 \par 
25 {\color{blue}47} {\color{blue}22}
 \par 
8 {\color{blue}25} {\color{blue}17}
 \par 
‘Fonte: Baseado na Tabela {\color{blue}16}. Taxas de alfabetização em percentagem da população.
 \par 
\section{Tabela 17}
 \par 
Renda per capita (em rúpias) em Kerala e em toda a Índia, vários anos, das décadas de 1950 a 1980'' * Kerala
 \par 
1950-51 1955-56 1960-61 198041'
 \par 
304 {\color{blue}312} 326 {\color{blue}1}.{\color{blue}382}
 \par 
2% {\color{blue}308} 336 1,571
 \par 
‘Fonte: Para {\color{blue}19}.50-1961: O Impacto do Desenvolvimento Social e Económico na Mortalidade; ver Tabela 2, nota de rodapé {\color{blue}6}. Para 1980-1981: Sistema de Saúde em Kerala e seu Impacto na Mortalidade Infantil. Em Good Heolrh ou Low Cost, editado por S. Halstead, I. Walsh e K. Warren. The Rockefeller Foundation, Nova York, 1985. Preços, salvo indicação em contrário.
 \par 
% vem em 1-1 ‘Estimativas em preços de 1981.
 \par 
Países da Europa Ocidental. Foi esta situação que forneceu a munição para aqueles que definem a disparidade de resultados como um fracasso do socialismo. A seguinte declaração, que apareceu em 1981 numa das publicações intelectuais mais influentes dos Estados Unidos, a New York Review of Books, é representativa: “Não há um único país em toda a Europa onde as vidas sejam tão curtas ou os bebés ' as mortes são tão altas que nem mesmo a empobrecida e semicivilizada Albânia. No domínio da saúde, os pares da União Soviética encontram-se na América Latina e na Ásia” (20).
 \par 
A informação empírica, amplamente disponível aos estudiosos nos Estados Unidos, não confirma esta afirmação. A esperança de vida na URSS em 1975 era de 70,4 anos, apenas {\color{blue}8} meses a menos que a esperança de vida nos Estados Unidos no mesmo ano. A esperança de vida soviética era superior à da Finlândia, Jugoslávia, Roménia, Polónia, Hungria, Checoslováquia, Albânia e Portugal. Foi consideravelmente maior do que na maioria dos países da América Latina (México, 64,7 anos; Chile, 62,6; Brasil, 61,4; Argentina, 68,2) e da Ásia (Afeganistão, {\color{blue}40} anos; Irã, 51; Turquia, 56,9). Na verdade, a esperança de vida na URSS era apenas ligeiramente inferior à dos principais países capitalistas avançados.
 \par 
\section{Taxa para 1951b Taxa para aumento de 1981}
 \par 
\section{Tabela 18}
 \par 
Taxas de mortalidade infantil (por {\color{blue}1}.{\color{blue}000} nascidos vivos), nas repúblicas soviéticas e em algumas grandes cidades, 1960-1974”
 \par 
\section{596 / Navarro}
 \par 
1960
 \par 
1%7
 \par 
1970
 \par 
1974
 \par 
\section{Tabela 19}
 \par 
\section{República (Cidade)}
 \par 
37,0 30,0 34,9
 \par 
25,0 18,4 21,0
 \par 
23,0 17,3 19,0
 \par 
23,0 17,4 (1973) 17,0
 \par 
\section{Repúblicas Eslavas}
 \par 
31,2 27,0 38,0
 \par 
19,2 17,0 20,5
 \par 
17,8 18,0 19,3
 \par 
17,6 19,0 19,4
 \par 
\section{S.F.S.R russo Ucrânia Bielorrússia Repúblicas Bálticas}
 \par 
\section{Estônia Letônia Lituânia}
 \par 
\section{Repúblicas Transcaucasianas}
 \par 
\section{Armênia}
 \par 
\section{(Erevan)}
 \par 
50,0
 \par 
36,8
 \par 
43,0
 \par 
28,0
 \par 
29,0
 \par 
(26,7) - (21,3) - (24,1)
 \par 
(21,4) - (33,9) - (20,7)
 \par 
\section{Geórgia}
 \par 
\section{(Tbilisi) Azerbaijão (Baku)}
 \par 
\section{Repúblicas da Ásia Central}
 \par 
\section{Cazaquistão}
 \par 
\section{(Alma-Ata)}
 \par 
\section{Quirguizia}
 \par 
\section{(Frunze) Tadjiquistão}
 \par 
\section{(Duschambe)}
 \par 
\section{Turquemenistão (Ashkhabad)}
 \par 
\section{Uzbequistão}
 \par 
\section{(Tashkent)}
 \par 
36,8
 \par 
30,0
 \par 
30,0
 \par 
28,0
 \par 
26,0
 \par 
43,0
 \par 
38,0
 \par 
31.{\color{blue}0}
 \par 
(26,7) - (25,3) - (46,7) - (32,4) - (40,0) - (1 6,8) 24,7
 \par 
(29,2) - (24,1) - (5 1,8) - (46,4) - (45,5) - (24,4) 27,9
 \par 
\section{Moldávia}
 \par 
35,3
 \par 
26,0
 \par 
‘Fonte: Davis, C., e Feshback, M. Rising Infanf Morfality in the U.S.S.R. in the 1970.q Tabelas {\color{blue}2} e {\color{blue}4}.
 \par 
TTTTTT, como Reino Unido (72,4 anos), Japão (72,9) e Alemanha Ocidental (71,3). Da mesma forma, a taxa de mortalidade infantil na URSS em 1974 (27,9/1.{\color{blue}000}) comparou-se favoravelmente com as taxas de 1975 da Áustria (21/1.{\color{blue}000}), da Alemanha Ocidental (20/1.{\color{blue}000}), da Itália (21/1.{\color{blue}000}), do Reino Unido (16 /1.{\color{blue}000}) e Austrália (17/1.{\color{blue}000}) (21). As condições de saúde na URSS melhoraram substancialmente desde a Segunda Guerra Mundial. Foi em meados da década de 1960 que a mortalidade infantil começou a aumentar e a esperança de vida a diminuir, especialmente nas repúblicas asiáticas; esta situação tem sido objecto de amplo debate. Mas a evidência disponível não confirma a conclusão da New York Review of Books de que os pares de saúde da União Soviética se encontravam no mundo subdesenvolvido.
 \par 
Ainda assim, o projecto socialista soviético não teve um desempenho tão bom como a maioria dos homólogos capitalistas no Ocidente. Publiquei uma crítica detalhada do modelo soviético
 \par 
TTTTTT em outro lugar (22). O fosso entre a União Soviética e os países capitalistas tornou o modelo soviético pouco atraente para as populações ocidentais e, eventualmente, também para as populações soviéticas.
 \par 
Esta pesquisa internacional mostra que, pelo menos no domínio do subdesenvolvimento, onde a fome e a desnutrição fazem parte da realidade quotidiana, o socialismo, e não o capitalismo, é a forma de organização da produção e distribuição de bens e serviços que melhor responde às necessidades socioeconómicas imediatas dos a maioria dessas populações. É claro que, apesar de melhorias importantes nos indicadores de saúde, a situação dos países subdesenvolvidos impõe sérias restrições ao socialismo e muitas vezes leva a limitações de direitos políticos, tais como direitos de organização, diversidade política e liberdade de imprensa. Isto explica o desencanto de grande parte da intelectualidade dos países capitalistas ocidentais desenvolvidos com esse tipo de socialismo. Mas a sua superioridade sobre o capitalismo na promoção dos direitos socioeconómicos, incluindo os direitos à saúde, explica a enorme atractividade do projecto socialista entre as populações do mundo subdesenvolvido. Testemunhe o enorme sucesso político dos recentes movimentos populares socialistas do partido de Lula no Brasil, Cardenas.in- México (cuja vitória nas eleições presidenciais do México foi roubada pelo actual regime antidemocrático), o Congresso Nacional Africano na África do Sul, e o forças socialistas no Nepal, para mencionar apenas alguns acontecimentos políticos no ano passado. O socialismo não tem sido inferior ao capitalismo no mundo do subdesenvolvimento e a sua atractividade para as populações dos países subdesenvolvidos continua elevada.
 \par 
\section{(Chisinau)}
 \par 
Embora o leninismo tenha sido, pelo menos até recentemente, a forma predominante de socialismo nos países subdesenvolvidos, a social-democracia tem sido a versão predominante nos países capitalistas desenvolvidos. É importante sublinhar que durante a maior parte deste século, as duas tradições socialistas diferiram nos seus meios, mas não nos seus fins. Na verdade, a social-democracia durante a maior parte deste século teve como objectivo o estabelecimento do socialismo. Tal como afirma um dos partidos social-democratas mais influentes, o Partido Social-democrata Sueco, “Aspiramos a transformar completamente a organização da sociedade burguesa e a promover a libertação social da classe trabalhadora” (citado em {\color{blue}23}). O projecto socialista apelava à “abolição da exploração, à destruição da divisão da sociedade em classes, ao fim do desperdício da produção capitalista e à erradicação de todas as fontes de injustiça e preconceito”. Isto exigiu a socialização (ou colectivização ou nacionalização, termos usados ​​com ambiguidade deliberada nos programas económicos da maioria dos partidos social-democratas) dos meios de produção. Os social-democratas e os leninistas divergiam principalmente quanto aos meios para atingir esse objectivo. Enquanto os leninistas apoiavam a insurreição e a tomada do Estado, os sociais-democratas favoreciam a via eleitoral, acreditando que “o sufrágio universal é incompatível com uma sociedade dividida numa pequena classe de proprietários e numa grande classe de proprietários. Ou os ricos e os possuidores tirarão o sufrágio universal, ou os pobres, com a ajuda do seu direito de voto, adquirirão para si uma parte das riquezas acumuladas” (23).
 \par 
O socialismo reformista, em oposição ao socialismo de insurreição, visava a transformação gradual da sociedade através do processo eleitoral. Mas, como diz Kautsky (24), o
 \par 
O principal teórico do TTTTTT da Internacional Socialista indicou que as reformas eram percebidas não como um substituto para a revolução social, mas como um caminho para ela. A constituição da maioria dos partidos social-democratas em países capitalistas desenvolvidos (exceto o Partido Social-Democrata Português) reivindica fidelidade ao projeto socialista e à necessidade de transcender ou romper com o capitalismo. Mesmo em 1981, o Partido Socialista Francês venceu as eleições (em aliança com o Partido Comunista) com um chamado para romper com o capitalismo!
 \par 
Os objectivos e tácticas originais dos partidos social-democratas tiveram de ser modificados devido à necessidade determinada pelo processo eleitoral que escolheram obedecer - estabelecer alianças eleitorais e alargar a sua base para alcançar as tão necessárias maiorias eleitorais. Esta necessidade explica as mudanças nas suas políticas económicas e sociais. Entre as políticas económicas, a mudança mais importante foi a redefinição do controlo colectivo dos meios de produção. Acreditava-se que o controle não exigia propriedade estatal real. O Estado era considerado o agente que poderia dirigir e regular os meios de produção sem possuí-los. Uma forma de regular a produção era controlar ou influenciar o crédito e determinar o nível de consumo global que mitigaria, através de medidas de bem-estar, as desigualdades estabelecidas pelo mercado.
 \par 
A nível social, a necessidade de alargar alianças levou ao estabelecimento do universalismo como princípio fundamental da política social. A introdução de programas de saúde universais e abrangentes, nos quais os movimentos laborais e os seus instrumentos políticos desempenham um papel fundamental, foi um resultado directo da necessidade de os partidos social-democratas se tornarem os partidos do povo e não apenas os partidos da classe trabalhadora. Este foco no domínio do consumo levou ao estabelecimento do Estado-providência, uma criação da maior parte dos partidos social-democratas no período pós-Primeira Guerra Mundial. A criação e expansão do Estado-Providência levaram, no final dos anos 1960 e início dos anos 1970 (estimulados por uma radicalização das reivindicações trabalhistas e populares, com o surgimento dos movimentos sociais) a um questionamento das relações de propriedade na produção. Mostrei noutro local como os partidos social-democratas na década de 1970 passaram da política de consumo para a política de produção (25). As famosas propostas Meidner do Partido Social Democrata Sueco, por exemplo, visavam coletivizar os meios de produção. Como afirmou um importante teórico do partido, “a implementação destas propostas tornaria a Suécia o primeiro país do mundo a dar passos decisivos e virtualmente irreversíveis em direcção a relações de produção socialistas de uma forma democrática e reformista” (26).
 \par 
\section{Toda a URSS}
 \par 
Essas políticas universalistas levaram a um crescimento na atração do eleitorado pelos partidos social-democratas. As décadas socialistas foram as décadas de 1970 e 1980. A Tabela {\color{blue}20} mostra o crescimento dos partidos socialistas na Europa. Em 1989, o bloco de esquerda (partidos social-democratas, partidos comunistas e os Verdes) tornou-se o bloco majoritário no Parlamento Europeu. No entanto, a experiência socialista em países capitalistas desenvolvidos durante este século foi curta. Até muito recentemente, a experiência governamental da social-democracia foi limitada porque a esmagadora maioria dos partidos social-democratas nunca governou em maioria. Eles tiveram que estabelecer alianças com partidos pró-capitalistas, o que restringiu a realização de seus programas socialistas. Somente na Suécia e na Noruega o período socialista de incumbência excedeu o dos partidos pró-capitalistas no período de 1945-1978, com a Suécia tendo o período mais longo de governo socialista. Não surpreendentemente, também é a Suécia que tem
 \par 
\section{Bureau of the Census dos EUA, Washington, DC, 1980.}
 \par 
\section{O socialismo falhou? / 597}
 \par 
Países onde o movimento laboral superou {\color{blue}50} por cento dos votos nas eleições parlamentares nacionais desde 1%5"
 \par 
\[SOCIALISMO NOS PAÍSES CAPITALISTAS DESENVOLVIDOS\]
 \par 
\[598 I Navarro\]
 \par 
1971,1975,1979 1981 1966 1981 1968,1970,1982 1%9 1982,1986 1976
 \par 
'Fonte: Therbom, G. As perspectivas do trabalho e a transformação do capitalismo avançado. NovoLefiRev. 145: 8,1984.
 \par 
Fez a única tentativa séria de ir além do estado de bem-estar social e concentrar-se novamente na política de produção (com o plano Meidner).
 \par 
\section{Partidos socialdemocratas jogados}
 \par 
A avaliação do socialismo nos países capitalistas desenvolvidos tem de ter em conta o grau de influência dos partidos socialistas nas políticas governamentais e também a sua força organizacional, medida pela densidade sindical e pela unidade do movimento operário, o principal eleitorado dos partidos socialistas (27 ). A influência nas políticas governamentais é medida pela estabilidade e duração do controlo do partido socialista sobre o governo e pelo nível de participação (maioria ou maioria substancial) durante o período em estudo. Outro elemento, que liga o poder político ao económico, é o nível de sindicalização e a articulação do movimento sindical com o partido ou partidos socialistas, ou seja, se os sindicatos estão organizados segundo linhas de classe e vêem os partidos como seus instrumentos políticos ou se estão organizados por interesses religiosos, políticos ou corporativistas. Até ao final da década de 1970, apenas três países - a Suécia, a Noruega e a Dinamarca - tinham governos de maioria social-democrata, e apenas na Suécia e na Noruega é que os pró-socialistas estiveram no poder durante mais tempo do que os pró-capitalistas. Em 1970, a Suécia tinha um governo socialista há {\color{blue}24} anos, a Noruega há {\color{blue}20} anos e a Dinamarca há {\color{blue}16} anos. Todos estes países apresentam um elevado nível de sindicalização; os sindicatos seguem linhas de classe sem divisões por motivos religiosos ou políticos e vêem os partidos políticos como os seus instrumentos políticos. Imediatamente após a Primeira Guerra Mundial, todos os três países tinham indicadores de saúde semelhantes ou até piores (no caso da Dinamarca) do que os dos Estados Unidos, o país capitalista com os governos pró-socialistas mais fracos e os governos pró-capitalistas mais fortes durante este período (1947). -1978). Em 1980, os três países tinham melhorado dramaticamente os seus indicadores de saúde, alcançando alguns dos melhores indicadores de saúde do mundo ocidental (Tabela {\color{blue}21}).
 \par 
\section{O socialismo falhou? / 599}
 \par 
\section{Tabela 20}
 \par 
Taxas de adesão à União e de mortalidade infantil nos países social-democratas e nos Estados Unidos, 1950-1980
 \par 
\section{Ano eleitoral)}
 \par 
TTTTTT durante a adesão de 1945-70, {\color{blue}76}
 \par 
1950
 \par 
1980
 \par 
\section{Áustria França Finlândia Grécia Suécia Noruega Espanha Portugal}
 \par 
24 {\color{blue}20} 16 {\color{blue}0}
 \par 
75 {\color{blue}52} 50 {\color{blue}23}
 \par 
20 {\color{blue}24} 31 {\color{blue}29}
 \par 
6,9 8,1 8,4 12,6
 \par 
\[COMO AVALIAR A EXPERIÊNCIA SOCIALISTA SOB O CAPITALISMO\]
 \par 
Com base nesta informação dificilmente se poderia concluir que o socialismo é menos eficaz do que o capitalismo na resposta às necessidades de saúde da população. Não nego que o capitalismo tenha sido eficaz em algumas partes do mundo, e que em alguns casos limitado possa ter sido ainda mais eficaz do que o socialismo. Mas a evidência empírica apresentada neste artigo mostra que, ao contrário do que é amplamente afirmado hoje, a experiência socialista (tanto nas suas tradições leninistas como nas suas tradições social-democratas) tem sido, na maioria das vezes, mais eficiente na resposta às necessidades humanas do que a experiência capitalista. Infelizmente, a experiência socialista também incluiu desenvolvimentos muito negativos que negaram componentes importantes do projecto socialista. A distância entre a teoria e a prática socialistas tem-se assemelhado com demasiada frequência à distância entre o Sermão da Montanha e o Cristianismo nos seus 2000 anos de existência. Ainda assim, a experiência histórica do socialismo é bastante curta. O capitalismo existe há mais de três séculos. O socialismo, por outro lado, apenas começou.
 \par 
\section{600 / Navarro}
 \par 
Agradecimentos - Agradecimentos a ha Diez e Suzzane McQueen pela assistência na coleta de informações para este artigo.
 \par 
\section{Tabela 21}
 \par 
1. Fukuyama, F. O fim da história. The Nationallinterest, verão de 1990. {\color{blue}2}. MacPherson, S. Quinhentos milhões de crianças. Em Pobreza e Crianças Estamos no Terceiro Mundo. Martin's Press, Nova York, 1987.
 \par 
3. Trechos das Encíclicas do Papa: Sobre dar uma face humana ao capitalismo. New York Times, {\color{blue}3} de maio de 1991, p. A10.
 \par 
4. Tabb, W. T. O Futuro do Socialismo: Perspectiva da Esquerda. Monthly Review Press, Nova York, 1990.
 \par 
5. Pmworski, A. Poderíamos alimentar a todos? A irracionalidade do capitalismo e a inviabilidade do socialismo. Política e Sociedade, Março de 1991, p. {\color{blue}14}.
 \par 
6. Roosevelt, F. D. Discurso presidencial ao Congresso dos EUA, {\color{blue}11} de janeiro de 1944. {\color{blue}7}. McKeown, T. O papel da medicina: sonho, miragem ou nêmesis? Princeton University Press, Princeton, NJ, 1979.