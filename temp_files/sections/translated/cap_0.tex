\begin{figure}
	\centering
	\includegraphics[width=1.\textwidth]{temp\_files/images/UP\_logo.png }
	\caption{Elena Lagadinova (à direita, com Angela Davis) (1930-2017): A mais jovem partidária feminina lutando contra a monarquia nazista aliada da Bulgária durante a Segunda Guerra Mundial. Ela obteve seu PhD em agrobiologia e trabalhou como cientista pesquisadora antes de se tornar presidente do Comitê do Movimento das Mulheres Búlgaras. Lagadinova liderou a delegação búlgara na Primeira Conferência Mundial das Nações Unidas sobre Mulheres em 1975. Como os mercados livres discriminam aquelas que têm filhos, Lagadinova acreditava que apenas a intervenção do Estado poderia apoiar as mulheres em seus papéis duplos como trabalhadoras e mães. Cortesia de Elena Lagadinova.}
	\label{ }
\end{figure}
 \par 
\chapter{NOTA DO AUTOR}\label{NOTA DO AUTOR}
 \par 
= nos últimos vinte anos, estudei os impactos sociais da transição política e econômica do socialismo de estado para o capitalismo na Europa Oriental. Embora eu tenha viajado pela região pela primeira vez apenas alguns meses após a queda do Muro de Berlim em 1989, meu interesse profissional começou em 1997, quando comecei a conduzir pesquisas sobre os impactos do colapso da ideologia comunista nas pessoas comuns. Primeiro como aluno de doutorado e depois como professor universitário, morei por mais de três anos na Bulgária e dezenove meses na Alemanha Oriental e Ocidental. No verão de 1990, também passei dois meses viajando pela Iugoslávia, Romênia, Hungria, Tchecoslováquia e pela República Democrática Alemã, que logo desapareceria. Nos anos seguintes, fui um visitante frequente da Europa Oriental, dando palestras convidadas em cidades como Belgrado, Bucareste, Budapeste e Varsóvia. Como viajo frequentemente de carro, ônibus e trem, vi em primeira mão a devastação do capitalismo neoliberal por toda a região: paisagens desoladas marcadas com os restos decrépitos de fábricas outrora prósperas dando lugar a novos subúrbios com megastores no estilo Walmart vendendo quarenta e dois tipos diferentes de xampu. Também estudei como a instituição de mercados livres não regulamentados em
 \par 
A Europa Oriental devolveu muitas mulheres a um posição subordinado, economicamente dependentes dos homens.
 \par 
Desde 2004, publiquei seis livros acadêmicos e mais de três dezenas de artigos e ensaios, utilizando provas empíricas recolhidas em arquivos, entrevistas e extenso trabalho de campo etnográfico na região. Neste livro, baseio-me em mais de vinte anos de investigação e ensino para escrever uma cartilha introdutória para um público interessado nas teorias feministas socialistas europeias, na experiência do socialismo de Estado do século XX e nas suas lições para agora. Após o sucesso inesperado de Bernie Sanders nas primárias democratas de 2016, as ideias socialistas estão a circular mais amplamente entre o público americano. É essencial que façamos uma pausa e aprendamos com as experiências do passado, examinando tanto as boas como as más. Porque acredito na busca de nuances históricas e que houve algumas qualidades redentoras do socialismo de Estado, serei inevitavelmente acusado de ser um apologista do estalinismo. Ataques mordazes ad hominem são a realidade do nosso clima político hiperpolarizado, e considero bastante irônico que aqueles que afirmam abominar o totalitarismo não tenham problemas em silenciar o discurso ou em desencadear multidões histéricas no Twitter. A teórica política alemã Rosa Luxemburgo disse certa vez: “A liberdade é sempre e exclusivamente liberdade para quem pensa diferente”. Este livro trata de aprender a pensar de forma diferente em relação ao passado socialista de estado, ao nosso presente capitalista neoliberal e ao caminho para o nosso futuro coletivo.
 \par 
Ao longo deste livro, uso o termo “socialismo de estado” ou “socialista de estado” para me referir aos estados da Europa Oriental e da União Soviética dominados pelos Partidos Comunistas governantes, onde as liberdades políticas foram restringidas. Eu uso o termo
 \par 
NOTA DO AUTOR termo “socialismo democrático” ou “socialista democrático” para se referir a países onde os princípios socialistas são defendidos por partidos que competem em eleições livres e justas e onde os direitos políticos são mantidos. Embora muitos partidos re-
 \par 
Chamados a si de “comunistas”, esse termo denota o ideal de uma sociedade onde todos os ativos econômicos são de propriedade coletiva e o estado e a lei desapareceram. Em nenhum caso o comunismo real foi alcançado e, portanto, tento evitar esse termo ao me referir a estados realmente existentes.
 \par 
Sobre o tópico da semântica, também me esforcei para ser sensível aos vocabulários interseccionais contemporâneos. Por exemplo, quando falo sobre “mulheres” neste livro, estou me referindo principalmente a mulheres cis gênero. A “questão da mulher” socialista dos séculos XIX e XX não considerou as necessidades únicas das mulheres trans, mas não tenho desejo de excluir ou alienar as mulheres trans da discussão atual. Da mesma forma, na minha discussão sobre maternidade, reconheço que estou discutindo aquelas que são mulheres designadas ao nascer (FAB), mas, por uma questão de simplicidade, uso a palavra “mulher”, embora esta categoria inclua algumas que se identificam como homens ou outros gêneros.
 \par 
Como este é um livro introdutório, haverá lugares no texto onde não entrarei em detalhes completos sobre os debates em torno de tópicos como Renda Básica Universal (UBI), extração de mais-valia ou cotas baseadas em gênero. Em particular, embora eu acredite que eles sejam essenciais, não gasto muito tempo discutindo assistência médica universal de pagador único ou educação pós-secundária pública gratuita, porque sinto que essas políticas já foram discutidas longamente em outros lugares. Espero que os leitores se inspirem a explorar mais sobre as questões levantadas nestas páginas, levando isso em consideração.
 \par 
xii
 \par 
NOTA DO AUTOR livro como um convite para uma exploração mais aprofundada das interseções do socialismo e do feminismo. Também quero deixar claro que este não é um tratado acadêmico; aqueles em busca de estruturas teóricas e debates metodológicos devem consultar os livros que publiquei em editoras universitárias. Também reconheço a longa e importante tradição do feminismo socialista ocidental, embora não seja discutido nestas páginas. Além disso, encorajo os leitores interessados ​​a consultar os livros listados nas sugestões para leitura adicional.
 \par 
Para todas as citações diretas e alegações estatísticas feitas ao longo do livro, incluo citações consolidadas em uma nota final no final do parágrafo relevante. Poucas notas finais substantivas acompanham este texto, então a maioria dos leitores pode se sentir livre para ignorar as notas finais, a menos que tenham uma pergunta sobre uma fonte específica. Material histórico geral pode ser encontrado nas sugestões para leitura adicional. Ao discutir anedotas pessoais, alterei os nomes e detalhes de identificação para preservar o anonimato.
 \par 
Finalmente, com os muitos males sociais que assolam o mundo hoje, alguns podem achar os capítulos sobre relações íntimas um pouco obscenos demais para seu gosto; alguns podem pensar que ter um sexo melhor é uma razão trivial para mudar de sistema econômico. Mas ligue a televisão, abra uma revista ou navegue na internet, e você encontrará um mundo saturado de sexo. O capitalismo não tem problema em mercantilizar a sexualidade e até mesmo se aproveitar de nossas inseguranças de relacionamento para nos vender produtos e serviços que não queremos ou precisamos. As ideologias neoliberais nos persuadem a ver nossos corpos, nossas atenções e nossas afeições como coisas a serem compradas e vendidas. Quero virar o jogo. Usar a discussão sobre sexualidade para expor as deficiências dos mercados livres sem restrições. Se pudermos
 \par 
NOTA DO AUTOR entender como o atual sistema capitalista cooptou
 \par 
E comercializamos emoções humanas básicas, demos o primeiro passo em direção à rejeição de avaliações de mercado que pretendem quantificar nosso valor fundamental como seres humanos. O político é pessoal.
 \par 
xiii