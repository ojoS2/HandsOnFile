Grande parte deste livro teve origem em periódicos e revistas literárias. Se não fossem os editores Alex Star, Paul Laity, Mary-Kay Wilmers, Paul Meyerscough, Adam Shatz, John Palattella e Jackson Lears, eu nunca teria escrito sobre o direito. Supõe-se frequentemente que os académicos que publicam em espaços não académicos estão a apresentar a sua investigação académica para consumo popular, simplificando ideias complexas inicialmente elaboradas nos laboratórios académicos. Para mim, o processo de escrita deste livro foi o inverso: o conservadorismo tornou-se um interesse académico meu através da minha escrita não académica, e a maior parte das minhas ideias sobre o direito foram formuladas em conversas e escritos para estes editores, especialmente Alex e John.
 \par 
Intelectualmente, este livro deve sua inspiração a Arno Mayer e Karen Orren. Não existem dois académicos que tenham feito mais para promover a minha compreensão da “persistência do antigo regime” – na Europa e nos Estados Unidos – do que Karen e Arno. Contra a sabedoria convencional da esquerda e da direita, que assume que o medievalismo foi eliminado pela modernidade, Karen e Arno abriram-me os olhos para o “feudalismo tardio” do nosso mundo pós-feudal. Embora eles sem dúvida discordassem da minha interpretação do conservadorismo, eu não teria chegado a esse ponto sem o seu trabalho extremamente produtivo.
 \par 
Ao escrever e revisar estes ensaios, fui apoiado por um amplo círculo de leitores: historiadores e políticos
 \par 
Cientistas, poetas e ensaístas, teóricos e filósofos, críticos literários e sociólogos, jornalistas e editores. Por suas contribuições para um ou mais desses ensaios, gostaria de agradecer a Jed Abrahamian, Bruce Ackerman, Joel Allen, Gaston Alonso, Joyce Appleby, Moustafa Bayoumi, Seyla Benhabib, Mar-shall Berman, Sara Bershtel, Akeel Bilgrami, Norman Birnbaum , Steve Bronner, Dan Brook, Sebastian Budgen, Josh Cohen, Peter Cole, Paisley Currah, Lizzie Donahue, Jay Driskell, Tom Dumm, John Dunn, Sam Farber, Liza Featherstone, Jason Frank, Steve Fraser, Josh Freeman, Paul Frymer, Sam Goldman, Manu Goswami, Alex Gourevitch, Pete Hallward, Harry Harootunian, Chris Hayes, Doug Henwood, Dick Howard, David Hughes, Judy Hughes, Allen Hunter, Jack Jacobs, Ira Katznelson, Gordon Lafer, Jill Lepore, Penny Lewis, Joe Lowndes, Steven Lukes, Kieko Matteson, Kevin Mattson, John Medeiras, Kathy New-man, Molly Nolan, Anne Norton, Jolie Olcott, Christian Parenti, Di Paton, Rick Perlstein, Ros Petchesky, Kim Phillips-Fein, Katha Pollitt, Aziz Rana, Andy Rich, Andrew Ross, Kristin Ross, Saskia Sassen, Ellen Schrecker, George Scialabba, Richard Seymour, Nikhil Singh, Quentin Skinner, Jim Sleeper, Rogers Smith, Katrina van den Heuvel, John Wallach, Eve Weinbaum, Keith Whittington, Daniel Wilkinson, Wesley Yang, Brian Young e Marilyn Young.
 \par 
Boa parte desse material foi apresentada em workshops e palestras em universidades de todo o país. Estou grato pelos comentários e sugestões que recebi nessas ocasiões de Arash Abizadeh, Anthony Appiah, Banu Bargu, Seyla Benhabib, Akeel Bilgrami, Elizabeth Cohen, Josh Cohen, Julie Cooper, do falecido Jack Diggins, Matt Evans, Nancy Fraser, Mark Graber, Nan Keohane, Steve Macedo, Karuna Mantena, Andrew March, Tom Medvetz, Andrew Murphy, Andrew Norris, Anne Norton, Joshua Ober, Philip Pettit, Andy Polsky, Robert Reich, Austin Sarat, Peter
 \par 
Gostaria de agradecer às seguintes instituições por fornecerem o tão necessário tempo de liberação do meu ensino: o Conselho Americano de Sociedades Científicas; o Centro de Valores Humanos da Universidade de Princeton; o Gabinete do Reitor do Brooklyn College; e o Congresso de Pessoal Profissional da City University of New York. Um agradecimento extra especial vai para o meu armário de cozinha dos primeiros leitores: Greg Grandin, Adina Hoffman, Robert Perkinson e Scott Saul; a Marco Roth, que deu o título do livro; a Charles Petersen, extraordinário editor de textos; aos meus alunos do Brooklyn College e do CUNY Graduate from Center, que trabalharam comigo nos textos e tomos da direita; a Alexandra Dauler e Marc Schneider da Oxford University Press (OUP); e a David McBride, meu editor na OUP, uma fonte infalível de excelentes conselhos que acreditou neste projeto desde o seu início e o conduziu até o fim com o que parece ser fruto de sabedoria, paciência e graça inúteis.
 \par 
Meus maiores agradecimentos vão para Laura Brahm, que ouviu essas ideias quando elas eram meias frases e as leu quando elas estavam meio cozidas. Ela trouxe para esses ensaios um olhar para o que importa e um senso de gosto infalível. Ela é sempre e inevitavelmente a única leitora que eu quero agradar.