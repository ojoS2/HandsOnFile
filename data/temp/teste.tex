\documentclass[twocolumn,amsmath,amssymb,aps,pre,floatfix]{revtex4-2}
\usepackage{url}
\usepackage[colorlinks=true, allcolors=blue]{hyperref}
\usepackage{float}
\usepackage[utf8]{inputenc}
\usepackage[T1]{fontenc}
\usepackage{lineno}
\usepackage{amsthm}
\usepackage{}
\newtheorem{theorem}{Auxiliary result}
\newtheorem{corollary}{Main result}
\newtheorem{definition}{Definition}
\newtheorem*{gen}{General properties}
\newtheorem{prop}{ Appendix result}
\newtheorem{secondary}{Appendix' secondary result}
\usepackage{nicematrix}
\usepackage{bm}
\usepackage{dsfont}
\usepackage{amsfonts}
\usepackage{indentfirst}
\usepackage{graphicx}
\usepackage{dcolumn}
\usepackage{bm}
\usepackage{color}
\usepackage{wasysym}
\begin{document}
\title{}
\author{Vicente Navarro}
\begin{abstract}Este artigo analisa a suposição amplamente difundida na academia e na grande imprensa de que o capitalismo provou ser superior ao socialismo na resposta às necessidades humanas. O autor analisa as condições de saúde das populações mundiais, continente por continente, e mostra que, contrariamente à ideologia dominante, o socialismo e as forças socialistas têm sido, na sua maior parte, mais capazes de melhorar as condições de saúde do que o capitalismo e as forças capitalistas. No mundo subdesenvolvido, as forças e os regimes socialistas têm, mais frequentemente, melhorado os indicadores sociais e de saúde melhor do que as forças e os regimes capitalistas, e no mundo desenvolvido, os países com fortes forças socialistas têm sido mais capazes de melhorar as condições de saúde do que esses países. Carentes ou com forças socialistas fracas. A experiência socialista também incluiu, evidentemente, desenvolvimentos negativos que negaram componentes importantes do projeto socialista. Ainda assim, as evidências apresentadas neste artigo mostram que a experiência histórica do socialismo não foi de fracasso. Pelo contrário: tem sido, na maior parte, mais bem-sucedido do que o capitalismo na melhoria das condições de saúde das populações mundiais.\end{abstract}
Uma importante posição intelectual reproduzida na imprensa acadêmica e na grande imprensa de hoje é que o conflito histórico entre duas abordagens ao desenvolvimento social humano foi resolvido em favor do capitalismo: o capitalismo provou ser superior ao socialismo na resposta às necessidades humanas. Esta posição, articulada pela primeira vez pelo funcionário do Departamento de Estado dos EUA, Francis Fukuyama (l), ganhou ampla aceitação nos centros intelectuais do mundo ocidental. A posição não é apenas descritiva, mas também não-nativa; o socialismo deve ser evitado e o capitalismo deve ser promovido para resolver as realidades dramáticas do nosso mundo, onde uma criança morre de fome a cada dois segundos e 15 milhões de crianças morrem de desnutrição todos os anos (2).
\par
Como afirma o Papa João Paulo I1 na sua encíclica Centesimus Annus (citada em 3):
\par
Deveria o capitalismo ser o objectivo dos países que atualmente estão a envidar esforços para reconstruir a sua economia e a sua sociedade? Será este o modelo que deveria ser proposto aos países do Terceiro Mundo que procuram o caminho do verdadeiro progresso econômico e civil? . . . Se por capitalismo se entende um sistema econômico que reconhece o papel fundamental e positivo das empresas, do mercado, da propriedade privada e do
\par
Esta é uma versão modificada de um artigo publicado na Science ano Society.
\par
International Journal of Health Services, Volume 22, número 4, páginas 583401, 1992 0 1992, Baywood Publishing Co., Inc.
\par
583 dói: 10.2190/B2TP-3 RAM-Q7UP-DUA2 http://baywood.com
\par
Responsabilidade resultante pelos meios de produção, bem como pela criatividade humana no sector econômico, então a resposta é certamente afirmativa.
\par
Nos círculos socialistas do mundo capitalista desenvolvido ocidental, duas posições defensivas tornaram-se dominantes. Uma é a negação do carácter socialista das sociedades que se afirmam socialistas (4). O socialismo não falhou, nunca sequer existiu. O suposto fracasso das sociedades “socialistas” em satisfazer as necessidades humanas não tem qualquer influência na realização do projeto socialista, uma vez que esse projeto ainda não foi testado. É importante notar que a esmagadora maioria das contribuições teóricas que sustentam esta posição foram feitas em países capitalistas desenvolvidos.
\par
A outra posição defensiva é questionar a viabilidade de comparar sistemas completamente. Como indicado por Adam Przeworski, “é impossível dizer se o modelo socialista ou capitalista tem sido mais bem-sucedido na prática” (5). Não está claro, no entanto, porque a questão extremamente importante de se o capitalismo é superior ao socialismo em responder às necessidades humanas não pode ser objeto de investigação científica? Como a principal preocupação de Przeworski, expressa no título de seu artigo mais recente “Poderíamos alimentar todos?”, é se o capitalismo ou o socialismo é um caminho melhor para resolver os problemas de fome e desnutrição, um pesquisador poderia comparar a evolução dos níveis nutricionais de populações que vivem atualmente sob dois regimes diferentes, mas que viveram sob condições capitalistas semelhantes no início do período histórico em estudo. A dificuldade de padronizar variáveis ​​pode enfraquecer a validade da comparação, mas raramente a ponto de tornar a comparação inútil. Também é provável que a comparação tenha um viés inevitável em favor do capitalismo, uma vez que a experiência socialista sempre evoca enorme hostilidade, bloqueio econômico e até mesmo intervenção militar. Tal comparação não seria de capitalismo contra socialismo em circunstâncias normais, mas sim de capitalismo em circunstâncias normais, articulado a um sistema mundial no qual as relações capitalistas são dominantes, contra socialismo sob circunstâncias mais anormais. Ainda assim, apesar desse viés intrínseco, acredito que tais comparações têm validade e podem ser apresentadas para mostrar a superioridade de um sistema sobre outro na resposta às principais necessidades humanas, que incluem a prevenção da fome, desnutrição, doenças e morte prematura. De fato, em países capitalistas desenvolvidos, onde ocorre a maioria da produção teórica ocidental, geralmente se esquece que a maioria dos seres humanos em nosso período histórico não tem direitos socioeconômicos básicos, como alimentação, água limpa, esgoto tratado e capacidade de ler. A ausência desses direitos básicos limita todos os outros direitos humanos, como direitos civis-políticos, incluindo os direitos de organização e liberdade de imprensa. O presidente Franklin Roosevelt colocou muito bem em sua mensagem ao Congresso em 11 de janeiro de 1944: “homens [e mulheres] necessitados não são homens [e mulheres] livres” (6).
\par
Ao contrário de Przeworski, acredito que a superioridade de um sistema sobre outro pode de facto ser demonstrada. E uma forma de o fazer é mostrar a evolução dos indicadores de saúde (tais como mortalidade infantil, esperança de vida, níveis de nutrição e baixo peso à nascença, sempre que tais dados estejam disponíveis) em países comparáveis ​​que seguiram diferentes caminhos de desenvolvimento, capitalistas contra socialista. Antes de nos concentrarmos nas informações empíricas, no entanto, vários pontos precisam ser levantados.
\par
Primeiro, contrariamente à crença predominante, o nível de saúde de uma população não é principalmente o resultado de intervenções médicas. Se o país A tiver melhores indicadores de saúde do que o país
\par
B, não é porque o país A tem maiores gastos médicos. Não há correlação entre o nível de gastos médicos e o nível de saúde. Nem o nível de saúde se correlaciona com o nível de consumo médico. (Há uma extensa bibliografia sobre o impacto limitado dos cuidados médicos no nível de mortalidade e morbidade das populações; veja 7.) Baltimore, por exemplo, tem uma taxa de utilização acima da média de serviços de cuidados pré-natais para todos os setores da população, incluindo os pobres, mas devido à sua pobreza generalizada tem uma das maiores taxas de mortalidade infantil nos Estados Unidos (8). Esta observação não pretende desconsiderar a importância dos cuidados médicos na melhoria da saúde da população. Em vez disso, é para indicar que a saúde da população é o resultado de todo um conjunto de intervenções sociais, econômicas e políticas, entre as quais os cuidados médicos desempenham um papel menor. O país A com melhores indicadores de saúde do que o país B tem, em geral, melhores condições sociais e econômicas para a maioria de seus cidadãos do que o país B. Assim, os indicadores de saúde são bons indicadores de desenvolvimento social e econômico.
\par
Em segundo lugar, devemos esclarecer o que se entende por capitalismo e socialismo. O capitalismo é a produção de bens e serviços para o lucro daqueles que possuem os meios pelos quais são produzidos. Nas sociedades capitalistas, os principais meios de produção são privados. O socialismo é um sistema de produção e distribuição em que os meios de produção são propriedade pública, com o Estado desempenhando o papel fundamental na produção. Este sistema é o resultado de um processo revolucionário autônomo em que grandes sectores da classe trabalhadora e/ou do campesinato foram as principais forças por trás do estabelecimento do Estado. Excluem-se desta definição os países onde o socialismo foi imposto de fora, como a Europa Oriental ou o Afeganistão, ou por um golpe militar, como a Etiópia. Estes casos têm sido os mais frequentemente utilizados para desacreditar todo o projeto socialista. Na minha opinião eles não são socialistas.
\par
Terceiro, a superioridade de um sistema sobre outro pode ser demonstrada não apenas olhando para países comparáveis ​​com regimes diferentes, mas também analisando países capitalistas comparáveis ​​com diferentes correlações de forças entre elementos pró-capitalistas e pró-socialistas. Por outras palavras, a superioridade, digamos, do socialismo sobre o capitalismo pode ser demonstrada comparando dois países capitalistas semelhantes, um com fortes forças socialistas (forças que reivindicam um compromisso com o socialismo) e o outro sem tais forças. Se o segundo país tiver maiores necessidades humanas não resolvidas do que o primeiro, acredito que a alegação de superioridade socialista é justificada. Por outras palavras, mesmo na ausência de formações socialistas no mundo capitalista desenvolvido, a superioridade do socialismo pode ser demonstrada.
\par

\section{UMA ANÁLISE CONTINENTE POR CONTINENTE DA EXPERIÊNCIA SOCIALISTA}

\par
Começando pelo nosso hemisfério, o desempenho socialista de Cuba pode ser medido em comparação com o desempenho de países latino-americanos comparáveis ​​com regimes capitalistas. A maioria destes países tinha uma distribuição demográfica semelhante e níveis de desenvolvimento econômico e social semelhantes ou até melhores do que Cuba em 1958, quando ocorreu a revolução cubana. Desde então, os indicadores de saúde melhoraram mais rapidamente em Cuba do que no resto da América Latina. Em 1955, a esperança de vida em Cuba era de 59,5 anos, inferior à do Paraguai (62 anos), Argentina (62 anos) e Uruguai (66 anos), os países com as maiores esperanças de vida na América Latina. Em 1985, a esperança de vida em Cuba era de 75 anos, superior à de estes países e, de facto, a mais elevada da América Latina (9, Tabela 16, p. 26). É também notável que ocorreram em Cuba aumentos na esperança de vida. Em níveis iniciais relativamente elevados, onde os aumentos são muitas vezes mais difíceis de alcançar, e as melhorias na esperança de vida foram maiores do que em países como a Argentina e o Uruguai, que tinham uma maior esperança de vida para começar e maiores rendimentos per capita do que Cuba. A melhora nas taxas de mortalidade ocorreu em todas as faixas etárias. Estas taxas são agora mais baixas em Cuba do que em qualquer outro país latino-americano (9, Tabela 17, p. 29). Os Estados Unidos também oferecem uma comparação interessante. Entre 1950-55 e 1985-90, a esperança de vida nos Estados Unidos aumentou de 69 para 75,4 anos. Durante o mesmo período, a esperança de vida em Cuba aumentou de 59,3 para 75,2 anos (9, Tabela 16, p. 26).
\par
Da mesma forma, em 1955, Cuba tinha uma taxa de mortalidade infantil de 81 mortes por 1.000 nados vivos, superior à de vários outros países latino-americanos, incluindo o Paraguai, o Uruguai e a Argentina (os países com menor mortalidade infantil). Em 1985, Cuba tinha a menor taxa de mortalidade infantil (13/1.000 nascidos vivos) da América Latina (9, Tabela 1, p. 53). A taxa de mortalidade de menores de cinco anos (por 1.000 nascidos vivos) caiu de 95 para 19 em Cuba entre 1960 e 1985. Mesmo os países com taxas de mortalidade de menores de cinco anos mais baixas em 1960 e rendimentos per capita mais elevados (Trindade e Tobago, Argentina, Uruguai) foram superados por Cuba (10, 11). Cuba também tem o nível mais baixo de desnutrição da América Latina
\par
Argentina Brasil Chile Costa Rica República Dominicana El Salvador Haiti Jamaica México Nicarágua Peru Uruguai Venezuela
\par
70 63 71 73 63 64 54 70 59 58 59 70 69
\par
35 71 24 20 63 70 107 28 82 84 82 38 39
\par
6,1 25,5 8,9 6,4 27,0 38,0 77,0 12,0 17,4 12,9 17,4 6,1' 15 (0,69 m),3
\par
6,1 5,3 17,2 6,6 20,3 30,0 40,0 25,9 11,8 16,3 11,8 13,1 14,3
\par
74
\par
13
\par
3.9
\par
3.4e
\par
«Fonte: Multinacional Moniror, abril de 1989. Estes são os últimos números disponíveis. BA mortalidade infantil é definida como a morte antes de 1 ano por 1.000 nascidos, arredondada para o número inteiro mais próximo. 'Taxa de analfabetismo para maiores de 15 anos. Figura de 1975. 'Desemprego urbano e rural (Censos 1981).
\par
América para todas as faixas etárias (9, Tabela 88, p. 91; tabela 91, p. 193), embora os níveis de desnutrição, especialmente nas áreas rurais, fossem elevados (12) e comparáveis ​​aos da maioria dos países latino-americanos durante a década de 1950. Cuba é o país latino-americano com a menor percentagem de bebês com baixo peso ao nascer e de crianças menores de 5 anos que sofrem de desnutrição leve a moderada a grave, e com a melhor oferta calórica diária per capita como percentagem de necessidades (13, Tabela 2, pág. 97). Além disso, em 1956 Cuba era um dos países latino-americanos com as piores condições ambientais. Apenas 35 por cento da população vivia em casas ligadas a sistemas de abastecimento de água (Censo de Cuba de 1953, citado em 14) (comparado, por exemplo, com 63 por cento na República Dominicana, 80 por cento em Honduras e 44 por cento na Argentina) (15). Além disso, apenas 42 por cento da população vivia em casas ligadas a sistemas de esgotos (citado em 14). Em 1980, Cuba tinha um dos melhores resultados em serviços ambientais. Setenta e quatro por cento da população vivia em habitações ligadas a sistemas de abastecimento de água (apenas Trinidad tinha uma percentagem maior na América Latina: 91 por cento), e 91 por cento da população tinha acesso a autoclismos, uma das percentagens mais elevadas da América Latina. (16). A taxa de mortalidade ajustada à idade por enterite e outras doenças diarreicas eram de 2,8 por 100.000 habitantes em 1988, uma das duas mais baixas da América Latina (9, Tabela 111, p. 370). Cuba também tem a maior taxa de alfabetização da América Latina (96% da população adulta). Na década de 1950, a taxa de alfabetização era comparável à do resto dos países das Caraíbas, variando entre 30 e 40 por cento da população adulta (13, p. 101). A Tabela 1 resume alguns indicadores sociais e de saúde atuais para Cuba e outros países latino-americanos.
\par
Tendo em conta esta informação, poder-se-ia concluir que a declaração do Papa João Paulo II no CenteshusAnnus definindo o capitalismo como o melhor sistema para responder às necessidades humanas no Terceiro Mundo, pelo menos para a América Latina, pode não ser totalmente justificada. A grande maioria dos camponeses e trabalhadores tem uma qualidade de vida mais elevada, com direitos humanos socioeconômicos mais substanciais sob o socialismo do que sob o capitalismo. Se o resto da América Latina tivesse a mesma taxa de mortalidade infantil que Cuba, mais de 2 milhões de vidas de crianças seriam salvas todos os anos. Cuba enfrenta atualmente grandes problemas econômicos, devidos principalmente à descontinuidade da rede internacional de apoio resultante das mudanças na União Soviética e na Europa Oriental. Mas estas dificuldades não são maiores do que na maioria dos países latino-americanos, que enfrentam uma das maiores depressões deste século. A desnutrição e a fome estão reaparecendo em países como a Argentina e o Uruguai, onde estes fenômenos de massa não existiram nos últimos 40 anos (17). O aparecimento da cólera ao nível continental é mais um sintoma desta deterioração socioeconômica (18).
\par
Na Ásia, a China Popular e a Índia podem ser comparadas com base em seu enorme tamanho populacional, composição multinacional e nível de desenvolvimento na época da revolução chinesa. As tabelas 2-5 mostram como as condições de vida eram piores na China pré-revolucionária do que na Índia. Desde a revolução, no entanto, os indicadores de bem-estar melhoraram muito mais rapidamente na China do que na Índia. A expectativa de vida na China era menor do que na Índia na década de 1950; hoje, a expectativa de vida na China é melhor do que na Índia
\par
(Mesa 2). Da mesma forma, as taxas de mortalidade infantil (Tabela 3), de mortalidade de menores de cinco anos e de mortalidade infantil (1 a 4 anos) na China eram piores do que as da Índia antes da revolução, mas são agora muito melhores do que as da Índia. E as taxas de mortalidade de menores de cinco anos e de mortalidade infantil na China melhoraram mais rapidamente do que as da Índia (Tabelas 4 e 5). Na década de 1980, a China também tinha melhores níveis nutricionais e melhores taxas de alfabetização do que a Índia (Tabela 6).
\par
É importante notar que se a taxa de mortalidade infantil da Índia, por exemplo, fosse igual à da China, 4 milhões de vidas infantis seriam salvas em apenas um ano. As melhorias na China foram, em parte, resultado de uma melhor nutrição. As Tabelas 7 e 8 mostram que, embora as condições nutricionais fossem piores na China do que na Índia antes da revolução chinesa, melhoraram mais rapidamente na China do que na Índia. As taxas de aumento da altura por década para crianças dos 5 aos 7 anos têm sido tão elevadas ou superiores na China ao longo dos últimos 20 anos do que os aumentos por década na experiência europeia do século XX, onde os rendimentos têm sido muito elevados e crescem rapidamente (19).
\par
Além disso, embora os indicadores de saúde tenham melhorado significativamente na China em comparação com a Índia, o fizeram a níveis semelhantes de PIB per capita. A China tem indicadores de saúde muito melhores, com níveis de PIB per capita semelhantes (Tabela 9).
\par
É também importante notar que a dramática taxa de melhoria da mortalidade infantil no período 1949-1980 abrandou desde a introdução de elementos do capitalismo na China no início da década de 1980. A mortalidade infantil diminuiu significativamente até 1981, altura em que a taxa de declínio nas zonas rurais abrandou consideravelmente, enquanto nas zonas urbanas
\par
194045 1945-50 1950-55 1955-60 1m 5 1%5-70 1970-75 1975-80 1982 1984 1986 1987
\par
27,7 30,5
\par
37,7 49,0 57,3 64,2 67,8 68,5 69,1 69,5
\par
45,5 (F) t
\par
47,9 (M)
\par
48,4 (1972) 51,7 (1977) 55,0 56,1 57,3 57,9
\par
(1961-70;
\par
“Fontes: Para a China, estimativas do Banco Mundial de 194CL1980 com base em dados oficialmente disponíveis; e Hill, K. China: An Evaluafion of. Demographic Trends-1950-82, PIIN Technical Note DEM 4. Para a China, 1982-1988, e Índia: Banco Mundial. World Tables, edição de 1989-90, a menos que indicado de outra forma.
\par
H i m a t e s para homens e mulheres. Fonte: O Impacto do Desenvolvimento Social e Econômico na Mortalidade. Em Good Health a Low Cosf, editado por S. Halstead, J. Walsh e K. Warren. Fundação Rockefeller, Nova York, 1985.
\par
Taxa de mortalidade infantil (por 1.000 nascidos vivos) na China e na Índia, 194 e 1987“
\par
1940-45 1945-50 1950-55 1955-60 1%5-70 1970-75 1975-80 1985' 1987
\par
290 265 236 229 137% 65 36 32
\par
192 (1941-50)
\par
140 (1951-60)
\par
135 (1972) 126 (1977) 105 99
\par
“Fontes: Para a China, 1940-1980, ver Tabela 1, nota de rodapé a. Para a Índia, 1940-1970, ver Tabela 1, nota de rodapé b. Anos restantes, para Índia e China: Banco Mundial. Tabelas Mundiais, Edição 1989-90, salvo indicação em contrário.
\par
Taxas de mortalidade de menores de cinco anos (por 1.000 nascidos vivos) na China e na Índia, 1960 e 1983'
\par
1960 1983
\par
340 55
\par
300 165
\par
Taxas de mortalidade infantil (de 1 a 4 anos) (por 1.000) na China e na Índia, vários anos, 1960-1981“
\par
1960 1%5 1970 1975 1977 1979 1981
\par
26,1 (1964-65) 17,7 10,7 10,3 9,0 7,4 7,2
\par
26,2 23,2 20,7 19,0 18,6 17,8 17,0
\par
“Fonte: Banco Mundial. Tabelas do Banco Mundial, Ed. 3. Nova York,
\par
1983.
\par
Taxa de alfabetização, matrícula escolar e níveis nutricionais na China e na Índia, vários anos“
\par
Percentagem de adultos alfabetizados, homens/mulheres, 1985
\par
82/56
\par
57/29
\par
Percentagem de matriculados na escola primária, homens/mulheres, 1982-84
\par
100/93
\par
Fornecimento de calorias Dailv Der capita como porcentagem das necessidades, 1983
\par
111
\par
“Fonte: referências 10 e 11, e UNICEF, The State olhe World’s Children, Oxford University Press,
\par
Oferta de calorias per capita (como percentagem das necessidades) na China e na Índia, vários anos, 196CL1983'
\par
1960 1%5 1970 1975 1977 1979 1981 1983'
\par
78,8 (1964-65) n/d 88,7 94,3 96,6 104,9 107,0 111,0
\par
95,6 (1961-65) n/d 90,4 81,8 88,7 94,2 87,5%.O
\par
“Fonte: Tabelas do Banco Mundial, Ed. 3, 1983, salvo indicação em contrário.
\par
Fornecimento de proteínas per capita (gramas por dia) para a China e a Índia, vários anos, 1960-1980“
\par
1960 1%5 1970 1975 1977 1979 1980
\par
49,6 (1964-65) n/d 53 58,1 59,7 65,5 66,8
\par
53,6 (1961-65) n/d 49,7 45 48,3 50,6 46,6
\par
PIB per capita atual (dólares americanos) da China e da Índia, vários anos, 1968-1988"
\par
1968 1970 1972 1974 1976 1978 1980 1982 1984 1986 1988
\par
90 120 130 160 170 220 300 320 330 310 340
\par
100 110 110 140 160 190 240 280 280 290 340
\par
'Souroe: Tabelas do Banco Mundial, 1989-90. Estimativas do PIB per capita em valores atuais de compra (preços de mercado) em dólares americanos correntes, calculadas conforme a metodologia atual do Atlas do Banco Mundial.
\par
Nessas áreas, a taxa de mortalidade infantil (para o período 1983-1989) inverteu o seu declínio e começou a aumentar (Tabela 10). A taxa de aumento do consumo de cereais nas zonas rurais diminuiu desde 1983, enquanto o consumo de cereais, carne e peixe atingiu um patamar após ter aumentado rapidamente desde a década de 1970 (Tabela 11). Nas áreas urbanas, o consumo destes bens também atingiu um patamar e diminuiu.
\par
Na Índia, um país capitalista, o estado em que o bem-estar social da população melhorou mais substancialmente é também um dos estados onde as forças pró-socialistas foram mais fortes. Desde 1957, as forças socialistas (de tradição leninista) estiveram no governo em Kerala durante longos períodos. As taxas de mortalidade infantil eram bastante semelhantes em Kerala e no resto da Índia, pelo menos até ao final da década de 1950 (Tabela 12). Os números da década de 1970 em diante – o período com a maior participação socialista no governo de Kerala – mostram uma redução dramática da mortalidade infantil em Kerala. Se compararmos as taxas de mortalidade infantil em Kerala e em toda a Índia na década de 1950 (antes da eleição das forças socialistas) com as taxas da década de 1980 (depois de quase três décadas de políticas predominantemente socialistas em Kerala), vemos que a taxa de mortalidade infantil em Kerala caiu 73 por cento durante este período, em comparação com uma queda de 26 por cento em toda a Índia (Tabela 13).
\par
Mudanças semelhantes podem ser observadas nos dados sobre a esperança de vida. Tal como na mortalidade infantil, não há diferença importante entre as taxas de aumento em Kerala e em toda a Índia até 1961-70 (Tabelas 14 e 15). As melhorias nas taxas de alfabetização, especialmente para as mulheres, entre as décadas de 1950 e 1980 também são marcantes (Tabelas 16 e 17). Todas as melhorias em Kerala ocorreram com rendimentos per capita semelhantes aos de toda a Índia (Tabela 18).
\par
Outro grande país da Ásia com uma grande diversidade de nacionalidades é a (antiga) União Soviética. Uma comparação das repúblicas asiáticas da URSS com países comparáveis ​​nas suas fronteiras mostra que os indicadores de saúde são muito melhores no que costumavam ser as repúblicas socialistas da URSS do que nos países capitalistas vizinhos, embora estes indicadores fossem igualmente fracos antes do socialismo ser estabelecido no
\par
UN.
\par
Antes de 1949 1950 1954 1955 1958
\par
195
\par
179
\par
200
\par
138,5'
\par
80Bb 50,8 89,1
\par
1960 1%5 1970 1973-1975 1975 1980 1981
\par
121 81 61
\par
41 38
\par
47,0'
\par
34,76
\par
1983
\par
Cidades [=I' Condados [58 em 12 províncias]
\par
13,6 26,5
\par
1985
\par
36 (1985)
\par
Cidades [36] Condados [72 em 15 províncias]
\par
14,0 25,1
\par
1989
\par
Cidades [82] Condados [72 em 15 províncias]
\par
13,8 21,7
\par
'De uma pesquisa realizada com 50.000 habitantes em 14 províncias. 'De uma pesquisa realizada na maioria das cidades e condados de 19 províncias. 'Do estudo retrospectivo nacional sobre mortalidade por câncer na China. dDo terceiro Censo (1982). 'O número de cidades ou condados é dado entre parênteses.
\par
URSS. A Tabela 19 mostra a evolução das taxas de mortalidade infantil nas repúblicas soviéticas, incluindo as repúblicas asiáticas. A taxa estimada de mortalidade infantil da Ásia Central de 46 por 1.000 nascidos vivos em 1975 foi consideravelmente melhor do que a da Turquia (153/1.000), Afeganistão (269/1.000) e Irã (120/1.000).
\par
Em resumo, os dados empíricos apresentados nesta discussão da experiência asiática não parecem confirmar a posição de que o capitalismo teve um desempenho melhor do que o socialismo na melhoria da saúde das populações.
\par
Em África, a experiência socialista é demasiado nova para ser capaz de detectar mudanças significativas. Na Europa, a comparação não é tão favorável ao socialismo. As repúblicas europeias da União Soviética não têm melhores indicadores de saúde do que a maioria dos países capitalistas
\par
Consumo alimentar (em quilogramas) nas populações rurais e urbanas na China Popular, vários anos, 1978-1988"
\par
1978 1982 1985 1988
\par
248,00 260,00 257,45 259,51
\par
123,01 192,14 209,31 210,46
\par
5,76 9,05 10,97 10,71
\par
0,25 0,78 1,03 1,25
\par
0,84 1,32 1,a 1,91
\par
1981 1982 1985 1988
\par
145,44 144,56 131,16 137,17
\par
18,60 18,67 20,16 19,75
\par
7,26 7,67 7,80 7,07
\par
1911-20 193140 1951-60 1971-75 1976-80 1981-85 1986-88
\par
242 173 120 57 46 32 27
\par
278 207 140 134 124 104 95
\par
'Fontes: Para 1911-1%0: compilado de várias publicações do Censo da Índia em The Impact of. Social ano Economic Development on Mortality; ver Tabela 1, nota de rodapé b. Para 1971-1988: Mari Bhat, P. N., e Irudaya Rajan, S. Transição demográfica em Kerala revisitada. Semanal Econômico e Político 25.1957-1980.1990.
\par
Diminuição da mortalidade infantil em Kerala e em toda a Índia entre as décadas de 1950 e 1980''
\par
Taxa para 1951-60, por 1.000 Taxa para 1981-85, por 1.000 Redução, como porcentagem da taxa de 1951-60
\par
120 32 73%
\par
140 104 26%
\par
1911-20 192 1-30 1951-60 1%1-70 1971-75 1976-80 1981-85 1986
\par
25,5 29,5 49 59,3 60,5 63,5 65,2 67,5
\par
27,4 32,7 48,3 59,3 63 67,4 71,5 73
\par
22,6 26,9 41,4 47,9 49,7 51,7 54,5 56
\par
23,3 26,6 40 45,5 48,3 51,8 54,9 56,5
\par
'Fontes: ver Tabela 12, nota de rodapé a
\par
Aumentos na esperança de vida, por sexo, em Kerala e em toda a Índia, entre as décadas de 1950 e 1980"
\par
49 65,2 16,2
\par
48,3 71,5 23,2
\par
41,4 54,5 13,1
\par
40 54,9 14,1
\par
Taxas de alfabetização (como porcentagem da população) em Kerala e em toda a Índia, vários anos, 1951-1981
\par
1951 1%1 1971 1981
\par
40 47 60
\par
50 55 67 75
\par
32 39 54 66
\par
17 24 30
\par
25 34 40 47
\par
8 13 19 25
\par
'Fonte: O Impacto do Desenvolvimento Social e Econômico na Mortalidade; ver Tabela 1, nota de rodapé b.
\par
Aumento da alfabetização, por sexo, em Kerala e em toda a Índia, entre as décadas de 1950 e 1980”
\par
50 75 25
\par
32 66 34
\par
25 47 22
\par
8 25 17
\par
‘Fonte: Baseado na Tabela 16. Taxas de alfabetização em percentagem da população.
\par
Renda per capita (em rúpias) em Kerala e em toda a Índia, vários anos, das décadas de 1950 a 1980'' * Kerala
\par
1950-51 1955-56 1960-61 198041'
\par
304 312 326 1.382
\par
2% 308 336 1,571
\par
‘Fonte: Para 19.50-1961: O Impacto do Desenvolvimento Social e Econômico na Mortalidade; ver Tabela 2, nota de rodapé 6. Para 1980-1981: Sistema de Saúde em Kerala e seu Impacto na Mortalidade Infantil. Em Good Heolrh ou De baixo custo, editado por S. Halstead, I. Walsh e K. Warren. The Rockefeller Foundation, Nova York, 1985. Preços, salvo indicação em contrário.
\par
%vem em 1-1 'Estimativas em preços de 1981.
\par
Países da Europa Ocidental. Foi esta situação que forneceu a munição para aqueles que definem a disparidade de resultados como um fracasso do socialismo. A seguinte declaração, que apareceu em 1981 numa das publicações intelectuais mais influentes dos Estados Unidos, a New York Review of Books, é representativa: “Não há um único país em toda a Europa onde as vidas sejam tão curtas ou os bebés' as mortes são tão altas que nem mesmo a empobrecida e semicivilizada Albânia. No domínio da saúde, os pares da União Soviética encontram-se na América Latina e na Ásia” (20).
\par
A informação empírica, amplamente disponível aos estudiosos nos Estados Unidos, não confirma esta afirmação. A esperança de vida na URSS em 1975 era de 70,4 anos, apenas 8 meses a menos que a esperança de vida nos Estados Unidos no mesmo ano. A esperança de vida soviética era superior à da Finlândia, Iugoslávia, Romênia, Polônia, Hungria, Checoslováquia, Albânia e Portugal. Foi consideravelmente maior do que na maioria dos países da América Latina (México, 64,7 anos; Chile, 62,6; Brasil, 61,4; Argentina, 68,2) e da Ásia (Afeganistão, 40 anos; irã, 51; Turquia, 56,9). Na verdade, a esperança de vida na URSS era apenas ligeiramente inferior à dos principais países capitalistas avançados.
\par
Taxas de mortalidade infantil (por 1.000 nascidos vivos), nas repúblicas soviéticas e em algumas grandes cidades, 1960-1974”
\par
1960
\par
1%7
\par
1970
\par
1974
\par
37,0 30,0 34,9
\par
25,0 18,4 21,o
\par
23,0 17,3 19,0
\par
23,0 17,4 (1973) 17,0
\par
31,2 27,0 38,0
\par
19,2 17,0 20,5
\par
17,8 18,0 19,3
\par
17,6 19,0 19,4
\par
50,0
\par
36,8
\par
43,0
\par
28,0
\par
29,0
\par
38,0
\par
(26,7) - (21,3) - (24,1)
\par
(21,4) - (33,9) - (20,7)
\par
36,8
\par
30,0
\par
30,0
\par
28,0
\par
26,0
\par
43,0
\par
38,0
\par
31.0
\par
(26,7) - (25,3) - (46,7) - (32,4) - (40,0) - (1 6,8) 24,7
\par
(29,2) - (24,1) - (5 1,8) - (46,4) - (45,5) - (24,4) 27,9
\par
35,3
\par
26.0
\par
‘Fonte: Davis, C., e Feshback, M. Rising Infanf Morfality in lhe U.S.S.R. in lhe 1970.q Tabelas 2 e 4.
\par
Como Reino Unido (72,4 anos), Japão (72,9) e Alemanha Ocidental (71,3). Da mesma forma, a taxa de mortalidade infantil na URSS em 1974 (27,9/1.000) comparou-se favoravelmente com as taxas de 1975 da Áustria (21/1.000), da Alemanha Ocidental (20/1.000), da Itália (21/1.000), do Reino Unido (16 /1.000) e Austrália (17/1.000) (21). As condições de saúde na URSS melhoraram substancialmente desde a Segunda Guerra Mundial. Foi em meados da década de 1960 que a mortalidade infantil começou a aumentar e a esperança de vida a diminuir, especialmente nas repúblicas asiáticas; esta situação tem sido objeto de amplo debate. Mas a evidência disponível não confirma a conclusão da New York Review of Books de que os pares de saúde da União Soviética se encontravam no mundo subdesenvolvido.
\par
Ainda assim, o projeto socialista soviético não teve um desempenho tão bom como a maioria dos homólogos capitalistas no Ocidente. Publiquei uma crítica detalhada do modelo soviético em outro lugar (22). O fosso entre a União Soviética e os países capitalistas tornaram o modelo soviético pouco atraente para as populações ocidentais e, eventualmente, também para as populações soviéticas.
\par
Esta pesquisa internacional mostra que, pelo menos no âmbito do subdesenvolvimento, onde a fome e a desnutrição fazem parte da realidade diária, o socialismo, e não o capitalismo, é a forma de organização da produção e distribuição de bens e serviços que melhor responde às necessidades socioeconômicas imediatas da maioria dessas populações. É claro que, apesar das melhorias importantes nos indicadores de saúde, a situação dos países subdesenvolvidos impõe sérias restrições ao socialismo e muitas vezes leva a limitações de direitos políticos, como direitos de organização, diversidade política e liberdade de imprensa. Isso explica o desencanto de grandes partes da intelectualidade dos países desenvolvidos capitalistas ocidentais com esse tipo de socialismo. Mas sua superioridade sobre o capitalismo na promoção de direitos socioeconômicos, incluindo direitos de saúde, explica a enorme atratividade do projeto socialista entre as populações do mundo subdesenvolvido. Testemunhe o enorme sucesso político dos recentes movimentos socialistas populares do partido de Lula no Brasil, Cárdenas... no México (cuja vitória nas eleições presidenciais do México foi roubada pelo atual regime antidemocrático), o Congresso Nacional Africano na África do Sul e as forças socialistas no Nepal, para mencionar apenas alguns eventos políticos no último ano. O socialismo não foi inferior ao capitalismo no mundo do subdesenvolvimento, e sua atratividade para as populações dos países subdesenvolvidos continua alta.
\par

\section{SOCIALISMO NOS PAÍSES CAPITALISTAS DESENVOLVIDOS}

\par
Embora o leninismo tenha sido, pelo menos até recentemente, a forma predominante de socialismo nos países subdesenvolvidos, a social-democracia tem sido a versão predominante nos países capitalistas desenvolvidos. É importante sublinhar que durante a maioria deste século, as duas tradições socialistas diferiram nos seus meios, mas não nos seus fins. Na verdade, a social-democracia durante a maioria deste século teve como objectivo o estabelecimento do socialismo. Tal como afirma um dos partidos social-democratas mais influentes, o Partido Social-democrata Sueco, “Aspiramos a transformar completamente a organização da sociedade burguesa e a promover a libertação social da classe trabalhadora” (citado em 23). O projeto socialista apelava à “abolição da exploração, à destruição da divisão da sociedade em classes, ao fim do desperdício da produção capitalista e à erradicação de todas as fontes de injustiça e preconceito”. Isto exigiu a socialização (ou colectivização, ou nacionalização, termos usados ​​com ambiguidade deliberada nos programas econômicos da maioria dos partidos social-democratas) dos meios de produção. Os social-democratas e os leninistas divergiam principalmente quanto aos meios para atingir esse objectivo. Enquanto os leninistas apoiavam a revolta insurrecional e a tomada do Estado, os sociais-democratas favoreciam a via eleitoral, acreditando que “o sufrágio universal é incompatível com uma sociedade dividida numa pequena classe de proprietários e numa grande classe de despossuídos. Ou os ricos e os possuidores tirarão o sufrágio universal, ou os pobres, com a ajuda do seu direito de voto, adquirirão para si uma parte das riquezas acumuladas” (23).
\par
O socialismo reformista, em oposição ao socialismo insurrecional, visava a transformação gradual da sociedade através do processo eleitoral. Mas, como indicou Kautsky (24), o principal teórico da Internacional Socialista, as reformas foram percebidas não como um substituto para a revolução social, mas como um caminho para a mesma. A constituição da maioria dos partidos social-democratas nos países capitalistas desenvolvidos (excepto o Partido Social Democrata Português) reivindica fidelidade ao projeto socialista e à necessidade de transcender ou romper com o capitalismo. Ainda em 1981, o Partido Socialista Francês venceu as eleições (em aliança com o Partido Comunista) com um apelo à ruptura com o capitalismo!
\par
Os objectivo e tácticas originais dos partidos social-democratas tiveram de ser modificados devido à necessidade determinada pelo processo eleitoral a que escolheram obedecer - estabelecer alianças eleitorais e alargar a sua base para alcançar as tão necessárias maiorias eleitorais. Esta necessidade explica as mudanças nas suas políticas econômicas e sociais. Entre as políticas econômicas, a mudança mais importante foi a redefinição do controlo colectivo dos meios de produção. Acreditava-se que o controle não exigia propriedade estatal real. O Estado era considerado o agente que poderia dirigir e regular os meios de produção sem possuí-los. Uma forma de regular a produção era controlar ou influenciar o crédito e determinar o nível de consumo global que mitigaria, por medidas de bem-estar, as desigualdades estabelecidas pelo mercado.
\par
Socialmente, a necessidade de alargar alianças levou ao estabelecimento do universalismo como princípio fundamental da política social. A introdução de programas de saúde universais e abrangentes, nos quais os movimentos laborais e os seus instrumentos políticos desempenham um papel fundamental, foi um resultado direito da necessidade de os partidos social-democratas se tornarem os partidos do povo e não apenas os partidos da classe trabalhadora. Este foco no domínio do consumo levou ao estabelecimento do Estado-providência, uma criação da maior parte dos partidos social-democratas no período pós-Primeira Guerra Mundial. A criação e expansão do Estado-Providência levaram, no final dos anos 1960 e início dos anos 1970 (estimulados por uma radicalização das reivindicações trabalhistas e populares, com o surgimento dos movimentos sociais) a um questionamento das relações de propriedade na produção. Mostrei noutro local como os partidos social-democratas na década de 1970 passaram da política de consumo para a política de produção (25). As famosas propostas Meidner do Partido Social Democrata Sueco, por exemplo, visavam coletivizar os meios de produção. Como afirmou um importante teórico do partido, “a implementação destas propostas tornaria a Suécia o primeiro país do mundo a dar passos decisivos e virtualmente irreversíveis em direção a relações de produção socialistas de uma forma democrática e reformista” (26).
\par
Estas políticas universalistas levaram a um crescimento na atração do eleitorado pelos partidos social-democratas. As décadas socialistas foram as décadas de 1970 e 1980. A Tabela 20 mostra o crescimento dos partidos socialistas na Europa. Em 1989, o bloco de esquerda (partidos social-democratas, partidos comunistas e Verdes) tornou-se o bloco maioritário no Parlamento Europeu. Contudo, a experiência socialista nos países capitalistas desenvolvidos durante este século foi curta. Até muito recentemente, a experiência governamental da social-democracia tem sido limitada porque a esmagadora maioria dos partidos social-democratas nunca governou por maioria. Tiveram de estabelecer alianças com partidos pró-capitalistas, o que restringiu a realização dos seus programas socialistas. Apenas na Suécia e na Noruega o período de mandato socialista excedeu o dos partidos pró-capitalistas no período 1945-1978, tendo a Suécia o período mais longo de governo socialista. Não é de surpreender que tenha sido também a Suécia que
\par
Países onde o movimento laboral superou 50 por cento dos votos nas eleições parlamentares nacionais desde 1%5"
\par
1971,1975,1979 1981 1966 1981 1968,1970,1982 1%9 1982,1986 1976
\par
'Fonte: Therbom, G. As perspectivas do trabalho e a transformação do capitalismo avançado. NovoLefiRev. 145: 8,1984.
\par
Fez a única tentativa séria de ir além do estado de bem-estar social e se concentrar novamente na política de produção (com o plano Meidner).
\par

\section{COMO AVALIAR A EXPERIÊNCIA SOCIALISTA SOB O CAPITALISMO}

\par
A avaliação do socialismo nos países capitalistas desenvolvidos tem de ter em conta o grau de influência dos partidos socialistas nas políticas governamentais e também a sua força organizacional, medida pela densidade sindical e pela unidade do movimento operário, o principal eleitorado dos partidos socialistas (27). A influência nas políticas governamentais é medida pela estabilidade e duração do controlo do partido socialista sobre o governo e pelo nível de participação (maioria ou maioria substancial) durante o período em estudo. Outro elemento, que liga o poder político ao econômico, é o nível de sindicalização e a articulação do movimento sindical com o partido ou partidos socialistas, ou seja, se os sindicatos estão organizados segundo linhas de classe e beém os partidos como seus instrumentos políticos ou se estão organizados por interesses religiosos, políticos ou corporativistas. Até ao final da década de 1970, apenas três países – a Suécia, a Noruega e a Dinamarca – tinham governos de maioria social-democrata, e apenas na Suécia e na Noruega é que os pró-socialistas estiveram no poder durante mais tempo do que os pró-capitalistas. Em 1970, a Suécia tinha um governo socialista há 24 anos, a Noruega há 20 anos e a Dinamarca há 16 anos. Estes países apresentam um elevado nível de sindicalização; os sindicatos seguem linhas de classe sem divisões por motivos religiosos ou políticos e beém os partidos políticos como os seus instrumentos políticos. Imediatamente após a Primeira Guerra Mundial, todos os três países tinham indicadores de saúde semelhantes ou até piores (no caso da Dinamarca) do que os dos Estados Unidos, o país capitalista com os governos pró-socialistas mais fracos e os governos pró-capitalistas mais fortes durante este período (1947). — 1978). Em 1980, os três países tinham melhorado dramaticamente os seus indicadores de saúde, alcançando alguns dos melhores indicadores de saúde do mundo ocidental (Tabela 21).
\par
Taxas de adesão à União e de mortalidade infantil nos países social-democratas e nos Estados Unidos, 1950-1980 durante a adesão de 1945-70, '76
\par
1950
\par
1980
\par
24 20 16 0
\par
75 52 50 23
\par
20 24 31 29
\par
6,9 8,1 8,4 12,6
\par

\section{CONCLUSÕES}

\par
Com base nesta informação dificilmente se poderia concluir que o socialismo é menos eficaz do que o capitalismo na resposta às necessidades de saúde da população. Não nego que o capitalismo tenha sido eficaz em algumas partes do mundo, e que em alguns casos limitado possa ter sido ainda mais eficaz do que o socialismo. Mas a evidência empírica apresentada neste artigo mostra que, ao contrário do que é amplamente afirmado hoje, a experiência socialista (tanto nas suas tradições leninistas como nas suas tradições social-democratas) tem sido, na maioria das vezes, mais eficiente na resposta às necessidades humanas do que a experiência capitalista. Infelizmente, a experiência socialista também incluiu desenvolvimentos muito negativos que negaram componentes importantes do projeto socialista. A distância entre a teoria e a prática socialistas tem-se assemelhado com demasiada frequência à distância entre o Sermão da Montanha e o Cristianismo nos seus 2000 anos de existência. Ainda assim, a experiência histórica do socialismo é bastante curta. O capitalismo existe há mais de três séculos. O socialismo, por outro lado, apenas começou.
\par
Agradecimentos - Agradecimentos a ha Diez e Suzzane McQueen pela assistência na coleta de informações para este artigo.
\par

\section{REFERÊNCIAS}

\par
1. Fukuyama, F. O fim da história. The Nationallinterest, verão de 1990. 2. MacPherson, S. Quinhentos milhões de crianças. Em Pobreza e Crianças Estamos no Terceiro Mundo. Martin's Press, Nova York, 1987.
\par
3. Trechos das Encíclicas do Papa: Sobre dar uma face humana ao capitalismo. New York Times, 3 de maio de 1991, p. A10.
\par
4. Tabb, W. T. O Futuro do Socialismo: Perspectiva da Esquerda. Monthly Review Press, Nova York, 1990.
\par
5. Pmworski, A. Poderíamos alimentar a todos? A irracionalidade do capitalismo e a inviabilidade do socialismo. Politics associei, março de 1991, p. 14.
\par
6. Roosevelt, F. D. Discurso presidencial ao Congresso dos EUA, 11 de janeiro de 1944. 7. McKeown, T. O papel da medicina: sonho, miragem ou nêmesis? Princeton University Press, Princeton, NJ, 1979.
\par
8. Departamento de Saúde da Cidade de Baltimore. Relatório sobre Mortalidade Infantil. 1991. 9. Organização Pan-Americana da Saúde. Condições de Saúde nas Américas, Vol. I. Washington, DC, 1990.
\par
10. UNICEF. Estatísticas Mundiais sobre Crianças: Bolso Estatístico da UNICEF. Nova Iorque, 1986. 11. UNICEF. Estatísticas sobre Crianças em Países Assistidos pela UNICEF. Nova York, 1987. 12. Pesquisa nutricional de alunos da sexta série de Cuba. J. Nutr. 64(3), março de 1958. 13. UNICEF. A situação da criança no mundo, Nova York. 1989. 14. Dim-Briquets, S. A Revolução da Saúde em Cuba. University of Texas Press, 1983. 15. Organização Pan-Americana da Saúde. Condições de Saúde nas Américas 1953-56. Washington, DC. 1956.
\par
16. Anuário Estatístico para a América Latina e o Caribe 1990. 17. Escudero, J. C. Desnutrição na América Latina. Manuscrito não publicado. Universidade de Buenos Aires, Argentina, 1986.
\par
18. Organização Pan-Americana da Saúde. Relatório sobre cólera. Washington, DC, 1991. 19. Banco Mundial. China: O Setor da Saúde. Um estudo nacional do Banco Mundial. Nova York, novembro de 1984.
\par
20. Eberstadt, N. A crise sanitária na URSS. Revisão de Bmks em Nova York, 19 de fevereiro de 1981. 21. Szymanski, A. Sobre os usos da desinformação para legitimar o renascimento da guerra fria: saúde na URSS. Internacional J. Healthh Servir. ^ 12 481496,1982.
\par
22. Navarro, V. Segurança Social e Medicina na URSS. Lexington Books, Lexington, Massachusetts, 1976.
\par
23. Przeworski, A. A social-democracia como fenômeno histórico. Novo LeJ Rev. 122 45,1980. 24. Kautsky, K. A Luta de Classes, p. 186. Norton, Nova York, 1971. 25. Navarro, V. Produção e estado de bem-estar social: O contexto político das reformas. Internacional J. Costura de Saúde. 21: 585414.1991.
\par
26. Himmelstrand, U. Suécia: Paraíso em apuros. Em Beyond lhe Weyare State, editado por I. Hare. Schocken Books, Nova York, 1982.
\par
27. Korpi, W. A Luta Democrática de Classes. Routledge e Kegan Paul, Boston, 1983.
\par
Dr. Vicente Navarro Departamento de Política e Gestão de Saúde Escola de Higiene e Saúde Pública da Universidade Johns Hopkins Hampton House, Sala 448 624 North Broadway Baltimore, MD 21205
\end{document}
